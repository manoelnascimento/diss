%% abtex2-modelo-trabalho-academico.tex, v-1.9.1 laurocesar
%% Copyright 2012-2013 by abnTeX2 group at http://abntex2.googlecode.com/ 
%%
%% This work may be distributed and/or modified under the
%% conditions of the LaTeX Project Public License, either version 1.3
%% of this license or (at your option) any later version.
%% The latest version of this license is in
%%   http://www.latex-project.org/lppl.txt
%% and version 1.3 or later is part of all distributions of LaTeX
%% version 2005/12/01 or later.
%%
%% This work has the LPPL maintenance status `maintained'.
%% 
%% The Current Maintainer of this work is the abnTeX2 team, led
%% by Lauro César Araujo. Further information are available on 
%% http://abntex2.googlecode.com/
%%
%% This work consists of the files abntex2-modelo-trabalho-academico.tex,
%% abntex2-modelo-include-comandos and abntex2-modelo-references.bib
%%

% ------------------------------------------------------------------------
% ------------------------------------------------------------------------
% abnTeX2: Modelo de Trabalho Academico (tese de doutorado, dissertacao de
% mestrado e trabalhos monograficos em geral) em conformidade com 
% ABNT NBR 14724:2011: Informacao e documentacao - Trabalhos academicos -
% Apresentacao
% ------------------------------------------------------------------------
% ------------------------------------------------------------------------

\documentclass[
	% -- opções da classe memoir --
	12pt,					% tamanho da fonte
	openright,			% capítulos começam em pág ímpar (insere página vazia caso preciso)
	twoside,			% para impressão em verso e anverso. Oposto a oneside
	a4paper,			% tamanho do papel. 
	% -- opções da classe abntex2 --
	%chapter=TITLE,		% títulos de capítulos convertidos em letras maiúsculas
	%section=TITLE,		% títulos de seções convertidos em letras maiúsculas
	%subsection=TITLE,	% títulos de subseções convertidos em letras maiúsculas
	%subsubsection=TITLE,% títulos de subsubseções convertidos em letras maiúsculas
	% -- opções do pacote babel --
	english,				% idioma adicional para hifenização
	french,				% idioma adicional para hifenização
	spanish,			% idioma adicional para hifenização
	brazil				% o último idioma é o principal do documento
	]{abntex2}



% ---
% Pacotes básicos 
% ---
\usepackage{lmodern}			% Usa a fonte Latin Modern			
\usepackage[T1]{fontenc}		% Selecao de codigos de fonte.
\usepackage[utf8]{inputenc}	% Codificacao do documento (conversão automática dos acentos)
\usepackage{lastpage}			% Usado pela Ficha catalográfica
\usepackage{indentfirst}		% Indenta o primeiro parágrafo de cada seção.
\usepackage{color}				% Controle das cores
\usepackage{graphicx}			% Inclusão de gráficos
\usepackage{microtype} 		% para melhorias de justificação
\usepackage{UFBA}
\usepackage{breakurl}
% ---
		
% ---
% Pacotes adicionais, usados apenas no âmbito do Modelo Canônico do abnteX2
% ---
\usepackage{lipsum}				% para geração de dummy text
% ---

% ---
% Pacotes de citações
% ---
\usepackage[brazilian,hyperpageref]{backref}	 % Paginas com as citações na bibl
\usepackage[alf]{abntex2cite}	% Citações padrão ABNT

% --- 
% CONFIGURAÇÕES DE PACOTES
% --- 

% ---
% Configurações do pacote backref
% Usado sem a opção hyperpageref de backref
\renewcommand{\backrefpagesname}{Citado na(s) página(s):~}
% Texto padrão antes do número das páginas
\renewcommand{\backref}{}
% Define os textos da citação
\renewcommand*{\backrefalt}[4]{
	\ifcase #1 %
		Nenhuma citação no texto.%
	\or
		Citado na página #2.%
	\else
		Citado #1 vezes nas páginas #2.%
	\fi}%
% ---


% ---
% Informações de dados para CAPA e FOLHA DE ROSTO
% ---
\titulo{O distrito soteropolitano de Brotas na Primeira República (1889-1930)} 
\autor{Manoel Maria do Nascimento Júnior}
\local{Salvador, Bahia}
\data{2015}
\orientador{Odete Dourado}
\instituicao{Universidade Federal da Bahia}
%\faculdade{Faculdade de Arquitetura}
%\programa{Programa de Pós-Graduação em Arquitetura e Urbanismo}
\tipotrabalho{Dissertação (Mestrado)}
% O preambulo deve conter o tipo do trabalho, o objetivo, 
% o nome da instituição e a área de concentração 
\preambulo{Dissertação apresentada ao Programa de Pós-Graduação em Arquitetura e Urbanismo da Faculdade de Arquitetura da UFBA como requisito parcial para obtenção do grau de Mestre em Arquitetura e Urbanismo}
% ---


% ---
% Configurações de aparência do PDF final

% alterando o aspecto da cor azul
\definecolor{blue}{RGB}{41,5,195}

% informações do PDF
\makeatletter
\hypersetup{
     	%pagebackref=true,
		pdftitle={\@title}, 
		pdfauthor={\@author},
    		pdfsubject={\imprimirpreambulo},
	    pdfcreator={LaTeX with abnTeX2},
		pdfkeywords={urbanismo}{Salvador}{República Velha}{Brotas}{conflitos sociais}, 
		colorlinks=true,       		% false: boxed links; true: colored links
    	linkcolor=blue,          	% color of internal links
    	citecolor=blue,        		% color of links to bibliography
    	filecolor=magenta,      		% color of file links
		urlcolor=blue,
		bookmarksdepth=4
}
\makeatother
% --- 

% --- 
% Espaçamentos entre linhas e parágrafos 
% --- 

% O tamanho do parágrafo é dado por:
\setlength{\parindent}{1.3cm}

% Controle do espaçamento entre um parágrafo e outro:
%\setlength{\parskip}{0.2cm}  % tente também \onelineskip

% ---
% compila o indice
% ---
\makeindex
% ---

\newcommand{\subtitulo}{\bfseries Conflitos sociais na produção, apropriação e uso do seu espaço urbano}


% ----
% Início do documento
% ----
\begin{document}

% Retira espaço extra obsoleto entre as frases.
\frenchspacing 

% ----------------------------------------------------------
% ELEMENTOS PRÉ-TEXTUAIS
% ----------------------------------------------------------
% \pretextual

% ---
% Capa
% ---




\begin{titlingpage}
\bfseries 
\centering

\fontsize{16}{18}\selectfont UNIVERSIDADE FEDERAL DA BAHIA

\fontsize{14}{16}\selectfont FACULDADE DE ARQUITETURA

\normalsize PROGRAMA DE PÓS-GRADUAÇÃO EM ARQUITETURA E URBANISMO





\vspace{3cm}

\MakeUppercase{\bfseries\fontsize{14}{16}\selectfont\imprimirautor}

\vspace{3cm}

\MakeUppercase{\fontsize{14}{16}\selectfont\bfseries\imprimirtitulo:}

\medskip
\MakeUppercase{\subtitulo}

\vfill

\normalfont\fontsize{14}{16}\selectfont
Salvador, Bahia

2015


\end{titlingpage}

%%% FOLHA DE ROSTO
\begin{titlingpage}
\centering
\MakeUppercase{\bfseries\fontsize{14}{16}\selectfont\imprimirautor}

\vspace{3cm}

\MakeUppercase{\fontsize{14}{16}\selectfont\bfseries\imprimirtitulo:}

\medskip
\MakeUppercase{\subtitulo}

\vspace*{\fill}

\normalfont

\normalsize

\begin{flushright}
	\begin{minipage}{.5\textwidth}
	\SingleSpacing
	\imprimirpreambulo
	
	\medskip
	Orientadora: Odete Dourado.
\end{minipage}%
\end{flushright}
\vspace*{\fill}
%

\vfill

\normalfont\fontsize{14}{16}\selectfont

Salvador, Bahia

2015
	

\end{titlingpage}


\setcounter{page}{2}



% ---

% ---
% Inserir a ficha bibliografica
% ---

% Isto é um exemplo de Ficha Catalográfica, ou ``Dados internacionais de
% catalogação-na-publicação''. Você pode utilizar este modelo como referência. 
% Porém, provavelmente a biblioteca da sua universidade lhe fornecerá um PDF
% com a ficha catalográfica definitiva após a defesa do trabalho. Quando estiver
% com o documento, salve-o como PDF no diretório do seu projeto e substitua todo
% o conteúdo de implementação deste arquivo pelo comando abaixo:
%
% \begin{fichacatalografica}
%     \includepdf{fig_ficha_catalografica.pdf}
% \end{fichacatalografica}
\begin{fichacatalografica}
		\vspace*{\fill}					% Posição vertical
	\hrule							% Linha horizontal
	\begin{center}					% Minipage Centralizado
	\begin{minipage}[c]{12.5cm}		% Largura
	
	\imprimirautor
	
	\hspace{0.5cm} \imprimirtitulo:\subtitulo  / \imprimirautor. --
	\imprimirlocal, \imprimirdata-
	
	\hspace{0.5cm} \pageref{LastPage} p. : il. (algumas color.) ; 30 cm.\\
	
	\hspace{0.5cm} \imprimirorientador~\imprimirorientador\\
	
	\hspace{0.5cm}
	\parbox[t]{\textwidth}{\imprimirtipotrabalho~--~\imprimirinstituicao,
	\imprimirdata.}\\
	
	\hspace{0.5cm}
		1. Conflitos sociais.
		2. Brotas (distrito de Salvador, Bahia).
		I. \imprimirorientador~\imprimirorientador.
		II. \imprimirinstituicao.
		III. Faculdade de Arquitetura.
		IV. Título\\ 			
	
	\hspace{8.75cm} CDU 02:141:005.7\\
	
	\end{minipage}
	\end{center}
	\hrule
\end{fichacatalografica}
% ---

% ---
% Inserir errata
% ---
% \begin{errata}
%
% \end{errata}
% ---

% ---
% Inserir folha de aprovação
% ---

% Isto é um exemplo de Folha de aprovação, elemento obrigatório da NBR
% 14724/2011 (seção 4.2.1.3). Você pode utilizar este modelo até a aprovação
% do trabalho. Após isso, substitua todo o conteúdo deste arquivo por uma
% imagem da página assinada pela banca com o comando abaixo:
%
% \includepdf{folhadeaprovacao_final.pdf}
%
\begin{folhadeaprovacao}

  \begin{center}
    \MakeUppercase{\fontsize{14}{16}\selectfont\bfseries\imprimirautor}

    \vspace{3cm}
 
      \MakeUppercase{\fontsize{14}{16}\selectfont\bfseries\imprimirtitulo:}
      
      \medskip
      \MakeUppercase{\subtitulo}
    \end{center}
    \bigskip
    \begin{quote}
        \imprimirpreambulo
    \end{quote}
        \bigskip
 
        
   \begin{flushright}
   	Aprovado em \imprimirlocal, INSERIR DATA DE EXAME:
   \end{flushright}

   \assinatura{\textbf{\imprimirorientador} \\ Orientadora} 
   \assinatura{\textbf{Fulano} \\ Convidado 1}
   \assinatura{\textbf{Beltrana} \\ Convidado 2}
   %\assinatura{\textbf{Professor} \\ Convidado 3}
   %\assinatura{\textbf{Professor} \\ Convidado 4}
      
 \end{folhadeaprovacao}
% ---

% ---
% Dedicatória
% ---
\begin{dedicatoria}
   \vspace*{\fill}
   \centering
   \noindent
   \textit{\lipsum[1].} \vspace*{\fill}
\end{dedicatoria}
% ---

% ---
% Agradecimentos
% ---
\begin{agradecimentos}

\lipsum[50]

\end{agradecimentos}
% ---

% ---
% Epígrafe
% ---
\begin{epigrafe}
    \vspace*{\fill}
	\begin{flushright}
		\textit{\lipsum[2]}
	\end{flushright}
\end{epigrafe}
% ---

% ---
% RESUMOS
% ---

% resumo em português
\setlength{\absparsep}{18pt} % ajusta o espaçamento dos parágrafos do resumo
\begin{resumo}
 \lipsum[50]
 \textbf{Palavras-chaves}: conflitos sociais. espaço urbano. Primeira República (1889-1930).
\end{resumo}

% resumo em inglês
\begin{resumo}[Abstract]
 \begin{otherlanguage*}{english}
\lipsum[50]
   \vspace{\onelineskip}
 
   \noindent 
   \textbf{Key-words}: latex. abntex. text editoration.
 \end{otherlanguage*}
\end{resumo}

% resumo em francês 
\begin{resumo}[Résumé]
 \begin{otherlanguage*}{french}
\lipsum[50].
 
   \textbf{Mots-clés}: latex. abntex. publication de textes.
 \end{otherlanguage*}
\end{resumo}

% resumo em espanhol
\begin{resumo}[Resumen]
 \begin{otherlanguage*}{spanish}
\lipsum[50].
  
   \textbf{Palabras clave}: latex. abntex. publicación de textos.
 \end{otherlanguage*}
\end{resumo}
% ---

% ---
% inserir lista de ilustrações
% ---
% \pdfbookmark[0]{\listfigurename}{lof}
% \listoffigures*
% \cleardoublepage
% ---

% ---
% inserir lista de tabelas
% ---
% \pdfbookmark[0]{\listtablename}{lot}
% \listoftables*
% \cleardoublepage
% ---

% ---
% inserir lista de abreviaturas e siglas por meio de um \item para cada uma
% ---
% \begin{siglas}
%
% \end{siglas}
% ---

% ---
% inserir lista de símbolos por meio de um \item para cada um
% ---
%\begin{simbolos}
%
% \end{simbolos}
% ---

% ---
% inserir o sumario
% ---
\pdfbookmark[0]{\contentsname}{toc}
\tableofcontents*
\cleardoublepage
% ---



% ----------------------------------------------------------
% ELEMENTOS TEXTUAIS
% ----------------------------------------------------------
\textual

% ----------------------------------------------------------
% Introdução (exemplo de capítulo sem numeração, mas presente no Sumário)
% ----------------------------------------------------------
\chapter*[Introdução]{Introdução}\label{intro}

\lipsum[50]
\addcontentsline{toc}{chapter}{Introdução}

Escrever a introdução


% ----------------------------------------------------------


\chapter[Salvador: sociedade, política e espaço urbano durante a Primeira República (1889-1930)]{Salvador: sociedade, política e espaço\\ urbano durante a Primeira República\\ (1889-1930)}\label{cap:1}

\lipsum[50]

\section[Sociedade]{Sociedade}\label{sec:1.1}
\lipsum[50]
\subsection[Caracterização demográfica e econômica]{Caracterização demográfica e econômica}\label{subsec:1.1.1}
\lipsum[50]
\subsection{Classes sociais}\label{subsec:1.1.2}
\lipsum[50]
\subsubsection{Transição do trabalho escravo ao trabalho livre}\label{subsubsec:1.1.2.1}
\lipsum[50]
\subsubsection{Caracterização da estrutura de classes}\label{subsubsec:1.1.2.2}
\lipsum[50]
\subsection{Cultura e espaço público}\label{subsec:1.1.3}
\lipsum[50]
\section{Política}\label{sec:1.2}
\lipsum[50]
\subsection{Instituições de governo}
\lipsum[50]
\subsubsection{Instituições municipais}
\lipsum[50]
\subsubsection{Instituições estaduais}
\lipsum[50]
\subsection{Agentes: os políticos e as grandes empresas}
\lipsum[50]
\subsubsection{Sistema partidário e lideranças carismáticas}
\lipsum[50]
\subsubsection{A política baiana e a banca internacional}
\lipsum[50]
\subsection{Dinâmica}
\lipsum[50]
\subsubsection{O severinismo}
\lipsum[50]
\subsubsection{O seabrismo}
\lipsum[50]
\subsubsection{Depois do seabrismo}
\lipsum[50]
\section{Espaço urbano}\label{sec:1.3}
\lipsum[50]
\subsection{Distritos urbanos de Salvador}\label{subsec:1.3.1}
\lipsum[50]
\subsection{Regime de terras}\label{subsec:1.3.2}
\lipsum[50]
\subsection{Agentes de produção do espaço urbano}\label{subsec:1.3.3}
\lipsum[50]
\subsubsection{Investidores privados}
\lipsum[50]
\subsubsection{Intervenções governamentais}
\lipsum[50]
\subsubsection{Intervenções da Intendência}
\lipsum[50]
\subsection{Reforma urbana (1906-1924)}
\lipsum[50]
\subsubsection{Antecedentes}
\lipsum[50]
\subsubsection{Plano de saneamento de Theodoro Sampaio}
\lipsum[50]
\subsubsection{Plano de melhoramentos de Alencar Lima}
\lipsum[50]
\subsubsection{O porto (1906-1912)}
\lipsum[50]
\subsubsection{As reformas do Centro (1912-1916)}
\lipsum[50]
\section{Conclusão}\label{sec:1.4}
\lipsum[50]
\chapter{Brotas: fronteira do urbano em Salvador}\label{cap:2}
\lipsum[50]
\section{Breve histórico e delimitação territorial}\label{sec:2.1}
\lipsum[50]
\section{Caracterização socioeconômica}\label{sec:2.2}
\lipsum[50]
\section{Esboço de caracterização fundiária}\label{sec:2.3}
\lipsum[50]
\section{Usos do espaço}\label{sec:2.4}
\lipsum[50]
\section{Como Salvador via o distrito}\label{sec:2.5}
\lipsum[50]
\subsection{Imprensa}\label{subsec:2.5.1}
\lipsum[50]
\subsubsection{Notícias policiais}\label{subsubsec:2.5.1.1}
\lipsum[50]
\subsubsection{Notícias políticas}\label{subsubsec:2.5.1.2}

\lipsum[50]

\subsection{Almanaques}\label{subsec:2.5.2}

\lipsum[50]

\subsubsection{Profissionais residentes}\label{subsubsec:2.5.2.1}

\lipsum[50]

\subsubsection{Serviços}\label{subsubsec:2.5.2.2}

\lipsum[50]

\chapter{Brotas e as reformas urbanas da República Velha}\label{cap:3}

\lipsum[50]

\section{Investimentos públicos}\label{sec:3.1}

\lipsum[50]

\subsection{Esgotamento e abastecimento de água}\label{subsec:3.1.1}

\lipsum[50]

\subsection{Arruamento e pavimentação}\label{subsec:3.1.2}

\lipsum[50]

\subsection{Iluminação pública}\label{subsec:3.1.3}

\lipsum[50]

\subsection{Transporte público}\label{subsec:3.1.4}

\lipsum[50]

\subsection{Cemitério}\label{subsec:3.1.5}

\lipsum[50]

\subsection{Serviços de saúde}\label{subsec:3.1.6}

\lipsum[50]

\subsection{Escolas}\label{subsec:3.1.7}

\lipsum[50]

\section{Ação privada}\label{sec:3.2}

\lipsum[50]

\subsection{Proprietários}\label{subsec:3.2.1}

\lipsum[50]

\subsection{Loteamentos}\label{subsec:3.2.2}

\lipsum[50]

\subsection{Construções, ampliações, reformas}\label{subsec:3.2.3}

\lipsum[50]
\cite{CARNEIRO1954}

\section{Ação recíproca: Brotas e as reformas, as reformas e Brotas}\label{sec:3.3}

\lipsum[50]
\cite{AZEVEDO1969}

\subsection{Houve influência da produção do espaço em Brotas sobre as reformas urbanas de Salvador?}\label{subsec:3.3.1}

\lipsum[50]
\cite{AGUIAR1957}

\subsection[Houve influência da produção do espaço pelas reformas de J. J. Seabra sobre Brotas?]{Houve influência da produção do espaço pelas reformas de J. J. Seabra sobre Brotas?}\label{subsec:3.3.2}

\lipsum[50]
\cite{AGUIAR1958}

% ----------------------------------------------------------
% Finaliza a parte no bookmark do PDF
% para que se inicie o bookmark na raiz
% e adiciona espaço de parte no Sumário
% ----------------------------------------------------------
\phantompart

% ---
% Conclusão (outro exemplo de capítulo sem numeração e presente no sumário)
% ---
\chapter*[Conclusão]{Conclusão}\label{concl}

\lipsum[50]

\addcontentsline{toc}{chapter}{Conclusão}

Escrever a conclusão

% ----------------------------------------------------------
% ELEMENTOS PÓS-TEXTUAIS
% ----------------------------------------------------------
\postextual
% ----------------------------------------------------------

% ----------------------------------------------------------
% Referências bibliográficas
% ----------------------------------------------------------
\bibliography{biblioteca}

% ----------------------------------------------------------
% Glossário
% ----------------------------------------------------------
%
% Consulte o manual da classe abntex2 para orientações sobre o glossário.
%
%\glossary

% ----------------------------------------------------------
% Apêndices
% ----------------------------------------------------------

% ---
% Inicia os apêndices
% ---
\begin{apendicesenv}

% Imprime uma página indicando o início dos apêndices
\partapendices

% ----------------------------------------------------------
\chapter{Tipo de apêndice}
% ----------------------------------------------------------

\lipsum[50]

% ----------------------------------------------------------
\chapter{Tipo de apêndice}
% ----------------------------------------------------------
\lipsum[55-57]

\end{apendicesenv}
% ---


% ----------------------------------------------------------
% Anexos
% ----------------------------------------------------------

% ---
% Inicia os anexos
% ---
\begin{anexosenv}

% Imprime uma página indicando o início dos anexos
\partanexos

% ---
\chapter{Tipo de documento}
\lipsum[50]
% ---



% ---
\chapter{Tipo de figura}
\lipsum[50]
% ---



% ---
\chapter{Tipo de facsímile}
\lipsum[50]
% ---



\end{anexosenv}

%---------------------------------------------------------------------
% INDICE REMISSIVO
%---------------------------------------------------------------------
\phantompart
\printindex
%---------------------------------------------------------------------

\end{document}
