% ---
% Agradecimentos
% ---
\begin{agradecimentos}
Numa pesquisa com apoio de instituições de fomento, os contribuintes e apoiadores são um vasto universo de pessoas anônimas, sequer sabedoras dos caminhos e descaminhos da investigação feita. Já numa pesquisa feita sem qualquer apoio de instituições de fomento à pesquisa, como a que se inicia com esta dissertação, os contribuintes e apoiadores são muito mais próximos, e cada pequeno gesto seu, cada pequena ação, tudo isto é importante demais para que passe sem o devido registro. Os agradecimentos neste caso são, necessariamente, abundantes. São o registro de que nem a mais solitária das pesquisas chega a bom termo sem o esforço coletivo, mesmo despretensioso, para que o pesquisador possa sistematizar conhecimento. São, igualmente, o registro de que uma pesquisa não se circunscreve aos limites de um curso de pós-graduação, mas resultam de inquietações de uma vida inteira, condensadas numa pesquisa que só parcialmente as satisfazem, e na verdade as revolvem e aprofundam.

Agradeço, em primeiro lugar, à professora Odete Dourado pela recepção acolhedora a um projeto que parecia fadado ao fracasso, e pela orientação segura, propositiva, paciente e, sobretudo, amistosa. Quaisquer falhas no processo são de minha inteira responsabilidade. Agradeço também à professora Ângela Gordilho pelo apoio, atenção e debates quentes; e à professora Any Brito Leal Ivo, pela atenção e pelas dicas acadêmicas e burocráticas. Ainda no corpo docente do PPG-AU, agradeço aos professores Marco Aurélio Gomes, Rodrigo Baeta, Heloísa Petti, Heliodório Sampaio, Ana Fernandes, Ângela Franco, Márcio Campos, Márcia Sant'anna e Fernando Ferraz pela saudável convivência no curso das disciplinas da pós-graduação.

Em segundo lugar, agradeço ao Centro de Estudos e Ação Social (CEAS) não apenas por ser a melhor organização à qual já vendi minha força de trabalho, mas igualmente por ultrapassar este papel e ser um espaço onde trabalhar questões fundamentais para o fortalecimento dos movimentos sociais de luta por moradia em Salvador; além disso, por custodiar uma das melhores bibliotecas especializadas em ciências humanas da Bahia e, franqueando-me a ela acesso, fermentar em mim dúvidas e questões que agora tomam a forma de trabalho acadêmico. Agradeço a meus companheiros de trabalho presentes e passados pelo estimulante convívio desde outubro de 2006, e especialmente a Joaci, por franquear acesso ao acervo digital pessoal construído durante sua pesquisa de doutorado, e a Nélia e Patrícia pelas preciosas dicas de pesquisa arquivística e pelas sutis correções de rumo que me indicaram.

Em terceiro lugar, agradeço a meus colegas de turma no PPG-AU, especialmente Gabriel Ramos, Maria Simone Soares, João Marques, Aline Tosta, Carlos Quinto, Lumena Addad, Maurício Felzemburgh, Santiago Cao, Milena Migliano e tantos outros, pelo convívio e pelas conversas onde muitas vezes as minhas dúvidas elucidavam as suas, e as suas as minhas.

Em quarto lugar, agradeço àqueles cuja amizade fui conquistando como um bônus das ações políticas que desenvolvemos juntos em tantos lugares. Em especial agradeço ao amigo João Bernardo Maia Viegas Soares pelos novos caminhos que mostra em cada conversa e pela notável firmeza, coerência e rigor intelectuais que marcam toda sua vida.

Em quinto lugar, agradeço às equipes dos arquivos e bibliotecas que consultei, pelo árduo trabalho de conservação da memória num tempo, num país e numa cidade geridos sob o signo do efêmero. Agradeço especialmente às equipes do Arquivo Público Municipal de Salvador, onde localizei os documentos essenciais para a caracterização fundiária exigida por esta pesquisa, e do Arquivo Público do Estado da Bahia, onde localizei inventários e outros documentos complementares à caracterização fundiária.

Em sexto lugar, agradeço --- um tanto anonimamente, é verdade --- aos fautores da trilha sonora que animou a pesquisa de base e a redação desta dissertação: Sikiru Adepoju, Carlos Alomar, Bülent Arel, Itamar Assumpção, Blixa Bargeld, Adrian Belew, Bezerra da Silva, José Mário Branco, Bill Bruford, Joseph Byrd, Chris Cutler, Holger Czukay, Halim Abdul Messieh El-Dabh, Rogério Duprat, Brian Eno, Leo Ferré, Robert Fripp, Fred Frith, Kim Gordon, Mozart Camargo Guarnieri, César Guerra-Peixe, Hans-Joachim Koellreutter, Edino Krieger, Fela Kuti, Toni Levin, Jaki Liebezeit, György Ligeti, Otto Luening, Antônio José Santana Martins, Joe Meek, Olivier Messiaen, Thurston Moore, Enio Morricone, Conlon Nancarrow, Babatunde Olatunji, Krzysztof Penderecki, Lee Ranaldo, Pierre Schaeffer, Lalo Schifrin, David Shire, Karlheinz Stockhausen, Demetrio Stratos, Chris Squire, Gábor Szabó, Univers Zero, Vladimir Ussachevsky, Edgard Varèse, Scott Walker, Tony Williams. 

Por último e mais importante, agradeço a minha família. A meu pai, Manoel Maria do Nascimento (\textit{in memoriam}), por estimular desde pequeno em mim o gosto pela leitura. A minha mãe, Florinda Lordello da Silva (\textit{in memoriam}), por estimular desde pequeno em mim o gosto pelas coisas práticas da vida. A minha avó, Neyde Lordello da Silva (\textit{in memoriam}), por superar suas próprias limitações e me ensinar as primeiras letras. A meus tios, Elba Celeste Lordello Dias e Geovane da Silva Dias, pelo cuidado nas horas de maior necessidade. A Rosa e Fernando, pela curiosidade ingênua com que acompanharam as ``tarefas de casa'' do pai. E, por fim, a Fernanda, por tudo, desde sempre.

A todos, meus mais sinceros agradecimentos.

\end{agradecimentos}
% ---