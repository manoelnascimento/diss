% ---
% RESUMOS
% ---

% resumo em português
\setlength{\absparsep}{18pt} % ajusta o espaçamento dos parágrafos do resumo
\begin{resumo}
Como os conflitos sociais característicos da Primeira República brasileira (1889-1930) e da política baiana coetânea interferiram na produção, apropriação e uso do espaço urbano no distrito soteropolitano de Brotas? Que parte coube a Brotas na descentralização urbana de Salvador ocorrida durante a Primeira República, que marcou o desenvolvimento urbano de Salvador no século XX? Esta pesquisa responde a estas perguntas, ainda que provisoriamente, tendo como base um amplo repertório de fontes, entre as quais destaca-se a totalidade dos pedidos de licenças para obras no distrito, custodiados pelo Arquivo Histórico Municipal de Salvador. Durante a Primeira República, ultimou-se o processo de desagregação fundiária das antigas fazendas e latifúndios de Brotas, dando origem a muitas das identidades territoriais que hoje marcam esta área de Salvador. Este processo foi atravessado pela proliferação, em áreas demarcadas por traços geográficos específicos, de ``moradias para operários'', de loteamentos irregulares, de ``cidades balneárias''; pela quieta permanência de antigos posseiros e pequenos roceiros, não raro libertos ou seus descendentes, em paralelo à luta de profissionais liberais, pequenos comerciantes e funcionários públicos de pequeno e médio escalões pela instalação de infraestruturas urbanas próximas às suas moradias. As intervenções urbanísticas impulsionadas pelo governador baiano José Joaquim Seabra, pelas quais a urbanização soteropolitana durante a Primeira República é mais conhecida, interferiram apenas tangencialmente no desenvolvimento urbano de Brotas, especialmente por serem causa forte de migrações intraurbanas e interdistritais pautadas pelo baixo valor da terra e pela proximidade do distrito dos postos de trabalho industriais, comerciais e portuários, facilitada pelas linhas de bonde. Espera-se, com esta pesquisa, ter sido possível explicar por que os conflitos e lutas sociais pela produção, apropriação e uso do espaço urbano na Primeira República contribuíram para fazer de Brotas o que é hoje -- de caso pensado, não ``espontaneamente''. 

\vspace{\onelineskip}
 
   \noindent 

 \textbf{Palavras-chaves}: Conflitos sociais. Espaço urbano. Salvador (Bahia). Primeira República (Brasil, 1889-1930).
\end{resumo}

% resumo em inglês
\begin{resumo}[Abstract]
 \begin{otherlanguage*}{english}
Did the social conflicts characteristic of the First Brazilian Republic and those of Bahian politics of the same period influence the production, appropriation and use of urban space in the Brotas soteropolitan district? What was Brotas' share in Salvador's urban decentralization process during the First Republic, which marked Salvador's urban development throughout the twentieth century? This dissertation tries to answer these questions, albeit tentatively, based on a broad repertoire of sources, including the totality of license applications for building, extension and reform works in the district. During the First Republic, the ownership fragmentation of ancient farms, mills and other large estates of Brotas was almost completed, giving rise to territorial identities that today mark this area. This process distinguished itself by the proliferation, in areas demarcated by specific geographic features, of ``houses for workers'', irregular allotments and ``balneary towns''; by the quiet permanence of former squatters and small tenants -- often recently freed ex-slaves or their descendants -- in parallel to the struggle of liberal professionals, small merchants and small- and medium-rank civil servants for the implementation of urban infrastructures near their homes. The urbanistic interventions promoted by the Bahian governor José Joaquim Seabra, for which the soteropolitan urbanization during the First Republic is better known, interfered only tangentially in the urban development of Brotas, mostly because these reforms were a strong cause of intraurban and interdistrict migrations, eased in Brotas by the low land value and the proximity, facilitated by tram lines, to industrial, commercial and port jobs. The author hopes to have explained why conflicts and social struggles for the production, appropriation and use of urban space in Salvador during the First Republic contributed to make Brotas what it is today -- purposedly, not ``spontaneously''.

   \vspace{\onelineskip}
 
   \noindent 
   \textbf{Key-words}: Social conflicts. Urban space. Salvador (Bahia). First Republic (Brazil, 1889-1930).
 \end{otherlanguage*}
\end{resumo}

% resumo em francês 
% \begin{resumo}[Résumé]
% \begin{otherlanguage*}{french}
%    Il s'agit d'un résumé en français.
% 
%   \textbf{Mots-clés}: Conflits sociaux. Espace urbain. Salvador (Bahia). Première République (Brésil, 1889-1930).
% \end{otherlanguage*}
%\end{resumo}

% resumo em espanhol
\begin{resumo}[Resumen]
 \begin{otherlanguage*}{spanish}
¿Cómo los conflictos sociales característicos de la Primera República Brasileña (1889-1930) y de la política bahiana de la misma época interfieren en la producción, apropiación y uso del espacio urbano en el distrito soteropolitano de Brotas? ¿Cuál fue la parte de Brotas en el proceso de descentralización urbana de Salvador que tuvo lugar durante la Primera República, que marcó el desarrollo urbano de Salvador a lo largo del siglo XX? Esta disertación responde a estas preguntas, aunque de manera tentativa, con base en un amplio repertorio de fuentes, incluida la totalidad de las solicitudes de permisos para obras en el distrito, custodiado por el Archivo Histórico Municipal de Salvador. Durante la Primera República, se completó el proceso de desagregación de la tierra de las antiguas granjas y fincas de Brotas, dando lugar a muchas de las identidades territoriales que hoy marcan esta área de Salvador. Este proceso se caracterizó por la proliferación, en áreas delimitadas por características geográficas específicas, de "viviendas para trabajadores", de asignaciones irregulares, de "pueblos costeros"; por la permanencia silenciosa de antiguos ocupantes ilegales y pequeños terratenientes, a menudo esclavos recién liberados o sus descendientes, en paralelo a la lucha de profesionales liberales, pequeños comerciantes y funcionarios públicos de pequeña y mediana escala para la instalación de infraestructuras urbanas cerca de sus hogares. Las intervenciones urbanísticas promovidas por el gobernador de Bahía José Joaquim Seabra, por las cuales la urbanización soteropolitana durante la Primera República es más conocida, solo interfirieron tangencialmente en el desarrollo urbano de Brotas, especialmente porque son una causa importante de migraciones intraurbanas e interdistritales basadas en el bajo valor de la tierra y en la proximidad de los trabajos industriales, comerciales y portuarios facilitados por las líneas de tranvía. Se espera, con esta investigación, que fuera posible explicar por qué los conflictos y las luchas sociales para la producción, apropiación y uso del espacio urbano en la Primera República contribuyeron a hacer de Brotas lo que es hoy -- conscientemente, no ``espontáneamente''.
  
   \textbf{Palabras clave}: Conflictos sociales. Espacio urbano. Salvador (Bahia). Primera República (Brasil, 1889-1930).
 \end{otherlanguage*}
\end{resumo}
% ---