% ---
% Epígrafe
% ---

\begin{epigrafe}
    \vspace*{\fill}
	\begin{flushright}
		\textit{O conflito é a causa e o arquiteto de todas as coisas. De uns faz deuses, de outros homens, de uns escravos, de outros cidadãos.}
		
		\textbf{Heráclito de Éfeso}
	\end{flushright}

	\begin{flushright}
		\textit{Gostaria de dizer coisas mais simples e mais abertas, mas falta-me uma das duas condições necessárias: confiar no futuro ou ser completamente desiludido.}
		
		\textbf{Artur Castro Neves}
	\end{flushright}
	\begin{flushright}
		\textit{Curiosa concepção de cidade onde uma freguesia de povoamento esparso, caracterizada por funções econômicas tipicamente rurais, e isto até meados do século XIX, acede à condição de núcleo urbano!}
		
		\textbf{Kátia M. de Queirós Mattoso}
	\end{flushright}
\end{epigrafe}
% ---