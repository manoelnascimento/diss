% ---
% Casa mista de negócios do capitão Valentin Duran Suarez na rua das Pitangueiras
% ---

\afterpage{
\begin{a3paisagem}
\begin{figure}[!h]
\centering
\caption{Projecto para a construção de uma casa na rua do Dr. Agrippino Dorea nº 22, districto de Brotas, pertencente ao Ilmo. Snr. Capitão Valentin Duran Suarez (1908). }
\includegraphics[height=0.9\textheight]{8-anexos/plantas/01-1distrito/04-agripino-dorea/agripinodorea-valentinduransuarez.jpg}{\footnotesize \par \textbf{Fonte:} \textbf{BR BAAHMS}, Fundo ``Intendência'', Série ``Processos de Licenciamento de Reforma e Ampliação de Edificações'', Subsérie ``Requerimentos e Plantas -- Brotas'', caixa 040. \par Projeto de Antonio Leite da Luz. Este projeto destacou-se entre todos pela intensa combinação no mesmo imóvel de uso residencial e comercial: o térreo abre-se com uma ``Area para negocio'' e  ``Bilhar'', separados por um corredor das duas salas, dispensa, cozinha e banheiro existentes no térrero, havendo ainda mais duas salas, cinco quartos, uma alcova e uma capela no ``pavimento nobre''. }
\label{fig:valentinduransuarez}
\end{figure}
\end{a3paisagem}
}

% ---
% Planta da casa de Manoel Amoedo y Amoedo, na Uruguaiana, ainda de pé
% ---

\afterpage{
\begin{a3paisagem}
\begin{figure}[!h]
\centering
\caption{Projecto para a construcção do predio sito entre as ruas Uruguayana e Asylo São João de Deus, no districto de Brotas, pertencente ao Snr. Manoel Amoedo e Amoedo (1909)}
\includegraphics[width=\textwidth]{8-anexos/plantas/02-boavista/01-uruguayana/1909-manoelamoedoyamoedo-casa.jpg}{\footnotesize \par \textbf{Fonte:} \textbf{BR BAAHMS}, Fundo ``Intendência'', Série ``Processos de Licenciamento de Reforma e Ampliação de Edificações'', Subsérie ``Requerimentos e Plantas -- Brotas'', caixa 01. \par No projeto original, o térreo da casa tinha destinação comercial (armazém e bilhares), enquanto o andar nobre tinha alcova, sala de jantar, gabinete, três quartos, sala de jantar, varanda, cozinha, dispensa, banheiro e \textit{water closet}. Esta casa ainda está de pé}
\label{fig:1909-manoelamoedoyamoedo-casa}
\end{figure}
\end{a3paisagem}
}