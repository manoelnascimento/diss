% ---
% Padaria de Domingos de Oliveira Reis, na Mariquita
% ---

\afterpage{
\begin{a3paisagem}
\begin{figure}[!h]
\centering
\caption{Projecto para a construcção de uma casa e forno de padaria, à rua da Mariquita, Rio Vermelho (1914)}
\includegraphics[height=0.9\textheight]{8-anexos/plantas/05-mariquita/06-mariquita/1914-domingosdeoliveirareis.jpg}{\footnotesize \par \textbf{Fonte:} \textbf{BR BAAHMS}, Fundo ``Intendência'', Série ``Processos de Licenciamento de Reforma e Ampliação de Edificações'', Subsérie ``Requerimentos e Plantas -- Brotas'', caixa 06. \par Projeto de Arthur Santos. O mesmo proprietário tinha uma casa de térreo e primeiro pavimento na Mariquita, vizinha à padaria.}
\label{fig:1914-domingosdeoliveirareis}
\end{figure}
\end{a3paisagem}
}

% ---
% Bilhar de Manoel Amoedo y Amoedo, na Uruguaiana
% ---

\afterpage{
\begin{a3paisagem}
\begin{figure}[!h]
\centering
\caption{Projecto para a construcção de uma casa terrea para bilhares, no terreno baldio sito na rua Uruguayana, districto de Brotas, pertencente ao Ilmo. Snr. Manoel Amoedo (1909)}
\includegraphics[height=0.9\textheight]{8-anexos/plantas/02-boavista/01-uruguayana/1909-manoelamoedoyamoedo.jpg}{\footnotesize \par \textbf{Fonte:} \textbf{BR BAAHMS}, Fundo ``Intendência'', Série ``Processos de Licenciamento de Reforma e Ampliação de Edificações'', Subsérie ``Requerimentos e Plantas -- Brotas'', caixa 01. \par Projeto de Rosalvo Celestino dos Santos. A existência de um salão de bilhar indica o grau de urbanização da rua Uruguaiana; as exigências de lazer de seus moradores oportunizaram a este espanhol, comerciante e terratenente (com várias casas na rua das Pitangueiras e adjacências), explorar também este ramo de negócios.}
\label{fig:1909-manoelamoedoyamoedo}
\end{figure}
\end{a3paisagem}
}