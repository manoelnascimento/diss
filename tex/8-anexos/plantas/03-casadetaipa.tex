% ---
% Casa de Theodomiro José Veríssimo, na rua do Céu
% ---

\afterpage{
\begin{figure}[!h]
\centering
\caption{Projecto para a construcção de uma casa de taipa que Theodomiro José Verissimo pretende fazer na rua do Céo no Rio Vermelho districto de Brotas (1906)}
\includegraphics[width=1\textwidth]{8-anexos/plantas/05-mariquita/03-ruadoceu/theodomirojoseverissimo-1906.jpg}{\footnotesize \par \textbf{Fonte:} \textbf{BR BAAHMS}, Fundo ``Intendência'', Série ``Processos de Licenciamento de Reforma e Ampliação de Edificações'', Subsérie ``Requerimentos e Plantas -- Brotas'', caixa 03. \par Projeto de Custódio Bandeira. Casas de taipa ainda eram toleradas pela Diretoria de Obras na primeira década do século XX. O próprio desenhista ``relaxou'' no projeto, e não apresentou as medidas de todos os cômodos. }
\label{fig:theodomirojoseverissimo-1906}
\end{figure}
}

% ---
% Casa de Úrsula das Virgens Guaresma, na Cruz da Redenção
% ---

\afterpage{
\begin{figure}[!h]
\centering
\caption{Projecto para a construcção de uma pequena casa que D. Ursula das Virgens Guaresma pretende fazer na rua da Cruz da Redempção no districto de Brotas (1903)}
\includegraphics[width=1\textwidth]{8-anexos/plantas/03-estbrotas/17-cruzdaredencao/1903-ursuladasvirgensguaresma.jpg}{\footnotesize \par \textbf{Fonte:} \textbf{BR BAAHMS}, Fundo ``Intendência'', Série ``Processos de Licenciamento de Reforma e Ampliação de Edificações'', Subsérie ``Requerimentos e Plantas -- Brotas'', caixa 15. \par Projeto de autor desconhecido. Outro projeto ``relaxado'', sem várias convenções exigidas pela Diretoria de Obras, conforme a legislação vigente, para projetos de outras casas. A casa é simplesmente minúscula, um cubículo com um aposento no meio e uma cozinha no fundo. }
\label{fig:1903-ursuladasvirgensguaresma}
\end{figure}
}

\afterpage{}