% ---
% Casa de José Cardozo da Silva na Ubarana
% ---

\afterpage{
\begin{a3paisagem}
\begin{figure}[!h]
\centering
\caption{Projecto da casa que o Ilmº. Snr. José Cardozo da Silva pretende construir em seu terreno, sito à Ubarana, no districto de Brotas (1915) }
\includegraphics[height=0.9\textheight]{8-anexos/plantas/09-amaralinapituba/19-amaralina/dsc04791.jpg}{\footnotesize \par \textbf{Fonte:} \textbf{BR BAAHMS}, Fundo ``Intendência'', Série ``Processos de Licenciamento de Reforma e Ampliação de Edificações'', Subsérie ``Requerimentos e Plantas -- Brotas'', caixa 19. \par Projeto de Archimedes Marques. Dois quartos, duas salas (``visita'' e ``jantar''), cozinha e banheiro nos fundos, fachada discretamente ornada: repete-se aqui o desenho comum para as casas de médio valor locativo no distrito. }
\label{fig:dsc04791}
\end{figure}
\end{a3paisagem}
}

% ---
% Casa de Prediliano Pitta, na rua do Lucaia
% ---

\afterpage{
\begin{figure}[!h]
\centering
\caption{Projecto para a construcção de uma casa em terreno baldio, sito à Lucaia freguesia de Brotas, pertencente ao Snr. Prediliano Pereira Pitta (1911)}
\includegraphics[width=1\textwidth]{8-anexos/plantas/05-mariquita/06-lucaia/predilianopitta-1911.jpg}{\footnotesize \par \textbf{Fonte:} \textbf{BR BAAHMS}, Fundo ``Intendência'', Série ``Processos de Licenciamento de Reforma e Ampliação de Edificações'', Subsérie ``Requerimentos e Plantas -- Brotas'', caixa 06. \par Não há assinatura de desenhista ou engenheiro/arquiteto, mas a caligrafia parece-se com a de Rosalvo Celestino dos Santos. }
\label{fig:predilianopitta-1911}
\end{figure}
}

% ---
% Duas casas de Maria Rosa Vianna Ferraz na rua do Castro Neves
% ---

\afterpage{
\begin{a3paisagem}
\begin{figure}[!h]
\centering
\caption{Projecto para modificação das fachadas dos predios nº 81 e 83 a rua Castro Neves districto de Brotas (1929). }
\includegraphics[height=0.9\textheight]{8-anexos/plantas/01-1distrito/16-castroneves/mariarosaviannaferraz-83e81-1929.jpg}{\footnotesize \par  Projeto de Lycerio Alfredo Schreiner. \textbf{Fonte:} \textbf{BR BAAHMS}, Fundo ``Intendência'', Série ``Processos de Licenciamento de Reforma e Ampliação de Edificações'', Subsérie ``Requerimentos e Plantas -- Brotas'', caixa 16. Uma das duas casas de Maria Rosa Vianna Ferraz permanece de pé. }
\label{fig:mariarosaviannaferraz1929}
\end{figure}
\end{a3paisagem}
}