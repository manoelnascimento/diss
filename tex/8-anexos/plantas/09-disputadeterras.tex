% ---
% Conflito de terras na Quinta das Beatas
% ---

\afterpage{
\begin{a3paisagem}
\begin{figure}[!h]
\centering
\caption{Croquis mostrando a cerca de arame feita recentemente, que intercepta a passagem da estrada que vai do Matatu a Brotas, pelas Quintas das Beatas (1920) }
\includegraphics[height=0.9\textheight]{8-anexos/plantas/06-matatu/21-matatugrandepequeno/1920-conflitodeterras-matatu.jpg}{\footnotesize \par \textbf{Fonte:} \textbf{BR BAAHMS}, Fundo ``Intendência'', Série ``Processos de Licenciamento de Reforma e Ampliação de Edificações'', Subsérie ``Requerimentos e Plantas -- Brotas'', caixa 21. \par Autor desconhecido. A situação dos proprietários limítrofes dá a entender que o ``riacho divisorio'' é o rio Bonocô, ou rio Campinas. }
\label{fig:dsc04791}
\end{figure}
\end{a3paisagem}
}

% ---
% Conflito de terras no Imbuhy
% ---

\afterpage{
\begin{a3paisagem}
\begin{figure}[!h]
\centering
\caption{Planta dos terrenos do Imbuhy requeridos à Camara Municipal por aforamento por Pedro Dias dos Santos (1910) }
\includegraphics[width=\textwidth]{8-anexos/plantas/10-armacoes/24-imbuhy/1910-imbuhy.jpg}{\footnotesize \par \textbf{Fonte:} \textbf{BR BAAHMS}, Fundo ``Intendência'', Série ``Processos de Licenciamento de Reforma e Ampliação de Edificações'', Subsérie ``Requerimentos e Plantas -- Brotas'', caixa 24. \par Autor desconhecido. }
\label{fig:1912-imbuhy}
\end{figure}
\end{a3paisagem}
}