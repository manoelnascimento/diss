% ---
% Casa de Theodomiro José Veríssimo, na rua do Céu
% ---

\afterpage{
\begin{a3paisagem}
\begin{figure}[!h]
\centering
\caption{Planta para construcção de uma casa que Bernardo Martins da Silva pretende construir no terreno baldio sito a rua Direita da Mariquita ao Rio Vermelho (1900)}
\includegraphics[height=0.9\textheight]{8-anexos/plantas/05-mariquita/06-mariquita/1900-bernardomartinsdasilva.jpg}{\footnotesize \par \textbf{Fonte:} \textbf{BR BAAHMS}, Fundo ``Intendência'', Série ``Processos de Licenciamento de Reforma e Ampliação de Edificações'', Subsérie ``Requerimentos e Plantas -- Brotas'', caixa 06. \par Projeto de desenhista e arquiteto/engenheiro desconhecidos. Casas de veraneio como esta foram comuns na rua Direita da Mariquita. }
\label{fig:1900-bernardomartinsdasilva}
\end{figure}
\end{a3paisagem}
}

% ---
% Casa de praia de A. Saffrey na Cidade Balneária Amaralina
% ---

\afterpage{
\begin{a3paisagem}
\begin{figure}[!h]
\centering
\caption{Propriedade de A. Saffrey, architecto, Amaralina. Lote 116 da planta geral da cidade d'Amaralina (1915) }
\includegraphics[height=0.9\textheight]{8-anexos/plantas/09-amaralinapituba/19-amaralina/dsc04777.jpg}{\footnotesize \par \textbf{Fonte:} \textbf{BR BAAHMS}, Fundo ``Intendência'', Série ``Processos de Licenciamento de Reforma e Ampliação de Edificações'', Subsérie ``Requerimentos e Plantas -- Brotas'', caixa 19. \par Projeto de A. Saffrey. A planta de situação mostra como esta grande casa de praia (121,50$m^{2}$), construída ao que parece em meio a um belo jardim, situava-se no lote 116 da Cidade Balneária Amaralina --- lote grande (525$m^{2}$), que não foi totalmente empregue no projeto. }
\label{fig:dsc04777}
\end{figure}
\end{a3paisagem}
}

% ---
% Casa de Lydia Dewald na Cidade Balneária Amaralina
% ---

\afterpage{
\begin{a3paisagem}
\begin{figure}[!h]
\centering
\caption{Projeto de uma casa em Amaralina de D. Lydia Dewald (1915) }
\includegraphics[height=0.9\textheight]{8-anexos/plantas/09-amaralinapituba/19-amaralina/DSC04783.jpg}{\footnotesize \par \textbf{Fonte:} \textbf{BR BAAHMS}, Fundo ``Intendência'', Série ``Processos de Licenciamento de Reforma e Ampliação de Edificações'', Subsérie ``Requerimentos e Plantas -- Brotas'', caixa 19. \par Desenho de Ciro Spínola, engenheiro/arquiteto desconhecido. Encontra-se bastante danificado pelo mofo este projeto de casa com três quartos, sala, dependências (uma delas o infame ``quarto de empregada''), copa, cozinha e pátio interno, o que impediu sua reprodução integral. Salvou-se entretanto a fachada, que destaca-se nesta área pelas duas escadarias abalaustradas. }
\label{fig:DSC04783}
\end{figure}
\end{a3paisagem}
}

% ---
% Casa de Chehadi Elias Kraychete na Cidade Balneária Amaralina
% ---

\afterpage{
\begin{a3paisagem}
\begin{figure}[!h]
\centering
\caption{Projecto de uma casa que o sr. Chehadi E. Kraycheti quer construir em Amaralina (1923)}
\includegraphics[width=\textwidth]{8-anexos/plantas/09-amaralinapituba/19-amaralina/dsc4805.jpg}{\footnotesize \par \textbf{Fonte:} \textbf{BR BAAHMS}, Fundo ``Intendência'', Série ``Processos de Licenciamento de Reforma e Ampliação de Edificações'', Subsérie ``Requerimentos e Plantas -- Brotas'', caixa 19. \par Projeto de Eurico da Costa Coutinho. O projeto de pouco adorno esconde uma casa de porte médio para grande, com duas salas e quartos de 12,58$m^{2}$ --- além do quarto ``dos fundos'', em frente ao sanitário e à cozinha, que terá servido a algum empregado doméstico. }
\label{fig:dsc4805}
\end{figure}
\end{a3paisagem}
}

% ---
% Mansão de João da Cunha Freire (planta baixa)
% ---

\afterpage{
\begin{a3paisagem}
\begin{figure}[!h]
\centering
\caption{Projecto para construcção do predio sito na Amaralina, propriedade do Ilmº Snr. João da Cunha Freire (1925), primeira parte  }
\includegraphics[height=0.9\textheight]{8-anexos/plantas/09-amaralinapituba/19-amaralina/dsc04813.jpg}{\footnotesize \par Projeto de Rossi Baptista. \textbf{Fonte:} \textbf{BR BAAHMS}, Fundo ``Intendência'', Série ``Processos de Licenciamento de Reforma e Ampliação de Edificações'', Subsérie ``Requerimentos e Plantas -- Brotas'', caixa 19. Verdadeira mansão adventícia, tem no primeiro andar \textit{cinco} quartos grandes (14,4$m^{2}$ o menor deles) com ``W.C.'' exclusivo, além de \textit{quatro} quartos ``dos fundos'' no térreo e enormes varanda e terraço. }
\label{fig:dsc04813}
\end{figure}
\end{a3paisagem}
}

% ---
% Mansão de João da Cunha Freire (fachada)
% ---

\afterpage{
\begin{a3paisagem}
\begin{figure}[!h]
\centering
\caption{Projecto para construcção do predio sito na Amaralina, propriedade do Ilmº Snr. João da Cunha Freire (1925), segunda parte }
\includegraphics[width=\textwidth]{8-anexos/plantas/09-amaralinapituba/19-amaralina/dsc04814.jpg}{\footnotesize \par Projeto de Rossi Baptista. \textbf{Fonte:} \textbf{BR BAAHMS}, Fundo ``Intendência'', Série ``Processos de Licenciamento de Reforma e Ampliação de Edificações'', Subsérie ``Requerimentos e Plantas -- Brotas'', caixa 19. \par Apesar de bastante danificada a planta, ainda é possível reconhecer elementos da fachada que permitem identificá-la na \autoref{fig:1930-amaralina02} (p. \pageref{fig:1930-amaralina02}). }
\label{fig:dsc04814}
\end{figure}
\end{a3paisagem}
}

% ---
% Casa de Elias Abdon
% ---

\afterpage{
\begin{a3paisagem}
\begin{figure}[!h]
\centering
\caption{Projecto para construcção de uma casa em Amaralina (1930) }
\includegraphics[width=\textwidth]{8-anexos/plantas/09-amaralinapituba/19-amaralina/amaralina-eliasabdon-1930.jpg}{\footnotesize \par \textbf{Fonte:} \textbf{BR BAAHMS}, Fundo ``Intendência'', Série ``Processos de Licenciamento de Reforma e Ampliação de Edificações'', Subsérie ``Requerimentos e Plantas -- Brotas'', caixa 19. \par Projeto de Rossi Baptista. Esta casa aparece na \autoref{fig:1930-amaralina02} (p. \pageref{fig:1930-amaralina02}), o que permite verificar a magnitude do projeto. }
\label{fig:amaralina-eliasabdon}
\end{figure}
\end{a3paisagem}
}