% ---
% Cinco casas de Manoel Amoedo y Amoedo na rua das Pitangueiras
% ---

\afterpage{
\begin{a3paisagem}
\begin{figure}[!h]
\centering
\caption{Projecto de construção para cinco casas na Rua 25 de Março, freguezia de Brotas (1892). }
\includegraphics[width=1\textwidth]{8-anexos/plantas/01-1distrito/04-agripino-dorea/agripinodorea-manoelamoedoyamoedo-1892.jpg}{\footnotesize \par \textbf{Fonte:} \textbf{BR BAAHMS}, Fundo ``Intendência'', Série ``Processos de Licenciamento de Reforma e Ampliação de Edificações'', Subsérie ``Requerimentos e Plantas -- Brotas'', caixa 04. \par Projeto de José Celestino dos Santos. O mais antigo entre todos os projetos pesquisados é exatamente o de construção de pequenas casas para alugar na rua das Pitangueiras, de propriedade do comerciante espanhol Manoel Amoedo y Amoedo. }
\label{fig:manoelamoedoyamoedo}
\end{figure}
\end{a3paisagem}
}

% ---
% Seis casas de Antonio Ribeiro da Cunha na rua da Alegria do Castro Neves
% ---

\afterpage{
\begin{a3paisagem}
\begin{figure}[!h]
\centering
\caption{Projecto para a construção de seis pequenas casas que o Snr. Antonio Ribeiro da Cunha pretende fazer na rua da Alegria districto de Brotas (1897). }
\includegraphics[width=1\textwidth]{8-anexos/plantas/01-1distrito/10-alegria/alegriadocastroneves-antonioribeirodacunha-seiscasas.jpg}{\footnotesize \par \textbf{Fonte:} \textbf{BR BAAHMS}, Fundo ``Intendência'', Série ``Processos de Licenciamento de Reforma e Ampliação de Edificações'', Subsérie ``Requerimentos e Plantas -- Brotas'', caixa 10. \par Projeto de autor desconhecido. Casas geminadas bem à moda das ``casas para operários'' que lhe sucederam: pequenas, desadornadas, e sobretudo baratas. }
\label{fig:antonioribeirodacunha}
\end{figure}
\end{a3paisagem}
}

% ---
% Vinte casas de José Antonio Ramos na rua do Castro Neves
% ---

\afterpage{
\begin{a3paisagem}
\begin{figure}[!h]
\centering
\caption{Projecto para a construção de 20 casas, para operarios, ao Castro Neves, districto de Brotas, que pretende fazer o snr. José Antonio Ramos (1919). }
\includegraphics[height=0.9\textheight]{8-anexos/plantas/01-1distrito/16-castroneves/joseantonioramos-20casas.jpg}{\footnotesize \par \textbf{Fonte:} \textbf{BR BAAHMS}, Fundo ``Intendência'', Série ``Processos de Licenciamento de Reforma e Ampliação de Edificações'', Subsérie ``Requerimentos e Plantas -- Brotas'', caixa 16. \par Projeto de Arthur Santos. Exceção à regra entre as ``casas para operários'' no que diz respeito ao apuro estético da fachada, estas casas, hoje inexistentes, foram verdadeiros cubículos, a julgar pela sua planta baixa. }
\label{fig:joseantonioramos-20casas}
\end{figure}
\end{a3paisagem}
}

% ---
% Oito casas de Crispiniano João Rogelio (parte 1)
% ---

\afterpage{
\begin{a3paisagem}
\begin{figure}[!h]
\centering
\caption{Projecto para construção de oito casas à Amaralina, primeira parte (1925)}
\includegraphics[height=0.9\textheight]{8-anexos/plantas/09-amaralinapituba/19-amaralina/dsc04810.jpg}{\footnotesize \par Projeto de Pedro Jayme David. \textbf{Fonte:} \textbf{BR BAAHMS}, Fundo ``Intendência'', Série ``Processos de Licenciamento de Reforma e Ampliação de Edificações'', Subsérie ``Requerimentos e Plantas -- Brotas'', caixa 19. \par Trata-se de oito cubículos de 38.75$m^{2}$ cada (sem contar a cozinha e o ``W.C.'' anexos), com dois quartos de 8,75$m^{2}$ e duas ``salas'' de 7,7$m^{2}$, com fachada desadornada. Com este tamanho, a julgar pela descrição do lote visto na \autoref{fig:dsc04777}, este projeto demonstra terem sido concebidas estas pequenas casas pelo requerente Crispiniano João Rogelio e por Pedro Jayme David no tamanho certo para caberem num só lote da Cidade Balneária Amaralina. }
\label{fig:dsc04810}
\end{figure}
\end{a3paisagem}
}

% ---
% Oito casas de Crispiniano João Rogelio (parte 2)
% ---

\afterpage{
\begin{a3paisagem}
\begin{figure}[!h]
\centering
\caption{Projecto para construção de oito casas à Amaralina, segunda parte (1925) }
\includegraphics[width=\textwidth]{8-anexos/plantas/09-amaralinapituba/19-amaralina/dsc04809.jpg}{\footnotesize \par Projeto de Pedro Jayme David. \par \textbf{Fonte:} \textbf{BR BAAHMS}, Fundo ``Intendência'', Série ``Processos de Licenciamento de Reforma e Ampliação de Edificações'', Subsérie ``Requerimentos e Plantas -- Brotas'', caixa 19. A segunda parte do projeto proposto por Crispiniano João Rogelio mostra que os cubículos desenhados por Pedro Jayme David compõem uma pequena avenida de casas. Como não se trata de casas para aluguel a veranistas, nem tampouco seriam tais cubículos atrativos ao habitual comprador de imóveis em Amaralina, fica evidente que trata-se de ``casas para operários'', muito provavelmente atendendo à demanda de trabalhadores para o serviço doméstico nas casas adventícias circunvizinhas.}
\label{fig:dsc04809}
\end{figure}
\end{a3paisagem}
}

% ---
% Vila operária da União Fabril
% ---

\afterpage{
\begin{a3paisagem}
\begin{figure}[!h]
\centering
\caption{Projecto da Companhia União Fabril para edificação de casas para operarios em terreno de sua propriedade na Fonte Nova e travessa do Sangradouro (1893)}
\includegraphics[width=\textwidth]{8-anexos/plantas/01-1distrito/15-sangradouro/vila-operaria-fabril/sangradouro-uniaofabril-vilaoperaria.jpg}{\footnotesize \par Projeto de autor desconhecido. \par \textbf{Fonte:} \textbf{BR BAAHMS}, Fundo ``Intendência'', Série ``Processos de Licenciamento de Reforma e Ampliação de Edificações'', Subsérie ``Requerimentos e Plantas -- Brotas'', caixa 15. A vila operária da União Fabril é única no distrito; não foi possível encontrar nenhum empreendimento imobiliário semelhante no distrito. No total, cinquenta e quatro casas foram projetadas. Algumas casas no local mantém ainda hoje certa padronização de fachadas, ainda que com tipos diferentes.}
\label{fig:sangradouro-uniaofabril-vilaoperaria}
\end{figure}
\end{a3paisagem}
}