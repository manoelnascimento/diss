% ---
% Casa de Edmundo Guimarães na ladeira dos Galés
% ---

\afterpage{
\begin{a3paisagem}
\hfill
\begin{figure}[!h]
\caption{Projecto para construção do 1ª andar do predio á Ladeira dos Galés nº 16. Propriedade do Illmº Snr. Edmundo Guimarães. Districto de Brotas (1930) }
\includegraphics[width=1\textwidth]{8-anexos/plantas/01-1distrito/12-gales/ladeiradosgales-edmundoguimaraes-casa.jpg}{\footnotesize \par  Projeto de Rossi Baptista (1930). \textbf{Fonte:} \textbf{BR BAAHMS}, Fundo ``Intendência'', Série ``Processos de Licenciamento de Reforma e Ampliação de Edificações'', Subsérie ``Requerimentos e Plantas -- Brotas'', caixa 12. Esta casa de Edmundo Guimarães permanece de pé. }
\label{fig:edmundoguimaraes1930}
\end{figure}
\hfill
\end{a3paisagem}
}

% ---
% Casa de José Veríssimo Alves na rua do Sangradouro
% ---

\afterpage{
\begin{a3paisagem}
\begin{figure}[!h]
\centering
\caption{Representação da casa de nº 24 sita a rua do Sangradouro com projecto de mais um pavimento (1920). }
\includegraphics[width=1\textwidth]{8-anexos/plantas/01-1distrito/15-sangradouro/sangradouro-joseverissimoalves-1920.jpg}{\footnotesize \par  \textbf{Fonte:} \textbf{BR BAAHMS}, Fundo ``Intendência'', Série ``Processos de Licenciamento de Reforma e Ampliação de Edificações'', Subsérie ``Requerimentos e Plantas -- Brotas'', caixa 15. \par Projeto de Custódio Bandeira (1920). A casa de José Veríssimo Alves é uma das poucas a permanecer de pé neste logradouro. }
\label{fig:joseverissimo1920}
\end{figure}
\end{a3paisagem}
}

% ---
% Casa de Maria Zifirina da Conceição
% ---

\afterpage{
\begin{a3paisagem}
\begin{figure}[!h]
\centering
\caption{Progeto de um predio a construir-se na Pedrinha de propriedade da Exma. Snra. D. Maria Zefirina da Conceição (1918)}
\includegraphics[width=1\textwidth]{8-anexos/plantas/05-mariquita/23-pedrinhas/1918-mariazifirinadaconceicao.jpg}{\footnotesize \par \textbf{Fonte:} \textbf{BR BAAHMS}, Fundo ``Intendência'', Série ``Processos de Licenciamento de Reforma e Ampliação de Edificações'', Subsérie ``Requerimentos e Plantas -- Brotas'', caixa 23. \par Projeto de José Portella Passos. Esta casa chama a atenção pelo número de quartos: \textit{oito}, variando entre os 7,92$m^{2}$ e os 14,91$m^{2}$. }
\label{fig:1918-mariazifirinadaconceicao}
\end{figure}
\end{a3paisagem}
}

% ---
% Casa de Aurelio Gonçalves Cal
% ---

\afterpage{
\begin{a3paisagem}
\begin{figure}[!h]
\centering
\caption{Representação da casa de nº 30 sita a rua direita da Mariquita, ao Rio Vermelho, com projecto de remodelação de sua fachada e de um pequeno pavimento (1922)}
\includegraphics[height=0.9\textheight]{8-anexos/plantas/05-mariquita/06-mariquita/1922-aureliogoncalvescal.jpg}{\footnotesize \par \textbf{Fonte:} \textbf{BR BAAHMS}, Fundo ``Intendência'', Série ``Processos de Licenciamento de Reforma e Ampliação de Edificações'', Subsérie ``Requerimentos e Plantas -- Brotas'', caixa 06. \par Projeto de Alfredo Vieira de Almeida. Esta casa destaca-se por ter \textit{seis} quartos no primeiro pavimento, e outros \textit{sete} no pavimento a construir, com cerca de 10$m^{2}$ cada.}
\label{fig:1922-aureliogoncalvescal.jpg}
\end{figure}
\end{a3paisagem}
}