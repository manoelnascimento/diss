% ---
% 03. Casa rural de Francisco Ventura na rua do Sangradouro
% ---

\afterpage{
\begin{a3paisagem}
\begin{figure}[!h]
\centering
\caption{Projecto de casa para o sr. Francisco Ventura (1901). }
\includegraphics[height=0.9\textheight]{8-anexos/plantas/01-1distrito/15-sangradouro/sangradouro-franciscoventura-casa01.jpg}{\footnotesize \par \textbf{Fonte:} \textbf{BR BAAHMS}, Fundo ``Intendência'', Série ``Processos de Licenciamento de Reforma e Ampliação de Edificações'', Subsérie ``Requerimentos e Plantas -- Brotas'', caixa 15. \par Projeto de Manoel R. F. Muniz. Hoje não mais existente, a construção é uma das típicas casas rurais do Sangradouro, um dos limites da urbanização no antigo 1º Distrito na alvorada da Primeira República. }
\label{fig:franciscoventura}
\end{figure}
\end{a3paisagem}
}

% ---
% 05. Casa rural de Ricardo da Silva Teixeira Machado na rua do Sangradouro
% ---

\afterpage{
\begin{a3paisagem}
\begin{figure}[!h]
\centering
\caption{Projecto para construcção de um andar da casa na roça ao Sangradouro, dist. de Brotas, pertencente ao Snr. Ricardo da Silva Teixeira Machado (1915). }
\includegraphics[height=0.9\textheight]{8-anexos/plantas/01-1distrito/15-sangradouro/sangradouro-ricardodasilvateixeiramachado-constroiandarnacasas.jpg}{\footnotesize \par \textbf{Fonte:} \textbf{BR BAAHMS}, Fundo ``Intendência'', Série ``Processos de Licenciamento de Reforma e Ampliação de Edificações'', Subsérie ``Requerimentos e Plantas -- Brotas'', caixa 15. \par Projeto de Arthur Santos. Ainda era possível falar de ``roças'' no Sangradouro em 1915, evidenciando até onde avançara a urbanização no antigo 1º Distrito. }
\label{fig:ricardodasilvateixeiramachado}
\end{figure}
\end{a3paisagem}
}