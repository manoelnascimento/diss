% ---
% 01. Casa de Theodomiro José Veríssimo, na rua do Céu
% ---

\afterpage{
\begin{a3paisagem}
\begin{figure}[!h]
\centering
\caption{Planta para construcção de uma casa que Bernardo Martins da Silva pretende construir no terreno baldio sito a rua Direita da Mariquita ao Rio Vermelho (1900)}
\includegraphics[width=1\textwidth]{8-anexos/plantas/05-mariquita/06-mariquita/1900-bernardomartinsdasilva.jpg}{\footnotesize \par \textbf{Fonte:} \textbf{BR BAAHMS}, Fundo ``Intendência'', Série ``Processos de Licenciamento de Reforma e Ampliação de Edificações'', Subsérie ``Requerimentos e Plantas -- Brotas'', caixa 06. \par Projeto de desenhista e arquiteto/engenheiro desconhecidos. Casas de veraneio como esta foram comuns na rua Direita da Mariquita. }
\label{fig:1900-bernardomartinsdasilva}
\end{figure}
\end{a3paisagem}
}

% ---
% 02. Casa de Theodomiro José Veríssimo, na rua do Céu
% ---

\afterpage{
\begin{figure}[!h]
\centering
\caption{Projecto para a construcção de uma casa de taipa que Theodomiro José Verissimo pretende fazer na rua do Céo no Rio Vermelho districto de Brotas (1906)}
\includegraphics[width=1\textwidth]{8-anexos/plantas/05-mariquita/03-ruadoceu/theodomirojoseverissimo-1906.jpg}{\footnotesize \par \textbf{Fonte:} \textbf{BR BAAHMS}, Fundo ``Intendência'', Série ``Processos de Licenciamento de Reforma e Ampliação de Edificações'', Subsérie ``Requerimentos e Plantas -- Brotas'', caixa 03. \par Projeto de Custódio Bandeira. Casas de taipa ainda eram toleradas pela Diretoria de Obras na primeira década do século XIX. Mesmo o desenhista ``relaxou'' e não apresentou as medidas de todos os cômodos. }
\label{fig:theodomirojoseverissimo-1906}
\end{figure}
}

% ---
% 03. Casa de Prediliano Pitta, na rua do Lucaia
% ---

\afterpage{
\begin{figure}[!h]
\centering
\caption{Projecto para a construcção de uma casa em terreno baldio, sito à Lucaia freguesia de Brotas, pertencente ao Snr. Prediliano Pereira Pitta (1911)}
\includegraphics[width=1\textwidth]{8-anexos/plantas/05-mariquita/06-lucaia/predilianopitta-1911.jpg}{\footnotesize \par \textbf{Fonte:} \textbf{BR BAAHMS}, Fundo ``Intendência'', Série ``Processos de Licenciamento de Reforma e Ampliação de Edificações'', Subsérie ``Requerimentos e Plantas -- Brotas'', caixa 06. \par Não há assinatura de desenhista ou engenheiro/arquiteto, mas a caligrafia é de Rosalvo Celestino dos Santos. }
\label{fig:predilianopitta-1911}
\end{figure}
}


% ---
% 04. Casa de Luiz Lucas da Costa, na rua da Fonte do Boi
% ---

\afterpage{
\begin{a3paisagem}
\begin{figure}[!h]
\centering
\caption{Projecto para construcção de uma cosinha, varanda, banheiro, latrina e platibanda, em um predio sito a Fonte do Boi no Rio Vermelho districto de Brotas, pertencente ao Sr. Luiz Lucas da Costa (1912)}
\includegraphics[height=0.9\textheight]{8-anexos/plantas/05-mariquita/06-fontedoboi/luizlucasdacosta-1912.jpg}{\footnotesize \par \textbf{Fonte:} \textbf{BR BAAHMS}, Fundo ``Intendência'', Série ``Processos de Licenciamento de Reforma e Ampliação de Edificações'', Subsérie ``Requerimentos e Plantas -- Brotas'', caixa 06. \par Na Mariquita os pedidos de reforma predominaram sobre as construções. Este é um dos poucos projetos em que a fachada original foi desenhada em separado, permitindo perceber as alterações. A casa aliás é pequena: o maior dos cômodos tem 9,79$m^{2}$ e o menor, 6,97$m^{2}$. }
\label{fig:luizlucasdacosta-1912}
\end{figure}
\end{a3paisagem}
}

% ---
% 05. Casa de Manoel Affonso Vianna, na Fonte do Boi
% ---

\afterpage{
\begin{figure}[!h]
\centering
\caption{Projecto de construcção --- propriedade do sr. Manoel Affonso Vianna --- Fonte do Boi (Rio Vermelho) (1913)}
\includegraphics[width=1\textwidth]{8-anexos/plantas/05-mariquita/06-fontedoboi/manoelvianna-1915.jpg}{\footnotesize \par \textbf{Fonte:} \textbf{BR BAAHMS}, Fundo ``Intendência'', Série ``Processos de Licenciamento de Reforma e Ampliação de Edificações'', Subsérie ``Requerimentos e Plantas -- Brotas'', caixa 06. \par Projeto de Humberto Badollato. }
\label{fig:manoelvianna-1915}
\end{figure}
}

% ---
% 06. Casa de Adolpho Moreira, na rua Direita da Mariquita
% ---

\afterpage{
\begin{a3paisagem}
\begin{figure}[!h]
\centering
\caption{Projecto para a reconstrucção da casa à Mariquita, Rio Vermelho, do Snr. Adolpho Moreira (1913)}
\includegraphics[width=0.9\textwidth]{8-anexos/plantas/05-mariquita/06-mariquita/1913-adolphomoreira.jpg}{\footnotesize \par \textbf{Fonte:} \textbf{BR BAAHMS}, Fundo ``Intendência'', Série ``Processos de Licenciamento de Reforma e Ampliação de Edificações'', Subsérie ``Requerimentos e Plantas -- Brotas'', caixa 06. }
\label{fig:1913-adolphomoreira.jpg}
\end{figure}
\end{a3paisagem}
}

% ---
% 07. Padaria de Domingos de Oliveira Reis, na Mariquita
% ---

\afterpage{
\begin{a3paisagem}
\begin{figure}[!h]
\centering
\caption{Rrojecto para a construcção de uma casa e forno de padaria, à rua da Mariquita, Rio Vermelho (1914)}
\includegraphics[height=0.9\textheight]{8-anexos/plantas/05-mariquita/06-mariquita/1914-domingosdeoliveirareis.jpg}{\footnotesize \par \textbf{Fonte:} \textbf{BR BAAHMS}, Fundo ``Intendência'', Série ``Processos de Licenciamento de Reforma e Ampliação de Edificações'', Subsérie ``Requerimentos e Plantas -- Brotas'', caixa 06. \par Projeto de Arthur Santos. O mesmo proprietário tinha uma casa de térreo e primeiro pavimento na Mariquita, vizinha à padaria.}
\label{fig:1914-domingosdeoliveirareis}
\end{figure}
\end{a3paisagem}
}

% ---
% 08. Casa de Maria Zifirina da Conceição
% ---

\afterpage{
\begin{a3paisagem}
\begin{figure}[!h]
\centering
\caption{Progeto de um predio a construir-se na Pedrinha de propriedade da Exma. Snra. D. Maria Zefirina da Conceição (1918)}
\includegraphics[width=1\textwidth]{8-anexos/plantas/05-mariquita/23-pedrinhas/1918-mariazifirinadaconceicao.jpg}{\footnotesize \par \textbf{Fonte:} \textbf{BR BAAHMS}, Fundo ``Intendência'', Série ``Processos de Licenciamento de Reforma e Ampliação de Edificações'', Subsérie ``Requerimentos e Plantas -- Brotas'', caixa 23. \par Projeto de José Portella Passos. Esta casa chama a atenção pelo número de quartos: \textit{oito}, variando entre 7,92$m^{2}$ até os 14,91$m^{2}$. }
\label{fig:1918-mariazifirinadaconceicao}
\end{figure}
\end{a3paisagem}
}

% ---
% 09. Casa de Aurelio Gonçalves Cal
% ---

\afterpage{
\begin{a3paisagem}
\begin{figure}[!h]
\centering
\caption{Representação da casa de nº 30 sita a rua direita da Mariquita, ao Rio Vermelho, com projecto de remodelação de sua fachada e de um pequeno pavimento (1922)}
\includegraphics[height=0.9\textheight]{8-anexos/plantas/05-mariquita/06-mariquita/1922-aureliogoncalvescal.jpg}{\footnotesize \par \textbf{Fonte:} \textbf{BR BAAHMS}, Fundo ``Intendência'', Série ``Processos de Licenciamento de Reforma e Ampliação de Edificações'', Subsérie ``Requerimentos e Plantas -- Brotas'', caixa 06. \par Projeto de Alfredo Vieira de Almeida. Esta casa destaca-se por ter \textit{seis} quartos no primeiro pavimento, e outros \textit{sete} no pavimento a construir, com cerca de 10$m^{2}$ cada.}
\label{fig:1922-aureliogoncalvescal.jpg}
\end{figure}
\end{a3paisagem}
}