% ---
% 01. Casa de praia de A. Saffrey na Cidade Balneária Amaralina
% ---

\afterpage{
\begin{a3paisagem}
\begin{figure}[!h]
\centering
\caption{Propriedade de A. Saffrey, architecto, Amaralina. Lote 116 da planta geral da cidade d'Amaralina (1915) }
\includegraphics[height=0.9\textheight]{8-anexos/plantas/09-amaralinapituba/19-amaralina/dsc04777.jpg}{\footnotesize \par \textbf{Fonte:} \textbf{BR BAAHMS}, Fundo ``Intendência'', Série ``Processos de Licenciamento de Reforma e Ampliação de Edificações'', Subsérie ``Requerimentos e Plantas -- Brotas'', caixa 19. \par Projeto de A. Saffrey. A planta de situação mostra como esta grande casa de praia (121,50$m^{2}$), construída ao que parece em meio a um belo jardim, situava-se no lote 116 da Cidade Balneária Amaralina --- lote grande (525$m^{2}$), que não foi totalmente empregue no projeto. }
\label{fig:dsc04777}
\end{figure}
\end{a3paisagem}
}

% ---
% 02. Casa de Lydia Dewald na Cidade Balneária Amaralina
% ---

\afterpage{
\begin{a3paisagem}
\begin{figure}[!h]
\centering
\caption{Projeto de uma casa em Amaralina de D. Lydia Dewald (1915) }
\includegraphics[height=0.9\textheight]{8-anexos/plantas/09-amaralinapituba/19-amaralina/DSC04783.jpg}{\footnotesize \par \textbf{Fonte:} \textbf{BR BAAHMS}, Fundo ``Intendência'', Série ``Processos de Licenciamento de Reforma e Ampliação de Edificações'', Subsérie ``Requerimentos e Plantas -- Brotas'', caixa 19. \par Desenho de Ciro Spínola, engenheiro/arquiteto desconhecido. Encontra-se bastante danificado pelo mofo este projeto de casa com três quartos, sala, dependências (uma delas o infame ``quarto de empregada''), copa, cozinha e pátio interno, o que impediu sua reprodução integral. Salvou-se entretanto a fachada, que destaca-se nesta área pelas duas escadarias abalaustradas. }
\label{fig:DSC04783}
\end{figure}
\end{a3paisagem}
}

% ---
% 03. Casa de José Cardozo da Silva na Ubarana
% ---

\afterpage{
\begin{a3paisagem}
\begin{figure}[!h]
\centering
\caption{Projecto da casa que o Ilmº. Snr. José Cardozo da Silva pretende construir em seu terreno, sito à Ubarana, no districto de Brotas (1915) }
\includegraphics[height=0.9\textheight]{8-anexos/plantas/09-amaralinapituba/19-amaralina/dsc04791.jpg}{\footnotesize \par \textbf{Fonte:} \textbf{BR BAAHMS}, Fundo ``Intendência'', Série ``Processos de Licenciamento de Reforma e Ampliação de Edificações'', Subsérie ``Requerimentos e Plantas -- Brotas'', caixa 19. \par Projeto de Archimedes Marques. Dois quartos, duas salas (``visita'' e ``jantar''), cozinha e banheiro nos fundos, fachada discretamente ornada: repete-se aqui o desenho comum para as casas de médio valor locativo no distrito. }
\label{fig:dsc04791}
\end{figure}
\end{a3paisagem}
}

% ---
% 04. Casa de Chehadi Elias Kraychete na Cidade Balneária Amaralina
% ---

\afterpage{
\begin{a3paisagem}
\begin{figure}[!h]
\centering
\caption{Projecto de uma casa que o sr. Chehadi E. Kraycheti quer construir em Amaralina (1923)}
\includegraphics[width=\textwidth]{8-anexos/plantas/09-amaralinapituba/19-amaralina/dsc4805.jpg}{\footnotesize \par \textbf{Fonte:} \textbf{BR BAAHMS}, Fundo ``Intendência'', Série ``Processos de Licenciamento de Reforma e Ampliação de Edificações'', Subsérie ``Requerimentos e Plantas -- Brotas'', caixa 19. \par Projeto de Eurico da Costa Coutinho. O projeto de pouco adorno esconde uma casa de porte médio para grande, com duas salas e quartos de 12,58$m^{2}$ --- além do quarto ``dos fundos'', em frente ao sanitário e à cozinha, que terá servido a algum empregado doméstico. }
\label{fig:dsc4805}
\end{figure}
\end{a3paisagem}
}

% ---
% 05. Oito casas de Crispiniano João Rogelio (parte 1)
% ---

\afterpage{
\begin{a3paisagem}
\begin{figure}[!h]
\centering
\caption{Projecto para construção de oito casas à Amaralina, primeira parte (1925)}
\includegraphics[height=0.9\textheight]{8-anexos/plantas/09-amaralinapituba/19-amaralina/dsc04810.jpg}{\footnotesize \par Projeto de Pedro Jayme David. \textbf{Fonte:} \textbf{BR BAAHMS}, Fundo ``Intendência'', Série ``Processos de Licenciamento de Reforma e Ampliação de Edificações'', Subsérie ``Requerimentos e Plantas -- Brotas'', caixa 19. \par Trata-se de oito cubículos de 38.75$m^{2}$ cada (sem contar a cozinha e o ``W.C.'' anexos), com dois quartos de 8,75$m^{2}$ e duas ``salas'' de 7,7$m^{2}$, com fachada desadornada. Com este tamanho, a julgar pela descrição do lote visto na \autoref{fig:dsc04777}, este projeto demonstra terem sido concebidas estas pequenas casas pelo requerente Crispiniano João Rogelio e por Pedro Jayme David no tamanho certo para caberem num só lote da Cidade Balneária Amaralina. }
\label{fig:dsc04810}
\end{figure}
\end{a3paisagem}
}

% ---
% 06. Oito casas de Crispiniano João Rogelio (parte 2)
% ---

\afterpage{
\begin{a3paisagem}
\begin{figure}[!h]
\centering
\caption{Projecto para construção de oito casas à Amaralina, segunda parte (1925) }
\includegraphics[width=\textwidth]{8-anexos/plantas/09-amaralinapituba/19-amaralina/dsc04809.jpg}{\footnotesize \par Projeto de Pedro Jayme David. \par \textbf{Fonte:} \textbf{BR BAAHMS}, Fundo ``Intendência'', Série ``Processos de Licenciamento de Reforma e Ampliação de Edificações'', Subsérie ``Requerimentos e Plantas -- Brotas'', caixa 19. A segunda parte do projeto proposto por Crispiniano João Rogelio mostra que os cubículos desenhados por Pedro Jayme David compõem uma pequena avenida de casas. Como não se trata de casas para aluguel a veranistas, nem tampouco seriam tais cubículos atrativos ao habitual comprador de imóveis em Amaralina, fica evidente que trata-se de ``casas para operários'', muito provavelmente atendendo à demanda de trabalhadores para o serviço doméstico nas casas adventícias circunvizinhas.}
\label{fig:dsc04809}
\end{figure}
\end{a3paisagem}
}

% ---
% 07. Mansão de João da Cunha Freire (planta baixa)
% ---



\afterpage{
\begin{a3paisagem}
\begin{figure}[!h]
\centering
\caption{Projecto para construcção do predio sito na Amaralina, propriedade do Ilmº Snr. João da Cunha Freire (1925), primeira parte  }
\includegraphics[height=0.9\textheight]{8-anexos/plantas/09-amaralinapituba/19-amaralina/dsc04813.jpg}{\footnotesize \par Projeto de Rossi Baptista. \textbf{Fonte:} \textbf{BR BAAHMS}, Fundo ``Intendência'', Série ``Processos de Licenciamento de Reforma e Ampliação de Edificações'', Subsérie ``Requerimentos e Plantas -- Brotas'', caixa 19. Verdadeira mansão adventícia, tem no primeiro andar \textit{cinco} quartos grandes (14,4$m^{2}$ o menor deles) com ``W.C.'' exclusivo, além de \textit{quatro} quartos ``dos fundos'' no térreo e enormes varanda e terraço. }
\label{fig:dsc04813}
\end{figure}
\end{a3paisagem}
}

% ---
% 08. Mansão de João da Cunha Freire (fachada)
% ---

\afterpage{
\begin{a3paisagem}
\begin{figure}[!h]
\centering
\caption{Projecto para construcção do predio sito na Amaralina, propriedade do Ilmº Snr. João da Cunha Freire (1925), segunda parte }
\includegraphics[width=\textwidth]{8-anexos/plantas/09-amaralinapituba/19-amaralina/dsc04814.jpg}{\footnotesize \par Projeto de Rossi Baptista. \textbf{Fonte:} \textbf{BR BAAHMS}, Fundo ``Intendência'', Série ``Processos de Licenciamento de Reforma e Ampliação de Edificações'', Subsérie ``Requerimentos e Plantas -- Brotas'', caixa 19. \par Apesar de bastante danificada a planta, ainda é possível reconhecer elementos da fachada que permitem identificá-la na \autoref{fig:1930-amaralina02} (p. \pageref{fig:1930-amaralina02}). }
\label{fig:dsc04814}
\end{figure}
\end{a3paisagem}
}

% ---
% 09. Casa de Elias Abdon
% ---

\afterpage{
\begin{a3paisagem}
\begin{figure}[!h]
\centering
\caption{Projecto para construcção de uma casa em Amaralina (1930) }
\includegraphics[width=\textwidth]{8-anexos/plantas/09-amaralinapituba/19-amaralina/amaralina-eliasabdon-1930.jpg}{\footnotesize \par \textbf{Fonte:} \textbf{BR BAAHMS}, Fundo ``Intendência'', Série ``Processos de Licenciamento de Reforma e Ampliação de Edificações'', Subsérie ``Requerimentos e Plantas -- Brotas'', caixa 19. \par Projeto de Rossi Baptista. Esta casa aparece na \autoref{fig:1930-amaralina02} (p. \pageref{fig:1930-amaralina02}), o que permite verificar a magnitude do projeto. }
\label{fig:amaralina-eliasabdon}
\end{figure}
\end{a3paisagem}
}