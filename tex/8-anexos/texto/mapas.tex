% ---
% Mapa de pontos notáveis
% ---

\afterpage{
\begin{a3paisagem}
\begin{figure}[!h]
\centering
\caption{Mapa ilustrativo com pontos notáveis e localidades trabalhadas na dissertação}
\includegraphics[height=0.9\textheight]{8-anexos/mapas/2019-pontosnotaveis.jpg}{\footnotesize \par \textbf{Fonte:} elaboração do autor a partir da base cartográfica do \textbf{Open Street Maps}.}
\label{fig:2019-pontosnotaveis}
\end{figure}
\end{a3paisagem}
}


% ---
% Mapa de Leal Teixeira
% ---

\afterpage{
\begin{a3paisagem}
\begin{figure}[!h]
\centering
\caption{``Carta hydrografica da Bahia de Todos os Santos na qual está situada a cidade de S. Salvador capital da provincia do mesmo nome'', de Leal Teixeira (1830)}
\includegraphics[height=0.9\textheight]{8-anexos/mapas/1830-lealteixeira.jpg}{\footnotesize \par \textbf{Fonte:} \citeonline{lealteixeira_carta_1830}. \par Este mapa é um dos poucos produzidos no século XIX a indicar adequadamente os cursos d'água da bacia dos rios das Pedras e Pituaçu, limite oriental da freguesia de Brotas.}
\label{fig:1830-lealteixeira}
\end{figure}
\end{a3paisagem}
}

% ---
% Mapa de Beaurepaire-Rohan
% ---

\afterpage{
\begin{a3paisagem}
\begin{figure}[!h]
\centering
\caption{``Planta do accampamento de Pirajá e Itapoan\dots'', de Henrique Pedro Carlos Beaurepaire-Rohan (1839)}
\includegraphics[height=0.9\textheight]{8-anexos/mapas/1839-beaurepairerohan.jpg}{\footnotesize \par \textbf{Fonte:} \citeonline{visconderohan_mapa_1839}. \par O mapa de Beaurepaire-Rohan, embora muito impreciso em vários aspectos, permite situar adequadamente as armações do Gregório (``Armação'') e do Saraiva (``Armação do Saraiva''), além da praia do Chega-Negro (``rio do Chega-Negro''). Indica também a existência de um caminho que saía de Brotas rumo à armação do Saraiva (a \textit{estrada das Armações}) e outro que ligava também a ``estrada do Cabula de Cima'' (rua Cristiano Buys / ladeira do Cabula) com esta mesma localidade (correspondendo à \textit{estrada de Pernambués}).}
\label{fig:1839-beaurepairerohan}
\end{figure}
\end{a3paisagem}
}

% ---
% Mapa de Weyll
% ---

\afterpage{
\begin{a3paisagem}
\begin{figure}[!htp]
\caption{``Mappa topographica da cidade de S. Salvador e seus subúrbios'', elaborada pelo engenheiro alemão Carlos Augusto Weyll e posteriormente datada como sendo de 1851.}
\centering
\includegraphics[width=\textwidth]{8-anexos/mapas/1855weyll.jpg}{\footnotesize \par \textbf{Fonte:} \citeonline{weyll_mappa_1851}. \par O mapa de Weyll ainda é a mais detalhada fonte cartográfica para o estudo de Salvador e seus subúrbios no período que vai até os anos 1930. \par} 
\label{fig:1851-weyll}
\end{figure}
\end{a3paisagem}
}

% ---
% Recorte do mapa de Theodoro Sampaio (1899)
% ---

\afterpage{
\begin{a3paisagem}
\begin{figure}[!htp]
\caption{Recorte da ``Carta do Recôncavo da Bahia\dots'', organizada por Theodoro Sampaio (1899).}
\centering
\includegraphics[height=0.9\textheight]{8-anexos/mapas/1899-theodorosampaio-recorte.jpg}{\footnotesize \par \textbf{Fonte:} \citeonline{sampaio_reconcavo_1899}. \par Apesar de pouco detalhado no que diz respeito a Salvador, pois de fato não foi objetivo do mapa detalhá-la, nota-se a indicação de ``Brotas'' e do ``Rio Vermelho'' bem distantes da malha urbana consolidada (representada por um polígono hachureado); e a ``Armação'' (do Saraiva) na foz do rio das Pedras, cujos afluentes e tributários foram representados estendendo-se até ``Pirajá''. Entre a bacia dos rios das Pedras e Pituaçu e a do Camarajipe (com foz no ``Rio Vermelho'') encontra-se o pequeno rio Pernambués. Pela convenção do mapa, projetava-se uma ``estrada de ferro'' para ligar ``Brotas'' à ``Armação'', e outra para ligar o ``Rio Vermelho'' a ``Itapoan''. \par}
\label{fig:1899-theodorosampaio}
\end{figure}
\end{a3paisagem}
}

% ---
% Atlas parcial, folha 24 (antigo 1º Distrito)
% ---

\afterpage{
\begin{a3paisagem}
\begin{figure}[!h]
\centering
\caption{``Atlas parcial da cidade do Salvador'', folha 24: 1º de Maio (1955)}
\includegraphics[height=0.9\textheight]{8-anexos/mapas/1955-ladeiradosbandeirantes.jpg}{\footnotesize \par \textbf{Fonte:} \citeonline{municipal_atlas_1955}. \par Neste mapa aparecem à direita a rua Djalma Dutra, que divide Brotas de Nazaré, e a ladeira dos Bandeirantes. }
\label{fig:1955-ladeiradosbandeirantes}
\end{figure}
\end{a3paisagem}
}

% ---
% Atlas parcial, folha 27 (Boa Vista)
% ---

\afterpage{
\begin{a3paisagem}
\begin{figure}[!h]
\centering
\caption{``Atlas parcial da cidade do Salvador'', folha 27: Fonte Nova (1955)}
\includegraphics[height=0.9\textheight]{8-anexos/mapas/1955-ladeiradopepino.jpg}{\footnotesize \par \textbf{Fonte:} \citeonline{municipal_atlas_1955}. \par Além da Vasco da Gama aparecem neste mapa, próximos à margem superior, da direita para a esquerda: a ladeira do Pepino (rua José Visco), a vila Santos, a vila Bela Vista, a vila Edinho, o pé da ladeira dos Galés (praça Francisco Vicente Viana) e a fábrica São Salvador, na rua Djalma Dutra.}
\label{fig:1955-ladeiradopepino}
\end{figure}
\end{a3paisagem}
}

% ---
% Atlas parcial, folha 28 (Cosme de Farias)
% ---

\afterpage{
\begin{a3paisagem}
\begin{figure}[!h]
\centering
\caption{``Atlas parcial da cidade do Salvador'', folha 28: Cosme de Farias (1955)}
\includegraphics[height=0.9\textheight]{8-anexos/mapas/1955-cosmedefarias.jpg}{\footnotesize \par \textbf{Fonte:} \citeonline{municipal_atlas_1955}. }
\label{fig:1955-cosmedefarias}
\end{figure}
\end{a3paisagem}
}

% ---
% Atlas parcial, folha 32 (Engenho Velho)
% ---

\afterpage{
\begin{a3paisagem}
\begin{figure}[!h]
\centering
\caption{``Atlas parcial da cidade do Salvador'', folha 32: Vasco da Gama}
\includegraphics[height=0.9\textheight]{8-anexos/mapas/1955-engenhovelho.jpg}{\footnotesize \par \textbf{Fonte:} \citeonline{municipal_atlas_1955}. \par Aparecem neste mapa quase todas as áreas integrantes do Engenho Velho e Boa Vista logo acima do dique do Tororó: capela do Deus Menino (rua Brígida do Vale e adjacências), avenida Paraguaçu, ladeira do Corrimão, Monte Belém (de Baixo, do Meio e de Cima), ladeira da União, alto do Moinho, vila Raimundo, vila Girassol e rua Caminho de Dentro.}
\label{fig:1955-engenhovelho}
\end{figure}
\end{a3paisagem}
}

% ---
% ``Mapa azul'' municipal de 1969
% ---

\afterpage{
\begin{a3paisagem}
\begin{figure}[!h]
\centering
\caption{Mapa da cidade do Salvador --- sistema viário}
\includegraphics[height=0.9\textheight]{8-anexos/mapas/1969-mapaprefeitura.jpg}{\footnotesize \par \textbf{Fonte:} \citeonline{salvador_mapa_1969}. \par Custodiado na mapoteca do \textbf{Arquivo Histórico Municipal de Salvador}, este mapa encomendado pela Prefeitura de Salvador contém informações importantes sobre os loteamentos existentes e projetados no período, além de permitir visualizar resquícios da malha viária que existiu ainda durante a Primeira República nas áreas mais afastadas do núcleo urbano de Salvador.}
\label{fig:1969-mapaprefeitura}
\end{figure}
\end{a3paisagem}
}