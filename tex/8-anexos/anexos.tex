% ---
% Inicia os anexos
% ---
\begin{anexosenv}

% Imprime uma página indicando o início dos anexos
\partanexos

% ---
\chapter{Tabelas originais}

\begin{tiny}
\begin{longtable}{cc}
\caption{Nome dos proprietários rurais do distrito de Brotas e localidade de suas terras (1920)}\label{tab:proprurais}\\
\hline Proprietários & Nome do estabelecimento (ou localidade) \\ \hline\hline \endhead
\hline \multicolumn{2}{c}{Continua na próxima página...} \\ \endfoot
\hline \endlastfoot
Joanna B. de Souza  & Brotas \\
Pedro F. H. Pires  & Brotas \\
José Visco  & Brotas \\
António S. Souza  & Brotas \\
Luiz Saraiva  & Brotas \\
Augusto R. de Senna  & Brotas \\
Juventino R. de Almeida  & Brotas \\
Ricardo A. Pereira  & Brotas \\
Antonio A. Cupim & Brotas \\
José A. Silva & Brotas \\
Avelino J. Pereira & Brotas \\
Luiz H. de Souza & Brotas \\
Salustiano R. Pinto & Brotas \\
João de A. Ramos & Brotas \\
Rodrigo A. Araújo & Brotas \\
Estevão G. da Encarnação & Brotas \\
Trifino P. de Souza & Brotas \\
Joaquim F. A. Rosa & Brotas \\
Affonso C. Mello & Brotas \\
João P. de Queiroz & Brotas \\
João P. dos Santos & Brotas \\
Hermenegildo P. Souza & Brotas \\
Silvann P. Ramos & Brotas \\
Jesuino A. Monteiro & Brotas \\
Francisco R. A Prata & Brotas \\
António P. Melchiades & Brotas \\
Raphael Alonso & Brotas \\
Pedro J. Mussitahyba & Brotas \\
João de D. Ribeiro & Brotas \\
Zacharias de Oliveira & Brotas \\
Raphael E. da Purificação & Brotas \\
Gabriel A. dos Santos & Brotas \\
Francisco X. da Silva & Brotas \\
Abel A. Amoedo & Brotas \\
Francelino S. B. Figueiredo & Brotas \\
João B. da Silva & Brotas \\
António F. Simões & Brotas \\
Esperidião B. de Argollo & Brotas \\
Olegário F. Cardoso & Brotas \\
João B. Wanderley & Brotas \\
Fortunato J. da Costa & Brotas \\
Joaquim F. Teixeira & Brotas \\
Segundo Garrido & Brotas \\
Pedro S. Nascimento & Brotas \\
Bernardo Lins & Brotas \\
Coronel Frederico A. Rodrigues da Costa & N. S. do Matatú \\
\hline
\end{longtable}
\end{tiny}

% ---



% ---
\chapter{Plantas selecionadas de Amaralina}
% ---




% ---
\chapter{Plantas selecionadas do Castro Neves, Ladeira dos Galés e vizinhança}
% ---





% ---
\chapter{Plantas selecionadas do Engenho Velho de Brotas e vizinhança}
% ---




% ---
\chapter{Plantas selecionadas da Estrada Dois de Julho e vizinhança}
% ---



% ---
\chapter{Plantas selecionadas do Rio Vermelho}
% ---



% ---
\chapter{Plantas selecionadas da Estrada de Brotas e vizinhança}
% ---



% ---
\chapter{Plantas selecionadas do Acupe e vizinhança}
% ---




% ---
\chapter{Plantas selecionadas da Quinta das Beatas e vizinhança}
% ---




% ---
\chapter{Plantas selecionadas do Matatu Grande e Matatu Pequeno}
% ---




% ---
\chapter{Plantas selecionadas da Quinta das Beatas e arredores}
% ---








% ---
\chapter{Documentos selecionados}
% ---




% ---
\chapter{Facsímiles}
% ---


% ---
\chapter{``A Boa Vista'', por Castro Alves}\label{cap:boavista}
% ---

\poemtitle{A Boa Vista}
\settowidth{\versewidth}{}

\begin{flushright}
\textit{Sonha, poeta, sonha! Aqui sentado \\
No tosco assento da janela antiga,\\
Apóias sobre a mão a face pálida,\\
Sorrindo — dos amores à cantiga.}\\
Álvares de Azevedo
\end{flushright}

\begin{verse}
Era uma tarde triste, mas límpida e suave... \\
Eu — pálido poeta — seguia triste e grave \\
A estrada, que conduz ao campo solitário, \\
Como um filho, que volta ao paternal sacrário, \\
\end{verse}

\begin{verse}
E ao longe abandonando o múrmur da cidade \\
— Som vago, que gagueja em meio à imensidade, — \\
No drama do crepúsculo eu escutava atento \\
A surdina da tarde ao sol, que morre lento. \\
\end{verse}

\begin{verse}
A poeira da estrada meu passo levantava, \\
Porém minh'alma ardente no céu azul marchava \\
E os astros sacudia no vôo violento \\
— Poeira, que dormia no chão do firmamento. \\
\end{verse}

\begin{verse}
A pávida andorinha, que o vendaval fustiga, \\
Procura os coruchéus da catedral antiga. \\
Eu — andorinha entregue aos vendavais do inverno, \\
Ia seguindo triste p'ra o velho lar paterno. \\
\end{verse}

\begin{verse}
Como a águia, que do ninho talhado no rochedo \\
Ergue o pescoço calvo por cima do fraguedo, \\
— (P'ra ver no céu a nuvem, que espuma o firmamento, \\
E o mar, — corcel que espuma ao látego do vento...) \\
Longe o feudal castelo levanta a antiga torre, \\
Que aos raios do poente brilhante sol escorre! \\
Ei-lo soberbo e calmo o abutre de granito \\
Mergulhando o pescoço no seio do infinito \\
E lá de cima olhando com seus clarões vermelhos \\
Os tetos, que a seus pés parecem de joelhos!... \\
\end{verse}

\begin{verse}
Não! Minha velha torre! Oh! atalaia antiga, \\
Tu olhas esperando alguma face amiga, \\
E perguntas talvez ao vento, que em ti chora: \\
``Por que não volta mais o meu senhor d'outrora? \\
Por que não vem sentar-se no banco do terreiro \\
Ouvir das criancinhas o riso feiticeiro, \\
E pensando no lar, na ciência, nos pobres \\
Abrigar nesta sombra seus pensamentos nobres? \\
\end{verse}

\begin{verse}
Onde estão as crianças — grupo alegre e risonho \\
— Que escondiam-se atrás do cipreste tristonho... \\
\end{verse}

\begin{verse}
Ou que enforcaram rindo um feio Pulchinello, \\
Enquanto a doce Mãe, que é toda amor, desvelo \\
Ralha com um rir divino o grupo folgazão, \\
Que vem correndo alegre beijar-lhe a branca mão?...'' \\
\end{verse}

\begin{verse}
É nisto que tu cismas, ó torre abandonada, \\
Vendo deserto o parque e solitária a estrada. \\
No entanto eu — estrangeiro, que tu já não conheces — \\
No limiar de joelhos só tenho pranto e preces. \\
\end{verse}

\begin{verse}
Oh! deixem-me chorar!... Meu lar... meu doce ninho! \\
Abre a vetusta grade ao filho teu mesquinho! \\
Passado — mar imenso!... inunda-me em fragrância! \\
Eu não quero lauréis, quero as rosas da infância. \\
\end{verse}

\begin{verse}
Ai! Minha triste fronte, aonde as multidões \\
Lançaram misturadas glórias e maldições... \\
Acalenta em teu seio, ó solidão sagrada! \\
Deixa est'alma chorar em teu ombro encostada! \\
\end{verse}

\begin{verse}
Meu lar está deserto... Um velho cão de guarda \\
Veio saltando a custo roçar-me a testa parda, \\
Lamber-me após os dedos, porém a sós consigo \\
Rusgando com o direito, que tem um velho amigo... \\
Como tudo mudou-se!... O jardim 'stá inculto \\
As roseiras morreram do vento ao rijo insulto... \\
A erva inunda a terra; o musgo trepa os muros \\
A ortiga silvestre enrola em nós impuros \\
Uma estátua caída, em cuja mão nevada \\
A aranha estende ao sol a teia delicada!... \\
Mergulho os pés nas plantas selvagens, espalmadas, \\
As borboletas fogem-me em lúcidas manadas... \\
E ouvindo-me as passadas tristonhas, taciturnas, \\
Os grilos, que cantavam, calaram-se nas furnas... \\
\end{verse}

\begin{verse}
Oh! jardim solitário! Relíquia do passado! \\
Minh'alma, como tu, é um parque arruinado! \\
Morreram-me no seio as rosas em fragrância, \\
Veste o pesar os muros dos meus vergéis da infância, \\
\end{verse}

\begin{verse}
A estátua do talento, que pura em mim s'erguia, \\
Jaz hoje — e nela a turba enlaça uma ironia!... \\
Ao menos como tu, lá d'alma num recanto \\
Da casta poesia ainda escuto o canto, \\
— Voz do céu, que consola, se o mundo nos insulta, \\
E na gruta do seio murmura um treno oculta. \\
\end{verse}

\begin{verse}
Entremos!... Quantos ecos na vasta escadaria, \\
Nos longos corredores respondem-me à porfia!... \\
\end{verse}

\begin{verse}
Oh! casa de meus pais!... A um crânio já vazio, \\
Que o hóspede largando deixou calado e frio, \\
Compara-te o estrangeiro — caminhando indiscreto \\
Nestes salões imensos, que abriga o vasto teto. \\
\end{verse}

\begin{verse}
Mas eu no teu vazio — vejo uma multidão \\
Fala-me o teu silêncio — ouço-te a solidão!... \\
Povoam-se estas salas... \\
\end{verse}

\begin{verse}
E eu vejo lentamente \\
No solo resvalarem falando tenuemente \\
Dest'alma e deste seio as sombras venerandas \\
Fantasmas adorados — visões sutis e brandas... \\
\end{verse}

\begin{verse}
Aqui... além... mais longe... por onde eu movo o passo, \\
Como aves, que espantadas arrojam-se ao espaço, \\
Saudades e lembranças s'erguendo — bando alado — \\
Roçam por mim as asas voando p'ra o passado. \\
\end{verse}

\attrib{Boa Vista, 18 de novembro de 1867.}

\end{anexosenv}