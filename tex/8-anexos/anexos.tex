% ---
% Inicia os anexos
% ---
\begin{anexosenv}

% Imprime uma página indicando o início dos anexos
\partanexos

% ---
\chapter{Proprietários de terras em Brotas (1858-1862)}\label{anexo1}
% ---
\begin{table}[!htp]
\IBGEtab{
\caption{Registros de terras constantes no Livro Eclesial de Registro de Terras da Freguesia de Brotas (parte 1)}\label{tab:livroterrabrotas1}}
{
\begin{minipage}{\textwidth}
\begin{tiny}
\begin{tabular}{p{4cm}p{4cm}p{4cm}ll}
\toprule
Registrante									&Posse					&Localidade atual			&Folhas			&Registro		\\
\midrule
\midrule
Herculano Nunes dos Reis							&Cruz das Almas				&Cruz das Almas				&02f			&1			\\
Antonio Mendes Junior e irmãos							&Montivideo				&(desconhecida)				&02f			&2			\\
Bernardo Xavier de Castro							&Candeal Grande				&Candeal Grande				&02v			&3			\\
João Fagundes de Farias								&Terras (estrada de Brotas)		&Estrada de Brotas			&02v			&4			\\
José Joaquim de Santa Tereza							&Cruz da Redenção			&Cruz da Redenção			&03f			&5			\\
Evaristo Ladislau e Silva (Dr. Coronel)						&Mineiro				&(desconhecida)				&03f			&6			\\
Beijamim Vieira d'Ortas								&Ladeira da Bôa-Vista			&Ladeira da Boa Vista			&03v			&7			\\
Maria Bernardina da Conceição Lima e irmãos					&Terrenos (estrada de Brotas)		&Estrada de Brotas			&03v			&8			\\
Irmandade do S. S. Sacramento e N. S. de Brotas					&Terrenos (largo de Brotas)		&Largo de Brotas			&04f			&9			\\
Tomasia Bemvinda de Aquino							&Terrenos (na ladeira do Beijú)		&Ladeira do Beiju			&04f			&10			\\
Luis da Rocha Dias 								&Recreio				&Estrada do Matatu			&04v			&11			\\
Rosa Ladislau de Figueredo e Mello, Virginia Ladislau de Figueredo e Mello	&Chacôco (Fazenda)			&Campinas de Brotas			&04v			&12			\\
Rosa Ladislau de Figueredo e Mello, Virginia Ladislau de Figueredo e Mello	&Campina Pequena			&Campinas de Brotas			&04v			&12			\\
Rosa Ladislau de Figueredo e Mello, Virginia Ladislau de Figueredo e Mello	&Campina Grande (Fazenda)		&Campinas de Brotas			&04v			&12			\\
Rosa Ladislau de Figueredo e Mello, Virginia Ladislau de Figueredo e Mello	&Carregado (Fazenda)			&Campinas de Brotas			&04v			&12			\\
Michelina Ladislau e Silva, Joanna Fausta Ladislau e Silva			&Campina Pequena			&Campinas de Brotas			&05f			&13			\\
Rosa Ladislau de Figueredo e Mello						&Cruz da Redenção			&Cruz da Redenção			&05v			&14			\\
Mem de Amorim Filgueiras							&Terras					&Estrada de Brotas			&06f			&15			\\
Pedro Joaquim de Santa Barbara (Major) (finado)					&Rocinha				&Estrada do Matatu Grande		&06f			&16			\\
Tomás da Silva Paranhos (Capitão)						&Fontinha				&Brotas					&06v			&17			\\
Antonio Ferreira Franco								&Terras					&Estrada de Brotas para o Rio Vermelho	&07f			&18			\\
José Simões de Belas								&Terras					&Estrada de Brotas			&07f			&19			\\
Joaquim Antonio de Amorim Viana							&Terras					&Estrada de Brotas			&07v			&20			\\
Josefa Catarina de Souza Arraia							&Matatu					&Matatu					&07v			&21			\\
Francisco Lourenço da Costa Lima (Comend.)					&Terrenos				&Armação				&07v			&21			\\
Lucio Casimiro da Fonseca Galvão						&Terras (no Matatu Pequeno)		&Estrada do Matatu Pequeno		&08f			&22			\\
\bottomrule
\end{tabular} 
\end{tiny}
\end{minipage}
}
{\fonte{Elaboração do autor, com base em \textbf{BR BAAPB}, fundo Colonial, série Registros de Terra, livro 4675.}}
\end{table}

\begin{table}
\IBGEtab{
\caption{Registros de terras constantes no Livro Eclesial de Registro de Terras da Freguesia de Brotas (parte 2)}\label{tab:livroterrabrotas2}}
{
\begin{minipage}{\textwidth}
\begin{tiny}
\begin{tabular}{p{4cm}p{4cm}p{4cm}ll}
\toprule
Registrante									&Posse					&Localidade atual			&Folhas			&Registro		\\
\midrule
\midrule
Joaquim do Vale Cabral								&Matatu					&Matatu					&08f			&23			\\
Joaquim do Vale Cabral								&Matatu Pequeno				&Matatu Pequeno				&08f			&23			\\
Joaquim do Vale Cabral								&Matatu Grande				&Matatu Grande				&08v			&23			\\
Joaquim do Vale Cabral								&Matatu Grande				&Matatu Grande				&08v			&23			\\
Tomé Mamede de Jesus								&Terras (no Matatu Pequeno)		&Matatu Pequeno				&08v			&24			\\
Caetano Rodrigues Banha								&Terras (no Matatu)			&Matatu					&09f			&25			\\
Luis Pereira da Silva								&Terrenos (na estrada do Matatú)	&Estrada do Matatu			&09f			&26			\\
Sebastião Alvares da Rocha							&Terrenos (na estrada do Matatú)	&Estrada do Matatu			&09f			&27			\\
Lucas Ramos									&Terrenos (na estrada do Matatú)	&Estrada do Matatu			&09v			&28			\\
Manuel de Macedo								&Terrenos (Matatú Pequeno)		&Matatu Pequeno				&09v			&29			\\
Sisnando Alvares da Rocha							&Terras (na estrada do Matatu Grande)	&Estrada do Matatu Grande		&10f			&30			\\
Manuel de Macêdo Junior e Cosme de Macêdo Junior				&Terrenos (No Matatu Pequeno)		&Matatu Pequeno				&10f			&31			\\
João Francisco Regis								&Terrenos (na estrada do Matatú)	&Estrada do Matatu			&10v			&32			\\
João José Lino									&Terrenos (Matatú)			&Matatu					&10v			&33			\\
Maria da Piedade Tabirá Bahiense						&Acupe					&Acupe					&11f			&34			\\
Francisco Moreira Sampaio (os Herdeiros)					&Terras					&Candeal				&11f			&35			\\
Manuel Zacharias de Santa Isabel						&Terrenos (Matatú)			&Matatu					&11v			&36			\\
Luisa Maria da Gloria								&Engenhoca				&Estrada de Brotas			&11v			&37			\\
João da Silva Lopes								&Terrenos (estrada da rua da Vala)	&Rua da Vala				&12f			&38			\\
José Carlos Martins Ferreira							&Terrenos				&Estrada do Matatu Pequeno		&12f			&39			\\
João da Cruz de Moraes								&Terrenos				&Estrada da Ubarana			&12v			&40			\\
Luiz Antonio Ferreira								&Terrenos (na estrada de Brotas)	&Estrada de Brotas			&12v			&41			\\
Antonio da Silva Quaresma							&Terrenos (no largo de Brotas)		&Largo de Brotas			&13f			&42			\\
Luiz Antonio Ferreira								&Terrenos (na estrada de Brotas)	&Estrada de Brotas			&13f			&41A			\\
Chantre Manuel Joaquim de Almeida						&Terrenos				&Estrada de Brotas para o Rio Vermelho	&13v			&43			\\
Joaquim José Fernandes Maciel							&Terrenos				&Estrada de Brotas			&14f			&44			\\
Salustiano Israel								&Terrenos (Matatú Pequeno)		&Matatu Pequeno				&14f			&45			\\
Joaquim Inacio Ribeiro dos Santos						&Terreno (Alto do Sangradouro)		&Alto do Sangradouro			&14v			&46			\\
José de Barros Reis								&Terras (Matatu)			&Matatu					&14v			&47			\\
Antonio José Teixeira Junior							&Terreno (no alto do Sangradouro)	&Alto do Sangradouro			&15f			&48			\\
Ana Ribeiro									&Terrenos (no Matatu)			&Matatu					&15v			&49			\\
Joana Antonia									&Terrenos (na estrada da União)		&Estrada da União			&15v			&50			\\
José Ricardo da Silva Terra							&Terreno (Na Quinta das Brotas)		&Quinta das Beatas			&16f			&51			\\
Henriqueta Flôres Claques Lobo							&Bulhoens (Roça)			&Bulhões				&16v			&52			\\
Felisardo Jeronimo Soares (Padre)						&Pitangueiras				&Pitangueiras				&16v			&53			\\
Francisco de Assis Gomes							&Terrenos (na estrada do Matatu)	&Estrada do Matatu			&17f			&54			\\
Marcellino de Souza Teles							&Terrenos (na estrada do Matatú)	&Estrada do Matatu			&17f			&55			\\
Viscondessa do Rio Vermelho							&Terreno (fronteira à Igreja de Brotas)	&Estrada de Brotas			&17v			&56			\\
Viscondessa do Rio Vermelho 							&Terras					&Estrada de Brotas para o Rio Vermelho	&17v			&56			\\
Francisca Mariana Rita Balthazar da Silveira					&Lucaya					&Estrada de Brotas para o Rio Vermelho	&18f			&57			\\
\bottomrule
\end{tabular} 
\end{tiny}
\end{minipage}
}
{\fonte{Elaboração do autor, com base em \textbf{BR BAAPB}, fundo Colonial, série Registros de Terra, livro 4675.}}
\end{table}

\begin{table}
\IBGEtab{
\caption{Registros de terras constantes no Livro Eclesial de Registro de Terras da Freguesia de Brotas (parte 3)}\label{tab:livroterrabrotas3}}
{
\begin{minipage}{\textwidth}
\begin{tiny}
\begin{tabular}{p{4cm}p{4cm}p{4cm}ll}
\toprule
Registrante									&Posse					&Localidade atual	 		&Folhas			&Registro		\\
\midrule
\midrule
Feliciano Primo Ferreira							&Terrenos (no Matatu Pequeno)		&Matatu Pequeno				&18f			&58			\\
José Martins da Costa								&Terras					&Estrada do Matatu			&18v			&59			\\
Rafael dos Anjos e mais herdeiros						&Terrenos (No Beco da Campina)		&Campinas de Brotas			&18v			&60			\\
Escolastica Maria de Santa Ana							&Terrenos				&Matatu Pequeno				&19f			&61			\\
Antonio Peixoto da Silva e Melo							&Acupe					&Acupe					&19f			&62			\\
Egidio Pires									&Terrenos				&Sangradouro				&19v			&63			\\
Antonio Peixoto da Silva Melo							&Terrenos (Acupe)			&Acupe					&19v			&62A			\\
Gemeniano Lopes Perdigão							&Terrenos (Matatú)			&Matatu					&20f			&64			\\
Antonio Joaquim da Costa							&Terrenos				&Matatu Pequeno				&20f			&65			\\
Maria Rosa Gomes da Silva e herdeiros						&Terrenos (Acú)				&Acupe					&20v			&66			\\
Maria Rosa Gomes da Silva e seus filhos menores				&Terrenos				&Acupe					&20v			&66			\\
Macario da Silva								&Terrenos (Matatú Pequeno)		&Matatu Pequeno				&21f			&67			\\
Francisca Romualda dos Santos							&Terrenos (Matatu Pequeno)		&Matatu Pequeno				&21v			&68			\\
Manuel Agostinho Cruz e Melo							&Terrenos				&Sangradouro				&21v			&69			\\
Marciana Ribeira da Silva							&Terrenos				&Acupe					&22f			&70			\\
Henrique Francisco de Oliveira							&Terrenos (Matatú Pequeno)		&Matatu Pequeno				&22f			&71			\\
Antonia Francisca Leopoldina de Novais Barata					&Terrenos				&Torre					&22v			&72			\\
Joaquim Teixeira de Oliveira							&Terrenos				&Estrada de Brotas			&23f			&73			\\
Francisco Antonio Bahia								&Terrenos				&Campinas de Brotas			&23f			&74			\\
Ana Francisca de Carvalho							&Engenho Velho (Fazenda)		&Engenho Velho de Brotas		&23v			&75			\\
Bernardino de Sena Moreira							&Terrenos				&Lucaia					&24f			&76			\\
Florencio Benjamim de Almeida Pires						&Terrenos				&Estrada de Brotas para o Rio Vermelho	&24f			&77			\\
Joaquim dos Santos								&Terrenos (Matatú Pequeno)		&Matatu Pequeno				&24v			&78			\\
José Hermogenes da Costa de Faria						&Terreno (Matatu Grande)		&Matatu Grande				&24v			&79			\\
Maria Francisca de Santana							&Terrenos				&Acupe					&25f			&80			\\
Rosendo Valentin da Cruz							&Terrenos (no Engenho Velho)		&Engenho Velho de Brotas		&25v			&81			\\
Manuel do Bonfim								&Terrenos (no Matatú Pequeno)		&Matatu Pequeno				&25v			&82			\\
Manuel Eloi Pontes								&Terrenos				&Estrada de Brotas			&26f			&83			\\
Bernardino José de Almeida							&Terrenos (na ladeira do Beijú)		&Ladeira do Beiju			&26f			&84			\\
Domingos Inacio da Conceição							&Terrenos				&Matatu Pequeno				&26v			&85			\\
Francisca de Sales Bahia							&Terrenos				&Cruz das Almas				&26v			&86			\\
Duarte de Oliveira								&Acú					&Acupe					&27f			&87			\\
José Gonçalves Monção								&Terrenos				&Estrada do Engenho Velho		&27v			&88			\\
Maria dos Santos Alves								&Terrenos (Matatú Pequeno)		&Matatu Pequeno				&27v			&89			\\
Vicente Ferreira de Santana e irmãos						&Terreno (no Matatú Pequeno)		&Matatu Pequeno				&28f			&90			\\
Maria Isabel do Ó Freire							&Terrenos				&Matatu					&28f			&91			\\
Angela Cardoso de Santa Barbara							&Terrenos				&Estrada do Matatu			&28v			&92			\\
José Antonio Pinto								&Matatu Grande (Fazenda)		&Matatu Grande				&28v			&93			\\
Antonio Ramos									&Terrenos				&--					&29f			&94			\\
Maria do Sacramento								&Terrenos				&--					&29f			&95			\\
Antonio Monteiro de Carvalho							&Terreno				&--					&29v			&96			\\
Manuel Patricio Xavier								&Terrenos				&Matatu Pequeno				&29v			&97			\\
Manuel Patricio Xavier								&Terrenos				&Matatu Pequeno				&30f			&98			\\
Manuel Patricio Xavier								&Terrenos				&Matatu Pequeno				&30f			&98A			\\
Luis José de Almeida								&Candeal				&Candeal				&30v			&99			\\
José Joaquim de Santa Tereza							&Pitangueiras				&Pitangueiras				&31f			&100			\\
Joaquim Antonio Pereira Barreto							&Terrenos (no Matatu Pequeno)		&Matatu Pequeno				&31f			&101			\\
Joaquim Barbosa de Oliveira							&Terrenos (Na Cruz da Redenção)		&Cruz da Redenção			&31v			&102			\\
Joaquim Olavo da Silva Rebelo (Tenente Coronel)					&Terrenos (No Sangradouro)		&Sangradouro				&31v			&103			\\
Custodia Angela Marinha								&Terrenos (no Matatu Pequeno)		&Matatu					&32f			&104			\\
Faustino José de Santana							&Terrenos (no largo de Brotas)		&Largo de Brotas			&32f			&105			\\
\bottomrule
\end{tabular} 
\end{tiny}
\end{minipage}
}
{\fonte{Elaboração do autor, com base em \textbf{BR BAAPB}, fundo Colonial, série Registros de Terra, livro 4675.}}
\end{table}

\begin{table}[ht]
\IBGEtab{
\caption{Registros de terras constantes no Livro Eclesial de Registro de Terras da Freguesia de Brotas (parte 4)}\label{tab:livroterrabrotas4}}
{
\begin{minipage}{\textwidth}
\begin{tiny}
\begin{tabular}{p{4cm}p{4cm}p{4cm}ll}
\toprule
Registrante									&Posse					&Localidade atual			&Folhas			&Registro		\\
\midrule
\midrule
Maria Amelia de Carvalho Martagão						&Terrenos (corredores do Acú)		&Acupe					&32v			&106			\\
Maria Constança Ebé								&Acú					&Acupe					&33f			&107			\\
Antonio Pereira do Rio								&Cruz da Redenção			&Cruz da Redenção			&33f			&108			\\
Anna Rosa Joaquina do Amor Divino							&Terrenos				&--					&33v			&109			\\
Thimothea Maria Lopes								&Terreno (Acupe)			&Acupe					&34f			&110			\\
Elias Lopes de São Jeronimo							&Terrenos (no Acú)			&Acupe					&34f			&111			\\
Caetana Rosa da Conceição							&Açú					&Acupe					&34v			&112			\\
Manuel de Santa Isabel								&Acú					&Acupe					&34v			&113			\\
Rosa Maria do Amor Divino							&Acupe					&Acupe					&35f			&114			\\
Cipriano Freire de Carvalho							&Terrenos (no Matatu Grande)		&Matatu Grande				&35f			&115			\\
Domingos José Garcia								&Terrenos				&Estrada de Brotas			&35v			&116			\\
Joaquim de Costa Pinheiro (Ten. Cel.)						&Ladeira da Boa Vista			&Ladeira da Boa Vista			&35v			&117			\\
João José de Azevedo Lima							&Terras (Ladeira da Boa Vista)		&Ladeira da Boa Vista			&36f			&118			\\
Antonio Joaquim da Silva e Abreu						&Santa Cruz (Fazenda)			&Santa Cruz				&36f			&119			\\
José Alves do Amaral								&Alagôa					&Amaralina				&36v			&119A			\\
José Rodrigues de Figueiredo							&Terras (Sangradouro)			&Sangradouro				&37f			&120			\\
Benjamim Pereira Marinho							&Casa (no Largo de Brotas)		&Largo de Brotas			&37f			&121			\\
Braz Balthazar da Silveira							&Matatu Pequeno (Roça)			&Matatu Pequeno				&37v			&122			\\
Viscondessa do Rio Vermelho, ou seu filho, Barão do Rio Vermelho		&Pituba (Fazenda)			&Pituba					&38f			&124			\\
José Joaquim de Santa Tereza							&Terrenos				&Largo de Brotas			&38f			&125			\\
Viscondessa do Rio Vermelho e herdeiro						&Terrenos (foreiros)			&Boca do Rio				&38v			&126			\\
Francisco Xavier dos Reis (Dr.)							&Acú					&Acupe					&39f			&127			\\
Francisco Antonio do Espirito Santo						&Terrenos (no Matatu)			&Matatu					&39f			&128			\\
Jacinto Muniz Barreto								&Quinta das Brotas (Fazenda)		&Quinta das Beatas			&39v			&129			\\
José Luis da Rocha								&Terrenos				&Estrada de Brotas					&39v			&130			\\
Manuel Inacio de Barros Paim							&Ubarana (Fazenda)			&Pituba e Amaralina			&40f			&131			\\
Manuel José de Santana (Cirurgião Mor)						&Terrenos				&Estrada de Brotas					&40f			&132			\\
Francisco Pires de Carvalho Albuquerque						&Torre (Roça)				&Torre					&40f			&133			\\
Ana Maria Madalena Rejente							&Quinta das Beatas (Fazenda)		&Quinta das Beatas			&40v			&134			\\
Angelo Francisco de Andrade							&Terrenos				&Estrada de Brotas					&40v			&135			\\
Francisco Gomes de Castro Dr.							&Terreno (no alto do Sangradouro)	&Sangradouro				&41f			&136			\\
Maria Joaquina do Espirito Santo						&Terrenos				&Sangradouro					&41f			&137			\\
Joaquim José de Santana Gomes por seus cunhados					&Terrenos				&Estrada do Matatu					&41v			&138			\\
Antonio Ramos de Silva e outro							&Terrenos (no Acú)			&Acupe					&41v			&139			\\
Firmino Pacifico Duarte Gameleira						&Palacete				&Estrada do Matatu					&42f			&140			\\
\bottomrule
\end{tabular} 
\end{tiny}
\end{minipage}
}
{\fonte{Elaboração do autor, com base em \textbf{BR BAAPB}, fundo Colonial, série Registros de Terra, livro 4675.}}
\end{table}


% ---
\chapter{Proprietários rurais de Brotas em 1920}
% ---
\begin{tiny}
\begin{longtable}{cc}
\caption{Nome dos proprietários rurais do distrito de Brotas e localidade de suas terras (1920)}\label{tab:proprurais}\\
\hline Proprietários & Nome do estabelecimento (ou localidade) \\ \hline\hline \endhead
\hline \multicolumn{2}{c}{Continua na próxima página...} \\ \endfoot
\hline \endlastfoot
Joanna B. de Souza  & Brotas \\
Pedro F. H. Pires  & Brotas \\
José Visco  & Brotas \\
António S. Souza  & Brotas \\
Luiz Saraiva  & Brotas \\
Augusto R. de Senna  & Brotas \\
Juventino R. de Almeida  & Brotas \\
Ricardo A. Pereira  & Brotas \\
Antonio A. Cupim & Brotas \\
José A. Silva & Brotas \\
Avelino J. Pereira & Brotas \\
Luiz H. de Souza & Brotas \\
Salustiano R. Pinto & Brotas \\
João de A. Ramos & Brotas \\
Rodrigo A. Araújo & Brotas \\
Estevão G. da Encarnação & Brotas \\
Trifino P. de Souza & Brotas \\
Joaquim F. A. Rosa & Brotas \\
Affonso C. Mello & Brotas \\
João P. de Queiroz & Brotas \\
João P. dos Santos & Brotas \\
Hermenegildo P. Souza & Brotas \\
Silvann P. Ramos & Brotas \\
Jesuino A. Monteiro & Brotas \\
Francisco R. A Prata & Brotas \\
António P. Melchiades & Brotas \\
Raphael Alonso & Brotas \\
Pedro J. Mussitahyba & Brotas \\
João de D. Ribeiro & Brotas \\
Zacharias de Oliveira & Brotas \\
Raphael E. da Purificação & Brotas \\
Gabriel A. dos Santos & Brotas \\
Francisco X. da Silva & Brotas \\
Abel A. Amoedo & Brotas \\
Francelino S. B. Figueiredo & Brotas \\
João B. da Silva & Brotas \\
António F. Simões & Brotas \\
Esperidião B. de Argollo & Brotas \\
Olegário F. Cardoso & Brotas \\
João B. Wanderley & Brotas \\
Fortunato J. da Costa & Brotas \\
Joaquim F. Teixeira & Brotas \\
Segundo Garrido & Brotas \\
Pedro S. Nascimento & Brotas \\
Bernardo Lins & Brotas \\
Coronel Frederico A. Rodrigues da Costa & N. S. do Matatú \\
\hline
\end{longtable}
\end{tiny}

% ---
\chapter{Plantas selecionadas de Amaralina}
% ---




% ---
\chapter{Plantas selecionadas do Castro Neves, Ladeira dos Galés e vizinhança}
% ---





% ---
\chapter{Plantas selecionadas do Engenho Velho de Brotas e vizinhança}
% ---




% ---
\chapter{Plantas selecionadas da Estrada Dois de Julho e vizinhança}
% ---



% ---
\chapter{Plantas selecionadas do Rio Vermelho}
% ---



% ---
\chapter{Plantas selecionadas da Estrada de Brotas e vizinhança}
% ---



% ---
\chapter{Plantas selecionadas do Acupe e vizinhança}
% ---




% ---
\chapter{Plantas selecionadas da Quinta das Beatas e vizinhança}
% ---




% ---
\chapter{Plantas selecionadas do Matatu Grande e Matatu Pequeno}
% ---




% ---
\chapter{Plantas selecionadas da Quinta das Beatas e arredores}
% ---








% ---
\chapter{Documentos selecionados}
% ---




% ---
\chapter{Facsímiles}
% ---


% ---
\chapter{``A Boa Vista'', por Castro Alves}\label{cap:boavista}
% ---

\poemtitle{A Boa Vista}
\settowidth{\versewidth}{}

\begin{flushright}
\textit{Sonha, poeta, sonha! Aqui sentado \\
No tosco assento da janela antiga,\\
Apóias sobre a mão a face pálida,\\
Sorrindo — dos amores à cantiga.}\\
Álvares de Azevedo
\end{flushright}

\begin{verse}
Era uma tarde triste, mas límpida e suave... \\
Eu — pálido poeta — seguia triste e grave \\
A estrada, que conduz ao campo solitário, \\
Como um filho, que volta ao paternal sacrário, \\
\end{verse}

\begin{verse}
E ao longe abandonando o múrmur da cidade \\
— Som vago, que gagueja em meio à imensidade, — \\
No drama do crepúsculo eu escutava atento \\
A surdina da tarde ao sol, que morre lento. \\
\end{verse}

\begin{verse}
A poeira da estrada meu passo levantava, \\
Porém minh'alma ardente no céu azul marchava \\
E os astros sacudia no vôo violento \\
— Poeira, que dormia no chão do firmamento. \\
\end{verse}

\begin{verse}
A pávida andorinha, que o vendaval fustiga, \\
Procura os coruchéus da catedral antiga. \\
Eu — andorinha entregue aos vendavais do inverno, \\
Ia seguindo triste p'ra o velho lar paterno. \\
\end{verse}

\begin{verse}
Como a águia, que do ninho talhado no rochedo \\
Ergue o pescoço calvo por cima do fraguedo, \\
— (P'ra ver no céu a nuvem, que espuma o firmamento, \\
E o mar, — corcel que espuma ao látego do vento...) \\
Longe o feudal castelo levanta a antiga torre, \\
Que aos raios do poente brilhante sol escorre! \\
Ei-lo soberbo e calmo o abutre de granito \\
Mergulhando o pescoço no seio do infinito \\
E lá de cima olhando com seus clarões vermelhos \\
Os tetos, que a seus pés parecem de joelhos!... \\
\end{verse}

\begin{verse}
Não! Minha velha torre! Oh! atalaia antiga, \\
Tu olhas esperando alguma face amiga, \\
E perguntas talvez ao vento, que em ti chora: \\
``Por que não volta mais o meu senhor d'outrora? \\
Por que não vem sentar-se no banco do terreiro \\
Ouvir das criancinhas o riso feiticeiro, \\
E pensando no lar, na ciência, nos pobres \\
Abrigar nesta sombra seus pensamentos nobres? \\
\end{verse}

\begin{verse}
Onde estão as crianças — grupo alegre e risonho \\
— Que escondiam-se atrás do cipreste tristonho... \\
\end{verse}

\begin{verse}
Ou que enforcaram rindo um feio Pulchinello, \\
Enquanto a doce Mãe, que é toda amor, desvelo \\
Ralha com um rir divino o grupo folgazão, \\
Que vem correndo alegre beijar-lhe a branca mão?...'' \\
\end{verse}

\begin{verse}
É nisto que tu cismas, ó torre abandonada, \\
Vendo deserto o parque e solitária a estrada. \\
No entanto eu — estrangeiro, que tu já não conheces — \\
No limiar de joelhos só tenho pranto e preces. \\
\end{verse}

\begin{verse}
Oh! deixem-me chorar!... Meu lar... meu doce ninho! \\
Abre a vetusta grade ao filho teu mesquinho! \\
Passado — mar imenso!... inunda-me em fragrância! \\
Eu não quero lauréis, quero as rosas da infância. \\
\end{verse}

\begin{verse}
Ai! Minha triste fronte, aonde as multidões \\
Lançaram misturadas glórias e maldições... \\
Acalenta em teu seio, ó solidão sagrada! \\
Deixa est'alma chorar em teu ombro encostada! \\
\end{verse}

\begin{verse}
Meu lar está deserto... Um velho cão de guarda \\
Veio saltando a custo roçar-me a testa parda, \\
Lamber-me após os dedos, porém a sós consigo \\
Rusgando com o direito, que tem um velho amigo... \\
Como tudo mudou-se!... O jardim 'stá inculto \\
As roseiras morreram do vento ao rijo insulto... \\
A erva inunda a terra; o musgo trepa os muros \\
A ortiga silvestre enrola em nós impuros \\
Uma estátua caída, em cuja mão nevada \\
A aranha estende ao sol a teia delicada!... \\
Mergulho os pés nas plantas selvagens, espalmadas, \\
As borboletas fogem-me em lúcidas manadas... \\
E ouvindo-me as passadas tristonhas, taciturnas, \\
Os grilos, que cantavam, calaram-se nas furnas... \\
\end{verse}

\begin{verse}
Oh! jardim solitário! Relíquia do passado! \\
Minh'alma, como tu, é um parque arruinado! \\
Morreram-me no seio as rosas em fragrância, \\
Veste o pesar os muros dos meus vergéis da infância, \\
\end{verse}

\begin{verse}
A estátua do talento, que pura em mim s'erguia, \\
Jaz hoje — e nela a turba enlaça uma ironia!... \\
Ao menos como tu, lá d'alma num recanto \\
Da casta poesia ainda escuto o canto, \\
— Voz do céu, que consola, se o mundo nos insulta, \\
E na gruta do seio murmura um treno oculta. \\
\end{verse}

\begin{verse}
Entremos!... Quantos ecos na vasta escadaria, \\
Nos longos corredores respondem-me à porfia!... \\
\end{verse}

\begin{verse}
Oh! casa de meus pais!... A um crânio já vazio, \\
Que o hóspede largando deixou calado e frio, \\
Compara-te o estrangeiro — caminhando indiscreto \\
Nestes salões imensos, que abriga o vasto teto. \\
\end{verse}

\begin{verse}
Mas eu no teu vazio — vejo uma multidão \\
Fala-me o teu silêncio — ouço-te a solidão!... \\
Povoam-se estas salas... \\
\end{verse}

\begin{verse}
E eu vejo lentamente \\
No solo resvalarem falando tenuemente \\
Dest'alma e deste seio as sombras venerandas \\
Fantasmas adorados — visões sutis e brandas... \\
\end{verse}

\begin{verse}
Aqui... além... mais longe... por onde eu movo o passo, \\
Como aves, que espantadas arrojam-se ao espaço, \\
Saudades e lembranças s'erguendo — bando alado — \\
Roçam por mim as asas voando p'ra o passado. \\
\end{verse}

\attrib{Boa Vista, 18 de novembro de 1867.}

\end{anexosenv}