% ---
% Inicia os anexos
% ---
\begin{anexosenv}

% Imprime uma página indicando o início dos anexos
\partanexos
\label{cap:anexos}

% ---
\chapter{Ficha de coleta de informações sobre os processos de licença de obra}
% ---

% ---
\chapter{Proprietários de terras em Brotas (1858-1862)}\label{anexo1}
% ---
\begin{table}[!htp]
\IBGEtab{
\caption{Registros de terras constantes no Livro Eclesial de Registro de Terras da Freguesia de Brotas (parte 1)}\label{tab:livroterrabrotas1}}
{
\begin{minipage}{\textwidth}
\begin{tiny}
\begin{tabular}{p{4cm}p{4cm}p{4cm}ll}
\toprule
Registrante									&Posse					&Localidade atual			&Folhas			&Registro		\\
\midrule
\midrule
Herculano Nunes dos Reis							&Cruz das Almas				&Cruz das Almas				&02f			&1			\\
Antonio Mendes Junior e irmãos							&Montivideo				&Estrada de Brotas para o Rio Vermelho				&02f			&2			\\
Bernardo Xavier de Castro							&Candeal Grande				&Candeal Grande				&02v			&3			\\
João Fagundes de Farias								&Terras (estrada de Brotas)		&Estrada de Brotas			&02v			&4			\\
José Joaquim de Santa Tereza							&Cruz da Redenção			&Cruz da Redenção			&03f			&5			\\
Evaristo Ladislau e Silva (Dr. Coronel)						&Mineiro				&Estrada de Brotas				&03f			&6			\\
Beijamim Vieira d'Ortas								&Ladeira da Bôa-Vista			&Ladeira da Boa Vista			&03v			&7			\\
Maria Bernardina da Conceição Lima e irmãos					&Terrenos (estrada de Brotas)		&Estrada de Brotas			&03v			&8			\\
Irmandade do S. S. Sacramento e N. S. de Brotas					&Terrenos (largo de Brotas)		&Largo de Brotas			&04f			&9			\\
Tomasia Bemvinda de Aquino							&Terrenos (na ladeira do Beijú)		&Ladeira do Beiju			&04f			&10			\\
Luis da Rocha Dias 								&Recreio				&Estrada do Matatu			&04v			&11			\\
Rosa Ladislau de Figueredo e Mello, Virginia Ladislau de Figueredo e Mello	&Chacôco (Fazenda)			&Campinas de Brotas			&04v			&12			\\
Rosa Ladislau de Figueredo e Mello, Virginia Ladislau de Figueredo e Mello	&Campina Pequena			&Campinas de Brotas			&04v			&12			\\
Rosa Ladislau de Figueredo e Mello, Virginia Ladislau de Figueredo e Mello	&Campina Grande (Fazenda)		&Campinas de Brotas			&04v			&12			\\
Rosa Ladislau de Figueredo e Mello, Virginia Ladislau de Figueredo e Mello	&Carregado (Fazenda)			&Campinas de Brotas			&04v			&12			\\
Michelina Ladislau e Silva, Joanna Fausta Ladislau e Silva			&Campina Pequena			&Campinas de Brotas			&05f			&13			\\
Rosa Ladislau de Figueredo e Mello						&Cruz da Redenção			&Cruz da Redenção			&05v			&14			\\
Mem de Amorim Filgueiras							&Terras					&Estrada de Brotas			&06f			&15			\\
Pedro Joaquim de Santa Barbara (Major) (finado)					&Rocinha				&Estrada do Matatu Grande		&06f			&16			\\
Tomás da Silva Paranhos (Capitão)						&Fontinha				&Brotas					&06v			&17			\\
Antonio Ferreira Franco								&Terras					&Estrada de Brotas para o Rio Vermelho	&07f			&18			\\
José Simões de Belas								&Terras					&Estrada de Brotas			&07f			&19			\\
Joaquim Antonio de Amorim Viana							&Terras					&Estrada de Brotas			&07v			&20			\\
Josefa Catarina de Souza Arraia							&Matatu					&Matatu					&07v			&21			\\
Francisco Lourenço da Costa Lima (Comend.)					&Terrenos				&Armação				&07v			&21			\\
Lucio Casimiro da Fonseca Galvão						&Terras (no Matatu Pequeno)		&Estrada do Matatu Pequeno		&08f			&22			\\
\bottomrule
\end{tabular} 
\end{tiny}
\end{minipage}
}
{\fonte{Elaboração do autor, com base em \textbf{BR BAAPB}, fundo Colonial, série Registros de Terra, livro 4675.}}
\end{table}

\begin{table}
\IBGEtab{
\caption{Registros de terras constantes no Livro Eclesial de Registro de Terras da Freguesia de Brotas (parte 2)}\label{tab:livroterrabrotas2}}
{
\begin{minipage}{\textwidth}
\begin{tiny}
\begin{tabular}{p{4cm}p{4cm}p{4cm}ll}
\toprule
Registrante									&Posse					&Localidade atual			&Folhas			&Registro		\\
\midrule
\midrule
Joaquim do Vale Cabral								&Matatu					&Matatu					&08f			&23			\\
Joaquim do Vale Cabral								&Matatu Pequeno				&Matatu Pequeno				&08f			&23			\\
Joaquim do Vale Cabral								&Matatu Grande				&Matatu Grande				&08v			&23			\\
Joaquim do Vale Cabral								&Matatu Grande				&Matatu Grande				&08v			&23			\\
Tomé Mamede de Jesus								&Terras (no Matatu Pequeno)		&Matatu Pequeno				&08v			&24			\\
Caetano Rodrigues Banha								&Terras (no Matatu)			&Matatu					&09f			&25			\\
Luis Pereira da Silva								&Terrenos (na estrada do Matatú)	&Estrada do Matatu			&09f			&26			\\
Sebastião Alvares da Rocha							&Terrenos (na estrada do Matatú)	&Estrada do Matatu			&09f			&27			\\
Lucas Ramos									&Terrenos (na estrada do Matatú)	&Estrada do Matatu			&09v			&28			\\
Manuel de Macedo								&Terrenos (Matatú Pequeno)		&Matatu Pequeno				&09v			&29			\\
Sisnando Alvares da Rocha							&Terras (na estrada do Matatu Grande)	&Estrada do Matatu Grande		&10f			&30			\\
Manuel de Macêdo Junior e Cosme de Macêdo Junior				&Terrenos (No Matatu Pequeno)		&Matatu Pequeno				&10f			&31			\\
João Francisco Regis								&Terrenos (na estrada do Matatú)	&Estrada do Matatu			&10v			&32			\\
João José Lino									&Terrenos (Matatú)			&Matatu					&10v			&33			\\
Maria da Piedade Tabirá Bahiense						&Acupe					&Acupe					&11f			&34			\\
Francisco Moreira Sampaio (os Herdeiros)					&Terras					&Candeal				&11f			&35			\\
Manuel Zacharias de Santa Isabel						&Terrenos (Matatú)			&Matatu					&11v			&36			\\
Luisa Maria da Gloria								&Engenhoca				&Estrada de Brotas			&11v			&37			\\
João da Silva Lopes								&Terrenos (estrada da rua da Vala)	&Rua da Vala				&12f			&38			\\
José Carlos Martins Ferreira							&Terrenos				&Estrada do Matatu Pequeno		&12f			&39			\\
João da Cruz de Moraes								&Terrenos				&Estrada da Ubarana			&12v			&40			\\
Luiz Antonio Ferreira								&Terrenos (na estrada de Brotas)	&Estrada de Brotas			&12v			&41			\\
Antonio da Silva Quaresma							&Terrenos (no largo de Brotas)		&Largo de Brotas			&13f			&42			\\
Luiz Antonio Ferreira								&Terrenos (na estrada de Brotas)	&Estrada de Brotas			&13f			&41A			\\
Chantre Manuel Joaquim de Almeida						&Terrenos				&Estrada de Brotas para o Rio Vermelho	&13v			&43			\\
Joaquim José Fernandes Maciel							&Terrenos				&Estrada de Brotas			&14f			&44			\\
Salustiano Israel								&Terrenos (Matatú Pequeno)		&Matatu Pequeno				&14f			&45			\\
Joaquim Inacio Ribeiro dos Santos						&Terreno (Alto do Sangradouro)		&Alto do Sangradouro			&14v			&46			\\
José de Barros Reis								&Terras (Matatu)			&Matatu					&14v			&47			\\
Antonio José Teixeira Junior							&Terreno (no alto do Sangradouro)	&Alto do Sangradouro			&15f			&48			\\
Ana Ribeiro									&Terrenos (no Matatu)			&Matatu					&15v			&49			\\
Joana Antonia									&Terrenos (na estrada da União)		&Estrada da União			&15v			&50			\\
José Ricardo da Silva Terra							&Terreno (Na Quinta das Brotas)		&Quinta das Beatas			&16f			&51			\\
Henriqueta Flôres Claques Lobo							&Bulhoens (Roça)			&Bulhões				&16v			&52			\\
Felisardo Jeronimo Soares (Padre)						&Pitangueiras				&Pitangueiras				&16v			&53			\\
Francisco de Assis Gomes							&Terrenos (na estrada do Matatu)	&Estrada do Matatu			&17f			&54			\\
Marcellino de Souza Teles							&Terrenos (na estrada do Matatú)	&Estrada do Matatu			&17f			&55			\\
Viscondessa do Rio Vermelho							&Terreno (fronteira à Igreja de Brotas)	&Estrada de Brotas			&17v			&56			\\
Viscondessa do Rio Vermelho 							&Terras					&Estrada de Brotas para o Rio Vermelho	&17v			&56			\\
Francisca Mariana Rita Balthazar da Silveira					&Lucaya					&Estrada de Brotas para o Rio Vermelho	&18f			&57			\\
\bottomrule
\end{tabular} 
\end{tiny}
\end{minipage}
}
{\fonte{Elaboração do autor, com base em \textbf{BR BAAPB}, fundo Colonial, série Registros de Terra, livro 4675.}}
\end{table}

\begin{table}
\IBGEtab{
\caption{Registros de terras constantes no Livro Eclesial de Registro de Terras da Freguesia de Brotas (parte 3)}\label{tab:livroterrabrotas3}}
{
\begin{minipage}{\textwidth}
\begin{tiny}
\begin{tabular}{p{4cm}p{4cm}p{4cm}ll}
\toprule
Registrante									&Posse					&Localidade atual	 		&Folhas			&Registro		\\
\midrule
\midrule
Feliciano Primo Ferreira							&Terrenos (no Matatu Pequeno)		&Matatu Pequeno				&18f			&58			\\
José Martins da Costa								&Terras					&Estrada do Matatu			&18v			&59			\\
Rafael dos Anjos e mais herdeiros						&Terrenos (No Beco da Campina)		&Campinas de Brotas			&18v			&60			\\
Escolastica Maria de Santa Ana							&Terrenos				&Matatu Pequeno				&19f			&61			\\
Antonio Peixoto da Silva e Melo							&Acupe					&Acupe					&19f			&62			\\
Egidio Pires									&Terrenos				&Sangradouro				&19v			&63			\\
Antonio Peixoto da Silva Melo							&Terrenos (Acupe)			&Acupe					&19v			&62A			\\
Gemeniano Lopes Perdigão							&Terrenos (Matatú)			&Matatu					&20f			&64			\\
Antonio Joaquim da Costa							&Terrenos				&Matatu Pequeno				&20f			&65			\\
Maria Rosa Gomes da Silva e herdeiros						&Terrenos (Acú)				&Acupe					&20v			&66			\\
Maria Rosa Gomes da Silva e seus filhos menores				&Terrenos				&Acupe					&20v			&66			\\
Macario da Silva								&Terrenos (Matatú Pequeno)		&Matatu Pequeno				&21f			&67			\\
Francisca Romualda dos Santos							&Terrenos (Matatu Pequeno)		&Matatu Pequeno				&21v			&68			\\
Manuel Agostinho Cruz e Melo							&Terrenos				&Sangradouro				&21v			&69			\\
Marciana Ribeira da Silva							&Terrenos				&Acupe					&22f			&70			\\
Henrique Francisco de Oliveira							&Terrenos (Matatú Pequeno)		&Matatu Pequeno				&22f			&71			\\
Antonia Francisca Leopoldina de Novais Barata					&Terrenos				&Torre					&22v			&72			\\
Joaquim Teixeira de Oliveira							&Terrenos				&Estrada de Brotas			&23f			&73			\\
Francisco Antonio Bahia								&Terrenos				&Campinas de Brotas			&23f			&74			\\
Ana Francisca de Carvalho							&Engenho Velho (Fazenda)		&Engenho Velho de Brotas		&23v			&75			\\
Bernardino de Sena Moreira							&Terrenos				&Lucaia					&24f			&76			\\
Florencio Benjamim de Almeida Pires						&Terrenos				&Estrada de Brotas para o Rio Vermelho	&24f			&77			\\
Joaquim dos Santos								&Terrenos (Matatú Pequeno)		&Matatu Pequeno				&24v			&78			\\
José Hermogenes da Costa de Faria						&Terreno (Matatu Grande)		&Matatu Grande				&24v			&79			\\
Maria Francisca de Santana							&Terrenos				&Acupe					&25f			&80			\\
Rosendo Valentin da Cruz							&Terrenos (no Engenho Velho)		&Engenho Velho de Brotas		&25v			&81			\\
Manuel do Bonfim								&Terrenos (no Matatú Pequeno)		&Matatu Pequeno				&25v			&82			\\
Manuel Eloi Pontes								&Terrenos				&Estrada de Brotas			&26f			&83			\\
Bernardino José de Almeida							&Terrenos (na ladeira do Beijú)		&Ladeira do Beiju			&26f			&84			\\
Domingos Inacio da Conceição							&Terrenos				&Matatu Pequeno				&26v			&85			\\
Francisca de Sales Bahia							&Terrenos				&Cruz das Almas				&26v			&86			\\
Duarte de Oliveira								&Acú					&Acupe					&27f			&87			\\
José Gonçalves Monção								&Terrenos				&Estrada do Engenho Velho		&27v			&88			\\
Maria dos Santos Alves								&Terrenos (Matatú Pequeno)		&Matatu Pequeno				&27v			&89			\\
Vicente Ferreira de Santana e irmãos						&Terreno (no Matatú Pequeno)		&Matatu Pequeno				&28f			&90			\\
Maria Isabel do Ó Freire							&Terrenos				&Matatu					&28f			&91			\\
Angela Cardoso de Santa Barbara							&Terrenos				&Estrada do Matatu			&28v			&92			\\
José Antonio Pinto								&Matatu Grande (Fazenda)		&Matatu Grande				&28v			&93			\\
Antonio Ramos									&Terrenos				&--					&29f			&94			\\
Maria do Sacramento								&Terrenos				&--					&29f			&95			\\
Antonio Monteiro de Carvalho							&Terreno				&Rua da Vala					&29v			&96			\\
Manuel Patricio Xavier								&Terrenos				&Matatu Pequeno				&29v			&97			\\
Manuel Patricio Xavier								&Terrenos				&Matatu Pequeno				&30f			&98			\\
Manuel Patricio Xavier								&Terrenos				&Matatu Pequeno				&30f			&98A			\\
Luis José de Almeida								&Candeal				&Candeal				&30v			&99			\\
José Joaquim de Santa Tereza							&Pitangueiras				&Pitangueiras				&31f			&100			\\
Joaquim Antonio Pereira Barreto							&Terrenos (no Matatu Pequeno)		&Matatu Pequeno				&31f			&101			\\
Joaquim Barbosa de Oliveira							&Terrenos (Na Cruz da Redenção)		&Cruz da Redenção			&31v			&102			\\
Joaquim Olavo da Silva Rebelo (Tenente Coronel)					&Terrenos (No Sangradouro)		&Sangradouro				&31v			&103			\\
Custodia Angela Marinha								&Terrenos (no Matatu Pequeno)		&Matatu					&32f			&104			\\
Faustino José de Santana							&Terrenos (no largo de Brotas)		&Largo de Brotas			&32f			&105			\\
\bottomrule
\end{tabular} 
\end{tiny}
\end{minipage}
}
{\fonte{Elaboração do autor, com base em \textbf{BR BAAPB}, fundo Colonial, série Registros de Terra, livro 4675.}}
\end{table}

\begin{table}[ht]
\IBGEtab{
\caption{Registros de terras constantes no Livro Eclesial de Registro de Terras da Freguesia de Brotas (parte 4)}\label{tab:livroterrabrotas4}}
{
\begin{minipage}{\textwidth}
\begin{tiny}
\begin{tabular}{p{4cm}p{4cm}p{4cm}ll}
\toprule
Registrante									&Posse					&Localidade atual			&Folhas			&Registro		\\
\midrule
\midrule
Maria Amelia de Carvalho Martagão						&Terrenos (corredores do Acú)		&Acupe					&32v			&106			\\
Maria Constança Ebé								&Acú					&Acupe					&33f			&107			\\
Antonio Pereira do Rio								&Cruz da Redenção			&Cruz da Redenção			&33f			&108			\\
Anna Rosa Joaquina do Amor Divino							&Terrenos				&--					&33v			&109			\\
Thimothea Maria Lopes								&Terreno (Acupe)			&Acupe					&34f			&110			\\
Elias Lopes de São Jeronimo							&Terrenos (no Acú)			&Acupe					&34f			&111			\\
Caetana Rosa da Conceição							&Açú					&Acupe					&34v			&112			\\
Manuel de Santa Isabel								&Acú					&Acupe					&34v			&113			\\
Rosa Maria do Amor Divino							&Acupe					&Acupe					&35f			&114			\\
Cipriano Freire de Carvalho							&Terrenos (no Matatu Grande)		&Matatu Grande				&35f			&115			\\
Domingos José Garcia								&Terrenos				&Estrada de Brotas			&35v			&116			\\
Joaquim de Costa Pinheiro (Ten. Cel.)						&Ladeira da Boa Vista			&Ladeira da Boa Vista			&35v			&117			\\
João José de Azevedo Lima							&Terras (Ladeira da Boa Vista)		&Ladeira da Boa Vista			&36f			&118			\\
Antonio Joaquim da Silva e Abreu						&Santa Cruz (Fazenda)			&Santa Cruz				&36f			&119			\\
José Alves do Amaral								&Alagôa					&Amaralina				&36v			&119A			\\
José Rodrigues de Figueiredo							&Terras (Sangradouro)			&Sangradouro				&37f			&120			\\
Benjamim Pereira Marinho							&Casa (no Largo de Brotas)		&Largo de Brotas			&37f			&121			\\
Braz Balthazar da Silveira							&Matatu Pequeno (Roça)			&Matatu Pequeno				&37v			&122			\\
Viscondessa do Rio Vermelho, ou seu filho, Barão do Rio Vermelho		&Pituba (Fazenda)			&Pituba					&38f			&124			\\
José Joaquim de Santa Tereza							&Terrenos				&Largo de Brotas			&38f			&125			\\
Viscondessa do Rio Vermelho e herdeiro						&Terrenos (foreiros)			&Boca do Rio				&38v			&126			\\
Francisco Xavier dos Reis (Dr.)							&Acú					&Acupe					&39f			&127			\\
Francisco Antonio do Espirito Santo						&Terrenos (no Matatu)			&Matatu					&39f			&128			\\
Jacinto Muniz Barreto								&Quinta das Brotas (Fazenda)		&Quinta das Beatas			&39v			&129			\\
José Luis da Rocha								&Terrenos				&Estrada de Brotas					&39v			&130			\\
Manuel Inacio de Barros Paim							&Ubarana (Fazenda)			&Pituba e Amaralina			&40f			&131			\\
Manuel José de Santana (Cirurgião Mor)						&Terrenos				&Estrada de Brotas					&40f			&132			\\
Francisco Pires de Carvalho Albuquerque						&Torre (Roça)				&Torre					&40f			&133			\\
Ana Maria Madalena Rejente							&Quinta das Beatas (Fazenda)		&Quinta das Beatas			&40v			&134			\\
Angelo Francisco de Andrade							&Terrenos				&Estrada de Brotas					&40v			&135			\\
Francisco Gomes de Castro Dr.							&Terreno (no alto do Sangradouro)	&Sangradouro				&41f			&136			\\
Maria Joaquina do Espirito Santo						&Terrenos				&Sangradouro					&41f			&137			\\
Joaquim José de Santana Gomes por seus cunhados					&Terrenos				&Estrada do Matatu					&41v			&138			\\
Antonio Ramos de Silva e outro							&Terrenos (no Acú)			&Acupe					&41v			&139			\\
Firmino Pacifico Duarte Gameleira						&Palacete				&Estrada do Matatu					&42f			&140			\\
\bottomrule
\end{tabular} 
\end{tiny}
\end{minipage}
}
{\fonte{Elaboração do autor, com base em \textbf{BR BAAPB}, fundo Colonial, série Registros de Terra, livro 4675.}}
\end{table}


% ---
\chapter{Proprietários rurais de Brotas em 1920}
% ---
\begin{tiny}
\begin{longtable}{cc}
\caption{Nome dos proprietários rurais do distrito de Brotas e localidade de suas terras (1920)}\label{tab:proprurais}\\
\hline Proprietários & Nome do estabelecimento (ou localidade) \\ \hline\hline \endhead
\hline \multicolumn{2}{c}{Continua na próxima página...} \\ \endfoot
\hline \endlastfoot
Joanna B. de Souza  & Brotas \\
Pedro F. H. Pires  & Brotas \\
José Visco  & Brotas \\
António S. Souza  & Brotas \\
Luiz Saraiva  & Brotas \\
Augusto R. de Senna  & Brotas \\
Juventino R. de Almeida  & Brotas \\
Ricardo A. Pereira  & Brotas \\
Antonio A. Cupim & Brotas \\
José A. Silva & Brotas \\
Avelino J. Pereira & Brotas \\
Luiz H. de Souza & Brotas \\
Salustiano R. Pinto & Brotas \\
João de A. Ramos & Brotas \\
Rodrigo A. Araújo & Brotas \\
Estevão G. da Encarnação & Brotas \\
Trifino P. de Souza & Brotas \\
Joaquim F. A. Rosa & Brotas \\
Affonso C. Mello & Brotas \\
João P. de Queiroz & Brotas \\
João P. dos Santos & Brotas \\
Hermenegildo P. Souza & Brotas \\
Silvann P. Ramos & Brotas \\
Jesuino A. Monteiro & Brotas \\
Francisco R. A Prata & Brotas \\
António P. Melchiades & Brotas \\
Raphael Alonso & Brotas \\
Pedro J. Mussitahyba & Brotas \\
João de D. Ribeiro & Brotas \\
Zacharias de Oliveira & Brotas \\
Raphael E. da Purificação & Brotas \\
Gabriel A. dos Santos & Brotas \\
Francisco X. da Silva & Brotas \\
Abel A. Amoedo & Brotas \\
Francelino S. B. Figueiredo & Brotas \\
João B. da Silva & Brotas \\
António F. Simões & Brotas \\
Esperidião B. de Argollo & Brotas \\
Olegário F. Cardoso & Brotas \\
João B. Wanderley & Brotas \\
Fortunato J. da Costa & Brotas \\
Joaquim F. Teixeira & Brotas \\
Segundo Garrido & Brotas \\
Pedro S. Nascimento & Brotas \\
Bernardo Lins & Brotas \\
Coronel Frederico A. Rodrigues da Costa & N. S. do Matatú \\
\hline
\end{longtable}
\end{tiny}

% ---
\chapter{Engenheiros, arquitetos e construtores atuantes em Brotas (1889-1930)}
% ---

\afterpage{
\begin{a3paisagem}
\begin{table}[!htp]
\IBGEtab{
\caption{Engenheiros atuantes em Brotas e suas obras, por grupo de logradouros)}\label{tab:engenheiros}}
{
% \begin{minipage}[c][0.8\textheight][c]{\textwidth}
\begin{tiny}
\begin{tabular}{lllllllllllll}
\toprule
Nome	&Antigo 1º Distrito	&Boa Vista / Engenho Velho	&Estrada de Brotas	&Estrada 2 de Julho	&Mariquita	&Matatu	&Acupe	&Campinas	&Alagoa-Pituba	&Armações / Várzea	&TOTAL	&\%\\
\midrule
\midrule
A. Carneiro da Rocha	&2	&1	&0	&0	&0	&1	&0	&0	&0	&0	&4	&0,66\\
A. J. de Souza Carneiro	&6	&6	&0	&0	&0	&1	&3	&0	&1	&0	&17	&2,81\\
Alberto Silva	&0	&0	&0	&2	&0	&0	&0	&0	&0	&0	&2	&0,33\\
Alfredo Vieira de Almeida	&4	&0	&0	&0	&0	&1	&0	&0	&0	&0	&5	&0,83\\
Allioni \& Cia.	&1	&0	&0	&0	&0	&0	&0	&0	&0	&0	&1	&0,17\\
André Saffrey	&0	&0	&0	&0	&0	&0	&0	&0	&2	&0	&2	&0,33\\
Antonio Augusto Machado	&2	&0	&0	&0	&0	&0	&0	&0	&0	&0	&2	&0,33\\
Antonio dos Santos	&0	&2	&0	&0	&0	&0	&0	&0	&0	&0	&2	&0,33\\
Antonio Ferrão Marques	&0	&0	&0	&0	&0	&1	&0	&0	&0	&0	&1	&0,17\\
Antonio Gonçalves	&0	&1	&0	&0	&0	&0	&0	&0	&0	&0	&1	&0,17\\
Antonio Leite da Luz	&6	&0	&3	&1	&2	&3	&1	&0	&0	&0	&16	&2,64\\
Antonio Lopes Rodrigues	&0	&2	&0	&0	&1	&0	&0	&0	&0	&0	&3	&0,50\\
Antonio Valentim Ferreira	&0	&1	&0	&0	&0	&0	&0	&0	&1	&0	&2	&0,33\\
Archimedes Marques	&16	&27	&1	&3	&2	&8	&2	&0	&9	&0	&68	&11,24\\
Arthur Santos	&14	&35	&6	&3	&4	&5	&0	&0	&6	&0	&73	&12,07\\
Barroso de Souza	&2	&1	&1	&0	&0	&0	&0	&0	&0	&0	&4	&0,66\\
Biaggio Bianco	&1	&0	&0	&0	&0	&0	&0	&0	&0	&0	&1	&0,17\\
Carlos Faria	&0	&0	&0	&0	&0	&0	&0	&0	&3	&0	&3	&0,50\\
Carlos Peixoto	&8	&4	&0	&1	&0	&1	&0	&0	&0	&0	&14	&2,31\\
Carlos Souza	&1	&7	&0	&0	&0	&2	&4	&0	&1	&0	&15	&2,48\\
Custódio Bandeira	&10	&21	&10	&5	&2	&1	&3	&0	&11	&0	&63	&10,41\\
Durval Fernandes	&0	&1	&0	&0	&0	&1	&0	&0	&0	&0	&2	&0,33\\
Durval Lima	&0	&1	&0	&0	&0	&0	&0	&0	&0	&0	&1	&0,17\\
Durval Neves da Rocha	&2	&0	&0	&0	&0	&1	&0	&0	&0	&0	&3	&0,50\\
Eduardo dos Santos Corrêa	&0	&0	&0	&0	&0	&0	&0	&0	&1	&0	&1	&0,17\\
Ernestino dos Santos Marques	&1	&1	&0	&1	&0	&0	&0	&0	&0	&0	&3	&0,50\\
Esmeraldo Coelho	&0	&0	&0	&0	&0	&1	&0	&0	&0	&0	&1	&0,17\\
Eurico da Costa Coutinho	&3	&1	&0	&0	&0	&0	&0	&0	&1	&0	&5	&0,83\\
F. Sampaio	&1	&0	&0	&0	&0	&0	&0	&0	&0	&0	&1	&0,17\\
Fillippe Silva	&0	&0	&0	&0	&0	&1	&1	&0	&0	&0	&2	&0,33\\
Francisco A. W. Silva	&0	&0	&0	&1	&0	&0	&0	&0	&0	&0	&1	&0,17\\
Francisco Martins	&0	&2	&0	&0	&0	&2	&0	&0	&1	&0	&5	&0,83\\
Francisco Theodoro Pereira das Neves	&0	&0	&0	&0	&0	&0	&0	&0	&1	&0	&1	&0,17\\
Frederico Saraiva	&0	&1	&1	&0	&0	&0	&0	&0	&1	&0	&3	&0,50\\
Frederico Theodoro Sampaio	&0	&0	&0	&0	&1	&0	&0	&0	&0	&0	&1	&0,17\\
Gustavo Pereira Santos	&2	&0	&0	&0	&0	&0	&0	&0	&0	&0	&2	&0,33\\
J. B. Vasconcellos	&0	&1	&0	&0	&0	&0	&0	&0	&0	&0	&1	&0,17\\
J. Barroso	&4	&3	&3	&2	&4	&4	&0	&0	&2	&0	&22	&3,64\\
J. Castro	&0	&0	&0	&1	&0	&0	&0	&0	&0	&0	&1	&0,17\\
J. Cyrillo de Souza	&0	&0	&0	&0	&0	&1	&0	&0	&0	&0	&1	&0,17\\
Jayme Bastos	&0	&2	&0	&0	&0	&0	&0	&0	&0	&0	&2	&0,33\\
Jayme Cerqueira Lima	&1	&1	&1	&0	&0	&1	&0	&0	&1	&0	&5	&0,83\\
M. Martins	&0	&0	&0	&1	&0	&0	&0	&0	&0	&0	&1	&0,17\\
João dos Santos Tuvo	&0	&0	&1	&0	&0	&0	&0	&0	&0	&0	&1	&0,17\\
João Pimenta Bastos Filho	&1	&0	&1	&1	&0	&0	&0	&0	&1	&0	&4	&0,66\\
Joaquim de Oliveira Júnior	&0	&1	&1	&0	&0	&1	&0	&0	&0	&0	&3	&0,50\\
Joaquim José Ribeiro d?Oliveira	&0	&0	&0	&1	&0	&0	&0	&0	&0	&0	&1	&0,17\\
José Allioni	&0	&0	&0	&2	&0	&0	&0	&0	&0	&0	&2	&0,33\\
José Celestino dos Santos	&2	&1	&0	&0	&1	&2	&0	&0	&1	&0	&7	&1,16\\
José Portella Passos	&1	&0	&0	&0	&2	&1	&0	&0	&1	&0	&5	&0,83\\
Justo J. David	&1	&0	&0	&0	&0	&0	&0	&0	&0	&0	&1	&0,17\\
Júlio Viveiros Brandão	&0	&0	&0	&0	&1	&0	&0	&0	&0	&0	&1	&0,17\\
Lamartine Portella Passos	&4	&3	&0	&0	&1	&0	&0	&0	&0	&0	&8	&1,32\\
Lopes Lima	&0	&1	&0	&0	&0	&1	&0	&0	&0	&0	&2	&0,33\\
Luiz Affonso de Sá	&0	&1	&0	&0	&0	&0	&0	&0	&1	&0	&2	&0,33\\
Luiz de Moura Bastos	&1	&2	&0	&2	&0	&1	&0	&0	&0	&0	&6	&0,99\\
Lycerio Alfredo Schreiner	&4	&5	&0	&1	&0	&5	&5	&0	&1	&0	&21	&3,47\\
M. Martins	&0	&1	&1	&0	&0	&0	&0	&0	&0	&0	&2	&0,33\\
Manoel R. F. Muniz	&3	&4	&1	&0	&1	&3	&0	&0	&0	&0	&12	&1,98\\
Manuel Querino	&0	&0	&0	&0	&0	&1	&0	&0	&0	&0	&1	&0,17\\
Mario de Souza Dias	&1	&0	&0	&0	&0	&3	&0	&0	&1	&0	&5	&0,83\\
Nogueira Passos	&0	&0	&0	&0	&0	&1	&0	&0	&0	&0	&1	&0,17\\
Oswaldo Gonçalves Martins	&3	&0	&1	&0	&0	&1	&6	&0	&0	&0	&11	&1,82\\
Pedro Jayme David	&23	&18	&4	&5	&1	&14	&4	&0	&10	&0	&79	&13,06\\
Quirino da Costa Coutinho	&0	&0	&0	&0	&1	&0	&0	&0	&0	&0	&1	&0,17\\
Rogério Baptista	&1	&0	&0	&0	&0	&0	&0	&0	&0	&0	&1	&0,17\\
Rosalvo Celestino dos Santos	&7	&12	&4	&2	&0	&5	&0	&0	&4	&0	&34	&5,62\\
Rossi Baptista	&1	&0	&0	&0	&0	&0	&0	&0	&0	&0	&1	&0,17\\
S. Lellis	&0	&0	&0	&0	&0	&0	&1	&0	&0	&0	&1	&0,17\\
Satyro Brandão	&0	&0	&0	&0	&0	&0	&1	&0	&0	&0	&1	&0,17\\
Urbano Rossi	&0	&2	&0	&0	&0	&0	&0	&0	&0	&0	&2	&0,33\\
Victorino d'Almeida	&2	&0	&0	&1	&0	&1	&1	&0	&0	&0	&5	&0,83\\
Victorio Joaquim de Meirelles	&2	&7	&3	&3	&1	&1	&1	&0	&2	&0	&20	&3,31\\
\midrule
TOTAL	&144	&180	&43	&39	&25	&77	&33	&0	&64	&0	&605	&100,00\\
\bottomrule
\end{tabular} 
\end{tiny}
% \end{minipage}
}
{\fonte{Elaboração do autor, com base em \textbf{BR BAAHMS}, Fundo ``Intendência e Prefeitura'', Série ``Processos de Licenciamento de Reforma e Ampliação de Edificações'', Subsérie ``Requerimentos e Plantas -- Brotas'', todas as caixas e processos.}}
\end{table}

\end{a3paisagem}
}

% ---
\chapter{Nomes antigos e modernos dos logradouros de Brotas}
% ---
\begin{table}[!htp]
\IBGEtab{\caption{Logradouros de Brotas em três tempos (parte 1)}\label{tab:logradouros01}}{
\begin{minipage}{0.9\textwidth}
\begin{tiny}
\begin{longtabu} to \textheight {m{4.5cm} m{5cm} m{4.5cm}}
\toprule 
Nome em 1935 				& Nome antigo (1889-1930) 		& Localização atual (2015) \\ 
\midrule
Acupe (acupe)				& Rua do Acupe 				& Ladeira do Acupe \\ 
Affonso de Taunay (rua) 		& Beco do General 			& Rua Affonso de Taunay \\
Agripino Dórea (rua) 			& Rua das Pitangueiras 			& Rua das Pitangueiras \\
Almirante Alves Câmara (rua) 		& Estrada do Engenho Velho de Brotas 	& Rua Almirante Alves Câmara \\
Altamira (vila) 			& Estr. Dois de Julho, 395 		& não localizado \\
Amaral Muniz (travessa) 		& Travessa das Pitangueiras 		& não localizado \\
Amaralina (lagoa) 			& idem 					& idem \\
Amaralina (avenida) 			& idem 					& idem \\
América (vila) 				& idem 					& idem \\
Arlindo Fragoso (rua) 			& Rua do Socorro 			& Rua Arlindo Fragoso \\
Armação (povoado) 			& Armação do Saraiva 			& Praias da Armação e da Boca do Rio\\
Asilo (ladeira) 			& Ladeira de Nanã 			& idem \\
Baixão (ladeira) 			& idem 					& idem \\
Bandeirantes (rua) 			& Rua do Fabrício 			& Ladeira dos Bandeirantes \\
Barro Vermelho 				& idem 					& Ao lado do Monte do Conselho \\
Barros Falcão (rua) 			& Estrada do Matatu Pequeno 		& Rua Barros Falcão \\
Basílio de Magalhães (travessa) 	& Travessa da Mariquita 		& -- \\
Belém de Baixo (alto) 			& Rua Monte Belém de Baixo 		& -- \\
Boa Esperança (avenida)			& Alto da Bola de Ouro 			& -- \\
Boca do Rio 				& idem 					& idem \\
Bonocô (baixa) 				& idem 					& idem \\
Brigadeiro Faria Lima (rua)		& Rua da Fonte do Boi e largo dos Dendezeiros		& Rua da Fonte do Boi e praça Brigadeiro Faria Lima
Brígida do Vale (rua) 			& Rua da Capelinha 			& Rua Brígida do Vale \\
Brongo (rua) 				& Avenida Sanches 			& idem \\
Brotas (largo) 				& Largo de Brotas 			& Avenida D. João VI \\
Cabuçu (estrada) 			& idem 					& trecho em aclive da rua Waldemar Falcão iniciado na rua Lucaia \\
Campinas (rua) 				& idem 					& idem \\
Campinas de Brotas (estrada) 		& idem 					& Rua Teixeira Barros \\
Candeal (ladeira) 			& idem 					& Rua Monsenhor Antonio Rosa \\
Candeal Grande (roça) 			& idem 					& idem \\
Capelinha (praça) 			& Largo da Capelinha 			& Praça da Capelinha \\
Capelinha (ladeira) 			& Ladeira da Capelinha 			& Rua Brígida do Vale \\
Capelinha (alto) 			& Alto da Capelinha 			& Rua Brígida do Vale \\
Castro Neves (rua) 			& idem 					& idem \\
Castro Neves (travessa) 		& idem 					& não identificado \\
Chega Negro (estrada) 			& Estrada do Chega Negro 		& Avenida Oceânica \\
Chile da Capelinha (rua) 		& Rua Chile da Capelinha 		& idem \\
Clião Arouca (rua) 			& Ladeira do Acupe 			& idem \\
Comendador Pereira da Silva (rua) 	& Avenida Saraiva 			& Rua Comendador Pereira da Silva \\
Conceição Foeppel (rua) 		& Rua Alegria do Castro Neves 		& idem \\
Corrimão (ladeira) 			& idem 					& idem \\
Cosme de Farias (rua) 			& Boca da Mata da Quinta das Beatas 	& Rua Cosme de Farias \\
Cruz da Redenção (largo) 		& Largo da Cruz da Redenção 		& Largo da Cruz da Redenção \\
Cruz das Almas (rua) 			& Avenida Redemptor 			& Rua Waldemar Falcão \\
D. João VI (avenida) 			& Estrada de Brotas / Rua D. Pedro II 	& Avenida D. João VI \\
Djalma Dutra (rua) 			& Ruas do Sangradouro, das Sete Portas e da Fonte Nova & Rua Djalma Dutra \\
Esperança (vila) 			& Estrada Dois de Julho, 491 		& não localizado \\
Fabrício (alto) 			& idem 					& idem \\
Fazenda Santa Cruz (estrada) 		& Estrada da Fazenda Santa Cruz 	& não localizado \\
Formoso (alto) 				& idem 					& idem \\
\bottomrule
\end{longtabu}
\end{tiny}
\end{minipage}
}
{\fonte{\citeonline{souza_guia_1935}, \citeonline{municipal_atlas_1955}, Open Street Map (\url{https://www.openstreetmap.org}) e Google Maps (\url{https://www.google.com/maps}).}}
\end{table}


\begin{table}[!htp]
\IBGEtab{\caption{Logradouros de Brotas em três tempos (parte 2)}\label{tab:logradouros02}}{
\begin{minipage}{0.9\textwidth}
\begin{tiny}
\begin{longtabu} to \textheight {m{4.5cm} m{5cm} m{4.5cm}}
\toprule 
Nome em 1935 				& Nome antigo (1889-1930) 		& Localização atual (2015) \\ 
\midrule
Francisco Vicente Vianna (praça)	& Largo da Fonte Nova 			& idem \\
Frederico Costa (rua) 			& idem 					& idem \\
Frei Apolônio de Todi 			& Rua do Meio da Mariquita 		& Rua do Meio \\
Gomes Brandão (rua) 			& Estrada das Máquinas 			& não localizado \\
Jangadeiros (praça) 			& Largo da Lagoa de Amaralina 		& Praça dos Ex-Combatentes \\
Joaquim dos Couros (ladeira) 		& idem 					& idem \\
José Ramos (rua) 			& inexistente 				& idem \\
José Visco (rua) 			& Ladeira do Pepino, rua Uruguaiana 	& Ladeira do Pepino \\
Luiz Anselmo (rua) 			& Estrada do Matatu Grande 		& Rua Luiz Anselmo \\
Machado (curva do) 			& inexistente 				& trecho inicial da ladeira dos Bandeirantes \\
Machado de Assis (rua) 			& Rua do Brongo, avenida Sanches 	& Rua Machado de Assis \\
Mangueira (rua) 			& Trecho da estrada do Beiju 		& Rua Teixeira Barros \\
Manoel Dias da Silva (avenida) 		& Rua Ubarana 				& Avenida Manoel Dias da Silva \\
Manoel Faustino (rua) 			& Rua de Baixo 				& Rua Manoel Faustino \\
Maria Felipa (rua) 			& inexistente 				& Rua Maria Felipa \\
Marquês de Abrantes (rua) 		& Rua da Boa Vista 			& idem \\
Marquês de Monte Santo (rua) 		& Rua das Pedrinhas 			& Rua Marquês de Monte Santo \\
Mata Escura (rua) 			& Trecho da Estrada 2 de Julho 		& Avenida Vasco da Gama \\
Matatu (rua) 				& Estrada da Pólvora 			& Rua Luiz Anselmo \\
Matatu Grande (rua) 			& Estrada do Matatu Grande 		& Rua Luiz Anselmo \\
Matatu Pequeno (rua) 			& Rua Barros Falcão 			& idem \\
Meio (rua) 				& inexistente 				& Rua Visconde de Itaboraí \\
Monte Belém de Baixo (rua) 		& idem 					& idem \\
Monte Belém de Cima (rua) 		& idem 					& idem \\
Monte Belo (rua) 			& inexistente 				& Rua Monte Belo de Baixo \\
Monte Conselho 				& idem 					& Rua Monte Conselho \\
Mulambo (rua) 				& Estrada das Ubaranas 			& Ladeira da Cruz da Redenção \\
Nanã (ladeira) 				& Ladeira do Asilo 			& Ladeira de Nanã \\
Nordeste (alto) 			& Alto do Nordeste 			& Nordeste de Amaralina \\
Nordeste (rua) 				& inexistente 				& Ladeira do Nordeste \\
Norte (rua) 				& inexistente 				& Rua do Norte \\
Odilon Santos (rua) 			& Rua Direita da Mariquita 		& Rua Odilon Santos \\
Oswaldo Cruz (rua) 			& Rua Dendezeiros da Mariquita 		& Rua Osvaldo Cruz \\
Olavo Bilac (rua) 			& inexistente 				& não localizado \\
Operária São Salvador (vila) 		& idem 					& idem \\
Padre Daniel Lisboa (rua) 		& idem 					& idem \\
Padre Eloy (rua) 			& Ladeira Joaquim dos Couros 		& Ladeira do Padre Eloy \\
Padre Luiz Filgueira (rua) 		& inexistente 				& Rua Padre Luiz Filgueira \\
Paraguaçu (avenida) 			& Baixa da Égua 			& Avenida Paraguaçu \\
Paz (baixa da) 				& Quinta das Beatas 			& Rua Baixa da Paz \\
Pedreiras (ladeira) 			& inexistente 				& Avenida Maria Dusá \\
Pirangy (travessa) 			& inexistente 				& não localizado \\
Pituba (estrada) 			& Estrada das Ubaranas 			& Trecho das avenidas Manoel Dias da Silva e Oceânica até o Jardim dos Namorados.\\
Primeiro de Maio (largo) 		& Largo das Sete Portas 		& idem \\
Professor Manoel Querino 		& Largo das Pitangueiras / do Paranhos 	& Largo do Paranhos \\
Professor Mussurunga 			& Travessa do Sangradouro 		& Rua do Sangradouro \\
Reis Príncipe (rua) 			& Ladeira do Sapoty 			& Rua Reis Príncipe \\
Saldanha (avenida) 			& Avenida Saldanha 			& não localizado \\
Saldanha (pequeno) 			& inexistente 				& não localizado \\
Santa Rita (avenida) 			& inexistente 				& Rua Santa Rita \\
Santa Therezinha de Jesus (vila)	& inexistente 				& não localizado \\
Santo Agostinho (rua)			& Rua do Bigode 			& Rua Santo Agostinho \\
Santo Antônio (rua) 			& inexistente 				& Rua Santo Antônio de Brotas \\
Santos (vila) 				& inexistente 				& Vila Santos \\
São João (rua) 				& inexistente 				& Trecho da rua Teixeira Barros \\
São Jorge (vila) 			& inexistente 				& Beco transversal à ladeira do Pepino \\
Trovador (rua) 				& Rua do Asilo 				& Rua do Trovador \\
Tupis (rua) 				& Ladeira do Fabrício 			& Rua dos Tupis \\
União (rua) 				& idem 					& idem \\
Vargem de Santo Antônio 		& Estrada da Várzea de Santo Antônio 	& Rua Várzea de Santo Antônio \\
Vasco da Gama (avenida) 		& Estrada Dois de Julho 		& Avenida Vasco da Gama \\
Vasco R. Caldas (travessa) 		& inexistente 				& Travessa Vasco Ribeiro Caldas, \\
Vila América 				& idem 					& Rua Vila América \\
Vila Rio Branco 			& inexistente 				& Trecho da avenida D. João VI \\
Vila Ziza 				& inexistente 				& não localizado \\
\bottomrule
\end{longtabu}
\end{tiny}
\end{minipage}
}
{\fonte{\citeonline{souza_guia_1935}, \citeonline{municipal_atlas_1955}, Open Street Map (\url{https://www.openstreetmap.org}) e Google Maps (\url{https://www.google.com/maps}).}}
\end{table}

% ---
\chapter{Mapas}\label{anexo-mapas}
% ---

\input{8-anexos/texto}

% ---
\chapter{Plantas selecionadas de obras no antigo 1º Distrito}
% ---

% ---
% 01. Cinco casas de Manoel Amoedo y Amoedo na rua das Pitangueiras
% ---

\afterpage{
\begin{a3paisagem}
\begin{figure}[!h]
\centering
\caption{Projecto de construção para cinco casas na Rua 25 de Março, freguezia de Brotas (1892). }
\includegraphics[width=1\textwidth]{8-anexos/plantas/01-1distrito/04-agripino-dorea/agripinodorea-manoelamoedoyamoedo-1892.jpg}{\footnotesize \par \textbf{Fonte:} \textbf{BR BAAHMS}, Fundo ``Intendência'', Série ``Processos de Licenciamento de Reforma e Ampliação de Edificações'', Subsérie ``Requerimentos e Plantas -- Brotas'', caixa 04. \par Projeto de José Celestino dos Santos. O mais antigo entre todos os projetos pesquisados é exatamente o de construção de pequenas casas para alugar na rua das Pitangueiras, de propriedade do comerciante espanhol Manoel Amoedo y Amoedo. }
\label{fig:manoelamoedoyamoedo}
\end{figure}
\end{a3paisagem}
}

% ---
% 02. Seis casas de Antonio Ribeiro da Cunha na rua da Alegria do Castro Neves
% ---

\afterpage{
\begin{a3paisagem}
\begin{figure}[!h]
\centering
\caption{Projecto para a construção de seis pequenas casas que o Snr. Antonio Ribeiro da Cunha pretende fazer na rua da Alegria districto de Brotas (1897). }
\includegraphics[width=1\textwidth]{8-anexos/plantas/01-1distrito/10-alegria/alegriadocastroneves-antonioribeirodacunha-seiscasas.jpg}{\footnotesize \par \textbf{Fonte:} \textbf{BR BAAHMS}, Fundo ``Intendência'', Série ``Processos de Licenciamento de Reforma e Ampliação de Edificações'', Subsérie ``Requerimentos e Plantas -- Brotas'', caixa 10. \par Projeto de autor desconhecido. Casas geminadas bem à moda das ``casas para operários'' que lhe sucederam: pequenas, desadornadas, e sobretudo baratas. }
\label{fig:antonioribeirodacunha}
\end{figure}
\end{a3paisagem}
}

% ---
% 03. Casa rural de Francisco Ventura na rua do Sangradouro
% ---

\afterpage{
\begin{a3paisagem}
\begin{figure}[!h]
\centering
\caption{Projecto de casa para o sr. Francisco Ventura (1901). }
\includegraphics[height=0.9\textheight]{8-anexos/plantas/01-1distrito/15-sangradouro/sangradouro-franciscoventura-casa01.jpg}{\footnotesize \par \textbf{Fonte:} \textbf{BR BAAHMS}, Fundo ``Intendência'', Série ``Processos de Licenciamento de Reforma e Ampliação de Edificações'', Subsérie ``Requerimentos e Plantas -- Brotas'', caixa 15. \par Projeto de Manoel R. F. Muniz. Hoje não mais existente, a construção é uma das típicas casas rurais do Sangradouro, um dos limites da urbanização no antigo 1º Distrito na alvorada da Primeira República. }
\label{fig:franciscoventura}
\end{figure}
\end{a3paisagem}
}

% ---
% 04. Casa mista de negócios do capitão Valentin Duran Suarez na rua das Pitangueiras
% ---

\afterpage{
\begin{a3paisagem}
\begin{figure}[!h]
\centering
\caption{Projecto para a construção de uma casa na rua do Dr. Agrippino Dorea nº 22, districto de Brotas, pertencente ao Ilmo. Snr. Capitão Valentin Duran Suarez (1908). }
\includegraphics[height=0.9\textheight]{8-anexos/plantas/01-1distrito/04-agripino-dorea/agripinodorea-valentinduransuarez.jpg}{\footnotesize \par \textbf{Fonte:} \textbf{BR BAAHMS}, Fundo ``Intendência'', Série ``Processos de Licenciamento de Reforma e Ampliação de Edificações'', Subsérie ``Requerimentos e Plantas -- Brotas'', caixa 040. \par Projeto de Antonio Leite da Luz. Este projeto destacou-se entre todos por ser o único no distrito inteiro a combinar uso residencial e comercial no mesmo imóvel: o térreo abre-se com uma ``Area para negocio'' e  ``Bilhar'', separados por um corredor das duas salas, dispensa, cozinha e banheiro existentes no térrero, havendo ainda mais duas salas, cinco quartos, uma alcova e uma capela no ``pavimento nobre''. }
\label{fig:valentinduransuarez}
\end{figure}
\end{a3paisagem}
}

% ---
% 05. Casa rural de Ricardo da Silva Teixeira Machado na rua do Sangradouro
% ---

\afterpage{
\begin{a3paisagem}
\begin{figure}[!h]
\centering
\caption{Projecto para construcção de um andar da casa na roça ao Sangradouro, dist. de Brotas, pertencente ao Snr. Ricardo da Silva Teixeira Machado (1915). }
\includegraphics[height=0.9\textheight]{8-anexos/plantas/01-1distrito/15-sangradouro/sangradouro-ricardodasilvateixeiramachado-constroiandarnacasas.jpg}{\footnotesize \par \textbf{Fonte:} \textbf{BR BAAHMS}, Fundo ``Intendência'', Série ``Processos de Licenciamento de Reforma e Ampliação de Edificações'', Subsérie ``Requerimentos e Plantas -- Brotas'', caixa 15. \par Projeto de Arthur Santos. Ainda era possível falar de ``roças'' no Sangradouro em 1915, evidenciando até onde avançara a urbanização no antigo 1º Distrito. }
\label{fig:ricardodasilvateixeiramachado}
\end{figure}
\end{a3paisagem}
}

% ---
% 06. Vinte casas de José Antonio Ramos na rua do Castro Neves
% ---

\afterpage{
\begin{a3paisagem}
\begin{figure}[!h]
\centering
\caption{Projecto para a construção de 20 casas, para operarios, ao Castro Neves, districto de Brotas, que pretende fazer o snr. José Antonio Ramos (1919). }
\includegraphics[height=0.9\textheight]{8-anexos/plantas/01-1distrito/16-castroneves/joseantonioramos-20casas.jpg}{\footnotesize \par \textbf{Fonte:} \textbf{BR BAAHMS}, Fundo ``Intendência'', Série ``Processos de Licenciamento de Reforma e Ampliação de Edificações'', Subsérie ``Requerimentos e Plantas -- Brotas'', caixa 16. \par Projeto de Arthur Santos. Exceção à regra entre as ``casas para operários'' no que diz respeito ao apuro estético da fachada, estas casas, hoje inexistentes, foram verdadeiros cubículos, a julgar pela sua planta baixa. }
\label{fig:joseantonioramos-20casas}
\end{figure}
\end{a3paisagem}
}

% ---
% 07. Casa de José Veríssimo Alves na rua do Sangradouro
% ---

\afterpage{
\begin{a3paisagem}
\begin{figure}[!h]
\centering
\caption{Representação da casa de nº 24 sita a rua do Sangradouro com projecto de mais um pavimento (1920). }
\includegraphics[width=1\textwidth]{8-anexos/plantas/01-1distrito/15-sangradouro/sangradouro-joseverissimoalves-1920.jpg}{\footnotesize \par  \textbf{Fonte:} \textbf{BR BAAHMS}, Fundo ``Intendência'', Série ``Processos de Licenciamento de Reforma e Ampliação de Edificações'', Subsérie ``Requerimentos e Plantas -- Brotas'', caixa 15. \par Projeto de Custódio Bandeira (1920). A casa de José Veríssimo Alves é uma das poucas a permanecer de pé neste logradouro. }
\label{fig:joseverissimo1920}
\end{figure}
\end{a3paisagem}
}

% ---
% 08. Casa de José Veríssimo Alves ainda de pé
% ---

\afterpage{
\begin{figure}[!h]
\centering
\caption{Antiga casa de José Veríssimo Alves, na rua do Sangradouro (2017). }
\includegraphics[width=1\textwidth]{8-anexos/plantas/01-1distrito/15-sangradouro/sangradouro-joseverissimoalves-2019n169.jpg}{\footnotesize \par   \textbf{Fonte:} Google Maps. }
\label{fig:joseverissimo2017}
\end{figure}
}

% ---
% 09. Casa de Domingos Gonçalves Cavalheiros na rua do Castro Neves, ainda de pé
% ---

\afterpage{
\begin{a3paisagem}
\begin{figure}[!htp]
	\caption{Casa de Domingos Gonçalves Cavalheiro, na rua do Castro Neves, em dois tempos}
	\centering
		\begin{subfigure}[t]{0.4\textwidth}
			\includegraphics[width=1\textwidth]{8-anexos/plantas/01-1distrito/16-castroneves/castroneves240-domingosgoncalvescavalheiro-1927.jpg}
			\caption{\footnotesize Projeto de Pedro Jayme David (1927). \textbf{Fonte:} \textbf{BR BAAHMS}, Fundo ``Intendência'', Série ``Processos de Licenciamento de Reforma e Ampliação de Edificações'', Subsérie ``Requerimentos e Plantas -- Brotas'', caixa 16. }
			\label{fig:domingosgoncalvescavalheiro-1927}
		\end{subfigure}
		\
		\begin{subfigure}[t]{0.4\textwidth}
			\includegraphics[width=1\textwidth]{8-anexos/plantas/01-1distrito/16-castroneves/castroneves240-domingosgoncalvescavalheiro-2017.jpg} 
			\caption{\footnotesize  O mesmo imóvel em 2017. \textbf{Fonte:} Google Maps. }
			\label{fig:domingosgoncalvescavalheiro-2017}
		\end{subfigure}
	\label{fig:domingosgoncalvescavalheiro}
\end{figure}
\end{a3paisagem}
}

% ---
% 10. Duas casas de Maria Rosa Vianna Ferraz na rua do Castro Neves
% ---

\afterpage{
\begin{a3paisagem}
\begin{figure}[!h]
\centering
\caption{Projecto para modificação das fachadas dos predios nº 81 e 83 a rua Castro Neves districto de Brotas (1929). }
\includegraphics[height=0.9\textheight]{8-anexos/plantas/01-1distrito/16-castroneves/mariarosaviannaferraz-83e81-1929.jpg}{\footnotesize \par  Projeto de Lycerio Alfredo Schreiner. \textbf{Fonte:} \textbf{BR BAAHMS}, Fundo ``Intendência'', Série ``Processos de Licenciamento de Reforma e Ampliação de Edificações'', Subsérie ``Requerimentos e Plantas -- Brotas'', caixa 16. Uma das duas casas de Maria Rosa Vianna Ferraz permanece de pé. }
\label{fig:mariarosaviannaferraz1929}
\end{figure}
\end{a3paisagem}
}

% ---
% 11. Casa de Maria Rosa Vianna Ferraz ainda de pé
% ---

\afterpage{
\begin{figure}[!h]
\centering
\caption{Antiga casa de Maria Rosa Vianna Ferraz, na rua do Castro Neves (2017). }
\includegraphics[height=0.9\textheight]{8-anexos/plantas/01-1distrito/16-castroneves/mariarosaviannaferraz-83e81-2017.jpg}{\footnotesize \par \textbf{Fonte:} Google Maps. }
\label{fig:mariarosaviannaferraz2017}
\end{figure}
}

% ---
% 12. Casa de Edmundo Guimarães na ladeira dos Galés
% ---

\afterpage{
\begin{a3paisagem}
\begin{figure}[!h]
\centering
\caption{Projecto para construção do 1ª andar do predio á Ladeira dos Galés nº 16. Propriedade do Illmº Snr. Edmundo Guimarães. Districto de Brotas (1930) }
\includegraphics[width=1\textwidth]{8-anexos/plantas/01-1distrito/12-gales/ladeiradosgales-edmundoguimaraes-casa.jpg}{\footnotesize \par  Projeto de Rossi Baptista (1930). \textbf{Fonte:} \textbf{BR BAAHMS}, Fundo ``Intendência'', Série ``Processos de Licenciamento de Reforma e Ampliação de Edificações'', Subsérie ``Requerimentos e Plantas -- Brotas'', caixa 12. Esta casa de Edmundo Guimarães permanece de pé. }
\label{fig:edmundoguimaraes1930}
\end{figure}
\end{a3paisagem}
}

% ---
% 13. Casa de Edmundo Guimarães ainda de pé
% ---

\afterpage{
\begin{figure}[!h]
\centering
\caption{Antiga casa de Edmundo Guimarães na ladeira dos Galés (2017)}
\includegraphics[width=1\textwidth]{8-anexos/plantas/01-1distrito/12-gales/ladeiradosgales-edmundoguimaraes-casa2017.jpg}{\footnotesize \par \textbf{Fonte:} Google Maps. }
\label{fig:edmundoguimaraes2017}
\end{figure}
}

% ---
\chapter{Plantas selecionadas de obras na Boa Vista, no Engenho Velho de Brotas e vizinhança}
% ---

\input{8-anexos/texto/02-boavista}

% ---
\chapter{Plantas selecionadas de obras na Estrada de Brotas e vizinhança}
% ---

\input{8-anexos/texto/03-estbrotas}

% ---
\chapter{Plantas selecionadas de obras na Estrada Dois de Julho e vizinhança}
% ---

\input{8-anexos/texto/04-e2j}

% ---
\chapter{Plantas selecionadas de obras na Mariquita e vizinhança}
% ---

% ---
% 01. Casa de Theodomiro José Veríssimo, na rua do Céu
% ---

\afterpage{
\begin{a3paisagem}
\begin{figure}[!h]
\centering
\caption{Planta para construcção de uma casa que Bernardo Martins da Silva pretende construir no terreno baldio sito a rua Direita da Mariquita ao Rio Vermelho (1900)}
\includegraphics[width=1\textwidth]{8-anexos/plantas/05-mariquita/06-mariquita/1900-bernardomartinsdasilva.jpg}{\footnotesize \par \textbf{Fonte:} \textbf{BR BAAHMS}, Fundo ``Intendência'', Série ``Processos de Licenciamento de Reforma e Ampliação de Edificações'', Subsérie ``Requerimentos e Plantas -- Brotas'', caixa 06. \par Projeto de desenhista e arquiteto/engenheiro desconhecidos. Casas de veraneio como esta foram comuns na rua Direita da Mariquita. }
\label{fig:1900-bernardomartinsdasilva}
\end{figure}
\end{a3paisagem}
}

% ---
% 02. Casa de Theodomiro José Veríssimo, na rua do Céu
% ---

\afterpage{
\begin{figure}[!h]
\centering
\caption{Projecto para a construcção de uma casa de taipa que Theodomiro José Verissimo pretende fazer na rua do Céo no Rio Vermelho districto de Brotas (1906)}
\includegraphics[width=1\textwidth]{8-anexos/plantas/05-mariquita/03-ruadoceu/theodomirojoseverissimo-1906.jpg}{\footnotesize \par \textbf{Fonte:} \textbf{BR BAAHMS}, Fundo ``Intendência'', Série ``Processos de Licenciamento de Reforma e Ampliação de Edificações'', Subsérie ``Requerimentos e Plantas -- Brotas'', caixa 03. \par Projeto de Custódio Bandeira. Casas de taipa ainda eram toleradas pela Diretoria de Obras na primeira década do século XIX. Mesmo o desenhista ``relaxou'' e não apresentou as medidas de todos os cômodos. }
\label{fig:theodomirojoseverissimo-1906}
\end{figure}
}

% ---
% 03. Casa de Prediliano Pitta, na rua do Lucaia
% ---

\afterpage{
\begin{figure}[!h]
\centering
\caption{Projecto para a construcção de uma casa em terreno baldio, sito à Lucaia freguesia de Brotas, pertencente ao Snr. Prediliano Pereira Pitta (1911)}
\includegraphics[width=1\textwidth]{8-anexos/plantas/05-mariquita/06-lucaia/predilianopitta-1911.jpg}{\footnotesize \par \textbf{Fonte:} \textbf{BR BAAHMS}, Fundo ``Intendência'', Série ``Processos de Licenciamento de Reforma e Ampliação de Edificações'', Subsérie ``Requerimentos e Plantas -- Brotas'', caixa 06. \par Não há assinatura de desenhista ou engenheiro/arquiteto, mas a caligrafia é de Rosalvo Celestino dos Santos. }
\label{fig:predilianopitta-1911}
\end{figure}
}


% ---
% 04. Casa de Luiz Lucas da Costa, na rua da Fonte do Boi
% ---

\afterpage{
\begin{a3paisagem}
\begin{figure}[!h]
\centering
\caption{Projecto para construcção de uma cosinha, varanda, banheiro, latrina e platibanda, em um predio sito a Fonte do Boi no Rio Vermelho districto de Brotas, pertencente ao Sr. Luiz Lucas da Costa (1912)}
\includegraphics[height=0.9\textheight]{8-anexos/plantas/05-mariquita/06-fontedoboi/luizlucasdacosta-1912.jpg}{\footnotesize \par \textbf{Fonte:} \textbf{BR BAAHMS}, Fundo ``Intendência'', Série ``Processos de Licenciamento de Reforma e Ampliação de Edificações'', Subsérie ``Requerimentos e Plantas -- Brotas'', caixa 06. \par Na Mariquita os pedidos de reforma predominaram sobre as construções. Este é um dos poucos projetos em que a fachada original foi desenhada em separado, permitindo perceber as alterações. A casa aliás é pequena: o maior dos cômodos tem 9,79$m^{2}$ e o menor, 6,97$m^{2}$. }
\label{fig:luizlucasdacosta-1912}
\end{figure}
\end{a3paisagem}
}

% ---
% 05. Casa de Manoel Affonso Vianna, na Fonte do Boi
% ---

\afterpage{
\begin{figure}[!h]
\centering
\caption{Projecto de construcção --- propriedade do sr. Manoel Affonso Vianna --- Fonte do Boi (Rio Vermelho) (1913)}
\includegraphics[width=1\textwidth]{8-anexos/plantas/05-mariquita/06-fontedoboi/manoelvianna-1915.jpg}{\footnotesize \par \textbf{Fonte:} \textbf{BR BAAHMS}, Fundo ``Intendência'', Série ``Processos de Licenciamento de Reforma e Ampliação de Edificações'', Subsérie ``Requerimentos e Plantas -- Brotas'', caixa 06. \par Projeto de Humberto Badollato. }
\label{fig:manoelvianna-1915}
\end{figure}
}

% ---
% 06. Casa de Adolpho Moreira, na rua Direita da Mariquita
% ---

\afterpage{
\begin{a3paisagem}
\begin{figure}[!h]
\centering
\caption{Projecto para a reconstrucção da casa à Mariquita, Rio Vermelho, do Snr. Adolpho Moreira (1913)}
\includegraphics[width=0.9\textwidth]{8-anexos/plantas/05-mariquita/06-mariquita/1913-adolphomoreira.jpg}{\footnotesize \par \textbf{Fonte:} \textbf{BR BAAHMS}, Fundo ``Intendência'', Série ``Processos de Licenciamento de Reforma e Ampliação de Edificações'', Subsérie ``Requerimentos e Plantas -- Brotas'', caixa 06. }
\label{fig:1913-adolphomoreira.jpg}
\end{figure}
\end{a3paisagem}
}

% ---
% 07. Padaria de Domingos de Oliveira Reis, na Mariquita
% ---

\afterpage{
\begin{a3paisagem}
\begin{figure}[!h]
\centering
\caption{Rrojecto para a construcção de uma casa e forno de padaria, à rua da Mariquita, Rio Vermelho (1914)}
\includegraphics[height=0.9\textheight]{8-anexos/plantas/05-mariquita/06-mariquita/1914-domingosdeoliveirareis.jpg}{\footnotesize \par \textbf{Fonte:} \textbf{BR BAAHMS}, Fundo ``Intendência'', Série ``Processos de Licenciamento de Reforma e Ampliação de Edificações'', Subsérie ``Requerimentos e Plantas -- Brotas'', caixa 06. \par Projeto de Arthur Santos. O mesmo proprietário tinha uma casa de térreo e primeiro pavimento na Mariquita, vizinha à padaria.}
\label{fig:1914-domingosdeoliveirareis}
\end{figure}
\end{a3paisagem}
}

% ---
% 08. Casa de Maria Zifirina da Conceição
% ---

\afterpage{
\begin{a3paisagem}
\begin{figure}[!h]
\centering
\caption{Progeto de um predio a construir-se na Pedrinha de propriedade da Exma. Snra. D. Maria Zefirina da Conceição (1918)}
\includegraphics[width=1\textwidth]{8-anexos/plantas/05-mariquita/23-pedrinhas/1918-mariazifirinadaconceicao.jpg}{\footnotesize \par \textbf{Fonte:} \textbf{BR BAAHMS}, Fundo ``Intendência'', Série ``Processos de Licenciamento de Reforma e Ampliação de Edificações'', Subsérie ``Requerimentos e Plantas -- Brotas'', caixa 23. \par Projeto de José Portella Passos. Esta casa chama a atenção pelo número de quartos: \textit{oito}, variando entre 7,92$m^{2}$ até os 14,91$m^{2}$. }
\label{fig:1918-mariazifirinadaconceicao}
\end{figure}
\end{a3paisagem}
}

% ---
% 09. Casa de Aurelio Gonçalves Cal
% ---

\afterpage{
\begin{a3paisagem}
\begin{figure}[!h]
\centering
\caption{Representação da casa de nº 30 sita a rua direita da Mariquita, ao Rio Vermelho, com projecto de remodelação de sua fachada e de um pequeno pavimento (1922)}
\includegraphics[height=0.9\textheight]{8-anexos/plantas/05-mariquita/06-mariquita/1922-aureliogoncalvescal.jpg}{\footnotesize \par \textbf{Fonte:} \textbf{BR BAAHMS}, Fundo ``Intendência'', Série ``Processos de Licenciamento de Reforma e Ampliação de Edificações'', Subsérie ``Requerimentos e Plantas -- Brotas'', caixa 06. \par Projeto de Alfredo Vieira de Almeida. Esta casa destaca-se por ter \textit{seis} quartos no primeiro pavimento, e outros \textit{sete} no pavimento a construir, com cerca de 10$m^{2}$ cada.}
\label{fig:1922-aureliogoncalvescal.jpg}
\end{figure}
\end{a3paisagem}
}

% ---
\chapter{Plantas selecionadas de obras no Matatu Grande, Matatu Pequeno, Quinta das Beatas e vizinhança}
% ---



%\afterpage{
%\begin{a3paisagem}
%\begin{figure}[!h]
%\centering
%\caption{}
%\includegraphics[height=0.9\textheight]{8-anexos/plantas/09-amaralinapituba/19-amaralina/.jpg}{\footnotesize \par \textbf{Fonte:} \textbf{BR BAAHMS}, Fundo ``Intendência'', Série ``Processos de Licenciamento de Reforma e Ampliação de Edificações'', Subsérie ``Requerimentos e Plantas -- Brotas'', caixa 19}
%\label{fig:}
%\end{figure}
%\end{a3paisagem}
%}

% ---
\chapter{Plantas selecionadas de obras no Acupe e vizinhança}
% ---

\input{8-anexos/texto/07-acupe}

% ---
\chapter{Plantas selecionadas de obras nas antigas fazendas Alagoa, Amaralina, Santa Cruz, Ubarana e Pituba}\label{cap:anexosamaralina}
% ---



%\afterpage{
%\begin{a3paisagem}
%\begin{figure}[!h]
%\centering
%\caption{}
%\includegraphics[width=\textwidth]{8-anexos/plantas/09-amaralinapituba/19-amaralina/.jpg}{\footnotesize \par \textbf{Fonte:} \textbf{BR BAAHMS}, Fundo ``Intendência'', Série ``Processos de Licenciamento de Reforma e Ampliação de Edificações'', Subsérie ``Requerimentos e Plantas -- Brotas'', caixa 19}
%\label{fig:}
%\end{figure}
%\end{a3paisagem}
%}

\afterpage{
\begin{a3paisagem}
\begin{figure}[!h]
\centering
\caption{Propriedade de A. Saffrey, architecto, Amaralina. Lote 116 da planta geral da cidade d'Amaralina (1915). Projeto de A. Saffrey.}
\includegraphics[height=0.9\textheight]{8-anexos/plantas/09-amaralinapituba/19-amaralina/dsc04777.jpg}{\footnotesize \par \textbf{Fonte:} \textbf{BR BAAHMS}, Fundo ``Intendência'', Série ``Processos de Licenciamento de Reforma e Ampliação de Edificações'', Subsérie ``Requerimentos e Plantas -- Brotas'', caixa 19.}
\label{fig:dsc04777}
\end{figure}
\end{a3paisagem}
}

\afterpage{
\begin{a3paisagem}
\begin{figure}[!h]
\centering
\caption{Projeto de uma casa em Amaralina de D. Lydia Dewald (1915). Desenho de Ciro Spínola, engenheiro desconhecido.}
\includegraphics[height=0.9\textheight]{8-anexos/plantas/09-amaralinapituba/19-amaralina/DSC04783.jpg}{\footnotesize \par \textbf{Fonte:} \textbf{BR BAAHMS}, Fundo ``Intendência'', Série ``Processos de Licenciamento de Reforma e Ampliação de Edificações'', Subsérie ``Requerimentos e Plantas -- Brotas'', caixa 19.}
\label{fig:DSC04783}
\end{figure}
\end{a3paisagem}
}

\afterpage{
\begin{a3paisagem}
\begin{figure}[!h]
\centering
\caption{Projecto da casa que o Ilmº. Snr. José Cardozo da Silva pretende construir em seu terreno, sito à Ubarana, no districto de Brotas (1915). Projeto de Archimedes Marques.}
\includegraphics[height=0.9\textheight]{8-anexos/plantas/09-amaralinapituba/19-amaralina/dsc04791.jpg}{\footnotesize \par \textbf{Fonte:} \textbf{BR BAAHMS}, Fundo ``Intendência'', Série ``Processos de Licenciamento de Reforma e Ampliação de Edificações'', Subsérie ``Requerimentos e Plantas -- Brotas'', caixa 19}
\label{fig:dsc04791}
\end{figure}
\end{a3paisagem}
}

\afterpage{
\begin{a3paisagem}
\begin{figure}[!h]
\centering
\caption{Projecto de uma casa que o sr. Chehadi E. Kraycheti quer construir em Amaralina (1923). Projeto de Eurico da Costa Coutinho.}
\includegraphics[width=\textwidth]{8-anexos/plantas/09-amaralinapituba/19-amaralina/dsc4805.jpg}{\footnotesize \par \textbf{Fonte:} \textbf{BR BAAHMS}, Fundo ``Intendência'', Série ``Processos de Licenciamento de Reforma e Ampliação de Edificações'', Subsérie ``Requerimentos e Plantas -- Brotas'', caixa 19}
\label{fig:dsc4805}
\end{figure}
\end{a3paisagem}
}

\afterpage{
\begin{a3paisagem}
\begin{figure}[!h]
\centering
\caption{Projecto para construção de oito casas à Amaralina, primeira parte (1925). Projeto de Pedro Jayme David.}
\includegraphics[height=0.9\textheight]{8-anexos/plantas/09-amaralinapituba/19-amaralina/dsc04810.jpg}{\footnotesize \par \textbf{Fonte:} \textbf{BR BAAHMS}, Fundo ``Intendência'', Série ``Processos de Licenciamento de Reforma e Ampliação de Edificações'', Subsérie ``Requerimentos e Plantas -- Brotas'', caixa 19}
\label{fig:dsc04810}
\end{figure}
\end{a3paisagem}
}

\afterpage{
\begin{a3paisagem}
\begin{figure}[!h]
\centering
\caption{Projecto para construção de oito casas à Amaralina, segunda parte (1925). Projeto de Pedro Jayme David.}
\includegraphics[width=\textwidth]{8-anexos/plantas/09-amaralinapituba/19-amaralina/dsc04809.jpg}{\footnotesize \par \textbf{Fonte:} \textbf{BR BAAHMS}, Fundo ``Intendência'', Série ``Processos de Licenciamento de Reforma e Ampliação de Edificações'', Subsérie ``Requerimentos e Plantas -- Brotas'', caixa 19}
\label{fig:dsc04809}
\end{figure}
\end{a3paisagem}
}

\afterpage{
\begin{a3paisagem}
\begin{figure}[!h]
\centering
\caption{Projecto para construcção do predio sito na Amaralina, propriedade do Ilmº Snr. João da Cunha Freire, primeira parte (1925). Projeto de Rossi Baptista.}
\includegraphics[height=0.9\textheight]{8-anexos/plantas/09-amaralinapituba/19-amaralina/dsc04813.jpg}{\footnotesize \par \textbf{Fonte:} \textbf{BR BAAHMS}, Fundo ``Intendência'', Série ``Processos de Licenciamento de Reforma e Ampliação de Edificações'', Subsérie ``Requerimentos e Plantas -- Brotas'', caixa 19}
\label{fig:dsc04813}
\end{figure}
\end{a3paisagem}
}

\afterpage{
\begin{a3paisagem}
\begin{figure}[!h]
\centering
\caption{Projecto para construcção do predio sito na Amaralina, propriedade do Ilmº Snr. João da Cunha Freire, segunda parte (1925). Projeto de Rossi Baptista.}
\includegraphics[width=\textwidth]{8-anexos/plantas/09-amaralinapituba/19-amaralina/dsc04814.jpg}{\footnotesize \par \textbf{Fonte:} \textbf{BR BAAHMS}, Fundo ``Intendência'', Série ``Processos de Licenciamento de Reforma e Ampliação de Edificações'', Subsérie ``Requerimentos e Plantas -- Brotas'', caixa 19}
\label{fig:dsc04814}
\end{figure}
\end{a3paisagem}
}

% ---
\chapter{Documentos selecionados}
% ---

%\afterpage{
%\begin{a3paisagem}
%\begin{figure}[!h]
%\centering
%\caption{}
%\includegraphics[width=\textwidth]{8-anexos/plantas/09-amaralinapituba/19-amaralina/.jpg}{\footnotesize \par \textbf{Fonte:} \textbf{BR BAAHMS}, Fundo ``Intendência'', Série ``Processos de Licenciamento de Reforma e Ampliação de Edificações'', Subsérie ``Requerimentos e Plantas -- Brotas'', caixa 19}
%\label{fig:}
%\end{figure}
%\end{a3paisagem}
%}

% ---
\chapter{``A Boa Vista'', por Castro Alves}\label{cap:boavista}
% ---

% \poemtitle{A Boa Vista}
\settowidth{\versewidth}{}

\begin{flushright}
\textit{Sonha, poeta, sonha! Aqui sentado \\
No tosco assento da janela antiga,\\
Apóias sobre a mão a face pálida,\\
Sorrindo — dos amores à cantiga.}\\
Álvares de Azevedo
\end{flushright}

\begin{verse}
Era uma tarde triste, mas límpida e suave... \\
Eu — pálido poeta — seguia triste e grave \\
A estrada, que conduz ao campo solitário, \\
Como um filho, que volta ao paternal sacrário, \\
\end{verse}

\begin{verse}
E ao longe abandonando o múrmur da cidade \\
— Som vago, que gagueja em meio à imensidade, — \\
No drama do crepúsculo eu escutava atento \\
A surdina da tarde ao sol, que morre lento. \\
\end{verse}

\begin{verse}
A poeira da estrada meu passo levantava, \\
Porém minh'alma ardente no céu azul marchava \\
E os astros sacudia no vôo violento \\
— Poeira, que dormia no chão do firmamento. \\
\end{verse}

\begin{verse}
A pávida andorinha, que o vendaval fustiga, \\
Procura os coruchéus da catedral antiga. \\
Eu — andorinha entregue aos vendavais do inverno, \\
Ia seguindo triste p'ra o velho lar paterno. \\
\end{verse}

\begin{verse}
Como a águia, que do ninho talhado no rochedo \\
Ergue o pescoço calvo por cima do fraguedo, \\
— (P'ra ver no céu a nuvem, que espuma o firmamento, \\
E o mar, — corcel que espuma ao látego do vento...) \\
Longe o feudal castelo levanta a antiga torre, \\
Que aos raios do poente brilhante sol escorre! \\
Ei-lo soberbo e calmo o abutre de granito \\
Mergulhando o pescoço no seio do infinito \\
E lá de cima olhando com seus clarões vermelhos \\
Os tetos, que a seus pés parecem de joelhos!... \\
\end{verse}

\begin{verse}
Não! Minha velha torre! Oh! atalaia antiga, \\
Tu olhas esperando alguma face amiga, \\
E perguntas talvez ao vento, que em ti chora: \\
``Por que não volta mais o meu senhor d'outrora? \\
Por que não vem sentar-se no banco do terreiro \\
Ouvir das criancinhas o riso feiticeiro, \\
E pensando no lar, na ciência, nos pobres \\
Abrigar nesta sombra seus pensamentos nobres? \\
\end{verse}

\begin{verse}
Onde estão as crianças — grupo alegre e risonho \\
— Que escondiam-se atrás do cipreste tristonho... \\
\end{verse}

\begin{verse}
Ou que enforcaram rindo um feio Pulchinello, \\
Enquanto a doce Mãe, que é toda amor, desvelo \\
Ralha com um rir divino o grupo folgazão, \\
Que vem correndo alegre beijar-lhe a branca mão?...'' \\
\end{verse}

\begin{verse}
É nisto que tu cismas, ó torre abandonada, \\
Vendo deserto o parque e solitária a estrada. \\
No entanto eu — estrangeiro, que tu já não conheces — \\
No limiar de joelhos só tenho pranto e preces. \\
\end{verse}

\begin{verse}
Oh! deixem-me chorar!... Meu lar... meu doce ninho! \\
Abre a vetusta grade ao filho teu mesquinho! \\
Passado — mar imenso!... inunda-me em fragrância! \\
Eu não quero lauréis, quero as rosas da infância. \\
\end{verse}

\begin{verse}
Ai! Minha triste fronte, aonde as multidões \\
Lançaram misturadas glórias e maldições... \\
Acalenta em teu seio, ó solidão sagrada! \\
Deixa est'alma chorar em teu ombro encostada! \\
\end{verse}

\begin{verse}
Meu lar está deserto... Um velho cão de guarda \\
Veio saltando a custo roçar-me a testa parda, \\
Lamber-me após os dedos, porém a sós consigo \\
Rusgando com o direito, que tem um velho amigo... \\
Como tudo mudou-se!... O jardim 'stá inculto \\
As roseiras morreram do vento ao rijo insulto... \\
A erva inunda a terra; o musgo trepa os muros \\
A ortiga silvestre enrola em nós impuros \\
Uma estátua caída, em cuja mão nevada \\
A aranha estende ao sol a teia delicada!... \\
Mergulho os pés nas plantas selvagens, espalmadas, \\
As borboletas fogem-me em lúcidas manadas... \\
E ouvindo-me as passadas tristonhas, taciturnas, \\
Os grilos, que cantavam, calaram-se nas furnas... \\
\end{verse}

\begin{verse}
Oh! jardim solitário! Relíquia do passado! \\
Minh'alma, como tu, é um parque arruinado! \\
Morreram-me no seio as rosas em fragrância, \\
Veste o pesar os muros dos meus vergéis da infância, \\
\end{verse}

\begin{verse}
A estátua do talento, que pura em mim s'erguia, \\
Jaz hoje — e nela a turba enlaça uma ironia!... \\
Ao menos como tu, lá d'alma num recanto \\
Da casta poesia ainda escuto o canto, \\
— Voz do céu, que consola, se o mundo nos insulta, \\
E na gruta do seio murmura um treno oculta. \\
\end{verse}

\begin{verse}
Entremos!... Quantos ecos na vasta escadaria, \\
Nos longos corredores respondem-me à porfia!... \\
\end{verse}

\begin{verse}
Oh! casa de meus pais!... A um crânio já vazio, \\
Que o hóspede largando deixou calado e frio, \\
Compara-te o estrangeiro — caminhando indiscreto \\
Nestes salões imensos, que abriga o vasto teto. \\
\end{verse}

\begin{verse}
Mas eu no teu vazio — vejo uma multidão \\
Fala-me o teu silêncio — ouço-te a solidão!... \\
Povoam-se estas salas... \\
\end{verse}

\begin{verse}
E eu vejo lentamente \\
No solo resvalarem falando tenuemente \\
Dest'alma e deste seio as sombras venerandas \\
Fantasmas adorados — visões sutis e brandas... \\
\end{verse}

\begin{verse}
Aqui... além... mais longe... por onde eu movo o passo, \\
Como aves, que espantadas arrojam-se ao espaço, \\
Saudades e lembranças s'erguendo — bando alado — \\
Roçam por mim as asas voando p'ra o passado. \\
\end{verse}

\attrib{Boa Vista, 18 de novembro de 1867.}

\end{anexosenv}