\begin{landscape}
\chapter{Salas de cinema em Salvador, 1897-1930}
\begin{longtabu} to \textheight {|m{3cm}|m{4cm}|m{1,5cm}|m{2cm}|m{10cm}|}
\hline Nome & Endereço & Abertura & Fechamento & Nota \\ \hline \endhead
\hline \multicolumn{5}{c}{Continua na próxima página...} \\ \endfoot
\hline \endlastfoot
Edison & Praça Castro Alves, por cima da Confeitaria Luso-Brasileira & 1898 & 1906 & -- \\
\hline
Cassino Castro Alves & Praça Castro Alves, onde depois foi instalado o Teatro Guarani & 1903 & 1906 & Funcionava ao ar livre \\
\hline
Santo Antonio & Praça Barão do Triunfo (antigo Largo do Santo Antônio) & 1907 & 1907 & Incendiou-se pouco depois de inaugurado \\
\hline
Salesianos & Rua Conselheiro Almeida Couto, 19 & 1907 & -- & Propriedade do Liceu Salesiano do Salvador. Possui teatrinho para diversão de seus alunos. \\
\hline
Bahia & Rua Chile, nº 1 & 1909 & 1911 & O mais elegante e bem montado de sua época \\
\hline
Jandaia & Rua Dr. Seabra & 1910 & -- & Um dos mais populares \\
\hline
Bijou Teatro-Cinema & Calçada do Bonfim & 1910 & 1911 & -- \\
\hline
Popular & Rua da Madragoa, nº 5, no arrabalde de Itapagipe & 1910 & 1919 & -- \\
\hline
Cinema Odeon & Calçada do Bonfim, antigo prédio Mira-Mar, próximo à estação da Estrada de Ferro & 1919 & 1920 & -- \\
\hline
Avenida & Travessa de Sant'Anna (Rio Vermelho) & 1910 & -- & -- \\
\hline
Castro Alves & Largo do Carmo & 1910 & 1911 & Funcionou primeiramente na cidade vizinha de Santo Amaro, ocorrendo ali, na noite da estreia (2 de outubro de 1910) o incêndio do antigo Teatro S. Pedro, causado por uma fita que se queimou \\
\hline
Central & Praça Castro Alves, na parte térrea do antigo Hotel Paris & 1910 & 1912 & -- \\
\hline
Recreio Fratelli Vita & Calçada do Bonfim, nº 20 & 1911 & 1919 & No mesmo prédio onde funcionava a grande fábrica de licores e gasosas dos mesmos proprietários \\
\hline
Bahia & Largo do Papagaio, nº 38 (Itapagipe) & 1911 & 1915 & -- \\
\hline
Rio Branco & Rua do Saldanha, nº 2 & 1911 & 1912 & -- \\
\hline
Iris-Teatro & Rua Dr. Seabra & 1912 & 1913 & Pouco depois de inaugurado, seu proprietário, coronel Ruben Guimarães, mudou-lhe o título para Eclair. De novembro de 1913 até 1915, quando de outros proprietários, teve, ainda, os títulos de Paz e Amor, Caraboo e Olímpia. \\
\hline
Soledade & Ladeira da Soledade, nº 112 & 1912 & 1913 & Funcionava no antigo prédio conhecido por Palacete Bandeira \\
\hline
Ideal & Ladeira de S. Bento, nº 3 & 1913 & 1921 & Construído pelo primeiro arrendatário, Thomás Antenor Borges da Mota (1913-1915). Reformado em 1915. Arrendado à Pimentel \& Lima em 1917. Funcionou, do fim de 1917 até 8 de novembro de 1921, sob a responsabilidade exclusiva e direção do coronel João Gaudêncio de Lima, quando foi extinto. Rui Barbosa foi batizado neste mesmo prédio em 5 de maio de 1850. \\
\hline
Petit-Cinema & Rua Dr. Agripino Dória (Brotas) & 1913 & 1914 & -- \\
\hline
Recreativo & Largo de Sant'Anna (Rio Vermelho) & 1913 & 1914 & Funcionava no Centro Recreativo. \\
\hline
Centro Católico & Largo de S. Antônio da Mouraria. & 1913 &  & Propriedade particular do Centro Católico, instalado no prédio da Socidade S. Vicente de Paula, junto à igreja de S. Antônio da Mouraria. \\
\hline
Parisiense & Praça Dois de Julho (antigo Campo Grande) & 1914 & 1914 & -- \\
\hline
Forte de São Pedro & Praça da Aclamação & 1914 &  & Funciona na grande área do quartel. Foi fundado pelo então comandante do extinto 50º Batalhão de Caçadores, Tenente Coronel Francisco Cabral da Silveira, com o intuito de desenvolver a educação cívica e moral dos soldados e praças. \\
\hline
Cinema da Barra & Rua Barão de Sergy, nº 22 & 1914 & 1918 & -- \\
\hline
Olímpia & Rua Dr. Seabra & 1915 &  & Recebeu este nome em 1915, quando o Iris-Téatre passou à empresa Amaral \& Oliveira. Em 1916 pasou a pertencer a Thomás Borges da Mota, que iniciou as obras de reforma e remodelação, transformando-o num dos melhores cinemas e casas públicas de diversão da capital. Possuía palco e caixa de teatro em condições de receber companhias bem constituídas. Inaugurado, depois das obras, em 27 de outubro de 1920. \\
\hline
Cine Venus & Rua Carlos Gomes, 25 & 1916 & 1916 & Chamado também de Lâmpada Vermelha, por ser esta sua única iluminação externa. \\
\hline
Recreio S. Jerônimo & Praça 15 de Novembro (antigo Terreiro de Jesus) & 1917 &  & Pertenceu à Obra Social Católica. Funcionava no salão (lado do mar) da Catedral, onde existiu a biblioteca dos jesuítas, e esteve depois instalado na Biblioteca Pública do Estado. Em 28 de setembro de 1922 passou a funcionar num grande edifício na Rua do Arcebispado. \\
\hline
Kursaal Baiano & Praça Castro Alves & 1919 &  & Projetado e construído por Filinto Santoro, com caixa e palco, inaugurado pela empresa Portella Passos \& Cia. Ltda. Em 24 de dezembro de 1919. Em 13 de maio de 1920 seu nome foi mudado para Teatro Guarani depois de concurso aberto pelo jornal A Manhã. \\
\hline
Cinema Itapagipe & Rua do Poço, nº 155 & 1920 &  & Empresa Cirne \& Cirne \\
\hline
Cinema Liceu & Rua do Liceu & 1921 &  & Propriedade do Liceu de Artes e Ofícios. Tinha lucros revertidos ao funcionamento do próprio Liceu. Este cinema e o Recreio S. Jerônimo foram isentados de impostos muncipais pela Resolução 516, de 7 de outubro de 1921. \\
\hline
Politeama Baiano & Politeama & 1897 &  & Aberto e fechado várias vezes depois de sua inauguração. Não era preferido pelo público por estar afastado do Centro da cidade, ou pelo grande tamanho da sala. \\
\hline
Teatro São João & Praça Castro Alves & 1899 & 1911 & Aberto, como cinema, por um italiano. Quase incendiado em uma de suas exibições. Entre 1906 e 1907 funcionou ali a Caixa Econômica Federal e Monte do Socorro, que o deixou em condições péssimas. Em 6 de março de 1908 foi arrendado ao Dr. João Rodrigues Germano. Reformado e usado como cinema até 1911, quando o contrato com o governo foi rescindido. Funcionou até 1920 sob responsabilidade do coronel Ruben Guimarães, sublocatário de João Rodrigues Germano, que assumiu o contrato com o governo. Como o governo requisitou o teatro em dezembro de 1920, Ruben se retirou definitivamente do prédio e retirou não somente os melhoramentos feitos, mas toda a mobília anteriormente existente. \\
\hline
\end{longtabu}
\end{landscape}
