\chapter{Sobre topônimos e certas liberalidades no uso de mapas}
\label{ap:mapas}

A história da produção, apropriação e uso de um território não se faz sem referência a lugares, áreas, regiões, ruas, estradas, caminhos, becos --- não se faz, em suma, sem algum esforço toponímico. Na tentativa de encontrar estes pontos notáveis, o melhor instrumento são sempre os \textit{mapas}. Os atuais, entretanto, apresentam um território que, em comparação com aquele existente durante o período estudado, aparentava num primeiro momento ter sido tão modificado pela expansão urbana dos anos 1970--1990 que tornaria inútil qualquer esforço neste sentido.

Salvador, entretanto, é uma cidade com particularidades muito propícias a quem se dedique a estudar sua história. Tendo sido capital da colônia, porto de importância internacional, segunda maior cidade do país até as primeiras décadas do século XX e metrópole regional de importância até os dias atuais, era de se esperar terem sido produzidos mapas em profusão para representar o território soteropolitano.

Pesquisando os mapas, apareceu imediatamente outro problema: a \textit{visão de cidade} dos cartógrafos não coincidia de forma alguma com a da atualidade --- e nem poderia ser diferente. Nos muitos mapas apresentados por \citeonline{VASCONCELOS2002}, nota-se abundância de representações das freguesias da Sé, São Pedro, Passo, Conceição da Praia e Pilar; só do século XVIII em diante as freguesias dos Mares e da Penha passam a ser representadas, e apenas no século XIX foi possível encontrar representações cartográficas da Vitória e do Santo Antônio.

Brotas?

Ora, que cartógrafo em sã consciência dedicaria seu tempo de trabalho a representar uma freguesia de povoação pontilhada, com uma casa aqui, outra lá adiante, espaçadas todas por quilômetros de mata, pasto e roça?

Somente no mapa de \citeonline{weyll_mappa_1851} aparecem as \textit{vias principais} de Brotas e de outros distritos periurbanos e suburbanos --- mas isto porque estava no escopo de Carlos Weyll representar ``S. Salvador e seus subúrbios'', e não apenas a cidade, a malha urbana. Ao que se pôde pesquisar, \textit{nenhum} outro cartógrafo voltou sua atenção para os mesmos lugares antes da década de 1930, quando a Prefeitura mandou fazer uma interessante planta da cidade que --- desbotada, quase apagada, de difícil leitura --- encontra-se custodiada na mapoteca do \textbf{Arquivo Histórico Municipal de Salvador}.

Que fazer? Como localizar os pontos notáveis que insistiam em aparecer na documentação?

Um primeiro caminho foi encontrar os muitos \textit{guias} de ruas da cidade. Experiência frustrante. Entre os mais recentes, nenhum tem pesquisa toponímica adequada, tudo feito com base no ``ouvi dizer'' ou na imaginação de seus autores. Entre os mais antigos, como o de \citeonline{moraes_ruas_1959}, encontrava-se um repositório do que era, então, o saber mais atual, vez que no século XIX conservavam as ruas soteropolitanas os mesmos nomes quase desde os tempos da colônia --- mas, úteis como fossem para as ruas da malha urbana consolidada, pouco ou nada diziam acerca das ruas, becos, estradas e caminhos distantes da Brotas rural de então.

Retornando ao mapa de \citeonline{weyll_mappa_1851}, uma surpresa: algumas vias suburbanas descritas no mapa pareciam \textit{rigorosamente iguais} a vias conhecidas, porque perduraram. A surpresa apenas confirmou o que já se sabe acerca das ruas mais antigas de Salvador, no distrito da Sé: a Direita do Palácio (atual rua Chile), há quantos séculos terá o mesmo trajeto? Não terá sido a avenida Sete de Setembro construída sobre o leito plurissecular do caminho do Conselho? Nada a estranhar: se, comparativamente, mesmo na Inglaterra de hoje algumas vias  Tal impressão se confirmou: conseguida uma cópia digital em alta resolução deste mapa junto à \textbf{Biblioteca Nacional}, sua sobreposição georreferenciada aos mapas disponibilizados pelo \textbf{Open Street Map} e pelo \textbf{Google Maps} demonstrou que, salvo uma ou outra distorção pequena de escala, a malha viária principal de Brotas, tal como hoje a conhecemos, estava quase toda ela \textit{pronta e acabada} em 1851. Restava dar nomes às ruas consultando os guias antigos.

Mesmo assim, havia muitas lacunas. Alguns caminhos pareciam resistir à identificação, pareciam querer guardar o anonimato a que Carlos Weyll lhes condenara.

Por acaso foi possível encontrar na biblioteca do Centro de Estudos e Ação Social (CEAS) um exemplar raríssimo do \textbf{Atlas parcial da cidade do Salvador} que a Prefeitura mandara confeccionar em 1955. Mesmo então, problemas: os mapas --- todos muito bons, completos até a escala dos lotes individuais --- não alcançavam o núcleo central de Brotas. Concentravam-se na área próxima ao centro consolidado de Salvador, não avançavam sequer até a Boa Vista. Quatro lâminas chamaram a atenção: ``1º de Maio'', com um pedaço da Djalma Dutra, da ladeira dos Bandeirantes e do Matatu; ``Fonte Nova'', com trecho significativo da ladeira do Pepino e da vila Santos; ``Cosme de Farias'', que na verdade contém apenas a entrada deste bairro mas apresenta quase todo o antigo 1º Distrito de Brotas; ``Vasco da Gama'', com quase todo o Engenho Velho de Brotas. Estas quatro folhas, por serem documento histórico importante e raro para a história da urbanização de Salvador, estão reproduzidas em anexo, bastante ampliadas em comparação com os originais.

Este mesmo atlas contém num apêndice uma lista enorme de nomes antigos e ``atuais'' de ruas (``atuais'' para a época, claro). Esta lista e a de \citeonline{moraes_ruas_1959} foram os primeiros passos para dar um ``lugar'' aos topônimos que se repetiam por toda a documentação pesquisada. Todas as ruas constantes no atlas foram checadas, uma a uma, com aquelas constantes no guia de Mello Moraes e com a base toponímica constituída com base na documentação pesquisada. Aquelas constantes no \textbf{Atlas} foram enfim localizadas, situadas e visualizadas, permitindo entender melhor o que havia perdurado desde a Primeira República até a década de 1950.

Outros logradouros, entretanto, recalcitravam. O jeito foi radicalizar o método. Percorrendo a documentação, encontrava-se um ou outro documento de terras ou pedido de licença situado na esquina de duas ruas. Como num quebra-cabeças, cada esquina, cada vizinhança, cada proximidade foi destacada, anotada e comparada com o mapa de Carlos Weyll, única referência cartográfica segura para o território de Brotas até a década de 1930. Esta paciente bricolagem chegou a bom termo, com quase 80\% das vias e localidades associadas aos nomes que tiveram ao longo do tempo.

Alguns pontos importantes ficaram sem qualquer referência. Tratava-se dos pontos mais afastados, na fronteira com os distritos de Itapuã e Santo Antônio, nos confins onde nenhum cartógrafo jamais esteve. Parecia ainda mais estranho a estranha ausência da bacia dos rios das Pedras e Pituaçu, limite oriental do distrito. Ausência? Não. Houve, sim, representações cartográficas importantes da bacia dos rios das Pedras e Pituaçu em mapas dedicados à \textit{hidrografia} da baía de Todos os Santos; entre todas, pareceram mais equilibradas aquelas de \citeonline{lealteixeira_carta_1830} e \citeonline{sampaio_reconcavo_1899}, reproduzidas em anexo (da última, apenas um recorte muito ampliado). Ali os rios das Pedras e Pituaçu aparecem enormes, cortando no sentido noroeste-sudeste quase toda a península onde Salvador se situa. Exagero? Não é hoje esta bacia apenas a \textit{quarta} maior de Salvador, atrás das bacias do Ipitanga, Jaguaribe e Camarajipe \cite{santos_aguas_2010}? Não se pode esquecer como a intensa urbanização ocorrida nas décadas de 1970 e 1980 em bairros como Doron, Narandiba, Beiru, Arenoso, Novo Horizonte, Barreiras, Engomadeira, Saboeiro, Imbuí e Boca do Rio fez-se por meio do aterro de nascentes, assoreamento de lagoas, tamponamento de cursos d'água, poluição e consequente desativação de barragens. Não se podia esperar outro resultado além da diminuição da vazão dos rios e consequente encolhimento da bacia.

Adicionados estes dois mapas, pareciam enfim bem desenhados os limites do distrito de Brotas. Dois felizes acontecimentos permitiram enfim conhecer \textit{todos} os pontos notáveis e logradouros encontrados na documentação pesquisada.

Em primeiro lugar, Odete Dourado, orientadora desta pesquisa, apresentou o \textbf{Guia da cidade do Salvador (Estado da Bahia --- Brasil)} de Raymundo Camillo de Souza. Publicado em 1935, este livreto foi concebido como um guia para guardas civis; pela sua própria descrição, continha ``dados indicativos de todas as ruas, avenidas, praças, largos, travessas, becos, etc., etc., trazendo a ``velha e nova nomenclatura'' e na rigorosa ordem alphabetica --- itinerário de todos os bonds --- fartas indicações commerciaes'' \cite{souza_guia_1935}. Historicamente próximo ao período recortado para esta pesquisa, completíssimo, deu nomes antigos e novos para literalmente \textit{todos} os logradouros soteropolitanos, sem uma exceção sequer. Mesmo as longínquas estradas rurais espalhadas pelo território de Salvador foram não somente listadas, mas situadas na malha viária, circunscritas, com indicações inclusive do \textit{itinerário} a ser feito para alcançá-las. Estavam aqui as informações necessárias para ligar os pontos da base de dados construída nesta pesquisa. Restavam ainda algumas dúvidas. O \textbf{Guia}, completo como fosse, pecava exatamente pelo \textit{excesso}: houve logradouros que apareceram em três ou quatro verbetes, às vezes com duas descrições diferentes a confundir o leitor. Em se tratando de nomes, o \textbf{Guia} é perfeito; como auxílio à cartografia, entretanto, apresentava falhas, dificuldades, vacilações. Faltaram apenas topônimos de relevância muito restrita para a pesquisa: \textit{baixa da Olaria} (Amaralina), \textit{Grão-Mogol} (Amaralina), \textit{Pomar} (Várzea de Santo Antônio). Estes, deixo a pesquisadores futuros a tarefa de localizar; para esta pesquisa, foram desimportantes.

Semanas depois, numa das últimas visitas ao \textbf{Arquivo Histórico Municipal de Salvador}, foram localizados alguns mapas antigos da cidade. Uma cópia do mapa de Carlos Weyll --- muito menor do que se imagina; alguns mapas originais, feitos à mão nas décadas de 1930 e 1940, infelizmente desbotados e imprestáveis à fotografia amadora de que me vali na pesquisa. Havia um mapa azul, datado de 1969, muito rico e detalhado, produzido com técnica topográfica moderna pelas mãos de Mário Martins de Oliveira por encomenda da Seção da Planta Cadadstral da Secretaria de Finanças da \citeonline{salvador_mapa_1969} --- quase quarenta anos depois do limite final do recorte histórico desta pesquisa. Parecia pouco útil, mas mesmo assim foi fotografado. Ao analisar as fotos, uma grata surpresa: no ``mapa azul'' estavam representadas, em traços muito finos, quase invisíveis, \textit{todas} as estradas rurais apresentadas no \textbf{Guia} de Raymundo Camillo de Souza, com nome e tudo! O desinteresse da burguesia e dos gestores pela expansão da malha urbana de Salvador a paragens tão distantes, qualquer que seja sua motivação, parecia ter resultado num tipo de ``congelamento'' dos velhos caminhos e estradas vicinais. Não se pode dizer ainda com absoluta segurança \textit{desde quando} tais vias existiam, mas \textit{com certeza} elas existiam em 1935, e a julgar por alguns documentos encontrados durante a pesquisa esta malha viária, ou parte significativa dela, \textit{estava lá desde pelo menos a década de 1910}!

Antes que a empolgação substituísse o método, era preciso submeter à crítica este achado, ao menos no que dizia respeito ao distrito de Brotas. De volta aos mapas antigos, destacou-se na comparação o mapa de \citeonline{visconderohan_mapa_1839}, que antes não despertara qualquer interesse. Mandado confeccionar pelo general João Chrisostomo Callado depois de suas tropas haverem aplastado a \textit{Sabinada}, foi terminado pelo desenhista António Pinto de Siqueira a mando do capitão Henrique de Beaurepaire-Rohan no Rio de Janeiro em 5 de julho de 1839. O grau de imprecisão corresponde ao de outros mapas da mesma época, ainda mais quando tudo indica ter sido feito muito rapidamente, apenas para registro das estradas percorridas pelas tropas imperiais (é o que diz o mapa), ou talvez misturando mapas anteriores (igualmente imprecisos) com a memória do topógrafo, ele próprio ao que tudo indica envolvido nas repressão ao levante já aos vinte e cinco anos de idade na patente de segundo-tenente. Como a \citeonline{VASCONCELOS2002} parecem ter interessado mais os mapas e plantas descritivas do \textit{centro} de Salvador, ele passou reto por este mapa em sua \textit{magna opera}, saltando direto do mapa de José Azevedo Galeão (1785) para o de Carlos \citeonline{weyll_mappa_1851}. A quem estude a hinterlândia soteropolitana mais próxima, entretanto, este mapa é uma pérola: com todos os cuidados em torno da imprecisão e da ligeireza do traço, pode-se encontrar nele a malha viária suburbana de Salvador \textit{anterior à construção da rua da Vala} (1849) e muitos pontos notáveis de outro modo ausentes da cartografia soteropolitana até a publicação do mapa de Carlos Weyll e mesmo depois\footnote{Destacam-se: o \textit{porto de Itacaranha}; o \textit{engenho do Cobre}; um \textit{candomblé} situado ao lado do mesmo \textit{Bate Folha} representado no mapa de Carlos Weyll; a localidade da \textit{Cajazeira}, antecessora do atual bairro de mesmo nome; a ``Estrada do Cabula de Cima'', atual rua Christiano Buys; o ``outeiro de S. Thomé'', por onde corre uma estrada que é a atual rua Waldemar Falcão; o \textit{rio Camarão}, situado onde hoje está o Calabar; o \textit{rio do Chega Negro}, atual desembocadura do Camarajipe no Costa Azul; a \textit{armação do Saraiva}, onde funcionou a antiga sede do Aeroclube da Bahia na Boca do Rio; a \textit{Cacunda da Yayá}, no Arenoso do Beiru; o alto da \textit{Muriçoca}, no São Rafael/Vale dos Lagos; a \textit{Bolandeira}, atual estação de tratamento de água homônima na Boca do Rio; e a \textit{estrada ``das Varges''}, atual rua Caminho de Areia, na península de Itapagipe.}. 

Quando comparado com o mapa de \citeonline{weyll_mappa_1851} e com o ``mapa azul'' da \citeonline{salvador_mapa_1969}, a malha viária do mapa de Beaurepaire-Rohan revelou-se surpreendentemente consistente --- mais uma vez tendo cautelas com a ligeireza do traço e algumas imprecisões, estavam ali vias importantes da zona rural de Salvador. Passado este teste de consistência, este mapa permitiu eliminar quaisquer dúvidas acerca do traçado das estradas de Pernambués, das Armações e do Saboeiro.

Parecia encerrado o levantamento de topônimos e sua localização. Havia um ou outro topônimo perdido, renitente, de difícil localização. Com que surpresa, ao retornar ao ``mapa azul'' da \citeonline{salvador_mapa_1969}, não foram encontrados todos eles! No aspecto fundiário, este mapa pareceu ter sido feito quase como quem sabe por que escreve o que escreve e indica o que indica. Nomes de fazendas antigas, marcos fundiários de outro modo esquecidos, estava tudo lá. Um exame detalhado mostra nele a existência de ``cercas'' e ``cercas vivas'' entre suas convenções. Ignora-se por completo que método terá seguido a equipe responsável por coletar as informações necessárias à sua confecção, mas de algum modo limites fundiários então existentes foram registrados para a posteridade. Terão sido empregues em sua preparação registros fundiários? Terá sido o \textbf{Livro Eclesial de Registro de Terras da Freguesia de Brotas} consultado para sua realização, ou ainda outros livros eclesiais de terras? O certo é que este mapa é posterior à promulgação da Lei Municipal 2.181, de dezembro de 1968, que abriu à aquisição particular em propriedade plena uma área de 25 milhões de metros quadrados de terras municipais, num processo de leilões que durou até 1975 e rendeu Cr\$ 57 milhões aos cofres da Prefeitura, a preços de 1976, e fez passar o regime fundiário hegemônico em Salvador das \textit{enfiteuses} para a \textit{propriedade plena} --- mantendo intacta, ou mesmo piorando, a concentração fundiária preexistente \cite{BRANDAO1980,SIMOES1985,VASCONCELLOS1974}. Neste contexto, a Prefeitura de Salvador decerto tinha em mãos pelo menos desde 1968 um diagnóstico fundiário, que cartografou neste mapa; o grau de precisão dos limites fundiários nele constantes, ou mesmo dos pontos notáveis delimitadores das velhas herdades, não pode ser avaliado adequadamente aqui, mas até o momento o mapa tem mostrado grande consistência tanto com a base de topônimos paulatinamente consolidada nesta pesquisa, quanto na comparação com pontos notáveis e topônimos soteropolitanos descritos em mapas atuais como o \textbf{Open Street Map} ou o \textit{Google Maps}, produzidos estes últimos a partir da comparação entre imagens de satélite e bases cartográficas digitais locais.

Percorrido todo este caminho, contando com mapas e listas de topônimos de épocas muito distintas, foi possível cercar a lista de logradouros encontrados na documentação pesquisada com um robusto aparato crítico, ao mesmo tempo sincrônico e diacrônico relativamente ao recorte temporal escolhido. De simples \textit{nomes}, de outro modo puramente abstratos, os logradouros puderam assim ser transformados em \textit{lugares}, cuja articulação numa \textit{estrutura espacial} se pôde ver ao longo desta dissertação. Procedimento decerto pouco ortodoxo, mas --- espera-se --- \textit{eficaz}.