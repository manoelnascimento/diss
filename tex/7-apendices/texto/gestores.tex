\chapter{Notas e comentários sobre os gestores na Primeira República}\label{ap:1}

Ainda não está plenamente consolidada na historiografia brasileira a análise das chamadas ``classes médias'' em suas partes constitutivas, menos ainda a caracterização de setores destas ``classes médias'' como \textit{gestores}, no sentido que João Bernardo dá à palavra (remeto à \autoref{subsec:cgpcsjobe} (p. \pageref{subsec:cgpcsjobe} para uma conceituação mais precisa). Se isto pode ser dito quanto ao emprego destas categorias na análise das sociedades contemporâneas, mais polêmico ainda é seu emprego na análise de sociedades pregressas. Por isto, faz-se necessário trazer, em apêndice, algum debate acerca do tema, centrado num artigo de \textit{José Evaldo de Mello Doin}; além do mérito de indicar expressamente sua divergência com a conceituação do Estado Amplo feita por João Bernardo, pois os ``conceitos do brilhante teórico português não se aplicam imediatamente a um rico processo histórico que não se deixa domar por modelos de interpretação não aclimatados'' \cite[pp.~103-104]{doin_bernardo_1996}, a crítica de José Doin permite contrapor certa leitura tradicional do desenvolvimento econômico brasileiro ao marco teórico estabelecido por João Bernardo.

A crítica de Doin funda-se em três argumentos centrais:

\begin{citacao}
Em nossa perspectiva, no que marca o movimento orgânico do desenvolvimento da estrutura econômico-social brasileira, o aparato conceitual bernardiano não se aplica em sua totalidade pelo menos com referência ao período que estamos enfocando (1891-1945), em virtude do desenvolvimento capitalista retardatário e dependente, que, por não estar fortemente suportado por uma considerável indústria pesada de bens de capital que garantisse uma reprodução endógena das forças reprodutivas, não criou "condições gerais da produção" que possibilitassem o advento do Estado Amplo. Indubitavelmente, entretanto, vamos assistir desde os primórdios republicanos a um processo de formação, lenta consolidação e finalmente hegemonia de uma classe de gestores, que espalmarão a exploração do proletariado crescente, abrigados nas instituições do Estado Restrito, que se multiplicam e se fortalecem, gerando, no caso brasileiro uma "ampliação" da unificação e do poder do Estado clássico, desfigurando-o e possibilitando o enraizamento da classe gestorial.

Travada a expansão do Estado Amplo pelo retardamento da indústria pesada de bens de produção (Departamento ou Setor I da economia na conceituação marxista), por outro lado, o bloqueio representado pela permanência do escravismo colonial na economia formalmente se inserindo no mercado capitalista mundial, impediu o processo de resistência da luta proletária que possibilitaria o  surgimento dos ``ciclos longos 
da mais-valia relativa'' a serem recuperados e adaptados pelo capital. Em que pese o movimento da resistência camponesa e operária (Canudos, Revolta da Chibata, movimentos paredistas de 1917 e 1919 etc.), ele não teve consistência suficiente para possibilitar uma avalancagem, mesmo que de proporções modestas, em direção à constituição do Estado Amplo. 

Um terceiro complicador se configura na aliança firme e inquebrantável da burguesia cafeeira e da crescente classe gestorial com os interesses do capital estrangeiro e com o mercado internacional. Esta é a ``face interna'' da acumulação capitalista brasileira, em que tanto os estratos da burguesia cafeeira, como dos setores urbanos envolvidos com o sistema financeiro, o comércio atacadista e o de importação e até mesmo as lideranças empresariais do modesto parque industrial que se firmava, estavam umbilicalmente articulados com os interesses do grande capital externo e em boa medida dele dependia para existir.

Portanto o principal ponto de articulação para a alavancagem e a interiorização do capitalismo no Brasil, passou a ser o Estado, tanto no seu papel de agente financiador da expansão, como no garantidor da ``ordem'' e repressor dos movimentos reivindicatórios, sem nos esquecermos de sua ação empresarial direta que a partir do Império cada vez mais se ampliava. É desta forma que o Estado torna-se a base essencial do poder de classe dos gestores, garantindo a exploração capitalista e promovendo até mesmo o aprofundamento e a expansão do proletariado com o apoio dado à imigração \cite[pp.~104-105]{doin_bernardo_1996}.
\end{citacao}

Tais argumentos, além de confundirem termos de debates distintos como se contidos num só, enraizam a classe dos gestores num lugar que não é propriamente o deles.

Em primeiro lugar, na conceituação marxista o chamado ``Departamento ou Setor I da economia'' não reproduz ``forças reprodutivas'', mas sim \textit{bens de produção}, ou seja, \textit{uma} entre as forças \textit{produtivas}. (É possível, a julgar pelo conteúdo do restante do artigo, que as tais ``forças reprodutivas'' não passem de um lapso do autor, pois adiante ele refere-se a ``forças produtivas'' para tratar do mesmo assunto.) Para piorar, nem a indústria de bens de produção é \textit{locus} exclusivo da classe dos gestores, nem tal indústria é, a rigor, uma das condições gerais de produção. Estas últimas, como visto, têm a ver não com os bens de produção, mas com o trabalho e a infraestrutura responsáveis pela \textit{integração entre empresas}; desta maneira, a classe dos gestores tem muito menos que ver com a indústria de bens de produção que com a pletora de guarda-livros, contadores, escriturários, contínuos, praticantes, autografistas, tesoureiros, fiéis, almoxarifes etc. tantas vezes mencionados na \autoref{subsubsec:clamed}, além dos engenheiros responsáveis, no final do Império e na Primeira República, pelos projetos e pela gestão e controle das obras infraestruturais (ferrovias, bondes, eletrificação, telefonia, portos etc.) tão procuradas pelos investidores britânicos, franceses, belgas, canadenses, suíços e alemães.

Aos problemas da escolástica marxista apresentados pelo autor como argumentos, soma-se uma concepção extremamente restrita do papel dos gestores na exploração da força de trabalho alheia. Segundo José Doin, os gestores fá-lo-iam a partir das instituições do Estado Restrito. Ora, a estrutura econômica brasileira durante a Primeira República é composta por empresas de portes variados, todas elas pulverizadas pelas cidades mais bem situadas na economia agroexportadora, e não se verificou no período qualquer interferência das instituições do Estado Restrito sobre os processos de trabalho em tais empresas antes de meados dos anos 1920, mesmo assim enfrentando muita resistência dos empresários a tal intromissão; como poderiam os gestores, então, interferir no processo de trabalho nestas empresas? Neste aspecto José Doin assumiu uma perspectiva a que, por falta de melhor conceito, chamaremos aqui de ``macroeconomia marxista ortodoxa'', ou seja, de uma \textit{análise da sociedade fundada exclusivamente nos processos macroeconômicos}, sem qualquer consideração teórica ou empírica acerca dos conflitos cotidianos entre trabalhadores, burgueses e gestores, ocorridos em esmagadora maioria no âmbito microeconômico das empresas.

Há outro aspecto passível de crítica nos argumentos de José Doin: a \textit{origem histórica} da classe dos gestores. Viu-se na \autoref{subsec:cgpcsjobe} que, na passagem do regime senhorial ao capitalismo na Europa, os gestores surgiram a partir de três lugares: a \textit{burocracia de corte}, a \textit{burocracia dos grandes soberanos e príncipes} e a \textit{burocracia das cidades}, responsáveis, as très, por criar as condições necessárias para a formação de empresas capitalistas propriamente ditas. De fato, como afirma José Doin, o escravismo colonial agroexportador -- ou, melhor dizendo, a aristocracia agrária que dele se beneficiava -- travou por longo tempo o desenvolvimento seja de indústrias, seja de empresas capitalistas tal como as conhecemos; para isto valeu-se de sua influência sobre a política econômica e fiscal brasileira por meio de sua hegemonia no parlamento imperial. Mas não se pode negar que mesmo neste cenário adverso a burocracia imperial e a burocracia urbana criaram condições mínimas para o deslanchar da atividade empresarial capitalista, bem aproveitadas pelos incipientes capitalistas brasileiros logo em seguida às reformas de Rui Barbosa nos primeiros anos da República. Um exemplo: as \textit{posturas municipais} regulamentadoras de horários e condições de trabalho, padrões para a produção de certas mercadorias, localização empresarial etc., descendo a minúcias como a estrutura, mobiliário e disposição interna dos prédios em determinados ramos de atividade; foi em torno das regras estabelecidas por tais posturas, naquilo a que se aplicavam, que se deram os primeiros conflitos laborais durante a República, antes mesmo da constituição de leis trabalhistas. 

Concordamos ainda com José Doin no que diz respeito à centralidade da cadeia produtiva cafeeira na economia da Primeira República. Há que se observar, entretanto, que \textit{centralidade} náo significa \textit{exclusividade}; a pesquisa no campo da história econômica do Brasil tem demonstrado nos últimos quarenta anos a existência de um mercado interno que, embora largamente suprido por importações no setor de bens de consumo suntuário e mesmo em alguns bens de consumo necessário (CITAR EXEMPLOS), era autossuficiente no setor de subsistência (CITAR EXEMPLOS), no qual se encontrava a maioria da população trabalhadora.

Já o historiador \textit{Marcello Felisberto Morais de Assunção}, diferentemente de José Doin, 