% ----------------------------------------------------------
\chapter{Crítica à análise da sociedade baiana feita por Francisco de\\ Oliveira em ``O elo perdido''}\label{ap:2}
% ----------------------------------------------------------

Para não prejudicar o fluxo de ideias no \autoref{cap:1}, mais especificamente na \autoref{subsec:1.1.2} o debate em torno dos equívocos perpetrados por Francisco de Oliveira na caracterização da estrutura social baiana da Primeira República foi transposto para este apêndice. Longas citações se farão necessárias para debater os equívocos do autor -- muitos deles, é preciso dizê-lo, causados por escassez de fontes secundárias na época da elaboração do ensaio \textbf{O elo perdido}, mas outros causados por distorções próprias ao autor.

Veja-se o primeiro equívoco:

\begin{citacao}
Uma cidade sede do capital bancário, que controla a circulação do excedente do cacau e do tabaco e uma indústria de pouca expressão, fundada na decadência do açúcar, nas poucas indústrias têxteis que restam, defendidas estas por ``barreiras'' regionais que serão desmanteladas pós-30, e umas poucas indústrias primárias, de transformação do cacau em manteiga, e de fabricação de charutos -- Danneman, Suerdieck. Uma fortíssima oligarquia, que vive faustosamente, letrada, cosmopolita, filhos estudando na Europa, dilapidadora e\dots investidora no Centro-Sul, Rio e São Paulo, principalmente.

A estrutura social pára, como num retrato amarelecido, no nível que se instaura no princípio do século. [\dots] Sob a impotência e a ostentação da oligarquia, Catarinos, Sás, Calmons, Calmons de Sá, Marianis, Bittencourts, Simões, Correias da Silva, Magalhães, e de outros arrivistas incorporados pela riqueza e pelo sobrenome estrangeiro, Dannemans, Suerdiecks, Wildbergers, vegeta uma população de não-reconhecidos. Sobretudo domésticas e domésticos, para fazer os quitutes das madamas e limpar as cocheiras dos \textit{messieurs}, negras de fartos seios para as amas-de-leite dos filhinhos de papai que depois vão pras Europas e voltam senhores, funcionários públicos [\dots], despachantes do porto de Salvador, bacharéis de títulos múltiplos [\dots], poucos operários, muitos biscateiros (os ``turcos'' da infância, os mascates), sacristãos das 365 igrejas, padres e freiras, para administrar a glória dos barrocos de ouro, a primeira comunhão dos filhinhos de papai que chegam `à idade da razão'', a extrema-unção para os que se despedem da vida farta procurando a certeza de continuarem assim ou até melhorarem -- fizeram tão boas obras de caridade na terra --, e as Irmandades e Ordens terceiras para conter a explosão dos cultos negro-africanos que coexistem, silenciosos, desde a colônia nas senzalas, e revoltados nos ``quilombos'' e na quase desconhecida rebelião dos malês, duramente reprimida e agora mais livres, pois que já não existe senzala mas não existe trabalho, um mundo em emergência, rompendo pela ação da passividade a ordem estabelecida e petrificada. \cite[p.~32-34]{OLIVEIRA1987}
\end{citacao}

\begin{citacao}
Uma divisão social do trabalho pouco desenvolvida, em termos capitalistas, em retrocesso mesmo. Predominância do que, na divisão social do trabalho em Salvador? Salvo as atividades diretamente ligadas ao setor capitalista, uma gota no oceano, o resto da cidade vive de ``expedientes''. É a circulação do excedente nas mãos da oligarquia financeira, e seus gastos suntuários, que alimenta a vida de ``expedientes''. E, no limite, o suntuário e o ostentário utilizam-se do vasto excedente de mão-de-obra para realçar o suntuário. Quem, na oligarquia financeira e nos seus agregados, gente do poder, funcionários de mais que meia-tigela, burocracia do capital e dos serviços, das escolas de medicina, direito e engenharia, não terá entre cinco e dez empregados domésticos? Menos que isso é sinal de pobreza dos\dots ricos \cite[p.~35]{OLIVEIRA1987}
\end{citacao}

\begin{citacao}
Trata-se de uma sociedade de classe ou é uma antecipação da ``sociedade sem classes''? [\dots] O olho arguto reconhecerá, ao contrário, uma sociedade de dissimulação, discurso que recorre ao plural para negar o singular. Objetividade da subjetividade: não se trata, ainda, de uma sociedade de classes, posto que não apenas a divisão social do trabalho não a legitima no plano das relações de produção, nem o discurso de dominantes para dominados, de dominados para dominantes, não se entre-reconhecem. Peso da herança da implantação primeva: a ``peça'', o preto escravo, não é um ser social, e sua descendência, passado o momento histórico de sua quase-metamorfose, hibernando secularmente no estágio de larva, não se abre ainda em borboleta.

Subjetividade da objetividade: como reconhecer um ``outro'' nas figuras que vivem da sobra dos banquetes do Corredor da Vitória? São poucos os que chegam de manhã, entram nas fábricas, almoçam a marmita fria, e partem no fim do dia. A maior parte não chega porque não saiu, está sempre ali, para levar um recado, preparar a comida, cuidar dos filhos das madamas e dos \textit{messieurs}, botar uma carta no correio, esperar pela hora do almoço, às vezes curtir uma sesta depois, e repetir o ritual até o fim do dia. Quando então um outro mundo, não reconhecido pelos brancos, começa sua ronda: os terreiros, os candomblés, visto pelos brancos e pelos poderosos como ameaça, caso de polícia, antros de magia e feitiçaria, emergência da desordem na \textit{pax} baiana. No outro pólo: como reconhecer um ``inimigo'', um ``outro'', nas figuras e nas casas onde se come, nos que não cobram honorários, nos que dão às vezes o próprio nome ou sobrenome, para os pretos e os serviçais? Nos que, em não dando o nome ou sobrenome, lhes dão os nomes mais gloriosos, Maria de Jesus, Manoel do Bonfim e lhes abrem as Irmandades? \cite[p.~36-37]{OLIVEIRA1987}.
\end{citacao}

\begin{citacao}
O movimento de constituição das classes passa, em primeiro lugar, pela descoberta, identificação, re-conhecimento do ``inimigo'': o outro é, antes de tudo, um inimigo. Os mitos fazem seu trabalho: se têm os nomes dos que são Senhor e Senhora dos senhores e senhoras, então é-se uma extensão deles, corpo apenas à parte de uma mesma identidade. Mesmo no cruzamento afro-cristão, Yemanjá é uma deusa\dots branca \cite[p.~37]{OLIVEIRA1987}.
\end{citacao}

\begin{citacao}
Uma re-produção que se faz sem necessidade de re-presentar: síntese de uma divisão social do trabalho abortada, resultando dela uma sociedade onde a maior parte dos dominados \textit{na cidade} do Salvador são, rigorosamente, não-explorados: vivem às custas dos banquetes da oligarquia, que por sua vez se alimenta do excedente produzido no cacau, no tabaco. Circulação do excedente de uma espoliação. Onde estão os interesses dos dominados? Não estão certamente na objetividade de uma convergência \textit{antagônica} de interesses, o capital se re-produzindo pela exploração, pela extração de mais-valia, absoluta ou relativa. A aparência é de harmonia.
\cite[p.~37-38]{OLIVEIRA1987}.
\end{citacao}

Por trás de um discurso novidadeiro, como se pode ver, oculta-se a mais ferrenha ortodoxia e o mais encarniçado atraso ideológico.