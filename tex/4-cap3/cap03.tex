\chapter{Brotas e as reformas urbanas da República Velha}\label{cap:3}

Tendo compreendido os conflitos sociais formadores do território do distrito de Brotas em perspectiva sincrônica e diacrônica, é possível, após um longo percurso, focar o escopo da análise na produção, apropriação e uso deste território por parte dos agentes de produção do espaço urbano durante a Primeira República. 

Tratou-se, adiantando desde já parte das conclusões, de um \textit{processo de reorganização espacial} no exato sentido que lhe dá o geógrafo Roberto Lobato Corrêa:

\begin{citacao}
incorporação de novas áreas ao espaço urbano, densificação do uso do solo, deterioração de certas áreas, renovação urbana, relocação diferenciada de infraestrutura e mudança, coercitiva ou não, do conteúdo social e econômico de determinadas áreas da cidade \cite[p.~7]{CORREA1985espa}.
\end{citacao}

Ao longo do período analisado, o que se verá é a transformação de áreas tidas até o ocaso do Império como eminentemente rurais, pontilhadas de fazendas, sítios, chácaras e pequenas roças, em uma área razoavelmente urbanizada tramada pela contraposição entre cidades balneárias e casas de campo para o deleite dos ricos da cidade, de um lado, e de outro as pequenas vilas e avenidas proletárias, casebres, casinholas, conjuntos de casas para aluguel e outras tantas formas de morar dos que sobreviviam às margens da cidade e da sociedade no inferno do pauperismo.

Para mais adequada compreensão do processo nos quase quarenta anos abrangidos pela pesquisa aqui apresentada, ele será segmentado com base nos agentes de produção do espaço urbano que lhes deram impulso. Tais agentes foram divididos duas grandes categorias -- \textit{agentes públicos} e \textit{agentes privados} -- a serem analisadas em separado num primeiro momento, para que se possa depois analisar a ação recíproca entre eles. Entre os agentes públicos, não somente os investimentos públicos serão destacados em pormenor para evidenciar o processo de reorganização espacial a partir da implementação de infraestruturas urbanas básicas -- eletricidade, esgotamento sanitário, abastecimento de água, arruamento, pavimentação, iluminação, transporte público, serviços de saúde, escolas e serviços telefônicos -- como a atuação de órgãos como a Diretoria Municipal de Obras e a Inspetoria Municipal de Higiene no controle do uso do solo, do loteamento das terras e da estética e salubridade dos imóveis será discutido. É com base na atuação de tais órgãos que é levantada a hipótese da existência de um \textit{planejamento urbano informal} na Salvador da Primeira República, já analisado em outros trabalhos \cite{almeida_victoria_1997,almeida_vitrinescomercio_2014}; este planejamento baseava-se tanto na legislação de controle urbanístico -- como o -- quanto em consensos técnicos e estéticos formados entre os engenheiros e médicos integrantes de tais órgãos.

A partir da apresentação, neste capítulo, dos resultados da pesquisa realizada, será possível, na conclusão que se segue como última parte desta dissertação, responder às perguntas que a nortearam.

Antes de prosseguir, é necessário adiantar parte de outra conclusão, desta vez acerca da produção do espaço urbano de Salvador na Primeira República:

\begin{citacao}
Algo a se considerar sobre o proletariado é que o período observado seguiu-se imediatamente à extinção do trabalho escravo. Tal fato pode ter ocasionado uma movimentação da massa liberta, uma parte da qual passaria a habitar zonas próprias, localizáveis principalmente nos distritos de Santo Antônio, Vitória e Brotas. Estes eram os de maiores áreas verdes, aptos, portanto, a ser explorados mais livremente. Ter-se-ia, desta forma, o início de bairros proletários como o da Liberdade (distrito de Santo Antônio). Outra inovação desta etapa é o surgimento de vilas operárias, embora elas não chegassem a se constituir modelo de habitação do proletariado, mesmo porque seu segmento fabril ainda era numericamente pouco significativo no conjunto \cite[pp.~21-22]{santos_habitacao_1990}.
\end{citacao}

A reorganização espacial que se acompanhará neste capítulo guarda as marcas assim descritas. Aquilo que o pesquisador citado levanta como hipótese, embora não plenamente confirmado pelos resultados da presente pesquisa, é encontrado novamente na produção, apropriação e uso do espaço urbano de Brotas durante a Primeira República, reforçando a hipótese por ele suscitada.

\section{Investimentos públicos}\label{sec:3.1}

Primeiramente, é preciso analisar o processo de reorganização espacial do distrito de Brotas no período a partir da instalação de infraestruturas urbanas. A presença de tais equipamentos constitui um indicador seguro da urbanização, na medida em que constituem 

\subsection{Esgotamento e abastecimento de água}\label{subsec:3.1.1}

USAR FALAS DE GOVERNADORES E RELATÓRIOS DE INTENDÊNCIA

\subsection{Arruamento e pavimentação}\label{subsec:3.1.2}

USAR FALAS DE GOVERNADORES E RELATÓRIOS DE INTENDÊNCIA

\begin{citacao}
Do Município desta Capital -- 32.000 metros de 
\end{citacao}

\subsection{Iluminação pública}\label{subsec:3.1.3}

USAR FALAS DE GOVERNADORES E RELATÓRIOS DE INTENDÊNCIA

\subsection{Transporte público}\label{subsec:3.1.4}

USAR FALAS DE GOVERNADORES E RELATÓRIOS DE INTENDÊNCIA

\subsection{Cemitério}\label{subsec:3.1.5}

USAR FALAS DE GOVERNADORES E RELATÓRIOS DA INTENDÊNCIA

DESTACAR OS REITERADOS ELOGIOS À ADMINISTRAÇÃO DO CEMITÉRIO NOS DOCUMENTOS OFICIAIS E OUTROS

\subsection{Serviços de saúde}\label{subsec:3.1.6}

USAR FALAS DE GOVERNADORES E RELATÓRIOS DA INTENDÊNCIA

\subsection{Escolas}\label{subsec:3.1.7}

USAR RELATÓRIOS DA INTENDÊNCIA, QUE DISCRIMINAM AS ESCOLAS

\subsection{Telefonia}\label{subsec:3.1.8}

\begin{citacao}
O serviço telephonico, tanto urbano, como interurbano, também está a cargo da Inspectoria de Viação.
O contracto respectivo para esta Capital provinha de uma concessão feita pela Monarchia, em 1884.
O Governo da Únião, em l924, transferiu suas obrigações e direitos ao Governo do Estado, que fez novo contracto com a \textit{Companhia Brasileira de Energia Eléctrica}, a 26 de Novembro do mesmo anno.
O serviço está actualmente bem montado e com alguma canalização subterrânea.
Há quatro estações: \textit{Central}, \textit{Garcia}, \textit{Roma} e \textit{Rio Vermelho} e por ellas o anno passado se fizeram 62.600 ligações diárias.
Uma nova estação, às Pitangueiras, em Brotas, acaba de ser installada para 120 linhas, com 4 telephonistas.
Encontra-se em remodelação a estação de Roma.
No fim do anno passado existiam 3.221 telephonios e 433 extensões.
A extensão da rêde aérea naquella época era de 24.700 metros e da subterrânea de 24.200, tendo augmentado a primeira no segundo semestre de 300 metros e a segunda
de 1.200, com capacidade para 600 telephonios.
Para a linha de Brotas serão precisos 3.500 metros de linha aérea. \cite[pp.~266-267]{bahia_rpe_1926}
\end{citacao}
\section{Ação privada}\label{sec:acaoprivada}

Paralelamente aos investimentos públicos, é necessário levar em conta a ação de \textit{agentes privados} na produção, apropriação e uso do espaço do distrito. A natureza de tal ação apresentou-se em campo distinto daquele analisado até o momento: enquanto os investimentos públicos estiveram voltados principalmente para a construção de equipamentos coletivos, infraestruturas de transporte, circulação e comunicação etc., os agentes privados apresentaram como prioridade neste distrito, em ordem decrescente de ocorrências, as construções de novos imóveis, as ampliações e reformas de imóveis existentes e os loteamentos. Cada qual será examinado em detalhe, não sem antes se estabelecerem alguns pontos acerca da natureza do uso e do valor da terra no distrito.

\subsection{Engenheiros, arquitetos, construtores e a produção do espaço urbano em Brotas}\label{subsec:gestprodespbrotas}

Já não se admitia durante a Primeira República uma prática ainda hoje comum: a realização de pequenos serviços construtivos ou mesmo a construção de prédios e casas inteiras com base no saber prático de mestres-de-obra, pedreiros e outros artífices. Vinha desde pelo menos o século XIX o enfrentamento na Bahia entre os artífices práticos e suas sociedades mutuárias, de um lado, e de outro os engenheiros, arquitetos e empreiteiros. 

A Junta Administrativa de Obras Públicas, criada pela Lei Provincial nº 91, de 25 de agosto de 1838, é um dos marcos deste conflito; composta por ``engenheiros de todas as classes a quem incorrerá a direção, inspeção, fiscalização e [conservação] de todas as obras públicas da Província'' \cite[pp.~244-245]{REIS2012}, esta junta é a predecessora da Repartição de Obras Públicas da província, ela própria a predecessora imperial da \textit{Diretoria Municipal de Obras e Viação} republicana, que durante o período pesquisado mudará de nome ainda algumas vezes.

Se durante o império o cerco profissional contra os artífices da construção civil na Bahia restringira-se às \textit{obras públicas}, durante os primeiros anos da república ele foi alargado para incluir também as \textit{obras privadas}. Criou-se assim uma \textit{reserva de mercado} benéfica aos profissionais da construção civil com educação universitária, fossem eles engenheiros, agrimensores, arquitetos ou empreiteiros ou quaisquer outros correlatos.

A \textit{Resolução Municipal nº 28}, de 12 de agosto de 1893, estabeleceu pela primeira vez na Salvador republicana a obrigatoriedade de \textit{planta} para a concessão de licença de obra, e é de tão grande importância seu curto texto que merece citação integral:

\begin{citacao}
\textbf{Art. 1º.} Fica a secção de Engenharia do Municipio obrigada a fornecer, mediante despacho do Intendente, planta a quem pretender edificar ou reedificar predios nesta cidade.

\textbf{Art. 2º.} Nesta planta procurará a referida secção conciliar os interesses e planos do particular, quanto possível, com as regras da hygiene e esthetica de accordo com a largura e amplitude das ruas e sua posição topographica.

\textbf{Art. 3º.} O proprietário pagará pela planta no acto da concessão da licença de 10\$000 a 200\$000 a titulo de emolumentos.

\textbf{Art. 4º.} O superintendente geral das obras municipaes organisará uma tabella para cobrança dentro daquelles limites de accordo com a extensão e valor do prédio a edificar.

\textbf{Art. 5º.} Revogam-se as disposições em contrario.
\end{citacao}

Com esta exigência, todo projeto de construção ou reforma a ser realizado em Salvador daí em diante passou não apenas a condicionar-se à ingerência de engenheiros para sua realização, como também, dada a falta de qualquer critério mais objetivo acerca da matéria, a submeter sua própria estética ao gosto dos engenheiros municipais.

Esta é a base dos \textit{pedidos de licença} para construção e reforma que compõem o núcleo desta pesquisa. Seu funcionamento era razoavelmente simples. Qualquer interessado em construir, ampliar, reformar, lotear ou parcelar imóveis precisava dirigir-se à Câmara Municipal com a planta a obra pretendida para que a Portaria da Câmara Municipal reduzisse o pedido a termo, indicando a natureza do pedido (construção, ampliação, reforma, loteamento ou parcelamento) e a localização do imóvel. Instaurava-se então um processo administrativo, encaminhado em primeiro lugar à Diretoria Municipal de Obras e Viação, onde os aspectos contrutivos e arquitetônicos seriam avaliados, em especial o alinhamento da obra com a rua, disciplinado por plantas de área disponíveis aos burocratas. Havendo parecer positivo, o processo era então encaminhado para a Inspetoria Municipal de Higiene, que dava outro parecer relativo às medidas higiênicas e sanitárias básicas necessárias para a construção; com a promulgação do Código Sanitário da Bahia em 1924, esta atribuição passou à Diretoria Estadual de Higiene (como o fora durante o Império), mas no que diz respeito à tramitação do processo nada se alterou. Estando tudo de acordo com as posturas municipais e com o \textit{Regulamento Sanitário Municipal} (Lei Municipal 797, de 28 jun. 1906), ou posteriormente com o \textit{Código de Posturas Municipais} (Ato 127, de 05 nov. 1920) e com o \textit{Código de Obras} (Lei Municipal 1.146, de 19 jun. 1926), a obra estava liberada.

Com tal reserva de mercado, não era de estranhar o aumento da demanda por engenheiros. É certamente este um dos fundamentos, entre tantos, para o início da mobilização de engenheiros em prol da fundação do Instituto Politécnico da Bahia em 12 de julho de 1896 e da Escola Politécnica da Bahia em 14 de março de 1897 \cite[pp.~9-11]{costa_politecnica_2005}. 

\subsubsection{Cenários da reserva de mercado}

Tendo isto em vista, é preciso descer ao miúdo, dar nomes aos bois. Foi recolhido na documentação pesquisada o quadro de profissionais atuantes no distrito (cf. \autoref{tab:engenheiros}, p. \pageref{tab:engenheiros}).

\afterpage{
\begin{a3paisagem}
\begin{table}[!htp]
\IBGEtab{
\caption{Engenheiros atuantes em Brotas e suas obras, por grupo de logradouros (parte 1))}\label{tab:engenheiros}}
{
\begin{tiny}
\begin{tabular}{lllllllllllll}
\toprule
Nome	&Antigo 1º Distrito	&Boa Vista / Engenho Velho	&Estrada de Brotas	&Estrada 2 de Julho	&Mariquita	&Matatu	&Acupe	&Campinas	&Alagoa-Pituba	&Armações / Várzea	&TOTAL	&\%\\
\midrule
\midrule
A. Carneiro da Rocha	&2	&1	&0	&0	&0	&1	&0	&0	&0	&0	&4	&0,66\\
A. J. de Souza Carneiro	&6	&6	&0	&0	&0	&1	&3	&0	&1	&0	&17	&2,81\\
Alberto Silva	&0	&0	&0	&2	&0	&0	&0	&0	&0	&0	&2	&0,33\\
Alfredo Vieira de Almeida	&4	&0	&0	&0	&0	&1	&0	&0	&0	&0	&5	&0,83\\
Allioni \& Cia.	&1	&0	&0	&0	&0	&0	&0	&0	&0	&0	&1	&0,17\\
André Saffrey	&0	&0	&0	&0	&0	&0	&0	&0	&2	&0	&2	&0,33\\
Antonio Augusto Machado	&2	&0	&0	&0	&0	&0	&0	&0	&0	&0	&2	&0,33\\
Antonio dos Santos	&0	&2	&0	&0	&0	&0	&0	&0	&0	&0	&2	&0,33\\
Antonio Ferrão Marques	&0	&0	&0	&0	&0	&1	&0	&0	&0	&0	&1	&0,17\\
Antonio Gonçalves	&0	&1	&0	&0	&0	&0	&0	&0	&0	&0	&1	&0,17\\
Antonio Leite da Luz	&6	&0	&3	&1	&2	&3	&1	&0	&0	&0	&16	&2,64\\
Antonio Lopes Rodrigues	&0	&2	&0	&0	&1	&0	&0	&0	&0	&0	&3	&0,50\\
Antonio Valentim Ferreira	&0	&1	&0	&0	&0	&0	&0	&0	&1	&0	&2	&0,33\\
Archimedes Marques	&16	&27	&1	&3	&2	&8	&2	&0	&9	&0	&68	&11,24\\
Arthur Santos	&14	&35	&6	&3	&4	&5	&0	&0	&6	&0	&73	&12,07\\
Barroso de Souza	&2	&1	&1	&0	&0	&0	&0	&0	&0	&0	&4	&0,66\\
Biaggio Bianco	&1	&0	&0	&0	&0	&0	&0	&0	&0	&0	&1	&0,17\\
Carlos Faria	&0	&0	&0	&0	&0	&0	&0	&0	&3	&0	&3	&0,50\\
Carlos Peixoto	&8	&4	&0	&1	&0	&1	&0	&0	&0	&0	&14	&2,31\\
Carlos Souza	&1	&7	&0	&0	&0	&2	&4	&0	&1	&0	&15	&2,48\\
Custódio Bandeira	&10	&21	&10	&5	&2	&1	&3	&0	&11	&0	&63	&10,41\\
Durval Fernandes	&0	&1	&0	&0	&0	&1	&0	&0	&0	&0	&2	&0,33\\
Durval Lima	&0	&1	&0	&0	&0	&0	&0	&0	&0	&0	&1	&0,17\\
Durval Neves da Rocha	&2	&0	&0	&0	&0	&1	&0	&0	&0	&0	&3	&0,50\\
Eduardo dos Santos Corrêa	&0	&0	&0	&0	&0	&0	&0	&0	&1	&0	&1	&0,17\\
Ernestino dos Santos Marques	&1	&1	&0	&1	&0	&0	&0	&0	&0	&0	&3	&0,50\\
Esmeraldo Coelho	&0	&0	&0	&0	&0	&1	&0	&0	&0	&0	&1	&0,17\\
Eurico da Costa Coutinho	&3	&1	&0	&0	&0	&0	&0	&0	&1	&0	&5	&0,83\\
F. Sampaio	&1	&0	&0	&0	&0	&0	&0	&0	&0	&0	&1	&0,17\\
Fillippe Silva	&0	&0	&0	&0	&0	&1	&1	&0	&0	&0	&2	&0,33\\
Francisco A. W. Silva	&0	&0	&0	&1	&0	&0	&0	&0	&0	&0	&1	&0,17\\
Francisco Martins	&0	&2	&0	&0	&0	&2	&0	&0	&1	&0	&5	&0,83\\
Francisco Theodoro Pereira das Neves	&0	&0	&0	&0	&0	&0	&0	&0	&1	&0	&1	&0,17\\
Frederico Saraiva	&0	&1	&1	&0	&0	&0	&0	&0	&1	&0	&3	&0,50\\
Frederico Theodoro Sampaio	&0	&0	&0	&0	&1	&0	&0	&0	&0	&0	&1	&0,17\\
Gustavo Pereira Santos	&2	&0	&0	&0	&0	&0	&0	&0	&0	&0	&2	&0,33\\
J. B. Vasconcellos	&0	&1	&0	&0	&0	&0	&0	&0	&0	&0	&1	&0,17\\
J. Barroso	&4	&3	&3	&2	&4	&4	&0	&0	&2	&0	&22	&3,64\\
J. Castro	&0	&0	&0	&1	&0	&0	&0	&0	&0	&0	&1	&0,17\\
J. Cyrillo de Souza	&0	&0	&0	&0	&0	&1	&0	&0	&0	&0	&1	&0,17\\
Jayme Bastos	&0	&2	&0	&0	&0	&0	&0	&0	&0	&0	&2	&0,33\\
Jayme Cerqueira Lima	&1	&1	&1	&0	&0	&1	&0	&0	&1	&0	&5	&0,83\\
M. Martins	&0	&0	&0	&1	&0	&0	&0	&0	&0	&0	&1	&0,17\\
João dos Santos Tuvo	&0	&0	&1	&0	&0	&0	&0	&0	&0	&0	&1	&0,17\\
João Pimenta Bastos Filho	&1	&0	&1	&1	&0	&0	&0	&0	&1	&0	&4	&0,66\\
Joaquim de Oliveira Júnior	&0	&1	&1	&0	&0	&1	&0	&0	&0	&0	&3	&0,50\\
Joaquim José Ribeiro d?Oliveira	&0	&0	&0	&1	&0	&0	&0	&0	&0	&0	&1	&0,17\\
José Allioni	&0	&0	&0	&2	&0	&0	&0	&0	&0	&0	&2	&0,33\\
José Celestino dos Santos	&2	&1	&0	&0	&1	&2	&0	&0	&1	&0	&7	&1,16\\
José Portella Passos	&1	&0	&0	&0	&2	&1	&0	&0	&1	&0	&5	&0,83\\
Justo J. David	&1	&0	&0	&0	&0	&0	&0	&0	&0	&0	&1	&0,17\\
Júlio Viveiros Brandão	&0	&0	&0	&0	&1	&0	&0	&0	&0	&0	&1	&0,17\\
Lamartine Portella Passos	&4	&3	&0	&0	&1	&0	&0	&0	&0	&0	&8	&1,32\\
Lopes Lima	&0	&1	&0	&0	&0	&1	&0	&0	&0	&0	&2	&0,33\\
Luiz Affonso de Sá	&0	&1	&0	&0	&0	&0	&0	&0	&1	&0	&2	&0,33\\
Luiz de Moura Bastos	&1	&2	&0	&2	&0	&1	&0	&0	&0	&0	&6	&0,99\\
Lycerio Alfredo Schreiner	&4	&5	&0	&1	&0	&5	&5	&0	&1	&0	&21	&3,47\\




M. Martins	&0	&1	&1	&0	&0	&0	&0	&0	&0	&0	&2	&0,33\\
Manoel R. F. Muniz	&3	&4	&1	&0	&1	&3	&0	&0	&0	&0	&12	&1,98\\
Manuel Querino	&0	&0	&0	&0	&0	&1	&0	&0	&0	&0	&1	&0,17\\
Mario de Souza Dias	&1	&0	&0	&0	&0	&3	&0	&0	&1	&0	&5	&0,83\\
Nogueira Passos	&0	&0	&0	&0	&0	&1	&0	&0	&0	&0	&1	&0,17\\
Oswaldo Gonçalves Martins	&3	&0	&1	&0	&0	&1	&6	&0	&0	&0	&11	&1,82\\
Pedro Jayme David	&23	&18	&4	&5	&1	&14	&4	&0	&10	&0	&79	&13,06\\
Quirino da Costa Coutinho	&0	&0	&0	&0	&1	&0	&0	&0	&0	&0	&1	&0,17\\
Rogério Baptista	&1	&0	&0	&0	&0	&0	&0	&0	&0	&0	&1	&0,17\\
Rosalvo Celestino dos Santos	&7	&12	&4	&2	&0	&5	&0	&0	&4	&0	&34	&5,62\\
Rossi Baptista	&1	&0	&0	&0	&0	&0	&0	&0	&0	&0	&1	&0,17\\
S. Lellis	&0	&0	&0	&0	&0	&0	&1	&0	&0	&0	&1	&0,17\\
Satyro Brandão	&0	&0	&0	&0	&0	&0	&1	&0	&0	&0	&1	&0,17\\
Urbano Rossi	&0	&2	&0	&0	&0	&0	&0	&0	&0	&0	&2	&0,33\\
Victorino d'Almeida	&2	&0	&0	&1	&0	&1	&1	&0	&0	&0	&5	&0,83\\
Victorio Joaquim de Meirelles	&2	&7	&3	&3	&1	&1	&1	&0	&2	&0	&20	&3,31\\
\midrule
TOTAL	&144	&180	&43	&39	&25	&77	&33	&0	&64	&0	&605	&100,00\\
\bottomrule
\end{tabular} 
\end{tiny}
}
{\fonte{Elaboração do autor, com base em \textbf{BR BAAHMS}, Fundo ``Intendência e Prefeitura'', Série ``Processos de Licenciamento de Reforma e Ampliação de Edificações'', Subsérie ``Requerimentos e Plantas -- Brotas'', todas as caixas e processos.}}
\end{table}

\end{a3paisagem}
}

Poucos entre os profissionais listados encontram-se nas listas de formados pela Escola Politécnica da Bahia: Antônio Joaquim (A. J.) de Souza Carneiro (turma de 1903)\footnote{Para registro, trata-se do pai do etnólogo e folclorista Edison Carneiro (1912-1972) e do político udenista Nelson de Souza Carneiro (1910-1996).}, Eurico da Costa Coutinho (turma de 1907), Carlos da Silva e Souza (turma de 1908), Durval Neves da Rocha (turma de 1916), Lamartine Portella Passos (turma de 1919) e Luiz de Moura Bastos (turma de 1925). 

A escassez de escolas de engenharia no Brasil levava os poucos nelas formados a espraiar-se pelas cidades em busca de obras; como o fenômeno do assalariamento massivo de profissionais com formação bacharelesca veio a se dar muito mais tarde, quase cinquenta anos à frente, cabia a estes jovens engenheiros muito mais o papel de \textit{empreiteiros} e \textit{construtores}, em especial no setor de obras públicas, que o de suplicantes de currículo à mão. É possível que muitos deles sejam como \textit{Lycerio Alfredo Schreiner}, engenheiro formado em 1920 pela Escola de Engenharia de Porto Alegre \cite{lersch_engenhariapoa_2014}, radicado em Salvador e mais conhecido pelas suas realizações como diretor da Escola de Aprendizes Artífices da Bahia entre 1926 e 1939 \cite{moreira_escolaartifices_2009} e, anos depois, como integrante da comissão interministerial para a educação profissional instituída pelo Decreto 1.238, de 2 de maio de 1939 \cite[p.~184]{almeida_ensinoindustrial_2010}. 

É possível ainda que muitos entre estes profissionais -- não é possível dizer ao certo quantos sem pesquisas que fugiriam ao escopo do presente estudo -- fossem ``desenhistas'' formados no Liceu de Artes e Ofícios, ``agrimensores'' ou  ``agrônomos'' formados na Escola Agrícola da Bahia, ``arquitetos'' formados na Escola de Belas-Artes da Bahia. Apesar do cerco profissional a que estavam submetidos por força de lei, parece ter havido alguma sensibilidade por parte da Intendência em 


Entre estes profissionais bacharelescos, destaca-se como caso extremo da circulação entre o público e o privado o engenheiro, arquiteto e construtor -- ele mesmo variava sua qualificação profissional de acordo com a função que exercia no momento -- \textit{Pedro Jayme David}.
 
\textit{Pedro Jayme David} (?-1925) era filho do agrimensor \textit{Pedro Julio David} (1839-1905), conhecido pela participação na construção do elevador Lacerda e da ladeira da Montanha; pela colaboração no projeto de abertura das estradas do Retiro (muito provavelmente o trecho da antiga rua da Vala correspondente às atuais avenidas Heitor Dias e Barros Reis) e do Rio Vermelho (hoje avenida Cardeal da Silva); pelo projeto, construção e fiscalização do Mercado do Ouro; por vários projetos de açudes de estradas no interior da Bahia; e por ser condutor de obras públicas, aposentado em 1896 depois de trinta anos de serviço \cite[pp.~327-328]{querino_artistas_2018}. Um \textit{portfolio} de peso, indicador de participação ativa no meio político e na alta sociedade baiana.

Não há notícias da data de nascimento de Pedro Jayme David. Em agosto de 1878, entretanto, encontramo-lo na ``1ª aula'' de francês do professor Henrique Monat com notas medianas\footnote{\textbf{O Atheneu Bahiano}, nº 6, ago. 1878, p. 15}. Às benesses do berço somou-se um verniz de cultura em meio a uma sociedade de pouquíssimos letrados, dando a entender uma cultura poliglota e quiçá cosmopolita (para os padrões da época) que o capacitava, em tese, a circular entre os escalões mais altos da sociedade baiana. Dito e feito: onze anos depois, em 1889, Pedro Jayme David já circulava na alta sociedade baiana, sendo encontrado em meio à membresia do Derby Club do Rio Vermelho ora como ``juiz de encilhamento'', ora como integrante da ``Commissão de Policia''\footnote{\textbf{Diário do Povo}, edições de 21 maio 1889, 29 maio 1889, 07 jun. 1889, 17 jun. 1889, 18 jun. 1889 e 22 jun. 1889.}. 

Não foi possível localizar onde Pedro Jayme David bacharelou-se, mas nos requerimentos de licença para obras mais antigos encontrados no \textbf{BR-BAAHMS}, de 1893, ele já despachava como parecerista, e num almanaque de 1898 descobre-se que ele era então ``ajudante'' da ``secção de engenharia'' da Intendência Municipal de Salvador, subordinado ao diretor de obras Francisco Lopes da Silva Lima e trabalhando junto com o ``condutor de obras'' Manoel Alves Nazareth \cite[p.~278]{reis_almanak_1898}; aí também indica-se sua residência como sendo no ``Campo dos Martyres'', antiga denominação do Campo da Pólvora, endereço indicado como seu escritório profissional nos carimbos com que assinava as plantas em projetos de obras\footnote{\textbf{BR-BAAHMS}, Fundos ``Intendência'' e ``Prefeitura'', Série ``Processos de Licenciamento de Reforma e Ampliação de Edificações'', Subsérie ``Requerimentos e Plantas -- Brotas'', vários documentos nas caixas 1 a 24.}. Em 1920, quando da aposentadoria de Francisco Lopes da Silva Lima e consequente vacância da Diretoria de Obras Públicas do município, Pedro Jayme David foi um dos concorrentes à vaga, e sua preterição na indicação ao cargo gerou protestos vivos na imprensa por se tratar do ``mais antigo empregado daquella repartição, que varias vezes occupou interinamente o logar de director e tem 28 annos de serviços''\footnote{\textbf{A Manhã}, ano I, nº 74, p. 1.}. Desde conjunto pode-se inferir que Pedro Jayme David tornou-se um burocrata na Intendência Municipal de Salvador muito provavelmente em 1892, que em 1893 já era atuante e que em 1898 tal \textit{status} era confirmado pela imprensa.

Pedro Jayme David era \textit{ao mesmo tempo} um \textit{burocrata de carreira} e um \textit{empreiteiro}. Ao que tudo indica, ele pode ter sido o \textit{único} entre os profissionais vinculados à Diretoria de Obras da Intendência de Salvador a viver este duplo papel de forma tão ostensiva. Em 1º de julho de 1891 Pedro Jayme David recebeu junto com João José Vaz, Américo de Freitas e Joaquim dos Santos Correia privilégio por trinta anos para construir e explorar uma estrada de ferro entre Passé, Candeias, e a estação Jacuípe da estrada de ferro de Santo Amaro\footnote{\textbf{Pequeno Jornal}, ano II, nº 402, p. 2.}. A julgar pelo ano mais provável em que Pedro Jayme David se tornou um burocrata, poder-se-ia muito facilmente argumentar que ele poderia ter saído do quadro societário para dedicar-se ao serviço público; se esta é uma hipótese plausível, pois não há qualquer notícia acerca de sua participação nesta sociedade depois de 1891, não é este o padrão de comportamento encontrado na restante documentação pesquisada. 

Em 4 de abril de 1899 -- seis a sete anos depois da data provável de entrada de Pedro Jayme David no serviço público -- este engenheiro apareceu numa ata eleitoral da Companhia União Fabril como suplente do conselho fiscal\footnote{\textbf{Cidade do Salvador}, ano III, nº 48, p. 1.}. Em 1910 encontramo-lo como alvo da catilinária de ninguém menos que seu colega de profissão, Theodoro Sampaio, envolvendo os planos para o abastecimento de água e esgoto de Salvador – uma querela de grande importância para os destinos da cidade, perdida numa troca de ofensas bem ao estilo da época, de envergonhar mesmo os mais boquirrotos \cite{sampaio_agua_1910}. Pedro
Jayme David chegou ao absurdo de aprovar, enquanto membro da Diretoria Municipal de Obras, um projeto de sua própria autoria! INSERIR A REFERÊNCIA

A sorte de Pedro Jayme David como construtor, todavia, não foi sempre das melhores. Um parecer de 1928 num pedido de licença de obra diz ser ele “o mesmo envolvido no caso do desabamento do Monte Serrat”\footnote{\textbf{BR-BAAHMS}, Fundo “Intendência”, Série “Processos de Licenciamento de Reforma e Ampliação de Edificações”, Subsérie “Requerimentos e Plantas – Brotas”, caixa 13, processo 1710 folha 254.}, do qual não foi possível encontrar vestígio nos diários e semanários pesquisados. Outro pedido de licença apresenta planta com seu nome riscado e substituído pelo de Oswaldo Gonçalves Martins, fato ressaltado pelo parecerista Mário Teixeira Rodrigues Lima por trazer agora “assinatura de construtor registrado neste departamento”\footnote{\textbf{BR-BAAHMS}, Fundo “Intendência”, Série “Processos de Licenciamento de Reforma e Ampliação de Edificações”, Subsérie “Requerimentos e Plantas – Brotas”, caixa 13, processo 1753 folha 31, datado de 03 jun. 1928.}. Preterido na seleção para a titular da Diretoria de Obras da Intendência em 1920, Pedro Jayme David parece ter em seguida saído do serviço público; é muito provável que este desabamento no Monte Serrat tenha marcado sua carreira ao ponto de ser também retirado da lista de construtores autorizados pela Diretoria de Obras.

Em 13 de dezembro de 1928 chegou à Intendência um requerimento, de autoria de Elísio Alves dos Santos, de substituição do projetista; ali se informava que Pedro Jayme David havia \textit{morrido}, e pedia-se sua substituição\footnote{\textbf{BR-BAAHMS}, Fundo “Intendência”, Série “Processos de Licenciamento de Reforma e Ampliação de Edificações”, Subsérie “Requerimentos e Plantas – Brotas”, caixa 20, processo 3268 folha 164, datado de 13 dez. 1928. José Allionni foi o construtor indicado neste caso. Outro documento de teor semelhante protocolado por Alexandre Alves da Silva encontra-se na mesma caixa, processo 3505 folha 70, datado de 31 dez 1928.}. Outras obras sob responsabilidade passaram para a responsabilidade de vários profissionais, talvez do seu círculo de relações, como Lycerio Alfredo Schreiner\footnote{\textbf{BR-BAAHMS}, Fundo “Intendência”, Série “Processos de Licenciamento de Reforma e Ampliação de Edificações”, Subsérie “Requerimentos e Plantas – Brotas”, caixa 20, processo 1925 folha 35.}.

Burocrata, empreiteiro, capitalista, Pedro Jayme David parece ter aproveitado bem a situação ambígua em que se encontrava. Pesquisas futuras poderão indicar o impacto dos projetos de Pedro Jayme David nos demais distritos urbanos e suburbanos de Salvador, mas em Brotas ele é o engenheiro mais cotado: encontram-se 79 projetos de sua autoria neste distrito entre 1893 e 1928, ano provável de sua morte -- ou seja, entre 2 a 3 projetos por ano \textit{somente em Brotas}, pois ele era profissional ativo também em distritos como Vitória, Pilar e Conceição da Praia \cite{almeida_victoria_1997, almeida_vitrinescomercio_2014}. Não há uma só entre as áreas urbanizadas ou em processo de urbanização em Brotas durante o período estudado que não tenha um projeto de sua autoria. Difícil achar mesmo uma rua que não tenha obra sob sua responsabilidade. Isto pressupõe algum nível de divisão de trabalho, algum nível de coordenação da cooperação entre trabalhadores distintos, alguma capacidade para assumir projetos -- em suma, isto pressupõe uma \textit{empreiteira} muito bem organizada, ou ao menos muito bons contatos com mestres-de-obra e sua coorte de alveneiros, pedreiros, marceneiros, marmoristas, pintores, ceramistas, encanadores, estucadores, carapinas, torneiros, oleiros, calceteiros, canteiros, funileiros, vidraceiros, gesseiros, entalhadores, ferreiros, ladrilheiros,  forjadores, serralheiros, taipeiros \dots

Uma hipótese para tamanho volume de trabalho, de difícil comprovação documental mas nem por isto menos plausível, é razoavelmente simples, e forma-se pelo encadeamento de outras hipóteses prévias com fatos bem estabelecidos pela documentação pesquisada. 

Como se verá na \autoref{subsec:constrampliref} (\pageref{subsec:constrampliref}), qualquer interessado em construir, ampliar, reformar, lotear ou parcelar imóveis precisava dirigir-se à Intendência Municipal para solicitar licença para a obra. Assim como havia, decerto, uma clientela bem situada socialmente e ávida pelas novidades arquitetônicas, entre outras razões por simbolizarem bem-estar material e distinção social, encontravam-se também em meio à clientela dos engenheiros, arquitetos e construtores no período pessoas que queriam simplesmente levantar um casebre simples, ou reformar uma parede arruinada, ou trocar um telhado etc., e que por força da \textit{reserva de mercado} imposta pela Resolução Municipal nº 28/1893 e legislação urbanística subsequente viam-se \textit{obrigados} a contratar os serviços de um engenheiro.

Ora, esta clientela que se via subitamente obrigada a contratar os serviços de engenheiros e arquitetos decerto não tinha o mesmo perfil daqueles que os procuravam voluntariamente. Talvez sequer soubesse onde encontrá-los, ou mesmo não soubesse ler os almanaques onde a cada ano anunciavam seus serviços e endereços profissionais e residenciais. A Diretoria Municipal de Obras, num tal contexto, não era somente um ponto de referência para estes clientes involuntários; pelo que se pôde verificar em diversos pareceres nos processos de licença de obras consultados, o órgão mantinha uma lista de engenheiros e construtores \textit{autorizados por ela} a trabalhar no mercado de obras soteropolitano. Era muito provável, portanto, que na própria portaria da Intendência Municipal qualquer um pudesse se informar sobre onde encontrar um engenheiro ou construtor; é igualmente provável que nosso burocrata-empreiteiro Pedro Jayme David beneficiasse desta clientela involuntária.

Trata-se, entretanto, de um caso extremo. A documentação pesquisada e a bibliografia consultada indicam que não parece ser esta a carreira profissional mais comum entre os bachareis da construção civil em Salvador.

Veja-se, por contraste, a postura de \textit{Manoel Alves Nazareth}, companheiro de trabalho de Pedro Jayme David por longos anos na Diretoria de Obras e Viação. Um dos filhos do ``ensaiador de metais'' Ignacio Alves Nazareth, reputado à época de seu falecimento em 1899 como ``ultimo que restava da geração patriota que pugnou, nos campos de batalha, pela independencia bahiana'' \cite{apontamentos_1899}, Manoel Alves Nazareth formou-se engenheiro agrônomo pela \textit{Escola Agrícola da Bahia} em 1885 \cite[p.~141]{araujo_agronomia_2010}, e desde 1895 pelo menos encontrava-se lotado na Diretoria de Obras e Viação da Intendência Municipal de Salvador, pois data deste ano o primeiro registro escrito encontrado na documentação pesquisada a mencionar seu nome \cite{salvador_relatorio_1895}. 

Não foi possível encontrar nos pedidos de licença de obra pesquisados nem um só projeto creditado a Manoel Alves Nazareth, nem tampouco seu nome aparece na imprensa baiana da época citado em qualquer circunstância, como a indicar uma vida de certo modo privada, ausente dos eventos sociais e políticos de maior destaque da vida pública soteropolitana de seu tempo. Sua residência na estrada da Cruz das Almas torna-o figura de ainda maior interesse para esta pesquisa: sendo ao mesmo tempo morador do distrito de Brotas e ``condutor de obras'' da Intendência Municipal, reunia assim elementos suficientes para fazer voltar a seu favor, e a de sua pequena residência afastada da zona urbana, os benefícios da ``modernidade''. O que se viu foi o contrário: mais uma vez, não há qualquer registro na imprensa baiana coetânea de qualquer manifestação pública de Manoel Alves Nazareth; nenhuma das tantas e quantas cartas abertas, manifestos, desagravos e libelos autorais que coloriram as páginas dos diários e hebdomadários baianos levou sua assinatura. Os pareceres de Manoel Alves Nazareth são via de regra bastante protocolares, repetitivos, monótonos até. Por outro lado, é dele o parecer mandando suspender todas as concessões de licença em terrenos de José Visco porque \footnote{\textbf{BR-BAAHMS}, Fundo ``Intendência e Prefeitura'', Série ``Processos de Licenciamento de Reforma e Ampliação de Edificações'', Subsérie ``Requerimentos e Plantas -- Brotas'', caixa 22, processo 2344 folha 110, de 30 maio 1905.}; é dele também o único parecer onde um funcionário da Diretoria de Obras e Viação \textit{declina de seu poder}, pois ``tratou-se de um alinhamento no local onde sou proprietário e portanto suspeito para resolver sobre o assunto''\footnote{\textbf{BR-BAAHMS}, Fundo ``Intendência e Prefeitura'', Série ``Processos de Licenciamento de Reforma e Ampliação de Edificações'', Subsérie ``Requerimentos e Plantas -- Brotas'', caixa 22, processo 2344 folha 110, de 30 maio 1905.}

Por outro lado, o próprio Pedro Jayme David disputava palmo a palmo a clientela com outros cinco profissionais cuja trajetória pode ajudar a explicar a situação: trata-se, por ordem de volume de obras, de \textit{Arthur Santos}, \textit{Archimedes Marques}, \textit{Custódio Bandeira}, \textit{Rosalvo Celestino dos Santos} e \textit{Ernestino dos Santos Marques}. Custódio Bandeira, a julgar pela sua \textit{performance} em meio ao universo pesquisado, é um empreiteiro sem maiores distinções, mas os outros quatro têm características que os destacam do conjunto.

\textit{Arthur Santos}, ao que tudo indica, era o mesmo vinculado ao \textit{Centro Operário}\footnote{\textbf{A Voz do Operário}, ano I, nº 1, 02 jan. 1894, p. 2.} e ao \textit{Liceu de Artes e Ofícios}. \textit{Archimedes Marques}, ao que tudo indica, era o mesmo que anos à frente foi noticiado como integrante do ``Syndicato de Officios Varios e Mutua União Operaria''\footnote{\textbf{O Imparcial}, ano I, nº 1377, 01 jul. 1935, p. 8.}. Não bastasse o envolvimento com o nascente movimento operário, desenvolvimento histórico durante o período republicano das formas associativas e mutualistas de apoio mútuo e expressão política inventadas pelos negros escravizados durante o Império, a clientela destes dois profissionais é bem demarcada. No Engenho Velho / Boa Vista, lugar de urbanização proletária como se verá adiante (\autoref{subsec:constrampliref}, p. \pageref{subsec:constrampliref}), são eles os principais autores de projetos; ali Arthur Santos chega a quase dobrar o número de projetos de Pedro Jayme David (35 projetos do primeiro contra 18 do segundo), e Archimedes Marques chega perto (27 projetos contra 18 de David).

\textit{Rosalvo Celestino dos Santos} e \textit{Ernestino dos Santos Marques} são dois casos à parte. O primeiro é ; o segundo é um \textit{desenhista} da ``Secção de Engenharia'' da Intendência Municipal, residente na rua das Pitangueiras \cite[p.~279]{reis_almanak_1898}, que em 1915 chegou a ser lotado interinamente como \textit{agrimensor} da Intendência\footnote{\textbf{A Notícia}, ano I, nº 165, 8 abr. 1915, p. 1}. São \textit{técnicos}, portanto, não \textit{bachareis}; foi impossível situar sua formação em meio ao universo documental e bibliográfico pesquisado, mas não é improvável que sejam artífices desenhistas ou agrimensores. Os projetos dos dois em Brotas são de uma simplicidade desadornada, quase rústica, via de regra situados em ruas de baixo valor locativo médio e voltados a pessoas cujos requerimentos são assinados a rogo (indicando analfabetismo). É altíssima a probabilidade de serem \textit{projetistas para pobres}, funcionários públicos colocados à disposição daqueles que de outro modo não teriam como arcar com o valor de um engenheiro ou arquiteto que lhes desenhasse o projeto e lhes coordenasse a obra posteriormente. Faziam com isto concorrência a Pedro Jayme David, mas o maior volume de projetos assinados por este último talvez se explique pela sua política de ``porta giratória'' entre o público e o privado, enquanto aos dois caberia talvez apenas o trabalho suplementar ao dos engenheiros por indicação da Intendência.

\subsubsection{Os gestores, a higiene, a estética e a polícia arquitetônica}

Por que insistir tanto no papel dos engenheiros, arquitetos e demais construtores? Porque são eles os responsáveis pela \textit{estética} de uma construção. Aqueles encarregados por dever de ofício de fiscalizar sua execução em nome da Intendência e da Prefeitura têm peso ainda maior, pois podiam, por força do artigo 2 o da Resolução Municipal 28/1893 e dos artigos 3 o e 4 o da Resolução Municipal 160/1905, interferir nos menores detalhes de uma planta a bem da “hygiene e esthetica” e em favor de “typos” de construção “que não poderão ser modificados em sua essencia, salvo quando a construcção obedecer a um estylo especial perfeitamente conhecido”. Não bastava as normas urbanísticas terem zoneado a cidade; os pareceristas da Diretoria de Obras e da Inspetoria de Higiene recebiam destas resoluções municipais o poder de \textit{normalizar} a aparência das construções e evitar as extravagâncias, de um lado, e a modéstia espartana, de outro. Formavam uma verdadeira \textit{polícia arquitetônica}, cuja atuação em Brotas se verá a seguir. 

Veja-se como exemplo desta “polícia arquitetônica” a situação do Engenho Velho / Boa Vista, onde houve muitos pareceres interventivos. Pedro Jayme David informou em março de 1900 que “a Rua do Asylo [ou seja, a rua da Boa Vista] é uma das mais largas (18,5m) desta capital”, e para ele “as propriedades ali construídas devem aparesentar igual elegância''\footnote{\textbf{BR-BAAHMS}, Fundo “Intendência”, Série “Processos de Licenciamento de Reforma e Ampliação de Edificações”, Subsérie “Requerimentos e Plantas – Brotas”, caixa 1, documento de 25 set. 1925.}, e a existência nesta rua de um pedido de licença cujo projeto foi desenhado por Urbano Rossi reforça tal impressão\footnote{\textbf{BR-BAAHMS}, Fundo “Intendência”, Série “Processos de Licenciamento de Reforma e Ampliação de Edificações”, Subsérie “Requerimentos e Plantas – Brotas”, caixa 2, processo 549, de 05 dez. 1913.}. Apesar disto, a regra eram alterações mínimas nos projetos, quando as havia. Porciúncula Maria do Espírito Santo, que solicitara em 1897 a construção de uma casa na área, teve seu pedido indeferido porque o projeto “não tem as dimensões exigidas”, o que seria sanado, segundo parecer do diretor de obras Francisco Lopes da Silva Lima, pela assinatura de um termo de responsabilidade em que a requerente se obrigaria a ajustar as dimensões da casa àquelas exigidas pela Diretoria de Obras\footnote{\textbf{BR-BAAHMS}, Fundo “Intendência”, Série “Processos de Licenciamento de Reforma e Ampliação de Edificações”, Subsérie “Requerimentos e Plantas – Brotas”, caixa 5. O documento não tem número de identificação, mas está datado como sendo de 9 set. 1897.}. Manoel Euphrazino da Paixão foi obrigado em 1913 a “adicionar claraboia na parte correspondente dos cômodos” da casa que pretendia construir\footnote{BR-BAAHMS, Fundo “Intendência”, Série “Processos de Licenciamento de Reforma e Ampliação de Edificações”, Subsérie “Requerimentos e Plantas – Brotas”, caixa 5, processo 503 folha 248.}. João Cardoso dos Santos teve de ajustar a largura da área lateral da casa que pretendia construir para que tivesse 1,5m\footnote{\textbf{BR-BAAHMS}, Fundo “Intendência”, Série “Processos de Licenciamento de Reforma e Ampliação de Edificações”, Subsérie “Requerimentos e Plantas – Brotas”, caixa 5, processo 2435 folha 266.}. André Argolo, em 1925, teve de ajustar o pé-direito a casa que pretendia construir na rua do Pepino, indicado pelo parecerista João dos Santos Tuvo como sendo de 3,5m\footnote{\textbf{BR-BAAHMS}, Fundo “Intendência”, Série “Processos de Licenciamento de Reforma e Ampliação de Edificações”, Subsérie “Requerimentos e Plantas – Brotas”, caixa 1, documento de 25 set. 1925.}. 

Cassiano Gomes Figueiredo caiu na categoria dos que realmente sofreram nas mãos dos pareceristas. A casa que pretendeu construir em 1906 na rua da Boa Vista recebeu um parecer demolidor de José Santos, da Inspetoria de Higiene, para quem os quartos não forrados deveriam ter telhas de vidro, a calha pluvial deveria ser estabelecida por trás da platibanda, os tubos pluviais deveriam ser embutidos na parede da fachada, os “agulheiros” deveriam ficar debaixo do passeio e a latrina “deve ser do sistema Unitas”, padrão da exigência dos pareceristas daquela inspetoria\footnote{\textbf{BR-BAAHMS}, Fundo “Intendência”, Série “Processos de Licenciamento de Reforma e Ampliação de Edificações”, Subsérie “Requerimentos e Plantas – Brotas”, caixa 2, processo 3488 folha 36, datado de 5 jul 1906.}. Outro nesta desditosa categoria foi Luiz Pedro Gonzaga: ao pedir licença para construir um chalé na rua Monte de Belém em 1927, foi convidado não somente a prestar maiores esclarecimentos sobre a obra, como também a mudar totalmente o projeto de chalé para o de uma casa com platibanda\footnote{\textbf{BR-BAAHMS}, Fundo “Intendência”, Série “Processos de Licenciamento de Reforma e Ampliação de Edificações”, Subsérie “Requerimentos e Plantas – Brotas”, caixa 15, processo 765 folha 295, datado de 23 maio 1927.}; como a rua Monte de Belém fica nas faldas do Engenho Velho, onde se costumava construir simples casebres para operários, uma hipótese para a mudança imposta é a de que o chalé talvez não fosse um “typo” compatível com a vizinhança. De modo semelhante, quando Domingos Pinheiro Alban solicitou em 1922 licença para construir uma casa comercial na Vila América, Manoel Alves Nazareth condicionou a aprovação a um parecer da Inspetoria de Higiene, pois considerava o local “insalubre”, e mesmo com a aprovação da Inspetoria mediante um “aterro com 50cm e calcada cimentada” impôs ao requerente que o telhado tivesse duas águas\footnote{\textbf{BR-BAAHMS}, Fundo “Intendência”, Série “Processos de Licenciamento de Reforma e Ampliação de Edificações”, Subsérie “Requerimentos e Plantas – Brotas”, caixa 11, processo 12899 folha 11, datado de 08 nov. 1911.}. 

Houve no Engenho Velho / Boa Vista pouquíssimos e excepcionais casos de rejeição total dos projetos, como o de uma casa a ser construída para Raymundo Gonçalves Martins que o parecerista da Diretoria de Obras, José Soares de Senna, reprovou por ``falta de esthetica”; a obra foi posteriormente aprovada por Aurelio de Britto Menezes, outro parecerista da diretoria\footnote{\textbf{BR-BAAHMS}, Fundo “Intendência”, Série “Processos de Licenciamento de Reforma e Ampliação de Edificações”, Subsérie “Requerimentos e Plantas – Brotas”, caixa 5, processo sem identificação nem data.}, indicando que havia alguma margem para a subjetividade dos engenheiros da Diretoria de Obras interferir nas aprovações e, portanto, na estética e arquitetura do bairro. A interferência da subjetividade dos pareceristas da Diretoria de Obras verificava-se mesmo quando o que estava em jogo eram questões eminentemente técnicas. Em 1928 o parecerista Mário Teixeira Rodrigues Lima apresentou várias correções no cálculo estrutural no projeto da licença para construção de uma casa requerida por Mário Peixoto; o requerente não apresentou qualquer nova memória de cálculo, mas o parecerista da Inspetoria de Higiene, João dos Santos Tuvo, aprovou-o assim mesmo\footnote{\textbf{BR-BAAHMS}, Fundo “Intendência”, Série “Processos de Licenciamento de Reforma e Ampliação de Edificações”, Subsérie “Requerimentos e Plantas – Brotas”, caixa 13, processo 362 folha 242, datado de 16 mar. 1928.}. 

Pior ainda foi o caso de Francisco Ventura, cujo projeto de construção de duas casas em 1913 recebeu parecer proibitivo por parte de Lucarinny Filho porque havia janelas “projetadas de má fé”; apresentado novo projeto com retificações nas janelas, ele foi aprovado, mas ficava claro que de pouco adiantava apresentar um projeto à Diretoria de Obras para depois executar obra diferente, ainda que apenas em detalhes\footnote{\textbf{BR-BAAHMS}, Fundo “Intendência”, Série “Processos de Licenciamento de Reforma e Ampliação de Edificações”, Subsérie “Requerimentos e Plantas – Brotas”, caixa 2, processo 99, datado de 19 maio 1913.}.

Comparativamente, no antigo 1º Distrito não se encontra um só parecer pela rejeição, e uns poucos pareceres modificativos. Veja-se o caso do pedido de “asseio geral” (reforma) na casa de Álvaro Pimenta da Cunha na rua da Alegria do Castro Neves, de 1928: além de ser mais um dos casos em que o parecerista José Soares de Senna alertou para o envolvimento de Pedro Jayme David com o desabamento no Monte Serrat, “cujo caso está dependendo de solução do Exmº. Sr. Dr. Intendente”, ele exigiu a do requerente remodelação da fachada “levando em conta não só o prédio como também o local”, sugerindo que “dando menos altura à fachada, esta apresentava melhor aspecto”\footnote{\textbf{BR-BAAHMS}, Fundo “Intendência”, Série “Processos de Licenciamento de Reforma e Ampliação de Edificações”, Subsérie “Requerimentos e Plantas – Brotas”, caixa 10, processo 979 folha 17, datado de 24 mar. 1928.}. 

Toda a documentação pesquisada segue apontando no mesmo sentido: eram os engenheiros funcionários da Diretoria de Obras a determinar o que se podia e o que não se podia fazer na construção civil, e portanto na arquitetura e no desenvolvimento urbano de Brotas e dos demais distritos urbanos de Salvador enquanto vigeu a legislação urbanística indicada.

Um outro exemplo no mesmo sentido é o de \textit{José Celestino dos Santos}, que nos anos de 1912 e 1913 aparece nos pedidos de licença como “auxiliar técnico do Intendente Municipal”. Sua tarefa era somente a de verificar se a obra se encontrava “compreendida no plano de melhoramentos municipais”; caso não se encontrasse, seu parecer, de que o fornecido num pedido de licença solicitado por Antônio Góes Tourinho dá exemplo, era invariável: “prevalece o  alinhamento determinado pela Diretoria de Obras Públicas Municipais”, que “não pode ser alterado”\footnote{\textbf{BR-BAAHMS}, Fundo “Intendência”, Série “Processos de Licenciamento de Reforma e Ampliação de Edificações”, Subsérie “Requerimentos e Plantas – Brotas”, caixa 2, processo 13850 folha 27, de 24 ago. 1912.}. Nenhuma dos pedidos de licença analisados no distrito de Brotas apresentou qualquer parecer onde José Celestino dos Santos indicou estar a obra “compreendida no plano de melhoramentos municipais”, demonstrando que Brotas estava fora dos planos da urbanização dirigida pela Intendência Municipal e pelo Governo da Bahia.

\subsection{Que apropriação da terra em Brotas?}\label{subsec:apropribrotas}

O regime de terras vigente em Brotas era o mesmo vigente em Salvador no mesmo período (cf. \autoref{subsubsec:polfundvalter}, p. \pageref{subsubsec:polfundvalter}): uma transição do regime da Lei 601/1850 para o regime de propriedade plena disciplinado pelo Código Civil de 1916, com todas as ``porosidades'' comuns ao período -- posses consuetudinárias, ocupação informal de terras, aforamentos, arrendamentos, proliferação de ``moradores'' etc. Na maioria dos pedidos de licença de obra pesquisados não se faz referência explícita à situação fundiária dos imóveis a construir ou reformar, pois não se exigia tal informação para a concessão das licenças necessárias; nos pouquíssimos pedidos de licença onde se menciona a situação fundiária do imóvel, contados literalmente nos dedos de duas mãos, é para registrar a situação de ``foreiro'' ou de ``arrendatário'' do requerente. 

No que diz respeito ao \textit{valor da terra} no distrito, apresenta-se, de forma parecida com o ocorrido na \autoref{subsubsec:polfundvalter} (p. \pageref{subsubsec:polfundvalter}), uma questão de método. A forma correta de auferir o valor da terra de forma direta é a pesquisa nos \textit{Livros de Décimas Urbanas} custodiados no Arquivo Público Municipal de Salvador, que por sinal encontram-se em excelente estado de conservação. Esta abordagem direta, entretanto, exigiria operações complexas e trabalhosas\footnote{Seria preciso, em primeiro lugar, estudar as categorias do lançamento tributário, ou seja, a forma que os funcionários da Intendência/Prefeitura empregavam para anotar os valores nos livros; depois, seria necessário o hercúleo trabalho de cópia do valor locativo de cada imóvel indicado pela Intendência/Prefeitura; em seguida, seria preciso construir tabelas para cada \textit{rua} (só no livro de 1893 estão registradas 41 ruas); logo após, todas as tabelas precisariam ser agrupadas num só banco de dados, modelado de forma a facilitar a consulta, a recuperação e o cruzamento de dados; por fim, os dados da décima urbana de cada ano precisariam ter sua consistência checada por meio de testes estatísticos diversos. Como se vê, trata-se de um trabalho cuja execução pede uma equipe dedicada, não um pesquisador solitário.}.

Por sorte, um tal trabalho já foi apresentado no \textbf{Annuario estatistico – annos de 1924 e 1925} organizado para o Governo da Bahia por M. Messias de \citeonline[pp.~263-264]{bahia_annuario_1926} com base nos valores da décima urbana de 1924, o que permitiu construir a \autoref{tab:imoveis1924brotas1} e a \autoref{tab:imoveis1924brotas2} (pp. \pageref{tab:imoveis1924brotas1} e \pageref{tab:imoveis1924brotas2}) para lançar luzes sobre a questão. 

\begin{sidewaystable}
\IBGEtab{
\caption{Relação dos imóveis arrolados pelo município de Salvador no distrito de Brotas em 1924 (parte 1)}\label{tab:imoveis1924brotas1}}
{
\begin{minipage}{\textwidth}
\begin{tiny}
\begin{tabular}{m{3cm} m{1cm} l l l l l l l l l l l}
\toprule
\multirow{2}{*}{Locais}	& \multirow{2}{*}{Valor}	& \multicolumn{10}{c}{Imóveis}\\
\cline{3-13}
	&	&Térreos	&Sobrados	&Abarracados	&Barracão	&Telheiros	&Galpões	&Em ruínas	&Em construção	&Em reconstrução	&Interditados	&TOTAL\\
\midrule
\midrule
Rua dos Galés							&26:184\$	&22	&0	&1	&0	&0	&0	&0	&0	&0	&0	&23\\
Rua Coronel Frederico Costa					&7:440\$	&7	&0	&0	&0	&0	&0	&0	&0	&0	&0	&7\\
Rua Uruguaiana							&108:984\$	&143	&2	&0	&0	&0	&0	&0	&9	&0	&0	&154\\
Trav. da Rua Uruguaiana					&30:304\$	&120	&0	&0	&0	&0	&0	&0	&1	&0	&0	&121\\
Rua da Boa Vista						&38:980\$	&60	&2	&0	&0	&0	&0	&0	&1	&0	&0	&63\\
Rua Agrippino Dorea						&67:832\$	&77	&0	&2	&0	&0	&0	&1	&0	&0	&0	&80\\
Beco do General							&4:800\$	&9	&0	&0	&0	&0	&0	&0	&0	&0	&0	&9\\
Rua do Socorro							&31:680\$	&59	&0	&0	&0	&0	&0	&0	&0	&1	&0	&60\\
Trav. Castro Neves					&4:620\$	&10	&0	&0	&0	&0	&0	&0	&0	&0	&0	&10\\
Rua do Castro Neves						&54:600\$	&83	&0	&0	&0	&0	&0	&1	&2	&0	&0	&86\\
Rua da Alegria							&16:676\$	&30	&0	&0	&0	&0	&0	&0	&0	&0	&0	&30\\
Trav. da Alegria						&14:124\$	&23	&0	&0	&0	&0	&0	&0	&0	&0	&0	&23\\
Trav. do Sangradouro para a Trav. da Alegria		&13:728\$	&16	&0	&0	&0	&0	&0	&0	&0	&0	&0	&16\\
Trav. do Sangradouro para a Rua da Alegria			&12:096\$	&18	&0	&0	&0	&0	&0	&0	&0	&0	&0	&18\\
Trav. do Sangradouro para a Lad. do Matatu Pequeno	&12:120\$	&27	&0	&0	&0	&0	&0	&0	&0	&0	&0	&27\\
Rua do Sangradouro						&46:072\$	&61	&3	&0	&0	&0	&0	&0	&0	&0	&0	&64\\
Trav. do Sangradouro						&19:228\$	&	&66	&0	&0	&0	&0	&0	&0	&0	&0	&66\\
Alto do Sangradouro						&9:348\$	&18	&0	&0	&0	&0	&0	&0	&1	&0	&0	&19\\
Rua da Vala ao Cabula						&34:592\$	&76	&6	&0	&0	&0	&0	&0	&1	&0	&0	&83\\
Estr. 2 de Julho						&94:186\$	&292	&0	&0	&0	&0	&0	&1	&5	&0	&0	&298\\
1ª Lad. do Engenho Velho					&6:684\$	&19	&0	&0	&0	&0	&0	&0	&0	&0	&0	&19\\
2ª Lad. do Engenho Velho					&4:524\$	&20	&0	&0	&0	&0	&0	&0	&0	&0	&0	&20\\
Rua do Engenho Velho						&30:576\$	&112	&0	&0	&0	&0	&0	&0	&0	&0	&0	&112\\
Capelinha do Deus Menino					&102:924\$	&416	&0	&0	&0	&0	&0	&0	&5	&0	&0	&421\\
Quinta das Beatas						&39:432\$	&149	&0	&0	&0	&0	&0	&0	&2	&0	&0	&151\\
1ª Trav. da Quinta das Beatas				&2:700\$	&13	&0	&0	&0	&0	&0	&0	&0	&0	&0	&13\\
2ª Trav. da Quinta das Beatas				&1:860\$	&7	&0	&0	&0	&0	&0	&0	&0	&0	&0	&7\\
Alto do Formoso							&11:340\$	&50	&0	&0	&0	&0	&0	&0	&1	&1	&0	&52\\
Rua do Matatu Pequeno						&57:920\$	&70	&0	&0	&0	&0	&0	&0	&4	&0	&0	&74\\
Rua do Matatu Grande						&33:672\$	&86	&0	&0	&0	&0	&0	&0	&0	&0	&0	&86\\
Casa da Pólvora							&13:836\$	&30	&0	&0	&0	&0	&0	&0	&0	&0	&0	&30\\
\bottomrule
(continua na parte 2) & & & & & & & & & & & & \\
\end{tabular} 
\end{tiny}
\end{minipage}
}
{\fonte{\textbf{Annuario estatistico – annos de 1924 e 1925} organizado para o Governo da Bahia por M. Messias de \citeonline[pp.~263-264]{bahia_annuario_1926}.}}
\end{sidewaystable}
\begin{table}[!htp]
\IBGEtab{
\caption{Relação dos imóveis arrolados pelo município de Salvador no distrito de Brotas em 1924 (parte 2)}\label{tab:imoveis1924brotas1}}
{
\begin{tiny}
\begin{tabular}{m{3cm} m{1cm} m{0,7cm} m{0,7cm} m{0,7cm} m{0,7cm} m{0,7cm} m{0,7cm} m{0,7cm} m{0,7cm} m{0,7cm} m{0,7cm} m{0,7cm}}
\hline
\multirow{2}{*}{Locais}	& \multirow{2}{*}{Valor}	& \multicolumn{10}{c}{Imóveis}\\
\cline{3-13}
	&	&Tér- reos	&Sobra- dos	&Abarra- cados	&Barra- cão	&Telhei- ros	&Gal- pões	&Em ruínas	&Em cons- trução	&Em re- cons- trução	&Inter- dita- dos	&TOTAL\\
\hline
\hline
Lad. do Fabrício						&28:620\$	&27	&0	&2	&0	&0	&0	&0	&0	&1	&0	&30\\
Lad. do Acupe						&5:700\$	&13	&0	&0	&0	&0	&0	&0	&0	&1	&0	&14\\
Rua do Acupe							&16:332\$	&19	&0	&0	&0	&0	&0	&0	&0	&0	&0	&19\\
Trav. do Acupe						&1:740\$	&3	&0	&0	&0	&0	&0	&0	&0	&0	&0	&3\\
Rua de Brotas							&62:476\$	&65	&0	&1	&0	&0	&0	&0	&0	&0	&0	&66\\
1ª Trav. da Rua de Brotas					&6:156\$	&6	&0	&0	&0	&0	&0	&0	&0	&0	&0	&6\\
Cruz da Redenção						&12:276\$	&25	&0	&0	&0	&0	&0	&0	&0	&0	&0	&25\\
Rua do Beiju							&9:204\$	&30	&0	&0	&0	&0	&0	&0	&0	&0	&0	&30\\
Trav. da Rua do Beiju					&180\$		&1	&0	&0	&0	&0	&0	&0	&0	&0	&0	&1\\
Rua das Campinas						&7:680\$	&12	&0	&0	&0	&0	&0	&0	&0	&0	&0	&12\\
Vargem de Santo Antônio						&1:200\$	&2	&0	&0	&0	&0	&0	&0	&0	&0	&0	&2\\
Trav. do Pomar						&1:500\$	&8	&0	&0	&0	&0	&0	&0	&0	&0	&0	&8\\
Pomar								&240\$		&2	&0	&0	&0	&0	&0	&0	&0	&0	&0	&2\\
Candeal Pequeno							&1:680\$	&9	&0	&0	&0	&0	&0	&0	&0	&0	&0	&9\\
Candeal Grande							&360\$		&1	&0	&0	&0	&0	&0	&0	&0	&0	&0	&1\\
Lad. da Cruz das Almas					&12:912\$	&26	&0	&0	&0	&0	&0	&0	&0	&0	&0	&26\\
Largo da Mariquita						&24:480\$	&14	&0	&1	&0	&0	&0	&0	&0	&1	&0	&16\\
Rua dos Dendezeiros						&38:430\$	&32	&0	&0	&0	&0	&0	&0	&0	&1	&0	&33\\
Trav. da Rua dos Dendezeiros para a Rua do Meio		&816\$		&2	&0	&0	&0	&0	&0	&0	&0	&0	&0	&2\\
Rua do Meio							&17:610\$	&20	&0	&0	&0	&0	&0	&1	&0	&0	&0	&21\\
Rua Direita							&37:656\$	&37	&0	&0	&0	&0	&0	&0	&0	&0	&0	&37\\
Rua Fonte do Boi						&10:248\$	&13	&0	&0	&0	&0	&0	&0	&0	&0	&0	&13\\
Rua das Pedrinhas						&22:200\$	&30	&0	&0	&0	&0	&0	&0	&0	&0	&0	&30\\
1ª Trav. da Rua das Pedrinhas				&5:220\$	&9	&0	&0	&0	&0	&0	&0	&0	&0	&0	&9\\
2ª Trav. da Rua das Pedrinhas				&2:640\$	&5	&0	&0	&0	&0	&0	&0	&0	&0	&0	&5\\
Rua da Lagoa							&8:916\$	&34	&0	&0	&0	&0	&0	&0	&1	&0	&0	&35\\
Rua Direita da Amaralina					&27:700\$	&18	&0	&3	&0	&0	&0	&0	&0	&0	&0	&21\\
Rua do Meio da Amaralina					&17:220\$	&23	&0	&0	&0	&0	&0	&0	&0	&0	&0	&23\\
Alto da Ubarana							&900\$		&3	&0	&0	&0	&0	&0	&0	&0	&0	&0	&3\\
Pituba								&10:140\$	&24	&0	&0	&0	&0	&0	&0	&0	&0	&0	&24\\
Armação Pequena							&720\$		&2	&0	&0	&0	&0	&0	&0	&0	&0	&0	&2\\
Armação Grande							&600\$		&1	&0	&0	&0	&0	&0	&0	&0	&0	&0	&1\\
\hline
TOTAL								&1:346:814\$	&2:705	&13	&10	&0	&0	&0	&4	&33	&6	&0	&2771\\
\hline
\end{tabular} 
\end{tiny}
}
{\fonte{\citeonline[pp.~263-264]{bahia_annuario_1926}.}}
\end{table}


Graças a este trabalho, foi possível encontrar mais uma vez o valor locativo médio dos imóveis arrolados, dividindo a massa do valor locativo dos imóveis de cada rua pelo número de imóveis da mesma rua; por rudimentar que seja o procedimento, foi a solução encontrada para evitar as distorções verificáveis quando foi tentada a comparação direta entre as massas de valores locativos de cada rua, por força de fatores como as diferenças nos valores locativos de cada imóvel e as grandes variações no número de imóveis por rua. Os resultados foram compilados na \autoref{tab:valorlocativomedio1924brotas} (p. \pageref{tab:valorlocativomedio1924brotas}).

\begin{table}[!htp]
\IBGEtab{
\caption{Valor locativo médio dos imóveis nas ruas arroladas pelo município de Salvador no distrito de Brotas em 1924}\label{tab:valorlocativomedio1924brotas}}
{
\begin{tiny}
\begin{tabular}{ll}
\toprule
Locais	&Valor locativo médio por imóvel\\
\midrule
\midrule
Rua dos Galés	&1:138\$430\\
Rua Coronel Frederico Costa	&1:062\$860\\
Rua Uruguaiana	&707\$690\\
Travessa da Rua Uruguaiana	&250\$450\\
Rua da Boa Vista	&618\$730\\
Rua Agrippino Dorea	&847\$900\\
Beco do General	&533\$330\\
Rua do Socorro	&528\$000\\
Travessa do Castro Neves	&462\$000\\
Rua do Castro Neves	&634\$880\\
Rua da Alegria	&555\$870\\
Travessa da Alegria	&614\$090\\
Travessa do Sangradouro para a Travessa da Alegria	&858\$000\\
Travessa do Sangradouro para a Rua da Alegria	&672\$000\\
Travessa do Sangradouro para a Ladeira do Matatu Pequeno	&448\$890\\
Rua do Sangradouro	&719\$880\\
Travessa do Sangradouro	&291\$330\\
Alto do Sangradouro	&492\$000\\
Rua da Vala ao Cabula	&416\$770\\
Estrada 2 de Julho	&316\$060\\
1ª Ladeira do Engenho Velho	&351\$790\\
2ª Ladeira do Engenho Velho	&226\$200\\
Rua do Engenho Velho	&273\$000\\
Capelinha do Deus Menino	&244\$480\\
Quinta das Beatas	&261\$140\\
1ª Travessa da Quinta das Beatas	&207\$690\\
2ª Travessa da Quinta das Beatas	&265\$710\\
Alto do Formoso	&218\$080\\
Rua do Matatu Pequeno	&782\$700\\
Rua do Matatu Grande	&391\$530\\
Casa da Pólvora	&461\$200\\
Ladeira do Fabrício	&954\$000\\
Ladeira do Acupe	&407\$140\\
Rua do Acupe	&859\$580\\
Travessa do Acupe	&580\$000\\
Rua de Brotas	&946\$610\\
1ª Travessa da Rua de Brotas	&1:026\$000\\
Cruz da Redenção	&491\$040\\
Rua do Beiju	&306\$800\\
Travessa da Rua do Beiju	&180\$000\\
Rua das Campinas	&640\$000\\
Vargem de Santo Antônio	&600\$000\\
Travessa do Pomar	&187\$500\\
Pomar	&120\$000\\
Candeal Pequeno	&186\$670\\
Candeal Grande	&360\$000\\
Ladeira da Cruz das Almas	&496\$620\\
Largo da Mariquita	&1:530\$000\\
Rua dos Dendezeiros	&1:164\$550\\
Travessa da Rua dos Dendezeiros para a Rua do Meio	&408\$000\\
Rua do Meio	&838\$570\\
Rua Direita	&1:017\$730\\
Rua Fonte do Boi	&788\$310\\
Rua das Pedrinhas	&740\$000\\
1ª Travessa da Rua das Pedrinhas	&580\$000\\
2ª Travessa da Rua das Pedrinhas	&528\$000\\
Rua da Lagoa	&254\$740\\
Rua Direita da Amaralina	&1:319\$050\\
Rua do Meio da Amaralina	&748\$700\\
Alto da Ubarana	&300\$000\\
Pituba	&422\$500\\
Armação Pequena	&360\$000\\
Armação Grande	&600\$000\\
\midrule
TOTAL	&568\$170\\
\bottomrule
\end{tabular} 
\end{tiny}
}
{\fonte{Elaboração do autor, com base no \textbf{Annuario estatistico – annos de 1924 e 1925} organizado para o Governo da Bahia por M. Messias de  \citeonline[pp.~263-264]{bahia_annuario_1926}.}}
\end{table}


O uso do valor locativo médio dos imóveis de cada rua do distrito como indicador do valor da terra permite classificar as ruas de Brotas segundo o valor médio de seus imóveis, estabelecendo assim aquelas com imóveis ``mais caros'' e aqueloutras com imóveis ``mais baratos'' (ressalvando-se sempre tratar-se de valores médios, não de valores reais). Os resultados deste procedimento podem ser verificados nas tabelas \autoref{tab:maiscaros1924brotas} e \autoref{tab:maisbaratos1924brotas} (\pageref{tab:maisbaratos1924brotas}), dos quais se pode extrair conclusões importantes.

\begin{table}[!htp]
\IBGEtab{
\caption{Valor locativo médio dos imóveis nas dez ruas de imóveis ``mais caros'' entre os arrolados pelo município de Salvador no distrito de Brotas em 1924}\label{tab:maiscaros1924brotas}}
{
\begin{tabular}{rr}
\toprule
Locais	&Valor locativo médio por imóvel\\
\midrule
\midrule
Largo da Mariquita	&1:530\$000\\
Rua Direita da Amaralina	&1:319\$050\\
Rua dos Dendezeiros	&1:164\$550\\
Rua dos Galés	&1:138\$430\\
Rua Coronel Frederico Costa	&1:062\$860\\
1ª Travessa da Rua de Brotas	&1:026\$000\\
Rua Direita	&1:017\$730\\
Ladeira do Fabrício	&954\$000\\
Rua de Brotas	&946\$610\\
Rua do Acupe	&859\$580\\
\bottomrule
\end{tabular} 
}
{\fonte{Elaboração do autor, com base no \textbf{Annuario estatistico – annos de 1924 e 1925} organizado para o Governo da Bahia por M. Messias de \citeonline[pp.~263-264]{bahia_annuario_1926}.}}
\end{table}

\begin{table}[!htp]
\IBGEtab{
\caption{Valor locativo médio dos imóveis nas dez ruas de imóveis ``mais baratos'' entre os arrolados pelo município de Salvador no distrito de Brotas em 1924}\label{tab:maisbaratos1924brotas}}
{
\begin{tabular}{rr}
\hline
Locais	&Valor locativo médio por imóvel\\
\hline
\hline
Alto da Ubarana	&300\$000\\
Travessa do Sangradouro	&291\$330\\
Rua do Engenho Velho	&273\$000\\
2ª Travessa da Quinta das Beatas	&265\$710\\
Quinta das Beatas	&261\$140\\
Rua da Lagoa	&254\$740\\
Travessa da Rua Uruguaiana	&250\$450\\
Capelinha do Deus Menino	&244\$480\\
2ª Ladeira do Engenho Velho	&226\$200\\
Alto do Formoso	&218\$080\\
1ª Travessa da Quinta das Beatas	&207\$690\\
Travessa do Pomar	&187\$500\\
Candeal Pequeno	&186\$670\\
Travessa da Rua do Beiju	&180\$000\\
Pomar	&120\$000\\
\hline
\end{tabular}
}
{\fonte{Elaboração do autor, com base em \citeonline[pp.~263-264]{bahia_annuario_1926}.}}
\end{table}

\begin{itemize}
\item Três entre as ruas com imóveis ``mais caros'' (Largo da Mariquita, Rua Direita e Rua dos Dendezeiros) encontram-se no Rio Vermelho, que manteve durante todo o período seu caráter de \textit{arrabalde de veraneio} dos soteropolitanos mais ricos. 
\item Duas entre as ruas com imóveis ``mais baratos'' (Alto da Ubarana e Rua da Lagoa) e uma dentre aquelas com imóveis ``mais caros'' (Rua Direita da Amaralina) encontram-se em Amaralina, fato que será investigado mais adiante; a Rua Direita da Amaralina, entretanto, é a principal da Cidade Balneária Amaralina, talvez um dos primeiros loteamentos formais de Salvador, totalmente voltado para atividades veranistas de alto padrão.
\item A larga Rua dos Galés segue entre as mais valorizadas do distrito, especialmente por ser eixo central da mais antiga área urbanizada do distrito.
\item Duas vias que hoje chamaríamos de ``arteriais'' ou ``coletoras'', a Rua de Brotas e a Rua do Acupe, encontram-se entre as mais valorizadas do distrito, talvez exatamente pela centralidade que exercem sobre a circulação no território.
\item Quatro das ruas que integram a área do Engenho Velho de Brotas (Rua do Engenho Velho, Travessa da Rua Uruguaiana, Capelinha do Deus Menino, 2ª Ladeira do Engenho Velho) estão entre aquelas com imóveis ``mais baratos'', enquanto uma delas (Rua Coronel Frederico Costa) encontra-se no grupo das ruas com imóveis ``mais caros''; é de se notar que as ruas menos valorizadas são exatamente aquelas que se encontram nas encostas e barrancos do Engenho Velho, portanto as mais insalubres e de difícil acesso, enquanto a mais valorizada entre elas encontra-se na cumeada, em terreno plano, fazendo a ligação desta área do distrito com a valorizada Rua de Brotas.
\item Todas as ruas abertas na Quinta das Beatas e áreas circunvizinhas (Alto do Formoso, 1ª e 2ª Travessas e a própria Quinta das Beatas) encontram-se entre aquelas com imóveis ``mais baratos''.
\end{itemize}

O cálculo de proporção de sobrados empregue na \autoref{subsubsec:polfundvalter} é inaplicável ao distrito, pois das 63 ruas cujos dados foram disponibilizados por \citeonline[pp.~263-264]{bahia_annuario_1926} apenas quatro apresentam este tipo de imóvel (Rua da Vala ao Cabula, com 6; Rua do Sangradouro, com 3; Travessa da Uruguaiana, com 2; e Rua da Boa Vista, com 2); conquanto este indicador sirva para reforçar o que já se viu a respeito do valor da terra no Sangradouro e na cumeada da Boa Vista, empregá-lo sem maiores cuidados equivaleria a considerar todo o restante do distrito com valor de terra ínfimo, o que não condiz com a situação encontrada em meio ao conjunto da documentação pesquisada. 

\subsection{Locação: forma principal de acesso à terra e à moradia}\label{subsec:locatermor}

Verificou-se em todo o período estudado que tanto os arrendamentos quanto os alugueis prosseguiram como a forma preferencial de acesso à terra e à moradia. Como exemplo da tendência, a \autoref{tab:taxaloc1893} (p. \pageref{tab:taxaloc1893}) apresenta os dados relativos a 1893, que permitem chegar às seguintes conclusões:

\afterpage{
\begin{table}[!htp]
\centering
\IBGEtab{
\caption{Imóveis do distrito de Brotas (1893), por condição da ocupação}\label{tab:taxaloc1893}}
{ \begin{tiny}
\begin{tabular}{m{4cm}llllll}
\toprule
Rua	&Proprietários	&Inquilinos	&Ausente	&Vazio/Fechado	&Ignorado	&TOTAL	\\
\midrule \midrule
Rua dos Galés	&10	&17	&1	&5	&0	&33	\\
Rua Uruguaiana	&5	&20	&0	&5	&0	&30	\\
Rua da Boa Vista	&13	&24	&6	&3	&0	&46	\\
Ladeira da Boa Vista	&1	&2	&0	&0	&1	&4	\\
Rua 1º de Março	&24	&52	&2	&5	&9	&92	\\
Rua do Socorro	&13	&30	&4	&5	&1	&53	\\
Travessa do Socorro	&0	&0	&0	&2	&0	&2	\\
Travessa do Castro Neves	&0	&7	&0	&0	&2	&9	\\
Direita do Castro Neves	&14	&42	&5	&4	&5	&70	\\
Rua da Alegria	&7	&14	&0	&3	&1	&25	\\
Travessa da Alegria para o Sangradouro	&4	&12	&0	&0	&0	&16	\\
Sangradouro	&7	&34	&5	&3	&0	&49	\\
Travessa do Sangradouro	&2	&15	&0	&1	&0	&18	\\
Estrada da Vala ao Cabula (lado direito)	&6	&49	&8	&5	&1	&69	\\
Estrada 2 de Julho	&20	&8	&5	&2	&5	&40	\\
Estrada da Quinta das Beatas	&6	&2	&0	&1	&2	&11	\\
Quinta das Beatas	&5	&0	&0	&0	&2	&7	\\
Matatu Pequeno	&13	&24	&2	&3	&7	&49	\\
Matatu Grande	&25	&8	&1	&5	&10	&49	\\
Estrada para a Casa da Pólvora	&1	&3	&0	&2	&2	&8	\\
Estrada do Engenho Velho	&11	&2	&1	&0	&3	&17	\\
Ladeira do Acupe	&12	&47	&2	&3	&0	&64	\\
Rua do Sangradouro	&1	&38	&2	&1	&1	&43	\\
Largo do Acupe	&0	&1	&0	&1	&1	&3	\\
Estrada do Acupe para a 2 de Julho	&4	&3	&0	&1	&0	&8	\\
Estrada de Brotas	&7	&7	&2	&2	&2	&20	\\
Estrada da Cruz das Almas	&8	&6	&3	&1	&4	&22	\\
Largo de Brotas	&9	&21	&3	&7	&5	&45	\\
Estrada para a Cruz da Redenção	&2	&8	&1	&1	&1	&13	\\
Largo da Cruz da Redenção	&0	&1	&4	&0	&2	&7	\\
Campinas	&1	&1	&0	&0	&0	&2	\\
Candeal	&3	&0	&0	&0	&0	&3	\\
Mariquita	&26	&35	&2	&31	&9	&103	\\
Estrada do Sangradouro para o Matatu	&3	&8	&0	&1	&6	&18	\\
Estrada da Ubarana	&1	&0	&0	&0	&1	&2	\\
Pomar	&4	&0	&0	&0	&2	&6	\\
Pituba	&2	&0	&0	&1	&0	&3	\\
Lagoa	&1	&0	&0	&0	&0	&1	\\
Estrada para Armação	&2	&3	&1	&1	&2	&9	\\
Várzea de Santo Antônio	&0	&0	&0	&1	&0	&1	\\
Armação	&1	&0	&1	&1	&0	&3	\\
\midrule
TOTAL	&274	&544	&61	&107	&87	&1073	\\
\bottomrule
\end{tabular}
\end{tiny}}
{\fonte{\textbf{BR BAAHMS}, Livro de Décimas Urbanas de 1893.}}
\end{table}
}

\begin{itemize}
\item 50,7\% dos imóveis em Brotas eram ocupados por ``inquilinos'' no início da Primeira República, contra 24,54\% ocupados por ``proprietários'' e 9,97\% que encontravam-se fechados ou vazios\footnote{A Lei Municipal nº 27, de 5 ago. 1893, dizia em seu art. 15 que a décima urbana deveria ser cobrada do ``proprietário'', e que caso o prédio fosse locado, seria cobrada sobre o valor do aluguel; além disso, o art. 11 da mesma Lei Municipal dizia que a décima urbana seria lançada por cinco lançadores e cinco auxiliares, distribuídos por cinco distritos (1º distrito: Sé e Conceição; 2º distrito: São Pedro e Vitória; 3º distrito: Paço e Sant'anna; 4º distrito: Pilar, Mares e Penha;  5º distrito: Santo Antônio e Brotas), enquanto o art. 12 dizia que o lançamento seria feito entre julho e outubro. Apesar deste regulamento por sinal bastante minucioso, as aspas são necessárias por não haver qualquer indicação de que os ``inquilinos'' ocupassem os imóveis respaldados exclusivamente por contratos de aluguel. Dado o regime de terras então existente, é provável que a Intendência entendesse como ``inquilino'' não apenas os locatários, mas também os \textit{arrendatários}, os \textit{foreiros}, os \textit{moradores}, os \textit{permissionários} e quaisquer outras formas de ocupação onde fosse resguardada a figura do \textit{possuidor indireto} do imóvel contra o \textit{possuidor direto} que o ocupava. O mesmo vale para a figura do ``proprietário'': o confuso regime de terras em vigor em 1893 não facilitava para os burocratas de médio e baixo escalão responsáveis pelo lançamento tributário e pela coleta dos impostos a distinção entre proprietários plenos, posseiros e sesmeiros irregulares, sendo fácil deduzir daí a hipótese de que a cobrança da décima urbana incidia sobre todos quantos reunissem em si as qualidades de posseiros diretos e indiretos. Para todos os efeitos, portanto, \textit{inquilino} foi entendido como quem detivesse apenas a posse direta, contra a posse indireta de um locador, arrendante etc., enquanto \textit{proprietário} foi entendido como quem detivesse a posse direta e a indireta, e ademais residisse no imóvel em questão.}.
\item As maiores proporções de casas ocupadas por seus ``proprietários'' foram encontradas nas ruas ou localidades marcadamente rurais (Candeal, Alagoa, Pomar, Pituba, Campinas etc.), ou naquelas onde o valor locativo era mais baixo (Quinta das Beatas, Matatu); inversamente, as ruas ou localidades com mais altas taxas de ocupação por ``inquilinos'' (Sangradouro, Castro Neves, Uruguaiana, Galés, Ladeira do Fabrício etc.) foram aquelas que em 1924 apresentaram mais alto valor locativo médio.
\item O maior número de imóveis encontra-se nas ruas e localidades onde era mais intenso o processo de valorização da terra (Mariquita, 1º de Março/Pitangueiras, Castro Neves, Sangradouro, Socorro, Galés, Uruguaiana, Alegria etc.), enquanto as localidades com menor número de imóveis eram precisamente aquelas onde tal processo ou era muito lento, ou não ocorria (Lagoa, Várzea de Santo Antônio, Ubarana, Campinas, Armação, Pituba, Candeal, Pomar etc.); as exceções resultam ou da pequena extensão de logradouros inseridos em contextos de valorização acelerada da terra (Travessa do Socorro, Travessa do Castro Neves, Travessa da Alegria para o Sangradouro), ou de um processo pretérito de fragmentação imobiliária em vizinhanças onde o valor da terra era baixo (Matatu, Quinta das Beatas).
\item O enorme número de imóveis vazios/fechados e de ``inquilinos'' na Mariquita explica-se pelo fato de tratar-se em 1893, ainda, de um arrabalde veranista, de uma estância balneária, de uma localidade cuja ocupação, descontados os pescadores e pequenos agricultores lá residentes, era eminentemente \textit{sazonal}, de \textit{temporada}, resultando em baixa ocupação permanente -- não sem riscos, pois a Intendência punha-se a derrubar as casas vazias da Mariquita deixadas arruinar pelos donos\footnote{\textbf{A Manhã}, ano I, nº 199, 03 dez. 1920, p. 2}.
\end{itemize}

A predominância da ``locação'' como forma de acesso à terra e à moradia expressa ainda a proliferação da especulação imobiliária sobre as terras do distrito. Durante a pesquisa das licenças para construção foi possível encontrar diversos modelos de ``casas de aluguel'', ou seja, corredores de pequenas casas geminadas, de poucos aposentos e aparência espartana, construídas como que num só molde com a única finalidade da locação, arrendamento, cessão etc. Às vezes tais casas sequer precisavam ser geminadas; uma vista rápida por qualquer dos \textbf{Livros de Décimas Urbanas} é suficiente para perceber como em certas ruas um só ``proprietário'' detém três, quatro, cinco, dez, vinte casas em sequência no mesmo logradouro

INSERIR PLANTAS DE CASAS DE ALUGUEL GEMINADAS

INSERIR PLANTAS DE CASAS DE ALUGUEL SIMPLES

A pesquisa mostrou também a construção em Brotas de uma \textit{vila operária} por parte da Companhia União Fabril (cf. ), mas de modo algum este modelo de habitação para proletários foi tão difundida no distrito quanto as ``casas de alugar''.

INSERIR PLANTA DA VILA OPERÁRIA DA UNIÃO FABRIL

\subsection{Loteamentos e parcelamentos: por que tão poucos?}\label{subsec:loteamentos}

Verifica-se via de regra que, em seguida à derrubada das salvaguardas à integridade de um patrimônio imobiliário, impõe-se com o passar dos anos uma tendência à desagregação e fragmentação deste patrimônio, seja pela via das \textit{partilhas hereditárias}, seja pela via dos \textit{loteamentos e parcelamentos}, ou mesmo por força de \textit{litígios judiciais} intermináveis \cite{costaporto_sesmaria_1980,sodero_diragrario_1990}. As partilhas hereditárias e os litígios judiciais interessam mais a uma historiografia do judiciário e da administração da justiça que a uma pesquisa sobre história urbana como a que aqui se expõe; são os \textit{loteamentos} e os \textit{parcelamentos} o assunto a ser tratado daqui por diante como tentativa de compreender a fragmentação imobiliária verificada no distrito de Brotas do final do século XIX à década de 1930.

Infelizmente, pouco foi possível de se encontrar sobre o assunto nos arquivos consultados. Tem-se notícia dos seguintes loteamentos oficiais:

\begin{itemize}
\item O famoso plano da \textit{Cidade Luz}, concebido em 1919 por Theodoro Sampaio, que traçou as linhas gerais do que hoje conhecemos como o bairro da Pituba.
\item O loteamento da \textit{Cidade Balneária Amaralina}, cuja planta original não foi possível encontrar mas que se sabe, por meio das licenças de construção, ampliação e obras consultadas, já ser existente em 1893.
\item O loteamento da fazenda \textit{Santa Cruz}, datado de XXXX.
\end{itemize} 

Tais loteamentos, entretanto, não conseguem explicar a a fragmentação imobiliária encontrada a uma simples comparação entre a quantidade de imóveis na freguesia em 1886 (\autoref{tab:decurb1886-1891}, p. \pageref{tab:decurb1886-1891}) e aquela encontrada em 1924 (\autoref{tab:imoveis1924brotas1} e \autoref{tab:imoveis1924brotas2}, p. \pageref{tab:imoveis1924brotas2}). Como explicá-la, portanto?

Não há explicação única, mas vários elementos dispersos parecem confirmar hipóteses complementares: a \textit{pressão demográfica e imobiliária causada pela abolição da escravidão}; a continuidade dos \textit{arrendamentos}, \textit{aforamentos} e \textit{alugueis} como forma preferencial de acesso à terra e à habitação pelos mais remediados; a opção pelos \textit{loteamentos informais} como confluência entre a sonegação fiscal pelos terratenentes e a satisfação da necessidade de acesso barato à posse da terra, ainda que insegura, pelos mais pobres; a persistência dos processos de \textit{ocupação informal da terra}, estes quase impossíveis de rastrear exatamente por não serem documentados.

Já se discutiu anteriormente () como a abolição da escravidão gerou uma onda migratória do Recôncavo em direção a Salvador; a pressão demográfica daí resultante exerceu força também sobre a malha urbana, demandando sua expansão. A massa de libertos recém-chegados nem cabia no espaço urbano preexistente, nem se enquadrava nas regras de sociabilidade cada vez mais rígidas -- e desabridamente racistas -- estabelecidas tanto pelos sucessivos blocos de poder a exercer a hegemonia política, quanto pelo horror da burguesia e dos gestores em todos os escalões, tomando a aparência pela essência, a tudo quanto remetesse à antiga associação entre as africanidades e o trabalho. Uma das soluções encontradas pelos recém-libertos para fixarem-se em Salvador foi o estabelecimento em ``zonas próprias'', em especial nos distritos com ``maiores áreas verdes, aptos, portanto, a ser explorados mais livremente'' \cite{santos_habitacao_1990}. 

Tal hipótese é reforçada pelo fato de a própria municipalidade soteropolitana entender a necessidade de dar ordem e sentido ao processo de desagregação e fragmentação das grandes herdades circundantes de Salvador, em especial quando eram os recém-libertos e a massa proletária a se beneficiar disto tudo. Em 8 de abril de 1905 promulgou a \textit{Resolução 160}, que entre outras providências já previstas pela ``lei das plantas''\footnote{\textit{Resolução Municipal nº 28}, de 12 de agosto de 1893, que impôs a apresentação ao município de plantas das obras de construção, ampliação ou reforma como condição para sua autorização; cf. a \autoref{subsec:constrampliref} (p. \pageref{subsec:constrampliref}) para o texto integral.} e pelo regulamento da décima urbana\footnote{\textit{Lei Municipal nº 27}, de 5 ago. 1893; cf. a \autoref{subsec:locatermor} (p. \pageref{subsec:locatermor}) para uma discussão mais pormenorizada de seu conteúdo. Esta lei poderia ser renovada quadrienalmente; daí que o texto da Resolução 160 fale num regulamento de 5 de agosto de 1903 para as décimas, autorizando sua revisão (art. 17).}, indicou o seguinte:

\begin{citacao}
\textbf{Art. 1º.} Fica o intendente autorisado, desde já, a mandar levantar as respectivas plantas das seguintes zonas urbanas: estradas 2 de Julho, Federação, Areia, Retiro, Cruz das Almas, Fazenda Garcia, Quinta da Barra, Quinta das Beatas, Cidade Nova, Pau Miudo, Resgate, S. Lazaro, Ondina, Amaralina, Ubaranas, Pituba e de todos os demais logares do perimetro urbano, onde se pretenda edificar, plantas que serão sujeitas á apreciação e approvação do Conselho Municipal. 

\textbf{Art. 2º.} Estas plantas constarão do actual traçado, bem como de todas as modificações de alinhamento que forem julgadas necessarias pela Directoria de Obras e viação no intuito de, fazendo desaparecer as curvas e sinuosidades, preparar-se as futuras avenidas de que tanto carece esta cidade, como garantia ao seu saneamento e embelezamento.

\textbf{Art. 3º.} Nenhuma construcção será feita nestas zonas ou em outra qualquer, sem que primeiramente seja levantada a necessaria planta e esta approvada.

\textbf{Art. 4º.} A Intendencia mandará levantar em plantas diversos typos de construcção, inclusive os das classes pobre e operaria destinados às differentes zonas, typos que não poderão ser modificados em sua essencia, salvo quando a construcção obedecer a um estylo especial perfeitamente conhecido.

\textbf{Art. 5º.} A Intendencia fica autorisada a construir, em particular, nos terrenos de sua propriedade, pequenas habitações hygienicas destinadas às classes pobre e operaria, podendo fazer cessão, desde quando seja indemnisada apenas, do valor do seu custo.
\end{citacao}

Os demais catorze artigos tratam de estímulos à construção em terrenos baldios, à reforma de imóveis arruinados, tudo alimentado por isenções decenais, quindecenais ou vintenais das décimas urbanas. Trata-se de um verdadeiro \textit{código de zoneamento urbano e de ocupação e uso do solo} o que se impunha já em 1905 com este regulamento; era ademais uma legislação que pretendia disciplinar a construção de novas vias as mais retas possíveis, prevendo seu uso futuro como avenidas. Como se vê, a fragmentação das herdades e a pressão demográfica pautaram como que um ``protoplanejamento'' da expansão da malha urbana --  ainda que, verificando-se as licenças de obra, em diversos momentos reclamem os engenheiros da Diretoria Municipal de Obras da falta, da incompletude ou do sumiço de algumas destas plantas\footnote{Até onde foi possível pesquisar no \textbf{BR-BAAHMS}, nenhuma destas plantas sobreviveu aos nossos dias.}.

A continuidade dos arrendamentos, aforamentos e locações já foi vista anteriormente (cf. \autoref{subsec:locatermor}, p. \pageref{subsec:locatermor}). Basta dizer, complementarmente, que os arrendamentos, aforamentos e locações tornaram-se com o tempo um fator de pressão pelo fracionamento das antigas herdades, na medida em que o valor da terra em determinadas vizinhanças do distrito aumentou por força da progressiva implementação de infraestruturas urbanas (eletricidade, bonde, telefonia, iluminação pública etc.) e o estabelecimento de residência em vizinhanças de Brotas como Castro Neves, Sangradouro e outras inseridas neste processo de valorização, conquanto não competisse com o \textit{status} conferido pelas moradias no corredor da Vitória, no distrito de São Pedro ou noutros lugares de altíssima valorização no período estudado, conferia a seus moradores ainda alguma distinção, alguma posição, algum \textit{status} que os diferenciasse da massa proletária famélica e doente. O salto no regime fundiário estabelecido pelo Código Civil de 1916 certamente terá influido na passagem de muitos destes arrendamentos, aforamentos e locações para um regime de propriedade plena, mas sem uma pesquisa cartorária rigorosa e minuciosa será impossível passar do campo das hipóteses para o das comprovações.

Ocorre que mesmo o aluguel, aforamento, arrendamento etc., num contexto de extremo pauperismo da maioria da população soteropolitana, pode ser proibitivo. Daí a proliferação em Brotas dos \textit{loteamentos informais} durante o período estudado. Especialmente no que diz respeito às vendas, num contexto onde a propriedade imobiliária implica numa série de obrigações tributárias e num processo extremamente burocratizado de transmissão, e onde parcela significativa da população não dispõe dos recursos necessários para acessar o mercado formal de terras, ela pressiona pela criação um mercado informal de terras, ao arrepio de qualquer mecanismo formal -- e de qualquer segurança para suas posses. 

Em terceiro lugar, a persistência dos processos de \textit{ocupação informal da terra} permaneceu como hipótese de trabalho, embora a documentação pesquisada não tenha fornecido senão indícios e pistas de sua existência. A melhor forma de abordagem direta da questão seria uma pesquisa e sistematização dos autos de infração da Diretoria Municipal de Obras, da Inspetoria Municipal de Higiene e da Diretoria Estadual de Higiene, mas os prazos da pesquisa tornaram proibitivo o emprego deste método; mais uma vez, fez-se necessário encontrar abordagens indiretas, fragmentárias, indiciárias. A fonte, desta vez, foram as \textit{imagens} do distrito, em especial cartões-postais.

\subsubsection{Loteamentos e parcelamentos formais}

Os loteamentos e parcelamentos formais foram encontrados na documentação pequisada quase que lateralmente, como exceções à regra. As duas únicas exceções, já destacadas, são os loteamentos da Cidade Luz, projetado neste período mas realizado décadas depois, e o controverso loteamento de Amaralina. Do primeiro muito já se disse e pesquisou CITAR BIBLIOGRAFIA, o que nos permite centrar a atenção no mais controverso.

INSERIR PLANTA DO LOTEAMENTO DA FAZENDA SANTA CRUZ

A totalidade da bibliografia sobre o desenvolvimento urbano de Salvador data o loteamento Cidade Balneária Amaralina como sendo de 1933, tendo como única fonte um levantamento feito pela Prefeitura \cite{salvador_loteamentos_1977}. 

INSERIR PLANTAS DE OBRAS DO LOTEAMENTO DA FAZENDA AMARALINA (1933)

De certa forma, estão corretos: está custodiada no Arquivo Histórico Municipal, ainda sem catalogação, a planta original deste loteamenteo de 1933. 

Acontece que este pode ter sido não o \textit{primeiro}, mas \textit{um} entre \textit{muitos} loteamentos sucessivos da velha fazenda Alagoa; o problema quanto a estes outros loteamentos, relativamente ao objeto daquele levantamento dos anos 1970, é que suas \textit{plantas} ou bem não sobreviveram à ação do tempo, ou foram perdidas, ou ainda não foram catalogadas e por isto mesmo não se consegue ter acesso a elas. 

A única forma de saber da existência de loteamentos mais antigos da fazenda Alagoa é a análise da totalidade dos pedidos de licença de obra relativos à área; por este meio foi possível verificar que o mais antigo entre tais pedidos, datado de 04 nov. 1897, refere-se ao ``lote 43'' da ``Cidade Balneária Amaralina'', e catorze outros pedidos relativos à mesma área, indicando o mesmo loteamento, cobrem o período que vai de 1897 a 1925\footnote{\textbf{BR-BAAHMS}, Fundo ``Intendência e Prefeitura'', Série ``Processos de Licenciamento de Reforma e Ampliação de Edificações'', Subsérie ``Requerimentos e Plantas -- Brotas'', caixa 19, documentos diversos. São eles, em ordem crescente de data: ``lote 43'' (04 nov. 1897); ``lote 49'' (12 fev. 1898); ``lotes 11 a 13'' (19 dez. 1901); ``lotes 69 e 70'' (12 set. 1910), empregue para a construção de seis casas; ``lote 81''(15 out 1910); ``lotes 8 e 64'' (28 abr 1910); ``lote 26'' (21 out. 1910); ``lote 143'' (26 dez. 1910); ``lote 142'' (09 mar. 1911); ``lotes 370 e 371'' (16 dez. 1911); ``lote 146'' (14 abr. 1913); ``lote 135'' (18 mar. 1914); ``lote 198'' (25 nov. 1915); ``lote 116'' (28 jul. 1915); ``lote 13'' (15 set. 1915), objeto de nova licença catorze anos depois da primeira; ``lote 28'' (out. 1925), do qual restou apenas a planta muito danificada; ``lote 393'' (02 mar. 1925). Na caixa 24 do mesmo fundo, série e subsérie há um pedido relativo ao ``lote 253'' (21 fev. 1925).}. Deste modo, se o levantamento da Prefeitura está correto em indicar a planta de 1933 como sendo talvez a mais antiga a ter chegado aos dias atuais, pode ser que a restrição do objeto do levantamento tenha impedido pesquisas mais aprofundadas em busca de outros documentos que não plantas de situação, resultando em que estes loteamentos anteriores da fazenda Alagoa permaneceram fora do conhecimento público até o presente momento.

INSERIR PLANTA DE A. SAFFREY

INSERIR PLANTA DE LYDIA DEWALD

INSERIR PLANTA DA VILA MARIA

INSERIR PLANTA DE CHEHADI KRAYCHETE

INSERIR PLANTA DO CHALÉ DE JOÃO DE MATTOS

INSERIR PLANTA DO CHALÉ DE JOÃO DA CUNHA FREIRE

INSERIR PLANTA DSC04835

No que diz respeito à imprensa, data de 1898 aquele que talvez seja o primeiro uso público do nome ``Amaralina'', quando da publicação de um decreto autorizativo da ``livre creação de gado \textit{vaccum}, cavallar,  lanígero e caprino na zona da costa do districto de Brotas comprehendida entre a fazenda Lagoa e o rio das Pedras exclusive as povoações de Amaralina e Mariquita''\footnote{\textbf{Jornal de Notícias}, ano XX, nº 5612, 23 set. 1898, p. 1}. A “fazenda Lagoa” e a “povoação de Amaralina” aparecem aí separados, como a indicar, possivelmente, que apenas parte da fazenda foi loteada, e não a fazenda inteira. Parece ser esta a hipótese mais plausível, como se verá. 

Ainda em 1912 a Companhia de Melhoramentos, em data muito próxima do anúncio da construção da avenida Oceânica por esta mesma companhia, mandou publicar no \textbf{Diário de Notícias} (12 jul. 1912) anúncio convocando os arrendatários e foreiros de terrenos situados na fazenda Alagoa ``a comparecerem no seu escritório, a fim de exibirem os seus títulos para registro nos livros da Companhia''; a fazenda serviria de garantia para um empréstimo de 6:000\$000 contraído por intermédio de seu diretor Francisco Marques de Góes Calmon para capitalizar a empresa \cite[p.~123]{CUNHA2011}. Mais adiante se verá por que não se pode dizer que esta empreiteira comprou \textit{toda} a fazenda Alagoa, mas \textit{parte} dela.

Em 1915, o major José Custódio da Silva, conhecido pela disputa com o velho José Álvares do Amaral em torno da propriedade da fazenda Alagoa, resolveu entrar no negócio dos loteamentos.  Deu entrada num pedido de licença para o loteamento chamado \textit{Cidade Balnear Hormendina} em terrenos de sua propriedade na fazenda Ubarana, resultando em curiosa negociação. Para facilitar a aprovação de seu loteamento, José Custódio fereceu gratuitamente ``um lote dessas terras ao Município para um predio escolar, e, ouvida a seção de higiene municipal no processo de licenciamento, esta exigiu seis lotes''; o loteador respondeu com uma oferta de três lotes de 24mX30m, ``desde que sejam concedidos ao suplicante os mesmos favores feitos ao sr. Dr. Bernardo Jambeiro, na concessão que obteve para as suas terras nas Quintas da Barra''; depois disto, não houve qualquer outro despacho no processo de licença\footnote{\textbf{BR-BAAHMS}, Fundo ``Intendência e Prefeitura'', Série ``Processos de Licenciamento de Reforma e Ampliação de Edificações'', Subsérie ``Requerimentos e Plantas -- Brotas'', caixa 19.} -- mas a imprensa baiana, sob o título ``O concelho acha que nós não precisamos mais de melhoramentos'', comentou o assunto, dizendo que José Custódio ``não deve estar muito satisfeito com os edis'' porque ``o concelho rejeitou o projeto''\footnote{\textbf{A Notícia}, ano I, nº 145, 15 mar. 1915, p. 145.}.

INSERIR PLANTA DA CIDADE BALNEÁRIA HORMENDINA

Em 1927 noticiava-se a doação pela Intendência de 10 mil paralelepípedos aos construtores Nivaldi, Allioni \& Cia., ``encarregados da construcção da futura cidade balneária em Amaralina''\footnote{\textbf{A Capital}, ano I, nº 109, 11 fev. 1927, p. 2.}. Teria fracassado o primeiro loteamento da “cidade balneária”? Teria sido necessário lançar novamente o loteamento ao público? Ora, se já em 1893 a “cidade balneária” era referenciada nos pedidos de licença, e não foram poucos a fazê-lo daí em diante, pode-se excluir de imediato a hipótese de fracasso do primeiro loteamento. Parecem mais plausíveis duas outras hipóteses: a de que esta nova “cidade balneária” seria ou uma \textit{extensão} ou \textit{continuidade} do loteamento já existente, ou que se trataria de \textit{outro loteamento com mesmo nome}. De toda forma, pode-se afirmar com alguma certeza: esta ``futura cidade balneária'' é o loteamento de 1933.

Que pode dizer a respeito deste loteamento o inventário de Maria Amália Paraíso do Amaral, esposa supérstite de José Álvares do Amaral? 

Sendo a viúva e herdeira do finado José Amaral, Maria Amália era proprietária de várias casas na rua Visconde de Itaboraí e “de meia parte dos terrenos de Amaralina”, sendo eles “o conjunto do loteamento já aprovado pela Prefeitura da Capital” e a outra parte os “terrenos e mais a casa e Igreja atualmente ocupados pelo Exército Nacional”, onde se instalara o “quartel de forças do mesmo Exército para defesa da costa”\footnote{\textbf{BR-BAAPB}, fundo “Judiciário”, seção “Inventários”, estante 6, caixa 2721, folha 255, documento 2.}. A conhecida “viúva Amaral” foi, portanto, a última proprietária conhecida da fazenda Alagoa/Amaralina. Em seu inventário podemos acompanhar, passo a passo, o fracionamento de uma das herdades históricas do distrito de Brotas. 

Os terrenos do segundo loteamento Cidade Balneária Amaralina foram transmitidos \textit{mortis causa} para José Ignacio do Amaral e Maria Emília Paraíso do Amaral\footnote{\textbf{BR-BAAPB}, fundo “Judiciário”, seção “Inventários”, estante 6, caixa 2721, folha 255, documento 2.}, filha de
José Álvares do Amaral e tão bem colocada na sociedade baiana quanto o resto de sua família\footnote{Um exemplo de sua rede de relacionamentos: quando de seu casamento com Julio Muniz Barretto em 6 de setembro de 1913, serviram como testemunhas, no religioso e no civil, figuras como Pedro Velloso Gordilho, Julieta Maia de Góes Calmon, Francisco Marques de Góes Calmon, Horácio Lucatelli Dórea e Francisco Eloy Paraíso Jorge (\textbf{Gazeta de Notícias}, ano III, nº 297, 6 set. 1913, p. 2).}. O inventário, ajuizado em 13 de junho de 1946, foi reaberto ainda várias vezes, e arrastou-se assim entre os arquivos e os cartórios em meio a disputas com o Exército, partilhas e sobrepartilhas até 1982, quando José do Amaral Moniz Barreto movimentou-o pela última vez em torno de uma questão escriturária. 

INSERIR PLANTA DA CIDADE LUZ

\subsubsection{Loteamentos e parcelamentos informais}

A existência dos loteamentos e parcelamentos formais em Brotas deu-se, ontem como hoje a julgar pela documentação pesquisada, em meio a número expressivo de loteamentos e parcelamentos \textit{informais}. 

Torna-se necessário destacar e ressaltar a importância deste fenômeno pela grande dificuldade de acompanhar e situar a desagregação e desmembramento de algumas das grandes herdades outrora existentes no distrito, ao que tudo indica por força de sua pulverização em pequenos lotes, avenidas, vilas e que tais. É precisamente a informalidade, entretanto, que dificulta rastrear, situar e mapear estes processos – embora não seja tarefa totalmente impossível.

Um exemplo clássico de loteador informal é \textit{José Visco}, que durante algum tempo emprestou seu nome à ladeira do Funil, no Engenho Velho de Brotas. Apenas para que fique demonstrada a importância de José Visco na produção do espaço urbano de Brotas, basta dizer que a ladeira do Pepino durante muitas décadas foi batizada com seu nome, e mesmo uma pesquisa em ferramentas de GPS atuais com a expressão ``José Visco'' ainda aponta para esta rua.

Tudo começa com a autorização dada em 1912 pelo conselho municipal a José Visco para construir uma ligação entre a rua Uruguaiana e a Estrada 2 de Julho através de terrenos de sua propriedade\footnote{\textbf{Gazeta de Notícias}, ano III, nº 42, 25 out. 1912, p. 3.}.

Ao que parece, tudo legal, formal e regular. Os pedidos de licença, por outro lado, mostram situação um tanto diferente.

Em março de 1911, o engenheiro da Diretoria de Obras, Manoel Alves Nazareth, recomendou ``que o Comissário do Distrito de Brotas vistorie a referida roça [de José Visco], a fim de verificar as inúmeras construções que estão sendo feitas sem licença'', além de exigir de José Visco a apresentação da ``planta geral da roça''\footnote{\textbf{BR-BAAHMS}, Fundo ``Intendência'', Série ``Processos de Licenciamento de Reforma e Ampliação de Edificações'', Subsérie ``Requerimentos e Plantas – Brotas'', caixa 20, processo 4264 folha 65, datado de 31 mar. 1911.}

Maiores detalhes da atividade imobiliária de José Visco serão vistos adiante (sub-seção 4.2.5, p. 346), assim como seu grande impacto no desenvolvimento urbano de Brotas. Importa apenas destacar, como prévia, o caráter informal dos loteamentos deste terratenente.

O inventário de José Visco, aberto em 1929, demonstra as benesses e complicações de sua atividade imobiliária. Ali está registrado o tamanho da fazenda Boa Vista:

\begin{citacao}
Uma roça denominada ``Chácara Boa Vista'', freguesia de Brotas d’esta Cidade, com os seguintes limites: frente para a Rua Frederico Costa, fundo no Dique Pequeno e Estrada Dous de Julho; lado direito com terreno de Raul Braga, Dique Pequeno; lado esquerdo, com terreno de Durval Leite, na Estrada Dous de Julho. Nos terrenos d’essa Chácara existe uma rua ligando a Rua Frederico Costa à Estrada Dous de Julho, onde foram divididos em vários lotes e d’esses, alguns, vendidos e outros arrendados com casas de rendeiros. Outras benfeitorias e terrenos existem na dita Chácara, tendo sido também alguns d’estes vendidos e outros arrendados. A casa que servia de residencia da dita chácara foi vendida pelo de cujus. Na chácara existem as seguintes casinhas pertencentes ao acervo a serem avaliadas. Lettra H: n o 32, 34, 40, 36, 38, 42 e 44. Lettra C: 4, 6, 8, 12, 14, 16, 18, 20, 22, 26, 28, 30, 32, 34, 36, 38, 40, 42, 50, 52, 2, 10, 24, 54. Lettra B: n o 4. Lettra D: n o 1, 2, 3, 58, 62, 64, 66, 68, 70, 76, 78, 80, 84, 88, 90, 92, 94, 96, 98, 100, 102, 104, 56, 82, 86, 60, 72, 74, 106, 108. Lettra N: n o 2, 22, 26, 28, 30, 34, 32, 36, 38, 42, 42, 44, 24, 20. Lettra P: n o 2, 4, 10, 12, 6, 8, 14\footnote{\textbf{BR-BAAPB}, fundo ``Judiciário'', série ``Inventários'', estante 7, caixa 2792, folha 97, documento 25.}.
\end{citacao}

As ``casinhas'' ainda de propriedade de José Visco quando de sua morte, cedidas via aluguel ou arrendamento a moradores diversos, eram oitenta e três. Se estes eram os lotes remanescentes de um conjunto identificado por ordem alfabética, pode-se imaginar quantas outras não foram vendidas.

Descobre-se também no inventário que José Visco residia num sobrado avaliado em 30:000\$000 na rua General Labatut, nos Barris, distrito de São Pedro; sua atividade imobiliária em Brotas era puramente rentista e especulativa. Este terratenente acumulava também casas na rua da Oração (distrito da Sé), na rua São Raimundo (distrito de São Pedro), na ladeira da Poeira (distrito de Nazaré)\dots Comparativamente, os terrenos da chácara Boa Vista foram avaliados em 10:000\$000; as casas aforadas na chácara também foram avaliadas: as casas de taipa, porta e janela do lote ``H'' em 500\$000 por casa, as casas semelhantes nos lotes ``B'' e ``D'' foram avaliadas em 400\$000 cada, e as do lote ``N'' em 700\$000 cada\footnote{\textbf{BR-BAAPB}, fundo ``Judiciário'', série ``Inventários'', estante 7, caixa 2792, folha 97, documento 25.}. O baixo valor das casas indica bem a quem se direcionavam: os mais pobres, os proletários, os trabalhadores braçais e artesanais dispostos a residir nas cercanias da área urbanizada de Salvador a preços módicos.

A ``novela'' deste inventário não acabou com o formal de partilha. Ao longo dos anos o inventário foi reaberto ainda outras vezes. Em 1939, já encerrado o inventário há muitos anos, foi acostado aos autos um pedido de concessão de alvará para que o comprador João Baptista de Lima pudesse lavrar a escritura da casa que comprara em 1927 ao falecido. Em 1951 o inventário foi novamente reaberto, desta vez por Antonio Anacleto Tavares, para finalidade semelhante, e pelos mesmos motivos\footnote{\textbf{BR-BAAPB}, fundo ``Judiciário'', série ``Inventários'', estante 7, caixa 2792, folha 97, documento 25.}.

INSERIR OS POSTAIS DO DIQUE E DA VASCO DA GAMA

Não bastasse a atividade loteadora informal, há também indícios de ocupação informal em Brotas, mas sua comprovação e mapeamento depende do cruzamento das informações dos loteamentos informais constantes nos pedidos de licença com outras constantes nos autos de infração da Diretoria de Obras de Salvador, que não foram adotados como base para esta pesquisa. Há também indícios desta mesma prática a serem verificados com atenção nos registros oficiais. A apartação social resultante da abolição da escravatura sem quaisquer cuidados com a inserção dos negros libertos na economia em outro papel senão o de braços livres para a exploração capitalista certamente terá empurrado grandes contingentes para a ocupação informal de terras e a construção informal de habitações, como se nota no Livro de Décimas Urbanas de 1893: lá nas mais extremas lonjuras, como um certo ``Matheus (africano)'', de um certo ``Thomas (africano)'' e de um certo ``Ignacio (africano)'' residentes na localidade conhecida como ``Pomar''.

\subsection{Construções, ampliações, reformas: urbanização desigual e a passos lentos}\label{subsec:constrampliref}

Se os investimentos públicos preparam a infraestrutura urbana, é preciso ver se tais investimentos foram exitosos ou baldados. Terão conseguido direcionar o desenvolvimento urbano em Brotas? Ou terá ele seguido outras direções?

O governo da Bahia, expondo resultados da estatística predial de 1892, dizia haver em Salvador 12.649 prédios, dos quais 307 encontravam-se em ruínas ou em construção, distribuídos estes últimos da seguinte forma: ``nas freguesias da Sé 1, Conceição da Praia 13, Victoria 60, Sant'Anna 56, Pilar 24, Mares 30, Rua do Passo 9, Brotas 31, S. Pedro 8, Penha 18, Santo Antonio 57'' \cite[relatório da inspetoria de higiene, p.~6]{bahia_rpe_1893}. Outro relatório, da Intendência Municipal, afirmava haver em Salvador 22.556
imóveis, distribuídos conforme consta na \autoref{tab:estpredom1915} (p. \pageref{tab:estpredom1915}). Vê-se por aí não apenas o aumento de 78,32\% no total de imóveis da cidade, mas também Brotas em quarto lugar em número de imóveis e em proporção de isenções de décimas urbanas, em terceiro lugar em número de imóveis em construção (ou em segundo, se considerarmos a proporção de imóveis em construção frente ao total) e em sexto lugar em número de ruínas (oitavo, se apenas a proporção de ruínas frente ao total for considerada). Brotas, na Primeira República, parece ser, ao menos nas estatísticas, um enorme canteiro de obras.

\begin{table}[!htp]
\centering
\IBGEtab{
\caption{Estatística predial soteropolitana (1915)}\label{tab:estpredom1915}}
{\begin{tiny}
\begin{tabular}{ccccc}
\toprule
Distrito			&Imóveis	&Isentos	&Ruínas	&Em construção	\\
\midrule \midrule
Santo Antônio		&4163		&32			&33		&180			\\
Vitória				&3952		&46			&24		&110			\\
Penha				&2534		&34			&32		&45				\\
Brotas				&2463		&121		&13		&104			\\
São Pedro			&1844		&27			&9		&7				\\
Santana				&1712		&51			&15		&18				\\
Mares				&1644		&74			&3		&8				\\
Nazaré				&1145		&62			&8		&12				\\
Pilar				&1042		&37			&16		&10				\\
Sé					&965		&66			&8		&7				\\
Paço				&639		&24			&1		&2				\\
Conceição da Praia 	&453		&71			&3		&0				\\
\bottomrule
\end{tabular}
\end{tiny}}
{\fonte{Elaboração do autor, com base no relatório de 1915 do intendente de Salvador ao Conselho Municipal \cite[p.~68]{salvador_relatorio_1916}.}}
\end{table}

A estatística demográfica já vista no \autoref{cap:1}, conquanto não se possa comparar diretamente com a estatístca predial, parece indicar notável pressão demográfica a influir no incremento no número de construções, ampliações e reformas em Brotas. 

Neste ponto do trabalho se tentará avaliar o impacto quantitativo das reformas urbanas de 1912-1916 sobre a produção, apropriação e uso do espaço urbano de Brotas, subdividindo a análise nas áreas definidas no início deste capítulo. Se a agitação do mercado imobiliário nas áreas diretamente afetadas pelas reformas urbanas de J. J. Seabra houver reverberado até Brotas, abrem-se duas hipóteses. A primeira (\textit{hipótese forte}) se verificará caso ocorra um aumento intenso no número de pedidos de licença para construção, ampliação e reforma em Brotas no período 1911-1920, ou seja, nos anos da reforma no Centro e aqueles imediatamente posteriores; a segunda (\textit{hipótese fraca}) se verificará caso ocorra um incremento gradual no número de pedidos de licença entre 1911 e 1920. Pode ser que nenhuma das duas hipóteses se verifique, e que não tenha havido incremento no número de pedidos de licença nos anos da reforma; esta \textit{hipótese negativa}, se verificada, infirmará a suposição da influência das reformas seabristas em Brotas, devendo ser seu desenvolvimento urbano explicado por outros processos sociais, políticos e econômicos.

Em todas as situações se tentará verificar, por área, a ocorrência dos processos de \textit{centralização}, \textit{descentralização}, \textit{magnetismo funcional}, \textit{segregação}, \textit{sucessão} e \textit{inércia} em meio à urbanização de Brotas (cf. \autoref{subsec:sociogeogrurb} (p. \autoref{subsec:sociogeogrurb}) para retomar os conceitos). Além disto, o estilo arquitetônico dos prédios construídos em cada área do distrito será brevemente comentado, relacionando-o com o debate sobre o \textit{caráter socialmente distintivo da estética arquitetônica} já feito no \autoref{cap:1}.

Antes de prosseguir, algumas palavras sobre a legislação aplicável e sobre certa tendência a considerar o desenvolvimento urbano de Salvador (e de Brotas) durante a Primeira República como ``desordenado''

Um pesquisador pregresso dos mesmos pedidos de licença disse que a falta de um ``código'' de obras ou de posturas ``implicava, de certo modo, no aumento da burocracia'', pois as plantas eram submetidas ``à apreciação de diversas diretorias e ao julgamento, quase que pessoal, de funcionários nem sempre qualificados para tal atribuição, o que, obviamente, resultava em grande perda de tempo para o interessado na construção'' \cite[p.~89]{cardoso_vilas_1991}. A documentação consultada confirma a burocratização das licenças e as atribulações dos requerentes, mas infirma a alegada desqualificação técnica dos burocratas analistas. Já se viu como \textit{todos} os pareceristas da Diretoria Municipal de Obras eram \textit{engenheiros}, e como \textit{todos} os pareceristas da Inspetoria Municipal e da Diretoria Estadual de Higiene eram \textit{médicos}, \textit{farmacêuticos} ou \textit{cirurgiões}, quando não também \textit{engenheiros}. 

De outro lado, a ausência de um código municipal de posturas até 1920 e de um código municipal de obras até 1926 não significou de modo algum o arbítrio completo dos burocratas: uma análise comparada do código de posturas de 1920 com os três livros de posturas municipais referentes ao período 1889-1930 sob custódia do Arquivo Histórico Municipal demonstra que quase nada se inovou com o código, que via de regra sistematizou numa só legislação posturas municipais antes esparsas, e o código de obras de 1926, se inovou no zoneamento urbano, de resto seguiu todos os preceitos construtivos e arquitetônicos já indicados pelo regulamento sanitário de 1906, pelas posturas municipais que o precederam e pela prática reiterada dos burocratas da Diretoria Municipal de Obras. Por isto mesmo, não está correta a impressão generalizada de um \textit{laissez-faire} urbanístico anterior aos códigos de posturas e de obras: não apenas havia legislação de controle do uso e ocupação do solo precedente aos dois em pelo menos dezessete anos (cf. a Resolução 160/1903 já citada), constantemente aplicada pelos burocratas da Diretoria de Obras, como normas municipais correlatas disciplinavam tanto os aspectos construtivo e arquitetônico (regulamento sanitário municipal de 1906) quanto usavam as isenções da décima urbana como incentivo ao melhor aproveitamento dos imóveis (construção em terrenos baldios, reformas de prédios arruinados, provisão de moradia para ``operários'' e ``pobres'' etc.). Esta impressão é, por assim dizer, uma projeção, na historiografia, da afirmação corrente, de senso comum, acerca de uma suposta ``falta de planejamento'' na produção do espaço urbano soteropolitano. Ora, assim como todos quantos tenham pesquisado a evolução da urbanização soteropolitana no século XX demonstram tanto uma sucessão de planos e projetos quanto, paralelamente, a ocupação e uso do espaço urbano pautada pela relação dialética entre as necessidades habitacionais dos trabalhadores mais pauperizados e a extração da renda da terra pelos latifundiários urbanos e promotores imobiliários por todos os meios, legais ou não \cite{BRANDAO1978, BRANDAO1978a, BRANDAO1980, CARVALHO1974, GORDILHO-SOUZA2008, SAMPAIO1999, VASCONCELLOS1974, VASCONCELOS2002}, esta mesma relação dialética se verifica na produção do espaço urbano de Salvador na Primeira República -- e Brotas era parte deste processo. 

Os loteamentos e parcelamentos informais eram fenômenos corriqueiros no desenvolvimento urbano de Brotas, às vezes até sacramentados pela lei como no caso de José Visco; a produção informal de moradia, difícil como seja de se deixar capturar pelas lentes da historiografia, encontrava-se igualmente neste processo. Seu fundamento não foi a falta de normas urbanísticas, não foi a ausência de mecanismos legais de controle da ocupação e uso do solo. Foi, isto sim, a combinação entre a pressão demográfica herdada em Salvador do processo de transição da escravidão para o trabalho assalariado; a expulsão das antigas moradias senhoriais e a consequente pressão pela produção de novas moradias e por novos alugueis, aforamentos e arrendamentos\footnote{Uma das características do modo de produção escravista é a obrigação por parte dos senhores de fornecer moradia para os escravos; no caso soteropolitano, a isto somava-se, em seguida à repressão contra os hauçás insurrectos de 1835, a proibição a todos os africanos e crioulos de apropriarem-se legalmente de bens de raiz. Extinta a escravidão, extinguiu-se com ela também a obrigação dos senhores de fornecer abrigo àqueles que antes escravizaram. Estes elementos configuram a hipótese de uma tendência de os recém-libertos pressionarem tanto a produção de novas moradias quanto o mercado de alugueis, arrendamentos e aforamentos na cidade.}; a necessidade habitacional de uma massa proletária extremamente pauperizada; foi tudo isto o que causou o ``crescimento desordenado'' de Salvador, e não a ausência de normas disciplinadoras deste processo.

Isto dito, é preciso registrar ainda um último cuidado metodológico antes de prosseguir. 

Um trabalho realmente minucioso em torno dos pedidos de licença exigiria a comparação entre os processos custodiados no \textbf{BR-BAAHMS} e a publicação na imprensa baiana da deliberação da Intendência acerca de cada processo (ao menos a \textbf{Gazeta de Notícias} veiculou tais deliberações entre 1912 e 1914, e \textbf{A Notícia} entre 1914 e 1915). Não obstante, interessam a esta pesquisa menos a certeza da aprovação que a \textbf{quantidade de pedidos}, a \textbf{natureza da obra pretendida}, a \textit{planta da obra} e seu \textit{estilo arquitetônico}, a \textit{data} do pedido, o \textit{teor dos pareceres} emitidos pela Diretoria de Obras e pela Inspetoria de Higiene, enfim, interessam a esta pesquisa informações sobre as quais a aprovação ou rejeição do pedido, conquanto importante, tem pouco impacto.

\subsubsection{Antigo 1º Distrito}

\begin{table}[!htp]
\IBGEtab{
\caption{Pedidos de licença por rua e década (antigo 1º distrito)}\label{tab:rd-1dist}}
{
\begin{tiny}
\begin{tabular}{llllll}
\toprule
\multirow{2}{*}{Logradouro}	& \multicolumn{4}{c}{Número de licenças}	& \multirow{2}{*}{TOTAL}\\
\cline{2-5}
	&1889-1900	&1901-1910	&1911-1920	&1921-1930	& \\
\midrule
\midrule
Ladeira da Fonte Nova	&0	&1	&6	&4	&11\\
Rua do Fabrício	&0	&4	&6	&8	&18\\
Ladeira dos Galés	&2	&4	&6	&9	&21\\
Ladeira dos Paranhos	&2	&2	&2	&1	&7\\
Largo das Sete Portas	&1	&1	&3	&4	&9\\
Rua do General	&0	&0	&1	&2	&3\\
Rua Agripino Dórea (ant. das Pitangueiras)	&0	&3	&8	&8	&19\\
Rua das Pitangueiras	&3	&1	&1	&4	&9\\
Rua do Socorro	&0	&0	&0	&0	&0\\
Rua da Alegria (Castro Neves)	&4	&6	&7	&14	&31\\
Rua Primeiro de Março (ant. das Pitangueiras)	&2	&1	&0	&1	&4\\
Rua Castro Neves	&5	&5	&10	&16	&36\\
Rua do Sangradouro	&5	&9	&10	&8	&32\\
Rua do Bigode	&0	&0	&0	&1	&1\\
Rua Santo Agostinho (ant. do Bigode)	&1	&0	&0	&0	&1\\
Rua das Sete Portas	&0	&0	&0	&0	&0\\
\midrule
TOTAL	&25	&37	&60	&80	&202\\
\bottomrule
\end{tabular} 
\end{tiny}
}
{\fonte{Elaboração do autor com dados de \textbf{BR BAAHMS}, Fundo ``Intendência e Prefeitura'', Série ``Processos de Licenciamento de Reforma e Ampliação de Edificações'', Subsérie ``Requerimentos e Plantas -- Brotas'', vários documentos nas caixas 1 a 24.} }
\end{table}

A \autoref{tab:rd-1dist} demonstra que no antigo 1º Distrito de Brotas o incremento no número de licenças foi contínuo, mas em taxas declinantes. Manteve-se a tendência inercial de urbanização acelerada nas Pitangueiras, no Castro Neves, nos Galés, no Sangradouro e na Ladeira do Fabrício. De igual modo, manteve-se a tendência desta área configurar-se como morada de burocratas de pequeno e médio escalão, somando-se a eles agora os comerciantes da avenida J. J. Seabra (a antiga Rua da Vala), para quem a relativa proximidade com seus locais de trabalho atuava como fator de magnetismo funcional.

Em 1903 as terras remanescentes da aquisição do solar do Castro Neves para a transferência do Hospital Militar entraram em pauta: o ministro da Fazenda pediu ao ministro da Guerra o levantamento de uma planta de situação ``de todo o terreno do proprio nacional em que funcciona o referido hospital'', discriminando-se nela ``a parte necessaria e a desnecessaria ao estabelecimento'' para que ficasse a última ``à disposição do ministerio da fazenda''\footnote{\textbf{Correio do Brasil}, ano I, nº 31, 29 set. 1903, p. 1.}; era o movimento que restava para que toda a extensão do Castro Neves fosse urbanizada.

Alguns imóveis podem ser destacados do conjunto, assim como os requerentes de
pedidos a eles relativos.

No Sangradouro, o comerciante e ``definidor'' da Ordem Terceira de São Francisco, Joaquim Ignacio Ribeiro dos Santos (69 anos em 1877), que já havia sido dispensado do pagamento da décima urbana sobre os terrenos de sua roça em 1875 \cite[p.~226]{bahia_assembleia_1875}, 

HOSPITAL MILITAR DO CASTRO NEVES



Elano Alfredo Durham

Manoel Amoedo y Amoedo, comerciante espanhol, 

\textit{Raymundo da Cunha Pacheco} e \textit{Durval Alves Fernandes} eram sócios na firma de materiais de construção \textit{Raymundo da Cunha Pacheco \& Comp.}\footnote{\textbf{Gazeta de Notícias}, ano III, nº 162, 25 mar. 1913, p. 3.}

José Antonio Ramos, comerciante (20 casas no Castro Neves!)

Empresa Construtura Ltda (lad Fabr)

\textit{Antonio Francisco dos Passos}, comerciante da casa exportadora \textit{Rossbeck Brothers}\footnote{\textbf{Correio do Brasil}, ano I, nº 46, 16 out. 1903, p. 2.}, tinha aí residência, onde ``guardava [\dots] o leito devido a uma violenta hepatite aguda da succumbio [\dots] na idade de 41 annos, deixando viuva e 11 filhos em orphandade''\footnote{textbf{Correio do Brasil}, ano I, nº 100, 21 dez. 1903, p. 2.}. 

Francisco de Assis Sepúlveda, despachante da Seção de Estatística Comercial da Alfândega \cite[p.~306]{reis_almanak_1898}, residia na ``rua do Bigode'' (Santo Agostinho), numa casa DESCREVER

João Ferraro Vaz

José Celestino dos Santos

Domingos Pacheco Leite

O major Emílio Cassiano da Silva\footnote{\textbf{Gazeta de Notícias}, ano III, nº 200, 10 maio 1913, p. 2}, integrante da mesa administrativa da Venerável Ordem Terceira do Carmo\footnote{\textbf{A Manhã}, ano I, nº 159, 17 out. 1920, p. 2} e da Sociedade Filantrópica dos Artistas\footnote{\textbf{Gazeta de Notícias}, ano III, nº 242, 01 jul. 1913, p. 2}, comerciante e proprietário de moinho \cite[p.~433]{reis_almanak_1898}, tinha uma venda na rua do Sangradouro\footnote{\textbf{Jornal de Notícias}, ano XIV, nº 3894, 13 nov. 1892, p. 2}. DESCREVER

Outro ilustre residente desta área era ninguém menos que o diretor da Inspetoria Municipal de Higiene, o médico Antonio do Amaral Ferrão Muniz

Manoel Correia Machado, comerciante de charutos na ``Loja Havaneza''\footnote{\textbf{Gazeta da Bahia}, ano VIII, nº 276, 17 dez. 1886, p. 4.} situada na rua J. J. Seabra\footnote{\textbf{A Notícia}, ano I, nº 292, 14 set. 1915, p. 2.} (destruída por um incêndio em 1915 e logo reconstruída\footnote{\textbf{A Notícia}, ano I, nº 284, 02 set. 1915, p. 1.}); e Augusto Correia Machado, proprietário da Mútua 7 de Setembro\footnote{\textbf{Correio do Povo}, nº 92, 04 ago. 1925, p. 1.} e da Casa Comercial 7 de Setembro\footnote{\textbf{A Capital}, ano I, nº 10, 30 set. 1926, p. 6.}, ambas situadas na rua J. J. Seabra, nº 184; ambos tinham imóvel no Sangradouro. 

Álvaro Pimenta da Cunha, farmacêutico da ``Pharmacia Universal'' \cite[p.~510]{reis_almanak_1903}, 


A tudo isto deve-se somar a \textit{Vila Operária da União Fabril}, no Sangradouro. DESCREVER (Sangr)



\subsubsection{Boa Vista / Engenho Velho}

\begin{table}[!htp]
\IBGEtab{
\caption{Pedidos de licença por rua e década (Boa Vista / Engenho Velho)}\label{tab:rd-boavista}}
{
\begin{tiny}
\begin{tabular}{llllll}
\toprule
\multirow{2}{*}{Logradouro}	& \multicolumn{4}{c}{Número de licenças}	& \multirow{2}{*}{TOTAL}\\
\cline{2-5}
	&1889-1900	&1901-1910	&1911-1920	&1921-1930	& \\
\midrule
\midrule
Av. Frederico Costa	&0	&0	&0	&23	&23\\
Engenho Velho	&3	&2	&15	&8	&28\\
Ladeira/Fazenda do Teixeira	&3	&5	&10	&0	&18\\
Rua do Saveiro	&0	&0	&4	&2	&6\\
Rua José Visco (antiga Ladeira do Pepino)	&0	&0	&2	&0	&2\\
Rua/Ladeira do Pepino	&0	&0	&0	&12	&12\\
Rua Uruguaiana	&1	&10	&15	&6	&32\\
Rua da Capelinha / Capela do Deus Menino	&0	&3	&19	&4	&26\\
Boa Vista	&10	&17	&19	&0	&46\\
Rua Monte Belém do Meio	&0	&0	&7	&2	&9\\
Vila América	&0	&0	&4	&6	&10\\
Dique Pequeno	&0	&0	&1	&3	&4\\
\midrule
TOTAL	&17	&37	&96	&66	&216\\
\bottomrule
\end{tabular} 
\end{tiny}
}
{\fonte{Elaboração do autor com dados de \textbf{BR BAAHMS}, Fundo ``Intendência e Prefeitura'', Série ``Processos de Licenciamento de Reforma e Ampliação de Edificações'', Subsérie ``Requerimentos e Plantas -- Brotas'', vários documentos nas caixas 1 a 24.}}
\end{table}



A área da Boa Vista / Engenho Velho de Brotas foi palco de um verdadeiro surto de loteamento informal nas terras de José Visco e dos Teixeira, iniciado entre 1901 e 1910 e arrefecido entre 1921 e 1930. A ocupação foi segregada em função da situação topográfica: a ``pequena e larga'' rua da Boa Vista\footnote{\textbf{Correio do Povo}, ano II, nº 176, p. 1.}, tida como uma das mais elegantes de Salvador PEGAR A CITAÇÃO DE PJD, foi rodeada por habitações proletárias no Moinho, na Vila América, no Dique Pequeno, na rua da Capelinha, no Monte de Belém e em toda a encosta que circunda o solar da Boa Vista.

Com a quantificação da \autoref{tab:rd-boavista} (p. \pageref{tab:rd-boavista}) é possível dimensionar o impacto das operações imobiliárias de José Visco e dos herdeiros de Antônio Teixeira na área: dos 216 pedidos de licença para obras ali situadas, 115 deles foram em terrenos em que José Visco aparecia como aforador, locador ou vendedor/loteador (53,24\% do total), enquanto outros 91 envolveram os herdeiros de Antônio Teixeira em posição similar (42,12\% do total). Somente dez pedidos de licença – \textit{dez}, entre \textit{duzentos e dezesseis}! – não os envolveram.

O loteamento informal por estas bandas tinha características bastante particulares – e especulativas.

Além de José Visco e dos herdeiros de Antônio Teixeira, aparece pela primeira vez nesta pesquisa o nome de \textit{Francisco Ventura}, proprietário de muitos imóveis espalhados em áreas diversas de Brotas. Despachante da Seção de Estatística Comercial da Alfândega \cite[p.~307]{reis_almanak_1898}, membro da Sociedade Beneficente dos Funcionários Públicos\footnote{\textbf{Cidade do Salvador}, ano I, nº 256, 6 dez. 1897, p. 1.}, 

\textit{centralização}, \textit{descentralização}, \textit{magnetismo funcional}, \textit{segregação}, \textit{sucessão} e \textit{inércia}

Surgem nesta área tanto \textit{Raymundo da Cunha Pacheco} quanto seu sócio \textit{Durval Alves Fernandes}. Desta feita, trata-se de casas de negócio – uma delas ainda hoje de pé, mesmo bastante descaracterizada COLOCAR PLANTA E IMAGEM ATUAL LADO A LADO. Outro comerciante a beneficiar da elegância da Boa Vista foi o grossista Francisco Amado Bahia, dono de um açougue na área COLOCAR A FONTE E O PROJETO.

Augusto Deocleciano Carigé, cirurgião-dentista filho do bibliotecário, publicista e abolicionista Eduardo Carigé\footnote{\textbf{Correio do Brasil}, ano III, nº 480, 14 abr. 1905, p. 1.}, 

Affonso Rui de Souza tem casa na Boa Vista

Francisco de Paula Nery

Durval Alves Fernandes

Domingos Pacheco Leite

Manoel Maria do Amaral (Boa Vista)

Santa Casa (Boa Vista)

Agostinho Paiva

Auria Pires de Almeida Couto

Manoel Martins Ribas

João Pereira de Carvalho


\subsubsection{Estrada de Brotas}

\begin{table}[!htp]
\IBGEtab{
\caption{Pedidos de licença por rua e década (Estrada de Brotas)}\label{tab:rd-estbrotas}}
{
\begin{tiny}
\begin{tabular}{llllll}
\toprule
\multirow{2}{*}{Logradouro}	& \multicolumn{4}{c}{Número de licenças}	& \multirow{2}{*}{TOTAL}\\
\cline{2-5}
	&1889-1900	&1901-1910	&1911-1920	&1921-1930	& \\
\midrule
\midrule
Rua do Cemitério	&0	&0	&0	&1	&1\\
Ladeira da Pedra	&2	&0	&3	&3	&8\\
Largo de Brotas	&0	&0	&0	&0	&0\\
Estrada de Brotas	&3	&5	&4	&1	&13\\
Rua D. Pedro II	&0	&0	&0	&0	&0\\
Av. Liberdade	&0	&0	&0	&3	&3\\
Rua/Estrada da Cruz das Almas	&0	&3	&2	&1	&6\\
Ladeira da Cruz da Redenção	&0	&5	&0	&0	&5\\
Estrada das Armações/do Beiju	&0	&3	&10	&3	&16\\
Candeal	&0	&1	&0	&0	&1\\
Avenida Saraiva	&0	&0	&0	&6	&6\\
\midrule
TOTAL	&5	&17	&19	&18	&59\\
\bottomrule
\end{tabular} 
\end{tiny}
}
{\fonte{Elaboração do autor com dados de \textbf{BR BAAHMS}, Fundo ``Intendência e Prefeitura'', Série ``Processos de Licenciamento de Reforma e Ampliação de Edificações'', Subsérie ``Requerimentos e Plantas -- Brotas'', vários documentos nas caixas 1 a 24.}}
\end{table}

\autoref{tab:rd-estbrotas}

O processo de urbanização na estrada de Brotas e arredores foi morno, contínuo e sem sobressaltos – se é que de uma evolução com média de um a dois pedidos por ano se pode dizer que seja realmente um processo de urbanização. As alterações concentraram-se em imóveis situados na própria estrada e na estrada do Beiju / das Armações; neste último caso, principalmente porque o \textit{cemitério de Brotas} era considerado um logradouro seu, e algumas casas foram construídas no seu entorno. Manteve-se portanto seu caráter de zona de transição entre o rural e o urbano, de lugar intermédio entre os extremos do distrito.

A julgar pela tradição historiográfica, deveria ser possível encontrar muitas casas de campo sendo no mínimo reformadas, pois seria esta a vocação do distrito desde o século XIX. Os pedidos de licença mostram, pelo contrário, muitos casebres, casinholas e quase barracos.

\textit{centralização}, \textit{descentralização}, \textit{magnetismo funcional}, \textit{segregação}, \textit{sucessão} e \textit{inércia}

Ursula das Virgens Guaresma faz casa de taipa

Emília Teixeira Couto

Joaquim da Costa Dórea

O coronel Bernardo Firmino Lins\footnote{\textbf{O Combate}, ano I, nº 115, 24 out. 1927, p. 4.}, proprietário de um moinho na estrada de Brotas \cite[p.~433]{reis_almanak_1898}

José Paulino de Carvalho Filho, coronel\footnote{\textbf{A Notícia}, ano II, nº 323, 21 out. 1915, p. 3.}, 

Standard Oil 

Teresa do Amaral Carvalho

Maria Amelia Costa Leite

Capela do cemitério foi feita por João dos Santos Tuvo em 1928

Candeal era roça de Candida Maria Castro dos Santos (cx 24, 13857/75, 28/11/1910)

\subsubsection{Estrada Dois de Julho}

\begin{table}[!htp]
\IBGEtab{
\caption{Pedidos de licença por rua e década (Estrada 2 de Julho)}\label{tab:rd-e2j}}
{
\begin{tiny}
\begin{tabular}{llllll}
\toprule
\multirow{2}{*}{Logradouro}	& \multicolumn{4}{c}{Número de licenças}	& \multirow{2}{*}{TOTAL}\\
\cline{2-5}
	&1889-1900	&1901-1910	&1911-1920	&1921-1930	& \\
\midrule
\midrule
Mata Escura do Rio Vermelho		&0	&4	&4	&6	&14\\
Estrada 2 de Julho				&9	&14	&15	&14	&52\\
Vila Santos						&0	&0	&0	&1	&1\\
Vila América					&0	&0	&4	&6	&10\\
\midrule
TOTAL	&9	&18	&23	&27	&77\\
\bottomrule
\end{tabular} 
\end{tiny}
}
{\fonte{Elaboração do autor com dados de \textbf{BR BAAHMS}, Fundo ``Intendência e Prefeitura'', Série ``Processos de Licenciamento de Reforma e Ampliação de Edificações'', Subsérie ``Requerimentos e Plantas -- Brotas'', vários documentos nas caixas 1 a 24.}}
\end{table}


A documentação pesquisada demonstra que a maioria dos pedidos de licença situados na Estrada Dois de Julho eram, na verdade, situados no Engenho Velho. Como a estrada margeia as encostas desta localidade até encontrar-se com a ladeira dos Galés, foi preciso reorganizar a contagem de pedidos de licença para ter real noção do incremento na construção, reforma ou ampliação de imóveis na área. O resultado desta operação encontra-se na \autoref{tab:rd-e2j} (p. \pageref{tab:rd-e2j}).

\textit{centralização}, \textit{descentralização}, \textit{magnetismo funcional}, \textit{segregação}, \textit{sucessão} e \textit{inércia}

Américo Furtado de Simas, presidente da ``Loja Theosophica Alcyone'', solicitou em 1928 licença para construir um ``pavilhão'' a entidade, projeto simples de autoria de Jaime Furtado de Simas e José Allionni.

José Vicente dos Santos, proprietário de um depósito de carne verde na rua da Vala \cite[p.~402]{reis_almanak_1898}

Maria Theresa da Silva Bottas pediu para construir dez ``casas para operários'' na Vila América\footnote{\textbf{BR-BAAHMS}, Fundo ``Intendência'', Série ``Processos de Licenciamento de Reforma e Ampliação de Edificações'', Subsérie ``Requerimentos e Plantas – Brotas'', caixa 20, processo 1659 folha 3130, datado de 22 maio 1926.}. Parece ter sido uma construção vizinha à de Alice Marinho Bacellar, que pediu licença para construir outras quatro casas semelhantes\footnote{\textbf{BR-BAAHMS}, Fundo “Intendência”, Série “Processos de Licenciamento de Reforma e Ampliação de Edificações”, Subsérie “Requerimentos e Plantas – Brotas”, caixa 20, processo 1678 folha 237, datado de 24 mar. 1926.}

Mata Escura fica dentro da fazenda Teixeira

Maria Augusta Ferreira

Durval de Souza Leite

Crescencio José

Antonio de Souza Pinto (PJD)

Izidoro José dos Santos

Maria Theresa da Silva Bottas

Américo Furtado de Simas

Francisca Xavier França

Alexandre Alves da Silva (PJD)

\subsubsection{Mariquita}

\begin{table}[!htp]
\IBGEtab{
\caption{Pedidos de licença por rua e década (Mariquita)}\label{tab:rd-mariquita}}
{
\begin{tiny}
\begin{tabular}{llllll}
\toprule
\multirow{2}{*}{Logradouro}	& \multicolumn{4}{c}{Número de licenças}	& \multirow{2}{*}{TOTAL}\\
\cline{2-5}
	&1889-1900	&1901-1910	&1911-1920	&1921-1930	& \\
\midrule
\midrule
Rua Banco de Areia	&0	&0	&0	&2	&2\\
Rua do Meio	&0	&0	&0	&0	&0\\
Rua Lucaia	&0	&0	&1	&0	&1\\
Rua das Pedrinhas	&4	&11	&7	&5	&27\\
Rua Fonte do Boi	&6	&0	&0	&0	&6\\
Rua da Mariquita	&0	&0	&0	&0	&0\\
Rua dos Dendezeiros	&0	&0	&0	&0	&0\\
Monte do Conselho	&0	&1	&0	&0	&1\\
Rua do Genipapeiro	&0	&0	&2	&0	&2\\
\midrule
TOTAL	&10	&12	&10	&7	&39\\
\bottomrule
\end{tabular} 
\end{tiny}
}
{\fonte{Elaboração do autor com dados de \textbf{BR BAAHMS}, Fundo ``Intendência e Prefeitura'', Série ``Processos de Licenciamento de Reforma e Ampliação de Edificações'', Subsérie ``Requerimentos e Plantas -- Brotas'', vários documentos nas caixas 1 a 24.}}
\end{table}


\textit{centralização}, \textit{descentralização}, \textit{magnetismo funcional}, \textit{segregação}, \textit{sucessão} e \textit{inércia}

Por falar neste hospital, trata-se de um belo projeto de Júlio Viveiros Brandão, mas seu pedido de licença para construção indica ter sido localizado no ``monte do Menino Jesus'', não no Grão-Mogol\footnote{\textbf{BR-BAAHMS}, Fundo ``Intendência e Prefeitura'', Série ``Processos de Licenciamento de Reforma e Ampliação de Edificações'', Subsérie ``Requerimentos e Plantas -- Brotas'', caixa 18, processo 2752 folha 214, de 26 nov. 1928.}. 

Affonso Ferreira Machado

Manuel Frederico Chiappe

Pedro Erasmo do Valle



\autoref{tab:rd-mariquita}

\subsubsection{Matatu}

\begin{table}[!htp]
\IBGEtab{
\caption{Pedidos de licença por rua e década (Matatu)}\label{tab:rd-matatu}}
{
\begin{tiny}
\begin{tabular}{llllll}
\toprule
\multirow{2}{*}{Logradouro}	& \multicolumn{4}{c}{Número de licenças}	& \multirow{2}{*}{TOTAL}\\
\cline{2-5}
	&1889-1900	&1901-1910	&1911-1920	&1921-1930	& \\
\midrule
\midrule
Matatu	&13	&14	&19	&30	&76\\
Rua da Pólvora	&0	&0	&0	&0	&0\\
Quinta das Beatas	&0	&4	&8	&24	&36\\
Rua/Estrada da Vala	&0	&0	&3	&1	&4\\
Rua Alto do Formoso	&0	&0	&2	&0	&2\\
Avenida Ribeiro dos Santos	&0	&0	&0	&2	&2\\
Estrada/Baixa do Cabula	&0	&1	&1	&0	&2\\
Rua Dr. J. J. Seabra	&0	&0	&0	&1	&1\\
Fazenda Saldanha	&0	&0	&0	&1	&1\\
Avenida Saraiva	&0	&0	&0	&6	&6\\
\midrule
TOTAL	&13	&19	&33	&59	&130\\
\bottomrule
\end{tabular} 
\end{tiny}
}
{\fonte{Elaboração do autor com dados de \textbf{BR BAAHMS}, Fundo ``Intendência e Prefeitura'', Série ``Processos de Licenciamento de Reforma e Ampliação de Edificações'', Subsérie ``Requerimentos e Plantas -- Brotas'', vários documentos nas caixas 1 a 24.}}
\end{table}


\autoref{tab:rd-matatu}

\textit{centralização}, \textit{descentralização}, \textit{magnetismo funcional}, \textit{segregação}, \textit{sucessão} e \textit{inércia}

8 casas na Ribeiro dos Santos

Bento Lobo (QB)

``Ribeiro Saldanha'' é o latifundiário local

Alinhamento QB é de 13/6/1929

Hermógenes Miguel dos Anjos fez casa em sua roça no Baixão (cx 21, 4669/165, 30/08/1906)

Antonio Prisco de Araújo Falcão

Frederico Costa tinha roça na rua da Pólvora (cx 21, 1381/141, 2/3/1909)

Disputa da servidão de passagem (cx 21, 4/10/1920)

Antonio Lopo Adan

Projeto de duas casas feito por Manuel Querino (cx 21, 784/191, 21/1/1899)

\subsubsection{Acupe}

\begin{table}[!htp]
\IBGEtab{
\caption{Pedidos de licença por rua e década (Acupe)}\label{tab:rd-acupe}}
{
\begin{tiny}
\begin{tabular}{llllll}
\toprule
\multirow{2}{*}{Logradouro}	& \multicolumn{4}{c}{Número de licenças}	& \multirow{2}{*}{TOTAL}\\
\cline{2-5}
	&1889-1900	&1901-1910	&1911-1920	&1921-1930	& \\
\midrule
\midrule
Avenida Sanches	&0	&0	&0	&6	&6\\
Ladeira do Acupe	&2	&4	&4	&18	&28\\
Ladeira do Padre Eloy	&0	&0	&0	&2	&2\\
\midrule
TOTAL	&2	&4	&4	&26	&36\\
\bottomrule
\end{tabular} 
\end{tiny}
}
{\fonte{Elaboração do autor com dados de \textbf{BR BAAHMS}, Fundo ``Intendência e Prefeitura'', Série ``Processos de Licenciamento de Reforma e Ampliação de Edificações'', Subsérie ``Requerimentos e Plantas -- Brotas'', vários documentos nas caixas 1 a 24.}}
\end{table}


\autoref{tab:rd-acupe}

\textit{centralização}, \textit{descentralização}, \textit{magnetismo funcional}, \textit{segregação}, \textit{sucessão} e \textit{inércia}

Augusto Sanches demorou a apresentar planta para o arruamento; só o fez em setembro de 1928

Demétrio José Ferreira

Valois Garcia

A roça da Torre estava em 1896 com Antonio Alves Câmara

Loteamento da ladeira do Padre Eloy é de 1927

Caixa Beneficente Imobiliária Caixeirial

\subsubsection{Alagoa, Amaralina, Sanra Cruz, Ubarana, Pituba}

\begin{table}[!htp]
\IBGEtab{
\caption{Pedidos de licença por rua e década (Alagoa e Pituba)}\label{tab:rd-alagoapituba}}
{
\begin{tiny}
\begin{tabular}{llllll}
\toprule
\multirow{2}{*}{Logradouro}	& \multicolumn{4}{c}{Número de licenças}	& \multirow{2}{*}{TOTAL}\\
\cline{2-5}
	&1889-1900	&1901-1910	&1911-1920	&1921-1930	& \\
\midrule
\midrule
Pituba	&0	&0	&0	&1	&1\\
Cidade Balneária Amaralina / Rua Amaralina	&4	&17	&21	&28	&70\\
Alto da Lagoa	&0	&0	&2	&2	&4\\
Grão-Mogol	&0	&2	&0	&2	&4\\
Rua 28 de março	&0	&0	&2	&0	&2\\
\midrule
TOTAL	&4	&19	&25	&33	&81\\
\bottomrule
\end{tabular} 
\end{tiny}
}
{\fonte{Elaboração do autor com dados de \textbf{BR BAAHMS}, Fundo ``Intendência e Prefeitura'', Série ``Processos de Licenciamento de Reforma e Ampliação de Edificações'', Subsérie ``Requerimentos e Plantas -- Brotas'', vários documentos nas caixas 1 a 24.}}
\end{table}


\autoref{tab:rd-alagoapituba}

\textit{centralização}, \textit{descentralização}, \textit{magnetismo funcional}, \textit{segregação}, \textit{sucessão} e \textit{inércia}

Primeira casa de alvenaria (cx 17, 914/69, 31/3/1914)

Marie Josephine Bernichon e o colégio das Mercês

Irmãs Ursulinas também tinham casas

Chehade Elias Kraychete

Trajano dos Santos Correia assinava pela cervejaria

Matteoni, Casali e Cia.

CERVEJARIA ANTARCTICA

De toda sorte, em 1927 era anunciada a venda da massa falida da cervejaria, que incluía imóveis entre a Fonte do Boi e Amaralina\footnote{\textbf{O Combate}, ano I, nº 112, 20 out. 1927, p. 2.}.

Nenhuma das fontes consultadas indica claramente onde fica o \textit{Grão Mogol}. Três delas, entretanto, dão a entender, a propósito da instalação de um emissário de esgoto, que trata-se do nome de um morro que forma com o Morro do Conselho as duas ribanceiras de um vale. Dois morros aparecem como possibilidades: um é aquele formado pelos vales onde se situam as atuais ruas Fonte do Boi e Barro Vermelho, em cujo topo por muitas décadas funcionou o \textit{Hosital Infantil Alfredo Magalhães}, ou \textit{Hospital da Criança}, ou \textit{Hospital Nita Costa} (nome de sua fundadora); o outro é o morro cortado pela atual alameda Morro da Margarida. 

\subsubsection{Campinas, Armações e Várzea de Santo Antônio}

\begin{table}[!htp]
\IBGEtab{
\caption{Pedidos de licença por rua e década (Várzea de Santo Antônio e Armações)}\label{tab:rd-armavargem}}
{
\begin{tiny}
\begin{tabular}{llllll}
\toprule
\multirow{2}{*}{Logradouro}	& \multicolumn{4}{c}{Número de licenças}	& \multirow{2}{*}{TOTAL}\\
\cline{2-5}
	&1889-1900	&1901-1910	&1911-1920	&1921-1930	& \\
\midrule
\midrule
Imbuhy	&0	&1	&0	&0	&1\\
Várzea / Vargem de Santo Antônio	&0	&0	&0	&0	&0\\
\midrule
TOTAL	&0	&1	&0	&0	&1\\
\bottomrule
\end{tabular} 
\end{tiny}
}
{\fonte{Elaboração do autor com dados de \textbf{BR BAAHMS}, Fundo ``Intendência e Prefeitura'', Série ``Processos de Licenciamento de Reforma e Ampliação de Edificações'', Subsérie ``Requerimentos e Plantas -- Brotas'', vários documentos nas caixas 1 a 24.}}
\end{table}


A \autoref{tab:rd-armavargem} atesta a completa ausência de pedidos de licença no conjunto formado por Campinas, pelas Armações e pela várzea de Santo Antônio, que poderia tranquilamente ser ignorada. Este “silêncio”, este “vazio”, entretanto, revelam até onde chegou o processo de urbanização em Brotas – não até aqui, com certeza. Verifica-se assim a \textit{continuidade inercial} do caráter rural, pesqueiro, ermo e isolado de todas estas áreas.

Na verdade, há um só processo registrado neste extremo de Brotas, verdadeira curiosidade histórica: trata-se da disputa entre Pedro Dias e um certo ``sr. Galiza''\footnote{Em pesquisa para outra finalidade, muitos anos atrás, localizei num dos \textbf{Livros de Notas} custodiado no \textbf{BR-BAAPB} um certo senhor José Joaquim de Santana Galiza, comprador em 1923 da Fazenda Cruz, situada entre os distritos suburbanos de Pirajá e Itapuã; adicionalmente, hoje uma das ruas principais do bairro da Boca do Rio chama-se Èmiliano Galiza. Não interessa em nada a esta pesquisa identificar quem seria este ``sr. Galiza'' a que se refere o documento, mas há grandes possibilidades de ser um destes dois.} em torno de um aforamento no lugarejo conhecido como \textit{Imbuhy}\footnote{\textbf{BR-BAAHMS}, Fundo ``Intendência'', Série ``Processos de Licenciamento de Reforma e Ampliação de Edificações'', Subsérie ``Requerimentos e Plantas -- Brotas'', caixa 24. O documento não tem número de identificação, mas está datado como sendo de 9 nov. 1910.}.

\subsubsection{Totalização e conclusões}

A comparação entre uma estatística tributária e uma estatística censitária traz problemas de diversas ordens, mas dá uma ideia razoável e aproximada do processo de urbanização em Brotas. Se o número de 326 imóveis encontrado na \autoref{tab:decurb1886-1891} (p. \pageref{tab:decurb1886-1891}) como referência para os anos 1886-1891 no distrito for tomado como base inicial para comparações, e o de 3.641 imóveis públicos e privados auferido na \autoref{tab:domsaldist1-1920} (p. \pageref{tab:domsaldist1-1920}) e na \autoref{tab:domsaldist2-1920} (p. \pageref{tab:domsaldist2-1920}) for tomado como base final, a urbanização de Brotas durante o quadridecênio 1889-1930 foi simplesmente avassaladora.

Mais complicada ainda é a comparação direta destes números com o de pedidos de licença protocolados; ela é, na verdade, impossível, por muitas razões. Encontram-se na ladeira do Padre Eloy, no Sangradouro, no Moinho, na Santa Cruz e em vários outros lugares, por exemplo, os chamados “pedidos plúrimos”, ou seja, pedidos para construção de vários imóveis ao mesmo tempo; como visto, só a vila operária da União Fabril, no Sangradouro, tem 54 casas em seu projeto. 

O que a análise dos pedidos de licença permite é a desagregação da análise da urbanização em Brotas por rua e área, para entender, ainda que de modo aproximativo, as particularidades do processo. Aquilo que as estatísticas tributárias e censitárias permitem ver como totalidades de certo modo inertes, os pedidos de licença permitem ver como partes articuladas de um processo.

Um esforço de síntese de tudo quanto exposto até o momento no que diz respeito à expansão da malha urbana por meio de construções, ampliações e reformas (cf. \autoref{tab:rd-total}, p. \pageref{tab:rd-total}) permite chegar às seguintes conclusões:

\begin{table}[!htp]
\IBGEtab{
\caption{Pedidos de licença por rua e década (Total)}\label{tab:rd-matatu}}
{
\begin{tiny}
\begin{tabular}{llllll}
\toprule
\multirow{2}{*}{Área}	& \multicolumn{4}{c}{Número de licenças}	& \multirow{2}{*}{TOTAL}\\
\cline{2-5}
	&1889-1900	&1901-1910	&1911-1920	&1921-1930	& \\
\midrule
\midrule
Antigo 1º Distrito	&25	&37	&60	&80	&202\\
Boa Vista / Engenho Velho	&17	&37	&96	&66	&216\\
Estrada de Brotas	&5	&17	&19	&18	&59\\
Estrada 2 de Julho	&9	&18	&19	&20	&66\\
Mariquita (Rio Vermelho)	&10	&12	&10	&7	&39\\
Matatu	&13	&19	&33	&59	&124\\
Acupe	&2	&4	&4	&26	&36\\
Campinas	&0	&0	&0	&0	&0\\
Alagoa, Amaralina, Santa Cruz, Ubarana, Pituba	&4	&19	&25	&33	&81\\
Armações / Várzea de Santo Antônio	&0	&1	&0	&0	&1\\
\midrule
TOTAL	&85	&164	&266	&309	&824\\
\bottomrule
\end{tabular} 
\end{tiny}
}
{\fonte{Elaboração do autor com dados de \textbf{BR BAAHMS}, Fundo ``Intendência e Prefeitura'', Série ``Processos de Licenciamento de Reforma e Ampliação de Edificações'', Subsérie ``Requerimentos e Plantas -- Brotas'', vários documentos nas caixas 1 a 24.}}
\end{table}


\begin{itemize}
\item A área com maior número de licenças foi a de Boa Vista / Engenho Velho, seguida pelo antigo 1º Distrito, pelo Matatu, pelo complexo fundiário que ia de Amaralina à Pituba, pela Estrada 2 de Julho, pela Mariquita e pelo Acupe. Nem nas Armações, nem na Várzea de Santo Antônio, nem tampouco em Campinas verificou-se qualquer processo, salvo por uma disputa fundiária no Imbuí. 
\item Em números absolutos, verifica-se incremento contínuo no número de pedidos de licença protocolados em todas as décadas, com ritmo desigual: 104,68\% de variação da primeira para a segunda décadas, 22,75\% de variação da segunda para a terceira décadas e 63,29\% de variação da terceira para a quarta décadas. Se em números absolutos fica demonstrada a \textit{hipótese forte}, a variação no incremento deixa sob suspeita esta conclusão.
\item A urbanização de Brotas é parte do processo de \textit{descentralização} urbana de Salvador, e o quadridecênio analisado é apenas seu começo. Internamente ao distrito, entretanto, pode-se dizer que o antigo 1º Distrito e a área da Boa Vista / Engenho Velho assumem \textit{função central}, concentrando pontos comerciais, linhas de transporte e equipamentos coletivos.
\item A \textit{segregação socioespacial} é traço marcante da urbanização de Brotas, tanto no que diz respeito à relação entre Brotas e Salvador quanto à relação entre áreas internas ao distrito. Globalmente considerado no processo de urbanização soteropolitano do período estudado, o distrito apresenta-se como uma \textit{área de expansão urbana} onde concentram-se pequenos e médios comerciantes, assim como funcionários públicos de médio escalão em localidades razoavelmente separadas daquelas onde trabalhadores braçais e artesanais das mais diversas naturezas residiam. O caráter balneário e
veranista da Mariquita e de Amaralina, assim como o caráter proletário do Engenho Velho / Boa Vista e o caráter burguês e gestorial do antigo 1º Distrito são indisfarçáveis.
\item Verificou-se a \textit{continuidade inercial} do processo de valorização verificado no antigo 1º Distrito já nas últimas décadas do regime imperial; do desenvolvimento da Mariquita como estância de vilegiatura; do caráter rural e pesqueiro da Pituba, das Armações, de Campinas e da Várzea de Santo Antônio. Há \textit{sucessões} importantes, como a transformação de Amaralina em cidade balneária; a metamorfose da Boa Vista / Engenho Velho e da Estrada 2 de Julho de zona eminentemente rural a área de moradia proletária periférica (com exceção da rua da Boa Vista).
\item O \textit{magnetismo funcional} em Brotas só é significante no antigo 1º Distrito, cuja proximidade à rua J. J. Seabra torna-o atrativo a comerciantes.
\item Quem “puxa” para cima a média total de pedidos de licença protocolados (82 pedidos, aproximadamente) são as áreas da Boa Vista / Engenho Velho, o antigo 1º Distrito e o Matatu; mesmo o loteamento da Cidade Balneária Amaralina fica ligeiramente abaixo da média de pedidos protocolados, e as demais áreas encontram-se todas abaixo da média.
\item Consideradas as médias de pedidos de licença protocolados por década (8 para a primeira, 16 para a segunda, 27 para a terceira e 31 para a quarta), mais uma vez são a Boa Vista / Engenho Velho, o antigo 1 o Distrito e o Matatu quem puxa as médias para cima. Na primeira década, ficam abaixo da média o complexo fundiário Amaralina-Pituba,
\item ESTÉTICA E CONFLITOS SOCIAIS
\end{itemize} 

\subsection{Outros usos do espaço em Brotas}

Se os usos residenciais, industriais e comerciais do espaço foram privilegiados até o momento, é porque a questão fundiária tem sido tratada ao longo de toda a pesquisa como central para a produção, apropriação e uso do espaço urbano, e estes três usos, por seu caráter estável, tendendo à permanência, ligam-se diretamente às questões de posse e propriedade da terra. Por outro lado, há outros usos do espaço de Brotas que, tangenciando ou não a questão fundiária, merecem destaque.

\subsubsection{Candomblés}

Fiéis às suas tradições, os praticantes do candomblé resistiram em Brotas durante a Primeira República a toda sorte de estratégias repressivas, e deixaram sua marca no território do distrito. 

Em 1890 um certo ``Zé do Ó'' noticiava, em tom chistoso, que a polícia tinha ido ``fazer estardalhaço no candomblé onde se adorava \textit{Gonocô} e Ossum-Ché'', a quem surpreendentemente defendeu em público por serem ``deuses que estão em pleno goso de seus direitos moraes, civis e politicos depois do decreto da liberdade de cultos''; ``Zé do Ó'' ainda acusou a polícia de ter comido e bebido à vontade no terreiro, além de ter soltado as sacerdotisas e prendido os ``marmanjos''\footnote{\textbf{Tribuna Popular}, ano I, nº 50, 29 jun. 1890, p. 3.}. Ironia? Defesa legítima? Importa a esta pesquisa apenas o fato de uma das divindades cultuadas ser \textit{Babá Gunukô}, indicando a possibilidade de este terreiro atacado ser o mesmo indicado na \autoref{subsubsec:matatu} (p. \pageref{subsubsec:matatu}).

Em setembro de 1914 noticiava-se -- com as habituais solicitações de providências por parte da polícia -- a existência de duas casas do ``maldito e ruidoso candomblé'', ``uma sita à rua Uruguayana e a outra em uma ladeira por detraz do Asylo São João de Deus''\footnote{\textbf{A Notícia}, ano I, nº 4, 23 set. 1914, p. 2.}. Como visto na \autoref{subsec:pontrel} (p. \pageref{subsec:pontrel}), o terreiro \textit{Tumba Junsara}, fundado anos depois, em 1919, foi instalado na Ladeira do Pepino por volta de 1920; é certo que o terreiro da Uruguaiana a que se refere a notícia não é ele, mas é muito provável que a existência de terreiros mais antigos na Boa Vista e no Engenho Velho tenha de algum modo influenciado a escolha de sua nova localização.

Em maio de 1920 chegou ao famigerado Pedro Gordilho, delegado de polícia, a informação de que ``no Matatú Grande'', ou mais especificamente na localidade conhecida ainda hoje como Baixão, ``todas as noites `tocam candomblé' ''. Em diligência ao local em alta madrugada\footnote{A notícia indica que à 1h se cantava, ao som de atabaques e agogôs, uma cantiga assim registrada: ``O Aruchachá / que relampuê / minha Santa Bárbara / que relampoá!''; o mais provável é que se tratasse de um culto a Iansã.}, o destacamento policial comandado pelo próprio Gordilho irrompeu terreiro adentro aos gritos de ``polícia! polícia!'', prendendo vinte pessoas (que se diz terem sido postas em liberdade no dia seguinte, sabe-se lá em que condições) e pondo em fuga outras tantas. O terreiro foi assim desmantelado, todos os objetos sagrados e litúrgicos do templo -- ``capacetes, coroas, settas, pandeiros, `tabaques', `afunchês', `agugôs', `ojás', chapanã, contas, santos horriveis de expressão physiognomica, pedaços de páo talhados, talhados de arremate'' -- foram apreendidos, e tudo foi entregue ao Instituto Geográfico e Histórico da Bahia por sugestão de Pedro Melo, do Gabinete de Identificação\footnote{\textbf{A Manhã}, ano I, nº 36, 20 maio 1920, p. 1.}. Nove dias depois da matéria, considerada ``um sucesso'' pelo jornal, foi impetrado \textit{habeas corpus} em favor do pai-de-santo, ironizado pelo jornal por apelar ao Judiciário ao invés dos santos do candomblé, este ``abuso anti-hygienico''\footnote{A Manhã, ano I, nº 44, 29 maio 1920, p. 3.}, A manchete em letras garrafais da incomum matéria de meia página da capa do jornal não deixa dúvidas quanto à localização do terreiro: ``Viva Ogunjá!'' Não se pode concluir daí outra coisa senão de que se tratava do \textit{Ilê Ogunjá}, fundado em 1906 por \textit{Procópio Xavier de Souza} (1880-1958), sacerdote ketu conhecido também como \textit{Procópio de Ogum}. Muito provavelmente foi esta a prisão, ou uma das prisões, que deu base a uma das mais famosas alegorias da obra literária de Jorge Amado, iniciado ele próprio no candomblé como ogã de Oxossi no Ilê Ogunjá: o afrontamento de Procópio ao delegado ``Pedrito Gordo'', baseado no próprio Pedro Gordilho \cite[p.~236-242]{amado_tenda_2010}. O Ilê Ogunjá deu seu nome iorubá ao vale entre o Acupe e o Matatu, de um lado, e o Engenho Velho de Brotas, do outro -- e até os dias atuais, quando sobre ele corre uma avenida bastante movimentada, não há quem a conheça por seu nome oficial.

Nem a documentação pesquisada, nem a bibliografia consultada apontam quaisquer outras ocorrências, que para emergirem da poeira dos arquivos precisariam de outro tipo de pesquisa, de outras fontes, de outro tempo. Cabe registrar, como nota final sobre o assunto, duas coisas. 

Em primeiro lugar, os terreiros mencionados, junto ao \textit{Alaketo} ainda existente no período no Matatu, são ou foram terreiros grandes, bastante conhecidos; está ainda em curso, em especial por meio da arqueologia, a construção de uma base historiográfica acerca dos terreiros menores, domésticos, mais fáceis de se ocultar num cenário político repressivo mas ao mesmo tempo mais vulneráveis à repressão quando a sofriam, e portanto mais efêmeros \cite{gordenstein_arqueterre_2016}; no caso de Brotas, uma tal pesquisa sequer existe, tanto pela falta de uma base de dados tal como a que vem sendo construída acerca dos candomblés do século XIX \cite{reis_candomble_2001}, quanto pelo fato de que, diferentemente do que se deu no Pelourinho e adjacências, onde tais pesquisas arqueológicas e arquivísticas logram maior sucesso, o desenvolvimento urbano em Brotas resultou em sucessivas demolições e reconstruções, arrancando assim do solo significativa matéria-prima arqueológica. Uma base de dados sobre os candomblés domésticos em Brotas pode ser construída, e certamente a base de dados relativa aos candomblés baianos do século XIX a contempla; o difícil é encontrar os traços e rastros arqueológicos dos terreiros menores. Veja-se, como exemplo destes terreiros menos conhecidos em Brotas, a notícia de que em junho de 1904 uma ``força de Urbanos'' atacou um terreiro na estrada da Cruz das Almas e prendeu oito pessoas\footnote{\textbf{Correio do Brasil}, ano II, nº 231, 07 jun. 1904, p. 2}; se o \textbf{Mapeamento dos Terreiros de Salvador} (\url{http://www.terreiros.ceao.ufba.br}) foi muito útil relativamente aos terreiros maiores e mais conhecidos, no que diz respeito a terreiros já desaparecidos como este a ferramenta mostrou suas limitações, pois aquilo que hoje deve constar quiçá apenas da memória genealógica de algumas casas não fica nele registrado, e este terreiro da estrada da Cruz das Almas não aparece em mais referência alguma. 

Em segundo lugar, não se pode deixar de registrar como a aguerrida resistência dos praticantes do candomblé em Brotas produziu um curioso efeito. Os mais conhecidos pontos notáveis de Brotas na atualidade não são, como na Primeira República, as igrejas e as grandes herdades, palacetes e mansões; são as \textit{avenidas de vale}, cuja importância transcende o território do distrito. Ora, duas das mais importantes delas, a Bonocô (Mário Leal Ferreira) e a Ogunjá (General Graça Lessa), devem seu nome mais conhecido não a uma ``origem popular'' de memória perdida, mas sim a dois terreiros de candomblé muito antigos. 

\subsubsection{Eventos cívico-políticos}

Prosseguiam as comemorações do Dois de Julho no distrito tão animadas -- ou mesmo mais, quem sabe! -- quanto nos tempos imperiais:

\begin{citacao}
\textbf{2 de Julho do Castro Neves}

Da comissão organisadora desses festejos, recebemos um delicado convite a que agradecemos, para assistirmos ao desfilar do prestito que se realizará no proximo domingo, 6 do corrente, a 1 hora da tarde, sahindo do Largo da Boa Vista e seguindo pelas ruas 1º de Março, Pitangueiras, Largo do Paranhos, Matatú Pequeno, Fabricio, Sangradouro, Sete Portas, regressando pela Ladeira do Santo Agostinho, rua da Alegria, Socorro, Ladeira dos Galés, Fonte Nova, donde se dirigirá para o Castro Neves, que estará festivamente embandeirado.

Ao passar pelo Matatu pequeno falará o acadêmico Lemos Britto.

Os festejos prolongar-se-ão até o dia 8.

O programma é o seguinte!

Abrira a comitiva civica um grupo numeroso de cavalheiros, trajados symbolicamente;

seguir-se-á uma banda de clarins, fanfarreando em toques estridentes, e annunciando à multidão anciosa a aproximação do imponente e magestoso prestito;

segue-se um piquete de cavallaria;

virá depois um pelotão de cornetas e caixas de guerra, surgindo então, da immensa massa do povo, o Carro Emblematico;

musica de S. Vicente de Paulo;

batalhão de graciosas meninas symbolisando as \og Heroinas Brasileiras \fg{};

banda do 2º corpo do Regimento Policial;

batalhão de meninos \og Defensores do Castro Neves \fg{};

musica dos Salesianos, seguindo-se o collegio encorporado da referida associação beneficente;

banda do 1º corpo policial;

banda do 5º batalhão de artilharia, etc., etc.\footnote{\textbf{Correio do Brazil}, ano III, nº 567, 4 ago. 1905, p. 4}
\end{citacao}

Os festejos bijulinos eram ainda realizados no Castro Neves em 1914\footnote{\textbf{Gazeta de Notícias}, ano IV, nº 230, 01 jul. 1914, p. 1.}, e em 1915 ocorriam também na vizinhança da Capela do Deus Menino\footnote{\textbf{A Notícia}, ano I, nº 260, 05 ago. 1915, p. 1.}.

As celebrações particulares também eram dignas de nota, em especial quando ligadas a políticos. Frederico Costa, presidente do Senado estadual, enraizava seu poder político nas terras de sua propriedade ao Matatu, onde tinha uma fazenda produtora de laranjas e vários imóveis urbanos; num seu aniversário, em 1927, um jornal notoriamente seabrista esculhambou-o de cima a baixo, chamando-o de ``homúnculo [\dots] [\textit{que}] de discursos [\dots] pouco entende'', ``rubicundo politiqueiro [\dots] cuja obtusidade é compacta, é integral, é profunda'' -- e entre um impropério e outro, registrou que naquela data, em seguida a uma missa na Sé, o Matatu amanhecia ``embandeirado em arco'' pela festa de aniversário do político, ``com gyrandolas, comidas e bebidas''\footnote{\textbf{O Combate}, ano I, nº 120, 29 out. 1927, p. 1. O mesmo jornal prosseguiu na catilinária contra Frederico Costa em várias edições posteriores.}.

\subsubsection{Eventos religiosos, festivos e carnaval}

O calendário de festas religiosas católicas soteropolitanas celebradas também no distrito de Brotas incluia o \textit{Natal}, celebrado em 1914 no ``aprazível arrabalde'' da Pituba com ``muitas diversões que se prolongarão até ao romper do dia, com o espoucar de foguetes, tocando um grupo musical, havendo kermesse e outras surpresas''; na Boa Vista estava programada a apresentação de um coral infantil, da banda do Corpo de Bombeiros e uma procissão infantil carregando a imagem do ``menino Deus'' (seria a mesma da Capela do Deus Menino, bastante próxima?); no ``lindo trecho'' da Alagoa seria realizada uma missa\footnote{\textbf{A Notícia}, ano I, nº 83, 24 dez. 1914, p. 7.}. Quanto a esta última, a tradição pode ter sido iniciada em 1898, ano em que ``um grupo de rapazes e moças passeiantes, moradores na Fonte do Boi e nas Pedrinhas'', foi pedir permissão ao ``coronel Juca Amaral'' para celebarar na capela de Nossa Senhora dos Mares da Lagoa uma missa natalina\footnote{\textbf{Jornal de Notícias}, ano , nº , 13 dez. 1898, p. 2.}.

Outra celebração em que Brotas se inseria no calendário soteropolitano era a \textit{queima do Judas}, como se deu no Sangradouro em 1905 \footnote{\textbf{Correio do Brasil}, ano III, nº 486, 24 maio 1905, p. 2.}. Comemorava-se também em Brotas a \textit{folia de Reis}: em 1914, além da saída do terno \textit{Concha de Ouro} da Alegria do Castro Neves\footnote{\textbf{Gazeta de Notícias}, ano IV, nº 94, 03 jan. 1914, p. 1.}, foi anunciado que no Rio Vermelho ``funccionarão, durante toda a noite, o \textit{roskoff} e a \textit{kermesse}, devendo alli comparecer os ranchos do \textit{Saramonete}, que sahirá do Grão-Mogol, e o do \textit{Bebê}, que sahirá do arrabalde de Amaralina''\footnote{\textbf{Gazeta de Notícias}, ano, nº , 5 jan. 1914}; no caso de Amaralina, é muito provável que participasse de algum modo na organização dos festejos o \textit{Centro Recreativo de Amaralina}, do qual era vice-presidente em dezembro de 1913 o jovem Alberto Magalhães\footnote{\textbf{Gazeta de Notícias}, ano IV, nº 71, 04 dez. 1913, p. 2.}.

Os ranchos, entretanto, eram vistos na época como ``elemento de desordem'': o \textit{Curuvina}, por exemplo, sediado no Dique Pequeno, foi destacado pela imprensa por causa de uma briga ocorrida numa de suas noites de ensaio, ``onde a \textit{pura} é o estimulante procurado para as libações do \textit{pessoal}''; postas as coisas em tom francamente moralista, perguntava-se o articulista, sarcasticamente, se ``a \textit{Curuvina} ainda virá á tona''\footnote{\textbf{A Notícia}, ano I, nº 42, 06 nov. 1914, p. 5.} -- ou seja, se a continuidade de seus ensaios seria permitida pelas autoridades. Como se vê, os ranchos dos bons moços de Amaralina e do Grão-Mogol eram tratados de forma muito diversa dos ranchos dos proletários, muito provavelmente negros, do Dique Pequeno.

Muito própria do distrito era a \textit{festa de Nossa Senhora da Luz}, assim anunciada em 1905:

\begin{citacao}
No proximo domingo realisar-se-à com toda a pompa a festa de N. S. da Luz, na Pituba, promovida por famílias que se acham veraneando no salubre e pitoresco arrabalde da Pituba.

Ás 5 horas da madrugada serão annunciadas as festas por 21 tiros, às 8 horas da manhã será realizada a missa e às 11 horas terá começo a festa religiosa, sendo pregador o Monsenhor Novaes.

Às 3 horas da tarde sahirá em procissão a imagem de N. Senhora, carregada por distinctas senhoras e tendo a frente a banda de música do Regimento Policial. A noite tocarà em um palanque uma musica, havendo leilão, kermesse e finalisando a festa com um bonito fogo de artifício.

Na segunda-feira haverá corrida de jangadas e outros divertimentos, terminando os festejos na terça-feira por passeiatas populares\footnote{\textbf{Correio do Brasil}, ano III, nº 420, 10 fev. 1905, p. 1.}.
\end{citacao}

O carnaval era também comemorado em Brotas durante a Primeira República, como se vê no anúncio de que em 1919 percorria a Estrada 2 de Julho na segunda e na terça de carnaval o cordão \textit{A Censura}\footnote{\textbf{A Hora}, ano II, nº 44, 01 mar. 1919, p. 3.}.

\subsubsection{Ponto de desova}

A lonjura e a derrelição dos limites de Brotas mais distantes da área urbanizada facilitavam seu uso como \textit{ponto de desova} de cadáveres. A \textit{várzea de Santo Antônio}, erma agora como anos antes, prestava-se ainda a este fim no período estudado: certo Ernesto Wolf, alemão chefe de máquinas no Gasômetro e residente à Calçada do Bonfim, desapareceu numa roça do distrito de Brotas e teve seu cadáver encontrado na várzea de Santo Antônio no dia seguinte\footnote{\textbf{Gazeta de Notícias}, ano III, nº 20, 30 set. 1912, p. 3}.
% % \section{Brotas e as reformas, as reformas e Brotas}\label{sec:3.3}

