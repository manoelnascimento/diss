\begin{table}[!htp]
\IBGEtab{
\caption{Valor locativo médio dos imóveis nas dez ruas de imóveis ``mais baratos'' entre os arrolados pelo município de Salvador no distrito de Brotas em 1924}\label{tab:maisbaratos1924brotas}}
{
\begin{tabular}{rr}
\toprule
Locais	&Valor locativo médio por imóvel\\
\midrule
\midrule
Alto da Ubarana	&300\$000\\
Travessa do Sangradouro	&291\$330\\
Rua do Engenho Velho	&273\$000\\
2ª Travessa da Quinta das Beatas	&265\$710\\
Quinta das Beatas	&261\$140\\
Rua da Lagoa	&254\$740\\
Travessa da Rua Uruguaiana	&250\$450\\
Capelinha do Deus Menino	&244\$480\\
2ª Ladeira do Engenho Velho	&226\$200\\
Alto do Formoso	&218\$080\\
1ª Travessa da Quinta das Beatas	&207\$690\\
Travessa do Pomar	&187\$500\\
Candeal Pequeno	&186\$670\\
Travessa da Rua do Beiju	&180\$000\\
Pomar	&120\$000\\
\bottomrule
\end{tabular}
}
{\fonte{Elaboração do autor, com base no \textbf{Annuario estatistico – annos de 1924 e 1925} organizado para o Governo da Bahia por M. Messias de \citeonline[pp.~263-264]{bahia_annuario_1926}.}}
\end{table}