\section{Investimentos públicos}\label{sec:3.1}

Primeiramente, é preciso analisar o processo de reorganização espacial do distrito de Brotas no período a partir da instalação de infraestruturas urbanas. A presença de tais equipamentos constitui um indicador seguro da urbanização, na medida em que sua implementação sinaliza mudança na forma do assentamento humano local: se a urbanização no período estudado era marcada sobretudo pela combinação entre a expansão da malha urbana e o avanço de infraestruturas viárias, comunicacionais e sanitárias sobre as áreas urbanas já consolidadas, 

\subsection{Investimentos: tipo, volume e impactos}\label{investbrotas}

\subsubsection{Esgotamento e abastecimento de água}

Em 1894 foram limpas a fonte e o riacho do Boi por 350\$000, e outros 5:200\$000 foram investidos na limpeza de 9.372 metros do rio Camarajipe e de outros 3.101 metros do rio das Tripas \cite[p.~156]{salvador_relatorio_1894}.

Em 1895 foi limpo o riacho do Sangradouro, a rua Primeiro de Março (Pitangueiras) recebeu um sifão e outros dois foram instalados também nas Pitangueiras sem se estabelecer com maior precisão o endereço da obra \cite[pp.~20, 137-138]{salvador_relatorio_1896}. 


Em 1906, como parte dos trabalhos de esgotamento e abastecimento de água conduzidos por Teodoro Sampaio, a Intendência Municipal desapropriou 14,9 milhões de metros quadrados de terra pertencentes à Ordem de São Bento ``entre a sesmaria de Rio Vermelho e a de Garcia D'Ávila''; a ordem religiosa terminou, além disto, vendendo em 1917 suas terras remanescentes na mesma área à Intendência \cite[p.~306]{VASCONCELOS2002}.


\subsubsection{Arruamento e pavimentação}

USAR FALAS DE GOVERNADORES E RELATÓRIOS DE INTENDÊNCIA

Registrou-se entre 1893 e 1895 sob a rubrica ``obras'' o investimento de 59:827\$676 no distrito, representando 18,6\% do total de investimento nesta rubrica; Sé, São Pedro receberam no mesmo período, respectivamente, investimentos nos montantes de 86:951\$544 (27,04\% do total), 50:556\$057 (15,72\% do total) e 68:465\$430 (21,29\% do total). Estas obras referem-se ao calçamento das ruas Primeiro de Março (Pitangueiras) e Uruguayana (Engenho Velho) \cite[pp.~21-23, 138]{salvador_relatorio_1896}, além de outras na estrada de Brotas, na Quinta das Beatas, no Castro Neves, na rua do Socorro, na ladeira do Inferno, na ladeira do Elevador e nos fundos do Hospital Militar \cite[p.~157]{salvador_relatorio_1894}.

No mesmo período a ponte da vargem de Santo Antônio, que ameaçava cair, foi demolida, e as pedras foram reaproveitadas pelo município \cite[p.~137]{salvador_relatorio_1896}; ainda outra ponte, na baixa da ladeira do Beiju, foi construída sobre o rio Camorogipe \cite[p.~156]{salvador_relatorio_1894}. Em 1901 foi construído um pontilhão sobre o rio Lucaia, na baixa do Acupe 

Adiantava-se, entretanto, a contratação do calçamento da rua do Castro Neves pelo valor de 20:270\$000, a ser realizada a partir de 1896 \cite[p.~23]{salvador_relatorio_1896}

No ano de 1895 a única localidade a ter recebido pavimentação nova em 1895 foi a Mariquita \cite[p.~16]{salvador_relatorio_1896}; da mesma forma, somente a ladeira da Fonte das Pedras recebeu conserto e calçamento neste ano \cite[p.~16]{salvador_relatorio_1896}. 



\begin{citacao}
Tornando-se necessário regularisar a íngreme communicação existente entre as ruas do Sangradouro, no distrito de Brotas, e do Cabral, de Sant'Anna, projectando-se o preciso melhoramento, referente somente ao movimento de terra e álveos ficando a rua com 9 metros e os passeios lateraes com 2 metros cada um.

D'esse trabalho, orçado em 3:207\$308 réis foi encarregado o cidadão Severiuno Vicente de Oliveira, a quem se passou em data de 19 de Novembro um attestado no valor de 842\$280.
\end{citacao}

\begin{citacao}
Do Município desta Capital -- 32.000 metros de 
\end{citacao}








\subsubsection{Iluminação pública}


Salvador tinha em 1894, segundo a \textit{Bahia Gas Company Ltd.} que prestava o serviço de iluminação pública, uma rede de encanamento de gás para os combustores que chegava a 99.894,45 metros de tubos de ferro fundido com diversos diâmetros \cite[p.~178]{salvador_relatorio_1894}.

A rede de canalização recebeu em 1895 mais 909,9 metros de tubos com 75mm de diâmetro no Castro Neves, Sangradouro e Socorro. Mesmo assim, a iluminação em Brotas ainda era fraca: de 74 novos combustores instalados naquele ano, o distrito recebeu 30\footnote{6 no Matatu, 6 na Boa Vista, 3 na ladeira dos Galés, 2 na rua do Socorro, 1 na ladeira da Fonte das Pedras e impressionantes 12 na rua do Castro Neves.} \cite[pp.~149-150]{salvador_relatorio_1896}, indicando forte ausência anterior; em 1894 dois novos combustores foram instalados no beco do Cego \cite[p.~179]{salvador_relatorio_1894}.

A iluminação domiciliar em Brotas começou a República tão tremeluzente quanto a pública:

\begin{citacao}
Os prédios d'esta cidade em que se faz uso da illuminaçâo por meio do gaz sâo em número de 1.246, dos quaes 370 na freguezia de S. Pedro e 219 no da Victoria. constituindo o primeiro districto dos cobradores; e 112 na da Penha 69 na dos Mares, 16 na do Pilar, 66 na da Conceição da Praia. 42 na do Santo Antonio 62 na da Rua do Passo, 163 na de Sant'Anna, 121 na da Sé e 6 na de Brotas formando o segundo districto \cite[p.~151]{salvador_relatorio_1896}. 
\end{citacao}

\begin{citacao}
Ao cidadão Antônio Fernandes Leitão, contractante do calçamento da rua Castro Neves, se firmou de obra feita attestado, no valor de 2:001\$440 rs. com folhas de trabalhadores empregados nos melhoramentos da estrada de Brotas se gastou a somma de 58\$200 \cite[p.~]{salvador_relatorio_1896}.
\end{citacao}



\subsubsection{Transporte público}

USAR FALAS DE GOVERNADORES E RELATÓRIOS DE INTENDÊNCIA


Em 1906 foram iniciadas as obras para implantação da linha de bonde entre Sete Portas e Brotas pela Trilhos Centrais, e em 1916 foi inaugurada a linha que levava ao Acupe \cite[p.~299]{VASCONCELOS2002}.


Um aspecto curioso em Brotas é a prolongada existência de seu \textit{cemitério}, ainda existente.

USAR FALAS DE GOVERNADORES E RELATÓRIOS DA INTENDÊNCIA

DESTACAR OS REITERADOS ELOGIOS À ADMINISTRAÇÃO DO CEMITÉRIO NOS DOCUMENTOS OFICIAIS E OUTROS

\subsubsection{Serviços de saúde}

USAR FALAS DE GOVERNADORES E RELATÓRIOS DA INTENDÊNCIA

\subsubsection{Escolas} 

do distrito, vimos na \autoref{subsec:matatubeatas} a inquietação causada pela transferência de uma escola do largo de Brotas para o povoado das Pitangueiras. Que teria acontecido daí em diante?



USAR RELATÓRIOS DA INTENDÊNCIA, QUE DISCRIMINAM AS ESCOLAS

\subsubsection{Telefonia} 

no distrito, encontramos a seguinte informação, datada de 1926:

\begin{citacao}
O serviço telephonico, tanto urbano, como interurbano, também está a cargo da Inspectoria de Viação [\textit{órgão do governo estadual}].
O contracto respectivo para esta Capital provinha de uma concessão feita pela Monarchia, em 1884.
O Governo da União, em l924, transferiu suas obrigações e direitos ao Governo do Estado, que fez novo contracto com a \textit{Companhia Brasileira de Energia Eléctrica}, a 26 de Novembro do mesmo anno.
O serviço está actualmente bem montado e com alguma canalização subterrânea.
Há quatro estações: \textit{Central}, \textit{Garcia}, \textit{Roma} e \textit{Rio Vermelho} e por ellas o anno passado se fizeram 62.600 ligações diárias.
Uma nova estação, às Pitangueiras, em Brotas, acaba de ser installada para 120 linhas, com 4 telephonistas.
Encontra-se em remodelação a estação de Roma.
No fim do anno passado existiam 3.221 telephonios e 433 extensões.
A extensão da rêde aérea naquella época era de 24.700 metros e da subterrânea de 24.200, tendo augmentado a primeira no segundo semestre de 300 metros e a segunda
de 1.200, com capacidade para 600 telephonios.
Para a linha de Brotas serão precisos 3.500 metros de linha aérea. \cite[pp.~266-267]{bahia_rpe_1926}
\end{citacao}

Como se vê, naquele ano o Rio Vermelho já era dotado de central telefônica, outra havia sido recém-instalada nas Pitangueiras e a rede ainda seria expandida.