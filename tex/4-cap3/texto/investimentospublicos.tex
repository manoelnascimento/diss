\section{Investimentos públicos}\label{sec:3.1}

Primeiramente, é preciso analisar o processo de reorganização espacial do distrito de Brotas no período a partir da instalação de infraestruturas urbanas. A presença de tais equipamentos constitui um indicador seguro da urbanização, na medida em que sua implementação sinaliza mudança na forma do assentamento humano local: se a urbanização no período estudado era marcada sobretudo pela combinação entre a expansão da malha urbana e o avanço de infraestruturas viárias, comunicacionais e sanitárias sobre as áreas urbanas já consolidadas, 

\subsection{Investimentos: tipo, volume e impactos}\label{investbrotas}

No que diz respeito ao \textit{esgotamento e abastecimento de água}

USAR FALAS DE GOVERNADORES E RELATÓRIOS DE INTENDÊNCIA

No que diz respeito ao \textit{arruamento e pavimentação}

USAR FALAS DE GOVERNADORES E RELATÓRIOS DE INTENDÊNCIA

\begin{citacao}
Do Município desta Capital -- 32.000 metros de 
\end{citacao}

No que diz respeito à \textit{iluminação pública}

USAR FALAS DE GOVERNADORES E RELATÓRIOS DE INTENDÊNCIA

No que diz respeito ao \textit{transporte público}

USAR FALAS DE GOVERNADORES E RELATÓRIOS DE INTENDÊNCIA

Um aspecto curioso em Brotas é a prolongada existência de seu \textit{cemitério}, ainda existente.

USAR FALAS DE GOVERNADORES E RELATÓRIOS DA INTENDÊNCIA

DESTACAR OS REITERADOS ELOGIOS À ADMINISTRAÇÃO DO CEMITÉRIO NOS DOCUMENTOS OFICIAIS E OUTROS

No que diz respeito aos \textit{serviços de saúde}

USAR FALAS DE GOVERNADORES E RELATÓRIOS DA INTENDÊNCIA

No que diz respeito às \textit{escolas} do distrito, vimos na \autoref{subsec:matatubeatas} a inquietação causada pela transferência de uma escola do largo de Brotas para o povoado das Pitangueiras. Que teria acontecido daí em diante?



USAR RELATÓRIOS DA INTENDÊNCIA, QUE DISCRIMINAM AS ESCOLAS

No que diz respeito à \textit{telefonia} no distrito, encontramos a seguinte informação, datada de 1926:

\begin{citacao}
O serviço telephonico, tanto urbano, como interurbano, também está a cargo da Inspectoria de Viação.
O contracto respectivo para esta Capital provinha de uma concessão feita pela Monarchia, em 1884.
O Governo da União, em l924, transferiu suas obrigações e direitos ao Governo do Estado, que fez novo contracto com a \textit{Companhia Brasileira de Energia Eléctrica}, a 26 de Novembro do mesmo anno.
O serviço está actualmente bem montado e com alguma canalização subterrânea.
Há quatro estações: \textit{Central}, \textit{Garcia}, \textit{Roma} e \textit{Rio Vermelho} e por ellas o anno passado se fizeram 62.600 ligações diárias.
Uma nova estação, às Pitangueiras, em Brotas, acaba de ser installada para 120 linhas, com 4 telephonistas.
Encontra-se em remodelação a estação de Roma.
No fim do anno passado existiam 3.221 telephonios e 433 extensões.
A extensão da rêde aérea naquella época era de 24.700 metros e da subterrânea de 24.200, tendo augmentado a primeira no segundo semestre de 300 metros e a segunda
de 1.200, com capacidade para 600 telephonios.
Para a linha de Brotas serão precisos 3.500 metros de linha aérea. \cite[pp.~266-267]{bahia_rpe_1926}
\end{citacao}

Como se vê, naquele ano o Rio Vermelho já era dotado de central telefônica, outra havia sido recém-instalada nas Pitangueiras e a rede ainda seria expandida.