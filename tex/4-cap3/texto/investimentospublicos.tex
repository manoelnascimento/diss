\section{Investimentos públicos}\label{sec:3.1}

Primeiramente, é preciso analisar o processo de reorganização espacial do distrito de Brotas no período a partir da \textit{instalação de infraestruturas urbanas por meio de investimentos públicos}. A presença de tais equipamentos infraestruturais constitui um indicador seguro da urbanização, na medida em que sua implementação sinaliza mudança na forma do assentamento humano local. 

Registrou-se entre 1893 e 1895 sob a rubrica ``obras'' o investimento de 59:827\$676 no distrito, representando 18,6\% do total apresentado;  estas obras referem-se ao calçamento das ruas Primeiro de Março (Pitangueiras) e Uruguaiana (Engenho Velho) \cite[pp.~21-23, 138]{salvador_relatorio_1895}, além de outras na estrada de Brotas, na Quinta das Beatas, no Castro Neves, na rua do Socorro, na ladeira do Inferno, na ladeira do Elevador e nos fundos do Hospital Militar \cite[p.~157]{salvador_relatorio_1894}. Por sua vez, e comparativamente, os distritos da Sé, São Pedro e Vitória receberam no mesmo período, respectivamente, investimentos nos montantes de 86:951\$544 (27,04\% do total), 50:556\$057 (15,72\% do total) e 68:465\$430 (21,29\% do total).

Brotas, como se vê, não era destino preferencial dos investimentos em infraestruturas urbanas. Uma listagem completa da série de investimentos municipais e estaduais no distrito, comparando-os com o investimento de cada ente federativo nos demais distritos de Salvador, comprovaria se esta tendência se manteve durante os anos abrangidos por esta pesquisa. Infelizmente, apesar de a coleção dos relatórios dos governadores da Bahia encontrar-se completa em várias bibliotecas físicas e digitais, a coleção dos relatórios dos intendentes municipais, que é onde se pode encontrar de forma comparativa tais investimentos ao longo do tempo, chegou até nossos dias com lacunas severíssimas, em especial nos anos onde as reformas de J. J. Seabra estavam em seu auge. O método mais adequado, portanto, não poderá ser seguido. Será preciso algum nível de inferência e interpretação, cotejadas com os achados na série de investimentos privados na urbanização do distrito. 

\subsection{Arruamento e pavimentação}

No que diz respeito ao arruamento e pavimentação, este esforço interpretativo torna-se ainda mais premente, pois a mesma rua pode ser pavimentada e reconstruída diversas vezes num dado período. Não é, portanto, a soma do investimento neste tipo de obras o elemento a destacar no conjunto, mas sim a recorrência destes investimentos em tal ou qual logradouro ou área do distrito.

Em 1893 foi “regularisada” a ladeira do Castro Neves “por occasião das festas
do 2 de Julho na localidade”, custando a obra 34\$500. Foi também continuada a obra, “quasi terminada”, da “importante ponte” de madeira sobre o Camorogipe na altura da baixa do Beiju, cuja construção, sob supervisão de Antonio Fernandes Leitão, custou 1:758\$888. Firmou-se também contrato com Osmundo Americano da Silva para limpar o rio Camorogipe, “da ponte do Retiro ao Rio Vermelho e do rio das Tripas no Arco do Barbalho ao entroncamento com aquelle rio”, custando 5:000\$000 a obra de limpeza do trecho de 9.372 metros do Camorogipe e de 3.103 metros do rio das Tripas. Neste mesmo ano a Intendência Municipal reconheceu a necessidade de “limpeza e regularização das estradas” de Brotas, além de “outros muitos melhoramentos” \cite[pp.~47-48; anexo 3, pp.~18-19]{salvador_relatorio_1893}.

Em 1894

No ano de 1895 a única localidade a ter recebido pavimentação nova no distrito foi a Mariquita; da mesma forma, somente a ladeira da Fonte das
Pedras recebeu conserto e calçamento neste ano \cite[p.~16]{salvador_relatorio_1895}. Adiantava-se, entretanto, a contratação do calçamento da rua do Castro Neves pelo valor de 20:270\$000, a ser realizada a partir de 1896 pelo contratante Antônio Fernandes Leitão \cite[p.~23]{salvador_relatorio_1895}. Em 1896 a ponte da vargem de Santo Antônio, que ameaçava cair, foi demolida, e as pedras foram reaproveitadas pelo município \cite[p.~137]{salvador_relatorio_1895}; ainda outra ponte, na baixa da ladeira do Beiju, foi construída sobre o rio Camarajipe \cite[p.~156]{salvador_relatorio_1894}. Em 1897 o Castro Neves ainda estava em obras, sob responsabilidade de Antônio Fernandes Leitão, que recebeu naquele ano 8:024\$620 pelo serviço \cite[p.~99]{salvador_relatorio_1897}. Em 1898 a Estrada 2 de Julho passou por mais um nivelamento “e mais trabalhos”, sob a coordenação do engenheiro Antonio Lopes da Silva Lima\footnote{\textbf{Jornal de Noticias}, ano XIX, n o 5.456, 14 mar. 1898, p. 1.}.

Em 1901 foi construído um pontilhão sobre o “riacho” Lucaia, na altura do Acupe, sob responsabilidade do empreiteiro Eduardo Soares de Campos, ao custo de 2:077\$639; este mesmo empreiteiro e seu colega Euthymio Candido dos Reis estiveram à frente do rebaixamento da ladeira dos Galés, “de ha muito reclamada pelos habitantes daquella zona”, para o qual foram nomeados como auxiliares da Junta administrativa do distrito para a fiscalização desta obra os srs. barão de São Marcos, monsenhor Manoel José de Novaes, Antonio Victorio de Araújo Falcão, João Agrippino da Costa Dórea e o capitão Anísio M. Gomes \cite[pp.~15, 162, 249-250]{salvador_relatorio_1901}.

A série de relatórios anuais da Intendência Municipal de Salvador custodiada no \textbf{BR-BAAHMS} interrompeu-se aqui, para ser retomada com o relatório de 1915. Nele, consta que foi orçada neste ano a enorme obra de reparos na estrada de Brotas, “a partir do portão do cemitério até a ladeira da Boquinha”, no total de 6:673\$896; o total de 466\$162 em obras executadas neste mesmo ano no distrito, entretanto, foi o mais baixo investimento em obras por parte da Intendência Municipal em 1915, nem de longe se comparam com os 11:158\$563 investidos por este órgão no distrito da Sé em 1915 durante as reformas de J. J. Seabra, ou com os 13:165\$190 investidos no Santo Antônio, já longe do perímetro das reformas \cite[pp.~127-128]{salvador_relatorio_1916}. Já a Diretoria de Higiene Municipal informou ter investido em Brotas o total de 3:217\$042 em Brotas em 1915, envolvendo desobstrução de vala e canalização de esgoto no Sangradouro, construção de esgoto no Rio Vermelho, limpeza, alargamento e retificação do Camorogipe e do rio das Tripas; no orçamento desta diretoria para o ano, Brotas ficou em quinto lugar \cite[pp.~195-196]{salvador_relatorio_1916}. Soma-se a isto um projeto de lei de autoria do conselheiro municipal Arnaldo Silvany, ele próprio morador do Castro Neves, que pedia o calçamento das ruas do Sangradouro, Alegria, Sete Portas, Socorro e outras circunvizinhas; ao ser materializado o projeto, entretanto, foram beneficiadas apenas as ruas da Alegria e do Sangradouro, ``quando a das Sete Portas é muito mais concorrida do que aquellas''\footnote{\textbf{A Noticia}, ano I, nº 206, 29 maio 1915, p. 1.}.

Em 1927 foi mandada abrir a estrada entre Amaralina e Pituba\footnote{\textbf{O Combate}, ano I, nº 99, 05 out. 1927, p. 1.}, cuja construção já era especulada em 1913\footnote{\textbf{Gazeta de Notícias}, ano III, nº 155, 14 mar. 1913, p. 1.}.

Verifica-se também a pressão e a mobilização de moradores como fator de abertura de novas vias, como na notícia a seguir, de 1926:

\begin{citacao}
De algum tempo vem se esforçando um grupo de moradores do aprazível bairro de Amaralina, tendo à frente como paladinos os srs. Arnaldo Moreira e João da Cunha Freire, para a construção de uma Avenida, aproveitando grande trecho de areial abandonado, e que se denomina Avenida Álvares do Amaral.
Várias são as vivendas ali edificadas, e mais ainda estão sendo construídas e outras remodeladas.
O primeiro trecho da projectada Avenida vae ser inaugurado em 1º de janeiro, compreendendo o perímetro do Bar daquelle local até à estação do telegrapho sem fio\footnote{\textbf{A Capital}, ano I, nº 80, 30 dez. 1926, p. 2. }.
\end{citacao}

Diga-se de passagem que no ano seguinte este mesmo jornal elogiaria a iniciativa de moradores do Tororó em prol da remodelação deste bairro, tendo justamente esta mobilização em Amaralina como modelo, afirmando: ``Só mesmo sendo assim. Quem não quizer vêr o logar em que mora com aspecto colonial, sujo, anti-esthetico, que tome a iniciativa de remodelal-o''\footnote{\textbf{A Capital}, ano I, nº 139, 25 mar. 1927, p. 6.}.

Na série de relatórios anuais da Intendência Municipal de Salvador custodiada no BR-BAAHMS, o próximo é o de 1929. Nele é possível observar como Brotas não era, propriamente, o distrito prioritário em termos de investimentos viários por parte da Prefeitura, conforme se vê pelos dados expostos na \autoref{tab:calcamento1929} (p. \autoref{tab:calcamento1929}): o distrito ficou em sexto lugar em área calçada, com metragem quadrada representando apenas 9,46\% do total (comparativamente, no mesmo ano a metragem quadrada do calçamento no distrito da Vitória, primeiro lugar entre todos, representou 28,58\% do total).

\begin{table}[!htp]
\centering
\IBGEtab{
\caption{Área calçada pela Prefeitura de Salvador em 1929, por distrito da cidade}\label{tab:calcamento1929}}
{\begin{tiny}
\begin{tabular}{cc}
\toprule
Distrito			&Área calçada ($m^{2}$)\\
\midrule \midrule
Vitória				&36.823,41\\
Mares				&20.482,87\\
Penha				&19.604,03\\
Santo Antônio		&15.234,82\\
São Pedro			&13.602,28\\
Brotas				&12.189,40\\
Nazaré				&2.985,95\\
Santana				&2.703,20\\
Conceição da Praia	&2.566,94\\
Pilar				&2.382,48\\
Sé					&270,79\\
\midrule
TOTAL				&128.846,17\\
\bottomrule
\end{tabular}
\end{tiny}
}
{\fonte{Elaboração do autor, com base no relatório de 1929 do intendente de Salvador ao Conselho Municipal \cite[pp.~11-12]{salvador_relatorio_1929}.}}
\end{table}

No relatório ficaram de fora, propositalmente destacados, os 1.870$m^{2}$ da obra de pavimentação das “rampas da nova estrada da Cruz das Almas”, concebida em função “do percurso em automovel para quem se dirige do Rio Vermelho a qualquer dos districtos centraes”, visto que “offerece a nova estrada a vantagem de encurtar esse percurso em relação ao que ordinariamente se faz pela Avenida Oceanica”; como o relatório afirma terem sido recolhidos 15:000\$000 dos proprietários na forma de terrenos marginais doados pelos proprietários para o alargamento da estrada, e que este valor teria representado “menos de 10\% da importância despendida pela Prefeitura”, é possível deduzir daí que a obra teria custado 150:000\$000 – valor que, comparado com os 246:805\$362 despendidos para a construção de um necrotério no Asilo de Mendicidade no mesmo ano, diminui da estatura exorbitante que parecia ter num primeiro momento \cite[pp.~22-23]{salvador_relatorio_1929}.

\subsection{Iluminação pública, energia elétrica, telefonia e telégrafos}

A iluminação pública em Brotas era tão precária quanto a própria iluminação pública no resto de Salvador, excetuados alguns logradouros onde residiam pessoas de influência.

Salvador tinha em 1894, segundo a \textit{Bahia Gas Company Ltd.} que prestava o serviço de iluminação pública, uma rede de encanamento de gás para os combustores que chegava a 99.894,45 metros de tubos de ferro fundido com diversos diâmetros \cite[p.~178]{salvador_relatorio_1894}. A rede de canalização recebeu em 1895 mais 909,9 metros de tubos com 75mm de diâmetro no Castro Neves, Sangradouro e Socorro. Mesmo assim, a iluminação em Brotas ainda era fraca: de 74 novos combustores instalados naquele ano, o distrito recebeu 30\footnote{6 no Matatu, 6 na Boa Vista, 3 na ladeira dos Galés, 2 na rua do Socorro, 1 na ladeira da Fonte das Pedras e impressionantes 12 na rua do Castro Neves.} \cite[pp.~149-150]{salvador_relatorio_1895}, indicando forte ausência anterior; em 1894 dois novos combustores foram instalados no beco do Cego \cite[p.~179]{salvador_relatorio_1894}. Teria tal preocupação do governo estadual com a iluminação nesta vizinhança resultado dos requerimentos dos moradores do Castro Neves à Diretoria Municipal de Obras em 1891\footnote{\textbf{Pequeno Jornal}, ano I, nº 283, 24 jan. 1891, p. 1.}? 

A Boa Vista vivia situação calamitosa, pois o serviço de iluminação pública nunca foi regular na localidade. Em 1926, por exemplo, publicou-se na imprensa um protesto dirigido à Intendência Municipal, segundo o qual ``diariamente falta energia na localidade''\footnote{\textbf{A Capital}, ano I, nº 27, 21 out. 1926, p. 2. O protesto seria reiterado em pelo menos duas edições posteriores do mesmo jornal (\textbf{A Capital}, ano I, nº 52, 19 nov. 1926, p. 2; ano I, nº 57, 26 nov. 1926, p. 2).}

O Matatu, então, mais distante do centro de Salvador que a Boa Vista, era um breu. Em 1920 matéria sobre o mau serviço de policiamento na localidade -- aliás a falta dele, pois depois das 21h ``nem sombra de soldado nas ruas'' -- dizia-se que o Matatu era ``logar solitario, à noite, pela sua natureza e deserto pela falta de illuminação'', verdadeiro ``perigo áquelles que, por necessidade, só podem regressar á casa ás dez horas''\footnote{\textbf{A Manhã}, ano I, nº 104, 18 ago. 1920, p. 1. O mau policiamento seguiria sendo pauta em 1926, quando o subdelegado de polícia pediu ao Secretário de Polícia para aumentar o número de policiais na vizinhança (\textbf{A Capital}, ano I, nº 41, 06 nov. 1926, p. 2).}.

A \textit{iluminação domiciliar} em Brotas começou a República tão tremeluzente quanto a pública:

\begin{citacao}
Os prédios d'esta cidade em que se faz uso da illuminaçâo por meio do gaz sâo em número de 1.246, dos quaes 370 na freguezia de S. Pedro e 219 no da Victoria. constituindo o primeiro districto dos cobradores; e 112 na da Penha 69 na dos Mares, 16 na do Pilar, 66 na da Conceição da Praia. 42 na do Santo Antonio 62 na da Rua do Passo, 163 na de Sant'Anna, 121 na da Sé e 6 na de Brotas formando o segundo districto \cite[p.~151]{salvador_relatorio_1895}. 
\end{citacao}

No que diz respeito à \textit{telefonia}, Brotas não era propriamente um distrito desprovido de atenção. Sobre ela, encontramos a seguinte informação, datada de 1926:

\begin{citacao}
O serviço telephonico, tanto urbano, como interurbano, também está a cargo da Inspectoria de Viação [\textit{órgão do governo estadual}].
O contracto respectivo para esta Capital provinha de uma concessão feita pela Monarchia, em 1884.
O Governo da União, em l924, transferiu suas obrigações e direitos ao Governo do Estado, que fez novo contracto com a \textit{Companhia Brasileira de Energia Eléctrica}, a 26 de Novembro do mesmo anno.
O serviço está actualmente bem montado e com alguma canalização subterrânea.
Há quatro estações: \textit{Central}, \textit{Garcia}, \textit{Roma} e \textit{Rio Vermelho} e por ellas o anno passado se fizeram 62.600 ligações diárias.
Uma nova estação, às Pitangueiras, em Brotas, acaba de ser installada para 120 linhas, com 4 telephonistas.
Encontra-se em remodelação a estação de Roma.
No fim do anno passado existiam 3.221 telephonios e 433 extensões.
A extensão da rêde aérea naquella época era de 24.700 metros e da subterrânea de 24.200, tendo augmentado a primeira no segundo semestre de 300 metros e a segunda
de 1.200, com capacidade para 600 telephonios.
Para a linha de Brotas serão precisos 3.500 metros de linha aérea. \cite[pp.~266-267]{bahia_rpe_1926}
\end{citacao}

Como se vê, naquele ano o Rio Vermelho já era dotado de central telefônica, outra havia sido recém-instalada nas Pitangueiras e a rede ainda seria expandida. Tratava-se, como se verá adiante, de duas áreas onde a valorização da terra era alta, e Pitangueiras encontrava-se no coração da área com urbanização mais antiga e contínua de todo o distrito.

Uma curiosidade distinguia o distrito de Brotas: a existência da \textit{estação telegráfica de Amaralina}, atestada em múltiplas notícias e documentos consultados, que servia não somente ao distrito mas a toda Salvador e também a Bahia.Um temporal em dezembro de 1913 danificou a antena da estação, interrompendo o serviço pois a antena ameaçava desabar\footnote{\textbf{Gazeta de Notícias}, ano IV, nº 69, 02 dez. 1913, p. 2.}; cerca de duas semanas depois o chefe da estação comunicou pela imprensa o pleno restabelecimento do serviço, com o conserto da antena\footnote{\textbf{Gazeta de Notícias}, ano IV, nº 77, 12 dez. 1913, p. 2.}.

\subsection{Transporte público}

A história das linhas de bonde em Brotas é bem documentada, pois insere-se na história geral do transporte público de Salvador. Aquilo de que não se dispunha sistematização era a história das lutas dos moradores de Brotas por melhor transporte, que se pôde localizar por meio da imprensa da época.


Em 1906 foram iniciadas as obras para implantação da linha de bonde entre Sete Portas e Brotas pela Trilhos Centrais, e em 1916 foi inaugurada a linha que levava ao Acupe \cite[p.~299]{VASCONCELOS2002}. Em 1926 Brotas era atendida por três linhas de bonde da Trilhos Centrais: a 11 (Brotas via Baixa dos Sapateiros e Sete Portas), a 15 (Rio Vermelho via São Pedro e Fonte Nova) e a 16 (Amaralina via São Pedro e Campo Grande)\footnote{\textbf{A Capital}, ano I, nº 22, 15 out. 1926, p. 3}.

No que diz respeito ao ramal operado pela Companhia Linha Circular, parecia tudo correr conforme o contratado com a Intendência em 1905, quando se sugeriu inclusive que esta empresa ``estabeleça communicação directa entre o largo de Nazareth e o alto da ladeira das Pitangueiras, facilitando consideravelmente assim a visita dos alumnos da Faculdade ao asylo de S. João de Deus'' e, por meio disto, da construção de um viaduto entre Nazaré e as Pitangueiras e da extensão da energia elétrica até a Boa Vista, resolveria a questão do acesso ao nosocômio, ``difficuldade que tem constituido verdadeiro espantalho para muitos''\footnote{\textbf{Correio do Brasil}, ano III, nº 528, 15 jun. 1905, p. 1.}. 

Acontece, entretanto, que enquanto o ramal de Amaralina havia sido inaugurado pela Companhia Linha Circular em 9 de setembro de 1911\footnote{\textbf{Gazeta de Notícias}, ano III, nº 9, 17 set. 1912, p. 3.}, em 1914 as obras do ramal de Brotas ainda não se achavam concluídas, levando a uma enxurrada de reclamações dos moradores do Sangradouro e das Pitangueiras contra a Intendência\footnote{\textbf{A Notícia}, ano I, nº 64, 01 dez. 1914, p. 1.}. Mesmo o ramal da Fonte Nova, aproveitado pelos moradores do Castro Neves, Sangradouro e Pitangueiras, causava esperas de trinta a quarenta minutos por um bonde que já vinha cheio do Rio Vermelho\footnote{\textbf{Gazeta de Noticias}, ano IV, nº 70, 03 dez. 1913, p. 1.}. A situação não agradava, e os protestos avolumavam-se, ainda mais quando eram respondidos pela Linha Circular apenas com paliativos: ``de quando em quando, um grupo de operarios ataca o serviço que paralysa dous dias depois, para ser recomeçado passados seis meses''\footnote{\textbf{Gazeta de Notícias}, ano IV, nº 188, 02 maio 1914, p. 2.}. 

Acuada, a empresa Trilhos Centrais, por meio de um seu engenheiro-fiscal, comunicou ao Intendente que ``o assentamento dos trilhos está bem feito e a linha tôda reconstruída até o alto da Capella, nas Pitangueiras''\footnote{Este ``alto da Capella'' é, muito provavelmente, o \textit{largo do Paranhos}, um dos pontos extremos da rua das Pitangueiras, onde se situa até hoje a igreja do Senhor Bom Jesus dos Milagres (cf. \autoref{subsec:pontrel}, p. \pageref{subsec:pontrel}), na época uma simples capela.}, e que, ``deante do estado em que estão as obras'', era provável que o trecho das Pitangueiras até a Boa Vista ficasse pronto em até um mês\footnote{\textbf{A Notícia}, ano I, nº 184, 01 maio 1915, p. 1.}. 

Não sem certo atraso, em 4 de junho foi inaugurado o ramal de bondes elétricos de Brotas, saindo do largo das Sete Portas e percorrendo ``Matatú Pequeno, Paranhos, rua do general e Pitangueiras, entrando por uma travessa que sae naquella rua''\footnote{Segundo a nomenclatura atual dos logradouros, e pela ordem apresentada, o trajeto seria o seguinte: largo das Sete Portas, rua dos Bandeirantes e rua Barros Falcão (``Matatú Pequeno'', largo do Paranhos, rua Affonso Taunay (``beco do general''), rua das Pitangueiras, rua Frederico Costa (atravessando o vale do Bonocô por um viaduto), rua Boa Vista de Brotas.}; a inauguração foi um verdadeiro festejo popular, literalmente apoteótico, com participação de autoridades diversas\footnote{Foram destacadados o intendente Azevedo Fernandes; o secretário Pedro Gordilho; o diretor técnico da Trilhos Centrais, Noronha Santos; o chefe de tráfego Álvaro Campos; vários conselheiros municipais (equivalente aos atuais vereadores) e representantes de grandes jornais da época (\textbf{Jornal de Notícias}, \textbf{A Tarde}, \textbf{Diário da Bahia}, \textbf{Gazeta do Povo}, \textbf{A Notícia}).}, fogos de artifício, ornamentação do trajeto com  ``bandeiras e folhagens'', música executada pela banda do 50º Batalhão de Caçadores na passagem pela ``chácara da viúva Correa Machado'' e pela banda da Polícia Militar no ponto terminal da linha, recepção das autoridades nas residências dos ``Tavares'', dos ``Machado'' e do sobrinho do intendente, o já conhecido conselheiro municipal Arnaldo Silvany, residente no Castro Neves\footnote{Toda esta cornucópia foi relatada numa notícia de três colunas e duas fotos enormes da primeira página d'\textbf{A Notícia} (ano I, nº 233, p. 1); tal extensão e intensa ilustração é pouco usual, reservada apenas para notícias de grandíssimos impacto e relevância.}. 

Mas ninguém se apresse a dizer que tudo terminou bem: ao mesmo tempo em que era comemorada a inauguração do novo trecho, vinha embutida na mesma notícia a reivindicação de um ponto de bonde ``no trecho do largo do Paranhos à rua do general''\dots e ainda em 1920 os moradores do Castro Neves protestavam pela imprensa e encaminhavam petições à diretoria da Linha Circular reclamando por bondes mais próximos de suas residências\footnote{\textbf{A Manhã}, ano I, nº 106, 15 ago. 1920, p. 6.}. Ademais, a concorrência entre a Trilhos Centrais e a Linha Circular, como se sabe, era uma simples fachada: o grupo \textit{Guinle \& Cia.} controlava as duas, e portanto monopolizava a prestação do serviço de transportes para o distrito de Brotas.

\subsection{Saneamento, esgotamento e abastecimento de água}

Em 1893 foi instalado um “cano” de esgoto no Santo Agostinho, além da limpeza e beneficiamento das fontes de Brotas e do riacho dos Bois \cite[pp.~6-7]{salvador_relatorio_1893}.

Em 1894 foram limpas a fonte e o riacho do Boi por 350\$000, e outros 5:200\$000 foram investidos na limpeza de 9.372 metros do rio Camarajipe e de outros 3.101 metros do rio das Tripas \cite[p.~156]{salvador_relatorio_1894}.

Em 1895 foi limpo o riacho do Sangradouro, a rua Primeiro de Março (Pitangueiras) recebeu um sifão e outros dois foram instalados também nas Pitangueiras sem se estabelecer com maior precisão o endereço da obra \cite[pp.~20, 137-138]{salvador_relatorio_1895}. 

Em 1897 foi reparada e limpa uma fonte no Matatu pelo “artista” Cassiano Godinho, ao custo de 186\$000; no mesmo ano, foi desobstruído o riacho dos Bois, no Rio Vermelho, pelo sr. Horácio Pinto de Barros Paim, que recebeu 150\$000 pelo serviço; também foi desobstruído o rio Camorogipe entre 16 de fevereiro a 1 o de maio, custando a obra o total de 1:326\$000/ e o “riacho da Lucaia” foi “dirivado“ no mesmo ano, obra executada pelo sr. Joaquim José da Silva Fialho que custou 1:815\$240 \cite[p.~99-100]{salvador_relatorio_1897}. Destaca-se neste ano a ausência de investimentos municipais em “canos de esgotos, syphões, etc.“, comuns em outros distritos.


Em 1906, como parte dos trabalhos de esgotamento e abastecimento de água conduzidos por Teodoro Sampaio, a Intendência Municipal desapropriou 14,9 milhões de metros quadrados de terra pertencentes à Ordem de São Bento ``entre a sesmaria de Rio Vermelho e a de Garcia D'Ávila''; a ordem religiosa terminou, além disto, vendendo em 1917 suas terras remanescentes na mesma área à Intendência \cite[p.~306]{VASCONCELOS2002}.

Em 1917 o serviço de água e esgoto chegou, enfim, ao Asilo São João de Deus \cite[p.~63]{bahia_rpe_1914}, o que, por consequência, tornava os imóveis circunvizinhos da Boa Vista aptos a receber o mesmo serviço; em 1925, todavia, uma comissão de moradores do Engenho Velho de Brotas (onde se localizam a Boa Vista e o Asilo São João de Deus) levou à imprensa seu protesto, pois fazia vários dias que o abastecimento de água na localidade era irregular, e a água quando ali chegava era ``barrenta e de máo cheiro, sem nenhuma serventia para o gasto''\footnote{\textbf{Correio do Povo}, ano II, nº 61, 29 jun. 1925, p. 2.}.

Em julho de 1920 noticiava-se, em tom de protesto, que um certo Adolpho Moreira havia fechado ``com cerca'' uma das ruas da Mariquita, ``impedindo, miseravelmente, a população dali proverse de agua da Fonte do Boi, posta dentro do cercado''\footnote{\textbf{A Manhã}, ano I, nº 68, 01 jul. 1920, p. 3.}. Não se sabe se a questão foi resolvida.

Em 1922 estavam em curso estudos para ``saneamento dos pântanos da Mariquita e de Amaralina'', sob responsabilidade da Diretoria de Engenharia Sanitária do governo estadual \cite[p.~409]{bahia_rpe_1922}

\subsection{Conclusões provisórias}
