\section{Investimentos públicos}\label{sec:3.1}

INTRODUZIR

\subsection{Esgotamento e abastecimento de água}\label{subsec:3.1.1}

USAR FALAS DE GOVERNADORES E RELATÓRIOS DE INTENDÊNCIA

\subsection{Arruamento e pavimentação}\label{subsec:3.1.2}

USAR FALAS DE GOVERNADORES E RELATÓRIOS DE INTENDÊNCIA

\begin{citacao}
Do Município desta Capital -- 32.000 metros de 
\end{citacao}

\subsection{Iluminação pública}\label{subsec:3.1.3}

USAR FALAS DE GOVERNADORES E RELATÓRIOS DE INTENDÊNCIA

\subsection{Transporte público}\label{subsec:3.1.4}

USAR FALAS DE GOVERNADORES E RELATÓRIOS DE INTENDÊNCIA

\subsection{Cemitério}\label{subsec:3.1.5}

USAR FALAS DE GOVERNADORES E RELATÓRIOS DA INTENDÊNCIA

DESTACAR OS REITERADOS ELOGIOS À ADMINISTRAÇÃO DO CEMITÉRIO NOS DOCUMENTOS OFICIAIS E OUTROS

\subsection{Serviços de saúde}\label{subsec:3.1.6}

USAR FALAS DE GOVERNADORES E RELATÓRIOS DA INTENDÊNCIA

\subsection{Escolas}\label{subsec:3.1.7}

USAR RELATÓRIOS DA INTENDÊNCIA, QUE DISCRIMINAM AS ESCOLAS

\subsection{Telefonia}\label{subsec:3.1.8}

\begin{citacao}
O serviço telephonico, tanto urbano, como interurbano, também está a cargo da Inspectoria de Viação.
O contracto respectivo para esta Capital provinha de uma concessão feita pela Monarchia, em 1884.
O Governo da Únião, em l924, transferiu suas obrigações e direitos ao Governo do Estado, que fez novo contracto com a \textit{Companhia Brasileira de Energia Eléctrica}, a 26 de Novembro do mesmo anno.
O serviço está actualmente bem montado e com alguma canalização subterrânea.
Há quatro estações: \textit{Central}, \textit{Garcia}, \textit{Roma} e \textit{Rio Vermelho} e por ellas o anno passado se fizeram 62.600 ligações diárias.
Uma nova estação, ás Pitangueiras, em Brotas, acaba de ser installada para 120 linhas, com 4 telephonistas.
Encontra-se em remodelação a estação de Roma.
No fim do anno passado existiam 3.221 telephonios e 433 extensões.
A extensão da rêde aérea naquella época era de 24.700 metros e da subterrânea de 24.200, tendo augmentado a primeira no segundo semestre de 300 metros e a segunda
de 1.200, com capacidade para 600 telephonios.
Para a linha de Brotas serão precisos 3.500 metros de linha aérea. \cite[pp.~266-267]{bahia_rpe_1926}
\end{citacao}