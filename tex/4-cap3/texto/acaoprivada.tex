\section{Ação privada}\label{sec:acaoprivada}

Paralelamente aos investimentos públicos, é necessário levar em conta a ação de \textit{agentes privados} na produção, apropriação e uso do espaço do distrito. A natureza de tal ação apresentou-se em campo distinto daquele analisado até o momento: enquanto os investimentos públicos estiveram voltados principalmente para a construção de equipamentos coletivos, infraestruturas de transporte, circulação e comunicação etc., os agentes privados apresentaram como prioridade neste distrito, em ordem decrescente de ocorrências, as construções de novos imóveis, as ampliações e reformas de imóveis existentes e os loteamentos. Cada qual será examinado em detalhe, não sem antes se estabelecerem alguns pontos acerca da natureza do uso e do valor da terra no distrito.

\subsection{Que apropriação da terra em Brotas?}\label{subsec:apropribrotas}

O regime de terras vigente em Brotas era o mesmo vigente em Salvador no mesmo período (cf. \autoref{subsubsec:polfundvalter}, p. \pageref{subsubsec:polfundvalter}): uma transição do regime da Lei 601/1850 para o regime de propriedade plena disciplinado pelo Código Civil de 1916, com todas as ``porosidades'' comuns ao período -- posses consuetudinárias, ocupação informal de terras, aforamentos, arrendamentos, proliferação de ``moradores'' etc. Na maioria dos pedidos de licença de obra pesquisados não se faz referência explícita à situação fundiária dos imóveis a construir ou reformar, pois não se exigia tal informação para a concessão das licenças necessárias; nos pouquíssimos pedidos de licença onde se menciona a situação fundiária do imóvel, contados literalmente nos dedos de duas mãos, é para registrar a situação de ``foreiro'' ou de ``arrendatário'' do requerente. 

No que diz respeito ao \textit{valor da terra} no distrito, apresenta-se, de forma parecida com o ocorrido na \autoref{subsubsec:polfundvalter} (p. \pageref{subsubsec:polfundvalter}), uma questão de método. A forma correta de auferir o valor da terra de forma direta é a pesquisa nos \textit{Livros de Décimas Urbanas} custodiados no Arquivo Público Municipal de Salvador, que por sinal encontram-se em excelente estado de conservação. Esta abordagem direta, entretanto, exigiria operações complexas e trabalhosas\footnote{Seria preciso, em primeiro lugar, estudar as categorias do lançamento tributário, ou seja, a forma que os funcionários da Intendência/Prefeitura empregavam para anotar os valores nos livros; depois, seria necessário o hercúleo trabalho de cópia do valor locativo de cada imóvel indicado pela Intendência/Prefeitura; em seguida, seria preciso construir tabelas para cada \textit{rua} (só no livro de 1893 estão registradas 41 ruas); logo após, todas as tabelas precisariam ser agrupadas num só banco de dados, modelado de forma a facilitar a consulta, a recuperação e o cruzamento de dados; por fim, os dados da décima urbana de cada ano precisariam ter sua consistência checada por meio de testes estatísticos diversos. Como se vê, trata-se de um trabalho cuja execução pede uma equipe dedicada, não um pesquisador solitário.}.

Por sorte, um tal trabalho já foi feito por \citeonline[pp.~263-264]{bahia_annuario_1926} com base nos valores da décima urbana de 1924, o que permitiu construir a \autoref{tab:imoveis1924brotas1} e a \autoref{tab:imoveis1924brotas2} (pp. \pageref{tab:imoveis1924brotas1} e \pageref{tab:imoveis1924brotas2}) para lançar luzes sobre a questão. 

\begin{sidewaystable}
\IBGEtab{
\caption{Relação dos imóveis arrolados pelo município de Salvador no distrito de Brotas em 1924 (parte 1)}\label{tab:imoveis1924brotas1}}
{
\begin{minipage}{\textwidth}
\begin{tiny}
\begin{tabular}{m{3cm} m{1cm} l l l l l l l l l l l}
\toprule
\multirow{2}{*}{Locais}	& \multirow{2}{*}{Valor}	& \multicolumn{10}{c}{Imóveis}\\
\cline{3-13}
	&	&Térreos	&Sobrados	&Abarracados	&Barracão	&Telheiros	&Galpões	&Em ruínas	&Em construção	&Em reconstrução	&Interditados	&TOTAL\\
\midrule
\midrule
Rua dos Galés							&26:184\$	&22	&0	&1	&0	&0	&0	&0	&0	&0	&0	&23\\
Rua Coronel Frederico Costa					&7:440\$	&7	&0	&0	&0	&0	&0	&0	&0	&0	&0	&7\\
Rua Uruguaiana							&108:984\$	&143	&2	&0	&0	&0	&0	&0	&9	&0	&0	&154\\
Trav. da Rua Uruguaiana					&30:304\$	&120	&0	&0	&0	&0	&0	&0	&1	&0	&0	&121\\
Rua da Boa Vista						&38:980\$	&60	&2	&0	&0	&0	&0	&0	&1	&0	&0	&63\\
Rua Agrippino Dorea						&67:832\$	&77	&0	&2	&0	&0	&0	&1	&0	&0	&0	&80\\
Beco do General							&4:800\$	&9	&0	&0	&0	&0	&0	&0	&0	&0	&0	&9\\
Rua do Socorro							&31:680\$	&59	&0	&0	&0	&0	&0	&0	&0	&1	&0	&60\\
Trav. Castro Neves					&4:620\$	&10	&0	&0	&0	&0	&0	&0	&0	&0	&0	&10\\
Rua do Castro Neves						&54:600\$	&83	&0	&0	&0	&0	&0	&1	&2	&0	&0	&86\\
Rua da Alegria							&16:676\$	&30	&0	&0	&0	&0	&0	&0	&0	&0	&0	&30\\
Trav. da Alegria						&14:124\$	&23	&0	&0	&0	&0	&0	&0	&0	&0	&0	&23\\
Trav. do Sangradouro para a Trav. da Alegria		&13:728\$	&16	&0	&0	&0	&0	&0	&0	&0	&0	&0	&16\\
Trav. do Sangradouro para a Rua da Alegria			&12:096\$	&18	&0	&0	&0	&0	&0	&0	&0	&0	&0	&18\\
Trav. do Sangradouro para a Lad. do Matatu Pequeno	&12:120\$	&27	&0	&0	&0	&0	&0	&0	&0	&0	&0	&27\\
Rua do Sangradouro						&46:072\$	&61	&3	&0	&0	&0	&0	&0	&0	&0	&0	&64\\
Trav. do Sangradouro						&19:228\$	&	&66	&0	&0	&0	&0	&0	&0	&0	&0	&66\\
Alto do Sangradouro						&9:348\$	&18	&0	&0	&0	&0	&0	&0	&1	&0	&0	&19\\
Rua da Vala ao Cabula						&34:592\$	&76	&6	&0	&0	&0	&0	&0	&1	&0	&0	&83\\
Estr. 2 de Julho						&94:186\$	&292	&0	&0	&0	&0	&0	&1	&5	&0	&0	&298\\
1ª Lad. do Engenho Velho					&6:684\$	&19	&0	&0	&0	&0	&0	&0	&0	&0	&0	&19\\
2ª Lad. do Engenho Velho					&4:524\$	&20	&0	&0	&0	&0	&0	&0	&0	&0	&0	&20\\
Rua do Engenho Velho						&30:576\$	&112	&0	&0	&0	&0	&0	&0	&0	&0	&0	&112\\
Capelinha do Deus Menino					&102:924\$	&416	&0	&0	&0	&0	&0	&0	&5	&0	&0	&421\\
Quinta das Beatas						&39:432\$	&149	&0	&0	&0	&0	&0	&0	&2	&0	&0	&151\\
1ª Trav. da Quinta das Beatas				&2:700\$	&13	&0	&0	&0	&0	&0	&0	&0	&0	&0	&13\\
2ª Trav. da Quinta das Beatas				&1:860\$	&7	&0	&0	&0	&0	&0	&0	&0	&0	&0	&7\\
Alto do Formoso							&11:340\$	&50	&0	&0	&0	&0	&0	&0	&1	&1	&0	&52\\
Rua do Matatu Pequeno						&57:920\$	&70	&0	&0	&0	&0	&0	&0	&4	&0	&0	&74\\
Rua do Matatu Grande						&33:672\$	&86	&0	&0	&0	&0	&0	&0	&0	&0	&0	&86\\
Casa da Pólvora							&13:836\$	&30	&0	&0	&0	&0	&0	&0	&0	&0	&0	&30\\
\bottomrule
(continua na parte 2) & & & & & & & & & & & & \\
\end{tabular} 
\end{tiny}
\end{minipage}
}
{\fonte{\textbf{Annuario estatistico – annos de 1924 e 1925} organizado para o Governo da Bahia por M. Messias de \citeonline[pp.~263-264]{bahia_annuario_1926}.}}
\end{sidewaystable}
\begin{table}[!htp]
\IBGEtab{
\caption{Relação dos imóveis arrolados pelo município de Salvador no distrito de Brotas em 1924 (parte 2)}\label{tab:imoveis1924brotas1}}
{
\begin{tiny}
\begin{tabular}{m{3cm} m{1cm} m{0,7cm} m{0,7cm} m{0,7cm} m{0,7cm} m{0,7cm} m{0,7cm} m{0,7cm} m{0,7cm} m{0,7cm} m{0,7cm} m{0,7cm}}
\hline
\multirow{2}{*}{Locais}	& \multirow{2}{*}{Valor}	& \multicolumn{10}{c}{Imóveis}\\
\cline{3-13}
	&	&Tér- reos	&Sobra- dos	&Abarra- cados	&Barra- cão	&Telhei- ros	&Gal- pões	&Em ruínas	&Em cons- trução	&Em re- cons- trução	&Inter- dita- dos	&TOTAL\\
\hline
\hline
Lad. do Fabrício						&28:620\$	&27	&0	&2	&0	&0	&0	&0	&0	&1	&0	&30\\
Lad. do Acupe						&5:700\$	&13	&0	&0	&0	&0	&0	&0	&0	&1	&0	&14\\
Rua do Acupe							&16:332\$	&19	&0	&0	&0	&0	&0	&0	&0	&0	&0	&19\\
Trav. do Acupe						&1:740\$	&3	&0	&0	&0	&0	&0	&0	&0	&0	&0	&3\\
Rua de Brotas							&62:476\$	&65	&0	&1	&0	&0	&0	&0	&0	&0	&0	&66\\
1ª Trav. da Rua de Brotas					&6:156\$	&6	&0	&0	&0	&0	&0	&0	&0	&0	&0	&6\\
Cruz da Redenção						&12:276\$	&25	&0	&0	&0	&0	&0	&0	&0	&0	&0	&25\\
Rua do Beiju							&9:204\$	&30	&0	&0	&0	&0	&0	&0	&0	&0	&0	&30\\
Trav. da Rua do Beiju					&180\$		&1	&0	&0	&0	&0	&0	&0	&0	&0	&0	&1\\
Rua das Campinas						&7:680\$	&12	&0	&0	&0	&0	&0	&0	&0	&0	&0	&12\\
Vargem de Santo Antônio						&1:200\$	&2	&0	&0	&0	&0	&0	&0	&0	&0	&0	&2\\
Trav. do Pomar						&1:500\$	&8	&0	&0	&0	&0	&0	&0	&0	&0	&0	&8\\
Pomar								&240\$		&2	&0	&0	&0	&0	&0	&0	&0	&0	&0	&2\\
Candeal Pequeno							&1:680\$	&9	&0	&0	&0	&0	&0	&0	&0	&0	&0	&9\\
Candeal Grande							&360\$		&1	&0	&0	&0	&0	&0	&0	&0	&0	&0	&1\\
Lad. da Cruz das Almas					&12:912\$	&26	&0	&0	&0	&0	&0	&0	&0	&0	&0	&26\\
Largo da Mariquita						&24:480\$	&14	&0	&1	&0	&0	&0	&0	&0	&1	&0	&16\\
Rua dos Dendezeiros						&38:430\$	&32	&0	&0	&0	&0	&0	&0	&0	&1	&0	&33\\
Trav. da Rua dos Dendezeiros para a Rua do Meio		&816\$		&2	&0	&0	&0	&0	&0	&0	&0	&0	&0	&2\\
Rua do Meio							&17:610\$	&20	&0	&0	&0	&0	&0	&1	&0	&0	&0	&21\\
Rua Direita							&37:656\$	&37	&0	&0	&0	&0	&0	&0	&0	&0	&0	&37\\
Rua Fonte do Boi						&10:248\$	&13	&0	&0	&0	&0	&0	&0	&0	&0	&0	&13\\
Rua das Pedrinhas						&22:200\$	&30	&0	&0	&0	&0	&0	&0	&0	&0	&0	&30\\
1ª Trav. da Rua das Pedrinhas				&5:220\$	&9	&0	&0	&0	&0	&0	&0	&0	&0	&0	&9\\
2ª Trav. da Rua das Pedrinhas				&2:640\$	&5	&0	&0	&0	&0	&0	&0	&0	&0	&0	&5\\
Rua da Lagoa							&8:916\$	&34	&0	&0	&0	&0	&0	&0	&1	&0	&0	&35\\
Rua Direita da Amaralina					&27:700\$	&18	&0	&3	&0	&0	&0	&0	&0	&0	&0	&21\\
Rua do Meio da Amaralina					&17:220\$	&23	&0	&0	&0	&0	&0	&0	&0	&0	&0	&23\\
Alto da Ubarana							&900\$		&3	&0	&0	&0	&0	&0	&0	&0	&0	&0	&3\\
Pituba								&10:140\$	&24	&0	&0	&0	&0	&0	&0	&0	&0	&0	&24\\
Armação Pequena							&720\$		&2	&0	&0	&0	&0	&0	&0	&0	&0	&0	&2\\
Armação Grande							&600\$		&1	&0	&0	&0	&0	&0	&0	&0	&0	&0	&1\\
\hline
TOTAL								&1:346:814\$	&2:705	&13	&10	&0	&0	&0	&4	&33	&6	&0	&2771\\
\hline
\end{tabular} 
\end{tiny}
}
{\fonte{\citeonline[pp.~263-264]{bahia_annuario_1926}.}}
\end{table}


Graças a este trabalho, foi possível encontrar mais uma vez o valor locativo médio dos imóveis arrolados, dividindo a massa do valor locativo dos imóveis de cada rua pelo número de imóveis da mesma rua; por rudimentar que seja o procedimento, foi a solução encontrada para evitar as distorções verificáveis quando foi tentada a comparação direta entre as massas de valores locativos de cada rua, por força de fatores como as diferenças nos valores locativos de cada imóvel e as grandes variações no número de imóveis por rua. Os resultados foram compilados na \autoref{tab:valorlocativomedio1924brotas} (p. \pageref{tab:valorlocativomedio1924brotas}).

\begin{table}[!htp]
\IBGEtab{
\caption{Valor locativo médio dos imóveis nas ruas arroladas pelo município de Salvador no distrito de Brotas em 1924}\label{tab:valorlocativomedio1924brotas}}
{
\begin{tiny}
\begin{tabular}{ll}
\toprule
Locais	&Valor locativo médio por imóvel\\
\midrule
\midrule
Rua dos Galés	&1:138\$430\\
Rua Coronel Frederico Costa	&1:062\$860\\
Rua Uruguaiana	&707\$690\\
Travessa da Rua Uruguaiana	&250\$450\\
Rua da Boa Vista	&618\$730\\
Rua Agrippino Dorea	&847\$900\\
Beco do General	&533\$330\\
Rua do Socorro	&528\$000\\
Travessa do Castro Neves	&462\$000\\
Rua do Castro Neves	&634\$880\\
Rua da Alegria	&555\$870\\
Travessa da Alegria	&614\$090\\
Travessa do Sangradouro para a Travessa da Alegria	&858\$000\\
Travessa do Sangradouro para a Rua da Alegria	&672\$000\\
Travessa do Sangradouro para a Ladeira do Matatu Pequeno	&448\$890\\
Rua do Sangradouro	&719\$880\\
Travessa do Sangradouro	&291\$330\\
Alto do Sangradouro	&492\$000\\
Rua da Vala ao Cabula	&416\$770\\
Estrada 2 de Julho	&316\$060\\
1ª Ladeira do Engenho Velho	&351\$790\\
2ª Ladeira do Engenho Velho	&226\$200\\
Rua do Engenho Velho	&273\$000\\
Capelinha do Deus Menino	&244\$480\\
Quinta das Beatas	&261\$140\\
1ª Travessa da Quinta das Beatas	&207\$690\\
2ª Travessa da Quinta das Beatas	&265\$710\\
Alto do Formoso	&218\$080\\
Rua do Matatu Pequeno	&782\$700\\
Rua do Matatu Grande	&391\$530\\
Casa da Pólvora	&461\$200\\
Ladeira do Fabrício	&954\$000\\
Ladeira do Acupe	&407\$140\\
Rua do Acupe	&859\$580\\
Travessa do Acupe	&580\$000\\
Rua de Brotas	&946\$610\\
1ª Travessa da Rua de Brotas	&1:026\$000\\
Cruz da Redenção	&491\$040\\
Rua do Beiju	&306\$800\\
Travessa da Rua do Beiju	&180\$000\\
Rua das Campinas	&640\$000\\
Vargem de Santo Antônio	&600\$000\\
Travessa do Pomar	&187\$500\\
Pomar	&120\$000\\
Candeal Pequeno	&186\$670\\
Candeal Grande	&360\$000\\
Ladeira da Cruz das Almas	&496\$620\\
Largo da Mariquita	&1:530\$000\\
Rua dos Dendezeiros	&1:164\$550\\
Travessa da Rua dos Dendezeiros para a Rua do Meio	&408\$000\\
Rua do Meio	&838\$570\\
Rua Direita	&1:017\$730\\
Rua Fonte do Boi	&788\$310\\
Rua das Pedrinhas	&740\$000\\
1ª Travessa da Rua das Pedrinhas	&580\$000\\
2ª Travessa da Rua das Pedrinhas	&528\$000\\
Rua da Lagoa	&254\$740\\
Rua Direita da Amaralina	&1:319\$050\\
Rua do Meio da Amaralina	&748\$700\\
Alto da Ubarana	&300\$000\\
Pituba	&422\$500\\
Armação Pequena	&360\$000\\
Armação Grande	&600\$000\\
\midrule
TOTAL	&568\$170\\
\bottomrule
\end{tabular} 
\end{tiny}
}
{\fonte{Elaboração do autor, com base no \textbf{Annuario estatistico – annos de 1924 e 1925} organizado para o Governo da Bahia por M. Messias de  \citeonline[pp.~263-264]{bahia_annuario_1926}.}}
\end{table}


O uso do valor locativo médio dos imóveis de cada rua do distrito como indicador do valor da terra permite classificar as ruas de Brotas segundo o valor médio de seus imóveis, estabelecendo assim aquelas com imóveis ``mais caros'' e aqueloutras com imóveis ``mais baratos'' (ressalvando-se sempre tratar-se de valores médios, não de valores reais). Os resultados deste procedimento podem ser verificados nas tabelas \autoref{tab:maiscaros1924brotas} e \autoref{tab:maisbaratos1924brotas} (\pageref{tab:maisbaratos1924brotas}), dos quais se pode extrair conclusões importantes.

\begin{table}[!htp]
\IBGEtab{
\caption{Valor locativo médio dos imóveis nas dez ruas de imóveis ``mais caros'' entre os arrolados pelo município de Salvador no distrito de Brotas em 1924}\label{tab:maiscaros1924brotas}}
{
\begin{tabular}{rr}
\toprule
Locais	&Valor locativo médio por imóvel\\
\midrule
\midrule
Largo da Mariquita	&1:530\$000\\
Rua Direita da Amaralina	&1:319\$050\\
Rua dos Dendezeiros	&1:164\$550\\
Rua dos Galés	&1:138\$430\\
Rua Coronel Frederico Costa	&1:062\$860\\
1ª Travessa da Rua de Brotas	&1:026\$000\\
Rua Direita	&1:017\$730\\
Ladeira do Fabrício	&954\$000\\
Rua de Brotas	&946\$610\\
Rua do Acupe	&859\$580\\
\bottomrule
\end{tabular} 
}
{\fonte{Elaboração do autor, com base no \textbf{Annuario estatistico – annos de 1924 e 1925} organizado para o Governo da Bahia por M. Messias de \citeonline[pp.~263-264]{bahia_annuario_1926}.}}
\end{table}

\begin{table}[!htp]
\IBGEtab{
\caption{Valor locativo médio dos imóveis nas dez ruas de imóveis ``mais baratos'' entre os arrolados pelo município de Salvador no distrito de Brotas em 1924}\label{tab:maisbaratos1924brotas}}
{
\begin{tabular}{rr}
\hline
Locais	&Valor locativo médio por imóvel\\
\hline
\hline
Alto da Ubarana	&300\$000\\
Travessa do Sangradouro	&291\$330\\
Rua do Engenho Velho	&273\$000\\
2ª Travessa da Quinta das Beatas	&265\$710\\
Quinta das Beatas	&261\$140\\
Rua da Lagoa	&254\$740\\
Travessa da Rua Uruguaiana	&250\$450\\
Capelinha do Deus Menino	&244\$480\\
2ª Ladeira do Engenho Velho	&226\$200\\
Alto do Formoso	&218\$080\\
1ª Travessa da Quinta das Beatas	&207\$690\\
Travessa do Pomar	&187\$500\\
Candeal Pequeno	&186\$670\\
Travessa da Rua do Beiju	&180\$000\\
Pomar	&120\$000\\
\hline
\end{tabular}
}
{\fonte{Elaboração do autor, com base em \citeonline[pp.~263-264]{bahia_annuario_1926}.}}
\end{table}

\begin{itemize}
\item Três entre as ruas com imóveis ``mais caros'' (Largo da Mariquita, Rua Direita e Rua dos Dendezeiros) encontram-se no Rio Vermelho, que manteve durante todo o período seu caráter de \textit{arrabalde de veraneio} dos soteropolitanos mais ricos. 
\item Duas entre as ruas com imóveis ``mais baratos'' (Alto da Ubarana e Rua da Lagoa) e uma dentre aquelas com imóveis ``mais caros'' (Rua Direita da Amaralina) encontram-se em Amaralina, fato que será investigado mais adiante; a Rua Direita da Amaralina, entretanto, é a principal da Cidade Balneária Amaralina, talvez um dos primeiros loteamentos formais de Salvador, totalmente voltado para atividades veranistas de alto padrão.
\item A larga Rua dos Galés segue entre as mais valorizadas do distrito, especialmente por ser eixo central da mais antiga área urbanizada do distrito.
\item Duas vias que hoje chamaríamos de ``arteriais'' ou ``coletoras'', a Rua de Brotas e a Rua do Acupe, encontram-se entre as mais valorizadas do distrito, talvez exatamente pela centralidade que exercem sobre a circulação no território.
\item Quatro das ruas que integram a área do Engenho Velho de Brotas (Rua do Engenho Velho, Travessa da Rua Uruguaiana, Capelinha do Deus Menino, 2ª Ladeira do Engenho Velho) estão entre aquelas com imóveis ``mais baratos'', enquanto uma delas (Rua Coronel Frederico Costa) encontra-se no grupo das ruas com imóveis ``mais caros''; é de se notar que as ruas menos valorizadas são exatamente aquelas que se encontram nas encostas e barrancos do Engenho Velho, portanto as mais insalubres e de difícil acesso, enquanto a mais valorizada entre elas encontra-se na cumeada, em terreno plano, fazendo a ligação desta área do distrito com a valorizada Rua de Brotas.
\item Todas as ruas abertas na Quinta das Beatas e áreas circunvizinhas (Alto do Formoso, 1ª e 2ª Travessas e a própria Quinta das Beatas) encontram-se entre aquelas com imóveis ``mais baratos''.
\end{itemize}

O cálculo de proporção de sobrados empregue na \autoref{subsubsec:polfundvalter} é inaplicável ao distrito, pois das 63 ruas cujos dados foram disponibilizados por \citeonline[pp.~263-264]{bahia_annuario_1926} apenas quatro apresentam este tipo de imóvel (Rua da Vala ao Cabula, com 6; Rua do Sangradouro, com 3; Travessa da Uruguaiana, com 2; e Rua da Boa Vista, com 2); conquanto este indicador sirva para reforçar o que já se viu a respeito do valor da terra no Sangradouro e na cumeada da Boa Vista, empregá-lo sem maiores cuidados equivaleria a considerar todo o restante do distrito com valor de terra ínfimo, o que não condiz com a situação encontrada em meio ao conjunto da documentação pesquisada. 

\subsection{Locação: forma principal de acesso à terra e à moradia}\label{subsec:locatermor}

Verificou-se em todo o período estudado que tanto os arrendamentos quanto os alugueis prosseguiram como a forma preferencial de acesso à terra e à moradia. Como exemplo da tendência, a \autoref{tab:taxaloc1893} (p. \pageref{tab:taxaloc1893}) apresenta os dados relativos a 1893, que permitem chegar às seguintes conclusões:

\begin{table}[!htp]
\centering
\IBGEtab{
\caption{Imóveis do distrito de Brotas (1893), por condição da ocupação}\label{tab:taxaloc1893}}
{ \begin{tiny}
\begin{tabular}{m{4cm}llllll}
\toprule
Rua	&Proprietários	&Inquilinos	&Ausente	&Vazio/Fechado	&Ignorado	&TOTAL	\\
\midrule \midrule
Rua dos Galés	&10	&17	&1	&5	&0	&33	\\
Rua Uruguaiana	&5	&20	&0	&5	&0	&30	\\
Rua da Boa Vista	&13	&24	&6	&3	&0	&46	\\
Ladeira da Boa Vista	&1	&2	&0	&0	&1	&4	\\
Rua 1º de Março	&24	&52	&2	&5	&9	&92	\\
Rua do Socorro	&13	&30	&4	&5	&1	&53	\\
Travessa do Socorro	&0	&0	&0	&2	&0	&2	\\
Travessa do Castro Neves	&0	&7	&0	&0	&2	&9	\\
Direita do Castro Neves	&14	&42	&5	&4	&5	&70	\\
Rua da Alegria	&7	&14	&0	&3	&1	&25	\\
Travessa da Alegria para o Sangradouro	&4	&12	&0	&0	&0	&16	\\
Sangradouro	&7	&34	&5	&3	&0	&49	\\
Travessa do Sangradouro	&2	&15	&0	&1	&0	&18	\\
Estrada da Vala ao Cabula (lado direito)	&6	&49	&8	&5	&1	&69	\\
Estrada 2 de Julho	&20	&8	&5	&2	&5	&40	\\
Estrada da Quinta das Beatas	&6	&2	&0	&1	&2	&11	\\
Quinta das Beatas	&5	&0	&0	&0	&2	&7	\\
Matatu Pequeno	&13	&24	&2	&3	&7	&49	\\
Matatu Grande	&25	&8	&1	&5	&10	&49	\\
Estrada para a Casa da Pólvora	&1	&3	&0	&2	&2	&8	\\
Estrada do Engenho Velho	&11	&2	&1	&0	&3	&17	\\
Ladeira do Acupe	&12	&47	&2	&3	&0	&64	\\
Rua do Sangradouro	&1	&38	&2	&1	&1	&43	\\
Largo do Acupe	&0	&1	&0	&1	&1	&3	\\
Estrada do Acupe para a 2 de Julho	&4	&3	&0	&1	&0	&8	\\
Estrada de Brotas	&7	&7	&2	&2	&2	&20	\\
Estrada da Cruz das Almas	&8	&6	&3	&1	&4	&22	\\
Largo de Brotas	&9	&21	&3	&7	&5	&45	\\
Estrada para a Cruz da Redenção	&2	&8	&1	&1	&1	&13	\\
Largo da Cruz da Redenção	&0	&1	&4	&0	&2	&7	\\
Campinas	&1	&1	&0	&0	&0	&2	\\
Candeal	&3	&0	&0	&0	&0	&3	\\
Mariquita	&26	&35	&2	&31	&9	&103	\\
Estrada do Sangradouro para o Matatu	&3	&8	&0	&1	&6	&18	\\
Estrada da Ubarana	&1	&0	&0	&0	&1	&2	\\
Pomar	&4	&0	&0	&0	&2	&6	\\
Pituba	&2	&0	&0	&1	&0	&3	\\
Lagoa	&1	&0	&0	&0	&0	&1	\\
Estrada para Armação	&2	&3	&1	&1	&2	&9	\\
Várzea de Santo Antônio	&0	&0	&0	&1	&0	&1	\\
Armação	&1	&0	&1	&1	&0	&3	\\
\midrule
TOTAL	&274	&544	&61	&107	&87	&1073	\\
\bottomrule
\end{tabular}
\end{tiny}}
{\fonte{\textbf{BR BAAHMS}, Livro de Décimas Urbanas de 1893.}}
\end{table}

\begin{itemize}
\item 50,7\% dos imóveis em Brotas eram ocupados por ``inquilinos'' no início da Primeira República, contra 24,54\% ocupados por ``proprietários'' e 9,97\% que encontravam-se fechados ou vazios\footnote{A Lei Municipal nº 27, de 5 ago. 1893, dizia em seu art. 15 que a décima urbana deveria ser cobrada do ``proprietário'', e que caso o prédio fosse locado, seria cobrada sobre o valor do aluguel; além disso, o art. 11 da mesma Lei Municipal dizia que a décima urbana seria lançada por cinco lançadores e cinco auxiliares, distribuídos por cinco distritos (1º distrito: Sé e Conceição; 2º distrito: São Pedro e Vitória; 3º distrito: Paço e Sant'anna; 4º distrito: Pilar, Mares e Penha;  5º distrito: Santo Antônio e Brotas), enquanto o art. 12 dizia que o lançamento seria feito entre julho e outubro. Apesar deste regulamento por sinal bastante minucioso, as aspas são necessárias por não haver qualquer indicação de que os ``inquilinos'' ocupassem os imóveis respaldados exclusivamente por contratos de aluguel. Dado o regime de terras então existente, é provável que a Intendência entendesse como ``inquilino'' não apenas os locatários, mas também os \textit{arrendatários}, os \textit{foreiros}, os \textit{moradores}, os \textit{permissionários} e quaisquer outras formas de ocupação onde fosse resguardada a figura do \textit{possuidor indireto} do imóvel contra o \textit{possuidor direto} que o ocupava. O mesmo vale para a figura do ``proprietário'': o confuso regime de terras em vigor em 1893 não facilitava para os burocratas de médio e baixo escalão responsáveis pelo lançamento tributário e pela coleta dos impostos a distinção entre proprietários plenos, posseiros e sesmeiros irregulares, sendo fácil deduzir daí a hipótese de que a cobrança da décima urbana incidia sobre todos quantos reunissem em si as qualidades de posseiros diretos e indiretos. Para todos os efeitos, portanto, \textit{inquilino} foi entendido como quem detivesse apenas a posse direta, contra a posse indireta de um locador, arrendante etc., enquanto \textit{proprietário} foi entendido como quem detivesse a posse direta e a indireta, e ademais residisse no imóvel em questão.}.
\item As maiores proporções de casas ocupadas por seus ``proprietários'' foram encontradas nas ruas ou localidades marcadamente rurais (Candeal, Alagoa, Pomar, Pituba, Campinas etc.), ou naquelas onde o valor locativo era mais baixo (Quinta das Beatas, Matatu); inversamente, as ruas ou localidades com mais altas taxas de ocupação por ``inquilinos'' (Sangradouro, Castro Neves, Uruguaiana, Galés, Ladeira do Fabrício etc.) foram aquelas que em 1924 apresentaram mais alto valor locativo médio.
\item O maior número de imóveis encontra-se nas ruas e localidades onde era mais intenso o processo de valorização da terra (Mariquita, 1º de Março/Pitangueiras, Castro Neves, Sangradouro, Socorro, Galés, Uruguaiana, Alegria etc.), enquanto as localidades com menor número de imóveis eram precisamente aquelas onde tal processo ou era muito lento, ou não ocorria (Lagoa, Várzea de Santo Antônio, Ubarana, Campinas, Armação, Pituba, Candeal, Pomar etc.); as exceções resultam ou da pequena extensão de logradouros inseridos em contextos de valorização acelerada da terra (Travessa do Socorro, Travessa do Castro Neves, Travessa da Alegria para o Sangradouro), ou de um processo pretérito de fragmentação imobiliária em vizinhanças onde o valor da terra era baixo (Matatu, Quinta das Beatas).
\item O enorme número de imóveis vazios/fechados e de ``inquilinos'' na Mariquita explica-se pelo fato de tratar-se em 1893, ainda, de um arrabalde veranista, de uma estância balneária, de uma localidade cuja ocupação, descontados os pescadores e pequenos agricultores lá residentes, era eminentemente \textit{sazonal}, de \textit{temporada}, resultando em baixa ocupação permanente -- não sem riscos, pois a Intendência punha-se a derrubar as casas vazias da Mariquita deixadas arruinar pelos donos\footnote{\textbf{A Manhã}, ano I, nº 199, 03 dez. 1920, p. 2}.
\end{itemize}

A predominância da ``locação'' como forma de acesso à terra e à moradia expressa ainda a proliferação da especulação imobiliária sobre as terras do distrito. Durante a pesquisa das licenças para construção foi possível encontrar diversos modelos de ``casas de aluguel'', ou seja, corredores de pequenas casas geminadas, de poucos aposentos e aparência espartana, construídas como que num só molde com a única finalidade da locação, arrendamento, cessão etc. Às vezes tais casas sequer precisavam ser geminadas; uma vista rápida por qualquer dos \textbf{Livros de Décimas Urbanas} é suficiente para perceber como em certas ruas um só ``proprietário'' detém três, quatro, cinco, dez, vinte casas em sequência no mesmo logradouro

INSERIR PLANTAS DE CASAS DE ALUGUEL GEMINADAS

INSERIR PLANTAS DE CASAS DE ALUGUEL SIMPLES

A pesquisa mostrou também a construção em Brotas de uma \textit{vila operária} por parte da Companhia União Fabril (cf. ), mas de modo algum este modelo de habitação para proletários foi tão difundida no distrito quanto as ``casas de alugar''.

INSERIR PLANTA DA VILA OPERÁRIA DA UNIÃO FABRIL

\subsection{Loteamentos e parcelamentos: por que tão poucos?}\label{subsec:loteamentos}

Verifica-se via de regra que, em seguida à derrubada das salvaguardas à integridade de um patrimônio imobiliário, impõe-se com o passar dos anos uma tendência à desagregação e fragmentação deste patrimônio, seja pela via das \textit{partilhas hereditárias}, seja pela via dos \textit{loteamentos e parcelamentos}, ou mesmo por força de \textit{litígios judiciais} intermináveis \cite{costaporto_sesmaria_1980,sodero_diragrario_1990}. As partilhas hereditárias e os litígios judiciais interessam mais a uma historiografia do judiciário e da administração da justiça que a uma pesquisa sobre história urbana como a que aqui se expõe; são os \textit{loteamentos} e os \textit{parcelamentos} o assunto a ser tratado daqui por diante como tentativa de compreender a fragmentação imobiliária verificada no distrito de Brotas do final do século XIX à década de 1930.

Infelizmente, pouco foi possível de se encontrar sobre o assunto nos arquivos consultados. Tem-se notícia dos seguintes loteamentos oficiais:

\begin{itemize}
\item O famoso plano da \textit{Cidade Luz}, concebido em 1919 por Theodoro Sampaio, que traçou as linhas gerais do que hoje conhecemos como o bairro da Pituba.
\item O loteamento da \textit{Cidade Balneária Amaralina}, cuja planta original não foi possível encontrar mas que se sabe, por meio das licenças de construção, ampliação e obras consultadas, já ser existente em 1893.
\item O loteamento da fazenda \textit{Santa Cruz}, datado de XXXX.
\end{itemize} 

Tais loteamentos, entretanto, não conseguem explicar a a fragmentação imobiliária encontrada a uma simples comparação entre a quantidade de imóveis na freguesia em 1886 (\autoref{tab:decurb1886-1891}, p. \pageref{tab:decurb1886-1891}) e aquela encontrada em 1924 (\autoref{tab:imoveis1924brotas1} e \autoref{tab:imoveis1924brotas2}, p. \pageref{tab:imoveis1924brotas2}). Como explicá-la, portanto?

Não há explicação única, mas vários elementos dispersos parecem confirmar hipóteses complementares: a \textit{pressão demográfica e imobiliária causada pela abolição da escravidão}; a continuidade dos \textit{arrendamentos}, \textit{aforamentos} e \textit{alugueis} como forma preferencial de acesso à terra e à habitação pelos mais remediados; a opção pelos \textit{loteamentos informais} como confluência entre a sonegação fiscal pelos terratenentes e a satisfação da necessidade de acesso barato à posse da terra, ainda que insegura, pelos mais pobres; a persistência dos processos de \textit{ocupação informal da terra}, estes quase impossíveis de rastrear exatamente por não serem documentados.

Já se discutiu anteriormente () como a abolição da escravidão gerou uma onda migratória do Recôncavo em direção a Salvador; a pressão demográfica daí resultante exerceu força também sobre a malha urbana, demandando sua expansão. A massa de libertos recém-chegados nem cabia no espaço urbano preexistente, nem se enquadrava nas regras de sociabilidade cada vez mais rígidas -- e desabridamente racistas -- estabelecidas tanto pelos sucessivos blocos de poder a exercer a hegemonia política, quanto pelo horror da burguesia e dos gestores em todos os escalões, tomando a aparência pela essência, a tudo quanto remetesse à antiga associação entre as africanidades e o trabalho. Uma das soluções encontradas pelos recém-libertos para fixarem-se em Salvador foi o estabelecimento em ``zonas próprias'', em especial nos distritos com ``maiores áreas verdes, aptos, portanto, a ser explorados mais livremente'' \cite{santos_habitacao_1990}. 

Tal hipótese é reforçada pelo fato de a própria municipalidade soteropolitana entender a necessidade de dar ordem e sentido ao processo de desagregação e fragmentação das grandes herdades circundantes de Salvador, em especial quando eram os recém-libertos e a massa proletária a se beneficiar disto tudo. Em 8 de abril de 1905 promulgou a \textit{Resolução 160}, que entre outras providências já previstas pela ``lei das plantas''\footnote{\textit{Resolução Municipal nº 28}, de 12 de agosto de 1893, que impôs a apresentação ao município de plantas das obras de construção, ampliação ou reforma como condição para sua autorização; cf. a \autoref{subsec:constrampliref} (p. \pageref{subsec:constrampliref}) para o texto integral.} e pelo regulamento da décima urbana\footnote{\textit{Lei Municipal nº 27}, de 5 ago. 1893; cf. a \autoref{subsec:locatermor} (p. \pageref{subsec:locatermor}) para uma discussão mais pormenorizada de seu conteúdo. Esta lei poderia ser renovada quadrienalmente; daí que o texto da Resolução 160 fale num regulamento de 5 de agosto de 1903 para as décimas, autorizando sua revisão (art. 17).}, indicou o seguinte:

\begin{citacao}
\textbf{Art. 1º.} Fica o intendente autorisado, desde já, a mandar levantar as respectivas plantas das seguintes zonas urbanas: estradas 2 de Julho, Federação, Areia, Retiro, Cruz das Almas, Fazenda Garcia, Quinta da Barra, Quinta das Beatas, Cidade Nova, Pau Miudo, Resgate, S. Lazaro, Ondina, Amaralina, Ubaranas, Pituba e de todos os demais logares do perimetro urbano, onde se pretenda edificar, plantas que serão sujeitas á apreciação e approvação do Conselho Municipal. 

\textbf{Art. 2º.} Estas plantas constarão do actual traçado, bem como de todas as modificações de alinhamento que forem julgadas necessarias pela Directoria de Obras e viação no intuito de, fazendo desaparecer as curvas e sinuosidades, preparar-se as futuras avenidas de que tanto carece esta cidade, como garantia ao seu saneamento e embelezamento.

\textbf{Art. 3º.} Nenhuma construcção será feita nestas zonas ou em outra qualquer, sem que primeiramente seja levantada a necessaria planta e esta approvada.

\textbf{Art. 4º.} A Intendencia mandará levantar em plantas diversos typos de construcção, inclusive os das classes pobre e operaria destinados às differentes zonas, typos que não poderão ser modificados em sua essencia, salvo quando a construcção obedecer a um estylo especial perfeitamente conhecido.

\textbf{Art. 5º.} A Intendencia fica autorisada a construir, em particular, nos terrenos de sua propriedade, pequenas habitações hygienicas destinadas às classes pobre e operaria, podendo fazer cessão, desde quando seja indemnisada apenas, do valor do seu custo.
\end{citacao}

Os demais catorze artigos tratam de estímulos à construção em terrenos baldios, à reforma de imóveis arruinados, tudo alimentado por isenções decenais, quindecenais ou vintenais das décimas urbanas. Trata-se de um verdadeiro \textit{código de zoneamento urbano e de ocupação e uso do solo} o que se impunha já em 1905 com este regulamento; era ademais uma legislação que pretendia disciplinar a construção de novas vias as mais retas possíveis, prevendo seu uso futuro como avenidas. Como se vê, a fragmentação das herdades e a pressão demográfica pautaram como que um ``protoplanejamento'' da expansão da malha urbana --  ainda que, verificando-se as licenças de obra, em diversos momentos reclamem os engenheiros da Diretoria Municipal de Obras da falta, da incompletude ou do sumiço de algumas destas plantas\footnote{Até onde foi possível pesquisar no \textbf{BR BAAHMS}, nenhuma destas plantas sobreviveu aos nossos dias.}.

A continuidade dos arrendamentos, aforamentos e locações já foi vista anteriormente (cf. \autoref{subsec:locatermor}, p. \pageref{subsec:locatermor}). Basta dizer, complementarmente, que os arrendamentos, aforamentos e locações tornaram-se com o tempo um fator de pressão pelo fracionamento das antigas herdades, na medida em que o valor da terra em determinadas vizinhanças do distrito aumentou por força da progressiva implementação de infraestruturas urbanas (eletricidade, bonde, telefonia, iluminação pública etc.) e o estabelecimento de residência em vizinhanças de Brotas como Castro Neves, Sangradouro e outras inseridas neste processo de valorização, conquanto não competisse com o \textit{status} conferido pelas moradias no corredor da Vitória, no distrito de São Pedro ou noutros lugares de altíssima valorização no período estudado, conferia a seus moradores ainda alguma distinção, alguma posição, algum \textit{status} que os diferenciasse da massa proletária famélica e doente. O salto no regime fundiário estabelecido pelo Código Civil de 1916 certamente terá influido na passagem de muitos destes arrendamentos, aforamentos e locações para um regime de propriedade plena, mas sem uma pesquisa cartorária rigorosa e minuciosa será impossível passar do campo das hipóteses para o das comprovações.

Ocorre que mesmo o aluguel, aforamento, arrendamento etc., num contexto de extremo pauperismo da maioria da população soteropolitana, pode ser proibitivo. Daí a proliferação em Brotas dos \textit{loteamentos informais} durante o período estudado. Especialmente no que diz respeito às vendas, num contexto onde a propriedade imobiliária implica numa série de obrigações tributárias e num processo extremamente burocratizado de transmissão, e onde parcela significativa da população não dispõe dos recursos necessários para acessar o mercado formal de terras, ela pressiona pela criação um mercado informal de terras, ao arrepio de qualquer mecanismo formal -- e de qualquer segurança para suas posses. 

Em terceiro lugar, a persistência dos processos de \textit{ocupação informal da terra} permaneceu como hipótese de trabalho, embora a documentação pesquisada não tenha fornecido senão indícios e pistas de sua existência. A melhor forma de abordagem direta da questão seria uma pesquisa e sistematização dos autos de infração da Diretoria Municipal de Obras, da Inspetoria Municipal de Higiene e da Diretoria Estadual de Higiene, mas os prazos da pesquisa tornaram proibitivo o emprego deste método; mais uma vez, fez-se necessário encontrar abordagens indiretas, fragmentárias, indiciárias. A fonte, desta vez, foram as \textit{imagens} do distrito, em especial cartões-postais.

Uma última questão antes de prosseguir: daqui por diante, a análise logradouro a logradouro se mostrará contraproducente. Os processos de loteamento e parcelamento, assim como adiante os processos de construção deles derivado, só se podem compreender a contento remetendo-se às herdades de origem, e não à pulverização fundiária subsequente. Por isto, tendo como base as divisões do esboço de caracterização fundiária (\autoref{sec:2.3}, p. \pageref{sec:2.3}), os logradouros serão agrupados em \textit{áreas}, organizadas da seguinte forma\footnote{Os nomes de ruas são aqueles encontrados nos processos examinados no \textbf{BR BAHMS} (Fundo "Intendência / Prefeitura", Série "Processos de licenciamento de reforma e ampliação de edificações", Subsérie "Requerimentos e plantas -- Brotas", Caixas 1 a 24). A classificação segue ao máximo as indicações topográficas de \citeonline{weyll_mappa_1851} e da Prefeitura de Salvador \cite{municipal_atlas_1955}, além de seguir a evolução toponímica indicada por \citeonline{moraes_ruas_1959}, por \citeonline{souza_guia_1935}, pela Prefeitura de Salvador \citeyear{municipal_atlas_1955} e por \citeonline{dorea_ruas_1999}.}

\begin{itemize}
\item \textbf{Antigo 1º Distrito:} Ladeira da Fonte Nova; Ladeira dos Galés; Largo do Paranhos / Praça Manuel Querino; Largo das Sete Portas; Rua do General / Afonso Taunay; Rua das Pitangueiras / 1º de Março / 25 de Março / Agripino Dórea;  Rua do Socorro / Arlindo Fragoso; Rua do Castro Neves; Rua Santo Agostinho / Rua do Bigode; Rua da Alegria do Castro Neves; Ladeira do Fabrício / Rua dos Tupys / Ladeira dos Bandeirantes.
\item \textbf{Boa Vista / Engenho Velho:} Rua Uruguaiana / Frederico Costa; Ladeira do Pepino / Rua José Visco; Ladeira / Fazenda / Roça do Teixeira\footnote{São três nomes para referir-se à \textit{fazenda Engenho Velho}.}; Rua Monte Belém; Capela do Deus Menino; Rua do Saveiro / Rua da União; Boa Vista / roça de José Visco; Vila América; Dique Pequeno.
\item \textbf{Estrada de Brotas:} Avenida Saraiva; Avenida Liberdade; Rua do Cemitério; Ladeira da Pedra; Rua/Estrada da Cruz das Almas; Ladeira/Largo da Cruz da Redenção; Estrada das Armações / do Beiju; Estrada/Largo de Brotas; Avenida D. Pedro II.
\item \textbf{Estrada Dois de Julho:} Estrada Dois de Julho; Mata Escura do Rio Vermelho.
\item \textbf{Mariquita:} Rua Banco de Areia; Rua do Meio; Rua Lucaia; Rua das Pedrinhas; Rua Fonte do Boi; Rua da Mariquita / Direita da Mariquita; Rua dos Dendezeiros; Rua do Papagaio\footnote{Alguns processos de licença de obra localizam esta rua no distrito de Brotas, mas ela não consta nos livros de décimas urbanas; tudo indica que faz parte do distrito da Vitória}.
\item \textbf{Matatu:} Fazenda/Alto do Saldanha; 
\item \textbf{Acupe:}
\item \textbf{Alagoa-Pituba:}
\item \textbf{Armações / Várzea de Santo Antônio:}
\end{itemize}

Esta classificação, como todas, é questionável. Muitas ruas eram classificadas pela população soteropolitana e pela própria Intendência/Prefeitura ora num distrito, ora noutro\footnote{As ruas do Papagaio e do Hipódromo (atual Conselheiro Pedro Luiz) ficam no ``além Lucaia'', no território demarcado como parte do distrito da Vitória, assim como a ladeira de São Gonçalo. Mesmo cientes do equívoco, ao receber os pedidos de licença para obras os burocratas da Diretoria Municipal de Obras -- a Portaria da Câmara Municipal, composta sempre por pessoas de poucas letras a julgar pela redação irregular de muitos dos pedidos por eles recebidos e encaminhados, não tinha obrigação alguma de saber as ruas integrantes de tais ou quais distritos -- davam curso aos pedidos sem quaisquer correções, alterações ou encaminhamentos para outros responsáveis, e a Inspetoria Municipal de Higiene -- esta sim, estruturada em torno de inspetores designados por distrito -- fazia o mesmo.}; dada a falta de uma malha urbana consolidade, com os múltiplos pontos de referência que dela decorrem, muitos imóveis eram indicados como perto de um ponto que, na verdade, ficava muito mais distante do que o que se dizia\footnote{Imóveis indicados como parte da Vila América, por exemplo, poderiam ser tanto aqueles rentes ao nível do rio Lucaia quanto aqueles dependurados nas ribanceiras do Engenho Velho. A Capela do Deus Menino era citada como ponto de referência tanto para imóveis ao nível do Dique do Tororó, margeando a Estrada Dois de Julho (atual avenida Vasco da Gama), quanto para imóveis nas faldas próximas a este templo. A própria Estrada Dois de Julho tinha -- e até hoje tem -- 4,6km, e um pedestre desavisado levaria cerca de uma hora de caminhada firme e contínua para sair de seu início ao pé da ladeira dos Galés até seu término rente ao cruzamento com a Estrada da Cruz das Almas (atual Waldemar Falcão). A Estrada de Brotas (atual avenida D. João VI) era dada como ponto de referência tanto para imóveis lindeiros com a Estrada do Beiju (atual rua Teixeira Barros) quanto para outros fronteiriços com a rua Uruguaiana, dois extremos da via entre os quais medeiam 2.5km. A Estrada da Cruz das Almas (atual rua Waldemar Falcão) era erma, ruralizada ao extremo, e de seu início no cruzamento com a Estrada de Brotas até seu fim nas margens do rio Lucaia, perto da Mariquita, leva-se cerca de meia hora a pé para atravessar seus 2,2km. Todas estas estimativas de tempo de caminhada têm como base as atuais vias, bem calçadas com cimento, pedras, concreto e asfalto, e não as sendas ermas, irregulares, poentas e lamacentas que eram tais caminhos durante toda a Primeira República.}; 

\subsubsection{Loteamentos e parcelamentos formais}

INSERIR PLANTA DO LOTEAMENTO DA FAZENDA SANTA CRUZ

No que diz respeito ao loteamento de \textit{Amaralina}, a vasta maioria da bibliografia consultada, com base num levantamento feito pela Prefeitura \cite{salvador_loteamentos_1977}, indica 1933 como sendo o ano de sua abertura ao mercado imobiliário. 

INSERIR PLANTAS DE OBRAS DO LOTEAMENTO DA FAZENDA AMARALINA (1933)

De certa forma, estão corretos: está custodiada no Arquivo Histórico Municipal, ainda sem catalogação, a planta original deste loteamenteo de 1933. Acontece que este pode ter sido não o primeiro, mas um entre muitos possíveis loteamentos sucessivos da velha fazenda Alagoa; o problema quanto a estes outros loteamentos, relativamente ao objeto daquele levantamento dos anos 1970, é que suas \textit{plantas} ou bem não sobreviveram à ação do tempo, ou foram perdidas, ou ainda não foram catalogadas e por isto mesmo não se consegue ter acesso a elas. A única forma de saber da existência de loteamentos mais antigos da fazenda Alagoa é a análise da totalidade dos pedidos de licença de obra relativos à área; por este meio foi possível verificar que o mais antigo entre tais pedidos, datado de 04 nov. 1897, refere-se ao ``lote 43'' da ``Cidade Balneária Amaralina'', e catorze outros pedidos relativos à mesma área, indicando o mesmo loteamento, cobrem o período que vai de 1897 a 1925\footnote{\textbf{BR BAAHMS}, Fundo ``Intendência e Prefeitura'', Série ``Processos de Licenciamento de Reforma e Ampliação de Edificações'', Subsérie ``Requerimentos e Plantas -- Brotas'', caixa 19, documentos diversos. São eles, em ordem crescente de data: ``lote 43'' (04 nov. 1897); ``lote 49'' (12 fev. 1898); ``lotes 11 a 13'' (19 dez. 1901); ``lotes 69 e 70'' (12 set. 1910), empregue para a construção de seis casas; ``lote 81''(15 out 1910); ``lotes 8 e 64'' (28 abr 1910); ``lote 26'' (21 out. 1910); ``lote 143'' (26 dez. 1910); ``lote 142'' (09 mar. 1911); ``lotes 370 e 371'' (16 dez. 1911); ``lote 146'' (14 abr. 1913); ``lote 135'' (18 mar. 1914); ``lote 198'' (25 nov. 1915); ``lote 116'' (28 jul. 1915); ``lote 13'' (15 set. 1915), objeto de nova licença catorze anos depois da primeira; ``lote 28'' (out. 1925), do qual restou apenas a planta muito danificada; ``lote 393'' (02 mar. 1925). Na caixa 24 do mesmo fundo, série e subsérie há um pedido relativo ao ``lote 253'' (21 fev. 1925).}. Deste modo, se o levantamento da Prefeitura está correto em indicar a planta de 1933 como sendo talvez a mais antiga a ter chegado aos dias atuais, pode ser que a restrição do objeto do levantamento tenha impedido pesquisas mais aprofundadas em busca de outros documentos que não plantas de situação, resultando em que estes loteamentos anteriores da fazenda Alagoa permaneceram fora do conhecimento público até o presente momento.

INSERIR PLANTA DE A. SAFFREY

INSERIR PLANTA DE LYDIA DEWALD

INSERIR PLANTA DA VILA MARIA

INSERIR PLANTA DE CHEHADI KRAYCHETE

INSERIR PLANTA DO CHALÉ DE JOÃO DE MATTOS

INSERIR PLANTA DO CHALÉ DE JOÃO DA CUNHA FREIRE

INSERIR PLANTA DSC04835

No que diz respeito à imprensa, data de 1898 aquele que talvez seja o primeiro uso público do nome ``Amaralina'', quando da publicação de um decreto autorizativo da ``livre creação de gado \textit{vaccum}, cavallar,  lanígero e caprino na zona da costa do districto de Brotas comprehendida entre a fazenda Lagoa e o rio das Pedras exclusive as povoações de Amaralina e Mariquita''\footnote{\textbf{Jornal de Notícias}, ano XX, nº 5612, 23 set. 1898, p. 1}.

Ainda em 1912 a Companhia de Melhoramentos, em data muito próxima do anúncio da construção da avenida Oceânica por esta mesma companhia, mandou publicar no \textbf{Diário de Notícias} (12 jul. 1912) anúncio convocando os arrendatários e foreiros de terrenos situados na fazenda Alagoa ``a comparecerem no seu escritório, a fim de exibirem os seus títulos para registro nos livros da Companhia''; a fazenda serviria de garantia para um empréstimo de 6:000\$000 contraído por intermédio de seu diretor Francisco Marques de Góes Calmon para capitalizar a empresa \cite[p.~123]{CUNHA2011}. Mais adiante se verá por que não se pode dizer que esta empreiteira comprou \textit{toda} a fazenda Alagoa, mas \textit{parte} dela.

Em 1915, o major José Custódio da Silva, conhecido pela disputa com o velho José Álvares do Amaral em torno da propriedade da fazenda Alagoa, resolveu entrar no negócio dos loteamentos.  Deu entrada num pedido de licença para o loteamento chamado \textit{Cidade Balnear Hormendina} em terrenos de sua propriedade na fazenda Ubarana, resultando em curiosa negociação. Para facilitar a aprovação de seu loteamento, José Custódio fereceu gratuitamente ``um lote dessas terras ao Município para um predio escolar, e, ouvida a seção de higiene municipal no processo de licenciamento, esta exigiu seis lotes''; o loteador respondeu com uma oferta de três lotes de 24mX30m, ``desde que sejam concedidos ao suplicante os mesmos favores feitos ao sr. Dr. Bernardo Jambeiro, na concessão que obteve para as suas terras nas Quintas da Barra''; depois disto, não houve qualquer outro despacho no processo de licença\footnote{\textbf{BR BAAHMS}, Fundo ``Intendência e Prefeitura'', Série ``Processos de Licenciamento de Reforma e Ampliação de Edificações'', Subsérie ``Requerimentos e Plantas -- Brotas'', caixa 19.} -- mas a imprensa baiana, sob o título ``O concelho acha que nós não precisamos mais de melhoramentos'', comentou o assunto, dizendo que José Custódio ``não deve estar muito satisfeito com os edis'' porque ``o concelho rejeitou o projeto''\footnote{\textbf{A Notícia}, ano I, nº 145, 15 mar. 1915, p. 145.}.


Em 1927 noticiava-se a doação pela Intendência de 10 mil paralelepípedos aos construtores Nivaldi, Allioni \& Cia., ``encarregados da construcção da futura cidade balneária em Amaralina''\footnote{\textbf{A Capital}, ano I, nº 109, 11 fev. 1927, p. 2.}. Desta vez pode-se afirmar com alguma certeza: esta ``futura cidade balneária'' é o loteamento de 1933.

Maria Emília Paraíso do Amaral, filha de José Álvares do Amaral e tão bem colocada na sociedade baiana quanto o resto de sua família\footnote{Um exemplo de sua rede de relacionamentos: quando de seu casamento com Julio Muniz Barretto em 6 de setembro de 1913, serviram como testemunhas, no religioso e no civil, figuras como Pedro Velloso Gordilho, Julieta Maia de Góes Calmon, Francisco Marques de Góes Calmon, Horácio Lucatelli Dórea e Francisco Eloy Paraíso Jorge (\textbf{Gazeta de Notícias}, ano III, nº 297, 6 set. 1913, p. 2).}, 

INSERIR PLANTA DA CIDADE LUZ

TRATAR DA COMPRA DA FAZENDA ALAGOA PELA COMPANHIA DE MELHORAMENTOS





INSERIR PLANTA DA CIDADE BALNEÁRIA HORMENDINA




\subsubsection{Loteamentos e parcelamentos informais}



Um exemplo clássico de loteador informal é \textit{José Visco}, que durante algum tempo emprestou seu nome à ladeira do Funil, no Engenho Velho de Brotas. 

Tudo começa com a autorização dada em 1912 pelo conselho municipal a José Visco para construir uma ligação entre a rua Uruguaiana e a Estrada 2 de Julho através de terrenos de sua propriedade\footnote{\textbf{Gazeta de Notícias}, ano III, nº 42, 25 out. 1912, p. 3.}.

Apenas para que fique demonstrada a importância do sr. José Visco na produção do espaço urbano de Brotas, basta dizer que a antiga Ladeira do Pepino, que serviu para ligar a rua Uruguaiana com a Estrada 2 de Julho, durante muitas décadas recebeu seu nome -- e mesmo uma pesquisa em ferramentas de GPS atuais com a expressão ``José Visco'' ainda aponta para esta rua.

OCUPAÇÃO INFORMAL



INSERIR CARTÕES POSTAIS DO DIQUE DO TORORÓ

INSERIR CARTÕES POSTAIS E FOTOS DO RIO VERMELHO

Por outro lado, há também indícios a serem verificados com atenção nos registros oficiais. Note-se no Livro de Décimas Urbanas de 1893, por exemplo, a presença de \textit{africanos} lá nas mais extremas lonjuras, como um certo ``Matheus (africano)'', de um certo ``Thomas (africano)'' e de um certo ``Ignacio (africano)'' residentes na localidade conhecida como ``Pomar''.

\subsection{Construções, ampliações, reformas: urbanização desigual e a passos lentos}\label{subsec:constrampliref}

O governo da Bahia, expondo resultados da estatística predial de 1892, dizia haver em Salvador 12.649 prédios, dos quais 307 encontravam-se em ruínas ou em construção, distribuídos estes últimos da seguinte forma: ``nas freguesias da Sé 1, Conceição da Praia 13, Victoria 60, Sant'Anna 56, Pilar 24, Mares 30, Rua do Passo 9, Brotas 31, S. Pedro 8, Penha 18, Santo Antonio 57'' \cite[relatório da inspetoria de higiene, p.~6]{bahia_rpe_1893}; este é o ponto de partida para o processo de construções, ampliações e reformas de imóveis em Brotas, mas antes de prosseguir, algumas palavras sobre a legislação aplicável e sobre certa tendência a considerar o desenvolvimento urbano de Salvador (e de Brotas) durante a Primeira República como ``desordenado''.

A \textit{Resolução Municipal nº 28}, de 12 de agosto de 1893, estabeleceu pela primeira vez na Salvador republicana a obrigatoriedade de planta para a concessão de licença, e é de tão grande importância seu curto texto que merece citação integral:

\begin{citacao}
\textbf{Art. 1º.} Fica a secção de Engenharia do Municipio obrigada a fornecer, mediante despacho do Intendente, planta a quem pretender edificar ou reedificar predios nesta cidade.

\textbf{Art. 2º.} Nesta planta procurará a referida secção conciliar os interesses e planos do particular, quanto possível, com as regras da hygiene e esthetica de accordo com a largura e amplitude das ruas e sua posição topographica.

\textbf{Art. 3º.} O proprietário pagará pela planta no acto da concessão da licença de 10\$000 a 200\$000 a titulo de emolumentos.

\textbf{Art. 4º.} O superintendente geral das obras municipaes organisará uma tabella para cobrança dentro daquelles limites de accordo com a extensão e valor do prédio a edificar.

\textbf{Art. 5º.} Revogam-se as disposições em contrario.
\end{citacao}

Com esta exigência, todo projeto de construção ou reforma a ser realizado em Salvador passou não apenas a condicionar-se à ingerência de engenheiros para sua realização, como também, dada a falta de qualquer critério mais objetivo acerca da matéria, a submeter sua própria estética ao gosto dos engenheiros municipais.

Esta é a base dos \textit{pedidos de licença} para construção e reforma que compõem o núcleo desta pesquisa. Seu funcionamento era razoavelmente simples. Qualquer interessado em construir, ampliar, reformar, lotear ou parcelar imóveis precisava dirigir-se à Câmara Municipal com a planta a obra pretendida para que a Portaria da Câmara Municipal reduzisse o pedido a termo, indicando a natureza do pedido (construção, ampliação, reforma, loteamento ou parcelamento) e a localização do imóvel. Instaurava-se então um processo administrativo, encaminhado em primeiro lugar à Diretoria Municipal de Obras e Viação, onde os aspectos contrutivos e arquitetônicos seriam avaliados, em especial o alinhamento da obra com a rua, disciplinado por plantas de área disponíveis aos burocratas. Havendo parecer positivo, o processo era então encaminhado para a Inspetoria Municipal de Higiene, que dava outro parecer relativo às medidas higiênicas e sanitárias básicas necessárias para a construção; com a promulgação do Código Sanitário da Bahia em 1924, esta atribuição passou à Diretoria Estadual de Higiene (como o fora durante o Império), mas no que diz respeito à tramitação do processo nada se alterou. Estando tudo de acordo com as posturas municipais e com o Regulamento Sanitário Municipal (Lei Municipal 797, de 28 jun. 1906), ou posteriormente com o Código de Posturas Municipais (Ato 127, de 05 nov. 1920) e com o Código de Obras (Lei Municipal 1.146, de 19 jun. 1926), a obra estava liberada.

Um pesquisador pregresso dos mesmos pedidos de licença disse que a falta de um ``código'' de obras ou de posturas ``implicava, de certo modo, no aumento da burocracia'', pois as plantas eram submetidas ``à apreciação de diversas diretorias e ao julgamento, quase que pessoal, de funcionários nem sempre qualificados para tal atribuição, o que, obviamente, resultava em grande perda de tempo para o interessado na construção'' \cite[p.~89]{cardoso_vilas_1991}. A documentação consultada confirma a burocratização das licenças e as atribulações dos requerentes, mas infirma a alegada desqualificação técnica dos burocratas analistas. Mais adiante se demonstrará como \textit{todos} os pareceristas da Diretoria Municipal de Obras eram \textit{engenheiros}, e como \textit{todos} os pareceristas da Inspetoria Municipal e da Diretoria Estadual de Higiene eram \textit{médicos}, \textit{farmacêuticos} ou \textit{cirurgiões}, quando não também \textit{engenheiros}. 

De outro lado, a ausência de um código municipal de posturas até 1920 e de um código municipal de obras até 1926 não significou de modo algum o arbítrio completo dos burocratas: uma análise comparada do código de posturas de 1920 com os três livros de posturas municipais referentes ao período 1889-1930 sob custódia do Arquivo Histórico Municipal demonstra que quase nada se inovou com o código, que via de regra sistematizou numa só legislação posturas municipais antes esparsas, e o código de obras de 1926, se inovou no zoneamento urbano, de resto seguiu todos os preceitos construtivos e arquitetônicos já indicados pelo regulamento sanitário de 1906, pelas posturas municipais que o precederam e pela prática reiterada dos burocratas da Diretoria Municipal de Obras. Por isto mesmo, não está correta a impressão generalizada de um \textit{laissez-faire} urbanístico anterior aos códigos de posturas e de obras: não apenas havia legislação de controle do uso e ocupação do solo precedente aos dois em pelo menos dezessete anos (cf. a Resolução 160/1903 já citada), constantemente aplicada pelos burocratas da Diretoria de Obras, como normas municipais correlatas disciplinavam tanto os aspectos construtivo e arquitetônico (regulamento sanitário municipal de 1907) quanto usavam as isenções da décima urbana como incentivo ao melhor aproveitamento dos imóveis (construção em terrenos baldios, reformas de prédios arruinados, provisão de moradia para ``operários'' e ``pobres'' etc.). Esta impressão é, por assim dizer, uma projeção na historiografia da afirmação corrente, de senso comum, acerca de uma suposta ``falta de planejamento'' na produção do espaço urbano soteropolitano. Ora, assim como todos quantos tenham pesquisado a urbanização soteropolitana na \textit{longue durée} do século XX demonstram tanto uma sucessão de planos e projetos quanto, paralelamente, a ocupação e uso do espaço urbano pautada pela relação dialética entre as necessidades habitacionais dos trabalhadores mais pauperizados e a extração da renda da terra pelos latifundiários urbanos e promotores imobiliários por todos os meios, legais ou não \cite{BRANDAO1978, BRANDAO1978a, BRANDAO1980, CARVALHO1974, GORDILHO-SOUZA2008, SAMPAIO1999, VASCONCELLOS1974, VASCONCELOS2002}, esta mesma relação dialética se verifica na produção do espaço urbano de Salvador na Primeira República -- e Brotas era parte deste processo. 

Os loteamentos e parcelamentos informais eram fenômenos corriqueiros no desenvolvimento urbano de Brotas, às vezes até sacramentados pela lei como no caso de José Visco; a produção informal de moradia, difícil como seja de se deixar capturar pelas lentes da historiografia, encontrava-se igualmente neste processo. Seu fundamento não foi a falta de normas urbanísticas, não foi a ausência de mecanismos legais de controle da ocupação e uso do solo. Foi, isto sim, a combinação entre a pressão demográfica herdada em Salvador do processo de transição da escravidão para o trabalho assalariado; a expulsão das antigas moradias senhoriais e a consequente pressão pela produção de novas moradias e por novos alugueis, aforamentos e arrendamentos\footnote{Uma das características do modo de produção escravista é a obrigação por parte dos senhores de fornecer moradia para os escravos; no caso soteropolitano, a isto somava-se, em seguida à repressão contra os hauçás insurrectos de 1835, a proibição a todos os africanos e crioulos de apropriarem-se legalmente de bens de raiz. Extinta a escravidão, extinguiu-se com ela também a obrigação dos senhores de fornecer abrigo àqueles que antes escravizaram. Estes elementos configuram a hipótese de uma tendência de os recém-libertos pressionarem tanto a produção de novas moradias quanto o mercado de alugueis, arrendamentos e aforamentos na cidade.}; a necessidade habitacional de uma massa proletária extremamente pauperizada; foi tudo isto o que causou o ``crescimento desordenado'' de Salvador, e não a ausência de normas disciplinadoras deste processo.

Isto dito, é preciso registrar ainda alguns poréns e senões metodológicos antes de prosseguir. Um trabalho realmente minucioso em torno dos pedidos de licença exigiria a comparação entre os processos custodiados no \textbf{BR BAAHMS} e a publicação na imprensa baiana da deliberação da Intendência acerca de cada processo (ao menos a \textbf{Gazeta de Notícias} veiculou tais deliberações entre 1912 e 1914, e \textbf{A Notícia} entre 1914 e 1915). Não obstante, interessam a esta pesquisa menos a certeza da aprovação que a \textbf{quantidade de pedidos}, a \textbf{natureza da obra pretendida}, a \textit{planta da obra} e seu \textit{estilo arquitetônico}, a \textit{data} do pedido, o \textit{teor dos pareceres} emitidos pela Diretoria de Obras e pela Inspetoria de Higiene, enfim, interessam a esta pesquisa informações sobre as quais a aprovação ou rejeição do pedido, conquanto importante, tem pouco impacto.

O ritmo do desenvolvimento urbano de cada uma das áreas já delimitadas está descrito, por década, na tabela

CRIAR TABELA COM OBRAS POR RUA, POR DÉCADA (1889-1900, 1900-1910, 1910-1920, 1920-1930)


A partir destes resultados, pode-se chegar às seguintes conclusões provisórias:

\begin{itemize}
\item
\item
\item
\end{itemize}

\subsubsection{Antigo 1º Distrito}

No Sangradouro, o comerciante e ``definidor'' da Ordem Terceira de São Francisco, Joaquim Ignacio Ribeiro dos Santos (69 anos em 1977), que já havia sido dispensado do pagamento da décima urbana sobre os terrenos de sua roça em 1875 \cite[p.~226]{bahia_assembleia_1875}, 

HOSPITAL MILITAR DO CASTRO NEVES

Em 1903 as terras remanescentes da aquisição do solar do Castro Neves para a transferência do Hospital Militar entraram em pauta: o ministro da Fazenda pediu ao ministro da Guerra o levantamento de uma planta de situação ``de todo o terreno do proprio nacional em que funcciona o referido hospital'', discriminando-se nela ``a parte necessaria e a desnecessaria ao estabelecimento'' para que ficasse a última ``à disposição do ministerio da fazenda''\footnote{Correio do Brasil, ano I, nº 31, 29 set. 1903, p. 1.}; era o movimento que restava para que toda a extensão do Castro Neves fosse urbanizada.

\subsubsection{Boa Vista / Engenho Velho}

Na ``pequena e larga'' rua da Boa Vista\footnote{\textbf{Correio do Povo}, ano II, nº 176, p. 1.}

RAYMUNDO DA CUNHA PACHECO

\subsubsection{Estrada de Brotas}


\subsubsection{Estrada Dois de Julho}


\subsubsection{Mariquita}


\subsubsection{Matatu}


\subsubsection{Acupe}


\subsubsection{Alagoa, Amaralina, Santra Cruz, Ubarana, Pituba}


CERVEJARIA ANTARCTICA

De toda sorte, em 1927 era anunciada a venda da massa falida da cervejaria, que incluía imóveis entre a Fonte do Boi e Amaralina\footnote{\textbf{O Combate}, ano I, nº 112, 20 out. 1927, p. 2.}.

\footnote{Nenhuma das fontes consultadas indica claramente onde fica o \textit{Grão Mogol}. Três delas, entretanto, dão a entender, a propósito da instalação de um emissário de esgoto, que trata-se do nome de um morro que forma com o Morro do Conselho as duas ribanceiras de um vale. Dois morros aparecem como possibilidades: um é aquele formado pelos vales onde se situam as atuais ruas Fonte do Boi e Barro Vermelho, em cujo topo por muitas décadas funcionou o Hosital Infantil Alfredo Magalhães, ou Hospital da Criança, ou Hospital Nita Costa (nome de sua fundadora); o outro é o morro cortado pela atual alameda Morro da Margarida.}


\subsection{Os gestores e o espaço público}









No que diz respeito aos \textit{requerentes},

REQUERENTES

BOA VISTA





No que diz respeito aos \textit{profissionais contratados}, observa-se o quadro de profissionais atuantes no distrito tal como exposto na \autoref{tab:profissionaisatuantesbrotas}

INSERIR TABELA COM PROFISSIONAIS E NÚMERO DE OBRAS

 . Entre estes profissionais, destaca-se como exemplo da circulação entre o público e o privado o engenheiro, arquiteto e construtor -- ele mesmo variava sua qualificação profissional de acordo com a função que exercia no momento -- \textit{Pedro Jayme David}. 

No que diz respeito aos \textit{órgãos fiscalizadores} do uso e ocupação do solo

INSERIR TABELA COM OS BUROCRATAS


\subsection{Outros usos do espaço em Brotas}

Se os usos residenciais, industriais e comerciais do espaço foram privilegiados até o momento, é porque a questão fundiária tem sido tratada ao longo de toda a pesquisa como central para a produção, apropriação e uso do espaço urbano, e estes três usos, por seu caráter estável, tendendo à permanência, ligam-se diretamente às questões de posse e propriedade da terra. Por outro lado, há outros usos do espaço de Brotas que, tangenciando ou não a questão fundiária, merecem destaque.

\subsubsection{Eventos cívico-políticos}

Prosseguiam as comemorações do Dois de Julho no distrito tão animadas -- ou mesmo mais, quem sabe! -- quanto nos tempos imperiais:

\begin{citacao}
\textbf{2 de Julho do Castro Neves}

Da comissão organisadora desses festejos, recebemos um delicado convite a que agradecemos, para assistirmos ao desfilar do prestito que se realizará no proximo domingo, 6 do corrente, a 1 hora da tarde, sahindo do Largo da Boa Vista e seguindo pelas ruas 1º de Março, Pitangueiras, Largo do Paranhos, Matatú Pequeno, Fabricio, Sangradouro, Sete Portas, regressando pela Ladeira do Santo Agostinho, rua da Alegria, Socorro, Ladeira dos Galés, Fonte Nova, donde se dirigirá para o Castro Neves, que estará festivamente embandeirado.

Ao passar pelo Matatu pequeno falará o acadêmico Lemos Britto.

Os festejos prolongar-se-ão até o dia 8.

O programma é o seguinte!

Abrira a comitiva civica um grupo numeroso de cavalheiros, trajados symbolicamente;

seguir-se-á uma banda de clarins, fanfarreando em toques estridentes, e annunciando à multidão anciosa a aproximação do imponente e magestoso prestito;

segue-se um piquete de cavallaria;

virá depois um pelotão de cornetas e caixas de guerra, surgindo então, da immensa massa do povo, o Carro Emblematico;

musica de S. Vicente de Paulo;

batalhão de graciosas meninas symbolisando as \og Heroinas Brasileiras \fg{};

banda do 2º corpo do Regimento Policial;

batalhão de meninos \og Defensores do Castro Neves \fg{};

musica dos Salesianos, seguindo-se o collegio encorporado da referida associação beneficente;

banda do 1º corpo policial;

banda do 5º batalhão de artilharia, etc., etc.\footnote{\textbf{Correio do Brazil}, ano III, nº 567, 4 ago. 1905, p. 4}
\end{citacao}

Os festejos bijulinos eram ainda realizados no Castro Neves em 1914\footnote{\textbf{Gazeta de Notícias}, ano IV, nº 230, 01 jul. 1914, p. 1.}, e em 1915 ocorriam também na vizinhança da Capela do Deus Menino\footnote{\textbf{A Notícia}, ano I, nº 260, 05 ago. 1915, p. 1.}.

As celebrações particulares também eram dignas de nota, em especial quando ligadas a políticos. Frederico Costa, presidente do Senado estadual, enraizava seu poder político nas terras de sua propriedade ao Matatu, onde tinha uma fazenda produtora de laranjas e vários imóveis urbanos; num seu aniversário, em 1927, um jornal notoriamente seabrista esculhambou-o de cima a baixo, chamando-o de ``homúnculo [\dots] [\textit{que}] de discursos [\dots] pouco entende'', ``rubicundo politiqueiro [\dots] cuja obtusidade é compacta, é integral, é profunda'' -- e entre um impropério e outro, registrou que naquela data, em seguida a uma missa na Sé, o Matatu amanhecia ``embandeirado em arco'' pela festa de aniversário do político, ``com gyrandolas, comidas e bebidas''\footnote{\textbf{O Combate}, ano I, nº 120, 29 out. 1927, p. 1. O mesmo jornal prosseguiu na catilinária contra Frederico Costa em várias edições posteriores.}.

\subsubsection{Eventos religiosos}

O calendário de festas religiosas soteropolitanas celebradas também no distrito de Brotas incluia o \textit{Natal}, celebrado em 1914 no ``aprazível arrabalde'' da Pituba com ``muitas diversões que se prolongarão até ao romper do dia, com o espoucar de foguetes, tocando um grupo musical, havendo kermesse e outras surpresas''; na Boa Vista estava programada a apresentação de um coral infantil, da banda do Corpo de Bombeiros e uma procissão infantil carregando a imagem do ``menino Deus'' (seria a mesma da Capela do Deus Menino, bastante próxima?); no ``lindo trecho'' da Alagoa seria realizada uma missa\footnote{\textbf{A Notícia}, ano I, nº 83, 24 dez. 1914, p. 7.}. Quanto a esta última, a tradição pode ter sido iniciada em 1898, ano em que ``um grupo de rapazes e moças passeiantes, moradores na Fonte do Boi e nas Pedrinhas'', foi pedir permissão ao ``coronel Juca Amaral'' para celebarar na capela de Nossa Senhora dos Mares da Lagoa uma missa natalina\footnote{\textbf{Jornal de Notícias}, ano , nº , 13 dez. 1898, p. 2.}.

Outra celebração em que Brotas se inseria no calendário soteropolitano era a \textit{queima do Judas}, como se deu no Sangradouro em 1905 \footnote{\textbf{Correio do Brasil}, ano III, nº 486, 24 maio 1905, p. 2.}. Comemorava-se também em Brotas a \textit{folia de Reis}: em 1914, além da saída do terno \textit{Concha de Ouro} da Alegria do Castro Neves\footnote{\textbf{Gazeta de Notícias}, ano IV, nº 94, 03 jan. 1914, p. 1.}, foi anunciado que no Rio Vermelho ``funccionarão, durante toda a noite, o \textit{roskoff} e a \textit{kermesse}, devendo alli comparecer os ranchos do \textit{Saramonete}, que sahirá do Grão-Mogol, e o do \textit{Bebê}, que sahirá do arrabalde de Amaralina''\footnote{\textbf{Gazeta de Notícias}, ano, nº , 5 jan. 1914}; no caso de Amaralina, é muito provável que participasse de algum modo na organização dos festejos o \textit{Centro Recreativo de Amaralina}, do qual era vice-presidente em dezembro de 1913 o jovem Alberto Magalhães\footnote{\textbf{Gazeta de Notícias}, ano IV, nº 71, 04 dez. 1913, p. 2.}. Os ranchos, entretanto, eram vistos na época como ``elemento de desordem'': o \textit{Curuvina}, por exemplo, sediado no Dique Pequeno, foi destacado pela imprensa por causa de uma briga ocorrida numa de suas noites de ensaio, ``onde a \textit{pura} é o estimulante procurado para as libações do \textit{pessoal}''; postas as coisas em tom francamente moralista, perguntava-se o articulista, sarcasticamente, se ``a \textit{Curuvina} ainda virá á tona''\footnote{\textbf{A Notícia}, ano I, nº 42, 06 nov. 1914, p. 5.} -- ou seja, se a continuidade de seus ensaios seria permitida pelas autoridades. Como se vê, os ranchos dos bons moços de Amaralina e do Grão-Mogol eram tratados de forma muito diversa dos ranchos dos proletários, muito provavelmente negros, do Dique Pequeno.

Muito própria do distrito era a \textit{festa de Nossa Senhora da Luz}, assim anunciada em 1905:

\begin{citacao}
No proximo domingo realisar-se-à com toda a pompa a festa de N. S. da Luz, na Pituba, promovida por famílias que se acham veraneando no salubre e pitoresco arrabalde da Pituba.
Ás 5 horas da madrugada serão annunciadas as festas por 21 tiros, às 8 horas da manhã será realizada a missa e às 11 horas terá começo a festa religiosa, sendo pregador o Monsenhor Novaes.
Às 3 horas da tarde sahirá em procissão a imagem de N. Senhora, carregada por distinctas senhoras e tendo a frente a banda de música do Regimento Policial. A noite tocarà em um palanque uma musica, havendo leilão, kermesse e finalisando a festa com um bonito fogo de artifício.
Na segunda-feira haverá corrida de jangadas e outros divertimentos, terminando os festejos na terça-feira por passeiatas populares\footnote{\textbf{Correio do Brasil}, ano III, nº 420, 10 fev. 1905, p. 1.}.
\end{citacao}

\subsubsection{Candomblés}

Fiéis às suas tradições, os praticantes do candomblé resistiram em Brotas durante a Primeira República a toda sorte de estratégias repressivas, e deixaram sua marca no território do distrito. 

Em 1890 um certo ``Zé do Ó'' noticiava, em tom chistoso, que a polícia tinha ido ``fazer estardalhaço no candomblé onde se adorava \textit{Gonocô} e Ossum-Ché'', a quem surpreendentemente defendeu em público por serem ``deuses que estão em pleno goso de seus direitos moraes, civis e politicos depois do decreto da liberdade de cultos''; ``Zé do Ó'' ainda acusou a polícia de ter comido e bebido à vontade no terreiro, além de ter soltado as sacerdotisas e prendido os ``marmanjos''\footnote{\textbf{Tribuna Popular}, ano I, nº 50, 29 jun. 1890, p. 3.}. Ironia? Defesa legítima? Importa a esta pesquisa apenas o fato de uma das divindades cultuadas ser \textit{Babá Gunukô}, indicando a possibilidade de este terreiro atacado ser o mesmo indicado na \autoref{subsubsec:matatu} (p. \pageref{subsubsec:matatu}).

Em setembro de 1914 noticiava-se -- com as habituais solicitações de providências por parte da polícia -- a existência de duas casas do ``maldito e ruidoso candomblé'', ``uma sita à rua Uruguayana e a outra em uma ladeira por detraz do Asylo São João de Deus''\footnote{\textbf{A Notícia}, ano I, nº 4, 23 set. 1914, p. 2.}. Como visto na \autoref{subsec:pontrel} (p. \pageref{subsec:pontrel}), o terreiro \textit{Tumba Junsara}, fundado anos depois, em 1919, foi instalado na Ladeira do Pepino por volta de 1920; é certo que o terreiro da Uruguaiana a que se refere a notícia não é ele, mas é muito provável que a existência de terreiros mais antigos na Boa Vista e no Engenho Velho tenha de algum modo influenciado a escolha de sua nova localização.

Em maio de 1920 chegou ao famigerado Pedro Gordilho, delegado de polícia, a informação de que ``no Matatú Grande'', ou mais especificamente na localidade conhecida ainda hoje como Baixão, ``todas as noites `tocam candomblé' ''. Em diligência ao local em alta madrugada\footnote{A notícia indica que à 1h se cantava, ao som de atabaques e agogôs, uma cantiga assim registrada: ``O Aruchachá / que relampuê / minha Santa Bárbara / que relampoá!''; o mais provável é que se tratasse de um culto a Iansã.}, o destacamento policial comandado pelo próprio Gordilho irrompeu terreiro adentro aos gritos de ``polícia! polícia!'', prendendo vinte pessoas (que se diz terem sido postas em liberdade no dia seguinte, sabe-se lá em que condições) e pondo em fuga outras tantas. O terreiro foi assim desmantelado, todos os objetos sagrados e litúrgicos do templo -- ``capacetes, coroas, settas, pandeiros, `tabaques', `afunchês', `agugôs', `ojás', chapanã, contas, santos horriveis de expressão physiognomica, pedaços de páo talhados, talhados de arremate'' -- foram apreendidos, e tudo foi entregue ao Instituto Geográfico e Histórico da Bahia por sugestão de Pedro Melo, do Gabinete de Identificação\footnote{\textbf{A Manhã}, ano I, nº 36, 20 maio 1920, p. 1.}. Nove dias depois da matéria, considerada ``um sucesso'' pelo jornal, foi impetrado \textit{habeas corpus} em favor do pai-de-santo, ironizado pelo jornal por apelar ao Judiciário ao invés dos santos do candomblé, este ``abuso anti-hygienico''\footnote{A Manhã, ano I, nº 44, 29 maio 1920, p. 3.}, A manchete em letras garrafais da incomum matéria de meia página da capa do jornal não deixa dúvidas quanto à localização do terreiro: ``Viva Ogunjá!'' Não se pode concluir daí outra coisa senão de que se tratava do \textit{Ilê Ogunjá}, fundado em 1906 por \textit{Procópio Xavier de Souza} (1880-1958), sacerdote ketu conhecido também como \textit{Procópio de Ogum}. Muito provavelmente foi esta a prisão, ou uma das prisões, que deu base a uma das mais famosas alegorias da obra literária de Jorge Amado, iniciado ele próprio no candomblé como ogã de Oxossi no Ilê Ogunjá: o afrontamento de Procópio ao delegado ``Pedrito Gordo'', baseado no próprio Pedro Gordilho \cite[p.~236-242]{amado_tenda_2010}. O Ilê Ogunjá deu seu nome iorubá ao vale entre o Acupe e o Matatu, de um lado, e o Engenho Velho de Brotas, do outro -- e até os dias atuais, quando sobre ele corre uma avenida bastante movimentada, não há quem a conheça por seu nome oficial.

Nem a documentação pesquisada, nem a bibliografia consultada apontam quaisquer outras ocorrências, que para emergirem da poeira dos arquivos precisariam de outro tipo de pesquisa, de outras fontes, de outro tempo. Cabe registrar, como nota final sobre o assunto, duas coisas. 

Em primeiro lugar, os terreiros mencionados, junto ao \textit{Alaketo} ainda existente no período no Matatu, são ou foram terreiros grandes, bastante conhecidos; está ainda em curso, em especial por meio da arqueologia, a construção de uma base historiográfica acerca dos terreiros menores, domésticos, mais fáceis de se ocultar num cenário político repressivo mas ao mesmo tempo mais vulneráveis à repressão quando a sofriam, e portanto mais efêmeros \cite{gordenstein_arqueterre_2016}; no caso de Brotas, uma tal pesquisa sequer existe, tanto pela falta de uma base de dados tal como a que vem sendo construída acerca dos candomblés do século XIX \cite{reis_candomble_2001}, quanto pelo fato de que, diferentemente do que se deu no Pelourinho e adjacências, onde tais pesquisas arqueológicas e arquivísticas logram maior sucesso, o desenvolvimento urbano em Brotas resultou em sucessivas demolições e reconstruções, arrancando assim do solo significativa matéria-prima arqueológica. Uma base de dados sobre os candomblés domésticos em Brotas pode ser construída, e certamente a base de dados relativa aos candomblés baianos do século XIX a contempla; o difícil é encontrar os traços e rastros arqueológicos dos terreiros menores. Veja-se, como exemplo destes terreiros menos conhecidos em Brotas, a notícia de que em junho de 1904 uma ``força de Urbanos'' atacou um terreiro na estrada da Cruz das Almas e prendeu oito pessoas\footnote{\textbf{Correio do Brasil}, ano II, nº 231, 07 jun. 1904, p. 2}; se o \textbf{Mapeamento dos Terreiros de Salvador} (http://www.terreiros.ceao.ufba.br) foi muito útil relativamente aos terreiros maiores e mais conhecidos, no que diz respeito a terreiros já desaparecidos como este a ferramenta mostrou suas limitações, pois aquilo que hoje deve constar quiçá apenas da memória genealógica de algumas casas não fica nele registrado, e este terreiro da estrada da Cruz das Almas não aparece em mais referência alguma. 

Em segundo lugar, não se pode deixar de registrar como a aguerrida resistência dos praticantes do candomblé em Brotas produziu um curioso efeito. Os mais conhecidos pontos notáveis de Brotas na atualidade não são, como na Primeira República, as igrejas e as grandes herdades, palacetes e mansões; são as \textit{avenidas de vale}, cuja importância transcende o território do distrito. Ora, duas das mais importantes delas, a Bonocô (Mário Leal Ferreira) e a Ogunjá (General Graça Lessa), devem seu nome mais conhecido não a uma ``origem popular'' de memória perdida, mas sim a dois terreiros de candomblé muito antigos. 

\subsubsection{Festas populares e carnaval}

Em 1919 percorria a Estrada 2 de Julho na segunda e na terça de carnaval o cordão \textit{A Censura}\footnote{\textbf{A Hora}, ano II, nº 44, 01 mar. 1919, p. 3.}.

\subsubsection{Ponto de desova}

Parecia, entretanto, não haver quaisquer planos para a \textit{várzea de Santo Antônio}. Tanto que, erma agora como anos antes, prestava-se a ser o que hoje chamamos de \textit{ponto de desova}: certo Ernesto Wolf, alemão chefe de máquinas no Gasômetro e residente à Calçada do Bonfim, desapareceu numa roça do distrito de Brotas e teve seu cadáver encontrado na várzea de Santo Antônio no dia seguinte\footnote{\textbf{Gazeta de Notícias}, ano III, nº 20, 30 set. 1912, p. 3}.

