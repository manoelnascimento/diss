\section{Ação privada}\label{sec:acaoprivada}

Paralelamente aos investimentos públicos, é necessário levar em conta a ação de \textit{agentes privados} na produção, apropriação e uso do espaço do distrito. A natureza de tal ação apresentou-se em campo distinto daquele analisado até o momento: enquanto os investimentos públicos estiveram voltados principalmente para a construção de equipamentos coletivos, infraestruturas de transporte, circulação e comunicação etc., os agentes privados apresentaram como prioridade neste distrito, em ordem decrescente de ocorrências, as construções de novos imóveis, as ampliações e reformas de imóveis existentes e os loteamentos. Cada qual será examinado em detalhe, não sem antes se estabelecerem alguns pontos acerca da natureza do uso e do valor da terra no distrito.

\subsection{Que apropriação da terra em Brotas?}\label{subsec:apropribrotas}

VER NA SEÇÃO DE DÉCIMA URBANA DA FGM-APMS

\begin{sidewaystable}
\IBGEtab{
\caption{Relação dos imóveis arrolados pelo município de Salvador no distrito de Brotas em 1924 (parte 1)}\label{tab:imoveis1924brotas1}}
{
\begin{minipage}{\textwidth}
\begin{tiny}
\begin{tabular}{m{3cm} m{1cm} l l l l l l l l l l l}
\toprule
\multirow{2}{*}{Locais}	& \multirow{2}{*}{Valor}	& \multicolumn{10}{c}{Imóveis}\\
\cline{3-13}
	&	&Térreos	&Sobrados	&Abarracados	&Barracão	&Telheiros	&Galpões	&Em ruínas	&Em construção	&Em reconstrução	&Interditados	&TOTAL\\
\midrule
\midrule
Rua dos Galés							&26:184\$	&22	&0	&1	&0	&0	&0	&0	&0	&0	&0	&23\\
Rua Coronel Frederico Costa					&7:440\$	&7	&0	&0	&0	&0	&0	&0	&0	&0	&0	&7\\
Rua Uruguaiana							&108:984\$	&143	&2	&0	&0	&0	&0	&0	&9	&0	&0	&154\\
Trav. da Rua Uruguaiana					&30:304\$	&120	&0	&0	&0	&0	&0	&0	&1	&0	&0	&121\\
Rua da Boa Vista						&38:980\$	&60	&2	&0	&0	&0	&0	&0	&1	&0	&0	&63\\
Rua Agrippino Dorea						&67:832\$	&77	&0	&2	&0	&0	&0	&1	&0	&0	&0	&80\\
Beco do General							&4:800\$	&9	&0	&0	&0	&0	&0	&0	&0	&0	&0	&9\\
Rua do Socorro							&31:680\$	&59	&0	&0	&0	&0	&0	&0	&0	&1	&0	&60\\
Trav. Castro Neves					&4:620\$	&10	&0	&0	&0	&0	&0	&0	&0	&0	&0	&10\\
Rua do Castro Neves						&54:600\$	&83	&0	&0	&0	&0	&0	&1	&2	&0	&0	&86\\
Rua da Alegria							&16:676\$	&30	&0	&0	&0	&0	&0	&0	&0	&0	&0	&30\\
Trav. da Alegria						&14:124\$	&23	&0	&0	&0	&0	&0	&0	&0	&0	&0	&23\\
Trav. do Sangradouro para a Trav. da Alegria		&13:728\$	&16	&0	&0	&0	&0	&0	&0	&0	&0	&0	&16\\
Trav. do Sangradouro para a Rua da Alegria			&12:096\$	&18	&0	&0	&0	&0	&0	&0	&0	&0	&0	&18\\
Trav. do Sangradouro para a Lad. do Matatu Pequeno	&12:120\$	&27	&0	&0	&0	&0	&0	&0	&0	&0	&0	&27\\
Rua do Sangradouro						&46:072\$	&61	&3	&0	&0	&0	&0	&0	&0	&0	&0	&64\\
Trav. do Sangradouro						&19:228\$	&	&66	&0	&0	&0	&0	&0	&0	&0	&0	&66\\
Alto do Sangradouro						&9:348\$	&18	&0	&0	&0	&0	&0	&0	&1	&0	&0	&19\\
Rua da Vala ao Cabula						&34:592\$	&76	&6	&0	&0	&0	&0	&0	&1	&0	&0	&83\\
Estr. 2 de Julho						&94:186\$	&292	&0	&0	&0	&0	&0	&1	&5	&0	&0	&298\\
1ª Lad. do Engenho Velho					&6:684\$	&19	&0	&0	&0	&0	&0	&0	&0	&0	&0	&19\\
2ª Lad. do Engenho Velho					&4:524\$	&20	&0	&0	&0	&0	&0	&0	&0	&0	&0	&20\\
Rua do Engenho Velho						&30:576\$	&112	&0	&0	&0	&0	&0	&0	&0	&0	&0	&112\\
Capelinha do Deus Menino					&102:924\$	&416	&0	&0	&0	&0	&0	&0	&5	&0	&0	&421\\
Quinta das Beatas						&39:432\$	&149	&0	&0	&0	&0	&0	&0	&2	&0	&0	&151\\
1ª Trav. da Quinta das Beatas				&2:700\$	&13	&0	&0	&0	&0	&0	&0	&0	&0	&0	&13\\
2ª Trav. da Quinta das Beatas				&1:860\$	&7	&0	&0	&0	&0	&0	&0	&0	&0	&0	&7\\
Alto do Formoso							&11:340\$	&50	&0	&0	&0	&0	&0	&0	&1	&1	&0	&52\\
Rua do Matatu Pequeno						&57:920\$	&70	&0	&0	&0	&0	&0	&0	&4	&0	&0	&74\\
Rua do Matatu Grande						&33:672\$	&86	&0	&0	&0	&0	&0	&0	&0	&0	&0	&86\\
Casa da Pólvora							&13:836\$	&30	&0	&0	&0	&0	&0	&0	&0	&0	&0	&30\\
\bottomrule
(continua na parte 2) & & & & & & & & & & & & \\
\end{tabular} 
\end{tiny}
\end{minipage}
}
{\fonte{\textbf{Annuario estatistico – annos de 1924 e 1925} organizado para o Governo da Bahia por M. Messias de \citeonline[pp.~263-264]{bahia_annuario_1926}.}}
\end{sidewaystable}
\begin{table}[!htp]
\IBGEtab{
\caption{Relação dos imóveis arrolados pelo município de Salvador no distrito de Brotas em 1924 (parte 2)}\label{tab:imoveis1924brotas1}}
{
\begin{tiny}
\begin{tabular}{m{3cm} m{1cm} m{0,7cm} m{0,7cm} m{0,7cm} m{0,7cm} m{0,7cm} m{0,7cm} m{0,7cm} m{0,7cm} m{0,7cm} m{0,7cm} m{0,7cm}}
\hline
\multirow{2}{*}{Locais}	& \multirow{2}{*}{Valor}	& \multicolumn{10}{c}{Imóveis}\\
\cline{3-13}
	&	&Tér- reos	&Sobra- dos	&Abarra- cados	&Barra- cão	&Telhei- ros	&Gal- pões	&Em ruínas	&Em cons- trução	&Em re- cons- trução	&Inter- dita- dos	&TOTAL\\
\hline
\hline
Lad. do Fabrício						&28:620\$	&27	&0	&2	&0	&0	&0	&0	&0	&1	&0	&30\\
Lad. do Acupe						&5:700\$	&13	&0	&0	&0	&0	&0	&0	&0	&1	&0	&14\\
Rua do Acupe							&16:332\$	&19	&0	&0	&0	&0	&0	&0	&0	&0	&0	&19\\
Trav. do Acupe						&1:740\$	&3	&0	&0	&0	&0	&0	&0	&0	&0	&0	&3\\
Rua de Brotas							&62:476\$	&65	&0	&1	&0	&0	&0	&0	&0	&0	&0	&66\\
1ª Trav. da Rua de Brotas					&6:156\$	&6	&0	&0	&0	&0	&0	&0	&0	&0	&0	&6\\
Cruz da Redenção						&12:276\$	&25	&0	&0	&0	&0	&0	&0	&0	&0	&0	&25\\
Rua do Beiju							&9:204\$	&30	&0	&0	&0	&0	&0	&0	&0	&0	&0	&30\\
Trav. da Rua do Beiju					&180\$		&1	&0	&0	&0	&0	&0	&0	&0	&0	&0	&1\\
Rua das Campinas						&7:680\$	&12	&0	&0	&0	&0	&0	&0	&0	&0	&0	&12\\
Vargem de Santo Antônio						&1:200\$	&2	&0	&0	&0	&0	&0	&0	&0	&0	&0	&2\\
Trav. do Pomar						&1:500\$	&8	&0	&0	&0	&0	&0	&0	&0	&0	&0	&8\\
Pomar								&240\$		&2	&0	&0	&0	&0	&0	&0	&0	&0	&0	&2\\
Candeal Pequeno							&1:680\$	&9	&0	&0	&0	&0	&0	&0	&0	&0	&0	&9\\
Candeal Grande							&360\$		&1	&0	&0	&0	&0	&0	&0	&0	&0	&0	&1\\
Lad. da Cruz das Almas					&12:912\$	&26	&0	&0	&0	&0	&0	&0	&0	&0	&0	&26\\
Largo da Mariquita						&24:480\$	&14	&0	&1	&0	&0	&0	&0	&0	&1	&0	&16\\
Rua dos Dendezeiros						&38:430\$	&32	&0	&0	&0	&0	&0	&0	&0	&1	&0	&33\\
Trav. da Rua dos Dendezeiros para a Rua do Meio		&816\$		&2	&0	&0	&0	&0	&0	&0	&0	&0	&0	&2\\
Rua do Meio							&17:610\$	&20	&0	&0	&0	&0	&0	&1	&0	&0	&0	&21\\
Rua Direita							&37:656\$	&37	&0	&0	&0	&0	&0	&0	&0	&0	&0	&37\\
Rua Fonte do Boi						&10:248\$	&13	&0	&0	&0	&0	&0	&0	&0	&0	&0	&13\\
Rua das Pedrinhas						&22:200\$	&30	&0	&0	&0	&0	&0	&0	&0	&0	&0	&30\\
1ª Trav. da Rua das Pedrinhas				&5:220\$	&9	&0	&0	&0	&0	&0	&0	&0	&0	&0	&9\\
2ª Trav. da Rua das Pedrinhas				&2:640\$	&5	&0	&0	&0	&0	&0	&0	&0	&0	&0	&5\\
Rua da Lagoa							&8:916\$	&34	&0	&0	&0	&0	&0	&0	&1	&0	&0	&35\\
Rua Direita da Amaralina					&27:700\$	&18	&0	&3	&0	&0	&0	&0	&0	&0	&0	&21\\
Rua do Meio da Amaralina					&17:220\$	&23	&0	&0	&0	&0	&0	&0	&0	&0	&0	&23\\
Alto da Ubarana							&900\$		&3	&0	&0	&0	&0	&0	&0	&0	&0	&0	&3\\
Pituba								&10:140\$	&24	&0	&0	&0	&0	&0	&0	&0	&0	&0	&24\\
Armação Pequena							&720\$		&2	&0	&0	&0	&0	&0	&0	&0	&0	&0	&2\\
Armação Grande							&600\$		&1	&0	&0	&0	&0	&0	&0	&0	&0	&0	&1\\
\hline
TOTAL								&1:346:814\$	&2:705	&13	&10	&0	&0	&0	&4	&33	&6	&0	&2771\\
\hline
\end{tabular} 
\end{tiny}
}
{\fonte{\citeonline[pp.~263-264]{bahia_annuario_1926}.}}
\end{table}


No que diz respeito ao \textit{valor da terra} no distrito, os dados da \autoref{tab:valorlocativomedio1924brotas}, construída com base nas informações apresentadas por \citeonline[pp.~263-264]{bahia_annuario_1926} num relatório estatístico publicado em 1926, ajudam a lançar luzes sobre a questão. Os valores nela apresentados foram obtidos dividindo-se a massa do valor locativo dos imóveis de cada rua pelo número de imóveis da mesma rua; por rudimentar que seja o procedimento, foi a solução encontrada para evitar as distorções verificáveis quando foi tentada a comparação direta entre as massas de valores locativos de cada rua, por força de fatores como as diferenças nos valores locativos de cada imóvel e as grandes variações no número de imóveis por rua.

\begin{table}[!htp]
\IBGEtab{
\caption{Valor locativo médio dos imóveis nas ruas arroladas pelo município de Salvador no distrito de Brotas em 1924}\label{tab:valorlocativomedio1924brotas}}
{
\begin{tiny}
\begin{tabular}{ll}
\toprule
Locais	&Valor locativo médio por imóvel\\
\midrule
\midrule
Rua dos Galés	&1:138\$430\\
Rua Coronel Frederico Costa	&1:062\$860\\
Rua Uruguaiana	&707\$690\\
Travessa da Rua Uruguaiana	&250\$450\\
Rua da Boa Vista	&618\$730\\
Rua Agrippino Dorea	&847\$900\\
Beco do General	&533\$330\\
Rua do Socorro	&528\$000\\
Travessa do Castro Neves	&462\$000\\
Rua do Castro Neves	&634\$880\\
Rua da Alegria	&555\$870\\
Travessa da Alegria	&614\$090\\
Travessa do Sangradouro para a Travessa da Alegria	&858\$000\\
Travessa do Sangradouro para a Rua da Alegria	&672\$000\\
Travessa do Sangradouro para a Ladeira do Matatu Pequeno	&448\$890\\
Rua do Sangradouro	&719\$880\\
Travessa do Sangradouro	&291\$330\\
Alto do Sangradouro	&492\$000\\
Rua da Vala ao Cabula	&416\$770\\
Estrada 2 de Julho	&316\$060\\
1ª Ladeira do Engenho Velho	&351\$790\\
2ª Ladeira do Engenho Velho	&226\$200\\
Rua do Engenho Velho	&273\$000\\
Capelinha do Deus Menino	&244\$480\\
Quinta das Beatas	&261\$140\\
1ª Travessa da Quinta das Beatas	&207\$690\\
2ª Travessa da Quinta das Beatas	&265\$710\\
Alto do Formoso	&218\$080\\
Rua do Matatu Pequeno	&782\$700\\
Rua do Matatu Grande	&391\$530\\
Casa da Pólvora	&461\$200\\
Ladeira do Fabrício	&954\$000\\
Ladeira do Acupe	&407\$140\\
Rua do Acupe	&859\$580\\
Travessa do Acupe	&580\$000\\
Rua de Brotas	&946\$610\\
1ª Travessa da Rua de Brotas	&1:026\$000\\
Cruz da Redenção	&491\$040\\
Rua do Beiju	&306\$800\\
Travessa da Rua do Beiju	&180\$000\\
Rua das Campinas	&640\$000\\
Vargem de Santo Antônio	&600\$000\\
Travessa do Pomar	&187\$500\\
Pomar	&120\$000\\
Candeal Pequeno	&186\$670\\
Candeal Grande	&360\$000\\
Ladeira da Cruz das Almas	&496\$620\\
Largo da Mariquita	&1:530\$000\\
Rua dos Dendezeiros	&1:164\$550\\
Travessa da Rua dos Dendezeiros para a Rua do Meio	&408\$000\\
Rua do Meio	&838\$570\\
Rua Direita	&1:017\$730\\
Rua Fonte do Boi	&788\$310\\
Rua das Pedrinhas	&740\$000\\
1ª Travessa da Rua das Pedrinhas	&580\$000\\
2ª Travessa da Rua das Pedrinhas	&528\$000\\
Rua da Lagoa	&254\$740\\
Rua Direita da Amaralina	&1:319\$050\\
Rua do Meio da Amaralina	&748\$700\\
Alto da Ubarana	&300\$000\\
Pituba	&422\$500\\
Armação Pequena	&360\$000\\
Armação Grande	&600\$000\\
\midrule
TOTAL	&568\$170\\
\bottomrule
\end{tabular} 
\end{tiny}
}
{\fonte{Elaboração do autor, com base no \textbf{Annuario estatistico – annos de 1924 e 1925} organizado para o Governo da Bahia por M. Messias de  \citeonline[pp.~263-264]{bahia_annuario_1926}.}}
\end{table}


O uso do valor locativo médio dos imóveis de cada rua do distrito como indicador do valor da terra permite classificar as ruas de Brotas segundo o valor médio de seus imóveis, estabelecendo assim aquelas com imóveis ``mais caros'' e aqueloutras com imóveis ``mais baratos'' (ressalvando-se sempre tratar-se de valores médios, não de valores reais). Os resultados deste procedimento podem ser verificados nas tabelas \autoref{tab:maiscaros1924brotas} e \autoref{tab:maisbaratos1924brotas} (\pageref{tab:maisbaratos1924brotas}), dos quais se pode extrair conclusões importantes.

\begin{table}[!htp]
\IBGEtab{
\caption{Valor locativo médio dos imóveis nas dez ruas de imóveis ``mais caros'' entre os arrolados pelo município de Salvador no distrito de Brotas em 1924}\label{tab:maiscaros1924brotas}}
{
\begin{tabular}{rr}
\toprule
Locais	&Valor locativo médio por imóvel\\
\midrule
\midrule
Largo da Mariquita	&1:530\$000\\
Rua Direita da Amaralina	&1:319\$050\\
Rua dos Dendezeiros	&1:164\$550\\
Rua dos Galés	&1:138\$430\\
Rua Coronel Frederico Costa	&1:062\$860\\
1ª Travessa da Rua de Brotas	&1:026\$000\\
Rua Direita	&1:017\$730\\
Ladeira do Fabrício	&954\$000\\
Rua de Brotas	&946\$610\\
Rua do Acupe	&859\$580\\
\bottomrule
\end{tabular} 
}
{\fonte{Elaboração do autor, com base no \textbf{Annuario estatistico – annos de 1924 e 1925} organizado para o Governo da Bahia por M. Messias de \citeonline[pp.~263-264]{bahia_annuario_1926}.}}
\end{table}

\begin{table}[!htp]
\IBGEtab{
\caption{Valor locativo médio dos imóveis nas dez ruas de imóveis ``mais baratos'' entre os arrolados pelo município de Salvador no distrito de Brotas em 1924}\label{tab:maisbaratos1924brotas}}
{
\begin{tabular}{rr}
\hline
Locais	&Valor locativo médio por imóvel\\
\hline
\hline
Alto da Ubarana	&300\$000\\
Travessa do Sangradouro	&291\$330\\
Rua do Engenho Velho	&273\$000\\
2ª Travessa da Quinta das Beatas	&265\$710\\
Quinta das Beatas	&261\$140\\
Rua da Lagoa	&254\$740\\
Travessa da Rua Uruguaiana	&250\$450\\
Capelinha do Deus Menino	&244\$480\\
2ª Ladeira do Engenho Velho	&226\$200\\
Alto do Formoso	&218\$080\\
1ª Travessa da Quinta das Beatas	&207\$690\\
Travessa do Pomar	&187\$500\\
Candeal Pequeno	&186\$670\\
Travessa da Rua do Beiju	&180\$000\\
Pomar	&120\$000\\
\hline
\end{tabular}
}
{\fonte{Elaboração do autor, com base em \citeonline[pp.~263-264]{bahia_annuario_1926}.}}
\end{table}

\begin{itemize}
\item Três entre as ruas com imóveis ``mais caros'' (Largo da Mariquita, Rua Direita e Rua dos Dendezeiros) encontram-se no Rio Vermelho, que manteve durante todo o período seu caráter de \textit{arrabalde de veraneio} dos soteropolitanos mais ricos. 
\item Duas entre as ruas com imóveis ``mais baratos'' (Alto da Ubarana e Rua da Lagoa) e uma dentre aquelas com imóveis ``mais caros'' (Rua Direita da Amaralina) encontram-se em Amaralina, fato que será investigado mais adiante; a Rua Direita da Amaralina, entretanto, é a principal da Cidade Balneária Amaralina, talvez um dos primeiros loteamentos formais de Salvador, totalmente voltado para atividades veranistas de alto padrão.
\item A larga Rua dos Galés segue entre as mais valorizadas do distrito, especialmente por ser eixo central da mais antiga área urbanizada do distrito.
\item Duas vias que hoje chamaríamos de ``arteriais'' ou ``coletoras'', a Rua de Brotas e a Rua do Acupe, encontram-se entre as mais valorizadas do distrito, talvez exatamente pela centralidade que exercem sobre a circulação no território.
\item Quatro das ruas que integram a área do Engenho Velho de Brotas (Rua do Engenho Velho, Travessa da Rua Uruguaiana, Capelinha do Deus Menino, 2ª Ladeira do Engenho Velho) estão entre aquelas com imóveis ``mais baratos'', enquanto uma delas (Rua Coronel Frederico Costa) encontra-se no grupo das ruas com imóveis ``mais caros''; é de se notar que as ruas menos valorizadas são exatamente aquelas que se encontram nas encostas e barrancos do Engenho Velho, portanto as mais insalubres e de difícil acesso, enquanto a mais valorizada entre elas encontra-se na cumeada, em terreno plano, fazendo a ligação desta área do distrito com a valorizada Rua de Brotas.
\item Todas as ruas abertas na Quinta das Beatas e áreas circunvizinhas (Alto do Formoso, 1ª e 2ª Travessas e a própria Quinta das Beatas) encontram-se entre aquelas com imóveis ``mais baratos''.
\end{itemize}

O cálculo de proporção de sobrados empregue na \autoref{subsubsec:polfundvalter} é inaplicável ao distrito, pois das 63 ruas cujos dados foram disponibilizados por \citeonline[pp.~263-264]{bahia_annuario_1926} apenas quatro apresentam este tipo de imóvel (Rua da Vala ao Cabula, com 6; Rua do Sangradouro, com 3; Travessa da Uruguaiana, com 2; e Rua da Boa Vista, com 2); conquanto este indicador sirva para reforçar o que já se viu a respeito do valor da terra no Sangradouro e na cumeada da Boa Vista, empregá-lo sem maiores cuidados equivaleria a considerar todo o restante do distrito com valor de terra ínfimo, o que não condiz com os dados encontrados. 

\subsection{Construções, ampliações, reformas: urbanização desigual e a passos lentos}\label{subsec:constrampliref}

Há dois aspectos da questão: os \textit{requerentes} e os \textit{profissionais contratados}.

REQUERENTES

No que diz respeito aos \textit{profissionais contratados}, observa-se o quadro de profissionais atuantes no distrito tal como exposto na \autoref{tab:profissionaisatuantesbrotas}

INSERIR TABELA COM PROFISSIONAIS E NÚMERO DE OBRAS





 . Entre estes profissionais, destaca-se como exemplo da circulação entre o público e o privado o engenheiro, arquiteto e construtor -- ele mesmo variava sua qualificação profissional de acordo com a função que exercia no momento -- \textit{Pedro Jayme David}. 

\subsection{Loteamentos: por que tão poucos?}\label{subsec:loteamentos}

No que diz respeito aos loteamentos, pouco foi possível de encontrar nos arquivos consultados. Na verdade, tem-se notícia dos seguintes loteamentos oficiais:

\begin{itemize}
\item O famoso plano da \textit{Cidade Luz}, concebido em 1919 por Theodoro Sampaio, que traçou as linhas gerais do que hoje conhecemos como o bairro da Pituba.
\item O loteamento da \textit{Cidade Balneária Amaralina}, cuja planta original não foi possível encontrar mas que se sabe, por meio das licenças de construção, ampliação e obras consultadas, já ser existente em 1893.
\item O loteamento da fazenda \textit{Santa Cruz}, datado de XXXX.
\end{itemize} 

Como explicar, portanto, a fragmentação imobiliária encontrada a uma simples comparação entre a quantidade de imóveis na freguesia em 1886 (\autoref{tab:decurb1886-1891}, \pageref{tab:decurb1886-1891}) e aquela encontrada em 1924 (\autoref{tab:imoveis1924brotas1} e \autoref{tab:imoveis1924brotas2}, \pageref{tab:imoveis1924brotas2})?

Não há explicação única, mas vários elementos dispersos parecem confirmar hipóteses complementares: a continuidade dos \textit{arrendamentos} e \textit{alugueis} como forma preferencial de acesso à terra e à habitação pelos mais remediados; a opção pelos \textit{loteamentos informais} como confluência entre a sonegação fiscal pelos terratenentes e a satisfação da necessidade de acesso barato à posse da terra, ainda que insegura, pelos mais pobres; a persistência dos processos de \textit{ocupação informal da terra}, estes quase impossíveis de rastrear exatamente por não serem documentados.

ARRENDAMENTOS E ALUGUEIS

LOTEAMENTOS INFORMAIS

Um exemplo clássico de loteador informal é \textit{José Visco}, que durante algum tempo emprestou seu nome à ladeira do Funil, no Engenho Velho de Brotas. 

OCUPAÇÃO INFORMAL

\subsection{Outros usos}

Prosseguiam as comemorações do Dois de Julho no distrito tão animadas -- ou mesmo mais, quem sabe! -- quanto nos tempos do império:

\begin{citacao}
\textbf{2 de Julho do Castro Neves}

Da comissão organisadora desses festejos, recebemos um delicado convite a que agradecemos, para assistirmos ao desfilar do prestito que se realizará no proximo domingo, 6 do corrente, a 1 hora da tarde, sahindo do Largo da Boa Vista e seguindo pelas ruas 1º de Março Pitangueiras, Largo do Paranhos, Matatú Pequeno, Fabricio, Sangradouro, Sete Portas, regressando pela Ladeira do Santo Agostinho, rua da Alegria, Socorro, Ladeira dos Galés, Fonte Nova, donde se dirigirá para o Castro Neves, que estará festivamente embandeirado.

Ao passar pelo Matatu pequeno falará o acadêmico Lemos Britto.

Os festejos prolongar-se-ão até o dia 8.

O programma é o seguinte!

Abrira a comitiva civica um grupo numeroso de cavalheiros, trajados symbolicamente;

seguir-se-á uma banda de clarins, fanfarreando em toques estridentes, e annunciando à multidão anciosa a aproximação do imponente e magestoso prestito;

segue-se um piquete de cavallaria;

virá depois um pelotão de cornetas e caixas de guerra, surgindo então, da immensa massa do povo, o Carro Emblematico;

musica de S. Vicente de Paulo;

batalhão de graciosas meninas symbolisando as \og Heroinas Brasileiras \fg{};

banda do 2º corpo do Regimento Policial;

batalhão de meninos \og Defensores do Castro Neves \fg{};

musica dos Salesianos, seguindo-se o collegio encorporado da referida associação beneficente;

banda do 1º corpo policial;

banda do 5º batalhão de artilharia, etc., etc.\footnote{\textbf{Correio do Brazil}, ano III, nº 567, 4 ago. 1905, p. 4}
\end{citacao}