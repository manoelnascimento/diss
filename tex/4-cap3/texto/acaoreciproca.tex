\section{Ação recíproca: Brotas e as reformas, as reformas e Brotas}\label{sec:3.3}

Nenhum dos agentes de produção do espaço urbano citados até o momento teria sido capaz de produzir, por si só, o processo de urbanização. Agiram todos eles de uma só vez sobre o mesmo processo, aliando-se às vezes, entrechocando-se noutras; fingindo concórdia às vezes, fingindo discórdia noutras; o que resulta daí é um espaço urbano produzido, apropriado e usado segundo as possibilidades de ação dos agentes envolvidos, fortemente constrangidas pela sua situação na estrutura de classes da sociedade soteropolitana de então, pelo regime fundiário vigente em cada conjuntura, pelos acordos e desacordos estabelecidos entre si, em suma, por condições sobre as quais nem sempre tiveram governabilidade.



\subsection{Natureza das pressões por investimentos públicos}

Há que se observar, quanto aos investimentos públicos analisados, dois tipos de iniciativa: aquela da própria municipalidade ou do governo estadual, atendendo a interesses os mais diversos; e aquela oriunda de pressão pública feita por moradores de determinadas localidades, em especial por meio da imprensa.

\subsection{Natureza das pressões por determinada ordem e sentido no espaço público}



\subsection{•}

INTRODUZIR A QUESTÃO

\subsection[Houve influência da produção do espaço em Brotas sobre as reformas\\ urbanas de Salvador?]{Houve influência da produção do espaço em Brotas sobre as reformas\\ urbanas de Salvador?}\label{subsec:3.3.1}

ESCREVER POR ÚLTIMO

\subsection[Houve influência da produção do espaço pelas reformas de J. J. Seabra\\sobre Brotas?]{Houve influência da produção do espaço pelas reformas de J. J. Seabra\\ sobre Brotas?}\label{subsec:3.3.2}

ESCREVER POR ÚLTIMO