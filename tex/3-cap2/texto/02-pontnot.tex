\section{Pontos notáveis}\label{sec:pontnot}

Para melhor situar os resultados da pesquisa sem a necessidade de recurso constante a mapas, serão expostos a seguir alguns \textit{pontos notáveis} da freguesia, agrupados em duas categorias: \textit{naturais} e \textit{construídos}.

\subsection{Naturais}\label{subsec:pontnat}

Os principais pontos notáveis da freguesia são cursos e espelhos d'água, que formam os vales e cumeadas distintivos das muitas áreas em que se subdivide a freguesia.

O \textit{Dique do Tororó} não é propriamente parte da freguesia; é uma de suas fronteiras, pois os limites da freguesia o margeiam. A lagoa que hoje conhecemos como o dique já foi muito maior, havendo quem afirme ter chegado a seis quilômetros de extensão antes do aterro de 1810 criar a ligação, por meio da atual ladeira dos Galés, entre as freguesias de Brotas e Sant'Anna \cite[p.~48]{santos_aguas_2010}. O dique era alimentado pelas águas de pequenos rios, mas hoje é alimentado principalmente por esgotos dos bairros circunvizinhos \cite[p.~41]{santos_aguas_2010}.

O rio \textit{Camarajipe}, \textit{Camaragipe} ou \textit{Camorogipe}\footnote{A grafia muda de acordo com a documentação encontrada; salvo quando constar na fonte original, será adotada a grafia ``Camarajipe'' sugerida por \citeonline{edelweiss_camarajipe_1969}.}, centro da terceira maior bacia hidrográfica de Salvador na atualidade\footnote{A maior é a do rio \textit{Ipitanga}, sub-bacia do rio \textit{Joanes} \cite[p.~311]{santos_aguas_2010}, seguida pela do rio \textit{Jaguaribe} \cite[p.~229]{santos_aguas_2010}.}, nasce longe de Brotas, nas cercanias do dique do Cabrito, e corta longitudinalmente o território soteropolitano até desembocar no Atlântico. O Camarajipe teve seu curso alterado na década de 1970 para evitar as constantes enchentes nas áreas mais baixas do Rio Vermelho; por meio de dragagem e rebaixamento do substrato do vale, seu curso foi estendido até o vale do rio Pernambués\footnote{O rio Pernambués corre ao lado da atual avenida Luiz Eduardo Magalhães e em seu curso original seguia rumo à área hoje ocupada pelo Shopping da Bahia e pela estação rodoviária de Salvador. Ainda existe, quase como um córrego.}, de onde foi possível canalizá-lo até sua foz atual, no bairro do Costa Azul.

Dentro da bacia do Camarajipe destacam-se quatro afluentes de extrema importância vistos no mapa de \citeonline{weyll_mappa_1851}; ainda que de curso mais curto, mais estreitos e de menor vazão que seu rio principal, sua importância está na demarcação de vales e cumeadas da freguesia. Em primeiro lugar, o \textit{rio das Tripas}, vizinho ao qual corria a rua da Vala, no trecho correspondente à atual avenida Heitor Dias; ele escava o vale que separa a freguesia de Brotas da freguesia do Santo Antônio. O segundo rio é indicado por \citeauthoronline{weyll_mappa_1851} como sendo o \textit{Santo Antônio}, hoje um esgoto que corre quase paralelamente à atual rua do Baixão, no Matatu \cite[p.~136]{santos_aguas_2010}; ele escava o vale que separa as cumeadas do Matatu Grande e do Matatu Pequeno. O terceiro rio é o \textit{Córrego das Beatas}, correspondente, grosso modo, à atual Baixa do Matatu, também transformado em esgoto \cite[p.~158]{santos_aguas_2010}; ele separa o Matatu Grande e a Quinta das Beatas por um vale. O Córrego das Beatas é afluente do rio \textit{Bonocô} ou \textit{Campinas}, que no mapa de \citeonline{weyll_mappa_1851} separa a Quinta das Beatas da cumeada cortada pela Estrada de Brotas e hoje corre sob a avenida Mário Leal Ferreira (popularmente conhecida pelo mesmo nome do rio).

\begin{landscape}
\begin{figure}[!htp]
\centering
\includegraphics[width=1\textwidth]{3-cap2/complementos/mapas/1858-pontecamorogipe.eps}{\par \textbf{Fonte:} \textbf{BR BAAPB}, Biblioteca, planta 262. \par}
\caption{``Ponte sobre o rio Camorogipe'', projeto de autoria de Manoel da Silva Pereira, capitão do corpo de engenheiros (1858).}
\end{figure}
\end{landscape}

O rio \textit{Lucaia} nasce nas encostas e grotões da cumeada onde hoje se localiza a avenida Joana Angélica, recebe contribuição do dique do Tororó e do riacho que escava o vale onde hoje se localiza a avenida Garibaldi, corta o vale que separa o Engenho Velho de Brotas do Engenho Velho da Federação e desemboca no largo da Mariquita (Rio Vermelho), onde se localizava a foz do rio Camarajipe antes de sua transposição nos anos 1970.

O último entre os rios tem menos importância para a história da freguesia de Brotas, mas importa por ser um de seus limites naturais: é o \textit{rio das Pedras}, cuja bacia é atualmente a quarta maior do município e inclui a sub-bacia do rio \textit{Pituaçu} \cite[p.~175]{santos_aguas_2010}, também tido em alguns documentos como um dos limites da freguesia. Este rio tem leito de curta extensão, pois trata-se de um curso d'água formado pela fusão dos rios \textit{Cascão}, \textit{Saboeiro} e Pituaçu, estes, sim, de maior extensão; sua foz é na Boca do Rio, no lugar onde se situava a antiga sede de praia do Esporte Clube Bahia \cite[p.~175]{santos_aguas_2010}.

\subsection{Religiosos}\label{subsec:pontrel}

\begin{figure}[!htp]
\centering
\subfloat[Fachada]{
\includegraphics[width=0.4\textwidth]{3-cap2/complementos/imagens/nsbrotas-fachada[www_igrejas-bahia_com].eps} 
\label{Fachada}
}
\  %espaco separador
\subfloat[Altar-mor]{
\includegraphics[width=0.4\textwidth]{3-cap2/complementos/imagens/nsbrotas-altarmor[www_igrejas-bahia_com].eps} 
\label{Altar-mor}
}
\  %espaco separador
\subfloat[Interior]{
\includegraphics[width=\textwidth]{3-cap2/complementos/imagens/nsbrotas-interior[www_igrejas-bahia_com].eps} 
\label{Interior}
}
\caption{Igreja de Nossa Senhora de Brotas em sua configuração atual. \textbf{Fonte:} http://www.igrejas-bahia.com}
\end{figure}

É inevitável tratar dos pontos notáveis deste distrito sem começar pela \textit{igreja de Nossa Senhora de Brotas}, cujo orago emprestou seu nome à freguesia. Não é tarefa simples, pois nem o próprio IPAC-BA, quando do tombamento da igreja pelo Decreto Estadual nº 11.673/2009, soube precisar sua data de fundação, remetendo-a, pela ``oralidade'', a 1714 \cite{ipac_brotas_2015}. Esta ``oralidade'', todavia, é controversa. 

Em primeiro lugar, a historiografia demonstra que sua construção nem foi tão rápida, nem tampouco precedeu a criação da freguesia em 1718; o mais provável é que a freguesia tenha sido criada primeiro, e o templo tenha vindo depois. As razões são muito comezinhas. Havia na paróquia ``apenas algumas fazendas e engenhos e poucos moradores, na maior parte de poucos recursos'' e, portanto, ``entravam poucas esmolas para a construção da matriz de pedra e cal'' \cite[p.~10]{ott_engenhos_1996}. Já em 1729 os paroquianos pediam ao rei João V uma ajuda de custo para a construção da igreja que viria a substituir a arruinada capela de taipa que sediava a freguesia \cite[p.~10]{ott_engenhos_1996}, e há ainda uma carta do Provedor-Mor da Fazenda Real do Estado do Brasil ao rei João V, datada de 1732, pedindo-lhe que socorresse as obras de construção do templo, vez que os paroquianos não dispunham de recursos \cite[p.~30]{vivas_botelho_2011}. Em 1739 o rei deu nove mil e quinhentos cruzados para acabar as obras da capela-mor, repartidos em pagamentos de três anos \cite[p.~10]{ott_engenhos_1996}; mesmo assim, em 1741 a Irmandade do Santíssimo Sacramento e Nossa Senhora de Brotas chegou a tomar empréstimo à Santa Casa de Misericórdia para terminar algumas obras na matriz, e tudo indica que em 1753 a obra não havia sido concluída pela falta de esmolas para continuá-la, resultando disto novo pedido de auxílio ao rei Jose I, que não se sabe se foi atendido \cite[p.~10]{ott_engenhos_1996}. Ora, se o templo aparentava não estar ainda concluído em 1753, malgrado os esforços hercúleos dos paroquianos, é impossível que houvesse sido fundado em 1714. 

Em segundo lugar, dois estudos antigos sobre este templo \cite{campos_brotas_1942,texbar_capellas_1930} apontam outros fatos elucidadores sobre sua construção. O mais antigo dos dois estudos diz que a igreja de Nossa Senhora de Brotas é de construção posterior à instituição da freguesia em 1718, pois sua matriz teria sido, nos primeiros tempos, a antiga igreja de São Paulo \cite[p.~344]{texbar_capellas_1930}. O segundo dos dois estudos não dá informações outras além da existência, antes de reforma já antiga, de uma campa na capela-mor onde se destacava 

\begin{citacao}
um escudo d'armas em relevo [\dots] [que] talvez lancasse alguma luz sobre a história do santuário, porque diziam os antigos moradores do `arraial' [o Largo de Brotas] que sob aquela pedra jaziam as cinzas do construtor da igreja \cite[p.~88]{campos_brotas_1942}
\end{citacao}

Como se vê, a data de início das obras do templo é incerta, estando em algum lugar entre 1714 e 1729; sua inauguração é ainda mais incerta, diante da sucessão de pedidos de auxílio para conclusão das obras. É possível que em 1757, quando do primeiro censo eclesiástico da freguesia, o templo já estivesse construído, mas a existência da data 1772 sobre o arco central da galilé põe em dúvida esta hipótese. Teria sido ela concluída ou meramente reformada em 1772? A documentação encontrada, bem como as fontes secundárias, não permitem estabelecer qualquer das duas hipóteses como definitiva; o que é possível concluir, entretanto, é que neste ano o templo chegou a um formato definitivo, sujeito a reformas posteriores, como a de abril de 1889\footnote{\textbf{Treze de Maio}, 01 abr. 1889, p. 3}. Em sua configuração atual, trata-se de uma igreja com fachada em estilo rococó e altares em estilo neoclássico. Já a Irmandade do Santíssimo Sacramento e Nossa Senhora de Brotas foi criada em 1781 \cite[p.~172]{VASCONCELOS2002}, que no último quarto do século XIX ainda era muito ativa\footnote{A pesquisa realizada nas edições do jornal \textbf{O Monitor} publicadas entre 1876 e 1881 evidenciou a atividade da irmandade durante o período, em especial na realização de funerais e missas solenes de seus integrantes}.

O mito fundador do templo diz que a devoção a Nossa Senhora de Brotas se deve ao fato de um homem muito pobre, que morava no vale atrás da atual igreja, ter apelado com sucesso à Virgem Maria para salvar do afogamento a única vaca que lhe dava sustento e à sua família; como a santa teria aparecido ao vaqueiro com seu filho ao colo nas \textit{grotas} existentes atrás do sítio da atual igreja, daí teria surgido o nome primitivo de ``Nossa Senhora das Grotas'', que sendo variado ao longo do tempo resultou na atual denominação do distrito \cite{campos_brotas_1942,texbar_capellas_1930}. A única fonte deste mito, um boletim paroquial de 1924 \cite[p.~345]{texbar_capellas_1930}, não menciona a longa cauda do mitema do apelo do vaqueiro à santa, que pode ser rastreado, passando por Juazeiro (1710), por Sergipe (1650) e pela igreja da Graça (onde se relata ter existido uma imagem de Nossa Senhora de Brotas muito antes da fundação da freguesia) \cite[p.~89-92]{campos_brotas_1942} até chegar à antiquíssima vila portuguesa de Brotas, atualmente uma freguesia do concelho de Mora, distrito de Évora, na regiao do Alentejo; esta vila foi fundada em torno da igreja de Nossa Senhora de Brotas, a primeira de todas, construida em 1424 \cite{campos_brotas_1942, correia_brotas_2010}. A Arquidiocese de São Salvador da Bahia, inclusive, reconhece esta ``filiação'' \cite{arqui_brotas_2015}.

\begin{figure}[!htp]
\centering
\includegraphics[width=1\textwidth]{3-cap2/complementos/imagens/deusmenino[wikimapia_org].eps} 
\caption{Igreja do Deus Menino em sua configuração atual. \textbf{Fonte:} http://www.wikimapia.org}
\end{figure}

A \textit{Capela do Deus Menino} é a precursora da atual igreja homônima, situada na atual rua Brígida do Vale, no Engenho Velho de Brotas. Apesar de dar nome a uma área inteira do distrito, como visto na documentação pesquisada (Alto da Capelinha, Rua da Capelinha etc.), não foi possível encontrar nem na documentação nem na bibliografia qualquer informação acerca de sua fundação ou de sua construção.  

\begin{figure}[!htp]
\centering
\includegraphics[width=1\textwidth]{3-cap2/complementos/imagens/bomjesusdosmilagres[salvador365igrejas_blogspot_com_br].eps} 
\caption{Igreja do Senhor Bom Jesus dos Milagres em sua configuração atual. \textbf{Fonte:} http://salvador365igrejas.blogspot.com.br}
\end{figure}

A \textit{Capela do Senhor Bom Jesus dos Milagres} é a precursora do templo homônimo existente ainda hoje no Largo dos Paranhos, na bifurcação entre o Matatu e Cosme de Farias. Pouco há de informações sobre ela, sabendo-se apenas que foi construída em meados do século XIX em estilo neoclássico. Sua forma atual resulta de reforma realizada em 1954.

Apesar de a Assembleia Provincial haver destinado 500\$000 para sua reforma em 1864 \cite[anexo~2, p.~2]{silvagomes_relatorio_1864} e outros 2:000\$000 para sua reforma em 11 de abril de 1877\footnote{\textbf{O Monitor}, 12 abr. 1877, p. 1}, em abril de 1889 encontrava-se em ruínas \footnote{\textbf{Treze de Maio}, 01 abr. 1889, p. 3}. 

Esta capela foi associada à Irmandade de Nosso Senhor dos Milagres, que em 1873 requereu à Assembleia Legislativa da província, junto com moradores das Pitangueiras, recursos para pagar um capelão para celebrar os atos religiosos, pois era ``distante da igreja matriz cerca de meia legua''; este requerimento, analisado pelos deputados, foi julgado merecedor da atenção da Assembleia, mas sua deliberação foi condicionada aos ``recursos que ministrar o orçamento provincial'' \cite[p.~46]{bahia_relatassleg_1873}. O pedido foi posteriormente aprovado \cite[p.~53]{bahia_relatassleg_1873}.

\begin{figure}[!htp]
\centering
\includegraphics[width=1\textwidth]{3-cap2/complementos/imagens/nsluz[www_igrejas-bahia_com].eps} 
\caption{Igreja de Nossa Senhora da Luz em sua configuração atual. \textbf{Fonte:} http://www.igrejas-bahia.com}.
\end{figure}

A \textit{Capela de Nossa Senhora da Luz} é a precursora da atual igreja de mesmo nome, situada na praça igualmente homônima, na Pituba. Há interessante registro histórico apresentado no \textit{website} da paróquia:

\begin{citacao}
Conforme a tradição, pelos anos de 1580, uma menina de uns doze anos de idade, andando para apanhar gravetos para cozinhar, sentindo sede, viu (ou imaginou ver) surgir entre a mata e a areia, uma figura de uma mulher linda, com um menino sentado no braço esquerdo e na mão direita uma vela acesa; atrás da figura veio um manancial de água, e todo o quadro como iluminado com uma luz azul. – Saciada a sede, correu para casa e comunicou aos seus pais o ocorrido, os quais vieram acompanhando-a; chegados ao lugar por ela indicado, constataram a existência do manancial de água, não sabendo explicar se já existia antes; e mais nada viram. Este lugar situa-se perto da confluência das ruas São Paulo com Rio Grande do Sul, bem perto da Praça Belo Horizonte.

Passados anos, a menina de então, já adulta, avistou em casa do capitão Felipe Correa uma Imagem de Nossa Senhora da Luz, trazida de Portugal, reconhecendo ser a mesma que tinha visto ou imaginado na fonte.

Pelos anos de 1600, o latifundiário e capitão Felipe Correa, proprietário da Fazenda Pituba, fez construir em terreno de sua propriedade, uma capela de taipa, no lugar que hoje seria entre as ruas Minas Gerais e Otávio Mangabeira, colocando na mesma a Imagem trazida de Portugal, de talha de madeira, medindo 53 centímetros, com o pedestal, conservada na sua Igreja da Pituba.

Durante os anos de 1610 a 1642, sendo atendente espiritual do litoral baiano, compreendido entre o Rio Vermelho e a Vila de Abrantes, o artista e religioso do Mosteiro de São Bento, Frei Agostinho da Piedade, o grande escultor e ceramista, fez para a capela da Pituba uma Imagem de Nossa Senhora da Luz, de barro cozido, e policromado, que é uma relíquia preciosa de quando o Brasil amanhecia, a qual no ano de 1949 foi restaurada, sendo reencarnada.

Os herdeiros do capitão Felipe Correa, capitão Manoel Gonçalves Saraiva e sua esposa Francisca Ferreira e o irmão desta, Francisco Ferreira, restauraram a capela pelos anos de 1663 \cite{fernandez_historia_1969}.
\end{citacao}

É preciso, entretanto, checar as informações com outras fontes para averiguar a precisão histórica do relato. Já encontramos a capela ereta em 1673, quando a Santa Casa de Misericórdia, responsável por pagar as côngruas dos sacerdotes deste templo entre 1676 e 1680, pagou também 13\$250 a Manoel Gonçalves Saraiva pelos consertos nela realizados. \cite[p.~11]{ott_engenhos_1996}. 

Vistos os principais pontos notáveis religiosos encontrados na documentação pesquisada e na bibliografia consultada, é preciso ressaltar uma lacuna séria nesta classificação. Qualquer templo ou local consagrado reflete a cosmovisão daqueles que em determinada formação social assumem o papel de \textit{dominantes} e \textit{exploradores}; aos \textit{dominados} e aos \textit{explorados}, conquanto tenham cosmovisão própria, costuma caber apenas a aceitação desta cosmovisão, ou a convivência com ela. 

Brotas tem longa história como território de prática das religiões afrobrasileiras. 

O terreiro Mutá Lambô ye Kaiongo, por exemplo, está hoje situado em Cajazeiras XI, mas remonta suas origens familiares a outros templos que funcionaram em ``Daniel Lisboa, Bonocô, Campinas de Brotas, Cosme de Farias'' \cite[p.~50]{alves_paquetan_2010}, embora no relato consultado estes terreiros ascendentes não estejam mais precisamente situados histórica ou geograficamente.

Otampê Ojarô - que recebeu o nome cristão de Maria do Rosário Francisca Régis -, foi a fundadora do Terreiro do Alaketu, no Matatu de Brotas



O terreiro do Alaketo,



\subsection{Civis}\label{subsec:pontciv}

Além dos pontos notáveis naturais e religiosos, foram encontrados na documentação pesquisada e na bibliografia consultada vários pontos notáveis \textit{civis}.

O \textit{Solar Boa Vista} é o palacete com torre encontrado ainda hoje no centro do Engenho Velho de Brotas. Ao longo de sua existência teve vários usos, a maior parte deles ligada a alguma função sanitária ou administrativa da cidade.

O solar foi erguido por volta das últimas décadas do século XVIII a mando do traficante de escravos \textit{Manuel José Machado}, que ordenou construírem-lhe torre alta o suficiente para acompanhar a chegada de navios ao porto de Salvador \cite[p.~127]{mattos_panorama_2011}. Em 1824 \textit{Joaquina Josefa de Santana Machado} recebeu o solar como herança; em 1831 o imóvel foi vendido a \textit{Joaquim Ramos de Araújo}; e em 1858 o médico \textit{Antônio José Alves}, cirurgião e professor de patologia externa da Faculdade de Medicina e pai de \textit{Antônio de Castro Alves}, o ``poeta dos escravos'', adquiriu a propriedade e investiu grande parte de seus recursos para transformá-la em uma casa de saúde. Castro Alves, o poeta, chegou inclusive a dedicar ao solar seu poema ``A Boa Vista'', transcrito no \autoref{cap:boavista} (p. \pageref{cap:boavista}).

O intento de Antônio José Alves com a aquisição do imóvel terminou sendo cumprido: em agosto de 1869 o governo da Bahia comprou o imóvel, com base na Lei provincial nº 1.089, para a instalação de um hospital. Em 24 de junho de 1874, foi inaugurado no local o Azylo São João de Deus, com um hospital, sob a responsabilidade da Santa Casa da Misericórdia; esta última, em 1912, entregou o nosocômio ao governo da Bahia devido a problemas financeiros.

Outro ponto notável de natureza civil é composto por dois pontos distintos, quase que invariavelmente referidos em conjunto na documentação pesquisada: a \textit{armação do Saraiva} e a \textit{armação do Gregório}.

As duas armações são mencionadas no censo de 1757 como pequenos aglomerados populacionais no vasto ermo que era a freguesia \cite[p.~183]{castralmeida_ultramar_1908}.

Tudo indica que o ``Saraiva'' da primeira armação era \textit{Manoel Gonçalves Saraiva}, já conhecido pelos consertos que fez na capela da Pituba em 1673 \cite[p.~11]{ott_engenhos_1996}. A antiga sede desta armação foi aproveitada pelo Aeroclube da Bahia como sua primitiva sede \cite[p.~III-11, verso]{teixeira_doacoes_1978}, quando radicada no local onde foi erguido o Shopping Aeroclube e hoje se pretende construir o Parque dos Ventos. Pode-se dizer, com base neste fato, que esta armação talvez seja uma antecessora remota da atual colônia de pescadores da Boca do Rio.

Já a armação do Gregório é mencionada em 1757 num censo eclesiástico, como já visto, mas não foi possível encontrar o nome completo de seu proprietário na documentação pesquisada, nem foi possível situá-la no território da freguesia. Da mesma forma, não é possível conceber esta armação como sendo alguma antecessora ou sucessora da armação do Saraiva, pois encontramo-las mencionadas simultaneamente no mesmo documento. É possível arriscar a hipótese de que esta armação pudesse situar-se onde hoje se localiza o Jardim dos Namorados, e que ela está para a colônia de pescadores ali localizada como a armação do Saraiva está para a colônia de pescadores da Boca do Rio, mas dada a escassez de evidências mais sólidas não se pode ultrapassar o campo puramente hipotético.