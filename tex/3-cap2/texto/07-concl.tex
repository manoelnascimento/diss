\section{Conclusões}\label{sec:conccap2}

A formação do território de Brotas mostrou alguns dos agentes de sua produção, elementos de sua caracterização fundiária, uma introdução às atividades econômicas desenvolvidas em seus limites e outros usos do território no período que precedeu a Primeira República. É possível chegar a algumas conclusões a partir de uma visão de conjunto.

Ficou vívida a impressão de que, durante todo o século XIX, a freguesia de Brotas -- à exceção do Rio Vermelho, considerado no século XIX como um dos melhores arrabaldes da cidade -- era vista pelo restante da cidade como um lugar ermo, distante, mesmo perigoso. Além do já visto, outros elementos pontuais vistos durante a pesquisa ajudarão a consolidar este julgamento.

Através de edital do Senado da Câmara de 27 de janeiro de 1811, que estabeleceu cobradores para os quarenta e dois ``talhos'' de Salvador e os sete de seu termo (ou seja, dos distritos rurais), temos noticia da existencia em Brotas de um destes pontos de venda de carne, que contava então com seu respectivo cobrador; comparativamente, o Caminho Novo tinha quatro ``talhos'', o Taboão tinha oito, e o Sao Bento dezoito\footnote{\textbf{Idade d'Ouro do Brazil}, nº 19, 06 mar. 1811, pp. 2-3}.

Em 1876 uma comissão encarregada de arrecadar fundos para o Asylo de Mendicidade localizado em Brotas veio a público agradecer pelos 572\$640 arrecadados, e informava já ter recolhido a quantia aos cofres da tesouraria municipal\footnote{\textbf{O Monitor}, 11 jul. 1876, p. 2}

O jornal \textbf{A Religião}, autoproclamado ``órgão da Igreja Catholica da Bahia'' e publicado sob os auspicios do arcebispo Luiz Antonio dos Santos, deixou de noticiar o movimento eclesiástico da freguesia de Brotas em 12 de junho de 1887 ``por ser quasi nenhum''.

Curiosa disposição legal dispensava funcionários do Tesouro Provincial de comparecer ao trabalho em seu horário regulamentar se morassem nas freguesias da ``Penha, Mares, Victoria, Santo Antonio e Brotas'' e obtivessem permissão de seus chefes para não comparecer diariamente ao trabalho (!), ``salvo porem o primeiro dia util de cada semana, em que ficarão sujeitos à regra geral''\footnote{\textbf{O Monitor}, 29 set. 1877, p. 1.}.

Brotas ainda era, no final do século XIX, lugar de sedução de menores:

\begin{citacao}
Apresentou-se esta manha ao sr. subdelegado da freguezia de Brotas uma senhora, moradora ao Sangradouro, queixando-se de um tal Mattos que lhe raptara uma sua filha menor, cujo nome ignoramos. Aquela autoridade, procedendo as respectivas diligencias, conseguiu descobrir o lugar onde se achava depositada a referida menor e trata de capturar o sedutor\footnote{ \textbf{Diario de Noticias}, ano 7, nº 23, 3 out. 1881, p. 1.}.
\end{citacao}

A edição de 25 de abril de 1880 do semanário \textbf{A Gargalhada} denunciava

\begin{citacao}
que o fiscal em exercicio na freguesia de Brotas, ao passo em que nao ve o estado das ruas, occupa-se em encher a estaçao de meninos e velhos, conductores de carroças, deixando impunes os valentoes e trampolineiros [\dots].
\end{citacao}

A iluminação pública, mesmo no distrito urbano da freguesia, era deficiente, como indica reclamação de moradores datada de 6 de fevereiro de 1881; ``grande parte da estrada das Pitangueiras até o largo do Paranhos'' estava às escuras, apesar de serem necessários apenas ``quatro ou cinco lampeões'' para resolver o problema. Os moradores, cujos prédios estavam ``sujeitos ao imposto da decima urbana'', reclamavam principalmente dos inconvenientes causados a seu trânsito por esta estrada durante o inverno, ``privada como também está de calçamento''\footnote{\textbf{O Monitor}, 6 fev. 1881, p. 1.}.

E assim chegou Brotas à república: um espaço marcadamente rural, com pouquíssimos serviços caracteristicamente urbanos, baixa implementação de infraestruturas, mas com progressiva ocupação de certas áreas -- em especial seu primeiro distrito, mais próximo ao núcleo urbanizado de Salvador -- por um misto de funcionários públicos de médio escalão, artífices e profissionais liberais, a indicar tendência de urbanização. Resume parcialmente a situação o pronunciamento de um deputado em 1887:

\begin{citacao}
Sr. presidente, a freguezia de Brotas é talvez a unica suburbana que está obrigada a pagar o imposto da decima urbana. Alli pagam esta contribuição verdadeiras casas de terra de estuque e ainda cobertas de palha.

Me parece, sr. presidente, que esta contribuição que alli se cobra é vexatória para aquella população pobre, cujos unicos recursos de vida são as choupanas em que se abrigam depois do trabalho quotidiano.

Muitas d'estas casas são edificadas em porções de terras que se dão de favor, outras são não habitaveis absolutamente, são destinadas a deposito de ferramentas, de utensilios de lavouras; e mesmo assim são arroladas e pagam decima urbana. Eu ainda hontem tive ensejo de examinar e de ver que são pessoas que não podem levantar a voz, porque nem ao menos tem o direito de votar [\dots] \cite[p.~335]{bahia_assembleia_1887}.
\end{citacao}

O ilustríssimo senhor deputado A. Bahia, com sua fala tocante, resumiu bem a situação dos moradores do segundo distrito -- mas em nome deles quis eximir da décima urbana também os moradores do primeiro distrito, este sim em franco processo de urbanização e com casas muito bem aprumadas.

No que diz respeito aos agentes de produção do espaço urbano, verificou-se a absoluta predominância dos \textit{latifundiários} neste processo durante todo o período que vai até o século XIX. Há entre eles, em especial na última metade do século XIX, a concentração em alguns pólos: 

\begin{itemize}
\item A \textit{casa de Nisa} e seu sucessor, \textit{Tomás da Silva Paranhos}, dono de vasto latifúndio;
\item O coronel \textit{João Ladislau de Figueiredo e Melo} e seus descendentes, que no conjunto assenhorearam-se das terras hoje integrantes dos bairros de Amaralina, Campinas de Brotas, parte da Pituba, do Nordeste de Amaralina e de Santa Cruz; 
\item O tenente-coronel, comerciante e armador \textit{Manuel Ignacio da Cunha e Meneses}, visconde do Rio Vermelho, e seus colaterais de descendentes, sesmeiros de vastas áreas do Rio Vermelho, Amaralina e Pituba, além de alguns terrenos na Estrada de Brotas;
\item O \textit{Recolhimento dos Perdões}, senhor da Quinta das Beatas;
\item O historiador \textit{José Álvares (ou Alves) do Amaral} e seus descendentes, que por casamentos com descendentes do visconde do Rio Vermelho e do coronel João Ladislau de Figueiredo e Melo, bem como por aquisições posteriores, assenhorearam-se da fazenda Alagoa;
\item O comerciante português \textit{Manoel de Castro Neves}, senhor das terras que deram origem à vizinhança que leva ainda hoje seu sobrenome.
\end{itemize}

O governo provincial interferiu na produção do espaço urbano apenas da década de 1870 em diante, e sua atuação foi fundamental para a instalação dos primeiros equipamentos urbanos da freguesia, equipamentos estes geridos por empresas nacionais e por comerciantes locais que se aventuraram na prestação de serviços públicos como transporte e iluminação pública (cf. \autoref{subsec:1.4.3}, p. \pageref{subsec:1.4.3}).