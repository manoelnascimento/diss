\section{Conclusões}

A formação do território de Brotas mostrou alguns dos agentes de sua produção, elementos de sua caracterização fundiária, uma introdução às atividades econômicas desenvolvidas em seus limites e outros usos do território no período que precedeu a Primeira República. É possível chegar a algumas conclusões a partir de uma visão de conjunto.

deixou vívida impressão de que, durante todo o século XIX, a freguesia de Brotas -- à exceção do Rio Vermelho, considerado no século XIX como um dos melhores arrabaldes da cidade -- era vista pelo restante da cidade como um lugar ermo, distante, mesmo perigoso. Além do já visto, outros elementos pontuais vistos durante a pesquisa ajudarão a consolidar este julgamento.

Através de edital do Senado da Câmara de 27 de janeiro de 1811, que estabeleceu cobradores para os quarenta e dois ``talhos'' de Salvador e os sete de seu termo (ou seja, dos distritos rurais), temos noticia da existencia em Brotas de um destes pontos de venda de carne, que contava então com seu respectivo cobrador; comparativamente, o Caminho Novo tinha quatro ``talhos'', o Taboao tinha oito, e o Sao Bento dezoito\footnote{\textbf{Idade d'Ouro do Brazil}, nº 19, 06 mar. 1811, pp. 2-3}.

Dois africanos que moravam no caminho de Brotas para o Rio Vermelho foram encontrados mortos, e a policia prendeu um crioulo de mais de 60 anos como suspeito do latrocínio \footnote{\textbf{Jornal da Bahia}, ano XXII, número ilegível, 10 mar. 1875, p. 2}.

Em 1876 uma comissão encarregada de arrecadar fundos para o Asylo de Mendicidade localizado em Brotas veio a público agradecer pelos 572\$640 arrecadados, e informava já ter recolhido a quantia aos cofres da tesouraria municipal\footnote{\textbf{O Monitor}, 11 jul. 1876, p. 2}

O jornal \textbf{A Religião}, autoproclamado ``órgão da Igreja Catholica da Bahia'' e publicado sob os auspicios do arcebispo Luiz Antonio dos Santos, deixou de noticiar o movimento eclesiástico da freguesia de Brotas em 12 de junho de 1887 ``por ser quasi nenhum''.

Curiosa disposição legal dispensava funcionários do Tesouro Provincial de comparecer ao trabalho em seu horário regulamentar se morassem nas freguesias da ``Penha, Mares, Victoria, Santo Antonio e Brotas'' e obtivessem permissão de seus chefes para não comparecer diariamente ao trabalho (!), ``salvo porem o primeiro dia util de cada semana, em que ficarão sujeitos à regra geral''\footnote{\textbf{O Monitor}, 29 set. 1877, p. 1.}.

Brotas ainda era, no final do século XIX, lugar de sedução de menores:

\begin{citacao}
Apresentou-se esta manha ao sr. subdelegado da freguezia de Brotas uma senhora, moradora ao Sangradouro, queixando-se de um tal Mattos que lhe raptara uma sua filha menor, cujo nome ignoramos. Aquela autoridade, procedendo as respectivas diligencias, conseguiu descobrir o lugar onde se achava depositada a referida menor e trata de capturar o sedutor\footnote{ \textbf{Diario de Noticias}, ano 7, nº 23, 3 out. 1881, p. 1.}.
\end{citacao}

A edição de 25 de abril de 1880 do semanário \textbf{A Gargalhada} denunciava

\begin{citacao}
que o fiscal em exercicio na freguesia de Brotas, ao passo em que nao ve o estado das ruas, occupa-se em encher a estaçao de meninos e velhos, conductores de carroças, deixando impunes os valentoes e trampolineiros [\dots].
\end{citacao}

A iluminação pública, mesmo no distrito urbano da freguesia, era deficiente, como indica reclamação de moradores datada de 6 de fevereiro de 1881; ``grande parte da estrada das Pitangueiras até o largo do Paranhos'' estava às escuras, apesar de serem necessários apenas ``quatro ou cinco lampeões'' para resolver o problema. Os moradores, cujos prédios estavam ``sujeitos ao imposto da decima urbana'', reclamavam principalmente dos inconvenientes causados a seu trânsito por esta estrada durante o inverno, ``privada como também está de calçamento''\footnote{\textbf{O Monitor}, 6 fev. 1881, p. 1.}.

E assim chegou Brotas à república: um espaço marcadamente rural, com pouquíssimos serviços caracteristicamente urbanos, baixa implementação de infraestruturas, mas com progressiva ocupação de certas áreas -- em especial seu primeiro distrito, mais próximo ao núcleo urbanizado de Salvador -- por um misto de funcionários públicos de médio escalão, artífices e profissionais liberais, a indicar tendência de urbanização.