\section{\(\dots\)e Brotas chega à República}\label{sec:conccap2}

A formação do território de Brotas mostrou alguns dos agentes de sua produção, elementos de sua caracterização fundiária, uma introdução às atividades econômicas desenvolvidas em seus limites e outros usos do território no período que precedeu a Primeira República. É possível chegar a algumas conclusões a partir de uma visão de conjunto.

Ficou vívida a impressão de que, durante todo o século XIX, a freguesia de Brotas --- à exceção do Rio Vermelho, considerado no século XIX como um dos melhores arrabaldes da cidade --- era vista pelo restante da cidade como um lugar ermo, distante, mesmo perigoso. Além do já visto, outros elementos pontuais vistos durante a pesquisa ajudarão a consolidar este julgamento.

Através de edital do Senado da Câmara de 27 de janeiro de 1811, que estabeleceu cobradores para os quarenta e dois ``talhos'' de Salvador e os sete de seu termo (ou seja, dos distritos rurais), temos noticia da existencia em Brotas de um destes pontos de venda de carne, que contava então com seu respectivo cobrador; comparativamente, o Caminho Novo tinha quatro ``talhos'', o Taboão tinha oito, e o Sao Bento dezoito\footnote{\textbf{Idade d'Ouro do Brazil}, nº 19, 06 mar. 1811, pp. 2-3}.

O jornal \textbf{A Religião}, autoproclamado ``órgão da Igreja Catholica da Bahia'' e publicado sob os auspicios do arcebispo Luiz Antonio dos Santos, deixou de noticiar o movimento eclesiástico da freguesia de Brotas em 12 de junho de 1887 ``por ser quasi nenhum''.

Curiosa disposição legal dispensava funcionários do Tesouro Provincial de comparecer ao trabalho em seu horário regulamentar se morassem nas freguesias da ``Penha, Mares, Victoria, Santo Antonio e Brotas'' e obtivessem permissão de seus chefes para não comparecer diariamente ao trabalho (!), ``salvo porem o primeiro dia util de cada semana, em que ficarão sujeitos à regra geral''\footnote{\textbf{O Monitor}, 29 set. 1877, p. 1.}.

Brotas ainda era, no final do século XIX, lugar de sedução de menores:

\begin{citacao}
Apresentou-se esta manha ao sr. subdelegado da freguezia de Brotas uma senhora, moradora ao Sangradouro, queixando-se de um tal Mattos que lhe raptara uma sua filha menor, cujo nome ignoramos. Aquela autoridade, procedendo as respectivas diligencias, conseguiu descobrir o lugar onde se achava depositada a referida menor e trata de capturar o sedutor\footnote{ \textbf{Diario de Noticias}, ano 7, nº 23, 3 out. 1881, p. 1.}.
\end{citacao}

A edição de 25 de abril de 1880 do semanário \textbf{A Gargalhada} denunciava

\begin{citacao}
que o fiscal em exercicio na freguesia de Brotas, ao passo em que nao ve o estado das ruas, occupa-se em encher a estaçao de meninos e velhos, conductores de carroças, deixando impunes os valentoes e trampolineiros [\dots].
\end{citacao}

A iluminação pública, mesmo no distrito urbano da freguesia, era deficiente, como indica reclamação de moradores datada de 6 de fevereiro de 1881; ``grande parte da estrada das Pitangueiras até o largo do Paranhos'' estava às escuras, apesar de serem necessários apenas ``quatro ou cinco lampeões'' para resolver o problema. Os moradores, cujos prédios estavam ``sujeitos ao imposto da decima urbana'', reclamavam principalmente dos inconvenientes causados a seu trânsito por esta estrada durante o inverno, ``privada como também está de calçamento''\footnote{\textbf{O Monitor}, 6 fev. 1881, p. 1.}.

E assim chegou Brotas à república: um espaço marcadamente rural, com pouquíssimos serviços caracteristicamente urbanos, baixa implementação de infraestruturas, mas com progressiva ocupação de certas áreas --- em especial seu primeiro distrito, mais próximo ao núcleo urbanizado de Salvador --- por um misto de funcionários públicos de médio escalão, artífices e profissionais liberais, a indicar tendência de urbanização. Resume parcialmente a situação o pronunciamento de um deputado em 1887:

\begin{citacao}
Sr. presidente, a freguezia de Brotas é talvez a unica suburbana que está obrigada a pagar o imposto da decima urbana. Alli pagam esta contribuição verdadeiras casas de terra de estuque e ainda cobertas de palha.

Me parece, sr. presidente, que esta contribuição que alli se cobra é vexatória para aquella população pobre, cujos unicos recursos de vida são as choupanas em que se abrigam depois do trabalho quotidiano.

Muitas d'estas casas são edificadas em porções de terras que se dão de favor, outras são não habitaveis absolutamente, são destinadas a deposito de ferramentas, de utensilios de lavouras; e mesmo assim são arroladas e pagam decima urbana. Eu ainda hontem tive ensejo de examinar e de ver que são pessoas que não podem levantar a voz, porque nem ao menos tem o direito de votar [\dots] \cite[p.~335]{bahia_assembleia_1887}.
\end{citacao}

O ilustríssimo senhor deputado A. Bahia, com sua fala tocante, resumiu bem a situação dos moradores do segundo distrito --- mas em nome deles quis eximir da décima urbana também os moradores do primeiro distrito, este sim em franco processo de urbanização e com casas muito bem aprumadas.

No que diz respeito aos agentes de produção do espaço urbano, verificou-se a absoluta predominância dos \textit{latifundiários} neste processo durante todo o período que vai até o século XIX. Há entre eles, em especial na última metade do século XIX, a concentração em alguns pólos: 

\begin{itemize}
\item A \textit{casa de Nisa} e seu sucessor, \textit{Tomás da Silva Paranhos}, dono de vasto latifúndio;
\item O coronel \textit{João Ladislau de Figueiredo e Melo} e seus descendentes, que no conjunto assenhorearam-se das terras hoje integrantes dos bairros de Amaralina, Campinas de Brotas, parte da Pituba, do Nordeste de Amaralina e de Santa Cruz; 
\item O tenente-coronel, comerciante e armador \textit{Manuel Ignacio da Cunha e Meneses}, visconde do Rio Vermelho, e seus colaterais de descendentes, sesmeiros de vastas áreas do Rio Vermelho, Amaralina e Pituba, além de alguns terrenos na Estrada de Brotas;
\item O \textit{Recolhimento dos Perdões}, senhor da Quinta das Beatas;
\item O historiador \textit{José Álvares (ou Alves) do Amaral} e seus descendentes, que por casamentos com descendentes do visconde do Rio Vermelho e do coronel João Ladislau de Figueiredo e Melo, bem como por aquisições posteriores, assenhorearam-se da fazenda Alagoa;
\item O comerciante português \textit{Manoel de Castro Neves}, senhor das terras que deram origem à vizinhança que leva ainda hoje seu sobrenome.
\end{itemize}

O governo provincial interferiu na produção do espaço urbano apenas da década de 1870 em diante, e sua atuação foi fundamental para a instalação dos primeiros equipamentos urbanos da freguesia, equipamentos estes geridos por empresas nacionais e por comerciantes locais que se aventuraram na prestação de serviços públicos como transporte e iluminação pública (cf. \autoref{subsec:1.4.3}, p. \pageref{subsec:1.4.3}).

Com todas as informações anteriores, a fronteira do distrito de Brotas fica mais precisamente delimitada que na descrição reproduzida e divulgada por \citeauthoronline{NASCIMENTO2007}. Retornemos a ela:

\begin{citacao}
A freguesia de Nossa Senhora de Brotas foi criada pelo arcebispo D. Sebastião Monteiro da Vide, em 1718, sendo a seguinte sua demarcação extrema com outras freguesias, no século XIX: com Santo Antônio Além do Carmo pela Estrada Nova, começando pela roça do comendador Barros Reis, vindo até a Fonte Nova no Dique, onde fazia diferentes limites com Santana e São Pedro. Daí, pela estrada Dois de Julho, seguia até a ponta da Mariquita, de onde se espraiava costeando a lagoa da Pituba, até Armação e o Rio das Pedras, quando se dividia com a freguesia de Itapuã, suburbana da cidade. Seguia a freguesia de Brotas até o Engenho da Bolandeira, onde novamente fazia divisa com Itapuã e com a freguesia de Santo Antônio Além do Carmo. Limitava-se com a Vitória na Mariquita \cite[p.~58]{NASCIMENTO2007}.
\end{citacao}

Compare-se-a, agora, com a descrição a seguir, onde pontos notáveis do século XIX serão empregues junto a outros da atualidade para facilitar a visualização:

\begin{itemize}
\item Adota-se arbitrariamente como marco inicial, entre aqueles que se verificou integrarem o território da freguesia, o pé da Ladeira dos Galés, onde Brotas fazia fronteira com a freguesia de Sant'Anna.
\item Partindo deste ponto, a freguesia de Brotas margeia o Dique do Tororó pelo lado do Engenho Velho de Brotas, limitando-se com a freguesia de Sant'Anna até o ponto onde o dique dá origem ao rio Lucaia.
\item A partir daí a fronteira de Brotas com a freguesia da Vitória margeia o rio Lucaia até sua foz, no Rio Vermelho.
\item Daí em diante Brotas limita-se com o mar até a foz do rio das Pedras, na Boca do Rio, que separava Brotas da freguesia suburbana de Itapuã.
\item A fronteira do distrito de Brotas segue margeando o rio das Pedras, sobe pelo rio Pituaçu, parte dele para o rio Cachoeirinha e segue-o até onde, no mapa da \citeonline{salvador_mapa_1969}, ele encontra com a ``estrada da Cachoeirinha''\footnote{Uma rua com este nome ainda existe, embora com traçado diferente daquele do mapa. O traçado original parece ter sido superposto, em parte, pelas atuais ruas Teódulo de Albuquerque e Anísio Melhor até o ponto qm que tocam na avenida Edgard Santos; daí em diante a estrada parece ter sido soterrada pela construção do conjunto Doron.}.
\item O limite do distrito seguia pela ``estrada da Cachoeirinha'' até encontrar o ``beco do Catarro''\footnote{Esta via de nome curioso parece, no mapa da \citeonline{salvador_mapa_1969}, seguir o mesmo traçado das atuais ruas José Ferreira e Leopoldo Tantu até o ponto onde esta última encontra-se com a rua Cidália Menezes; o ponto em que este beco encontrava-se com a estrada da Cachoeirinha parece ter sido soterrado pela malha viária do conjunto Doron.}, por onde subia até encontrar a ``estrada do Saboeiro''\footnote{A antiga estrada do Saboeiro é a atual rua Silveira Martins, que em tempos mais antigos parece ter seguido pela atual avenida Jorge Amado até dar na atual praia dos Artistas \cite{souza_guia_1935}. O mapa da \citeonline{salvador_mapa_1969} indica que, ao ser cortada pela avenida Luiz Vianna Filho, dividiu-se em duas: a estrada do Saboeiro propriamente dita, correspondente à atual rua Silveira Martins, e a avenida Vale do Cascão, correspondente à atual avenida Jorge Amado.}.
\item O limite de Brotas passa a ficar um tanto nebuloso a partir daqui  pois sabe-se que o Cabula pertenceu ao distrito de Santo Antônio e a estrada do Saboeiro circunda o Cabula inteiro. Parece mais prudente a hipótese pela qual o limite, ao invés de subir a ``estrada do Saboeiro'', simplesmente a atravessou, seguindo adiante pelo que o mapa da \citeonline{salvador_mapa_1969} indicava ser o ``beco da Coruja''\footnote{O logradouro ainda existe e mantém o mesmo nome: inicia-se na rua Silveira Martins e segue por dentro da mata espessa até encontrar com o muro do 19º Batalhão de Caçadores.}.
\item Como o que hoje é a parte do leito do rio Camarajipe que vai de Pernambués à sua atual foz no Costa Azul era, no período estudado, o leito do \textit{rio Pernambués}\footnote{Parte significativa da atual avenida Luiz Eduardo Magalhães foi construída sobre o vale escavado pelo rio Pernambués, segundo \cite{santos_aguas_2010}.}, é factível seguir desenhando os limites da freguesia de Brotas traçando uma linha que vá do ``beco da Coruja'' até o rio Pernambués, depois construir uma linha imaginária que ligue o Pernambués ao Camarajipe, e daí em diante seguir Camarajipe acima, rumo à região onde se situam atualmente a rodoviária de Salvador e o Shopping da Bahia. Em todo este traçado Brotas confronta-se com a freguesia do Santo Antônio.
\item A partir daí é possível, como se vê em \citeonline{weyll_mappa_1851}, margear o Camarajipe até que ele encontre com a rua da Vala na altura da atual rótula do Abacaxi, novamente confrontando Brotas e Santo Antônio.
\item A rua da Vala limita as freguesias de Brotas e Santo Antônio até o ponto onde se encontra com a atual rua Djalma Dutra.
\item Deste ponto em diante a rua Djalma Dutra segue delimitando as freguesias de Brotas e Santo Antônio até encerrar-se no pé da ladeira dos Galés, ponto inicial desta descrição.
\end{itemize}

É fora de questão que os dois únicos historiadores a tratar com algum detalhe do território da freguesia de Brotas \cite{NASCIMENTO2007,ott_engenhos_1996} foram induzidos a erro pela falta de referências precisas nos documentos em que se basearam, e a descrição acima, reconhecendo seus esforços, pretende complementá-los. \citeonline{ott_engenhos_1996}, por sua vez, situa o engenho Santo Antônio na freguesia de Itapuã, quando o \textbf{Livro Eclesial de Registro de Terras da Freguesia de Brotas} situa-o na freguesia de Brotas. \citeonline{NASCIMENTO2007} indica corretamente os limites da freguesia até chegar à atual estação Bolandeira --- mas o polígono por ela descrito não ``fecha'', talvez por faltarem na documentação consultada por ela a descrição dos limites \textit{interiores} da freguesia (ou seja, da Bolandeira até a Christiano Buys). Respeitados os méritos destes historiadores em tantas outras questões, as lacunas na descrição da freguesia de Brotas, ainda que representem o máximo esforço de pesquisa até o momento em que produziram suas respectivas obras, refletem o enorme desconhecimento, persistente em historiadores até o dia de hoje, sobre a geografia das áreas soteropolitanas mais afastadas da malha urbana consolidada. Trata-se de lacuna a preencher com urgência, tanto mais por tratar-se de áreas que se supunha ``vazias'' até a expansão urbana dos anos 1970--1990 --- mas onde desenvolvia-se extensa atividade agropecuária e pesqueira, além de abrigar um sem-número de quilombos que somente hoje vão sendo desvendados por historiadores, acadêmicos ou não, nascidos e criados nos bairros que os sucederam.

Diante de tantas dificuldades, é grande a tentação de reconstruir o território antigo do distrito de Brotas pela simples justaposição dos atuais subdistritos de Brotas e Amaralina --- desmembrado de Brotas pela Lei Municipal nº 502, de 12 de agosto de 1954. Bastaria georreferenciar o mapa de subdistritos produzido pela \apudonline{conder_subdistritos_1984}{VASCONCELOS2002}, sobrepô-lo a qualquer dos tantos mapas digitais hoje disponíveis ao público e sair elencando os limites. Simples como seja, o procedimento induz a erro. Este método ignora completamente os cuidados tantas vezes ressaltados por \citeonline{NASCIMENTO2007} ao analisar as freguesias soteropolitanas, pois elas tinham seus limites \textit{constantemente alterados} já no século XIX. Mesmo assim, a comparação é válida para que se perceba as alterações nos limites distritais de Brotas: de acordo com o mapa da \apudonline{conder_subdistritos_1984}{VASCONCELOS2002}, a justaposição dos territórios de Brotas e Amaralina resultaria um território demarcado a leste pela avenida Vasco da Gama e pela rua Djalma Dutra; a norte, pela avenida Heitor Dias, pelo trecho da avenida Barros Reis que vai até a rua Christiano Buys; daí para o leste pela rua Thomaz Gonzaga e pela estrada do Curralinho, até encontrar o rio das Pedras na altura da estação Bolandeira; a leste, pelo rio das Pedras, do trecho que vai da estação Bolandeira até a foz; do leste ao sul pelo mar, do trecho que vai da foz do rio das Pedras até a foz do rio Lucaia. Parece assim, à primeira vista, que Pernambués inteiro fazia parte de Brotas, e que as áreas da bacia dos rios das Pedras e Pituaçu. Fontes mais próximas do período estudado, entretanto \cite{souza_guia_1935}, indicam que Pernambués pertencia ao distrito de \textit{Santo Antônio}, assim como o Cabula e o Saboeiro.