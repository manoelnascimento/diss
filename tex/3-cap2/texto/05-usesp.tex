\section{Usos do espaço}\label{sec:2.4}

Até agora, nas entrelinhas do já exposto, foi possível perceber diversos usos para o espaço da freguesia: desde a simples moradia, agricultura e pesca até a especulação imobiliária do Recolhimento dos Perdões. Há, além destes, outros \textit{usos menos explícitos do espaço da freguesia} que importa destacar.

\subsection{Prisão}

É verdade que a proposta não chegou a ser concretizada, mas, diante da necessidade ``inadiável'' de retirar a cadeia do palácio da Câmara, foi cogitado movê-la para a Casa da Pólvora, no Matatu. A ideia, recusada pela primeira vez em 1809, ressurgiu em 1830 e 1831, sendo recusada nas duas vezes; mesmo a cessão do forte do Barbalho para este fim, em negociação, não acalmou o espírito de alguns vereadores, que argumentaram não haver espaço suficiente na fortaleza; ao fim e ao cabo, foi escolhido o forte de Santo Antônio, quase ao mesmo tempo em que a Câmara perdeu para a Província a competência de abrigar os presos \cite[pp.~304-305]{ruy_camara_1953}.

\subsection{Refúgio}

Sendo jornal de portugueses, é de estranhar, à primeira vista de um leitor hodierno desavisado, que \textbf{Idade d'Ouro do Brazil} tomasse posição favorável a uma monarquia constitucional no Brasil. A posição só se explica pelo fato de que o Reino Unido de Brasil, Portugal e Algarves passara, desde a Revolução Liberal do Porto (1820) até a promulgação de sua primeira constituição (1822), do absolutismo à monarquia constitucional; as lutas pela independência do Brasil representavam, para os portugueses interessados na regressão do Brasil ao \textit{status} de colônia, uma ruptura do pacto constituinte. Parecia ser este o caso dos redatores e editores de \textbf{Idade d'Ouro do Brazil}: seus conselhos e invectivas visavam manter a ``ordem constitucional'', o que significava tomar partido pela monarquia portuguesa; desta posição, atacavam tanto absolutistas quanto republicanos, e nutriam verdadeiro ódio aos independentistas brasileiros. Entre perorações a um lado e queixas ao outro, lá iam os redatores do periódico servir de inteligência às tropas portuguesas ainda estacionadas na Bahia, em dezembro de 1822:

\begin{citacao}
Os rebeldes armados, que andam para as bandas de Brotas, têm armado suas traições aos nossos soldados, que vão de manhã à descoberta. Parecia-nos muito fácil armar uma cilada aos tais traidores. Que o diga quem conhece bem o caminho e os desvios das Brotas. [/dots] Portugal não tarda com o remédio. Juízo, leis e força. No entanto estamos seguros na cidade\footnote{\textbf{Idade d'Ouro do Brazil}, nº 102, 20 dez. 1822, p. 2}.
\end{citacao}

\subsection{Aquilombamento, refugio de escravos fugidos e palco de ``ajuntamentos'' de africanos amotinados}

Não é de negligenciar o uso das matas de Brotas como refúgio dos que escapavam à escravidão. Quilombos, fugas, ``seduções'', motins, batuques, candomblés, acoitamentos, nada disso é estranho à história da freguesia.

Já em 1814 

Em 1845, dois motins de escravos ocorridos por volta de maio deixaram os soteropolitanos brancos em polvorosa, falando-se inclusive em ``três caixas de depósitos e fundos africanos'', tudo isto -- denunciava o \textbf{O Guaycuru} indignadíssimo -- sob o consentimento do presidente da província, XXXXX, que ``continua a consentir, melhor diremos continha a autorizar'' o ajuntamento de ``turmas de  600, 800 e mil para dançar nas praças públicas, e até nos arrabaldes mais solitários vizinhos à cidade'', Brotas entre eles\footnote{\textbf{O Guaycuru}, ano 3, nº 99, 10 jun. 1845, p. 390}.

Veja-se, por exemplo, o caso de Fiel, que desde 9 de junho de 1866 evadiu-se do engenho das Brotas, em Santo Amaro, e segundo anuncio jornalistico ``tem sido visto no Sangradouro, Cabula, Matatu''\footnote{\textbf{O Alabama}, ano 4, série 8, nº 73, 14 jul 1866, p. 4}

\subsection{Palco de festas religiosas}

Em 1838 o calendário da arquidiocese de Salvador indicou que a festa do Rosário era comemorada na matriz da freguesia de Brotas em 27 de dezembro \cite[p.~44]{arcebis_diario_1837}, e em 1842 que a do Santíssimo Sacramento na mesma freguesia era comemorada em 26 de dezembro \cite[p.~55]{arcebis_folhinha_1841}. Já um almanaque para o ano de 1878 somou a estas festividades a de Nossa Senhora do Rosario, no dia 1º de janeiro, e a de Nossa Senhora de Brotas no dia 27 de dezembro \cite[pp.~53,~109]{macosta_almana_1877}.

Em 1871 o cemitério da freguesia de Brotas já demandava ser alargado\footnote{Jornal da Bahia, ano XIX, nº 5.446, 22 set. 1871, p. 1}.

Em 1877 era ainda Ernesto de Oliveira Valle o pároco da freguesia \cite[p.~176]{macosta_almana_1877}.

Ainda em 1877 funcionava o velho esquema dos sinais de incendio, detalhados num regulamento municipal de 11 de novembro de 1853. Em caso de incendio, o sinal de fogo seria dado pelo toque do maior sino da igreja que primeiro dele soubesse, do maior sino da matriz da freguesia em que se manifestasse o incendio e do maior sino das demais igrejas que dele tivessem notícia; este toque seria de trinta badaladas apressadas, depois das quais seriam dada breve pausa seguida por toques em número convencionado por freguesia: um para a Se, dois para Sao Pedro, tres para Santana, quatro para Conceiçao, cinco para Pilar, seis para Passo, sete para Santo Antônio, oito para Vitória, nove para Brotas, dez para Penha e onze para Mares. Enquanto durasse o incendio, esta sinalizaçao seria repetida a cada quatro minutos; ao debelar-se o sinistro, a igreja que primeiro deu o sinal soaria um breve repique para indicar o fim da emergencia \cite[pp.~192-193]{macosta_almana_1877}.

\subsection{Palco de ajuntamentos africanos}