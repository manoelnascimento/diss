\section{Usos do espaço}\label{sec:2.4}

Até agora, nas entrelinhas do já exposto, foi possível perceber diversos usos para o espaço da freguesia: desde a simples moradia, agricultura e pesca até a especulação imobiliária do \index{Quinta das Beatas!Recolhimento dos Perdões}Recolhimento dos Perdões. Há, além destes, outros \textit{usos menos explícitos do espaço da freguesia} que importa destacar.

\subsection{Usos pontuais}

O primeiro caso é o da instalação de uma \textit{prisão} em Brotas. É verdade que a proposta não chegou a ser concretizada, mas, diante da necessidade ``inadiável'' de retirar a cadeia do palácio da Câmara, foi cogitado movê-la para a \index{\index{Matatu}Matatu!Casa da Pólvora}Casa da Pólvora, no \index{Matatu}Matatu. A ideia, recusada pela primeira vez em 1809, ressurgiu em 1830 e 1831, sendo recusada nas duas vezes; mesmo a cessão do forte do Barbalho para este fim, em negociação, não acalmou o espírito de alguns vereadores, que argumentaram não haver espaço suficiente na fortaleza; ao fim e ao cabo, foi escolhido o forte de Santo Antônio para acolher os presos, quase ao mesmo tempo em que a Câmara perdeu para a Província a competência de abrigá-los \cite[pp.~304-305]{ruy_camara_1953}.

O segundo caso é o de \textit{valhacouto de ``rebeldes''}. Nas lutas pela independência havidas entre 1822 e 1823 na Bahia foi exatamente este o caso. 

Durante os conflitos da independência, em setembro de 1822, a desembocadura da Estrada de Brotas foi guardada pelo 1º Batalhão Constitucional de Lisboa, por força de notícias de que batalhões milicianos da \index{Garcia D'Ávila!Torre dos Garcia D'Ávila}Torre dos Garcia D'Ávila estariam já em Brotas, a légua e meia de Salvador\footnote{\textbf{O Espelho}, nº 83, 03 set. 1822, p. 1; \textit{idem}, nº 110, 06 dez. 1822, p. 2}. Já no mês seguinte foram publicadas noticias de escaramuças entre tropas portuguesas e brasileiras no \index{Rio Vermelho}Rio Vermelho e em Brotas\footnote{\textbf{O Espelho}, nº 98, 25 out. 1822, p. 1}, e em dezembro tropas portuguesas incendiaram a casa de uma fazenda chamada Torre, homônima à dos \index{Garcia D'Ávila}Garcia D'Ávila, proxima daquela de um certo ``Machado da Boa Vista''\footnote{Trata-se, com quase absoluta certeza, de Manuel José Machado, proprietário do Solar Boa Vista.}, e fizeram o mesmo com outra fazenda chamada ``roça dos Mansos''\footnote{\textbf{O Espelho}, nº 8, 28 dez. 1822, p. 1}.

Sendo jornal de portugueses, é de estranhar, à primeira vista de um leitor moderno desavisado, que \textbf{Idade d'Ouro do Brazil} tomasse posição favorável a uma monarquia constitucional no Brasil. A posição só se explica pelo fato de que o Reino Unido de Brasil, Portugal e Algarves passara, desde a Revolução Liberal do Porto (1820) até a promulgação de sua primeira constituição (1822), do absolutismo à monarquia constitucional; as lutas pela independência do Brasil representavam, para os portugueses interessados na regressão do Brasil ao \textit{status} de colônia, uma ruptura do pacto constituinte. Parecia ser este o caso dos redatores e editores de \textbf{Idade d'Ouro do Brazil}: seus conselhos e invectivas visavam manter a ``ordem constitucional'', o que significava tomar partido pela monarquia portuguesa; desta posição, atacavam tanto absolutistas quanto republicanos, e nutriam verdadeiro ódio aos independentistas brasileiros. Entre perorações a um lado e queixas ao outro, lá iam os redatores do periódico servir de inteligência às tropas portuguesas ainda estacionadas na Bahia, em dezembro de 1822:

\begin{citacao}
Os rebeldes armados, que andam para as bandas de Brotas, têm armado suas traições aos nossos soldados, que vão de manhã à descoberta. Parecia-nos muito fácil armar uma cilada aos tais traidores. Que o diga quem conhece bem o caminho e os desvios das Brotas. [/dots] Portugal não tarda com o remédio. Juízo, leis e força. No entanto estamos seguros na cidade\footnote{\textbf{Idade d'Ouro do Brazil}, nº 102, 20 dez. 1822, p. 2}.
\end{citacao}

As tropas ``milicianas'' de ``rebeldes armados'' foram, muito provavelmente, aquelas comandadas por \textit{Felisberto Gomes Caldeira}, que em fevereiro de 1823 assentava posições na Cruz do Cosme e em Brotas; sabia-se também que \textit{José de Barros Reis} entrincheirara suas tropas na Fazenda Grande do Retiro e em Campinas \cite[p.~248]{ruy_camara_1953}. Durante a Sabinada (1837-1838) Brotas foi novamente palco de combates \cite[p.~536]{ruy_politica_1949}.

Esses dois casos, entretanto, são de usos \textit{pontuais} do espaço. Houve usos mais \textit{contínuos} e significativos, ainda que intermitentes, dos quais foram selecionados dois casos -- um ilegal, outro legal.

\subsection{Refúgio contra a escravidão}\label{subsec:refugioescrav}

O primeiro uso destacado, ilegal porém legítimo, foi o uso das matas de Brotas como \textit{refúgio para os que escapavam à escravidão} e para \textit{práticas culturais e religiosas africanas proibidas por lei}. \index{quilombos}Quilombos, fugas, ``seduções'', motins, batuques, candomblés, acoitamentos, nada disso é estranho à história da freguesia.

Antes de prosseguir, uma contextualização necessária: as matas e pequenas herdades circunvizinhas a Salvador, nos segundos distritos das freguesias do Santo Antônio, da Vitória e de Brotas, assim como a integralidade do território das freguesias rurais, eram o lugar de escolha para negros em luta pela liberdade viverem livres e a seu modo, ao menos enquanto o braço da polícia e dos capitães-do-mato não os alcançava para arrastá-los de volta ao cativeiro. Ainda em 1885, três anos antes de decretado o fim da escravidão, este era um fato corriqueiro e vituperado pela imprensa conservadora de Santo Amaro da Purificação:

\begin{citacao}
\textbf{COMMUNICADO}

\textbf{Aos Srs. proprietarios de engenho}

Fazemos scientes que existem espalhados lá pelas roças da capital, principalmente n'aquelles sitios em paragens mais remotas e exquisitas, como Cabulla, Páo Miudo, fundos da Matança, estrada para a Boa Vista, Pitangueiras, Rio Vermelho, \&, \&, um crescido numero de escravos fugidos e que vivem alugados trabalhando por baixo preço para gozarem protecção.

Tambem por outros logares dos mais remotos das immediações d'aquella cidade existem muitas casas reunidas e com moradores que prestam auxilio em occasiões difficeis, em nome do falso abolicionismo.

Alguns ha, principalmente dos africanos livres, que se prestam a occultar os escravos com o fim de os libertar por baixo preço, depois de muito tempo, com parte de dinheiro por elles ganho e a outra parte que lhes emprestam, mettendo por medianeiros junto aos Srs. dos escravos, uns certos typos de procuradores, que vivem exclusivamente disso.

Os libertos por esse meio submettem-se, entretanto, ao mais ferrenho jugo o cruel dos captiveiros, porque ficam encerrados nas roças, como presos, trabalhando \textit{dia e noite} debaixo da maior vigilancia e inspecção deste mundo, para pagarem o principal e os juros de 2 e 4 por cento ao mez, e por uma forma que nunca se acaba, por ser \textit{pela conta do não chega}, isto é, por mais que amortizem o debito, estão sempre sugeitos ao premio do capital primitivo.

E os negros a tudo se sugeitam e chegam a morrer as vezes como entalados, só por amor da liberdade!\footnote{\textbf{O Popular}, serie XXVII, nº 524, 18 jun. 1885, pp. 2 e 3.}.
\end{citacao}

A notícia acima apresenta \textit{uma} entre tantas e quantas formas de luta pela liberdade encontrada pelos negros escravizados. Vistas as coisas com os olhos daqueles nascidos e criados numa sociedade que abomina a escravidão -- mas é hipócrita o suficiente para conviver com a exploração assalariada do trabalho de quem, por não controlar os meios de produção, não tem outra alternativa de sobrevivência -- o que espanta não é a sujeição dos negros a um \textit{regime de barracão}, à \textit{escravização pela dívida}; espanta, sim, creditar-se à fraude e à enganação aquilo que pode ter sido, ao menos em tese, a troca consciente de uma escravidão aberta e legalizada por uma liberdade precária, mas possibilitadora de outras estratégias de sobrevivência.

Neste contexto Brotas ganhava destaque como área de exercício da liberdade pelos negros escravizados. O \index{Matatu}Matatu era reconhecido pelos ``cidadãos de bem'' desde as primeiras décadas do século XIX, possivelmente mesmo antes, como uma área quilombola, assim como também o eram Cabula e Itapuã \cite[p.~377]{schwartz_1814_1996}. Já em 1814 o ``Corpo do Comércio e mais Cidadãos da Praça da Bahia'', aterrorizados pelo levante escravo de fevereiro do mesmo ano, escreveram ao rei João VI numa representação temerosa:

\begin{citacao}
A \index{Matatu!Casa da Pólvora}Casa da Pólvora, um dos pontos tão interessantes e perigosos que sempre em tempo dos Governos anteriores foi guardada por um numeroso piquete de oficial, desde que chegou este Governo até ao presente está reduzido a uma patrulha de oito soldados recrutas inertes e estão no centro de um mato que só é rodeado de quilombos de negros [\dots] \cite[pp.~103-106]{ott_formaet2_1957}.
\end{citacao}

Em 1843 o subdelegado de Brotas era publicamente admoestado pelos seguintes fatos:

\begin{citacao}
\dots sendo frequentes as reuniões ou batuques de africanos ou crioulos de ambos os sexos dentro da \index{Quinta das Beatas}Quinta das Beatas, muito se recomendava á S. S. que mediante providencia energica fizesse despersal-os, e se responsabilizasse o inspector de quarteirão respectivo se pontual e religiosamente não cumprisse as ordens de S. S.\footnote{\textbf{Correio Mercantil}, ano 10, nº 239, 4 nov. 1843, p. 1.}.
\end{citacao}

Viu-se na \autoref{subsec:matatubeatas} que estas ``reuniões ou batuques'', segundo a tradição oral \textit{ketu}, deveria referir-se a ``um cemitério angolano onde se realizava o culto de Tempo Kiamuilo'', ou mesmo aos ``cultos de Orixá Okô e dos ancestrais Babá Gunukô e sua esposa Abakô Laí, no local onde hoje se encontra a Avenida Bonocô'' \cite[pp.~373-374]{silveira_alaketo_2003}. Viu-se, de igual modo, como tudo indica terem sido majoritariamente libertos os muitos posseiros do Matatu, e que muito provavelmente constituíam uma rede de apoio e suporte à prática das religiões afrobrasileiras e às fugas. Admitindo-se como reais, ainda que com muitíssimas reservas, os temores expressos no documento de 1814 visto acima, pode-se adicionar mais um elemento: não apenas os posseiros do Matatu, ao que tudo indica, eram libertos, como é também muito provável que suas posses guardassem relação com os quilombos a que se referiu o documento. Parece mais próxima da verdade a hipótese levantada na \autoref{subsec:matatubeatas} (p. \pageref{subsec:matatubeatas}), de que os dois Matatus e mesmo as vizinhanças da Quinta das Beatas eram uma ``pequena África'' no território de Salvador.

Brotas era, de uma ponta a outra, ponto de quilombagem --- tanto assim que já em 1819 um furibundo Manoel Marques da Rocha e Queiroz mandava noticiar que

\begin{citacao}
\([\dots]\) he senhor de um pardo que está fugido ha 25 annos, o qual diz se chama \textit{Joaquim Xavier de Santa Anna}, sendo seu verdadeiro nome \textit{Joaquim Ferreira}; o referido escravo tem assistido no \textit{Rio das Pedras} em caminho da \textit{Itapoã}, em companhia de hum filho que tem, e já esteve prezo na Cadeia desta Cidade em Julho de 1816, sendo conduzido a ella por \textit{Luiz Henrique}, Cabo de ronda naquelle tempo do referido sitio \([\dots]\)\footnote{\textbf{Idade d'Ouro do Brazil}, nº 23, 16 mar. 1819, p. 04}.
\end{citacao}

Não se pode afirmar que Joaquim Xavier de Santa Anna houvesse fixado residência no rio das Pedras durante todos os vinte e cinco anos em que viveu livre; o que se pode afirmar é que em algum momento este homem resolveu ali fincar suas raízes, criar seu filho, viver. Ignora-se sua profissão, seus saberes, suas crenças --- sabe-se apenas que as últimas fronteiras do distrito de Brotas pareceram acolhedoras a suas práticas de liberdade. Quantos não terão feito o mesmo, tão distante da malha urbana consolidada era o rio das Pedras? Que outras quilombagens terão frutificado em suas matas ciliares?

Ainda em 1843 era preso José Manoel Pimentel, ``que alem de inculcar se inspetor do Sangradouro, para mediante captura de escravos extorquir dinheiro dos respectivos Srs., tem cometido furtos, e promovido desordens''\footnote{\textbf{Correio Mercantil}, ano X, nº 7, 10 jan. 1843, p. 1.}. Em 1845, dois motins de escravos ocorridos por volta de maio deixaram os soteropolitanos brancos em polvorosa, falando-se inclusive em ``três caixas de depósitos e fundos africanos'', tudo isto -- denunciava \textbf{O Guaycuru} indignadíssimo -- sob o consentimento do presidente da província, que ``continua a consentir, melhor diremos continha a autorizar'' o ajuntamento de ``turmas de  600, 800 e mil para dançar nas praças públicas, e até nos arrabaldes mais solitários vizinhos à cidade'', Brotas entre eles\footnote{\textbf{O Guaycuru}, ano 3, nº 99, 10 jun. 1845, p. 390}.

Não são poucos os casos, ainda, de negros que se libertavam do cativeiro fugindo para Brotas. Em toda a pesquisa de imprensa realizada encontram-se relatos os mais diversos acerca de negros evadidos que encontravam nas matas, roças e ermos de Brotas um refúgio para sua liberdade, por temporária que fosse. Veja-se, por exemplo, o caso de Vicente, ``preto de nação nagô'' fugido em 1839 e ainda perseguido pelo antigo senhor em 1845, que ``anda bem calçado e bem vestido, intitulando-se ser forro, e que já possui um moleque seu'', cujo paradeiro, segundo se anunciava, era uma roça no ``Cabulla, Brotas, Itapoan, Armação ou \index{Rio Vermelho}Rio Vermelho''\footnote{\textbf{O Mercantil}, 29 mar. 1845, p. 5.}. Ou, ainda, o caso de Fiel, que desde 9 de junho de 1866 evadira-se do engenho das Brotas, em Santo Amaro (em Sergipe), e segundo anuncio jornalistico ``tem sido visto no \index{Sangradouro}Sangradouro, Cabula, \index{Matatu}Matatu''\footnote{\textbf{O Alabama}, ano 4, série 8, nº 73, 14 jul 1866, p. 4}. 

Certo E. M. Pestana publicou n'\textbf{A União Liberal} de 19 de março de 1853 um poema chamado ``Meu Sentir'', declamando de Estância (Sergipe) as saudades que sentia da Bahia: ``Eu tenho saudades também de Brotas / Qu'o collo não curva à vil servidão''. Linhas talvez enigmáticas, não fossem os fatos que as explicam.

\subsection{Palco de festas}

O segundo uso, plenamente legal e normatizado, é o de \textit{palco de festividades religiosas, cívicas ou profanas}. O afastamento de Brotas da malha urbana consolidada de Salvador não significou, de forma alguma, qualquer cisão na sua malha de sociabilidades; não apenas seguia-se na freguesia o calendário geral de festas da cidade, como também eram realizados na freguesia festejos próprios a que acorriam moradores de outras freguesias.

Em 1838 o calendário da arquidiocese de Salvador indicou que a festa do Rosário era comemorada na matriz da freguesia de Brotas em 27 de dezembro \cite[p.~44]{arcebis_diario_1837}, e em 1842 que a do Santíssimo Sacramento na mesma freguesia era comemorada em 26 de dezembro \cite[p.~55]{arcebis_folhinha_1841}. Já um almanaque para o ano de 1878 apresenta datas diferentes: a de Nossa Senhora do Rosario, no dia 1º de janeiro, e a de Nossa Senhora de Brotas no dia 27 de dezembro \cite[pp.~53,~109]{macosta_almana_1877}. Há uma descrição da realização destes festejos datada de janeiro de 1881:

\begin{citacao}
\textbf{Procissão e festa}

A mesa da irmandade do SS. Sacramento e Nossa Senhora das Brotas, tendo de festejar no sabbado, 8 do corrente, com procissão solemne, que sahirá da \index{Igrejas de Brotas!Capela de Bom Jesus dos Milagres}capella do Senhor dos Milagres ao \index{Matatu}Matatú, ás cinco da tarde; e no domingo 9 com festa, tambem solemne, na propria matriz, a sua excelsa padroeira, convida a todos os seus irmãos e as irmandades do Senhor dos Milagres, Rosario e S. Benedicto da referida freguezia, afim de que todos reunidos em acto de irmandade tomem parte nos aludidos festejos.

Outrossim, a mesa, convicta da piedade religiosa para o culto de Maria Santissima, que tanto distingue os habitantes d'esta cidade, que, na phrase eloquente do finado D. Romualdo, foi apelidada de cidade de Maria; convida, esperando numeroso concurso, a todos em geral de seus habitantes e a cada um d'elles em particular a vir louvar Maria Santissima, sob o titulo de Senhora das Brotas, em os aprazados dias.

Brotas da Bahia, 5 de janeiro de 1881 -- O escrivão, Padre Ernesto d'Oliveira Valle\footnote{\textbf{O Monitor}, 5 jan. 1881, p. 2.}.
\end{citacao}

No \index{Rio Vermelho}Rio Vermelho, em 6 de fevereiro de 1881, saía pelas ruas o ``bando annunciador'' da festa de Santana\footnote{\textbf{O Monitor}, 6 fev. 1881, p. 1.}; a proximidade de datas indica relação entre esta festa católica e a hoje famosa festa de Iemanjá (2 de fevereiro).

Em 1871 o \textit{Dois de Julho} era comemorado em Itapagipe e Brotas logo depois dos festejos oficiais\footnote{\textbf{O Reverbero}, ano 1, nº 14, 6 ago. 1871, p. 7}, e em 1876 o ``bando annunciador'' da festa era ``concorrido''\footnote{\textbf{O Monitor}, 18 jul. 1876, p. 1}. Não custa esquecer que os festejos do Dois de Julho, como qualquer outro festejo cívico, pressupunham uma comunidade relativamente populosa e desejosa não apenas de rememorar os feitos do passado e inseri-los em sua vida presente, mas igualmente desejosa de entretenimento. O anúncio do festejo na imprensa, dada sua pequena circulação em meio a uma multidão analfabeta, ressalta outro aspecto: o desejo de fazer ver aos moradores de outros distritos a civilidade e a integração da vida de Brotas à vida pública urbana de Salvador. Diferente da festa da padroeira da freguesia, o Dois de Julho não era uma festa local a que concorriam apenas os paroquianos, era uma festa ``nacional'' da Bahia inteira, cuja comemoração era momento de grande júbilo em toda a capital; ao invés de acorrerem aos festejos em outras freguesias, os moradores de Brotas parecem ter preferido demarcar seu território, ou melhor, demarcar Brotas como um território civilizado da capital ao inserir seu território no calendário cívico.

Ainda em 1876 se observa, curiosamente, que o Dois de Julho era comemorado em Brotas quase vinte dias depois da data oficial, e era festança das boas:

\begin{citacao}
\textbf{Festejos do Dous de Julho de Brotas --} No domingo proximo sahirão do largo de Brotas, às 7 horas da noite, os carros triumphaes, seguidos por uma musica até o largo do Paranhos, \index{Matatu}Matatu, onde vão a ser depositados. \\
No dia 23, domingo, às 10 horas da manhã, impreterivelmente, partirão os carros do largo do Paranhos percorrendo as ruas do \index{Matatu}Matatú, \index{Pitangueiras}Pitangueiras e \index{Castro Neves}Castro Neves, seguindo até o largo de Brotas, onde se acha um palanque primorosamente preparado. \\
Nesse trajecto tocará uma excellente banda de musica, bem como à noite no palanque, na occasião da illuminação. \\
No dia 24 haverá illuminação no palanque e tocará a banda de musica escolhidas peças de seu repertorio. \\
No dia 25 haverá illuminação no palanque e às 9 horas da noite sahirão os carros para percorrer as ruas do costume e recolher-se\footnote{\textbf{O Monitor}, 22 jul. 1876, p. 1.}.
\end{citacao}

Em 1881 ainda se comemorava o ``Dous de Julho'' em Brotas, ainda no final do mês, e ainda por três dias seguidos\footnote{\textbf{O Monitor}, 31 jul. 1881, p. 1.}. Viu-se também na \autoref{subsubsec:pitangueiras} que em 1883 o Dois de Julho ainda era comemorado, desta vez saindo do Castro Neves.

Não custa lembrar, a este respeito, a associação entre os caboclos dos festejos e os \textit{caboclos do candomblé} \cite[p.~88-91]{albuquerque_doisdejulho_1997}; numa freguesia pontilhada por quilombos, terreiros e esconderijos, a comemoração do Dois de Julho em Brotas poderia ter servido não somente para inserir a freguesia no calendário cívico soteropolitano, mas igualmente para que negros escravizados disputassem o uso ritual do espaço público com seus senhores católicos puro-sangue. 

Em Brotas comemorava-se, além do Dois de Julho, o \textit{carnaval}: em 1879 anúncio indicou que a banda de música do ``16º de linha'' acompanharia no domingo de carnaval os bandos carnavalescos das freguesias da Sé, Sant'Anna e Brotas, e que já se encontraria postada desde as duas horas da tarde daquele dia na praça da Independência\footnote{\textbf{O Monitor}, 21 fev. 1879, p. 1.}.

