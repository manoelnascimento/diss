\section{Como Salvador via o distrito}\label{sec:2.5}

A pesquisa de imprensa deixou vívida impressão de que, durante todo o século XIX, a freguesia de Brotas -- à exceção do Rio Vermelho, considerado no século XIX como um dos melhores arrabaldes da cidade -- era vista pelo restante da cidade como um lugar ermo, distante, mesmo perigoso. Algumas passagens escolhidas ajudarão a compor esta imagem.

Através de edital do Senado da Câmara de 27 de janeiro de 1811, que estabeleceu cobradores para os quarenta e dois ``talhos'' de Salvador e os sete de seu termo (ou seja, dos distritos rurais), temos noticia da existencia em Brotas de um destes pontos de venda de carne, que contava então com seu respectivo cobrador; comparativamente, o Caminho Novo tinha quatro ``talhos'', o Taboao tinha oito, e o Sao Bento dezoito\footnote{\textbf{Idade d'Ouro do Brazil}, nº 19, 06 mar. 1811, pp. 2-3}.

Certo E. M. Pestana publicou n'\textbf{A União Liberal} de 19 de março de 1853 um poema chamado ``Meu Sentir'', declamando de Estância as saudades que sentia da Bahia: ``Eu tenho saudades também de Brotas / Qu'o collo não curva à vil servidão''. Enigmáticas linhas.

Dois africanos que moravam no caminho de Brotas para o Rio Vermelho foram encontrados mortos, e a policia prendeu um crioulo de mais de 60 anos como suspeito do latrocínio \footnote{\textbf{Jornal da Bahia}, ano XXII, número ilegível, 10 mar. 1875, p. 2}.

Em 1876 uma comissão encarregada de arrecadar fundos para o Asylo de Mendicidade localizado em Brotas veio a público agradecer pelos 572\$640 arrecadados, e informava já ter recolhido a quantia aos cofres da tesouraria municipal\footnote{\textbf{O Monitor}, 11 jul. 1876, p. 2}

Certo tenente Bernardino Jose de Almeida, que antes possuira bens e agora vivia na pobreza, morreu numa roça em Brotas e, por falta de meios, ficou tres dias insepulto, inumado por ato de benemerencia de certo escrivao Gularte\footnote{\textbf{O Alabama}, ano 14, serie 163, nº 1625, 25 nov. 1876.}

O jornal \textbf{A Religião}, autoproclamado ``órgão da Igreja Catholica da Bahia'' e publicado sob os auspicios do arcebispo Luiz Antonio dos Santos, deixou de noticiar o movimento eclesiástico da freguesia de Brotas em 12 de junho de 1887 ``por ser quasi nenhum''.

Curiosa disposição legal dispensava funcionários do Tesouro Provincial de comparecer ao trabalho em seu horário regulamentar se morassem nas freguesias da ``Penha, Mares, Victoria, Santo Antonio e Brotas'' e obtivessem permissão de seus chefes para não comparecer diariamente ao trabalho (!), ``salvo porem o primeiro dia util de cada semana, em que ficarão sujeitos à regra geral''\footnote{\textbf{O Monitor}, 29 set. 1877, p. 1.}.

Brotas ainda era, no final do século XIX, lugar de sedução de menores\footnote{  ``Apresentou-se esta manha ao sr. subdelegado da freguezia de Brotas uma senhora, moradora ao Sangradouro, queixando-se de um tal Mattos que lhe raptara uma sua filha menor, cujo nome ignoramos. Aquela autoridade, procedendo as respectivas diligencias, conseguiu descobrir o lugar onde se achava depositada a referida menor e trata de capturar o sedutor.'' (\textbf{Diario de Noticias}, ano 7, nº 23, 3 out. 1881, p. 1.)}, 

A edição de 25 de abril de 1880 do semanário \textbf{A Gargalhada} denunciava

\begin{citacao}
\dots que o fiscal em exercicio na freguesia de Brotas, ao passo em que nao ve o estado das ruas, occupa-se em encher a estaçao de meninos e velhos, conductores de carroças, deixando impunes os valentoes e trampolineiros \dots
\end{citacao}

A iluminação pública, mesmo no distrito urbano da freguesia, era deficiente, como indica reclamação de moradores datada de 6 de fevereiro de 1881; ``grande parte da estrada das Pitangueiras até o largo do Paranhos'' estava às escuras, apesar de serem necessários apenas ``quatro ou cinco lampeões'' para resolver o problema. Os moradores, cujos prédios estavam ``sujeitos ao imposto da decima urbana'', reclamavam principalmente dos inconvenientes causados a seu trânsito por esta estrada durante o inverno, ``privada como também está de calçamento''\footnote{\textbf{O Monitor}, 6 fev. 1881, p. 1.}.

Em 3 de março de 1889 petição de moradores da localidade ao chefe de polícia implorava providências contra ``aos abusos, ás desordens, aos sambas, ás palavras, gestos immoraes, e a toda sorte de escandalos que se praticão n'este infeliz logar, de onde a policia parece ter fugido espavorida''; reclamam da ``constante mutação de physionomias que, á semelhança de morcegos, apparecem e desapparecem quotidianamente'', e pedem ``amplas e energicas medidas que os garantam da sanha dos desordeiros, para que continuem a viver tranquilos como até aqui têm vivido n'esta agreste solidão''\footnote{\textbf{Diário da Bahia}, nº 50, 3 mar. 1889, p. 2.}. Ditos ``moradores ordeiros'' voltariam à carga ainda outra vez em 14 de março para falar de um ``crioulo conhecido em toda a freguezia de Brotas, principalmente do largo da matriz até á Cruz da Redempção, como perigoso desordeiro'' que teria sido responsável pelo assassinato do fiscal João Cancio Vergne Baptista e era ``capaz de commeter crime de qualquer natureza que lhe seja ordenado''; esta mesma pessoa teria causado desordem na festa da Pituba, ocorrida no mês anterior, ``pondo em sobresalto aos moradores da povoação''. Apontando diversas vezes relações de proximidade entre as autoridades policiais e a pessoa em questão, suplicavam ao presidente da província para que fizesse sentir ao chefe de polícia que ``o povo do largo da Matriz e da Cruz da Redempção, em Brotas, tem direito tambem a ser mantido pela força publica'' ; como o subdelegado da freguesia, diversas vezes procurado sobre a questão, ``limita-se a dizer que não dispõe de força publica'', chegaram mesmo a oferecer casa para que o destacamento de polícia da freguesia fosse dividido, ficando metade dele no largo de Brotas ou na Cruz da Redenção\footnote{\textbf{Diário da Bahia}, 14 mar. 1889, p. 2.}.


Em 30 de novembro de 1881, o jornal \textbf{O Preceptor} anunciou em sua página 4 que o médico Eduardo Feliciano Castilho residia à rua 25 de Março, nº 119, e dava ``consultas grátis aos pobres''.