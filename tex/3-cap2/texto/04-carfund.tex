\section{Esboço de caracterização fundiária}\label{sec:2.3}

Já no primeiro jornal publicado na Bahia, \textbf{Idade d'Ouro do Brazil} \cite[p.~162]{souza_imprensa_1972}, notamos interessante movimento de compra e venda de terras em Brotas. Em 1811 a roça da finada Maria Eufrasia do Carmo, cujo tamanho não foi informado, estava sendo leiloada desde o dia 17 de fevereiro por 3 contos de réis\footnote{\textbf{Idade d'Ouro do Brazil}, nº 19, 06 mar. 1811, p. 4}. 

Ao contrário do que dá a entender este anúncio, não parecia ser costume na época anunciar publicamente o preço dos imóveis à venda, ou, se assim não fosse, ao menos no jornal onde foi veiculado o anúncio isto não fazia parte da praxe dos emigrados portugueses que o redigiam e editavam. Em junho de 1813, por exemplo, foi anunciada no mesmo jornal a venda uma ``morada de casas'' na Estrada de Brotas, ``perto da Cruz'', embora não se anunciasse o preço; os interessados deveriam ir à tal morada acertar diretamente os valores\footnote{\textbf{Idade d'Ouro do Brazil}, nº 44, 01 jun. 1813, p. 4}. Em abril de 1814 Marcos Antonio Fernandes, morador de Nazaré, anunciou no jornal a venda de uma roça na Estrada de Brotas, ``bem povoada de arvoredo de espinho'', com ``fonte de bica'' e ``tanques para banho'', além de ``boa casa de vivenda e senzala separada''\footnote{\textbf{Idade d'Ouro do Brazil}, nº 33, 26 abr. 1814, p. 4}. Em janeiro de 1815 o mesmo jornal anunciou que estava para ser leiloada no dia 17 uma roça ``das Brotas para o Rio Vermelho'', em ``terras próprias, com seu arvoredo, bom brejo, casa de vivenda''\footnote{\textbf{Idade d'Ouro do Brazil}, nº 3, 10 jan. 1815, p. 4}, igualmente sem anunciar o lance mínimo. José Ramos, morador da Saúde, anunciou no jornal em novembro de 1816 a venda de uma roça em Brotas, de ``terras proprias, com arvoredo de espinho e outras plantações''\footnote{\textbf{Idade d'Ouro do Brazil}, nº 95, 26 nov. 1816, p. 4}, também sem indicar valores. 

A ``discrição'' chegava a ser curiosa: em abril de 1817, o jornal anunciou a venda de uma roça no ``caminho das Brotas, logo depois da Boa-Vista'', com o valor agregado dos ``chãos próprios'', muito atrativo numa cidade onde quase tudo era arrendamento, aforamento ou sesmaria; que além disso, a herdade dispunha de ``grande casa de vivenda'' , ``10 quartos contíguos à porteira, em parte murada a frente, com sua cocheira, brejo e fonte de beber''; os interessados deveriam dirigir-se à Typographia de Manoel Antonio da Silva Serva, onde se imprimia o jornal, para negociar o feito\footnote{\textbf{Idade d'Ouro do Brazil}, nº 31, 22 abr. 1817, p. 4}. Fizeram o mesmo com uma roça na Estrada de Brotas cuja venda anunciaram em outubro de 1818\footnote{\textbf{Idade d'Ouro do Brazil}, nº 84, 23 out. 1818, p. 4}. De editores, tanto Manoel Serva quanto seu colega de redação, Diogo Soares da Silva e Bivar, parecem ter pretendido ser também corretores imobiliários \textit{avant la lettre}\footnote{Outros anuncios no mesmo jornal davam a entender que seus editores agenciavam também a compra e venda de escravos, mas o detalhamento desta atividade foge ao objeto desta pesquisa.}.

Descobre-se também no \textbf{Idade d'Ouro do Brazil} que o solicitador da Casa da Fazenda, Joao Baptista de Faria, tinha uma ``roça com casa de vivenda'' em Brotas, e em janeiro de 1819 pretendia vênde-la\footnote{\textbf{Idade d'Ouro do Brazil}, nº 5, 15 jan. 1819, p. 4}. Descobre-se igualmente que Manoel da Silva Friandes aforava terras na freguesia, inclusive a ``morada de casas'' com ``sete braças de frente'' cuja venda, anunciada em março de 1819, teve como corretor o comerciante Manoel Jose Fontes Braga, dono de um armazem de secos e molhados na rua do Bispo\footnote{\textbf{Idade d'Ouro do Brazil}, nº 23, 16 mar. 1819, p. 4}.

De tudo isto, percebe-se que já no início do século XIX havia intensa movimentação de terras na freguesia de Brotas. A compreensão da urbanização ocorrida nas primeiras décadas do século XX requer a compreensão prévia da reconfiguração fundiária ocorrida na freguesia no século anterior. Para melhor compreensão do processo, ele será subdividido em áreas pautadas pela existência de grandes herdades, pontos notáveis ou denominações históricas que persistiram, em certos casos, até os dias atuais.

\subsection{O morgado da Nisa, o morgado da Foz e o latifúndio de Tomás da Silva Paranhos}

O \textit{morgado}\footnote{O morgado é uma antiga instituição do direito de propriedade, extinta no Brasil em 1835 e em Portugal em 1863, para cujo melhor compreensão torna-se necessária longa citação de comentário ao Título C do Livro IV das \textbf{Ordenações Filipinas}: 

``\textit{Morgados e bens vinculados.}

Coelho da Rocha no \textit{Dir. Civ.} § 491 tratando da propriedade limitada, dá as seguintes prelecções sobre a especie mais impportante dessa propriedade os \textit{bens vinculados}:

`A palavra \textit{vinculo}, tomada \textit{subjectivamente}, significa a instituição, ou condição de certos bens, que devem andar perpetuamente annexos em uma familia determinada, por uma forma especial de successão sem poderem ser divididos, nem alienados.

`Tomada \textit{objectivamente} significa os bens sujeitos a estes estabelecimentos ou \textit{vinculados}.

`Para se dar vinculo, he necessario:

`1. -- Instituição.

`2. -- a condição da propriedade, e portanto da indivisibilidado e da moralidade.

`Os Vínculos ou são \textit{Morgados} ou \textit{Capellas}.'

`Chama·se \textit{Morgado} o vínculo, que tem por fim principalmente a conservação do lustre e nobreza de uma família; em contraposição de uma \textit{Capella}, cujo fim he a epressão da piedade do instituidor.

`Comtudo em quasi todas as instituições de \textit{Morgados} costumão andar annexos alguns encargos pios; ainda quando não estivessem determinados na instituição, \textit{a)} os administradores são obrivados à gastar em obras de piedade a centesima parte do rendimento do vínculo (L. de 3 de Agosto de 1770, § 27); \textit{b)} nos Morgados \textit{unidos} em virtude do § 28 da citada Lei podem os encargos pios ser reduzidos à esta quantia, se a excedem.'

No scholio ao § 498 diz o mesmo Jurista o seguinte:

`A palavra \textit{Morgado} em phrase jurídica significa tambem o direito de succeder no vínculo, e na prhase vulgar muitas vezes costuma por ella designar-se a pessoa do administrador.

`Os Francezes definem o \textit{Morgado}: um fideicomisso gradual, sucessivo, perpetuo e indivisível, destinado a conservar o nome e explendor de uma família. [\dots]

Os Morgados depois de soffrerem uma reforma na presente Ord., tiverão outra na Lei de 3 de agosto de 1770, e assim se forão conservando tanto em Portugal, como no Brazil; mas entre nós forão abolidos, assim como as Capellas, pela Lei n. 57 -- de 6 de Outubro de 1835 [\dots]'' \cite[p.~990]{ordfil_1870}. Uma apreciação sumária da longa história deste instituto jurídico, dos séculos XIII ao XIX, pode ser vista em \citeonline{coelho_vincular_1980}. } \textit{de Nisa}\footnote{É comum encontrar a grafia ``Niza'', ou mesmo ``Nizza'' na documentação pesquisada. Por isto mesmo, não será adotado padrão, citando-se a grafia encontrada tal qual está no documento ou fonte secundária referida, dando-se preferência à grafia ``Nisa'' quando não se fizer referência a algum documento de época ou fonte secundária.} é importantíssimo para a constituição fundiária da freguesia. 

O que interessa a esta pesquisa neste morgado, na verdade, é apenas parte do total dos bens do marquesado de Nisa; trata-se do antigo \textit{morgado da Foz}, recebido pelo marquesado de Nisa -- ao que tudo indica por Rodrigo Xavier Teles de Castro da Gama Ataide Noronha Silveira e Sousa (1744-1784), seu 6º titular \cite{wiki_nisa_2015} -- por força do falecimento da última marquesa de Cascais. 

Este morgado é de longa história. Álvaro Gonçalves de Ataíde, primeiro conde da Castanheira, filho quarto\footnote{A \textit{primogenitura} foi um sistema simultaneamente familiar e patrimonial por meio do qual, para evitar a fragmentação do patrimônio familiar que tantos problemas causara, entre outros casos registrados mundo afora, no período de dissolução do império carolíngio, o primogênito e seus descendentes tinham precedência na sucessão de bens, enquanto os demais descendentes viam-se excluídos da linha sucessória. A estes últimos costumava-se chamar de \textit{filhos segundos}, independente de serem, de fato, os segundos mais velhos entre a prole. Numa família com três filhos, por exemplo, haveria sempre um primogênito e dois filhos segundos. Os \textit{filhos quartos}, como o próprio Álvaro Gonçalves de Ataíde, são uma espécie de filhos segundos. Por estarem excluídos da sucessão de bens, os filhos segundos não dispunham de meios para montar suas próprias famílias, e ou bem viviam como agregados na domesticidade do primogênito, ou saíam de casa em busca de novas terras para desbravar, de aventuras militares que lhes garantissem vantagens patrimoniais, de filiação a alguma ordem religiosa que lhes funcionasse como família substituta etc. \cite{BERNARDO1997,coelho_vincular_1980} . } ascendido à nobreza por casamentos (foram vários), integrava um círculo familiar que incluía como seus primos Tomé de Souza (primeiro governador-geral do Brasil), Martim Afonso de Souza (donatário da capitania de São Vicente) e Pero Lopes de Souza (donatário da capitania de Santo Amaro), a quem protegia e recomendava junto ao rei João III \cite[p.~III-3]{teixeira_doacoes_1978}. Logo que nomeado governador-geral do Brasil, Tomé de Souza envidou seus bons ofícios para que o seguinte pedido de seu primo Álvaro ao rei João III fossem expeditamente atendido:

\begin{citacao}
D. Antonio de Ataíde, Conde de Castanheira, faz saber a V. S. que ele quer mandar fazer engenho de açucar nesta Capitania da Baia de Todos os Santos e quer mandar povoar e fazer criações de toda a sorte de gados; assim vacum como porcos e outro gado miúdo, para o que tem necessidade da ilha de Itaparica, que está defronte desta Cidade do Salvador, com suas águas, matos, pastos e logradouros para os engenhos e povoados; e assim, tem necessidade da ilha pequena que está por traz dela na boca do Jaguaripe, da banda do sudoeste, com suas águas e matos nela conteúdos e inclusos, assim para fazer o que cumpre o que determina de povoar; tem também necessidade da ribeira que se chama Rio Vermelho que está da banda de Leste além desta cidade, com uma légua de terra para a costa do mar para Leste, e pela dita ribeira duas léguas de terra para o sertão do dito Rio, para contra esta cidade a que estiver por dar, e não se achar donos: pelo que -- Pede a V. S. lhe dê o conteúdo nesta petição, e as alcadarias das vilas, e povoações que nas ditaa povoações fizer para si e seos descendentes \cite[p.~III-3]{teixeira_doacoes_1978}.
\end{citacao}

O pedido foi atendido em sua íntegra em 29 de abril de 1552 \cite[p.~III-3 - III-4]{teixeira_doacoes_1978}, e o conde da Castanheira criou um \textit{morgado} com estes bens. Tão poderoso era o conde da Castanheira que mesmo os Garcia d'Ávila lhe pagavam foro pela Casa da Torre de Tatuapara\cite[p.~III-5]{teixeira_doacoes_1978}, hoje conhecida como o castelo Garcia d'Ávila, que dá nome à Praia do Forte. A casa da Castanheira, entretanto, teve sua linha sucessória matrimonialmente unificada com a da casa de Cascais já no século XVII, e no século XVIII encontramos estes bens transferidos para a casa de Nisa.

Os documentos não indicam datas para esta última transferência, mas é possivel situá-las aproximativamente. 

O primeiro fato relevante é a extinção da casa de Cascais pelo falecimento, em 1745, de Luís José de Castro Noronha Ataíde e Sousa, 4º marquês de Cascais e 11º conde de Monte Santo \cite{wiki_cascais_2015}; não tendo outros herdeiros além de sua esposa Anna José Maria da Graça -- é o que se pode entender lendo a documentacão citada por \citeonline[p.~592]{castralmeida_ultramar_1910} -- a morte desta resultou na reversão da casa à Coroa portuguesa \cite{wiki_cascais_2015}. 

O segundo elemento relevante é o fato de de Eugênia Maria Josefa Xavier Teles de Castro da Gama (1776-1839) ser listada como 11ª Condessa da Vidigueira, 7ª Marquesa de Nisa e 7ª Condessa do Unhao \cite{wiki_nisa_2015}. 

Sabe-se, além disso, que os marqueses de Cascais eram senhores do morgado da Foz, instituído pela já então falecida Condessa da Castanheira e composto pelas ``Ilhas de Taparica, Taramanda e Ilha Pequena na Ribeira e terras do Rio Vermelho, continente da Cidade da Bahia''; o marquês de Nisa, diante da extinção do marquesado de Cascais, solicitou à coroa portuguesa seu reconhecimento como sucessor no morgado da Foz, requerendo ``carta de sucessão em seu nome para poder entrar na posse e receber todos os rendimentos desde o falecimento da última Marquesa de Cascais'' \cite[p.~592]{castralmeida_ultramar_1910}. 

Tendo falecido em 1784 o marquês de Nisa, sua viúva, Maria Anna Josefa Xavier de Lima, requereu no ano seguinte confirmação da doação destas terras a seu falecido marido para que a doação aproveitasse a sua filha e herdeira do \textit{de cujus}, Eugenia Maria Josefa Xavier Telles -- aparentemente ainda menor de idade neste momento, dado o fato de sua mãe apresentar-se como sua ``tutora'' \cite[p.~592]{castralmeida_ultramar_1910}. 

Ora, a disputa entre o marquesado de Nisa e a coroa portuguesa origina-se num titulo jurídico que remonta a data que a documentação consultada não certifica, mas possibilita situar entre 1745, quando falece o último marquês de Cascais, e 1785, quando o assunto é ressuscitado. É certo, entretanto, que a disputa se resolveu em favor do marquesado de Nisa, pois em 1797 outro documento registra o marquês de Nisa como ``donatário [\dots] das terras do Rio Vermelho'' \cite[p.~543]{ramiz_expos_1881}.

Em 1839, o capitao Tomás da Silva Paranhos\footnote{Foram encontradas várias grafias para o nome desta personagem: ``Tomás'', ``Thomás'' e ``Thomaz''. Elas serão usadas indistintamente como constam nas fontes citadas, mas sempre que o texto não remeter a algum documento, será dada preferência à grafia ``Tomás''.} comprou ao marquesado de Nisa todas suas terras no Império brasileiro \cite[pp.~III-7 - III-12]{teixeira_doacoes_1978}, incluindo as do antigo morgado da Foz, e registrou-as na freguesia de Brotas, onde tinha residência \cite[p.~10]{ott_engenhos_1996}. 

A descrição das terras de Tomás da Silva Paranhos no \textbf{Livro Eclesial de Registro de Terras da Freguesia de Brotas} é impressionante, e vale transcrever na íntegra pela relevância histórica\footnote{A transcrição feita por \citeonline{teixeira_doacoes_1978} é da escritura de compra, que infelizmente não especifica os limites do velho morgado.}

\begin{citacao}
Terras do Capitão Thomas da Silva Paranhos

Ilustríssimo Senhor Vigário da Freguesia de Brotas

O abaixo assignado vem registrar as terras que possue nesta Freguesia, por compra feita à Casa de Niza, as quaes principião no marco que se acha collocado perto à casa do mar e armação do finado Visconde do Rio Vermelho, seguindo a mesma costa do mar até o rochedo que se acha na barra do Rio Vermelho, no lugar denominado ``Fontinha'', seguindo o Rio Vermelho acima até defronte do rio Camorogipe, antigamente São Lourenço, na estrada de Brotas para Itapoan, seguindo ao longo do dito rio, atravessando a estrada do Cabula, seguindo a mesma direção a estrada chamada de Manuel Ramos, seguindo a mesma direção atravessando a estrada das boiadas à encontrar o marco de [ilegível] preta, seguindo o rumo de Nordeste até onde divide as terras desta sesmaria com as dos Frades de São Bento, seguindo o mesmo rumo até o outeiro chamado de ``mavida'', onde morou o foreiro de São Bento por nome Aleixo, seguindo o rumo Sueste a subir na costa do mar, no lugar onde mora Antônio José Coelho, pela banda do Sul, seguindo o mesmo rumo até encontrar na costa do mar o marco donde partiu. Bem como declara que nesta comprehensão existem alguns foreiros a quem o abaixo assignado tem vendido o domínio direto, aos quaes em cumprimento da lei dará a Registro as confrontações de suas compras, e que nesta comprehensão igualmente existem terras na Freguesia de Santo Antônio Além do Carmo, onde igualmente estão registradas.

Bahia, vinte e hum de maio de mil oitocentos e cincoenta e oito.

Thomas da Silva Paranhos

E nada mais continhão as declarações que me foram enviadas.

Brotas da Bª, 9 de junho de 1858.

Vigº Ernesto de Olivª Valle\footnote{BR BAAPB, fundo Colonial, série Registros de Terra, livro 4675, f. 6 verso.}
\end{citacao}

Ainda que a extrema dificuldade de localizar no atual território soteropolitano marcos como o ``outeiro chamado de 'mavida''', a ``Fontinha'', as terras de São Bento, a ``estrada chamada de Manuel Ramos'' e outros torne tarefa hercúlea delimitar precisamente os limites desta vasta herdade, somente o fato de um marco encontrar-se no Rio Vermelho\footnote{Considerando ``Fontinha'' como a atual rua Fonte do Boi.} e outro na Estrada das Boiadas\footnote{Considerando-a iniciada no atual Largo da Lapinha.}, a 6,92km de distância um do outro, e também o registro destas terras nas duas maiores freguesias de Salvador em sua época, dão ideia aproximada da vastidão das terras de Tomás da Silva Paranhos. Talvez seja o único imóvel na freguesia a receber, com total e absoluta propriedade, o epíteto de \textit{latifúndio}. O velho morgado foi sendo fracionado em disputas com foreiros e sucessores, e o que restou destas terras foi, em parte, cedido a José Ribeiro Saldanha\footnote{É dele, e de suas terras, que deriva o nome do Alto do Saldanha.}, e noutra parte desapropriado pela Prefeitura de Salvador em 1906 \cite[p.~III-13 - III-14]{teixeira_doacoes_1978}\footnote{Diga-se de passagem que nas peças do processo de desapropriação transcritas por \citeonline{teixeira_doacoes_1978} encontra-se uma das primeiras menções a um lugar chamado ```Imbuhy', districto de Brotas'', a ser revisitado no capítulo seguinte.}.

\subsection{As fazendas Alagoa, Amaralina, Santa Cruz, Ubarana e Pituba}

\begin{figure}[!htp]
\centering
\subfloat[Em 1851]{
\includegraphics[width=0.7\textwidth]{3-cap2/complementos/mapas/armacaodalagoa-1851.eps} 
\label{armacaodalagoa-1851}
}
\  %espaco separador
\subfloat[Atualmente]{
\includegraphics[width=0.7\textwidth]{3-cap2/complementos/mapas/armacaodalagoa-hoje.eps} 
\label{armacaodalagoa-hoje}
}
\caption{Duas representações cartográficas do território correspondente às fazendas Alagoa, Amaralina, Santa Cruz, Ubarana e Pituba. Não foi possível avançar além do que o mapa mais antigo permitiu. \textbf{Fonte:} \citeonline{weyll_mappa_1851} e Google Earth.}
\end{figure}

A história fundiária destas cinco fazendas é inseparável, indissolúvel, indivisível. Limítrofes, integrantes do antigo morgado da Foz, seus antigos terrenos constituem, somados, os atuais bairros de Amaralina, Pituba, Nordeste de Amaralina, Santa Cruz, Rio Vermelho, Itaigara, Iguatemi, Chapada do Rio Vermelho e Vale das Pedrinhas.

Um dos grandes imóveis saídos do morgado da Foz foi a \textit{fazenda Alagoa}, localizada onde hoje estão os bairros do Rio Vermelho e Amaralina.

Quando ainda integrava o morgado foram construídos dentro dela, em 1768, um tanque de captação de água, um engenho (há muito desaparecido) e uma ermida \cite[p.~118]{campos_alagoa_1942}. A ermida, transformada em capela dedicada a Nossa Senhora dos Mares, ainda existe nos dias de hoje, incorporada ao Quartel de Amaralina, localizado no mesmo sítio onde se erguia a casa-grande do engenho.

Já desmembrada do morgado, foi adquirida em 1797 por Alexandre Teotônio de Sousa, tenente-coronel de granadeiros da guarnição de Salvador; alguns seus herdeiros remotos venderam-na em 1854 a José Álvares do Amaral\footnote{O primeiro sobrenome desta personagem histórica encontra-se grafado como ``Alves'' ou ``Álvares'' a depender da fonte. Foi mantida a grafia tal como foi encontrada.} \cite[p.~118]{campos_alagoa_1942}, com os seguintes limites constantes do \textbf{Livro Eclesial de Registro de Terras da Freguesia de Brotas}:

\begin{citacao}
Terreno de José Alves do Amaral

Em obediência ao despacho do senhor Vice-Presidente Missias de Leão, de 29 de abril de 1859, exarei o seguinte registro:

José Alves do Amaral, tendo o domínio útil da fazenda ``Alagôa'', situada nesta freguesia das Brotas, vem apresentala ao registro das terras. Principia o limite da dita fazenda, do lado da Ubarana, de propriedade útil do Major Manoel de Barros Paim, por um marco de pedra de cantaria collocado em mil setecentos e oitenta e nove na costa do mar em direção NO 35gº no qual se acha gravada a seguinte inscripção '1789', e partindo dahi no rumo que mostra o dito marco a encontrar a valla mestra divisória que passa nos fundos da dita fazenda Ubarana, e seguindo pela dita valla a encontrar a área nativa que limita com a fazenda ``Pituba'', de propriedade do Visconde do Rio Vermelho, e acompanhando a dita cerca até encontrar a estrada que conduz para a Cruz da Redempção em Brotas, dividindo sempre por esta estrada com a fazenda ``Santa Cruz'', de propriedade de Antonio Joaquim da Silva e Abreu, até o lugar chamado ``tanque'', e seguindo dahi pela valla mestra a desembocar no rio Camorogipe, e por este abaixo até sua foz no mar pela parte da Mariquita, confrontando por este lado com terras pertencentes ao Mosteiro de São Bento, tendo a dita fazenda toda a frente para o mar, pela costa as terras da dita fazenda [\textit{ilegível}] do senhorio direto Thomas da Silva Paranhos. Cumpre declarar que os limites da mencionada fazenda se encontram em litígio com os [\textit{ilegível}] confrontantes da Ubarana e Santa Cruz. Bahia, quatro de abril de mil oitocentos e cincoenta e nove. José Álvares do Amaral.

E nada mais continhão as declarações que me foram transmitidas. Brotas da Bahia, 3 de maio de 1859.

Vigº Ernesto de Olivª Valle\footnote{BR BAAPB, fundo Colonial, série Registros de Terra, livro 4675, f. 36 verso e 37.}
\end{citacao}

No mapa de \citeonline{weyll_mappa_1851} há a indicação de uma ``Armação da Lagoa'', compatível com os relatos da existência de uma armação de pesca nesta fazenda desde os idos do século XVIII \cite[p.~120-121]{campos_alagoa_1942}. O mesmo mapa mostra uma estrada que a liga ao Largo de Brotas. Se seguirmos seu trajeto desde este largo até o mar e o compararmos com o trajeto de ruas atuais, é plausível conceber alguns possíveis caminhos remanescentes desta antiga estrada\footnote{Com a base documental pesquisada até o momento é impossível descrever precisa e minuciosamente as ruas remanescentes desta estrada, e é possível mesmo que tal descrição documental sequer exista; a reconstituição desta estrada exigiria uma pesquisa \textit{in loco} com antigos moradores da Santa Cruz e do Nordeste de Amaralina para tentar recompor alguns traços perdidos desta estrada, profundamente modificada pela ocupação popular do território destes bairros, pelos loteamentos resultantes nos bairros de Amaralina e Pituba, e pela construção da avenida Juracy Magalhães e do Parque da Cidade.}

\begin{itemize}
\item Os dois caminhos saem do largo da Cruz da Redenção, descendo sua ladeira até o leito do rio Camorogipe, e o atravessam por meio de uma ponte (atualmente inexistente);
\item Uma primeira hipótese para um traçado remanescente desta estrada segue pela rua Onze de Novembro, ou Estrada da Santa Cruz, e daí pela rua do Futuro e pela avenida Nova República;
\item Uma segunda hipótese para o traçado corta caminho desde o leito do Camorogipe por dentro do atual Parque da Cidade, encontrando a avenida Nova República;
\item Da avenida Nova República o traçado segue encontrando-se com a rua Victorio Rossi para continuar pelo Beco da Cultura;
\item Uma linha imaginária ligaria o Beco da Cultura à rua Reinaldo de Matos, encontrando-se com a rua Cristóvão Ferreira;
\item Outra forma de ligar a avenida Nova República com a rua Cristóvão Ferreira seria sair da avenida para a rua Nova República, daí para a rua Valdomiro, depois para a travessa 20 de Junho, daí para as ruas do Areal e Ipanema, pela ruas Onze de Novembro e Francisco Sales até a rua dos Posseiros, e daí por uma linha imaginária até a rua Alto do Capim e, enfim, a rua Cristóvão Ferreira;
\item O caminho terminaria, por fim, na atual rua do Norte, ligada ao litoral por uma linha imaginária. 
\end{itemize}

É desta fazenda que deriva o loteamento, de final do século XIX, chamado Cidade Balneária Amaralina, a ser detalhado no capítulo seguinte.

A compra da fazenda Alagoa foi fortemente contestada durante quase quarenta anos por Tomás da Silva Paranhos, comprador, como visto, do que restara do morgado da Foz \cite[p.~118]{campos_alagoa_1942}. Terminada a querela, diz um memorialista célebre que a fazenda teria sido rebatizada em 1912 como fazenda \textit{Amaralina} \cite[p.~118]{campos_alagoa_1942}; como se verá no capítulo seguinte, a documentação consultada durante esta pesquisa dá a entender que já na década de 1890 havia duas fazendas distintas, a \textit{Alagoa} e a \textit{Amaralina}.

Menos célebre que a fazenda Alagoa, a \textit{fazenda Santa Cruz} é assim descrita no \textbf{Livro Eclesial de Registro de Terras da Freguesia de Brotas}:

\begin{citacao}
Fazenda de Antonio Joaquim da Silva e Abreu

Antonio Joaquim da Silva e Abreu possue na freguesia de Nossa Senhora de Brotas da Capital da Bahia uma fazenda denominada ``Santa Cruz'', em terrenos proprios, aqual limita-se pelo nascente com a fazenda Ubarana, pelo poente com o rio Camorogipe, pelo norte com a fazenda Pituba e pelo sul com o mesmo rio Camorogipe na povoação da Mariquita. Bahia e Freguesia de Brotas, vinte e sete de novembro de mil oitocentos e cincoenta e oito. Antonio Joaquim da Silva e Abreu.

E nada mais continhão as declarações que me foram transmitidas. Brotas da Bahia, 27 de novembro de 1858.

Vigº Ernesto de Olivª Valle\footnote{BR BAAPB, fundo Colonial, série Registros de Terra, livro 4675, f. 36 e 36 verso.}
\end{citacao}

A \textit{fazenda Ubarana} é assim descrita no \textbf{Livro Eclesial de Registro de Terras da Freguesia de Brotas}, num registro que se encontra severamente atingido pela ação do tempo:

\begin{citacao}
A fazenda denominada Ubarana, sita na f[regue]sia de N[oss]a Senhora de [Bro]tas, da Capital [da] Bahia, divide-se pelos la[do]s do Sul com a[s fa]zendas Alagôa e Santa Cruz, a saber [ilegível]dindo do [ilegível] da Ubarana do lado do Sul em [li]nha recta a Pedra da Marca pelo caminh[o] que vae para Brotas, afindar-se no rio Camorogipe, a encontrar do lado norte com a [ilegível] nativa de antigas árvores que faz a [divi]za com a fazenda da Pituba, descendo athe vizinhanças do mar, [ilegível] desta os limites [ilegível] referida fazenda do [ilegível] [ilegível]. Brotas, vinte e oito de novembro de mil oitocentos e cincoenta e nove. Manuel [ilegível] de Barros Paim.

E nada mais continhão as declarações que me foram transmitidas. Brotas da Bahia, 15 de janeiro de 1860.

Vigº Ernesto de Olivª Valle\footnote{BR BAAPB, fundo Colonial, série Registros de Terra, livro 4675, f. 40.}
\end{citacao}

Um resquício de memória desta fazenda ainda pode ser encontrado no nome da atual rua das Ubaranas, lindeira entre os bairros da Pituba e Amaralina, que corre paralela à avenida Manoel Dias da Silva entre as ruas Pará e Vandick Badaró.

Já a \textit{fazenda Pituba} tem seus limites assim descritos no \textbf{Livro Eclesial de Registro de Terras da Freguesia de Brotas}:

\begin{citacao}
Terras da Exmª Viscondessa do Rio Vermelho

Limites da legoa de terra que pertence a Excellentíssima Viscondessa do Rio Vermelho e o condomino seo filho Barão do Rio Vermelho. Na freguesia de Nossa Senhora das Brotas está a fazenda denominada Pituba em um [ilegível] de terras que pertenceu a Casa da Excellentíssima Marquesa de Niza, hoje ao Capitão Thomas da Silva Paranhos, e [ilegível] de posse por escriptura de foro perpetuo para a Excellentíssima Viscondessa do Rio Vermelho e seu filho o Barão do Rio Vermelho. Tem por limites a dita fazenda ao sul divide com a fazenda Ubarana, ao norte com terras do Engenho Santo Antônio, ao nordeste com terras de São Bento, onde está a Armação do Gregório, e a leste com o mar. Bahia, quinze de julho de mil oitocentos e cincoenta e nove. Barão do Rio Vermelho.

E nada mais continhão as declarações que me foram transmitidas. Brotas da Bahia, 15 de julho de 1859.

Vigº Ernesto de Olivª Valle\footnote{BR BAAPB, fundo Colonial, série Registros de Terra, livro 4675, f. 38.}
\end{citacao}

Como se pode inferir dos limites constantes nos registros de terra, a povoação da Mariquita era o ponto de convergência, e portanto de conflito, entre os limites das fazendas Santa Cruz e Alagoa. A deterioração do registro da fazenda Ubarana dificulta compreender os limites entre ela e as demais fazendas, pois a descrição de alguns marcos foi corroída pela tinta.

\subsection{Campinas e as terras dos Ladislau}

\begin{figure}[!htp]
\centering
\subfloat[Em 1851]{
\includegraphics[width=0.7\textwidth]{3-cap2/complementos/mapas/campinas-1851.eps} 
\label{campinas-1851}
}
\  %espaco separador
\subfloat[Atualmente]{
\includegraphics[width=0.7\textwidth]{3-cap2/complementos/mapas/campinas-hoje.eps} 
\label{campinas-hoje}
}
\caption{Duas representações cartográficas do território correspondente às terras dos Ladislau. \textbf{Fonte:} \citeonline{weyll_mappa_1851} e Google Earth.}
\end{figure}

Toda a área que hoje conhecemos como \textit{Campinas de Brotas} era no século XIX de propriedade de integrantes da família \textit{Ladislau}, cujo patriarca, João Ladislau de Figueredo e Melo, vinha a ser sogro do mesmo José Álvares do Amaral reivindicante da fazenda Alagôa. O próprio nome da área -- Campinas -- deriva de duas das herdades da família.

A primeira delas era a fazenda Campina Grande:

\begin{citacao}
Fazenda Campina Grande

Rosa Ladislau de Figueredo e Melo e Virgínia Ladislau de Figueredo e Melo possuem em condomínio nesta Freguesia de Nossa Senhora das Brotas uma Fazenda denominada Campina Grande, em que há Engenho de fabricar assucar, e que comprehende a fazenda do mesmo nome Campina Grande, [ilegível] Carregado e Chacôco, terras proprias, que de [ilegível] [ilegível] um lado com a roça da dita Rosa Ladislau de Figueredo e Melo no lugar da Cruz da Redempção e com outra denominada Campina Pequena de dona Michelina Ladislau e Silva e dona Joanna Fausta Ladislau e Silva, de outro com terras do Matatu de José Antonio Pinto pelo riacho de mesmo nome, e mais com terras do Girão de Joaquim Caetano de Almeida Couto, ou quem mais direito for, pelo riacho Camorogipe, outra com terras que forão de dona Maria de Argôlo, e com as do Engenho Santo Antônio da Viscondessa do Rio Vermelho e sua filha dona Judith Constança da Cunha, pelo dito Camorogipe, e pelo outro com a estrada que sobe da ponte do mesmo Camorogipe e com terras que forão de João Paulo e seu irmão Fabião. Esta declaração vae por uma de nós feita e por ambas assignada. Bahia e Freguesia de Nossa Senhora das Brotas, primeiro de junho de mil oitocentos e cincoenta e oito. Rosa Ladislau de Figueredo e Melo, Virgínia Ladislau de Figueredo e Melo.

E nada mais continhão as declarações que me foram enviadas. Brotas da Bª, 5 de junho de 1858.

Vigº Ernesto de Olivª Valle\footnote{BR BAAPB, fundo Colonial, série Registros de Terra, livro 4675, f. 4 verso e 5.}
\end{citacao}

A outra, a fazenda Campina Pequena:

\begin{citacao}
Roça Campina Pequena

Michelina Ladislau e Silva e Joanna Fausta Ladislau e Silva possuem nesta Freguesia de Nossa Senhora das Brotas uma roça denominada Campina Pequena com casa de vivenda e outras benfeitorias, terras próprias, e que se divide pela frente com terras de Raphael e José Joaquim, e por outro lado com a Quinta das Beatas pelo riacho que a separa, por outro com a estrada que entra para o Engenho da Campina Grande, e pelo fundo com terras do mesmo Engenho. Esta declaração vae feita por uma de nós e por ambas assignada. Bahia e Freguesia de Nossa Senhora das Brotas, primeiro de junho de mil oitocentos e cincoenta e oito. Michelina Ladislau e Silva, Joanna Fausta Ladislau e Silva.

E nada mais continhão as declarações que me forão enviadas. Brotas da Bahia, 4 de junho de 1858.

Vigº Ernesto de Olivª Valle\footnote{BR BAAPB, fundo Colonial, série Registros de Terra, livro 4675, f. 5 e 5 verso.}
\end{citacao}

No mapa de \citeonline{weyll_mappa_1851}, vê-se nitidamente que o ``Engº'' marcado perto de ``Prambeé'' trata-se do Engenho Campina Grande, situado na baixada hoje correspondente à rua Santiago de Compostela. No cume logo abaixo, onde hoje se situam  o cemitério Jardim da Saudade e o Abrigo Salvador, tem início uma estrada que corresponde à atual rua Campinas de Brotas. À atual rua Teixeira Barros corresponde a antiga Estrada do Beijú; se o mapa de Weyll continuasse mais ao leste, seria possível observar como esta estrada se unia à Estrada das Armações no trecho onde se cruzam, hoje, as avenidas Paulo VI e Antônio Carlos Magalhães, tal como o demonstra um mapa da Prefeitura de Salvador datado de 1969.

\subsection{As terras de Joaquim José de Oliveira}

O mesmo Joaquim José de Oliveira cuja roça serviu como marco divisório do distrito é mencionado durante as lutas pela independência:

\begin{citacao}
Pelo mesmo tempo, marchava pela estrada do Rio Vermelho a divisão da esquerda, comandada pelo coronel Felisberto Gomes Caldeira, precedida, bem como a primeira, por uma partida de exploradores tirada do 4º batalhão [\textit{conhecido como Pitanga}], e comandada pelo tenente Manoel Rocha Galvão, menos, porém, o batalhão nº 1, do comando do major José Leite Pacheco, que, pelo lado das Brotas, passou a ocupar os entrincheiramentos da roça de José Joaquim de Oliveira [\dots] \cite[p.~58]{vieira_memorias_1903}
\end{citacao}

\subsection{A Estrada de Brotas e seus arredores}

\begin{figure}[!htp]
\centering
\subfloat[Em 1851]{
\includegraphics[width=0.4\textwidth]{3-cap2/complementos/mapas/estbrotas-1851.eps} 
\label{estbrotas-1851}
}
\  %espaco separador
\subfloat[Atualmente]{
\includegraphics[width=0.4\textwidth]{3-cap2/complementos/mapas/estbrotas-hoje.eps} 
\label{estbrotas-hoje}
}
\caption{Duas representações cartográficas do território correspondente à Estrada de Brotas (atual av. D. João VI). \textbf{Fonte:} \citeonline{weyll_mappa_1851} e Google Earth.}
\end{figure}

No século XVIII saía do Portão da Piedade uma estrada então conhecida como \textit{Caminho Grande}, correspondente ao que veio depois ser a \textit{Estrada de Brotas}, atual Avenida D. João VI; era por aí que se partia da cidade ao Rio Vermelho, passando pelo paço do Acupe, integrante do morgado da Casa da Torre \cite[p.~85]{campos_brotas_1942}. Patente de sargento-mor da freguesia de ``Nossa Senhora de Brotas do Caminho Grande'' concedida a Veríssimo de Campos de Carvalho em 1725 \cite[p.~114]{texmel_manusbn_1896} mostra como era conhecida a freguesia em seus primórdios.

Infelizmente não foi possível encontrar registros seguros do traçado completo do Caminho Grande, exceto por uma referência que o qualificou como `perigoso'' e disse passar ele ``pela Lapa e atual Campo da Pólvora, pela crista dos montes e pelos divisores de águas, passando em Fonte Nova e por Brotas até aquele ponto da costa oceânica'' \cite[p.~488]{sampaio_salvador_2016}. Com base no mapa de \citeonline{weyll_mappa_1851}, pode-se conjecturar, entretanto, que a ligação entre a Piedade e o atual Largo dos Paranhos, onde tinha início a Estrada de Brotas, fosse feita pelo trecho da atual avenida Joana Angélica que vai até a ladeira da Fonte Nova e por esta própria ladeira, chegando, através da atual ladeira dos Galés, até o referido largo, completando assim o primeiro trecho. Daí em diante, pode-se apenas conjecturar, inconclusivamente, por onde o Caminho Grande desceria rumo ao Rio Vermelho.

Entre 1848 e 1849 a presidência da província investiu 2:269\$344 na Estrada de Brotas, comprometendo-se a investir outros 5:000\$000 em melhoramentos na via; investiu também 1:414\$000 no encanamento do rio Camorogipe, empenhando-se a investir outros 177:539\$000 na mesma finalidade \cite{bahia_rpe_1849}.

Caetano Vicente de Almeida Galião, juiz de paz do segundo distrito da Sé em 1835, possuía um pequeno engenho e uma pequena fazenda em Brotas \cite[p.~239]{REIS2004males}.

\subsubsection{Cruz das Almas}



\subsubsection{Largo de Brotas}



\subsubsection{Cemitério de Brotas}



Em 1841 o \textit{largo da Cruz da Redenção} foi mandado abrir pelo coronel João Ladislau de Figueredo e Mello, dono do engenho Campinas \cite[p.~88]{campos_brotas_1942}. 

\subsubsection{Estrada e Alto do Beijú}


\subsection{A fazenda Boa Vista e seus arredores}

\begin{figure}[!htp]
\centering
\subfloat[Em 1851]{
\includegraphics[width=\textwidth]{3-cap2/complementos/mapas/boavista-sangradouro-1851.eps} 
\label{boavista-sangradouro-1851}
}
\  %espaco separador
\subfloat[Atualmente]{
\includegraphics[width=\textwidth]{3-cap2/complementos/mapas/boavista-sangradouro-hoje.eps} 
\label{boavista-sangradouro-hoje}
}
\caption{Duas representações cartográficas do território correspondente à fazenda Boa Vista (atual Engenho Velho de Brotas) e ao Sangradouro (atual Santo Agostinho). \textbf{Fonte:} \citeonline{weyll_mappa_1851} e Google Earth.}
\end{figure}

O mapa de Weyll mostra ainda uma longa estrada saindo das terras da Boa Vista em direção a uma estrada que corresponde à atual avenida Cardeal da Silva. Remanescentes desta estrada são as ruas Almirante Alves Câmara e Padre Luiz Figueira, no Engenho Velho de Brotas, e Sérgio de Carvalho, no Vale da Muriçoca; mais ou menos no ponto onde a Sérgio de Carvalho faz esquina com a atual av. Edite, também no Vale da Muriçoca, o mapa de Weyll indica uma ponte sobre um riacho, e daí em diante a estrada seguia um curso hoje inexistente, que terminava aproximadamente na altura da atual Ladeira das Carmelitas, na Federação. Seria esta a Estrada da Boa Vista, estendendo-se desde o Largo dos Paranhos até a atual Cardeal da Silva?

\subsubsection{Moinho}



\subsubsection{Capela do Deus Menino}



\subsubsection{Dique Pequeno}



\subsubsection{Monte de Belém}



\subsubsection{Porto dos Saveiros}



\subsubsection{Estrada da União}



\subsection{A Estrada Dois de Julho}

\begin{figure}[!htp]
\centering
\subfloat[Em 1851]{
\includegraphics[width=0.7\textwidth]{3-cap2/complementos/mapas/e2j-1851.eps} 
\label{e2j-1851}
}
\  %espaco separador
\subfloat[Atualmente]{
\includegraphics[width=0.7\textwidth]{3-cap2/complementos/mapas/e2j-hoje.eps} 
\label{e2j-hoje}
}
\caption{Duas representações cartográficas da Estrada Dois de Julho (atual av. Vasco da Gama). Em 1851 ela não estava construída, mas corresponde ao leito do rio Lucaia. \textbf{Fonte:} \citeonline{weyll_mappa_1851} e Google Earth.}
\end{figure}


\subsection{O Matatu Grande, o Matatu Pequeno, Quinta das Beatas e arredores}

\begin{figure}[!htp]
\centering
\subfloat[Em 1851]{
\includegraphics[width=0.7\textwidth]{3-cap2/complementos/mapas/matatu-1851.eps} 
\label{matatu-1851}
}
\  %espaco separador
\subfloat[Atualmente]{
\includegraphics[width=0.7\textwidth]{3-cap2/complementos/mapas/matatu-hoje.eps} 
\label{matatu-hoje}
}
\caption{Duas representações cartográficas do Matatu, mostrando a rua da Valla (atual av. Heitor Dias), a Estrada da Pólvora (atual rua Raul Leite), a Estrada do Matatu Grande (atual rua Luiz Anselmo) e a Quinta das Beatas (atual Cosme de Farias). \textbf{Fonte:} \citeonline{weyll_mappa_1851} e Google Earth.}
\end{figure}

Em 17 de junho de 1799 o Conselho Ultramarino deu parecer favorável a requerimento de porte de armas defensivas feito pelo capitão Pedro Gomes Ferreira; o militar residia ``na sua fazenda do Matatu'' \cite[p.~228]{castralmeida_ultramar_1914}, indicando que já no século XVIII a área era reconhecida por este nome. 

Há duas versões para a etimologia do topônimo. A primeira e mais conhecida diz ser ele de origem tupi, significando ``mata escura'', ``floresta negra'' \cite[p.~281]{sampaio_tupi_1987}. A segunda, menos conhecida, diz que se trata de um africanismo de origem bantu, significando ``lugar deserto, isolado'' \cite[p. 46]{dorea_ruas_2006}. Qualquer das duas versões, seja pela existência de mata fechada, seja escassez populacional, passam a impressão de um lugar distante, ermo, pouco povoado, e é bem possível que assim o fosse no século XVIII quando encontramos a primeira referência ao nome; no século XIX, entretanto, o \textbf{Livro Eclesial de Registro de Terras da Freguesia de Brotas} indica a existência de muitos pequenos proprietários de terras na área, sendo ela a que mais tem registros fundiários em toda a freguesia\footnote{BR BAAPB, fundo Colonial, série Registros de Terra, livro 4675.}.

No mapa de \citeonline{weyll_mappa_1851}, lido no sentido NNE-SSE, a área é representada por três cumeadas. A primeira tem uma estrada abre para três pequenos caminhos -- correspondentes ao que hoje seriam as esquinas das ruas Laura Costa e Professor Osvaldo O'Dwyer, na Vila Laura -- e termina na ``Casa da Povora''; trata-se do ``Matatu Pequeno'', e a estrada corresponde à atual rua Raul Leite. A segunda tem uma estrada sem nenhuma esquina ou bifurcação -- correspondente à atual rua Luiz Anselmo -- que vai dar no ``Matatu Grande''. A terceira tem uma estrada -- correspondente à atual rua Cosme de Farias -- que vai dar na ``Quinta das Biatas''\footnote{Há indicação de que a Estrada do Matatu Grande corresponderia às atuais ruas Luiz Anselmo e Raul Leite, e a Estrada do Matatu Pequeno à atual rua Barros Falcão \cite[p.~124]{valladares_beaba_2012}; a indicação, entretanto, não faz sentido, porque a junção da Luiz Anselmo com a Raul Leite resultaria numa grande via, em forma aproximada de ``U'', que reuniria as cumeadas dos morros que no mapa de \citeonline{weyll_mappa_1851}, contemporâneo das antigas denominações, são separadas como ``Matatu Grande'', à direita, e ``Casa da Povora'', à esquerda. Na falta de documentos comprobatórios da mudança toponímica, não encontrados até onde foi possível avançar na pesquisa ora exposta, parece muito mais plausível que a Estrada do Matatu Grande corresponda à rua Luiz Anselmo e a Estrada do Matatu Pequeno à rua Raul Leite.}. No mesmo mapa é possível encontrar símbolos indicativos de construções pontilhando a cumeada do Matatu Grande, embora a cumeada onde se localiza a Casa da Pólvora apresente-se pouco povoada, assim como a da Quinta das Beatas.

Novamente lendo o mapa de Weyll no sentido NNE-SSE, quatro rios escavam os vales circundantes destas cumeadas. Temos em primeiro lugar o \textit{rio das Tripas}, afluente do Camaragipe, vizinho ao qual corria a rua da Vala, no trecho correspondente à atual avenida Heitor Dias. O segundo rio é indicado por Weyll como sendo o \textit{Santo Antônio}, hoje um esgoto \cite[p.~136]{santos_aguas_2010} que corre quase paralelamente à atual rua do Baixão. O terceiro rio é o \textit{Córrego das Beatas} \cite[p.~158]{santos_aguas_2010}, correspondente, grosso modo, à atual Baixa do Matatu, também transformado em esgoto. O Córrego das Beatas é afluente do rio \textit{Bonocô}, que no mapa de \citeonline{weyll_mappa_1851} separa a Quinta das Beatas da cumeada cortada pela Estrada de Brotas.

\subsubsection{Matatu Grande}



\subsubsection{Matatu Pequeno}

Em 1802 o governador e capitão-general da Capitania da Bahia, Francisco da Cunha e Meneses, expediu portaria ao capitão-de-mar-e-guerra, intendente da marinha e armazéns gerais, ordenando-o a construir uma ``casa de pólvora'' no sítio do Matatu \cite[p.~93]{oliveira_ultramar_1977}\footnote{A julgar pelo mapa de \citeonline{weyll_mappa_1851}, a Casa da Pólvora localizava-se no sítio onde hoje está a \textit{Vila Militar do Matatu}, administrada pelo Exército.}.

\subsubsection{Quinta das Beatas}

A fazenda que ficou conhecida como \textit{Quinta das Beatas}, no atual bairro de Cosme de Farias, tem sua descrição no \textbf{Livro Eclesial de Registro de Terras da Freguesia de Brotas} extremamente danificada pela ação do tempo:

\begin{citacao}
Quinta das Beatas

[ilegível] fazenda denominada Quinta das Beatas é propriedade do Recolhimento do Senhor Bom Jesus dos [Perd]oens, está situada na freguesia de Nossa Senhora  [de] Brotas, confina pelo nascente com a fazenda de[no]minada Campina, dos herdeiros do coronel João Ladis[la]u de Figueredo, e Matatu Grande, pertencente ao [ilegível] Pinto, pelo poente com a roça do tenente[-coron]el Pinheiro e com o capitão Paranhos, pelo sul [com] a roça de Amorim Vianna, com a Torre e com a [ilegível] e pelo norte com o Matatu pequeno e com [ilegível] [ilegível] Paranhos. Bahia, dezesseis de março de [mil oitocentos e] sessenta. Anna Maria Magdalena Re[jente]. 

[E nada mais] se continha em as ditas declarações [que me for]am enviadas.

Brotas da Bª, 20 de [março de 1860].

Vigº Ernesto de Olivª Valle\footnote{BR BAAPB, fundo Colonial, série Registros de Terra, livro 4675, f. 40 verso.}
\end{citacao}

A antiga sede da fazenda, a julgar pelo que mostra o mapa de \cite{weyll_mappa_1851}, estaria em algum lugar no trecho da atual rua Cosme de Farias situado entre as esquinas das ruas Jaguarari e Lima Durval.

A relação entre o Recolhimento dos Perdões e a Quinta das Beatas é um dos mais acabados exemplos de rentismo religioso.  dos terrenos arrendados e aforados ACCIOLY

\subsubsection{Pitangueiras}

Em 1871 já se noticiava na área a existência do ``povoado das Pitangueiras'', para onde havia sido removida uma escola primaria masculina, medida elogiada pelo rápido crescimento das turmas e pela abertura de turmas noturnas para artistas-operários\footnote{\textbf{Jornal da Bahia}, ano XIX, nº 5.445, 21 set. 1871, p. 1}.

\subsection{A fazenda Torre e os remanescentes da fazenda Acupe}

\begin{figure}[!htp]
\centering
\subfloat[Em 1851]{
\includegraphics[width=0.4\textwidth]{3-cap2/complementos/mapas/acupe-1851.eps} 
\label{acupe-1851}
}
\  %espaco separador
\subfloat[Atualmente]{
\includegraphics[width=0.4\textwidth]{3-cap2/complementos/mapas/acupe-hoje.eps} 
\label{acupe-hoje}
}
\caption{Duas representações cartográficas do território correspondente às fazendas Torre e Acupe. Note-se, pouco abaixo da palavra ``Acú'', a pequena estrada correspondente à atual ladeira do Acupe, e o rio Bonocô à esquerda. \textbf{Fonte:} \citeonline{weyll_mappa_1851} e Google Earth.}
\end{figure}

Há notícias de que a Casa da Torre possuía, já no século XVIII, uma roça na região hoje conhecida como \textit{Acupe de Brotas}, que a viúva de Garcia d'Ávila Pereira vendeu em 1765 por 500\$000 \cite[p.~10]{ott_engenhos_1996}. É muito provável ser esta a roça conhecida como ``Torre'', cujo registro no \textbf{Livro Eclesial de Registro de Terras da Freguesia de Brotas} anda bem danificado:

\begin{citacao}
Roça da Torre

Vem o abaixo assignado [ilegível] [ilegível] [uma linha inteira ilegível] [ilegível]ada Roça da Torre [ilegível] [duze]ntas e vinte e seis braças, limitandose pelo [ilegível] da Cidade com a roça da viúva Amorim [ilegível] lado das Brotas com a roça do finado [Mem] de Amorim [Filgu]eiras e pelo fundo com a [Q]uinta das Beatas. Bahia, oito de março de [mil] oitocentos e sessenta. Francisco Pires de [Carv]alho Albuquerque. 

E nada mais [ilegível]tinha em as declarações que me foram [ilegível].

Brotas da Bª, 17 de março de [1860].

Vigº Ernesto de Olivª Valle\footnote{BR BAAPB, fundo Colonial, série Registros de Terra, livro 4675, f. 40.}
\end{citacao}

O jornal \textbf{Idade d'Ouro do Brazil} anunciou, em julho de 1817, que Victorino dos Santos Pereira -- dono de muitas outras coisas expostas no mesmo anúncio\footnote{O sr. Victorino aparentava ser comerciante, pois anunciou no \textbf{Idade d'Ouro do Brazil} vender ``breu de muito boa qualidade'', ``alcatrão d'América'', ``cabos sortidos'', lonas ``da Suécia'' e ``da Rússia'', ferro ``redondo'' e ``em barra'', pregos e aço. Além disso, Victorino Pereira aparentava ser muito bem provido de bens, pois ``não duvida vender a dinheiro, ou com prazo, um barco de 66 palmos de quilha muito bem construido''; se o comprador quisesse, ainda poderia ``comprar o Mestre e quatro Marinheiros escravos''. Reforça esta impressão o fato de vender também, no mesmo anuncio, vários sitios e fazendas: \textit{Murici}, em Itapicuru, com duas léguas; \textit{Rio de Paus}; a fazenda \textit{Ramalho}, no distrito de Carinhanha, ``Termo da Vila de Jacobina''; as fazendas \textit{Riacho} e \textit{Porto de João Pereira}, no Rio Preto; no Lagarto, as fazendas \textit{Curral Novo}, ``\textit{Ingola caxorro}'', \textit{Palma} e \textit{Pé de Serra}, alem dos sitios \textit{Macuna}, \textit{Tapeirinha} e \textit{Piauí}, ``próprios para criar gado'' (\textbf{Idade d'Ouro do Brazil}, nº 55, 15 jul. 1817, p. 4).} -- prometia a recompensa de dez mil-reis para quem encontrasse ``um cavalo ruço queimado de bom tamanho marca DM na pata direita, assendeirado cauda curta, crina sem estar aparada'' \footnote{\textbf{Idade d'Ouro do Brazil}, nº 55, 15 jul. 1817, p. 4}. O cavalo pertencia aos bens da roça \textit{Torre}, que o abastado sr. Victorino dizia ser de sua propriedade.

Já a fazenda Acupe encontramos dividida no \textbf{Livro Eclesial de Registro de Terras da Freguesia de Brotas}, como se vê:

\begin{citacao}
Dona Maria da Piedade Tabirá Bahiense, viúva do Coronel Antonio Lopes Tabirá Bahiense, possui um terreno no lugar denominado Acupe, na freguesia de Nossa Senhora das Brotas desta Capital, com sete braças de frente linha recta, segundo sua escriptura, que dá para a mesma estrada. Esta roça outrora foi parte da fazenda denominada Acupe e como pertencesse a diversos esses venderam suas partes, teve de fazer-se uma estrada pelo centro, e veio a tornarse a frente em uma linha diagonal contendo doze braças de frente, por que em uma das extremidades faz um funil, confinando por um lado com a roça de Elias Lopes de São Jerônimo, linha recta da pedra marca do rego mestre e por elle acima até encontrar com terras da roça de dona Maria Rosa Gomes da Silva, seguindo-se outra recta da valla que desce da estrada do Engenho Velho, atravessando o rego mestre até a estrada do Acupe, onde teve princípio esta demarcação, e pelo fundo com Barnabé Arraes, pelo mesmo rego mestre. Bahia, primeiro de junho de mil oitocentos e cincoenta e oito. Dona Maria da Piedade Tabirá Bahiense.

E nada mais continhão as declarações que me forão enviadas.

Brotas da Bª, 21 de junho de 1858.

Vigº Ernesto de Olivª Valle\footnote{BR BAAPB, fundo Colonial, série Registros de Terra, livro 4675, f. 11.}
\end{citacao}

No mapa de \citeonline{weyll_mappa_1851} já se vê a referida estrada, de onde se deduz que a divisão da fazenda Acupe se deu bem antes do seu registro.

Veja-se o seguinte anúncio, de 1876:

\begin{citacao}
ROÇA

Aluga-se na estrada de Brotas, lugar denominado Acú, uma roça com casa de morar, arvoredos frutiferos, boa fonte de bica, brejo para plantação. Quem a pretender dirija-se a venda do beco que achará com quem tratar\footnote{\textbf{Diário de Notícias}, ano 2, nº 199, 02 set. 1876, p. 3}.
\end{citacao}








