\section{Fundação e delimitação territorial da freguesia}\label{sec:2.1}

\subsection{Dos tempos pré-cabralinos à fundação da freguesia}

A primeira notícia que se tem de qualquer forma de povoação do território que viria a compor a freguesia de Brotas remete à existência, muito provavelmente pré-cabralina, das aldeias tupinambás de \textit{Mairaqui}, ou \textit{Mairiquiig}, de onde deriva o nome do atual \textit{Largo da Mariquita}, e que significa, em tupi, ``naufrágio dos franceses''; \textit{Ubarana}, muito provavelmente no lugar onde se situou a fazenda cujos limites serão descritos adiante; e \textit{Pituba}, no bairro que hoje leva o mesmo nome \cite{azevedo_povoamento_1969,dorea_ruas_1999,sampaio_salvador_2016,VASCONCELOS2002}. Da parte dos colonizadores portugueses, a primeira notícia que se tem é a de um engenho, existente já no período anterior à chegada de Tomé de Souza em 1549, na localidade que hoje conhecemos como Engenho Velho \cite[p.~235]{sampaio_salvador_2016}.

Após o estabelecimento da cabeça-de-ponte portuguesa na Conceição da Praia e a construção, entre 1549 e 1551, das primeiras edificações e muros no cume da falha geológica soteropolitana, onde hoje se localiza a Praça Municipal, ainda se fazia necessário para estes primeiros portugueses, por questão de sobrevivência mesmo, submeter o ``gentio'' a seu domínio\footnote{Não se pode interpretar de outro modo instruções constantes no Regimento de Tomé de Souza como a de tomar posse da antiga Vila do Pereira, à força se necessário (item 3 do Regimento); localizar os tupinambás, distinguir entre eles os dispostos à paz para alianças e os dispostos à guerra para os castigar, ``destruindo-lhes suas aldeias e povoações, matando e cativando aquela parte deles que vos parecer que basta para seu castigo e exemplo de todos'' e mandando enforcar os chefes nas suas respectivas aldeias (item 6 do Regimento); em tomada a Vila do Pereira, construir imediatamente uma fortaleza e povoação em lugar mais para dentro da Baía de Todos os Santos, em ``sítio sadio e de bons ares'' com ``abastança de águas e porto'', apta a prover as outras capitanias (item 8 do Regimento), com seis léguas de termo para cada lado (item 9 do Regimento). Além da sujeição pela força, a ordem dada a Tomé de Souza quanto às relações com os tupinambás era de conceder a paz aos assim submissos, desde que eles ``reconheçam sujeição e vassalagem, com encargo de darem em cada ano alguns mantimentos para a gente da povoação'' \cite[pp.~81-101]{ruy_politica_1949}.}.

Secular tradição política europeia recomendava a soberanos, senhores e magnates, ao menos desde a Idade Média e nos territórios onde vigeu plenamente o regime senhorial, estabelecer igrejas paroquiais como instrumento para aglutinação populacional e exercício de soberania \cite[pp.~193-205]{BERNARDO1997}. O rei português João III, sabedor e praticante desta tradição, ordenou no seu Regimento a Tomé de Souza replicá-la por estas bandas, adaptando-a às novas circunstâncias: neste documento, levou em conta o ``grande inconveniente'' de se fazer os índios convertidos morar na terra de outros índios não-cristãos e de permitir qualquer contato entre uns e outros. Por isto, recomendou expressamente trazer os índios cristianizados para perto das povoações portuguesas (item 45 do Regimento). A forma encontrada para isolar os recém-convertidos e submetê-los à soberania portuguesa, quebrando seus padrões culturais ao tempo em que se lhes impunham os hábitos e costumes reinois, foi a criação dos \textit{aldeamentos}, comunidades indígenas tuteladas por colonos ou padres \cite[pp.~72-76]{santos_salvador_2004}.

Justo a primeira experiência de aldeamento conduzida pelos portugueses foi a povoação de \textit{São Paulo}. Em carta datada de 5 de julho de 1559, Manuel da Nóbrega, que dá a esta comunidade religiosa primazia entre outras três criadas depois da chegada do governador-geral Mem de Sá em 1557\footnote{As outras são as de São João, ``a três léguas da cidade'', e a de Sancti Spiritus, a ``sete léguas''. A primeira localizava-se próximo ao atual bairro soteropolitano de Plataforma, e a última deu início à povoação de Abrantes, em Camaçari.}, descreve em breves linhas sua rotina, mas não sua forma: diz que ali já estava instalada uma escola de meninos, com algo entre oitenta e cento e vinte alunos que estudavam à tarde, ``porque tem o mar longe e vão pelas manhãs pescar para si e para seus pares que não se mantêm d'outra coisa''; estes jovens eram encarregados pelos jesuítas de catequizar seus pais e os mais velhos à noite, depois dos estudos \cite[p.~179]{nobrega_cartas_1931}. Afirma-se que este aldeamento localizava-se, tomando como base marcos atuais, ou no Alto do Cruzeiro (no atual bairro de Cosme de Farias), ou no Largo da Cruz da Redenção \cite[p.~87]{campos_brotas_1942}.

Este aldeamento se constituiu à base de índios trazidos das aldeias do Rio Vermelho e arredores -- e o Rio Vermelho foi outro núcleo populacional importante da freguesia, existente mesmo antes de sua constituição\footnote{Como se verá adiante, trata-se do trecho do Rio Vermelho que, usando marcos atuais, está compreendido entre a foz do rio Lucaia e a Praça dos Jangadeiros, em frente ao Quartel de Amaralina. O trecho entre a Praia da Paciência e a foz do rio Lucaia pertenceu, desde sempre, à freguesia da Vitória.}. Esta localidade tem registros históricos que remontam ao século XVI, e apontam para povoação ainda mais recuada no tempo: cabe registrar a existência de aldeias e aldeamentos no Morro do Conselho (aldeamento de Nossa Senhora do Rio Vermelho) e proximidades (aldeamento de São Lourenço, ou Tamandaré) \cite[p.~75]{santos_salvador_2004}.

Ao contrário do que afirma a historiografia clássica sobre a invasão holandesa a Salvador (1624-1625), induzida a erro por uma confusão linguística e topográfica e por desatenção quanto a medidas de época, nem o Morro do Conselho, nem a localidade onde se situa o atual bairro do Rio Vermelho foram usados como refúgio pelos portugueses durante a invasão holandesa de 1624-1625. A confusão linguística se dá pelo fato de o rio Camorogipe, um dos principais a cortar o território soteropolitano, ter seu nome vertido para o português como ``rio vermelho'' em função da abundância em suas margens do \textit{camará vermelho}, ou \textit{lantana}, planta de flor vermelha. O Camorogipe, antes de lhe imporem mudanças ao leito, nascia próximo da enseada do Lobato, na Boa Vista, e confluía com o Lucaia para desembocar num estuário comum, onde hoje é o Largo da Mariquita; desde 1866 o Camorogipe passou a correr por baixo da Estrada do Retiro, atual Avenida San Martin, e assim perdeu-se, com o tempo, a noção de que o ``rio vermelho'' atravessa o território soteropolitano em sentido latitudinal desde o mar interno da Baía de Todos os Santos até as praias atlânticas. Daí a confusão. Quanto ao Arraial do Rio Vermelho, reduto dos portugueses na luta contra os holandeses, há vários indícios históricos de que foi na verdade construído onde hoje fica o Largo do Tanque, na altura do que deveria ser, à época, o tanque do Engenho Conceição\footnote{Sobre a confusão linguística, cf. \citeonline{edelweiss_camarajipe_1969}; sobre a construção do Arraial do Rio Vermelho no atual Largo do Tanque, cf. \citeonline[p.~54-62]{magalhaes_equus_2010}; para visualizar o leito do Camorogipe antes do tamponamento, assim como a proximidade entre uma das nascentes do Camorogipe e o tanque do Engenho Conceição, cf. o mapa de \citeonline{weyll_mappa_1851}.}. O que parece ter havido no local, ao que indica a historiografia mais recente, foi um verdadeiro êxodo de soteropolitanos corridos da cidade pelo invasor flamengo, todos em busca de refúgio.

Finda a ocupação holandesa em 1628 o tráfego de tropas e boiadas exigiu do Senado da Câmara o conserto da ponte sobre o Rio Vermelho, no caminho que levava à casa da Torre dos Garcia D'Ávila; era por aí que Salvador era abastecida de gado antes que a Estrada das Boiadas fosse inaugurada, em 1652 \cite[pp.~315-316]{azevedo_povoamento_1969}. Em 1632 foi destruído um quilombo no Rio Vermelho \cite[p.~67]{VASCONCELOS2002}, indicando que o afastamento da cidade-fortaleza soteropolitana era suficiente para que africanos em fuga da escravidão ali encontrassem refúgio (o assunto será retomado na \autoref{subsec:refugioescrav}, p. \pageref{subsec:refugioescrav}). Em 1653 a Câmara concedeu a Natal Cascão e Mateus Tavares monopólio da compra e venda de peixe nas paragens da Pituba, Ubarana e Rio Vermelho, para garantir o fornecimente destes víveres à cidade \cite[p.~259]{azevedo_povoamento_1969}, indicando uma das vocações econômicas da futura freguesia.

\subsection{Da fundação da freguesia ao início do século XIX}

Exceto pelo fato da existência, em 1711, de uma fortificação no Rio Vermelho sob responsabilidade do coronel Garcia de Ávila -- por sinal muito descuidada, pela pouca gente que o militar lá enviava de seu regimento \cite[p. 99]{cabral_manuscritos_1883} --, a historiografia consultada não aponta o acontecimento de outros fatos relevantes no território de Brotas antes de sua constituição como freguesia em 1718 pelo rei João V, a pedido insistente do arcebispo Sebastião Monteiro da Vide. O arcebispo percebia a necessidade de canalizar mais rendas para a arquidiocese de Salvador, vez que havia ainda muitos párocos isolados nos sertões e dificuldades de pagar as côngruas em valores aptos a garantir a sobrevivência digna dos padres; havia, entretanto, poucas paróquias na arquidiocese para reforçar a captação de dízimo, e forte desequilíbrio entre o clero secular (diocesano) e regular (jesuítas, franciscanos, carmelitas etc.) em favor destes últimos. Sob tais condições, o funcionamento das igrejas paroquiais como instrumento de registro civil (e portanto registro populacional e controle censitário), ratificado por vários dispositivos das \textbf{Constituições primeiras do Arcebispado da Bahia}\footnote{No Livro I das Constituições, o Título 20 trata dos livros de ``assentos de batizados'', única forma de registro civil antes da separação entre Igreja e Estado \cite[pp.~28-31]{vide_const_2007}, e o Título 73 estabelecia a obrigação de cada paróquia um livro ``em que se assentem os casados'' \cite[pp.~130-131]{vide_const_2007}. No Livro IV, o Título 49 tratava dos chamados ``assentos dos defuntos'', ou seja, de livros de registro de óbitos \cite[pp.~292-293]{vide_const_2007}. Na ausência de registros civis -- as Ordenações Filipinas, cuja legislação civil (Livros II e IV) vigeu no Brasil até o advento do Código Civil de 1916, são absolutamente silentes quanto ao assunto -- era nas igrejas e nestes livros que se encontravam os documentos capazes de provar filiações, núpcias, divórcios e falecimentos.}, estaria seriamente comprometido. É evidente que Brotas foi fundada -- junto com outras vinte novas paróquias -- dentro de um processo de reorganização da malha paroquial, ele próprio parte de uma estratégia de consolidação da soberania portuguesa sobre o território colonizado \cite[pp.~26-30]{vivas_botelho_2011}. A estratégia, como sabemos, funcionou.

Em 1757 uma noticia sobre a freguesia, escrita pelo vigario Pedro Barbosa Gondim, indicou a existência nela de 1.045 ``almas'', das quais apenas vinte e cinco ``não são de comunhão'', que habitavam esparsamente as ``três léguas de extensão'' do território da freguesia, concentrando-se, além do entorno da igreja matriz, no ``sítio da Pituba'' e, mais afastadamente, nas armações do Saraiva e do Gregório \cite[p.~183]{castralmeida_ultramar_1908}. Descrição da freguesia de São Bartolomeu de Pirajá feita no mesmo ano pelo vigário Francisco Baptista da Silva indica que o rio ``Pituassu'' cortava o território de Brotas \cite[p.~218]{castralmeida_ultramar_1908}; como o topônimo designa atualmente o mesmo rio, principal afluente do Rio das Pedras, com foz localizada na Boca do Rio (que por isto leva este nome) \cite[p.~175-177]{santos_aguas_2010}, é possível perceber como desde muito cedo os limites atlânticos da freguesia foram muito grandes, alavancando enormemente sua já mencionada vocação pesqueira. Noutro censo, de 1775, a freguesia contava com 1.063 almas e 189 fogos \cite[p.~183]{castralmeida_ultramar_1910}; no mesmo ano, há indicação de que os ``pescadores forros'' de Brotas eram, com os da ``Vitoria [\dots], S. Pedro do Sanipe (sic) e Santo Amaro de Ipitanga'', dos poucos que sabiam pescar em alto-mar, demonstrando continuar a pesca na freguesia \cite[p.~294]{castralmeida_ultramar_1910}.

No início do século XIX, mais precisamente em 1802, eram registradas a existência de três igrejas filiais na freguesia: Nossa Senhora da Luz, Nossa Senhora da Boa Vista e Santo Antônio \cite[p.~172]{VASCONCELOS2002}\footnote{A primeira corresponde à atual igreja homônima, no bairro da Pituba; a segunda é a capela existente no Solar Boa Vista, hoje dessacrada; e a terceira era uma pequena igreja de beneditinos, já ruída, cuja localização precisa é atualmente ignorada.}. Em 1854 Brotas tinha 12 quarteirões, contra 29 da Vitória e 36 de Santo Antônio, duas freguesias que lhe podem ser comparadas sem maiores mediações; em 1857, estes números passaram a 13 (Brotas) e 30 (Vitória), sem registro para Santo Antônio; e em 1863, ficaram em 20 (Brotas), 30 (Vitória) e 34 (Santo Antônio). Há que se observar, entretanto, que o \textit{número} de quarteirões nada dizia a respeito da \textit{extensão} de cada um; comparativamente, o Paço, menor freguesia da cidade, tinha 11 quarteirões em 1854, 10 em 1857 e novamente 10 em 1863 \cite[p.~46]{NASCIMENTO2007}.

\begin{sidewaysfigure}[!htp]
\includegraphics[width=1\textwidth]{3-cap2/complementos/mapas/1855weyll-brotassegundoweyll.eps}{\footnotesize \par \textbf{Fonte:} Elaboração do autor, sobre mapa de \citeonline{weyll_mappa_1851}. Compare-se com a malha urbanizada de Salvador, representada pela mancha de pontos pretos logo abaixo. \par}
\caption{O território da Freguesia de Brotas em 1851, segundo \citeonline{NASCIMENTO2007}.}
\end{sidewaysfigure}

\subsection{Uma delimitação dos limites da freguesia (e uma crítica)}

A instituição da décima urbana, em 1857, resultou da necessidade de cobrir os custos com a construção de novas ruas como a rua da Vala (1851), a Ladeira da Independência (1854) e a Barroquinha (1859) \cite[p.~309]{ruy_camara_1953}. A comissão que delimitou o perímetro de cobrança do tributo terminou cindindo a freguesia de Brotas em dois \textit{distritos}, um \textit{urbano}, outro \textit{rural}, sendo a fronteira entre ambos marcada pela linha formada pela rua da Vala\footnote{``[\dots] do qual descerá pela ladeira que vai passar pela frente do edifício da Quinta dos Lázaros, seguindo por aí até a travessa que comunica com a rua da Valla, pela qual subirá a linha de limites até a trifurcação da mesma rua da Valla [\dots]'' \cite[pp.~309-310]{ruy_camara_1953}.}, pela rua das Sete Portas\footnote{``[\dots] onde tomará o rumo que se dirige à fonte das Pedras [\dots]'' \cite[pp.~309-310]{ruy_camara_1953}.}, pela rua do Sangradouro\footnote{``[\dots] e por ele seguirá até o cruzamento com a travessa ou rua do Sangradouro [\dots]'' \cite[p.~310]{ruy_camara_1953}.}, pelo Matatu\footnote{``[\dots] por onde continuará até o Matatu, limitando-se neste em frente da roça de Thomaz da Silva Paranhos, inclusive esta, e daí voltará pelo mesmo Matatu e seguirá até o alto de Joaquim José de Oliveira'' \cite[p.~310]{ruy_camara_1953}. Encontraremos adiante com as terras de Tomás da Silva Paranhos.}, pela Estrada de Brotas\footnote{``[\dots] e daí pela estrada de Brotas [\dots]'' \cite[p.~310]{ruy_camara_1953}.} e pela Boa Vista\footnote{`` [\dots] até a casa da Boa Vista, por cuja estrada se dirigirá ao Dique [\dots]'' \cite[p.~310]{ruy_camara_1953}.}. 

Esta delimitação, quando comparada com a mais conhecida -- e acriticamente replicada -- descrição do perímetro da freguesia de Brotas no século XIX, permite visualizar a divisão territorial da freguesia em dois distritos. Eis a descrição, assim apresentada por uma notável historiadora baiana:

\begin{citacao}\label{nascimentodescreve}
A freguesia de Nossa Senhora de Brotas foi criada pelo arcebispo D. Sebastião Monteiro da Vide, em 1718, sendo a seguinte sua demarcação extrema com outras freguesias, no século XIX: com Santo Antônio Além do Carmo pela Estrada Nova, começando pela roça do comendador Barros Reis, vindo até a Fonte Nova no Dique, onde fazia diferentes limites com Santana e São Pedro. Daí, pela estrada Dois de Julho, seguia até a ponta da Mariquita, de onde se espraiava costeando a lagoa da Pituba, até Armação e o Rio das Pedras, quando se dividia com a freguesia de Itapuã, suburbana da cidade. Seguia a freguesia de Brotas até o Engenho da Bolandeira, onde novamente fazia divisa com Itapuã e com a freguesia de Santo Antônio Além do Carmo. Limitava-se com a Vitória na Mariquita \cite[p.~58]{NASCIMENTO2007}.
\end{citacao}

Qualifico como ``acrítica'' a reprodução desta descrição sumária sem ver quaisquer deméritos na autora nem na descrição sucinta; o problema está em quem a reproduz, pois como, na maioria dos casos, quem a reproduz -- e são muitos -- tem fins outros que não a descrição mais precisa possível dos limites da freguesia, não enxergam alguns pontos cegos da descrição:
\begin{enumerate}
\item Respeitando o conhecimento de causa de quem a escreveu, não há uma só indicação de fontes, para que se possa visitá-las novamente e compará-las com outros documentos;
\item Ainda que a descrição seja a simples citação indireta do texto de documentos oficiais -- da igreja? do Estado? --, como tantos documentos da época os pontos notáveis, assim como os limites, são bastante vagos para quem não conheça de antemão o território da freguesia;
\item A descrição corresponde quase passo a passo a certas linhas tracejadas, de função desconhecida, existentes no mapa topográfico de Carlos Weyll, copiando-lhe inclusive alguns erros ortográficos \cite{weyll_mappa_1851}\footnote{O engenheiro Carlos Augusto Weyll (1815 - 1855), embora brasileiro, era filho do alemão Pedro Weyll (? - 1839), engenheiro-arquiteto e um dos primeiros colonizadores alemães na região Sul da Bahia, cuja presença na área antecede inclusive a implantação da colônia Leopoldina-Franckental. Em 1816, o príncipe austríaco Maximiliano de Wied-Neuwied foi recebido pelo Weyll pai no povoado indígena abandonado Almada, ao norte de Ilhéus; a família Weyll havia chegado há pouco da Holanda, tendo seu patriarca recebido concessão de uma légua quadrada de terras para plantar café e algodão. Dois anos depois, a família Weyll recebeu em sua casa os naturalistas alemães Johann Baptist Ritter von Spix e Carl Friedrich Philipp von Martius \cite[pp.~458-460]{oberacker_leopoldina_1972}. As desventuras posteriores da família Weyll com a construção da Casa de Prisão com Trabalho pouco interessam ao escopo desta pesquisa; importa é que o ambiente onde cresceu Carlos Weyll nas colônias alemãs do Sul da Bahia, apesar do contato constante com índios e negros de quem poderia ter aprendido outras línguas além do português oficial, estava certamente mais impregnado da língua alemã que do português, e o seu mapa, produzido em 1845 \cite[p.~31]{bahia_rpe_1846} e publicado muito provavelmente em 1851, contém vários erros ortográficos incompatíveis com o padrão linguístico da época. Exemplos: ``Resgato'' por \textit{Resgate}, ``Casa da Povora'' por \textit{Casa da Pólvora}, ``Moraria'' por \textit{Mouraria}, ``Quermado'' por \textit{Queimado} etc. O provável erro ortográfico que mais chamou a atenção, porque reproduzido por Anna Amélia Nascimento em sua descrição da poligonal da freguesia de Santo Antônio Além do Carmo \cite[p.~55]{NASCIMENTO2007}, foi ``Prambeé''; não foi possível encontrar esta localidade constante no mapa de 1851 em qualquer dos documentos consultados nesta pesquisa, mesmo nos jornais de época. A localidade, entretanto, fica aproximadamente no mesmo lugar conhecido hoje como \textit{Pernambués}, de que já achamos notícia num relatório governamental de 1926, já com esta grafia \cite[p.~185]{bahia_rpe_1926}.}.
\end{enumerate}

É possível perceber, pela comparação entre o perímetro inicial de cobrança da décima urbana e a descrição comentada, que ficaram fora do perímetro da décima urbana áreas pertencentes à freguesia de Brotas que hoje conhecemos como Amaralina, Pituba, Armação, Boca do Rio, Imbuí, Campinas de Brotas, Vila Laura, Horto Florestal, Candeal, Cosme de Farias, Vale do Matatu, Vasco da Gama \dots Em suma, a maior parte do território da freguesia no século XIX integrava seu \textit{segundo distrito}, que mantinha o caráter \textit{rural} e \textit{pesqueiro} herdado já então de séculos.

\subsection{Outras particularidades da freguesia no século XIX}

Ainda em 1877 funcionava em Brotas (e em Salvador inteira) o velho esquema dos \textit{sinais de incendio}, detalhados num regulamento municipal de 11 de novembro de 1853. Em caso de incendio, o sinal de fogo seria dado pelo toque do maior sino da igreja que primeiro dele soubesse, do maior sino da matriz da freguesia em que se manifestasse o incendio e do maior sino das demais igrejas que dele tivessem notícia; este toque seria de trinta badaladas apressadas, depois das quais seriam dada breve pausa seguida por toques em número convencionado por freguesia: um para a Se, dois para Sao Pedro, tres para Santana, quatro para Conceiçao, cinco para Pilar, seis para Passo, sete para Santo Antônio, oito para Vitória, nove para Brotas, dez para Penha e onze para Mares. Enquanto durasse o incendio, esta sinalizaçao seria repetida a cada quatro minutos; ao debelar-se o sinistro, a igreja que primeiro deu o sinal soaria um breve repique para indicar o fim da emergencia \cite[pp.~192-193]{macosta_almana_1877}. Em 8 de maio de 1878 o delegado do 1ª distrito, no expediente de polícia, o sr. José Álvares do Amaral, fez notar que ``pelo modo irregular por que é actualmente feito traz grande confusão e dificilmente se pode saber a freguesia onde o incendio se manifesta'', e por isto revisou o número de badaladas por freguesia: 8 para a Se, 9 para Sao Pedro, 10 para Santana, 11 para Conceiçao, 12 para Pilar, 13 para Passo, 14 para Santo Antônio, 15 para Vitória, 16 para Brotas, 17 para Penha e 18 para Mares\footnote{\textbf{O Monitor}, 11 mai. 1878, p. 1.}.