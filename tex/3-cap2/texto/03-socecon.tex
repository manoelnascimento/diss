\section{Caracterização socioeconômica}\label{sec:2.2}

Brotas era até o século XIX um distrito eminentemente \textit{rural}. Informe sobre o serviço postal datado de janeiro de 1881 indicou que Brotas, assim como o Rio Vermelho e Mont-Serrat, era uma das localidades onde ``não ha serviço urbano''\footnote{\textbf{O Monitor}, 04 fev. 1881, p. 1.}.

\begin{citacao}
Em Brotas, abrigava-se uma rarefeita população rural, que era também típica do 2º distrito de Santo Antônio. Roças, fazendas e mesmo engenhos eram encontrados nos segundos distritos de Brotas e Santo Antônio \cite[p.~52]{NASCIMENTO2007}
\end{citacao}

Já se viu, em seção anterior, que todo o caminho entre o Rio Vermelho e Itapuã era margeado por fazendas de gado, e também foi exposta a vocação pesqueira do distrito. A estas duas atividades econômicas exercidas no território soma-se outra, pouco lembrada: a \textit{produção de laranjas}. Num relatório de 1917 sobre a produção de laranjas na Bahia do século XIX, uma equipe estadunidense de pesquisadores apresentou os seguintes dados:

\begin{citacao}
Os principais distritos citríferos dentro da cidade [\dots] são os seguintes: Cabula, contendo cerca de 30.000 árvores; Saboeiro, com 12.000 árvores; Cruz do Cosme, 7.000 árvores; Matatu, 8.500 árvores; Brotas, 6.000 árvores; São Gonçalo, 2.000 árvores; e Vitória (incluindo Barra, Graça e Rio Vermelho), 1.500 árvores \cite[p.~3]{dorsett_orange_1917}\footnote{Texto original: ``The principal orange districts within the municipality, as shown upon the map (p. 8), are as follows: Cabulla, containing about 30,000 trees; Saboeiro, with 12,000 trees; Cruz do Cosme, 7,000 trees; Matatu, 8,500 trees; Brotas, 6,000 trees; Sao Gongalo, 2,000 trees; and Victoria (including Barra, Graca, and Rio Vermelho), 1,500 trees''.}.
\end{citacao}

Conquanto a produção de laranjas no Cabula e no Saboeiro seja conhecidíssima da memória coletiva dos soteropolitanos, o relatório faz notar um fato que, se passou despercebido pelas vistas dos pesquisadores, deve aqui ser ressaltado: sendo Matatu e Brotas partes do mesmo distrito, o número de árvores dos dois deve ser somado, resultando em 14.500 árvores --- o que faz de Brotas o segundo maior distrito citrífero de Salvador, atrás do de Santo Antônio\footnote{Cabula, Saboeiro, Cruz do Cosme e São Gonçalo eram localidades do distrito de Santo Antônio; somadas suas árvores, chega-se a um total de 51 mil árvores.}

Brotas não era freguesia das mais queridas pelos soteropolitanos, ao que parece nem mesmo pelo clero. Veja-se o caso de certo pároco ``português lusitaníssimo'' denunciado nas paginas do jornal \textbf{O Guaycuru} em 1845. Ele, além de ``viver em constante guerra com esse pobre povo'', há muito estava fora da paróquia ``sob pretexto de moléstia'', embora fosse público e notório ter ele ``uma capelania na cidade e nos dias desobrigados ir dizer regularmente sua missa a Conceiçao da Praia''. Este mesmo pároco tentou fazer-se substituir por dois sacerdotes coadjutores, que cedo abandonaram o posto ao ver que o vigário não queria dividir com eles a côngrua. Por isso, desde 1842 que estava a freguesia de Brotas 

\begin{citacao}
\dots sem nenhum recurso da Igreja; não há ao menos a missa conventual, não há quem ministre os Sacramentos na hora extrema, não há absolutamente culto algum --- e assim tem estado durante toda a quaresma. Não há muito que esteve um cadáver exposto à porta da matriz por um dia inteiro, e aí ficaria até que os cães o devorassem, se um agente da polícia não o fizesse sepultar. \footnote{\textbf{O Guaycuru}, ano 3, nº 46, 22 mar. 1845, p. 342}. 
\end{citacao}

A situação só veio a ser resolvida por decreto imperial de 13 de dezembro de 1850, quando o padre \textit{Ernesto de Oliveira Valle}, originalmente vigário na freguesia de Urubú, no interior da Bahia, foi apresentado como novo pároco de Brotas\footnote{\textbf{O Noticiador Catholico}, ano 3, nª 121, 18 jan. 1851, p. 252.}. Em 1877 era ainda ele o pároco da freguesia \cite[p.~176]{macosta_almana_1877}, e em 1881 figurava não apenas como pároco, mas também como escrivão da irmandade do Santíssimo Sacramento de Nossa Senhora de Brotas\footnote{\textbf{O Monitor}, várias edições de 1881.}. Este vigário será encontrado adiante, na \autoref{sec:2.3} (p. \pageref{sec:2.3}), exercendo as funções notariais impostas aos párocos de sua época pela Lei 601/1850.