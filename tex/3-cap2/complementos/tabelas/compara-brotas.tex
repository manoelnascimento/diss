\begin{table}
\centering
\IBGEtab{ \caption{Comparativo entre dados imobiliários de 1858-1862 e 1886-1891}
\label{tab:compara-brotas} }
{
\begin{tabular}{m{4cm}llll}
\toprule
Localidade	& Imóveis (1858-1862)	& Imóveis (1886-1891)	& Variação	& Variação (\%) \\
\midrule \midrule
Acupe	& 17	& 22	& 5	& 29,41\% \\
Alagôa	& 1	& 1	& 0	& 0,00\% \\
Armação	& 1	& 3	& 9	& 900,00\% \\
Beiju / Estrada das Armações	& 2	& 7	& -2	& -100,00\% \\
Brotas	& 1	& 1	& 0	& 0,00\% \\
Bulhões	& 1	& 0	& -1	& -100,00\% \\
Campinas	& 7	& 2	& -5	& -71,43\% \\
Candeal	& 3	& 3	& 0	& 0,00\% \\
Cruz da Redenção	& 4	& 8	& 4	& 100,00\% \\
Cruz das Almas / Estrada Brotas/Rio Vermelho	& 7	& 14	& 7	& 100,00\% \\
Engenho Velho / Boa Vista	& 6	& 18	& 12	& 200,00\% \\
Estrada da União	& 2	& 0	& -2	& -100,00\% \\
Estrada de Brotas	& 13	& 15	& 2	& 15,38\% \\
Largo de Brotas	& 5	& 39	& 34	& 680,00\% \\
Lucaia	& 1	& 0	& -1	& -100,00\% \\
Mariquita	& 0	& 65	& 65	& 6500,00\% \\
Matatu	& 50	& 66	& 16	& 32,00\% \\
Pitangueiras	& 2	& 0	& -2	& -100,00\% \\
Pituba	& 1	& 0	& -1	& -100,00\% \\
Pomar	& 0	& 5	& 5	& 500,00\% \\
Quinta das Beatas	& 3	& 14	& 11	& 366,67\% \\
Rua da Vala	& 1	& 0	& -1	& -100,00\% \\
Sangradouro	& 7	& 15	& 8	& 114,29\% \\
Santa Cruz	& 1	& 0	& -1	& -100,00\% \\
(sem nome)	& 0	& 25	& 25	& 2500,00\% \\
Torre	& 2	& 0	& -2	& -100,00\% \\
Ubarana	& 2	& 2	& 0	& 0,00\% \\
Várzea / Engenho de Santo Antônio	& 1	& 1	& 0	& 0,00\% \\
\midrule
TOTAL	& 136	& 326	& 190	& 58,28\% \\
\bottomrule
\end{tabular} 
}
{\fonte{Elaboração do autor, com base em dados do \textbf{Livro Eclesial de Registro de Terras da Freguesia de Brotas} (\textbf{BR BAAPB}, fundo Colonial, série Registros de Terra, livro 4675) e da lista de imóveis cadastrados para pagamento das décimas urbanas do quinquênio 1886-1891 apresentada pela Recebedoria Provincial da Bahia (\textbf{Gazeta da Bahia}, ano VIII, nº 273, 14 dez. 1886, p. 2).}}
\end{table}