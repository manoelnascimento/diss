\begin{table}[!htp]
\centering
\IBGEtab{
\caption{Logradouros cadastrados para coleta da décima urbana no quinquênio 1886-1891}\label{tab:decurb1886-1891}}
{
\begin{tabular}{ccc}
\toprule
Logradouro				&Total de móveis cadastrados	&\% da freguesia	&\\
\midrule \midrule
(sem nome)				&25				&8,36\%			&\\
Estrada da Quinta das Beatas		&8				&2,68\%			&\\
Quinta das Beatas			&6				&2,01\%			&\\
Matatu Pequeno				&24				&8,03\%			&\\
Matatu Grande				&38				&12,71\%		&\\
Estrada para a Casa da Pólvora		&4				&1,34\%			&\\
Estrada do Engenho Velho		&8				&2,68\%			&\\
Boa Vista				&8				&2,68\%			&\\
Ladeira da Boa Vista			&2				&0,67\%			&\\
Ladeira do Acú				&2				&0,67\%			&\\
Estrada do Acú				&13				&4,35\%			&\\
Largo do Acú				&2				&0,67\%			&\\
Estrada do Acú para a 2 de Julho	&5				&1,67\%			&\\
Estrada de Brotas			&15				&5,02\%			&\\
Estrada da Cruz das Almas		&14				&4,68\%			&\\
Largo de Brotas				&39				&13,04\%		&\\
Estrada para a Cruz da Redenção		&4				&1,34\%			&\\
Largo da Cruz da Redenção		&4				&1,34\%			&\\
Campina					&2				&0,67\%			&\\
Candeal					&3				&1,00\%			&\\
Mariquita				&65				&21,74\%		&\\
Estrada do Sangradouro para o Matatu	&11				&3,68\%			&\\
Alto do Sangradouro			&4				&1,34\%			&\\
Estrada da Ubarana			&1				&0,33\%			&\\
Pomar					&5				&1,67\%			&\\
Lagoa					&1				&0,33\%			&\\
Estrada para a Armação			&7				&2,34\%			&\\
Várzea de Santo Antônio			&1				&0,33\%			&\\
Armação					&3				&1,00\%			&\\
\midrule
TOTAL					&324				&100,00\%		&\\
\bottomrule
\end{tabular} 
}
\end{table}
