\section{Como Salvador via o distrito}\label{sec:2.5}

PESQUISA NA IMPRENSA DE ÉPOCA, EM CURSO

\subsection{Imprensa}\label{subsec:2.5.1}




Certo E. M. Pestana publicou n'\textbf{A União Liberal} de 19 de março de 1853 um poema chamado ``Meu Sentir'', declamando de Estância as saudades que sentia da Bahia: ``Eu tenho saudades também de Brotas / Qu'o collo não curva à vil servidão''. Enigmáticas linhas.

Certo tenente Bernardino Jose de Almeida, que antes possuira bens e agora vivia na pobreza, morreu numa roça em Brotas e, por falta de meios, ficou tres dias insepulto, inumado por ato de benemerencia de certo escrivao Gularte\footnote{\textbf{O Alabama}, ano 14, serie 163, nº 1625, 25 nov. 1876}

Em 1871 o Dois de Julho era comemorado em Itapagipe e Brotas logo depois dos festejos oficiais\footnote{\textbf{O Reverbero}, ano 1, nº 14, 6 ago. 1871, p. 7}

O jornal \textbf{A Religião}, autoproclamado ``órgão da Igreja Catholica da Bahia'' e publicado sob os auspicios do arcebispo Luiz Antonio dos Santos, deixou de noticiar o movimento eclesiástico da freguesia de Brotas em 12 de junho de 1887   ``por ser quasi nenhum''.

\subsubsection{Notícias policiais}\label{subsubsec:2.5.1.1}

PESQUISA NA IMPRENSA DE ÉPOCA, EM CURSO

Brotas ainda era, no final do século XIX, lugar de sedução de menores\footnote{  ``Apresentou-se esta manha ao sr. subdelegado da freguezia de Brotas uma senhora, moradora ao Sangradouro, queixando-se de um tal Mattos que lhe raptara uma sua filha menor, cujo nome ignoramos. Aquela autoridade, procedendo as respectivas diligencias, conseguiu descobrir o lugar onde se achava depositada a referida menor e trata de capturar o sedutor.'' (\textbf{Diario de Noticias}, ano 7, nº 23, 3 out. 1881, p. 1.)}, 

A edição de 25 de abril de 1880 do semanário \textbf{A Gargalhada} denunciava

\begin{citacao}
\dots que o fiscal em exercicio na freguesia de Brotas, ao passo em que nao ve o estado das ruas, occupa-se em encher a estaçao de meninos e velhos, conductores de carroças, deixando impunes os valentoes e trampolineiros \dots
\end{citacao}

Dois africanos que moravam no caminho de Brotas para o Rio Vermelho foram encontrados mortos, e a policia prendeu um crioulo de mais de 60 anos como suspeito do latrocínio \footnote{Jornal da Bahia, ano XXII, número ilegível, 10 mar. 1875, p. 2}.

\subsubsection{Notícias políticas}\label{subsubsec:2.5.1.2}

PESQUISA NA IMPRENSA DE ÉPOCA, EM CURSO

Durante os conflitos da Independência, em setembro de 1822, a desembocadura da Estrada de Brotas foi guardada pelo 1º Batalhão Constitucional de Lisboa, por força de notícias de que batalhões milicianos da Torre dos Garcia D'Ávila estariam já em Brotas, a légua e meia de Salvador\footnote{\textbf{O Espelho}, nº 83, 03 set. 1822, p. 1; \textit{idem}, nº 110, 06 dez. 1822, p. 2}. Já no mês seguinte foram publicadas noticias de escaramuças entre tropas portuguesas e brasileiras no Rio Vermelho e em Brotas\footnote{\textbf{O Espelho}, nº 98, 25 out. 1822, p. 1}, e em dezembro tropas portuguesas incendiaram a casa de uma fazenda chamada Torre, homônima à dos Garcia D'Ávila, proxima daquela de um certo ``Machado da Boa Vista'', e fizeram o mesmo com outra fazenda chamada ``roça dos Mansos''\footnote{\textbf{O Espelho}, nº 8, 28 dez. 1822, p. 1}.

o caso de Cesar Zama

\subsection{Almanaques}\label{subsec:2.5.2}

DIFICULDADES DE FONTES: CONVERSAR COM ODETE

\subsubsection{Profissionais residentes}\label{subsubsec:2.5.2.1}

DIFICULDADES DE FONTES: CONVERSAR COM ODETE

Em 30 de novembro de 1881, o jornal \textbf{O Preceptor} anunciou em sua página 4 que o médico Eduardo Feliciano Castilho residia à rua 25 de Março, nº 119, e dava ``consultas grátis aos pobres''.

\subsubsection{Serviços}\label{subsubsec:2.5.2.2}

DIFICULDADES DE FONTES: CONVERSAR COM ODETE

Atraves de edital do Senado da Camara de 27 de janeiro de 1811, que estabeleceu cobradores para os quarenta e dois ``talhos'' de Salvador e os sete de seu termo (ou seja, dos distritos rurais), temos noticia da existencia de um destes pontos de venda de carne, que contava entao com seu respectivo cobrador; comparativamente, o Caminho Novo tinha quatro ``talhos'', o Taboao tinha oito, e o Sao Bento dezoito\footnote{\textbf{Idade d'Ouro do Brazil}, nº 19, 06 mar. 1811, pp. 2-3}.

O jornal quinzenal \textbf{A Lei}, num breve perfil biografico, indicou que em 1848 o brigadeiro Evaristo Ladislau e Silva ``concorreu para o melhoramento da estrada de Brotas''\footnote{\textbf{A Lei}, ano 2, nº 2, fev. 1877, pp. 2-3}.