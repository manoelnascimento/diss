\chapter*[Conclusão]{Conclusão}\label{concl}

\addcontentsline{toc}{chapter}{CONCLUSÃO}

O método empregue nesta pesquisa levou a uma longa contextuação sincrônica no primeiro capítulo, a um igualmente longo enraizamento territorial diacrônico no segundo capítulo e, enfim, à análise pormenorizada dos conflitos sociais na produção, apropriação e uso do espaço urbano em Brotas no terceiro capítulo, indo desde os investimentos em infraestruturas urbanas e serviços públicos até a evolução comparada da urbanização em cada logradouro, década a década, passando pela atuação dos engenheiros da Diretoria de Obras do município na produção do espaço urbano por meio de suas interferências ou omissões.

Este percurso permite, enfim, extrair conclusões. 

A urbanização de Brotas é parte do processo de \textit{descentralização} urbana de Salvador ocorrido durante todo o século XX, e o quadridecênio analisado é apenas seu começo. A estatística demográfica mostrou como outros distritos fronteiriços à mancha urbana consolidada --- Santo Antônio, Vitória, Pirajá, Mares --- também registraram incremento populacional no período, ao contrário dos distritos situados na mancha urbana consolidada. O destaque dado a Brotas apenas demonstra a importância de sua participação neste processo: embora somente o terceiro maior distrito em termos de população, foi aquele onde a urbanização parece ter ocorrido mais intensa, a julgar pela sua taxa de incremento populacional.

Internamente ao distrito, pode-se dizer que o antigo 1º Distrito e a área da Boa Vista / Engenho Velho assumem \textit{função central}, concentrando pontos comerciais, linhas de transporte e equipamentos coletivos. Outras áreas como Acupe, Matatu e o entorno da Cidade Balneária Amaralina permaneceram em seu caráter rural, ainda que a urbanização de áreas próximas possa tê-las afetado. 

O largo de Brotas, bem mais adiante, poderá ter exercido alguma centralização sobre áreas como Campinas, Várzea de Santo Antônio, Candeal, Pituba e outras mais afastadas, em especial por ser um dos pontos finais do bonde e pela quantidade de casas comerciais ali instaladas. \textit{Centralidade} na \textit{ruralidade}? Sim, pois talvez tenha sido exatamente o afastamento acentuado de todas estas áreas relativamente ao centro consolidado, bem como a enorme disponbilidade de terras e imóveis no entorno mais próximo para satisfazer a demanda habitacional crescente, os principais fatores contrários à urbanização destas áreas. Mesmo se a urbanização destas áreas mais afastadas fosse cogitada a sério como solução para as demandas habitacionais do período, o nível de investimento nas infraestruturas necessárias à sua urbanização era incompatível com um contexto de turbulências financeiras como aquelas causadas pelas reformas do porto e pelas reformas patrocinadas pelo governo estadual nos distritos da Sé e de São Pedro, ambas envolvendo alto comprometimento dos tesouros municipal e estadual e drenagem de recursos públicos por empresas privadas.

A \textit{segregação socioespacial} é traço marcante da urbanização de Brotas, tanto no que diz respeito à relação entre Brotas e Salvador quanto à relação entre áreas internas ao distrito. 

Todas as localidades componentes do distrito encontravam-se ligadas à malha urbana de Salvador por meio de longas vias terrestres --- estrada de Brotas, estrada da Boa Vista, estrada do Engenho Velho, estrada do Matatu, estrada da Pólvora, estrada do Acupe, estrada da Cruz das Almas, estrada Dois de Julho, estrada das Ubaranas, estrada das Armações, estrada da Várzea de Santo Antônio etc. --- algumas posteriormente usadas como leito para trilhos de bonde; não bastasse isso e também a geografia irregular do distrito, mesmo os moradores das localidades mais próximas à malha urbana tinham diante de si o dique do Tororó a estorvar-lhes a livre circulação. 

A esta segregação \textit{física} somou-se a segregação \textit{social}.Urbaniza-se o distrito, sim, pouco a pouco --- mas cada classe social urbaniza-o a seu modo. Formam territórios socialmente distintos dentro do próprio distrito. Partes do distrito como Boa Vista, Engenho Velho e o 1º Distrito apresentam-se como \textit{áreas de expansão urbana} onde concentraram-se pequenos e médios comerciantes, assim como funcionários públicos de médio escalão em localidades razoavelmente separadas daquelas onde trabalhadores braçais e artesanais das mais diversas naturezas residiam. O caráter balneário e veranista da Mariquita e de Amaralina, assim como o caráter proletário do Engenho Velho e de parte da Boa Vista, ou o caráter burguês e gestorial do antigo 1º Distrito, são todos indisfarçáveis. A urbanização de Brotas foi marcada por usos que não se mostraram, todavia, capazes de substituir concentração de postos de trabalho existente na malha urbana consolidada; por isto, sua urbanização concretizou mais uma etapa da separação entre trabalhadores e seus respectivos locais de trabalho em Salvador, iniciada durante a abolição da escravatura. 

Foram verificadas \textit{continuidades} importantes na urbanização do distrito. A mais notável é a continuidade do processo de urbanização observado no antigo 1º Distrito já nas últimas décadas do regime imperial. Continuou seu curso, também, o desenvolvimento da Mariquita como estância de vilegiatura; vendo as estatísticas imobiliárias, pode-se dizer que foi na Primeira República que o veranismo característico do Rio Vermelho expandiu-se além do rio Lucaia, levando à proliferação de novas construções na Mariquita. 

Permaneceram também durante a Primeira República o caráter rural e/ou pesqueiro do Matatu, da Quinta das Beatas, da Pituba, das Armações, de Campinas e da Várzea de Santo Antônio, mas trata-se de dois processos distintos. No Matatu e Quinta das Beatas, trata-se de uma urbanização pontilhística, esparsa, feita em meio às chácaras e laranjais, incapaz portanto de reverter o caráter semirrural destas áreas existente desde o século XIX; no caso da Pituba, das Armações, de Campinas de Brotas e da Várzea de Santo Antônio, dada a distância destas localidades frente à malha urbana consolidada, trata-se da manutenção de seu caráter rural e pesqueiro por força não das grandes distâncias, mas da manifesta falta de interesse dos capitalistas soteropolitanos em estender para tão longe a infraestrutura necessária a seu desenvolvimento.

No plano das \textit{sucessões}, destaca-se a transformação da velha fazenda Alagoa no loteamento veranístico Cidade Balneária Amaralina. Sobressaiu-se tal loteamento ainda mais por ter sido implantado numa herdade cuja cadeia sucessória de proprietários é complicadíssima; o loteamento poderá ter sido decidido, como em tantas outras situações, para resolver as questões reivindicatórias pela transformação da terra em dinheiro. Destaca-se também no campo das sucessões a urbanização acelerada na Boa Vista, no Engenho Velho de Brotas e na Estrada 2 de Julho, com sua transformação de zonas eminentemente rurais a áreas de moradia proletária periférica (com exceção da rua da Boa Vista). 

O \textit{magnetismo funcional} em Brotas só é significante no antigo 1º Distrito, cuja proximidade à rua J. J. Seabra tornou-o atrativo a comerciantes.

As características do desenvolvimento urbano visto em Brotas na Primeira República marcaram profundamente o território do distrito, ao ponto de lançar as bases para a determinação do caráter social futuro de áreas do distrito. Veja-se na atualidade a permanência do caráter proletário do Engenho Velho de Brotas, do Matatu, de Cosme de Farias; a passagem do caráter balneário e veranista de Amaralina para outro mais propriamente residencial, conservando um valor da terra compatível com a renda das mesmas frações de classe ao longo do tempo (comerciantes, funcionários públicos de médio escalão, profissionais liberais etc.); o ermo a que permaneceram condenadas as terras além de Amaralina até cerca de cinquenta anos atrás, excetuado os loteamentos Cidade Luz (1919/1937) e Boca do Rio (década de 1960); o ``vazio'' da área do Iguatemi até meados dos anos 1970\dots Se o caráter, a morfologia, a tipologia construtiva etc. não podem ser determinados somente observando-se o que antes existia no sítio construtivo, como a própria história das transformações territoriais ocorridas em Brotas entre os séculos XIX e XX o demonstram, a permanência do caráter social destas áreas ao longo do tempo, expresso por meio de tipologias construtivas características de cada classe social em momentos históricos bem delimitados, diz algo sobre os processos de urbanização de Salvador que ultrapassam a urbanização de Brotas; é neste distrito que se pode ver como o sentido da urbanização soteropolitana se deu em torno de tendências demográficas e imobiliárias muito anteriores ao período moderno ``canônico'' iniciado com a Semana de Urbanismo (1935) e consolidado pelo exaustivo, minudente, competente e marcante trabalho do Escritório do Plano de Urbanismo da Cidade do Salvador (EPUCS, 1942-1948).

Há que se observar, quanto aos investimentos públicos analisados, dois tipos de iniciativa: aquela da própria municipalidade ou do governo estadual, atendendo a interesses os mais diversos; e aquela oriunda de pressão pública feita por moradores de determinadas localidades.

Da parte da municipalidade e do governo estadual, tudo indica que só se movimentaram em seguida à mobilização dos moradores. Note-se, mais uma vez, que tal mobilização era feita por meio de protestos na imprensa; em toda a documentação pesquisada não foi possível encontrar nem uma só notícia tratando de qualquer das ações coletivas comuns ao repertório moderno (passeatas, comícios etc.). A julgar pela mobilização em prol da extensão dos ramais de bonde, é mais factível crer que esta ação política fosse obra de pessoas bem situadas na estratificação social e na estrutura de classes, aptas portanto a alcançar facilmente os gabinetes dos vereadores e mesmo do intendente com suas reivindicações, e que por isto não lançaram mão de artifícios tendentes a abalar seu próprio prestígio e posição. Daqueles que nada tinham a perder, de quem se poderia esperar este tipo de ação política --- os trabalhadores --- nenhuma notícia. A grande greve de 1919 e os muitos protestos com destruição de bondes não afetaram diretamente o território do distrito nem tiveram qualquer reivindicação --- exceto a melhoria nos transportes públicos -- que pudesse alterar significativamente os rumos da urbanização concertada entre a ação dos latifundistas loteadores, dos locadores especuladores e dos engenheiros protoplanejadores.

A esta altura já é possível responder às perguntas em torno das quais foi construída esta pesquisa. Como as reformas urbanas vividas por Salvador durante a Primeira República influenciaram a produção, apropriação e uso do espaço urbano em Brotas? E como o desenvolvimento urbano em Brotas influenciou as reformas urbanas?

Viu-se exaustivamente no terceiro capítulo como surgiram no distrito agentes de produção do espaço urbano cuja atuação foi bastante circunscrita ao distrito de Brotas, e como entre os grandes proprietários de terra do século XIX apenas os Amaral e os herdeiros de Antonio Teixeira de Carvalho tiveram algum papel de destaque na urbanização do distrito. Os Figueiredo e Melo e seus sucessores mantiveram o caráter rural das fazendas Campina Grande e Campina Pequena; se a Campina Grande passou a ser o Abrigo do Salvador e a Campina Pequena uma área de pequenos posseiros, foi pré-requisito de ambas as situações a preservação da \textit{integridade} destas duas herdades. O recolhimento dos Perdões não parece ter abandonado a prática de concessão de pequenos aforamentos rurais. Os descendentes e sucessores do visconde do Rio Vermelho deixaram como principal contribuição sua para a urbanização soteropolitana -- talvez única contribuição mesmo --- a venda de suas terras para a instalação dos equipamentos do sistema de abastecimento de água de Teodoro Sampaio. Sequer os Ribeiro Saldanha, legatários dos últimos fragmentos do latifúndio de Tomás da Silva Paranhos, embarcaram sem restrições na moda loteadora. 

Por outro lado, novos atores entraram em cena: José Visco, Emílio Cassiano da Silva, Raymundo da Cunha Pacheco, Clião Arouca, Durval de Souza Leite e tantos outros com investimentos de menor vulto nas ``avenidas'' e corredores de casas para alugar, representam, todos eles, um tipo de investidor já não mais vinculado às velhas herdades coloniais, mas o investidor imobiliário \textit{moderno}, cuja atividade centrava-se não mais no aforamento de terras, mas nos novos imóveis que construíam --- seja alugando-os, seja vendendo-os.

Nenhum dos agentes de produção do espaço urbano citados até o momento teria sido capaz de produzir, por si só, o processo de urbanização em Brotas. Agiram todos eles de uma só vez sobre o mesmo processo, aliando-se às vezes, entrechocando-se noutras; fingindo concórdia às vezes, fingindo discórdia noutras. O que resulta desta contínua confrontação é um espaço urbano produzido, apropriado e usado a cada momento segundo as possibilidades de ação dos agentes envolvidos em cada conjuntura, fortemente constrangidas pela sua situação na estrutura de classes da sociedade soteropolitana de então, pelo regime fundiário vigente em cada conjuntura, pelos acordos e desacordos estabelecidos entre si, em suma, por condições sobre as quais nem sempre tiveram governabilidade.

Viu-se que o território de Brotas foi produzido por meio de disputas pelos melhores lugares para moradia ou negócios, disputas estas expressivas de conflitos sociais mais profundos. Cumeadas e orla atlântica para a moradia e deleite da burguesia, dos latifundiários e dos gestores; encostas e vales para os proletários. Concentração das infraestruturas nas áreas destinadas à burguesia, aos latifundiários e aos gestores; atraso na sua implementação – quando a houve, claro – nas áreas ocupadas por proletários. Brotas exemplifica como a urbanização não é um processo unificado ou unívoco, mas um processo conflituoso, cheio de \textit{démarches} mesmo na escala distrital. As raízes destes conflitos, entretanto, só podem ser encontradas inserindo-se o território do distrito na dinâmica mais ampla da urbanização soteropolitana.

Diga-se, complementarmente, que a urbanização em Brotas envolve tanto conflitos na produção de novos espaços urbanos quanto na modificação das formas herdadas de territorialização. Se, como visto, a orla atlântica onde a família Amaral lançou seu famoso loteamento encontrava-se povoada talvez apenas pelo pessoal da velha fazenda Alagoa e seus posseiros e foreiros, que pouco ou nada pôde opor às mudanças no uso do espaço, no Matatu a parca urbanização foi sendo feita, ainda que a passos lentos, sobre um território formado por dezenas de pequenas posses rurais, elas próprias quiçá novas formas jurídicas, posteriores à Lei de Terras de 1850, de formas outras de apropriação do espaço diferentes daquelas previstas nas velhas Ordenações Filipinas e no Código Civil de 1916. Foi preciso, portanto, superar as formas pretéritas antes mesmo de transformar as antigas posses rurais em lotes urbanos.

Nada a estranhar. Aquilo que se manifestava em meio às classes capitalistas soteropolitanas durante a Primeira República como aversão ao ``passado colonial'', ao ``atraso'', era a aversão a formas de produção da vida cotidiana com fortíssima influência africana e proletária. A superação das formas de apropriação territorial avessas às regras jurídicas, dos usos do espaço capazes de fomentar qualquer veleidade de autonomia entre os trabalhadores, das formas de produção do espaço urbano que não implicassem na mercantilização de tal processo; a superação de tudo isto, enfim, foi um dos traços mais marcantes de todo este processo. O constante aperfeiçoamento dos marcos legais urbanísticos no período, demonstra como estes mecanismos de disciplinamento urbano 

Tomou-se como ponto de partida seguro e estabelecido para fixar um hipotético ``ponto de mutação'' da produção, apropriação e uso do espaço urbano de Brotas na Primeira República o curto período das reformas urbanas capitaneadas por José Joaquim Seabra. Viu-se aqui, ao contrário do que se esperava, como o ``pré-modernismo'' seabrista, tomado como eixo, campo precisamente delimitado, ponto de referência de tantas e quantas pesquisas até o momento realizadas, está longe de cumprir estas funções. Longe, inclusive, de ser o princípio motor de um \textit{zeitgeist} difuso a animar a produção, apropriação e uso do espaço urbano de Salvador como um todo. As determinações históricas do período devem ser buscadas não somente aí, mas também alhures.

Como explicar, por exemplo, o processo de urbanização de Brotas por meio do impulso seabrista quando a totalidade dos pareceres emitidos nos processos de autorização para obras de construção ou reforma no distrito durante as intervenções seabristas dizia estarem elas ``fora do perímetro'' das referidas intervenções, e que considerações de qualquer tipo acerca de tais pedidos (alinhamento principalmente, mas também a estética, a volumetria, a higiene etc.) estariam sob o arbítrio da Diretoria de Obras e da Inspetoria de Higiene, ambas da Intendência de Salvador? Aquilo que serve de ponto de partida seguro para explicar as intervenções nos distritos da Sé, São Pedro, Vitória, Conceição da Praia e Pilar, ao que tudo indica, pode não ter a mesma força explicativa para a urbanização em outros distritos urbanos e suburbanos de Salvador --- ao menos não diretamente. Daí ser possível extrapolar, ainda que hipoteticamente, uma das determinantes da urbanização de Brotas para distritos como Mares, Penha e Santo Antônio: \textbf{as intervenções seabristas interferiram apenas oblíqua e tangencialmente no desenvolvimento urbano destes distritos, por meio de migrações intraurbanas e interdistritais pautadas pelo baixo valor da terra e pela proximidade dos postos de trabalho industriais, comerciais e portuários garantida pelas linhas de bonde.}

Outra extrapolação: conquanto o volume de investimentos mobilizados nas intervenções seabristas tenha sido altíssimo e o Estado seja, indiscutivelmente, um agente importantíssimo de produção do espaço urbano soteropolitano no período estudado, é contraproducente tomá-lo como único agente no período. Ainda está por ser construída uma historiografia da urbanização de Salvador que esquadrinhe os investimentos \textit{privados} com o mesmo rigor a que foi submetida a série de melhoramentos urbanos conduzida por José Joaquim Seabra, e mesmo esta, a rigor e empregando anacronicamente uma expressão atual na falta de outra coetânea com significado parecido, deu-se nos moldes de uma gigantesca ``parceria público-privada''. Está também por ser construída uma historiografia semelhante que identifique tais agentes privados de acordo com suas classes sociais, que faça o mesmo quanto às tipologias arquitetônicas --- não custa lembrar que a \textit{casa de taipa} e a \textit{casa de palha} requerem uma arquitetura tanto quanto o palacete eclético\dots

Especialmente no que diz respeito ao processo de urbanização ocorrido nos distritos dos Mares, Penha e Santo Antônio, pode-se muito seguramente extrapolar os achados de Brotas como hipótese: conquanto o Estado fornecesse infraestruturas básicas --- arruamento, alguma iluminação, alguma pavimentação etc. --- coube a agentes privados não o papel \textit{indutor} do desenvolvimento, mas o de \textit{consolidador} deste desenvolvimento nos eixos induzidos. Do proprietário de fazendas, engenhos e chácaras aos pequenos posseiros periurbanos; do proletário em sua casinhola constantemente ameaçada de demolição por ``insalubridade'' e ``desacordo com as posturas municipais'' ao burguês encastelado em seu palacete eclético; do especulador das evoneias e corredores de casas ao loteador clandestino; todos, literalmente todos, materializaram e deram sentido ao desenvolvimento urbano em Brotas, ora seguindo as diretrizes propostas pelo Estado, ora ignorando-as, ora enfrentando-as. A curiosa dialética assim formada, embora passível de descrição no puro plano das hipóteses, pode ser tomada como ponto de partida para a investigação sobre a urbanização noutros distritos, pois no que diz respeito a Brotas o pouco que foi possível pesquisar indica tratar-se esta ``dialética'' de uma realidade insofismável. Poderá ser a base para uma leitura da história da urbanização soteropolitana vista não mais a partir de eventos-chave ou da assunção de um ou outro fenômeno como indutor do desenvolvimento. Poderá ser a base para uma leitura deste fenômeno \textit{enquanto processo}: ou seja, pela compreensão da \textit{totalidade} em que se inseriu, pela caracterização dos \textit{sujeitos históricos} que nela interferiram, pela identificação dos \textit{conflitos sociais} que os mobilizaram e pela análise das \textit{relações espaciais} que por meio deles se constituíram. O resultado poderá ser uma história que, enfim, dê o merecido lugar a quem a construiu.

Mas por quê fazê-la? A quem interessa? 

Exemplifica tal necessidade uma característica territorial encontrada primeiro quase por acaso, depois tornada um dos eixos desta pesquisa: a pertinência dos limites fundiários das antigas fazendas e de seus processos de fragmentação e loteamento como critério objetivo para a definição da identidade de bairros. A lei municipal 9.278/2017, que ``dispõe sobre a delimitação e denominação dos bairros do Município de Salvador'', identifica em seu artigo 2º determinada área como ``bairro'' a partir dos seguintes critérios: \textit{identidade histórica} e \textit{relativa autonomia no contexto da cidade}; \textit{identidade e pertencimento dos residentes e usuários}, \textit{utilização dos mesmos equipamentos e serviços comunitários}, \textit{relações de vizinhança} e \textit{reconhecimento dos limites pelo mesmo nome}. Em suma: ``bairro'' é \textit{aquilo que seus moradores disserem que é}. Louvável como seja tamanha racionalização em torno da identidade local, sabe-se que isto tudo é cópia quase literal daquilo anteriormente definido pela equipe do monumental estudo \textbf{Caminho das Águas em Salvador} como bairro:

\begin{citacao}
O conceito de bairro nesse trabalho, fruto de uma construção coletiva, reporta-nos a um conjunto de relações socioambientais com as seguintes características: \textit{unidade territorial, com densidade histórica e relativa autonomia no contexto urbano-ambiental, que incorpora as noções de identidade e pertencimento dos moradores que o constituem; que utilizam os mesmos equipamentos e serviços comunitários; que mantêm relações de vizinhança e que reconhecem seus limites pelo mesmo nome.} Assim circunscrito esse conceito articula elementos de natureza objetiva e subjetiva – sendo a expressão das relações instituídas no cotidiano da vida da cidade, como das relações entre sociedade e natureza. Por localidade compreendemos uma porção menor do território, inserida parcial ou totalmente em um bairro, sem centralidade definida e que apresenta características socioeconômicas similares. A localidade possui elementos específicos da estruturação e complexidade urbana, podendo ser um loteamento ou um conjunto habitacional de pequeno ou médio porte que se tornou referência; uma pequena ocupação informal ou uma ocupação ao longo de uma avenida. Além disso, foi utilizado por esse trabalho o conceito de centro de bairro, aqui compreendido como uma área para a qual convergem e se articulam os principais fluxos do bairro ou da região, dotado de variedade de serviços, infraestrutura e acessibilidade.

Além das noções de pertencimento e de identidade definiu-se como critério a existência de unidade de saúde (pública, privada ou comunitária); a existência de unidade de ensino que ofereça a partir da sexta série do ensino fundamental; a existência de logradouro categorizado pela Prefeitura Municipal do Salvador como via coletora (ou equivalente) e a existência de transporte público regulamentado. Destes critérios, de caráter mais objetivo, fez-se necessária a presença de 3 destes, para a conversão de uma localidade em bairro \cite[pp.~3-4]{santos_aguas_2010}.
\end{citacao}

Nota-se nas definições da lei e do estudo como a relação entre a ``densidade'' ou ``identidade histórica'' e os ``loteamentos'' e ``conjuntos habitacionais'' parece cindida. Estes dois últimos definem a identidade de uma localidade, mas só o fazem quanto a bairros inteiros depois de muitas mediações. Do ponto de vista adotado nesta pesquisa, esta ``identidade histórica'' não é critério subjetivo, mas \textit{objetivo}, porque calcado na longa história da fragmentação das herdades tomadas aqui como unidade territorial de referência. As localidades, os logradouros, na pesquisa aqui exposta, só existem porque precedidos pelo processo que as originou --- a que hoje chamaríamos de ``incorporação imobiliária'', mas que o evidente anacronismo conceitual não permite remeter ao período estudado.

Se o ``loteamento'' e o ``conjunto habitacional'' explicam a ``identidade histórica'' de uma ``localidade'', os resultados desta pesquisa mostram como a história fundiária explica também o bairro, bastando a longa, paciente e criteriosa reconstituição histórica para evidenciá-lo. Se tal reconstituição se fizer pela ótica dos conflitos sociais, explicará também, como visto, a persistência de sua ocupação por determinadas classes sociais e suas frações; explicará a crônica deficiência de infraestruturas urbanas, equipamentos comunitários, serviços públicos e áreas de lazer e cultura em alguns lugares e sua concentração em outros; explicará por que alguns bairros são ``bonitos'', ``ordenados'', ``planejados'' enquanto outros criam no senso comum a noção --- aliás falsa --- de que cidades como Salvador ``não têm planejamento''; explicará por que alguns moram em lugares salubres e outros, nos pântanos, charcos, ribeiras e fundos de vale.

A análise da história fundiária e territorial de Brotas pela ótica dos conflitos sociais mostrou também como certas ``fronteiras'' foram instituídas pelos limites das herdades. Não pela simples pertença, mas pela complexa interação entre os muitos processos de fragmentação e loteamento de herdades não raro contíguas: as defasagens em seus inícios, as disparidades no preço dos lotes, as velocidades distintas com que eram totalmente vendidos os lotes, as particularidades dos padrões construtivos e as consequentes diferenças nos preços das primeiras construções, tudo isto e outras circunstâncias mais terminam por criar as bases materiais para que sejam os integrantes de uma classe ou fração de classe social, e não de outras, a ocupar aquele espaço por longa duração. É o que fez a urbanização ``pequeno-burguesa'' da Boa Vista distinguir-se da urbanização ``para proletários'' verificada no Engenho Velho; é o que fez do Nordeste algo diferente de Amaralina; é o que tornou tão desiguais a Pituba e a Boca do Rio, outrora igualadas na derrelição; e por aí vai. Mesmo bairros contíguos tanto no espaço quanto na composição social, como Engenho Velho de Brotas, Acupe, Fazenda Garcia e Engenho Velho da Federação (estes dois últimos situados no distrito da Vitória), separados pelos transpassáveis vales do Lucaia e do Ogunjá, encontram explicação para sua distinção identitária nos processos de fragmentação e loteamento das herdades sobre as quais foram construídos, e não apenas nos acidentes geográficos que as distinguem. São estes processos que explicam a ``identidade histórica'' formativa de um bairro, que deste modo passa a ter um conteúdo mais objetivo.

Tudo isto ao mesmo tempo em que este processo, espera-se, poderá ter fornecido aos moradores dos logradouros, localidades e bairros integrantes do atual subdistrito de Brotas alguma noção de pertencimento histórico, tanto ao lugar quanto à história da cidade, dos seus conflitos sociais constitutivos, do longo caudal de disputas pelo espaço em que estão inescapavelmente envolvidos. Espera-se, com esta pesquisa, ter sido possível explicar por que os conflitos e lutas sociais pela produção, apropriação e uso do espaço urbano fizeram Brotas ser o que é hoje --- de caso pensado, não ``espontaneamente''.