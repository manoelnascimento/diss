\chapter*[Conclusão]{Conclusão}\label{concl}

\addcontentsline{toc}{chapter}{CONCLUSÃO}

É prciso dizê-lo sem rodeios: a pesquisa até aqui exposta ultrapassou em muito aquilo que de início fora projetado. Como em qualquer expedição exploratória, entretanto, aquilo que está por ser descoberto termina sendo muito mais rico, muito mais complexo, muito mais plural que o esperado; serve-lhe -- ao explorador -- apenas para estabelecer outros pontos de partida para viagens e explorações futuras.

Falemos primeiramente daquilo que diz respeito ao distrito para em seguida extrapolar algumas conclusões.

Viu-se que o território de Brotas foi produzido por meio de disputas pelos melhores lugares para moradia ou negócios, disputas estas expressivas de conflitos sociais mais profundos. Cumeadas e orla atlântica para a moradia e deleite da burguesia, dos latifundiários e dos gestores, encostas e vales para os proletários. Concentração das infraestruturas nas áreas destinadas à burguesia, aos latifundiários e aos gestores, atraso na sua implementação -- quando houve implementação, claro -- nas áreas ocupadas por proletários. Brotas exemplifica como a urbanização não é um processo unificado ou unívoco, mas um processo conflituoso, cheio de \textit{démarches} mesmo na escala distrital. As raízes destes conflitos, entretanto, só podem ser encontradas inserindo-se o território do distrito na dinâmica mais ampla da urbanização soteropolitana, como visto XXXX

Diga-se, complementarmente, que a urbanização em Brotas envolve tanto conflitos na produção de novos espaços urbanos quanto na modificação das formas herdadas de territorialização. Se, como visto, a orla atlântica onde a família Amaral lançou seu famoso loteamento encontrava-se povoada apenas pelo pessoal da velha fazenda Amaralina, que pouco ou nada puderam opor às mudanças no uso do espaço, no Matatu a urbanização foi feita sobre um território formado por dezenas de pequenas posses rurais, elas próprias quiçá novas formas jurídicas, posteriores à Lei de Terras de 1850, de formas outras de apropriação do espaço diferentes daquelas previstas nas velhas Ordenações Filipinas e no Código Civil de 1916. Foi preciso, portanto, superar as formas pretéritas antes mesmo de transformar as antigas posses rurais em lotes urbanos. Nada a estranhar. Aquilo que se manifestava na cultura soteropolitana como aversão ao ``passado colonial'', ao ``atraso'', como visto em XXXXXX, era a aversão a formas de produção da vida cotidiana com fortíssima influência africana e proletária. A superação das formas de apropriação territorial avessas às regras jurídicas, dos usos do espaço capazes de fomentar qualquer veleidade de autonomia entre os trabalhadores, das formas de produção do espaço urbano que não implicassem na mercantilização de tal processo; a superação de tudo isto, enfim, foi um dos traços mais marcantes de todo este processo.

Tomou-se como ponto de partida seguro e estabelecido o curto período das reformas urbanas capitaneadas por José Joaquim Seabra para fixar um hipotético ``ponto de mutação'' da produção, apropriação e uso do espaço urbano de Brotas na Primeira República.  O ``pré-modernismo'' seabrista, tomado como eixo, campo precisamente delimitado, ponto de referência de tantas e quantas pesquisas até o momento realizadas, está longe de ser, inclusive, o princípio motor de um como que \textit{zeitgeist} difuso a animar a produção, apropriação e uso do espaço urbano de Salvador como um todo. As determinações históricas do período devem ser buscadas não somente aí, mas também alhures. 

Como explicar, por exemplo, o processo de urbanização de Brotas por meio do impulso seabrista quando a totalidade dos pareceres emitidos nos processos de autorização para obras de construção ou reforma no distrito durante as intervenções seabristas dizia estarem elas ``fora do perímetro'' das referidas intervenções, e que considerações de qualquer tipo acerca de tais pedidos (alinhamento principalmente, mas também a estética, a volumetria, a higiene etc.) estariam sob o arbítrio da Diretoria de Obras e da Inspetoria de Higiene, ambas da Intendência de Salvador? Aquilo que serve de ponto de partida seguro para explicar as intervenções nos distritos da Sé, São Pedro, Vitória, Conceição da Praia e Pilar, ao que tudo indica, pode não ter a mesma força explicativa para a urbanização em outros distritos urbanos e suburbanos de Salvador. É possível inclusive extrapolar, ainda que hipoteticamente, uma das determinantes da urbanização de Brotas para distritos como Mares, Penha e Santo Antônio: \textbf{as intervenções seabristas interferem apenas de modo indireto no desenvolvimento urbano destes distritos, por meio de migrações intraurbanas e interdistritais pautadas na proximidade dos postos de trabalho industriais e no baixo valor da terra} (ARTIGO SOBRE VALORES E PREÇOS EM MARX).

Outro ponto de partida passível de extrapolação: conquanto o volume de investimentos mobilizados nas intervenções seabristas tenha sido altíssimo e o Estado seja, indiscutivelmente, um agente importantíssimo de produção do espaço urbano soteropolitano no período estudado, é contraproducente tomá-lo como único agente no período. Ainda está por ser construída uma historiografia da urbanização de Salvador que esquadrinhe os investimentos \textit{privados} com o mesmo rigor a que foi submetida a série de melhoramentos urbanos conduzida por José Joaquim Seabra, e mesmo esta, a rigor e empregando anacronicamente uma expressão atual na falta de outra coetânea com significado parecido, deu-se nos moldes de uma gigantesca ``parceria público-privada''. Está também por ser construída uma historiografia semelhante que identifique tais agentes privados de acordo com suas classes sociais, que faça o mesmo quanto às tipologias arquitetônicas -- não custa lembrar que a \textit{casa de taipa} e a \textit{casa de palha} requerem uma arquitetura tanto quanto o palacete eclético \dots Especialmente no que diz respeito ao processo de urbanização ocorrido nos distritos dos Mares, Penha e Santo Antônio, pode-se muito seguramente extrapolar os achados de Brotas como hipótese: conquanto o Estado fornecesse infraestruturas básicas -- arruamento, alguma iluminação, escolas, alguma pavimentação etc. -- coube a agentes privados não o papel \textit{indutor} do desenvolvimento, mas o de \textit{consolidador} deste desenvolvimento nos eixos induzidos. Do proprietário de fazendas, engenhos e chácaras aos pequenos posseiros periurbanos; do proletário em sua casinhola constantemente ameaçada de demolição por ``insalubridade'' e ``desacordo com as posturas municipais'' ao burguês encastelado em seu palacete eclético; do especulador das evoneias e corredores de casas ao loteador clandestino; todos, literalmente todos, materializaram e dram sentido ao desenvolvimento urbano em Brotas, ora seguindo as diretrizes propostas pelo Estado, ora ignorando-as, ora enfrentando-as. A curiosa dialética assim formada, embora passível de descrição no puro plano das hipóteses, pode ser tomada como ponto de partida para a investigação sobre a urbanização noutros distritos, pois no que diz respeito a Brotas o pouco que foi possível pesquisar indica tratar-se de uma realidade insofismável.

Ainda outro aspecto. O desenvolvimento urbano visto em Brotas no período marcou profundamente o território do distrito, ao ponto de lançar as bases para a determinação do caráter futuro de áreas do distrito. Veja-se a permanência do caráter proletário do Engenho Velho de Brotas, do Matatu, de Cosme de Farias; a passagem do caráter balneário e veranista de Amaralina para outro mais propriamente residencial, conservando um valor da terra compatível com a renda das mesmas frações de classe ao longo do tempo (comerciantes, funcionários públicos de médio escalão, profissionais liberais etc.); o ermo a que permaneceram condenadas as terras além de Amaralina até cerca de cinquenta anos atrás, excetuado os loteamentos Cidade Luz (1919/1937) e Boca do Rio (década de 1960); o ``vazio'' da área do Iguatemi até meados dos anos 1970 \dots Se o caráter, a morfologia, a tipologia construtiva etc. não podem ser determinados somente observando-se o que lá existia no passado, como a própria história das transformações territoriais ocorridas em Brotas entre os séculos XIX e XX o demonstram, a permanência do caráter \textit{social} destas áreas ao longo do tempo, expresso por meio de tipologias construtivas características de cada classe social em momentos históricos bem delimitados, diz algo sobre os processos de urbanização de Salvador que ultrapassam a urbanização de Brotas; é neste distrito que se pode ver como o sentido da urbanização soteropolitana se deu em torno de tendências demográficas e imobiliárias muito anteriores ao período moderno ``canônico'' iniciado com a Semana de Urbanismo (1935) e consolidado pelo exaustivo, minudente, competente e marcante trabalho do Escritório do Plano de Urbanismo da Cidade do Salvador (EPUCS, 1942-1948).