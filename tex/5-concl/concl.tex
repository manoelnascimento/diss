\chapter*[Conclusão]{Conclusão}\label{concl}

\addcontentsline{toc}{chapter}{CONCLUSÃO}

O método empregue nesta pesquisa levou a uma longa contextuação sincrônica no primeiro capítulo, a um igualmente longo enraizamento territorial diacrônico no segundo capítulo e, enfim, à análise detalhada dos conflitos sociais na produção, apropriação e uso do espaço urbano em Brotas no terceiro capítulo, indo desde os investimentos em infraestruturas urbanas e serviços públicos até a evolução comparada da urbanização em cada logradouro, década a década.

Este percurso permite, enfim, extrair conclusões.

\section{Centralização, descentralização, sucessão, inércia, segregação, coesão}\label{subsec:3.3.1}

A urbanização de Brotas é parte do processo de \textit{descentralização} urbana de Salvador ocorrido durante todo o século XX, e o quadridecênio analisado é apenas seu começo. A estatística demográfica mostrou como outros distritos fronteiriços à mancha urbana consolidada --- Santo Antônio, Vitória, Pirajá, Mares --- também registraram incremento populacional no período, ao contrário dos distritos situados na mancha urbana consolidada. O destaque dado a Brotas apenas demonstra a importância de sua participação neste processo: embora somente o terceiro maior distrito em termos de população, foi aquele onde a urbanização parece ter ocorrido mais intensa, a julgar pela sua taxa de incremento populacional.

Internamente ao distrito, pode-se dizer que o antigo 1º Distrito e a área da Boa Vista / Engenho Velho assumem \textit{função central}, concentrando pontos comerciais, linhas de transporte e equipamentos coletivos. Outras áreas como Acupe, Matatu e o entorno da Cidade Balneária Amaralina permaneceram em seu caráter rural, ainda que a urbanização de áreas próximas possa tê-las afetado. 

O largo de Brotas, bem mais adiante, poderá ter exercido alguma centralização sobre áreas como Campinas, Várzea de Santo Antônio, Candeal, Pituba e outras mais afastadas, em especial por ser um dos pontos finais do bonde e pela quantidade de casas comerciais ali instaladas. \textit{Centralidade} na \textit{ruralidade}? Sim, pois talvez tenha sido exatamente o afastamento acentuado de todas estas áreas relativamente ao centro consolidado, bem como a enorme disponbilidade de terras e imóveis no entorno mais próximo para satisfazer a demanda habitacional crescente, os principais fatores contrários à urbanização destas áreas. Mesmo se a urbanização destas áreas mais afastadas fosse cogitada a sério como solução para as demandas habitacionais do período, o nível de investimento nas infraestruturas necessárias à sua urbanização era incompatível com um contexto de turbulências financeiras como aquelas causadas pelas reformas do porto e pelas reformas patrocinadas pelo governo estadual nos distritos da Sé e de São Pedro, ambas envolvendo alto comprometimento dos tesouros municipal e estadual e drenagem de recursos públicos por empresas privadas.

A \textit{segregação socioespacial} é traço marcante da urbanização de Brotas, tanto no que diz respeito à relação entre Brotas e Salvador quanto à relação entre áreas internas ao distrito. 


Globalmente considerado no processo soteropolitano de urbanização referente ao período estudado, partes do distrito como Boa Vista, Engenho Velho e o 1º Distrito apresentam-se como \textit{áreas de expansão urbana} onde concentraram-se pequenos e médios comerciantes, assim como funcionários públicos de médio escalão em localidades razoavelmente separadas daquelas onde trabalhadores braçais e artesanais das mais diversas naturezas residiam. O caráter balneário e veranista da Mariquita e de Amaralina, assim como o caráter proletário do Engenho Velho / Boa Vista e o caráter burguês e gestorial do antigo 1º Distrito são indisfarçáveis. Urbaniza-se o distrito, sim, a pouco e pouco -- mas cada classe social urbaniza-o a seu modo, quando o fazem. Formam territórios socialmente distintos dentro do próprio distrito.

Verificou-se a \textit{continuidade} do processo de valorização observado no antigo 1º Distrito já nas últimas décadas do regime imperial; pode-se dizer o mesmo quanto ao desenvolvimento da Mariquita como estância de vilegiatura, e também ao caráter rural e/ou pesqueiro do Matatu, da Quinta das Beatas, da Pituba, das Armações, de Campinas e da Várzea de Santo Antônio. Há \textit{sucessões} importantes, como a transformação de Amaralina em cidade balneária e a metamorfose da Boa Vista / Engenho Velho e da Estrada 2 de Julho des zonas eminentemente rurais a áreas de moradia proletária periférica (com exceção da rua da Boa Vista).

O \textit{magnetismo funcional} em Brotas só é significante no antigo 1º Distrito, cuja proximidade à rua J. J. Seabra torna-o atrativo a comerciantes.

Quem ``puxa'' para cima a média total de pedidos de licença protocolados (82 pedidos, aproximadamente) são as áreas da Boa Vista / Engenho Velho, o antigo 1º Distrito e o Matatu; mesmo o loteamento da Cidade Balneária Amaralina fica ligeiramente abaixo da média de pedidos protocolados, e as demais áreas encontram-se todas abaixo da média.

Consideradas as médias de pedidos de licença protocolados por década (8 para a primeira, 16 para a segunda, 27 para a terceira e 31 para a quarta), mais uma vez são a Boa Vista / Engenho Velho, o antigo 1 o Distrito e o Matatu quem puxa as médias para cima. Na primeira década, ficam abaixo da média o complexo fundiário Amaralina-Pituba, 

Ainda outro aspecto: o desenvolvimento urbano visto em Brotas no período marcou profundamente o território do distrito, ao ponto de lançar as bases para a determinação do caráter futuro de áreas do distrito. Veja-se na atualidade a permanência do caráter proletário do Engenho Velho de Brotas, do Matatu, de Cosme de Farias; a passagem do caráter balneário e veranista de Amaralina para outro mais propriamente residencial, conservando um valor da terra compatível com a renda das mesmas frações de classe ao longo do tempo (comerciantes, funcionários públicos de médio escalão, profissionais liberais etc.); o ermo a que permaneceram condenadas as terras além de Amaralina até cerca de cinquenta anos atrás, excetuado os loteamentos Cidade Luz (1919/1937) e Boca do Rio (década de 1960); o ``vazio'' da área do Iguatemi até meados dos anos 1970\dots Se o caráter, a morfologia, a tipologia construtiva etc. não podem ser determinados somente observando-se o que antes existia no sítio construtivo, como a própria história das transformações territoriais ocorridas em Brotas entre os séculos XIX e XX o demonstram, a permanência do caráter social destas áreas ao longo do tempo, expresso por meio de tipologias construtivas características de cada classe social em momentos históricos bem delimitados, diz algo sobre os processos de urbanização de Salvador que ultrapassam a urbanização de Brotas; é neste distrito que se pode ver como o sentido da urbanização soteropolitana se deu em torno de tendências demográficas e imobiliárias muito anteriores ao período moderno ``canônico'' iniciado com a Semana de Urbanismo (1935) e consolidado pelo exaustivo, minudente, competente e marcante trabalho do Escritório do Plano de Urbanismo da Cidade do Salvador (EPUCS, 1942-1948).

\section{Natureza das pressões por investimentos públicos}\label{subsec:3.3.2}

Há que se observar, quanto aos investimentos públicos analisados, dois tipos de iniciativa: aquela da própria municipalidade ou do governo estadual, atendendo a interesses os mais diversos; e aquela oriunda de pressão pública feita por moradores de determinadas localidades, em especial por meio da imprensa.

\section{Tendência por determinada ordem e sentido no espaço público}\label{subsec:3.3.3}



\section[As reformas de Seabra e Brotas: influências recíprocas]{As reformas de Seabra e Brotas: influências recíprocas}\label{subsec:3.3.4}

A esta altura já é possível responder às perguntas em torno das quais foi construída esta pesquisa. Como as reformas urbanas vividas por Salvador durante a Primeira República influenciaram a produção, apropriação e uso do espaço urbano em Brotas? E como o desenvolvimento urbano em Brotas influenciou as reformas urbanas?

Viu-se exaustivamente no terceiro capítulo como surgiram no distrito agentes de produção do espaço urbano cuja atuação foi bastante circunscrita ao distrito de Brotas, e como entre os grandes proprietários de terra

Nenhum dos agentes de produção do espaço urbano citados até o momento teria sido capaz de produzir, por si só, o processo de urbanização em Brotas. Agiram todos eles de uma só vez sobre o mesmo processo, aliando-se às vezes, entrechocando-se noutras; fingindo concórdia às vezes, fingindo discórdia noutras. O que resulta desta contínua confrontação é um espaço urbano produzido, apropriado e usado a cada momento segundo as possibilidades de ação dos agentes envolvidos em cada conjuntura, fortemente constrangidas pela sua situação na estrutura de classes da sociedade soteropolitana de então, pelo regime fundiário vigente em cada conjuntura, pelos acordos e desacordos estabelecidos entre si, em suma, por condições sobre as quais nem sempre tiveram governabilidade.

Viu-se que o território de Brotas foi produzido por meio de disputas pelos melhores lugares para moradia ou negócios, disputas estas expressivas de conflitos sociais mais profundos. Cumeadas e orla atlântica para a moradia e deleite da burguesia, dos latifundiários e dos gestores; encostas e vales para os proletários. Concentração das infraestruturas nas áreas destinadas à burguesia, aos latifundiários e aos gestores; atraso na sua implementação – quando a houve, claro – nas áreas ocupadas por proletários. Brotas exemplifica como a urbanização não é um processo unificado ou unívoco, mas um processo conflituoso, cheio de \textit{démarches} mesmo na escala distrital. As raízes destes conflitos, entretanto, só podem ser encontradas inserindo-se o território do distrito na dinâmica mais ampla da urbanização soteropolitana.

Diga-se, complementarmente, que a urbanização em Brotas envolve tanto conflitos na produção de novos espaços urbanos quanto na modificação das formas herdadas de territorialização. Se, como visto, a orla atlântica onde a família Amaral lançou seu famoso loteamento encontrava-se povoada talvez apenas pelo pessoal da velha fazenda Alagoa e seus posseiros e foreiros, que pouco ou nada pôde opor às mudanças no uso do espaço, no Matatu a parca urbanização foi sendo feita, ainda que a passos lentos, sobre um território formado por dezenas de pequenas posses rurais, elas próprias quiçá novas formas jurídicas, posteriores à Lei de Terras de 1850, de formas outras de apropriação do espaço diferentes daquelas previstas nas velhas Ordenações Filipinas e no Código Civil de 1916. Foi preciso, portanto, superar as formas pretéritas antes mesmo de transformar as antigas posses rurais em lotes urbanos.

Nada a estranhar. Aquilo que se manifestava em meio às classes capitalistas soteropolitanas durante a Primeira República como aversão ao ``passado colonial'', ao ``atraso'', era a aversão a formas de produção da vida cotidiana com fortíssima influência africana e proletária. A superação das formas de apropriação territorial avessas às regras jurídicas, dos usos do espaço capazes de fomentar qualquer veleidade de autonomia entre os trabalhadores, das formas de produção do espaço urbano que não implicassem na mercantilização de tal processo; a superação de tudo isto, enfim, foi um dos traços mais marcantes de todo este processo. O constante aperfeiçoamento dos marcos legais urbanísticos no período, demonstra como estes mecanismos de disciplinamento urbano 

Tomou-se como ponto de partida seguro e estabelecido para a pesquisa cuja exposição já se vai concluindo o curto período das reformas urbanas capitaneadas por José Joaquim Seabra para fixar um hipotético ``ponto de mutação'' da produção, apropriação e uso do espaço urbano de Brotas na Primeira República. O ``pré-modernismo'' seabrista, tomado como eixo, campo precisamente delimitado, ponto de referência de tantas e quantas pesquisas até o momento realizadas, está longe de ser, inclusive, o princípio motor de um \textit{zeitgeist} difuso a animar a produção, apropriação e uso do espaço urbano de Salvador como um todo. As determinações históricas do período devem ser buscadas não somente aí, mas também alhures.

Como explicar, por exemplo, o processo de urbanização de Brotas por meio do impulso seabrista quando a totalidade dos pareceres emitidos nos processos de autorização para obras de construção ou reforma no distrito durante as intervenções seabristas dizia estarem elas ``fora do perímetro'' das referidas intervenções, e que considerações de qualquer tipo acerca de tais pedidos (alinhamento principalmente, mas também a estética, a volumetria, a higiene etc.) estariam sob o arbítrio da Diretoria de Obras e da Inspetoria de Higiene, ambas da Intendência de Salvador? Aquilo que serve de ponto de partida seguro para explicar as intervenções nos distritos da Sé, São Pedro, Vitória, Conceição da Praia e Pilar, ao que tudo indica, pode não ter a mesma força explicativa para a urbanização em outros distritos urbanos e suburbanos de Salvador --- ao menos não diretamente. Daí ser possível extrapolar, ainda que hipoteticamente, uma das determinantes da urbanização de Brotas para distritos como Mares, Penha e Santo Antônio: \textbf{as intervenções seabristas interferiram apenas de modo indireto no desenvolvimento urbano destes distritos, por meio de migrações intraurbanas e interdistritais pautadas pelo baixo valor da terra e pela proximidade dos postos de trabalho industriais, comerciais e portuários garantida pelas linhas de bonde.}

Outra extrapolação: conquanto o volume de investimentos mobilizados nas intervenções seabristas tenha sido altíssimo e o Estado seja, indiscutivelmente, um agente importantíssimo de produção do espaço urbano soteropolitano no período estudado, é contraproducente tomá-lo como único agente no período. Ainda está por ser construída uma historiografia da urbanização de Salvador que esquadrinhe os investimentos \textit{privados} com o mesmo rigor a que foi submetida a série de melhoramentos urbanos conduzida por José Joaquim Seabra, e mesmo esta, a rigor e empregando anacronicamente uma expressão atual na falta de outra coetânea com significado parecido, deu-se nos moldes de uma gigantesca ``parceria público-privada''. Está também por ser construída uma historiografia semelhante que identifique tais agentes privados de acordo com suas classes sociais, que faça o mesmo quanto às tipologias arquitetônicas – não custa lembrar que a \textit{casa de taip}a e a \textit{casa de palha} requerem uma arquitetura tanto quanto o palacete eclético\dots

Especialmente no que diz respeito ao processo de urbanização ocorrido nos distritos dos Mares, Penha e Santo Antônio, pode-se muito seguramente extrapolar os achados de Brotas como hipótese: conquanto o Estado fornecesse infraestruturas básicas – arruamento, alguma iluminação, alguma pavimentação etc. – coube a agentes privados não o papel \textit{indutor} do desenvolvimento, mas o de \textit{consolidador} deste desenvolvimento nos eixos induzidos. Do proprietário de fazendas, engenhos e chácaras aos pequenos posseiros periurbanos; do proletário em sua casinhola constantemente ameaçada de demolição por ``insalubridade'' e ``desacordo com as posturas municipais'' ao burguês encastelado em seu palacete eclético; do especulador das evoneias e corredores de casas ao loteador clandestino; todos, literalmente todos, materializaram e dram sentido ao desenvolvimento urbano em Brotas, ora seguindo as diretrizes propostas pelo Estado, ora ignorando-as, ora enfrentando-as. A curiosa dialética assim formada, embora passível de descrição no puro plano das hipóteses, pode ser tomada como ponto de partida para a investigação sobre a urbanização noutros distritos, pois no que diz respeito a Brotas o pouco que foi possível pesquisar indica tratar-se de uma realidade insofismável. 