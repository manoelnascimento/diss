\begin{table}[!htp]
\IBGEtab{
\caption{Valor locativo médio dos imóveis nos distritos arrolados pelo município de Salvador em 1924}\label{tab:valorlocativo1924geral}}
{
\begin{tabular}{rr}
\hline
Locais	&Valor locativo médio por imóvel\\
\hline
\hline
\multicolumn{2}{c}{Urbanos}\\
\hline
Sé	&2:990\$060\\
São Pedro	&2:045\$480\\
Passo	&1:923\$270\\
Conceição da Praia	&8:688\$030\\
Pilar	&1:897\$360\\
Mares	&674\$900\\
Nazaré	&1:291\$690\\
Vitória	&1:007\$210\\
Santana	&1:118\$060\\
Penha	&666\$840\\
Santo Antônio	&475\$910\\
Brotas	&493\$540\\
Valor médio urbanos	&1:939\$360\\
\hline
\multicolumn{2}{c}{Suburbanos}\\
\hline
Itapuã	&14\$210\\
Pirajá	&974\$100\\
Paripe	&84\$320\\
Passé	&611\$380\\
Matoim	&39\$810\\
Maré	&374\$260\\
Cotegipe	&3\$080\\
Valor médio suburbanos	&300\$170\\
\hline
VALOR MÉDIO GERAL	&1:335\$450\\
\hline
\end{tabular} 
}
{\fonte{Elaboração do autor, com base em \citeonline[pp.~263-264]{bahia_annuario_1926}.}}
\end{table}