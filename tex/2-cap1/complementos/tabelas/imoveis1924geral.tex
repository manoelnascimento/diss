\begin{sidewaystable*}[!htp]
\centering
\IBGEtab{
\caption{Relação dos imóveis arrolados pelo município de Salvador nos distritos urbanos e suburbanos em 1924}\label{tab:imoveis1924geral}}
{
\begin{tiny}
\begin{tabular}{rrrrrrrrrrrrr}
\hline
\multirow{2}{*}{Distritos}&\multirow{2}{*}{Valor locativo}&\multicolumn{11}{c}{Imóveis}\\
\cline{3-13}
	&	&Térreos	&Sobrados	&Abarracados	&Barracão	&Telheiros	&Galpões	&Em ruínas	&Em construção	&Em reconstrução	&Interditados	&TOTAL\\
\hline 
\hline 
\multicolumn{13}{c}{Urbanos}\\
\hline 
Sé	&2855:505\$	&323	&573	&25	&1	&8	&1	&5	&6	&13	&0	&955\\
São Pedro	&3714:596\$	&1113	&660	&24	&1	&2	&4	&3	&4	&5	&0	&1816\\
Passo	&1280:900\$	&373	&277	&12	&0	&0	&0	&2	&0	&2	&0	&666\\
Conceição da Praia	&3675:038\$	&69	&326	&7	&4	&0	&0	&15	&0	&0	&2	&423\\
Pilar	&1874:596\$	&716	&238	&3	&2	&0	&1	&23	&5	&0	&0	&988\\
Mares	&1245:862\$	&1742	&67	&0	&10	&3	&1	&8	&15	&0	&0	&1846\\
Nazaré	&1559:070\$	&981	&197	&3	&1	&3	&1	&10	&11	&0	&0	&1207\\
Vitória	&4676:477\$	&4010	&558	&5	&8	&4	&0	&26	&32	&0	&0	&4643\\
Santana	&1951:010\$	&1571	&169	&1	&2	&1	&0	&1	&0	&0	&0	&1745\\
Penha	&1740:457\$	&2416	&126	&5	&2	&15	&1	&23	&14	&3	&5	&2610\\
Santo Antônio	&2401:906\$	&4773	&210	&24	&2	&3	&3	&27	&5	&0	&0	&5047\\
Brotas	&1346:364\$	&2662	&13	&10	&0	&0	&0	&4	&33	&6	&0	&2728\\
TOTAL 	&28321:781\$	&20749	&3414	&119	&33	&39	&12	&147	&125	&29	&7	&24674\\
\hline
\multicolumn{13}{c}{Suburbanos}\\
\hline
Itapuã	&25:456\$	&210	&0	&0	&0	&0	&0	&0	&3	&0	&0	&213\\
Pirajá	&314:634\$	&1732	&0	&0	&0	&0	&0	&0	&56	&4	&0	&1792\\
Paripe	&50:678\$	&308	&0	&0	&0	&0	&0	&1	&12	&2	&0	&323\\
Passé	&70:920\$	&578	&0	&0	&0	&0	&0	&0	&23	&0	&0	&601\\
Matoim	&11:148\$	&114	&0	&0	&0	&0	&0	&0	&1	&1	&0	&116\\
Maré	&29:192\$	&272	&0	&0	&0	&0	&0	&0	&5	&3	&0	&280\\
Cotegipe	&10:488\$	&75	&0	&0	&0	&0	&0	&0	&3	&0	&0	&78\\
TOTAL 	&512:516\$	&3289	&0	&0	&0	&0	&0	&1	&103	&10	&0	&3403\\
\hline
TOTAL GERAL	&28834:297\$	&27327	&3414	&119	&33	&39	&12	&149	&331	&49	&7	&31480\\
\hline
\end{tabular} 
\end{tiny}
}
{\fonte{\textbf{Annuario estatistico – annos de 1924 e 1925} organizado para o Governo da Bahia por M. Messias de \citeonline[p.~249]{bahia_annuario_1926}.}}
\end{sidewaystable*}