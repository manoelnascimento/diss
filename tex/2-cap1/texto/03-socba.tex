\section{A sociedade baiana e soteropolitana num mundo em convulsão}\label{sec:sobasotconv}

É só depois de empreender este longo percurso que é possível entender a situação das sociedades baiana e soteropolitana no período. Estabelecidas as linhas gerais da economia, da política e da cultura, a narrativa do \textit{desajuste} da economia baiana frente aos núcleos mais dinâmicos do capitalismo, assim como a permanência atávica de certos traços culturais e políticos, pode ser compreendida sem maiores problemas.

Será preciso, para isto, compreender a inserção da Bahia na evolução nacional nos aspectos econômico, demográfico, social e cultural e a posição de Salvador neste processo; as razões da estagnação econômica do Estado e da paralisia demográfica de Salvador nas primeiras décadas do século XX; a formação de uma estratificação e de uma estrutura de classes sociais correspondente a uma economia baseada no trabalho livre, em substituição a uma estratificação e a uma estrutura de classes sociais correspondente a uma economia baseada no trabalho escravo; os lazeres públicos e as festas cívicas, laicas e religiosas, momentos onde expressavam-se de forma bastante ritualizada as distinções comportamentais e estéticas entre classes sociais.

\subsection{Demografia}\label{subsec:demogbasa}

A análise da população soteropolitana será feita com base nos censos de 1872, 1890, 1900, 1920 e 1940. O primeiro e o último serão incluídos, apesar de ultrapassarem o lapso temporal escolhido para esta dissertação, pelo fato de os censos programados para os anos de 1880 e 1930 não terem sido realizados por força de dificuldades políticas conjunturais \cite{oliveirasimoes_censos_2005}; da mesma forma, a precária apresentação dos censos de 1890 e 1900 mediante simples sinopse, sem qualquer tabela de detalhamento, levou a considerá-los nesta pesquisa apenas para simples contagem populacional municipal, sem qualquer outro item que se lhes aproveite \cite{reisetal_areascensos_2011}. 

Viu-se no século XIX a população soteropolitana mais que triplicar entre 1800 (50 mil hab.) e 1890 (174 mil hab.) \cite[p.~70]{sampaio_formas_1999}; é com base no puro impacto da triplicação que se costuma afirmar a existência de um forte crescimento demográfico no período. Ora, sem entrar na discussão das causas das variações populacionais sazonais, já bastante comentadas na historiografia -- cólera (1855, 1866), febre amarela (1849-1854), peste bubônica (1855), secas prolongadas no interior baiano (1857-1861, 1869-1870, 1877-1879, 1888-1890) etc. --, trata-se do acréscimo médio de aproximadamente 1.378 habitantes ao ano, e de uma taxa de crescimento populacional de 1,4\% ao ano (a.a.)\footnote{A taxa de crescimento populacional é calculada usando a metodologia do IBGE: \[ r = \Bigg[ \Bigg( \sqrt[n]{\frac{P_{0}}{P_{t}}} \Bigg) - 1 \Bigg] X 100 \] onde \textit{r} é a taxa de crescimento anual, \textit{$P_{0}$} é a população ao início do perído, \textit{$P_{t}$} é a população ao final do período e \textit{n} é o número de anos do período.}. Uma comparação com o crescimento populacional em outros períodos ajuda a situar a questão (cf. \autoref{tab:1}, p. \pageref{tab:1}).

\begin{table}[!htp]
\centering
\IBGEtab{
\caption{Evolução populacional de Salvador 1872-1940}\label{tab:1}}
{\begin{tabular}{|c|c|c|c|}
\hline 
Ano & População (hab.) & Variação (hab.) & Taxa média de crescimento (em \% a.a.) \\ 
\hline 
1872 & 129.109 & -- & -- \\ 
1890 & 174.412 & 45.303 & 1,68 \\ 
1900 & 205.813 & 58.401 & 1,67 \\ 
1920 & 283.422 & 77.609 & 1,61 \\ 
1940 & 290.443 & 7.021 & 0,12 \\ 
\hline 
\end{tabular} }{
\fonte{Elaboração do autor, com base nos Recenseamentos de 1872, 1890, 1900, 1920 e 1940.}}
\end{table}

Por esta perspectiva, apenas a taxa de crescimento verificada entre 1920 e 1940 pode ser considerada como ``ponto fora da curva'' por força das epidemias de gripe espanhola (1918), varíola (1919) e febre amarela (1919), causadoras de óbitos e de emigração; exceto por este período, a velocidade do crescimento demográfico soteropolitano manteve-se durante toda a Primeira República consistente com o verificado no século anterior, sendo mesmo ligeiramente maior.

Vista a questão em termos comparativos com outras cidades brasileiras, entretanto, aparece uma enorme defasagem entre este incremento demográfico e os do Rio de Janeiro e São Paulo, únicas cidades brasileiras demograficamente comparáveis a Salvador no período.

\begin{table}[!htp]
\centering
\IBGEtab{
\caption{Incremento demográfico segundo recenseamentos oficiais}\label{tab:2}}
{\begin{tabular}{|c|c|c|c|c|c|c|c|c|}
\cline{1-9}
\multirow{3}{*}{Capitais} &\multicolumn{8}{c|}{Aumentos populacionais}\\
\cline{2-9} & \multicolumn{2}{|c|}{1872-1890} & \multicolumn{2}{|c|}{1890-1900} & \multicolumn{2}{|c|}{1900-1920} & \multicolumn{2}{|c|}{1920-1940}	\\
\cline{2-9} &nº &\% &nº &\% &nº &\% &nº &\%\\
\cline{1-9} São Paulo &33.549 &106,89 &147.886 &269,32 &339.183 &141,43 &747.258 &129,05\\
\cline{1-9} Rio &247.679 &90,07 &288.792 &55,25 &346.430 &42,69 &606.268 &52,36\\
\cline{1-9} Salvador &45.303 &35,08 &31.401 &18 &77.609 &37,7 &7.021 &2,47\\
\hline
\end{tabular} }
{ \fonte{\cite[p.~14]{santos_repovo_2001}} }
\end{table}

O incremento populacional no Rio de Janeiro pode-se facilmente explicar pelo fato de ser a capital federal e principal porto do país. O de São Paulo pode-se explicar pelo crescimento industrial da cidade e pela preponderância da economia cafeeira sobre a economia brasileira no período, que atraiu massas de migrantes em fuga da crise e da fome na Europa, no Oriente Próximo e no Japão. A discrepância do incremento populacional soteropolitano frente ao carioca e ao paulistano explica-se principalmente pelo baixo dinamismo da economia baiana, pouco capaz de criar novos postos de trabalho e atrair população migrante. Descontado o fluxo migratório rumo a Salvador imediatamente após a abolição da escravatura, constantemente referenciado pela literatura a respeito do tema \cite{fraga_encruzilhadas_2014,souza_trabalholivre_2011}, não foram registrados nos censos quaisquer outras causas para movimentos migratórios na direção de Salvador no período, mesmo em épocas de seca (1898-1900); pelo contrário, a tendência global era \textit{emigratória}. Fugia-se das epidemias, das elevadas taxas de mortalidade, da insegurança alimentar endêmica, mas, fundamentalmente, da estagnação econômica \cite{santos_repovo_2001}.

\subsection{Economia e classes sociais}\label{subsubsec:ecobasa}

Passando da demografia à economia, na Primeira República a economia baiana participou da economia global fundamentalmente por meio de importações e exportações de produtos agrícolas. Entretanto, desde o Império verificou-se tendência ao declínio da participação baiana na pauta de exportações brasileira, como decorrência da crise na lavoura açucareira que era o esteio da economia baiana desde os tempos da colônia.

A economia baiana acompanhou os ciclos da economia global vivendo seus mesmos altos e baixos; se em termos de ciclo econômico os anos 1874-1895 são de \textit{recessão}, os anos 1895-1928 são de relativa \textit{prosperidade}, alavancada pela recuperação das importações, pela revalorização do mil-réis, pela aceleração no ritmo das exportações (especiamente do cacau) e pelo aumento da receita e das despesas públicas do Estado \cite[p.~28-29]{CPE1980}. Em tal conjuntura, entretanto, não foi significativamente alterada a inserção da Bahia nas economias brasileira e global, pouco se alterou em sua estrutura produtiva, e menos ainda nas condições de vida da população baiana. Durante toda a República Velha a Bahia permaneceu aferrada à \textit{agricultura para exportação}, especialmente via (por ordem de importância) cacau, fumo, açúcar, café, coco e coquilhos, piaçava e outros \cite[p.~77;110]{CPE1980}. Tal vinculação ao mercado exterior, ao mesmo tempo em que fez das regiões produtoras dos principais produtos da pauta (Recôncavo e Sul) as regiões mais economicamente dinâmicas do Estado, atrelou a Bahia às flutuações e mesmo às menores crises cíclicas dos mercados destinatários de seus produtos. As demais regiões baianas (Nordeste, São Francisco, Chapada Diamantina e Sertão), conquanto vivessem um ou outro momento de prosperidade graças a produtos sazonais, tinham papel complementar ou mesmo marginal diante das duas regiões principais \cite[p.~77]{CPE1980}. 

As exportações declinavam em consequência da decadência canavieira. Se a participação brasileira no mercado internacional do açúcar diminuiu desde os últimos anos do Império, a participação baiana declinou ainda mais velozmente. A transição tecnológica do \textit{engenho} para a \textit{usina} incrementaria a produtividade no setor, como se deu em Pernambuco (ainda que por meio de empréstimos tomados ao governo estadual, feitos em literais doações ao longo do tempo) \cite[p.~31]{gorender_burguesia_1990}; observa-se, todavia, que praticamente todos os engenhos baianos durante a Primeira República encontravam-se aparelhados com máquinas deficientes e obsoletas, grande parte proveniente de engenhos que haviam sido desmontados no Egito e instalados na Bahia por firmas inglesas \cite[p.~74]{sampaio_legislativo_1985}, e a demora dos sucrocultores baianos em fazer acontecer em tempo hábil esta transição, por quaisquer meios que fossem\footnote{O último esforço para soerguer a produção açucareira manifestou-se no projeto de lei 2 do Senado da Bahia, de 1898, ``autorizando o governo a contratar como pessoas idôneas, proprietários e lavradores, a construção de seis usinas aperfeiçoadas para a fabricação do açúcar''; tal projeto estabelecia que ``cada usina deveria ter uma capacidade para moer 200 toneladas de cana em 24 horas'', e dispor de ``instrumentos agrários aperfeiçoados e adaptados ao cultivo da cana, destinados a funcionar nas propriedades agrícolas que fornecerem matéria-prima''. Submetido a emendas e levado á discussão, foi considerado ``inexequível'', mas mesmo assim aprovado e promulgado como a Lei Estadual 255; foi todavia ineficaz, e a produção açucareira baiana permaneceu na estagnação tecnológica em que se encontrava \cite[pp.~74-75]{sampaio_legislativo_1985}.}, afastou-os das novas centralidades tecnológicas das condições gerais de produção, perdendo posição para os sucrocultores pernambucanos, agora capazes de produzir açúcar mais barato e portanto de quebrá-los por meio da concorrência no mercado, e no mercado internacional tornando seus preços incapazes de competir com o açúcar cubano. O Recôncavo, antes principal região agrícola do Estado, passou a sobreviver economicamente da policultura de exportação (fumo, algodão etc.) que, durante a hegemonia açucareira, era secundária, de menor porte, subsidiária da economia dos engenhos. As demais regiões baianas (Nordeste, São Francisco, Chapada Diamantina e Sertão), onde se praticava há séculos a pecuária extensiva, a produção extensiva de gêneros alimentícios e a agricultura de subsistência, mantiveram seu caráter \textit{complementar} ao restante da produção agrícola baiana, não se operando nelas quaisquer alterações seja em suas técnicas produtivas, seja em sua escolha de culturas agrícolas em função do mercado, seja em sua estrutura fundiária -- em suma, permaneceram durante toda a Primeira República tão marginais quanto o foram na Colônia e no Império. 

Abriu-se uma possibilidade de reversão do processo involutivo mediante a boa fase da \textit{lavoura cacaueira} durante a Primeira República. O aumento da procura internacional pelo produto fez do Sul da Bahia, entre 1907 e 1925, nova fronteira agrícola, onde a disputa pela terra materializou-se em incêndios de cartórios, grilagem, corrupção de autoridades e toda sorte de fraudes e violências para legitimar os \textit{caxixes} \cite[pp.~78-79]{CPE1980}; de 1923 em diante, razoavelmente consolidadas as forças políticas e econômicas locais, passaram os latifundiários cacaueiros a pautar o governo a apoiá-los por meio de concessão de isenções tributárias, de recursos para a construção de estradas de rodagem e prédios escolares na região Sul baiana, e mesmo por meio de um convênio estadual em defesa do cacau \cite[p.~77]{sampaio_legislativo_1985}. É importante, adicionalmente, ressaltar que a economia cacaueira, por si só, não teria sido capaz de brecar a involução econômica baiana, pois a massa de excedente econômico criada pelo cacau na Bahia nunca alcançou o tamanho da produzida pelo café em São Paulo, ou pelo algodão e açúcar no Nordeste. Em 1929, no final do auge das exportações de cacau, as vendas desse produto no exterior representavam apenas 6\% das exportações totais do país \cite[p.~20]{CPE1980}. Soma-se a isto o fato de que, não obstante a boa fase cacaueira, repetiram-se nesta cadeia produtiva os mecanismos da \textit{subordinação dos fazendeiros aos comerciantes} vistos em outras lavouras intensivas voltadas à exportação: a baixa oferta de crédito, comum também na Bahia, fez com que os produtores de cacau vendessem sua produção aos intermediários e exportadores quando o cacau estava ainda em floração; eram as casas exportadoras, majoritariamente estrangeiras, quem estava em posição adequada para retirar as melhores vantagens da situação ao financiar os fazendeiros, funcionando quase como bancos de investimento ao emprestar-lhes dinheiro contra as safras ainda pendentes nos cacaueiros, safras estas não raro insuficientes para saldar as dívidas -- fazendo dos cacauicultores dependentes de novos créditos, ou mesmo reiterados suplicantes em torno de novos prazos ou renegociações de valores \cite[p.~229]{perissinotto_cladom_1994}. 

O \textit{setor industrial} baiano também viveu crise e descenso durante a Primeira República. Primeiro polo têxtil da indústria brasileira \cite{stein_textil_1979}, de terceira força industrial do país em 1892, a Bahia caiu em vinte anos para o 12º posto, empregando, em 1912, dez mil operários, envolvendo um capital de 28:000\$000 e produzindo bens no valor de 25:000\$000 \cite[p.~29-30]{CPE1980}. Este parque industrial, entretanto, era fundamentalmente \textit{acessório da produção agrícola} -- engenhos e usinas de açúcar, fábricas e manufaturas de fumo etc. --, exceto em cidades como Salvador, Valença e Santo Amaro; nelas, pequenas fábricas de sabão, papel, pólvora e rapé concorriam com fundições de ferro e cobre cuja produção centrava-se em ferramentas para a lavoura e maquinismos para os engenhos e embarcações a vapor \cite[p.~30]{CPE1980}. Observa-se, adicionalmente, a \textit{pequenez} das indústrias baianas na Primeira República, expressa em seu baixo capital de giro (vasta maioria entre 1\$000 a 500\$000 de 1898 a 1914), sua baixa densidade tecnológica, seu baixo número de empregados, a natureza arcaica de sua força motriz e a baixa complexidade das instalações \cite[p.~55]{CPE1980}. Em Salvador, mais especificamente, proliferavam-se os chamados \textit{artífices}, ou seja, profissionais envolvidos em processos artesanais de trabalho cujas tradições remontavam pelo menos ao século XVIII: alfaiates, costureiras, chapeleiras, cabeleireiros, bombeiros hidráulicos, ferreiros, funileiros, encanadores, latoeiros, sapateiros e tipógrafos etc. \cite{CPE1980,REIS2012}.

A \textit{infraestrutura viária} e os \textit{transportes}, duas condições gerais de produção fundamentais para alavancar o dinamismo econômico, eram precaríssimas na Bahia. Ainda que a \textit{rede ferroviária} concluída em 1896 permitisse ligar Salvador a Juazeiro e, portanto, ao rio São Francisco; ainda que tal ligação interligasse a praça comercial soteropolitana aos 1.700km navegáveis do mais importante rio do Centro-Norte brasileiro; ainda assim, havia dois problemas. Em primeiro lugar, o crescimento da rede ferroviária para além da linha Salvador-Juazeiro mostrou-se insuficiente para atender a todas as regiões do Estado \cite[p.~31]{CPE1980}. Já a \textit{navegação no Vale do São Francisco}, infraestrutura decisiva para a praça comercial de Salvador por conectá-la aos mercados ribeirinhos e também de Goiás, Piauí, Pernambuco, Minas Gerais e Maranhão por meio de ramais navegáveis partindo do Rio Preto e chegando ao rio Sapão \cite[p.~220]{CUNHA2011}, começou bem o século XX, mas já durante a Primeira Guerra Mundial mostrou-se defasada, mal-administrada, incapaz de atender às demandas da praça soteropolitana, que reclamou inclementemente ao Governo da Bahia por mudanças até que, por fim, em 1921, foi completamente arrendada à iniciativa privada e relegada a segundo plano frente à navegação marítima, esta última finalmente posta, depois de muita contenda epistolar, parlamentar, jornalística e publicitária, sob o controle direto de elementos hegemônicos dentro da Associação Comercial da Bahia \cite[p.~221-223]{CUNHA2011}. No plano \textit{rodoviário}, desimportante no período que vai até os anos 1950, a Bahia seguiu a tendência brasileira de pouca valorização deste modal de transporte no período: dos 11.517km de estradas no Estado existentes em 1936 (ano da primeira estatística do setor), 10 mil deles ``não passavam de caminhos para as tropas de burros'' \cite[p.~31]{CPE1980}.

A predominância da agricultura na estrutura produtiva baiana resultou, necessariamente, em alguma \textit{atividade comercial} para escoar a produção, seguindo os mesmos mecanismos oligopsonistas já vistos e repisados no tocante a outras lavouras; tal atividade era hegemônica nas praças de Salvador, Ilhéus e Cachoeira, cidades portuárias por excelência. No caso soteropolitano, o comércio era o setor econômico mais numericamente extenso e também o que mais contribuía com a renda pública por meio de tributos \cite[p.~55]{CPE1980}. Ressalta-se a \textit{concentração de capitais}, pois os \textit{pequenos comerciantes}, apesar de mais numerosos -- eram mais da metade dos contribuintes do Imposto de Indústrias e Profissões arrolados pela Fazenda estadual -- pagavam muito menos ao Estado em rendas tributárias que os \textit{escritórios comerciais}, onde se realizavam os negócios mais vultosos \cite[p.~56]{CPE1980}. Os mais numerosos entre os pequenos comerciantes soteropolitanos eram os que exploravam bares, tavernas, cafés ou restaurantes; os que vendiam alimentos e bebidas; os armazéns de gêneros não-alimentícios; e as lojas de tecidos e roupas. 

No que diz respeito à \textit{presença do capital estrangeiro} na Bahia, com a ressalva de que muitas firmas comerciais estrangeiras não se registravam na Junta Comercial, é curioso observar a enorme presença do capital \textit{português}, alhures insignificante e aqui persistente desde os tempos do Império; e do capital \textit{alemão}, algo esperado numa economia agroexportadora \cite[p.~69-70]{CPE1980}. O capital francês, em especial, dominou o comércio exportador de cacau por meio da firma suíço-brasileira \textit{Wildberger \& Cia.} -- e as articulações desta empresa com outros setores da economia serão vistas em momento oportuno.

No setor \textit{bancário}, até 1910 apenas o \textit{Banque l'Union Parisienne} despontava numa praça hegemonizada pelos ingleses; os mais estáveis entre estes últimos foram o \textit{The British Bank of South America}, o \textit{London and Brazilian Bank}, \textit{The London and River Plate Bank} e o \textit{Bank of London and South America}. Em 1910 chegou à praça soteropolitana o \textit{Brazilianische Bank für Deutschland}, inaugurando a presença alemã na praça bancária soteropolitana, reforçada em 1930 com a chegada do \textit{Banco Alemão Transatlântico}. Em 1918 abriu as portas uma filial do estadunidense \textit{The National City Bank of New York}. Todos, independentemente da nacionalidade, operavam com alta margem de lucro, mas o conjunto de suas operações não superava o das casas de comércio, que desde o Império faziam as vezes de instituições financeiras \cite[p.~55]{CPE1980}.

Como resultado de tantos e tamanhos limites e constrangimentos, mesmo em conjuntura economicamente favorável a população da Bahia inteira, afora os grandes latifundiários exportadores, tinha poder aquisitivo ``limitadíssimo'' \cite[p.~189]{azevedolins_bancoba_1969}. Num cenário de tamanha estagnação econômica, sem outras regiões que não o Sul e o Recôncavo a dinamizar a economia baiana, Salvador manteve-se não apenas como capital política do Estado, mas igualmente como capital econômica, cabeça de região, porta de entrada e saída do Estado -- em suma, aparecia como centro hiperdimensionado relativamente aos demais municípios \cite[p.~63]{CPE1980}. 

A estrutura das classes sociais na Bahia durante a Primeira República segue as linhas gerais das classes sociais brasileiras, já desenhadas na \autoref{subsec:clapolprire}. Da estrutura econômica baiana seria razoavelmente fácil fazer derivar uma estrutura de classes regional e local que opusesse, no setor agrícola, os grandes latifundiários e a miríade de pequenos agricultores e assalariados agrícolas deles dependentes sob variadas formas; no setor industrial, os artesãos, os grandes industriais e os proletários; no setor comercial e bancário, os donos das grandes casas comerciais e bancos (nacionais ou estrangeiros), seus gerentes e o restante pessoal dos escalões inferiores; e assim por diante, a cada setor correspondendo grupos e categorias profissionais a somar, de acordo com sua posição no processo de trabalho, a um ou outro lado da exploração econômica e da opressão política. Vários autores já o fizeram antes \cite{castellucci_salvador_2001,CPE1980,santos_repovo_2001}, e seria de pouco interesse a esta pesquisa repetir o que alhures já se disse. Já se viu na \autoref{subsec:clapolprire}, entretanto, que há certas precauções a tomar, e serão repetidas aqui no contexto baiano e soteropolitano.

Se formalmente a escravidão fora abolida em 1888, não se pode dizer o mesmo do \textit{comportamento escravocrata} e da \textit{persistência de fomas de trabalho oriundas do tempo do cativeiro}. Desde antes da abolição formas de resistência escrava eram comuníssimas, e o caso baiano não foge à regra: fugas (especialmente para as cidades, onde poderiam lançar-se ao trabalho em obras públicas assumindo a identidade de trabalhadores livres), ações de liberdade, rebeliões, negaças, tudo foi tentado pelos escravizados tanto para obter sua liberdade, quanto para obter melhorias provisórias em sua situação \cite[p.~45-52]{fraga_encruzilhadas_2014}. Nos campos, conquanto a pequena roça destas pessoas escravizadas fosse absolutamente subsidiária à plantagem \cite{gorender_escracolo_2010}, surgia entre os cativos a partir dela um senso de direitos mínimos, de direito à terra, materializado imediatamente após a escravidão na expectativa, infelizmente vã, de que ao fim do cativeiro corresponderia o acesso à terra \cite{fraga_encruzilhadas_2014}. Da mesma forma, houve na Bahia movimento abolicionista que, se não alcançou os mesmos resultados conseguidos no Ceará ou o mesmo engajamento intenso vivido no Rio de Janeiro e em São Paulo, conseguiu por outro lado impulsionar alforrias gratuitas e promover eventos arrecadatórios em prol de alforrias; envolveu sociedades recreativas como a Filarmônica Euterpe, o Clube Cruz Vermelha ou o Clube Fantoches; fundou a Sociedade Abolicionista 25 de Junho (Cachoeira, 1870), a Sociedade Libertadora Baiana (Salvador, 1883) e jornais abolicionistas como \textit{O Asteroide} (1887-1889); o movimento abolicionista baiano, enfim, representou bem a corrente ideológica e política em voga no final do Império na província com maior número de escravos no Nordeste brasileiro \cite{brito2003abolicao}.

O fim da escravidão no Recôncavo baiano\footnote{Foge ao escopo desta pesquisa analisar o processo abolicionista na Bahia inteira, e há mesmo dificuldade de acesso a estudos sobre o assunto, dada a concentração do olhar dos historiadores sobre o perímetro que vai de Salvador e seu Recôncavo até o Sul baiano -- significando, portanto, poucos estudos sobre o processo abolicionista no Além São Francisco, no sertão, nas Lavras Diamantinas e noutras regiões baianas.} teve uma cadeia de efeitos de curto prazo e outra de médio prazo. No curto prazo, os recém-libertos, como visto, esperavam que o fim do cativeiro resultasse-lhes também no seu acesso à terra, o que não aconteceu; guardando ou não esta expectativa, por meses a fio após o 13 de maio de 1888 negaram-se peremptoriamente a realizar qualquer trabalho que vagamente lembrasse a condição escrava. Recusaram-se a trabalhar nos canaviais, enjeitaram os serviços domésticos, desfilaram (sozinhos ou em grupos) pelas ruas dando vivas à liberdade e -- muito provocativamente -- à igualdade, quebraram a dominação senhorial (mesmo o mais dengoso paternalismo era tratado com desdém) e reivindicavam terras. Não por acaso, policiais, autoridades políticas e senhores de engenho chamavam-lhes de ``insubordinados'', ``rebeldes'' e -- sem anacronismo algum -- ``comunistas'' \cite[p.~119-160]{fraga_encruzilhadas_2014}. No médio prazo, a liberdade implicou numa onda de saques a engenhos, incêndios e conflitos entre recém-libertos sitiantes e senhores de engenho, amoldadores da estrutura agrária da região e dos sistemas de assalariamento e arrendamento rurais; e também na migração em massa de ex-cativos para as cidades em busca de trabalho, onde, descapitalizados, ingressaram com as qualificações de que dispunham nas pequenas indústrias e oficinas, nos serviços e obras públicas, nos poucos grandes empreendimentos fabris (\textit{Dannemann}, \textit{União Fabril}, \textit{Empório do Norte} etc.) \cite[p.~161-241]{fraga_encruzilhadas_2014}.

Não é de surpreender, deste modo, que a classe trabalhadora soteropolitana -- ousaria dizer a classe trabalhadora brasileira como um todo, mas faltam-me os dados empíricos comprobatórios da hipótese -- tenha sua formação fortemente condicionada pelo processo de transição do trabalho escravo para o trabalho livre. As formas e processos de trabalho, as profissões exercidas, o controle sobre os trabalhadores (e sua intensidade), seu exercício profissional no espaço público e suas manifestações culturais coletivas, tudo isto foi, pelo menos durante as duas primeiras décadas República (1889-1910), fortemente condicionado pela tentativa de disciplinamento dos recém-libertos para limitar sua recém-conquistada liberdade aos termos de uma sociedade pautada pelas relações assalariadas de exploração do trabalho. Havia inclusive soluções ditadas pelo mais puro e simples desespero, que somado ao autoritarismo e racismo velhos de séculos resultava em verdadeiros absurdos. Veja-se, por exemplo, que para resolver a escassez de mão-de-obra nos anos imediatamente posteriores à abolição os latifundiários baianos, em especial os sucrocultores por meio de seus representantes no Legislativo estadual, conceberam soluções drásticas como o projeto de lei 173, que autorizava o governo a ``fazer regulamento para a colocação de desocupados, homens e mulheres que não tivessem ocupação conhecida'', e a criar colônias correcionais para aqueles ``delinquentes'' que infringissem os novos contratos de trabalho estabelecidos em sequência ao retorno de recém-libertos ao trabalho nos engenhos de açúcar. Este projeto do barão de Lacerda Paim, verdadeira legislação punitiva voltada a forçar ao trabalho os ex-escravos que migravam em massa para as cidades baianas e ao lá chegar viam-se subempregados ou desempregados, foi aprovado em plenário na primeira discussão, ainda que fortemente contestado em segunda discussão pelo deputado oposicionista Lacerda Medrado, e enfim engavetado, pondo cobro a esta tentativa desesperada dos latifundiários açucareiros de arrebanhar à força trabalhadores para seus engenhos decadentes \cite[pp.~73-74]{sampaio_legislativo_1985}.

Por outro lado, o isolamento comercial das regiões baianas, as barreiras mercantis, tudo isto fazia com que a economia baiana mostrasse-se cenário fértil para fatores produtivos locais e gerasse, no âmbito urbano, um grande número de ocupações capazes de inforporar a mão de obra e mesmo pequenos capitais oriundos de negros e mestiços. A própria configuração urbana de Salvador, com seus hábitos, serviços e mestrias construídos ao longo dos séculos, facilitou a criação de empregos, e nesta perspectiva a decadência da economia agroexportadora -- com seus engenhos endividados, safras empenhadas etc. -- não prejudicou, mas beneficiou a expansão da estrutura de serviços urbanos e pequenas manufaturas, estabelecendo-se portanto uma diferenciação socioeconômica interna à comunidade negra até que, passado o primeiro terço do século XX, a concorrência com bens e serviços oriundos do Centro-Sul, por força dos ganhos de escala de tecnologias mais produtivas e das vantagens de seu confronto no mercado com produtos e serviços decorrentes das tecnologias arcaicas empregues na Bahia, empurrou definitivamente tais setores para a proletarização \cite[pp.~71-72]{sodre_terreiro_1988}.

\subsection{Cultura e espaço público}\label{subsec:cultespubsaba}

Salvador, como qualquer outro assentamento humano em todos os tempos e lugares, não viveu somente de sua economia; em torno dela foram produzidas formas de cultura que, conquanto comunguem de elementos centrais à cultura brasileira, guardam especificidades capazes de permitir a forja e portanto também a invenção de tradições -- e, como em qualquer invenção de tradições, oculta-se a dimensão conflituosa da sociedade \cite{mariano_baianidade_2009,pinho_baianidade_1998}. Pode-se dizer que havia na Salvador da Primeira República uma distinção, um combate mal-disfarçado mesmo, entre a cultura \textit{dos salões} e a cultura \textit{da rua}. À espurcícia das vias, ainda estruturadas \textit{grosso modo} conforme o legado colonial, descalçadas e côncavas, onde uma vala central captava indistintamente todas as águas pluviais e servidas (apesar da proibição municipal ao seu lançamento em via pública, era esta a rotina), formando córregos pútridos de lama e excrementos; à azáfama dos \textit{moleques compradores de tempero} e ao vaivém das \textit{caixinheiras} e \textit{lavadeiras}; às cantilenas altissonantes dos \textit{mascates} ``árabes'' ou ``russos'', dos \textit{homens das folhas} e das \textit{mulheres de saia} a vender abará, aberem, acaçá, acarajé, amendoim, amoda, bolo, roletes de cana, cocada, cuscuz, mingau e outros acepipes trazidos nos baús e gamelas que lhes vacilavam por sobre as cabeças ou nos tabuleiros que lhes recurvavam a figura; aos cheiros, à profusão de cheiros, à indistinção acrimoniosa dos fedores emanados das valas pseudossanitárias em meio ao odor acre de peixe e de vísceras bovinas trazidas daqui para ali pelas \textit{peixeiras} e \textit{fateiras} a mercadejar, tudo imiscuindo-se entre as essências  olorosas das bandejas dos \textit{pulgas-prenhas}; a tudo isto, a todo este sensualismo das ruas, tão perigoso em tempos de miasmas e maus ares, opunham-se os lazeres como que ``assépticos'' das \textit{salas de estar}, dos \textit{clubes}, dos \textit{salões}, admirados sôfrega e desejosamente por quem assistia às \textit{funções} de suas cadeiras cativas no \textit{sereno} \cite{vianna_bahia_1973}.

É certo que os aspectos \textit{privados} da cultura são marcantes, mas para esta pesquisa importam mais os aspectos necessariamente \textit{públicos} da cultura, ou seja, aqueles que para sua manifestação ou existência necessitam dos espaços públicos, ou abertos ao público. Destes, foram destacados a \textit{imprensa} e a formação da camada de \textit{intelectuais} que através dela se manifestava; espaços votados à diversão pública, como os \textit{teatros} e \textit{cinemas}, então novidade; as \textit{procissões} e as \textit{festas} públicas, fossem profanas ou religiosas, populares ou oficiais; por fim, a mais duradoura tradição festiva soteropolitana, o \textit{carnaval}.

Radicava-se na Bahia uma seção da chamada ``República das Letras'', ciosa de ser ``o segundo centro cultural do país'' \cite[p.~263]{machadoneto_bahiaint_1972}. Estes intelectuais eram em geral \textit{polígrafos}, ou seja, ``franco-atiradores'' generalistas, com especialização de sua produção cultural correspondente ao fraco grau de especialização profissional e à pequena complexificação da divisão social do trabalho; altamente influenciados pela cultura europeia, particularmente a francesa; gravitando em torno do Rio de Janeiro como centro intelectual, seja para elogiá-lo, seja para desdenhá-lo; ideologicamente formados pelo positivismo então reinante, nas vertentes de Haeckel, Comte e Buchner;  \cite{MachadoNeto1966,machadoneto_bahiaint_1972}.  Nestas e noutras características, não destoavam da restante intelectualidade brasileira da época \cite{martins_intelv5_1977,martins_intelv6_1978}. No pensamento e nas letras, esta fração da burguesia era agitada pela juventude frequentadora da Faculdade de Medicina do Terreiro de Jesus, da Faculdade Livre de Direito da Bahia, da Escola Normal, da Academia de Letras, da Escola de Belas Artes, do Instituto de Música \cite[p.~272]{machadoneto_bahiaint_1972}; na prática, esta agitação se dava através da imprensa, cada jornal assumindo a defesa de tal ou qual bloco político ao sabor das alianças de momento (p. ex., o \textbf{Diário da Bahia} defendia posições dos partidários de Severino Vieira, \textbf{A Tarde} teve início como órgão associado ao ex-governador Luiz Vianna, \textbf{O Imparcial} tomava o lado da Associação Comercial da Bahia que financiava sua publicação etc.) \cite{souza_imprensa_1972,machadoneto_bahiaint_1972}.

No campo das \textit{diversões públicas}, uma de suas formas, democrática de certa maneira, foram os \textit{teatros}. \citeauthoronline{ruy_teatro_1959} registrou uma crise no teatro baiano na passagem do Império para a República, derivada segundo ele da ``depressão econômica advinda da abolição em 1888'' e das ``reformas radicais impostas pela nova forma de governo em 1889''; neste momento, o público baiano, ``preocupado com o jogo desenfreado da bolsa'', perdeu o interesse pela comédia de costumes, pela burleta (ou comédia musicada), ``repudiava as peças de tese que obrigavam a pensar'', ``voltava as costas ao teatro romântico, já em desuso'', e também ``fugia do gênero lírico''. Como saída, os empresários do ramo lançaram a ``revista'', forma teatral bastante licenciosa para os costumes da época, aplaudida todavia pela crítica e aclamada pelo público, entusiasmado pelas ``coplas licenciosas'' \cite[p.~48-49]{ruy_teatro_1959}. No período estudado abrem-se os primeiros \textit{cinemas} da capital, que \citeonline[p.~89]{boccanera_teatro_2008} considerava ``o maior inimigo do teatro''. Viu-se no período uma profusão de salas, especialmente entre 1910 e 1914 (cf. \autoref{tab:cinemas}).

\begin{table}[!htp]
\IBGEtab{\caption{Nome, endereço, data de abertura e de fechamento de salas de cinema em Salvador (1897-1930)}\label{tab:cinemas}}{
\begin{minipage}{0.9\textwidth}
\begin{tiny}
\begin{longtabu} to \textheight {m{3cm} m{9cm} m{1cm} m{1cm}}
\hline Nome & Endereço & Abriu & Fechou \\ \hline \endhead
\hline \multicolumn{4}{c}{Continua na próxima página...} \\ \endfoot
\hline \endlastfoot
Edison & Praça Castro Alves, por cima da Confeitaria Luso-Brasileira & 1898 & 1906 \\
Cassino Castro Alves & Praça Castro Alves, onde depois foi instalado o Teatro Guarani & 1903 & 1906 \\
Santo Antonio & Praça Barão do Triunfo (antigo Largo do Santo Antônio) & 1907 & 1907 \\
Salesianos & Rua Conselheiro Almeida Couto, 19 & 1907 & -- \\
Bahia & Rua Chile, nº 1 & 1909 & 1911 \\
Jandaia & Rua Dr. Seabra & 1910 & -- \\
Bijou Teatro-Cinema & Calçada do Bonfim & 1910 & 1911 \\
Popular & Rua da Madragoa, nº 5, no arrabalde de Itapagipe & 1910 & 1919 \\
Cinema Odeon & Calçada do Bonfim, antigo prédio Mira-Mar, próximo à estação da Estrada de Ferro & 1919 & 1920 \\
Avenida & Travessa de Sant'Anna (Rio Vermelho) & 1910 & -- \\
Castro Alves & Largo do Carmo & 1910 & 1911 \\
Central & Praça Castro Alves, na parte térrea do antigo Hotel Paris & 1910 & 1912 \\
Recreio Fratelli Vita & Calçada do Bonfim, nº 20 & 1911 & 1919 \\
Bahia & Largo do Papagaio, nº 38 (Itapagipe) & 1911 & 1915 \\
Rio Branco & Rua do Saldanha, nº 2 & 1911 & 1912 \\
Iris-Teatro & Rua Dr. Seabra & 1912 & 1913 \\
Soledade & Ladeira da Soledade, nº 112 & 1912 & 1913 \\
Ideal & Ladeira de S. Bento, nº 3 & 1913 & 1921 \\
Petit-Cinema & Rua Dr. Agripino Dória (Brotas) & 1913 & 1914 \\
Recreativo & Largo de Sant'Anna (Rio Vermelho) & 1913 & 1914 \\
Centro Católico & Largo de S. Antônio da Mouraria. & 1913 & \\
Parisiense & Praça Dois de Julho (antigo Campo Grande) & 1914 & 1914 \\
Forte de São Pedro & Praça da Aclamação & 1914 &  \\
Cinema da Barra & Rua Barão de Sergy, nº 22 & 1914 & 1918 \\
Olímpia & Rua Dr. Seabra & 1915 & \\
Cine Venus & Rua Carlos Gomes, 25 & 1916 & 1916 \\
Recreio S. Jerônimo & Praça 15 de Novembro (antigo Terreiro de Jesus) & 1917 & \\
Kursaal Baiano & Praça Castro Alves & 1919 &  \\
Cinema Itapagipe & Rua do Poço, nº 155 & 1920 & \\
Cinema Liceu & Rua do Liceu & 1921 & \\
Politeama Baiano & Politeama & 1897 & \\
Teatro São João & Praça Castro Alves & 1899 & 1911 \\
\hline
\end{longtabu}
\end{tiny}
\end{minipage}
}
{\fonte{\cite{boccanera_teatro_2008}}}
\end{table}

Teatro e cinema, entretanto, são diversões, digamos, \textit{semipúblicas}, onde a sociabilidade festiva se dava em espaços fechados, ainda que abertos ao público. Poderiam, certamente, afetar seu entorno, pois seu público cativo, ontem como hoje, deixava impressões no espaço circunvizinho; estavam longe, entretanto, de causar o mesmo impacto das \textit{diversões públicas}, das que tomavam as ruas e impunham seu \textit{joie de vivre} dionisíaco às rotinas pretensamente ``civilizadas'' que políticos e intelectuais pretendiam impor à vida urbana de Salvador -- e por isto mesmo tornadas inimigas públicas numa luta sem quartel.

As mais importantes entre tais diversões públicas são, inequivocamente, as \textit{festas de largo}, manifestações multitudinárias antiquíssimas da religiosidade popular baiana, algumas datando de séculos. Nelas é possível perceber as inúmeras tentativas de controle e disciplinamento do desregramento festivo, tudo em prol das aparências de ``civilização''.

Todas as datas festivas soteropolitana costumavam ser precedidas em alguns dias por \textit{bandos anunciadores}, grupos de foliões mascarados e fantasiados a prenunciar os festejos enquanto gozavam do momento ao som de quadrinhas e serenatas; numa tentativa de contê-los, a postura municipal 146, de 1920, estabeleceu multa de 30\$000 a ser cobrada de seus cabecilhas, ressalvados os bandos de São Pedro, São Gonçalo do Bonfim, Sant’Anna do Rio Vermelho e Santo Antônio da Barra \cite[pp.~42-43]{albuquerque_doisdejulho_1997}. 

Outra prática festiva soteropolitana da época eram as \textit{máscaras} e os \textit{mascarados}, integrantes inescapáveis dos \textit{bandos anunciadores}. Foi tudo isto proibido em 1901, seguido, em 1920, por regulamentação severa, necessitando os mascarados, para escapar às punições caso pegos disfarçados passadas as 18h, de estarem confinados aos bailes carnavalescos dos clubes ou de obterem licença do intendente (prefeito) \cite[p.~44]{albuquerque_doisdejulho_1997}. 

Outra festa pública a destacar é o \textit{Dois de Julho}. Teve ele próprio seu bando anunciador em tempos pretéritos, igualmente proibido com a república, ainda que se tentasse disfarçá-lo como bando de São Pedro \cite[p.~43]{albuquerque_doisdejulho_1997}. Se durante o Império a populaça dominava a festa, com a república os festejos bi-julinos foram tomados como oportunidade educativa pelo \textit{Instituto Geográfico e Histórico da Bahia} (IGHB), pela \textit{Liga de Educação Cívica}, por acadêmicos de medicina e direito e \textit{tutti quanti} perfilados em cortejo ordeiro e garboso, com a ``crioulada'', a ``mulataria'', os ``africanos, maltrapilhos e selvagens'' seguindo ao lado dos carros dos caboclos bem depois das alas organizadas, inclusive das bandas de música, como que a representar a hierarquia social do período \cite[p.~49]{albuquerque_doisdejulho_1997} \dots

Quem conheça o ciclo soteropolitano de festas populares certamente terá notado a ausência do \textit{carnaval} neste relato. Simples: o carnaval é uma \textit{tradição inventada}, não uma festa legitimamente popular. Foi em fevereiro de 1884 que o governo provincial baiano tomou para si a prerrogativa de criar um desfile festivo em substituição ao então popular \textit{entrudo}, considerado bárbaro, incivilizado e assustador pelas ``boas famílias''. A ideia era extinguir a festa popular espontânea, onde se guerreava nas ruas com os famosos ``limões de cera'' perfumados ou mesmo bexigas de tripa de animal cheias d'água, e estimular um préstito no estilo dos \textit{carnavais europeus}. E assim, por iniciativa governamental, consolidou-se o \textit{carnaval}, hoje tido como modelo de festa popular graças a um processo de fabricação de tradições.

Desde a década de 1850 essa transição do entrudo para o carnaval já era realidade no Rio de Janeiro. Na Bahia demorou a se implantar esse modelo de brincadeira ``civilizada'', mas em 1884 foi anunciado através de uma portaria um novo formato da festa momesca; desfilaram em sua estreia os clubes carnavalescos Cruz Vermelha e Fantoches de Euterpe, com carros alegóricos no estilo dos carnavais de Nice, Paris e Veneza. Era o que se queria. Um desfile suntuoso para o povo ver e admirar, bestializado pela opulência e magnanimidade do cortejo. Desfilaram também outras agremiações que já participavam do Carnaval desde a década de 1870 como Sarrabulhada, por exemplo, mas sem os ``carros de ideias'' -- ou seja, os carros alegóricos -- das agremiações mais chiques. A iniciativa do governo vingou em parte. É que esqueceu-se de oferecer alternativas ao povo além da prerrogativa de assistir, até por os despossuídos não frequentarem os mesmos espaços de convivência de quem tinha posses e ainda se vivia num regime escravagista. Isto garantiu mais duas décadas de sobrevida ao entrudo, praticado desde então com a convivência da policia --- que fazia-lhe vistas grossas --- e da alta sociedade baiana, que condenava a pratica, mas não abria mão de brincar também, no seu modo, jogando limões de cera nos amigos ou até em desconhecidos. Desfile de carros alegóricos era muito bonito de se ver, mas entrudar os outros era tocar nas pessoas, tinha um ar de intimidade, de cumplicidade -- no dizer de um cronista, ``era o abre alas da paquera'' \cite{cadena_130carnaval_2017}.

Com o fim do entrudo, os libertos também reivindicaram sua parte nos festejos. O primeiro tempo dos blocos afros e afoxés ocorreu a partir de 1895, no 11º ano do carnaval oficial, assim considerado em função do poder público assumir a prerrogativa de ordenar o desfile das agremiações carnavalescas; naquele ano desfilou a \textit{Embaixada Africana}, organizada por Marcos Carpinteiro a partir de um terreiro do Engenho Velho de Brotas; era um bloco de ``misturados'' como se dizia naquele tempo \cite{cadena_doistemposafro_2017}. Já no carnaval de 1898 saíam às ruas, além da Embaixada Africana, os blocos \textit{Pândegos da África}, \textit{Filhos D'África} e \textit{Chegada Africana}\footnote{\textbf{A Coisa}, ano 1, nº 27, 27 fev. 1898.}, e até 1905, quando as agremiações com temática africana foram proibidas de desfilar por decreto, surgiram outros blocos como \textit{Africanos em Pândega}, \textit{Congada Africana}, \textit{Folia Africana}, \textit{Guerreiros da África}, \textit{Império da África}, \textit{Lembrança dos Africanos}, \textit{Lanceiros da África}, \textit{Lutadores da África}, \textit{Mamãe Arrumaria} e \textit{Papai Folia}.  A maioria era ligada ao candomblé e assim eram reconhecidos pejorativamente pela imprensa, que os chamava ``candomblés de rua''. Algum desses blocos desfilavam com carros alegóricos e muitas fantasias, pertinentes ao tema-enredo que evocava epopeias do continente negro \cite{cadena_doistemposafro_2017}. Com a proibição dos blocos africanos em 1905, todos encerraram suas atividades; os \textit{Filhos da Bahia} chegaram a tentar desfilar sem licença, mas não deu certo. Depois da Primeira Guerra Mundial, entretanto, voltaram a sair no carnaval soteropolitano chamando-se \textit{cordões}. São desse tempo os \textit{Nagôs em Folia}, \textit{Congos D’África} (vinculado a um terreiro de Omolú no Dique do Tororó) e os \textit{Pândegos da África} (que não é o mesmo de antes da guerra) \cite{cadena_doistemposafro_2017}.