\section{Espaço intraurbano de Salvador em momento de reformas}\label{sec:1.4}

Se o influxo populacional e o dinamismo econômico do século XIX resultaram na percepção da necessidade de mudanças na malha urbana de Salvador e na efetiva realização de reformas, como a construção do Teatro São João e do Passeio Público (1812), o código de posturas de 1844, o calçamento do Largo do Teatro (1846), a abertura da \index{rua da Vala}rua da Vala (1849), a construção do Campo Grande (1851), o início do abastecimento de água pelos chafarizes da Companhia do Queimado (1852) e a instalação de uma malha viária de \index{infraestruturas urbanas!transporte público}bondes \cite{fernandesgomes1992, fernandessampaiogomes1999, NASCIMENTO2007, SAMPAIO2005}, elas estiveram aquém do necessário. Salvador começou o século XX como uma cidade drasticamente afetada por problemas sanitários e habitacionais.

Salvador era subdividida, no período estudado, nos distritos urbanos de \textit{Brotas} (fundado em 1718), \index{distritos!urbanos!Conceição da Praia}\textit{Conceição} (f. 1623), \index{distritos!urbanos!Mares}\textit{Mares} (f. 1870), \index{distritos!urbanos!Nazaré}\textit{Nazaré} (f. 1897), \index{distritos!urbanos!Paço}\textit{Paço} (f. 1718), \index{distritos!urbanos!Penha}\textit{Penha} (f. 1745), \index{distritos!urbanos!Pilar}\textit{Pilar} (f. 1720), \index{distritos!urbanos!Santana}\textit{Santana} (1679), \index{distritos!urbanos!Santo Antônio}\textit{Santo Antônio} (1646), \index{distritos!urbanos!São Pedro}\textit{São Pedro Velho} (1679), \index{distritos!urbanos!Sé}\textit{Sé} (f. 1549) e \index{distritos!urbanos!Vitória}\textit{Vitória} (f. 1561) \cite[259-307]{VASCONCELOS2002}, e nos distritos suburbanos de \index{distritos!rurais!Cotegipe}\textit{Cotegipe} (f. 1608), \index{distritos!rurais!Itapuã}\textit{Itapuã} (f. 1608), \index{distritos!rurais!Maré}\textit{Maré} (f. 1832), \index{distritos!rurais!Matoim}\textit{Matoim} (f. 1609, hoje integrante do município de Candeias), \index{distritos!rurais!Paripe}\textit{Paripe} (f. 1608), \index{distritos!rurais!Passé}\textit{Passé} (f. 1608, hoje integrante do município de Candeias) e \index{distritos!rurais!Pirajá}\textit{Pirajá} (f. 1608) \cite[p.~53-62]{NASCIMENTO2007}. O nome, tamanho, limites e outras informações sobre os distritos foram estabelecidas por uma Lei Municipal de 05 de agosto de 1892\footnote{Embora tal informação conste no site do IBGE, todas as seis pastas de projetos de lei do período 1889-1930 constantes no Arquivo Histórico Municipal de Salvador foram consultadas em busca desta lei, sem sucesso.}.

\subsection{População e vetores de crescimento populacional}\label{subsubsec:populacaosalvador}

Os únicos entre os censos mais antigos que permitem destrinchar a população soteropolitana por distrito são os de 1872 e 1920\footnote{Sabe-se que estudos como os de \citeonline{NASCIMENTO2007} e \citeonline{COSTA1989} basearam-se em censos ainda mais antigos que inclusive desagregam os dados censitários em quarteirões, mas estas duas autoras apontaram o caráter precário de conservação em que se encontravam os documentos por elas consultados, além de inúmeras lacunas neles constantes; desta forma, optou-se por considerar o censo de 1872 como o mais antigo da série.}. Desta forma, é possível desenhar um ``antes'' e um ``durante'' dos distritos soteropolitanos no período estudado. A conhecida inexistência de um censo de 1930 e a falta dos dados distritais no censo de 1940 não permitem avançar muito além desta data. 

A \autoref{tab:evodisal1872} (p. \pageref{tab:evodisal1872}) e a \autoref{tab:evodisal1920} (p. \pageref{tab:evodisal1920}), que sintetizam os dados destes dois censos para a população dos distritos soteropolitanos, permitem comparar sua evolução populacional e observar as diferenças no desenvolvimento dos diferentes distritos. 

\begin{table}[!htp]
\IBGEtab{
\caption{População dos distritos soteropolitanos (1872)}\label{tab:evodisal1872}}
{
\begin{minipage}{18cm}
\begin{tiny}
\begin{tabular}{m{1.3cm} m{1cm} m{1cm} m{1.1cm}  m{1cm} m{0.7cm} m{0.7cm} m{1.1cm} m{1cm} m{1cm} m{1cm}}
\hline
\multirow{2}{*}{Distrito} & \multicolumn{4}{c}{Livres} & \multicolumn{4}{c}{Escravos} & \multicolumn{2}{c}{TOTAL} \\
\cline{2-11}	&Hom.	&Mul.	&TOTAL	&\% do segm.	&Hom.	&Mul. 	&TOTAL	&\% do segm.	&absol.	& \% da pop. \\
\hline
\multicolumn{11}{c}{Urbanos}\\
\hline
Brotas	&3.490	&1.006	&4.496	&3,48\%	&317	&277	&594	&0,46\%	&5.090	&3,94\%	\\
Conceição	&3.330	&1.010	&4.340	&3,36\%	&415	&735	&1.150	&0,89\%	&5.490	&4,25\%	\\
Mares	&1.828	&1.750	&3.578	&2,77\%	&84	&60	&144	&0,11\%	&3.722	&2,88\%	\\
Paço	&1.602	&1.596	&3.198	&2,48\%	&210	&228	&438	&0,34\%	&3.636	&2,82\%	\\
Penha	&2.341	&2.412	&4.753	&3,68\%	&543	&471	&1.014	&0,79\%	&5.767	&4,47\%	\\
Pilar	&3.868	&3.569	&7.437	&5,76\%	&490	&419	&909	&0,70\%	&8.346	&6,46\%	\\
Santana	&9.447	&8.047	&17.494	&13,55\%	&296	&164	&460	&0,36\%	&17.954	&13,91\%	\\
Santo Antônio	&7.257	&8.246	&15.503	&12,01\%	&515	&595	&1.110	&0,86\%	&16.613	&12,87\%	\\
São Pedro	&5.989	&6.408	&12.397	&9,60\%	&1.121	&1.225	&2.346	&1,82\%	&14.743	&11,42\%	\\
Sé	&5.874	&7.139	&13.013	&10,08\%	&1.105	&993	&2.098	&1,62\%	&15.111	&11,70\%	\\
Vitória	&5.493	&3.935	&9.428	&7,30\%	&989	&1.249	&2.238	&1,73\%	&11.666	&9,04\%	\\
TOTAL	&50.519	&45.118	&95.637	&74,07\%	&6.085	&6.416	&12.501	&9,68\%	&108.138	&83,76\%	\\
\hline
\multicolumn{11}{c}{Rurais}\\
\hline
Cotegipe	&1.052	&700	&1.752	&1,36\%	&180	&120	&300	&0,23\%	&2.052	&1,59\%	\\
Itapuã	&2.015	&2.266	&4.281	&3,32\%	&270	&384	&654	&0,51\%	&4.935	&3,82\%	\\
Maré	&496	&453	&949	&0,74\%	&95	&80	&175	&0,14\%	&1.124	&0,87\%	\\
Matoim	&837	&596	&1.433	&1,11\%	&447	&566	&1.013	&0,78\%	&2.446	&1,89\%	\\
Paripe	&1.189	&1.065	&2.254	&1,75\%	&488	&366	&854	&0,66\%	&3.108	&2,41\%	\\
Passé	&2.465	&1.334	&3.799	&2,94\%	&464	&180	&644	&0,50\%	&4.443	&3,44\%	\\
Pirajá	&1.246	&1.290	&2.536	&1,96\%	&172	&155	&327	&0,25\%	&2.863	&2,22\% \\
TOTAL	&9.300	&7.704	&17.004	&13,17\%	&2.116	&1.851	&3.967	&3,07\%	&20.971	&16,24\%	\\
\hline
\multicolumn{11}{c}{TOTAL GERAL}\\
\hline
Salvador	&59.819	&52.822	&112.641	&87,24\%	&8.201	&8.267	&16.468	&12,76\%	&129.109	&100\% \\
\hline
\end{tabular} 
\end{tiny}
\end{minipage}
}
{\fonte{Elaboração do autor, com dados de \citeonline[p.~508-510]{brasil_censo3_1876}.}}
\end{table}

\begin{table}[!htp]
\IBGEtab{
\caption{População dos distritos soteropolitanos (1920)}\label{tab:evodisal1920}}
{
\begin{tiny}
\begin{tabular}{m{1.3cm} m{1cm} m{1cm} m{1.1cm}  m{1cm} m{0.7cm} m{0.7cm} m{1.1cm} m{1cm} m{1cm} m{1cm} }
\hline
\multirow{2}{*}{Distrito} & \multicolumn{4}{c}{Brasileiros} & \multicolumn{4}{c}{Estrangeiros} & \multicolumn{2}{c}{TOTAL} \\
\cline{2-11}	&Hom.	&Mul.	&TOTAL	&\% do segm.	&Hom.	&Mul. 	&TOTAL	&\% do segm.	&absol.	& \% da pop. \\
\hline
\multicolumn{11}{c}{Urbanos}	\\
\hline
Brotas	&10.466	&12.274	&22.740	&8,02\%	&300	&81	&381	&0,13\%	&23.121	&8,16\% \\
Conceição	&2.429	&1.949	&4.378	&1,54\%	&177	&34	&211	&0,07\%	&4.589	&1,62\% \\
Mares	&6.331	&7.587	&13.918	&4,91\%	&252	&102	&354	&0,12\%	&14.272	&5,04\% \\
Nazaré	&5.189	&7.735	&12.924	&4,56\%	&391	&123	&514	&0,18\%	&13.438	&4,74\% \\
Paço	&2.948	&3.668	&6.616	&2,33\%	&392	&66	&458	&0,16\%	&7.074	&2,50\% \\
Penha	&8.413	&10.987	&19.400	&6,84\%	&276	&75	&351	&0,12\%	&19.751	&6,97\% \\
Pilar	&4.708	&5.059	&9.767	&3,45\%	&279	&62	&341	&0,12\%	&10.108	&3,57\% \\
Santana	&6.259	&9.064	&15.323	&5,41\%	&346	&70	&416	&0,15\%	&15.739	&5,55\% \\
Santo Antônio	&26.389	&29.620	&56.009	&19,76\%	&642	&191	&833	&0,29\%	&56.842	&20,06\% \\
São Pedro	&6.944	&10.727	&17.671	&6,23\%	&717	&278	&995	&0,35\%	&18.666	&6,59\% \\
Sé	&6.042	&8.175	&14.217	&5,02\%	&897	&294	&1.191	&0,42\%	&15.408	&5,44\% \\
Vitória	&18.999	&21.908	&40.907	&14,43\%	&1.135	&498	&1.633	&0,58\%	&42.540	&15,01\% \\
TOTAL	&105.117	&128.753	&233.870	&82,52\%	&5.804	&1.874	&7.678	&2,71\%	&241.548	&85,23\% \\
\hline
\multicolumn{11}{c}{Rurais}	\\
\hline
Cotegipe	&2.637	&1.616	&4.253	&1,50\%	&9	&1	&10	&0,00\%	&4.263	&1,50\% \\
Itapuã	&1.702	&1.744	&3.446	&1,22\%	&8	&3	&11	&0,00\%	&3.457	&1,22\% \\
Maré	&1.310	&1.400	&2.710	&0,96\%	&14	&5	&19	&0,01\%	&2.729	&0,96\% \\
Matoim	&1.664	&1.515	&3.179	&1,12\%	&6	&1	&7	&0,00\%	&3.186	&1,12\% \\
Paripe	&2.166	&1.965	&4.131	&1,46\%	&4	&0	&4	&0,00\%	&4.135	&1,46\% \\
Passé	&4.001	&3.981	&7.982	&2,82\%	&32	&15	&47	&0,02\%	&8.029	&2,83\% \\
Pirajá	&7.543	&8.388	&15.931	&5,62\%	&111	&33	&144	&0,05\%	&16.075	&5,67\% \\
TOTAL	&21.023	&20.609	&41.632	&14,69\%	&184	&58	&242	&0,09\%	&41.874	&14,77\% \\
\hline
\multicolumn{11}{c}{TOTAL GERAL} \\
\hline
Salvador	&126140	&149362	&275502	&97,21\%	&5988	&1932	&7920	&2,79\%	&283422	&100\% \\
\hline
\end{tabular} 
\end{tiny}
}
{ \fonte{Elaboração do autor, com dados de \citeonline[p.~32-43]{brasil_censo421_1920} .} }
\end{table}

Alguns fatos observados nestas tabelas são de especial interesse para esta pesquisa.

O primeiro deles é surpreendente crescimento populacional do distrito rural de Pirajá (461,47\%, média de 9,61\% a.a.), maior que o crescimento de qualquer outro distrito soteropolitano, seja ele rural ou urbano. É possível, a julgar pelo fato de o EPUCS, décadas depois, haver projetado concentrar a moradia operária neste distrito \cite{PREFEITURA1978,sampaio_formas_1999}, e pela proximidade dos empreendimentos industriais da península de Itapagipe, levantar a hipótese de que este crescimento tenha sido majoritariamente causado pela \textit{moradia operária} \cite{cardoso_vilas_1991}.

Destaca-se também o salto populacional dos distritos de Mares (283,45\%, média de 5,9\% a.a.), Penha (242,48\%, média de 5,05\% a.a.) e Santo Antônio (242,15\%, média de 5,04\% a.a.), refletindo, ao que indicam os dados sobre a distribuição da moradia proletária entre os distritos da cidade \cite[p.~126]{cardoso_vilas_1991}, o crescimento populacional no entorno dos empreendimentos industriais e o surgimento da \textit{moradia operária}, em suas diversas formas, tanto como demanda quanto como modelo de atuação do mercado imobiliário local, ainda que precário\footnote{As formas clássicas da moradia operária -- \index{casa!vila operária}vilas operárias, \index{casa!evoneia}evoneias e \index{casa!avenida}avenidas -- não esgotam a questão: ainda há que se considerar os casebres individuais, os conjuntos de casas para aluguel, as habitações coletivas e os cortiços, estes últimos combatidos pelo menos desde a Postura nº 39 do \index{legislação urbanística!Código de Posturas Municipais de Salvador (1921)}Código de Posturas Municipais de 1921 \cite{PREFEITURA1921}. A \index{casa!vila operária}\textit{vila operária} é uma quantidade numerosa de casas dispostas em ruas contíguas dentro de um mesmo terreno, via de regra construídas por alguma empresa fabril e próximas do local de trabalho; foram feitas sob encomenda para que os trabalhadores residissem mais próximos de seus locais de trabalho --- e para que fossem mais facilmente supervisionáveis em seus inter-relacionamentos sociais genéricos. A \index{casa!avenida}\textit{avenida} é uma fileira de casas pequenas, geralmente com fachadas muito semelhantes e divisões internas iguais, construídas como que ``em série'' para aluguel a preços módicos, geralmente destinadas a trabalhadores. A \index{casa!evoneia}\textit{evoneia} é a típica casa para trabalhadores, pequena, térrea e de poucos cômodos; seu nome vem da \textit{Companhia Evoneas Fluminenses}, que entre 1890 e 1900 dedicou-se à construção de casas para trabalhadores no Rio de Janeiro. }.

O acelerado crescimento populacional no distrito da \index{distritos!urbanos!Vitória}Vitória (264,65\%, média de 5,51\% a.a.) reflete o desenvolvimento de áreas populares como a Fazenda Garcia (405 construções entre 1889 e 1930) e Federação (93 construções), bem como a expansão da cidade rumo às áreas nobres do Campo Grande (24 construções), Vitória (74 construções), Graça/Barra Avenida (145 construções) e Barra (225 construções) \cite[p.~295]{almeida_victoria_1997}.

No distrito do \index{distritos!urbanos!Paço}Paço, o incremento populacional explica-se em grande parte pela singularidade deste distrito: sendo o menor distrito de Salvador em área, uma variação populacional que em outros distritos seria de menor monta (3.438 pessoas) toma aqui maiores proporções. Por outro lado, trata-se de um incremento --- 94,55\%, ou 1,40\% a.a. --- que fez a população do distrito quase \textit{dobrar} em quarenta e oito anos.

A relativa estagnação populacional dos distritos de \index{distritos!urbanos!São Pedro}São Pedro (26,61\%, média de 0,55\% a.a.) e \index{distritos!urbanos!Pilar}Pilar (21,11\%, média de 0,44\% a.a.) quando comparado seu crescimento populacional com o dos demais, só não foi pior que a completa estagnação populacional do distrito da Sé (1,97\%, média de 0,04\% a.a.), com variação de 297 pessoas nos quarenta e oito anos que separam os dois censos empregues (comparativamente, Brotas recebeu acréscimo de 18.031 pessoas no mesmo período). Enquanto o incremento populacional em \index{distritos!urbanos!São Pedro}São Pedro explica-se em parte pela migração de antigos moradores dos distritos da \index{distritos!urbanos!Sé}Sé e \index{distritos!urbanos!Paço}Paço, o fraco incremento populacional no \index{distritos!urbanos!Pilar}Pilar explica-se pela consolidação da mudança em seu perfil de uso, com redução dos usos residenciais preexistentes.

Embora tenha sido notável a perda de população no distrito de \index{distritos!urbanos!Santana}Santana (-12,35\%, média de -0,22\% a.a.), ela é facilmente explicada pelo desmembramento do distrito de \index{distritos!urbanos!Nazaré}Nazaré. Já a perda de população no distrito da \index{distritos!urbanos!Conceição da Praia}Conceição (-16,41\%, média de -0,55\% a.a.) reflete mudanças no perfil do seu uso, pois intensificou-se o uso comercial e foi sendo paulatinamente eliminado o uso residencial.

O incremento populacional concentrou-se nos distritos urbanos industriais (\index{distritos!urbanos!Mares}Mares, \index{distritos!urbanos!Penha}Penha, \index{distritos!urbanos!Santo Antônio}Santo Antônio) ou nos distritos semiurbanos, compostos por um setor urbano e um setor rural (\index{distritos!urbanos!Santo Antônio}Santo Antônio, Brotas) ou nos distritos rurais fronteiriços com os distritos industriais (\index{distritos!rurais!Pirajá}Pirajá), reflexos simultaneamente de pequena mudança no perfil de trabalho da população soteropolitana e do esgotamento da malha urbana consolidada, já incapaz de suportar o acréscimo populacional.

Para a presente pesquisa, entretanto, o fato mais importante a destacar é o vertiginoso crescimento populacional --- da ordem de 354,28\% (média de 7,38\% a.a.) --- no distrito de \index{Brotas!distrito de}Brotas, maior expansão demográfica entre todos os distritos urbanos de Salvador no período, a ser tratado com maior detalhe nos capítulos seguintes.

\subsection{Regime fundiário, densidade imobiliária e aproximações ao valor da terra}\label{subsubsec:polfundvalter}

No caso soteropolitano, a ocupação da terra se dava de maneira relativamente simples, dada a abundância do espaço e sua mercantilização quase inexistente \cite[p.~25]{MOURA1990}; o complicado era a documentação, o registro, a titulação da posse ou da propriedade. As \index{terra}terras de Salvador pertenciam basicamente a algumas ordens religiosas, a poucos \index{propriedade!proprietários}proprietários individuais e à Prefeitura \cite{CEDURB1978}; sendo assim, era comum que o soteropolitano, mesmo quando proprietário de sua casa, fosse mero \index{posse!posseiros}``foreiro'', ``rendeiro'' ou ``morador'' de terras de terceiros \cite[p.~139]{BRANDAO1980}. 

Seria interessante encontrar alguma forma direta de acesso ao \index{terra!valor da}\textit{valor da terra} nos distritos soteropolitanos no período pesquisado, mas diversos obstáculos impossibilitaram a tarefa: em primeiro lugar, nenhum dos documentos consultados trata a questão da forma direta como hoje se trata, por meio do preço do metro quadrado; em segundo lugar, a consulta à imprensa da época evidenciou que, ao contrário do que é hoje usual nos classificados imobiliários, o valor dos imóveis ofertados para venda ou aluguel nos jornais nunca era apresentado, sendo usual a orientação aos interessados de comparecerem ou bem à redação do jornal, ou aos endereços indicados nos anúncios para terem acesso aos preços e proceder às negociações; em terceiro lugar, a análise minuciosa das anotações dos livros das \textit{\index{décima urbana}décimas urbanas} custodiados no \textbf{Arquivo Histórico Municipal de Salvador}, única fonte onde o valor de cada imóvel é individuado e registrado ano a ano, seria trabalho já para outra dissertação com este foco específico, mesmo em se reduzindo a amostras quinquenais ou decenais o universo de livros de \index{décima urbana}décimas urbanas a serem pesquisados.

Não sendo possível trabalhar com o \index{terra!valor da}valor da terra tal como efetivamente foi registrado na época, é possível, todavia, conhecer o \textit{processo de valorização da terra} em Salvador de forma ao menos aproximativa, por caminhos oblíquos; nomeadamente, por meio da correlação entre o \index{valor locativo}\textit{valor locativo médio dos imóveis} e da \textit{taxa de ocupação residencial}.

Primeiro, é preciso encontrar o \Alsoindex{valor locativo}{terra!valor da}\index{valor locativo}valor locativo médio dos imóveis. 

A \autoref{tab:imoveis1924geral} (p. \pageref{tab:imoveis1924geral}) lista e compara os imóveis cadastrados pelo Município de Salvador para fim de arrecadação da \index{décima urbana}\Alsoindex{décima urbana}{terra!valor da}décima urbana em 1924, e por isto mesmo emprega critérios diferentes daqueles constantes no Censo e apresenta dados que por vezes com ele divergem\footnote{Não se pode esquecer o caráter precário dos censos realizados anteriormente a 1940, já referidos \cite{oliveirasimoes_censos_2005, reisetal_areascensos_2011}; a discrepância entre dados obtidos em fontes diversas num período bastante próximo é comum nas fontes encontradas, sendo possível trabalhar com eles apenas de modo aproximativo e com muitas cautelas.}. Ela permite uma série de análises diretas e extrapolações capazes de indicar não apenas o valor da terra de forma aproximativa, como também de lançar luzes sobre outras questões relativas ao desenvolvimento urbano de Salvador na Primeira República.

\begin{sidewaystable*}[!htp]
\centering
\IBGEtab{
\caption{Relação dos imóveis arrolados pelo município de Salvador nos distritos urbanos e suburbanos em 1924}\label{tab:imoveis1924geral}}
{
\begin{tiny}
\begin{tabular}{rrrrrrrrrrrrr}
\hline
\multirow{2}{*}{Distritos}&\multirow{2}{*}{Valor locativo}&\multicolumn{11}{c}{Imóveis}\\
\cline{3-13}
	&	&Térreos	&Sobrados	&Abarracados	&Barracão	&Telheiros	&Galpões	&Em ruínas	&Em construção	&Em reconstrução	&Interditados	&TOTAL\\
\hline 
\hline 
\multicolumn{13}{c}{Urbanos}\\
\hline 
Sé	&2855:505\$	&323	&573	&25	&1	&8	&1	&5	&6	&13	&0	&955\\
São Pedro	&3714:596\$	&1113	&660	&24	&1	&2	&4	&3	&4	&5	&0	&1816\\
Passo	&1280:900\$	&373	&277	&12	&0	&0	&0	&2	&0	&2	&0	&666\\
Conceição da Praia	&3675:038\$	&69	&326	&7	&4	&0	&0	&15	&0	&0	&2	&423\\
Pilar	&1874:596\$	&716	&238	&3	&2	&0	&1	&23	&5	&0	&0	&988\\
Mares	&1245:862\$	&1742	&67	&0	&10	&3	&1	&8	&15	&0	&0	&1846\\
Nazaré	&1559:070\$	&981	&197	&3	&1	&3	&1	&10	&11	&0	&0	&1207\\
Vitória	&4676:477\$	&4010	&558	&5	&8	&4	&0	&26	&32	&0	&0	&4643\\
Santana	&1951:010\$	&1571	&169	&1	&2	&1	&0	&1	&0	&0	&0	&1745\\
Penha	&1740:457\$	&2416	&126	&5	&2	&15	&1	&23	&14	&3	&5	&2610\\
Santo Antônio	&2401:906\$	&4773	&210	&24	&2	&3	&3	&27	&5	&0	&0	&5047\\
Brotas	&1346:364\$	&2662	&13	&10	&0	&0	&0	&4	&33	&6	&0	&2728\\
TOTAL 	&28321:781\$	&20749	&3414	&119	&33	&39	&12	&147	&125	&29	&7	&24674\\
\hline
\multicolumn{13}{c}{Suburbanos}\\
\hline
Itapuã	&25:456\$	&210	&0	&0	&0	&0	&0	&0	&3	&0	&0	&213\\
Pirajá	&314:634\$	&1732	&0	&0	&0	&0	&0	&0	&56	&4	&0	&1792\\
Paripe	&50:678\$	&308	&0	&0	&0	&0	&0	&1	&12	&2	&0	&323\\
Passé	&70:920\$	&578	&0	&0	&0	&0	&0	&0	&23	&0	&0	&601\\
Matoim	&11:148\$	&114	&0	&0	&0	&0	&0	&0	&1	&1	&0	&116\\
Maré	&29:192\$	&272	&0	&0	&0	&0	&0	&0	&5	&3	&0	&280\\
Cotegipe	&10:488\$	&75	&0	&0	&0	&0	&0	&0	&3	&0	&0	&78\\
TOTAL 	&512:516\$	&3289	&0	&0	&0	&0	&0	&1	&103	&10	&0	&3403\\
\hline
TOTAL GERAL	&28834:297\$	&27327	&3414	&119	&33	&39	&12	&149	&331	&49	&7	&31480\\
\hline
\end{tabular} 
\end{tiny}
}
{\fonte{\textbf{Annuario estatistico – annos de 1924 e 1925} organizado para o Governo da Bahia por M. Messias de \citeonline[p.~249]{bahia_annuario_1926}.}}
\end{sidewaystable*}

No que diz respeito a análises diretas, verificam-se fatos de interesse para esta pesquisa.

O maior número de imóveis \textit{em construção} -- ou seja, segundo os critérios encontrados na documentação pesquisada, imóveis \textit{novos}, construídos a partir de \textit{terrenos baldios} -- estava no distrito de \index{distritos!rurais!Pirajá}Pirajá (56), seguido de perto pelos da \index{distritos!urbanos!Vitória}Vitória (33) e \index{Brotas!distrito de}Brotas (32) e de longe por \index{distritos!urbanos!Mares}Mares (15), \index{distritos!urbanos!Penha}Penha (14) e \index{distritos!urbanos!Nazaré}Nazaré (11), reforçando a hipótese de uma tendência ao crescimento de Salvador para fora de seu núcleo tradicional.

O maior número de imóveis \textit{em reconstrução} -- ou seja, segundo os critérios de época, imóveis existentes que passavam por reformas parciais ou reconstruções completas -- estava no distrito da \index{distritos!urbanos!Sé}Sé (13), seguido de longe por Brotas (6), \index{distritos!urbanos!São Pedro}São Pedro (5) e \index{distritos!rurais!Pirajá}Pirajá (4); tal fato se dá, de um lado, como consequência das intervenções urbanas sobre a área central consolidada (a serem vistas na \autoref{subsec:1.4.3}, na p. \pageref{subsec:1.4.3}), e de outro como claro sintoma do incremento da atividade imobiliária em distritos vetores do crescimento urbano de Salvador no período estudado.

Partindo para as extrapolações a partir dos dados da tabela, é possível achegar-se, senão do valor da terra considerado por metro quadrado como deveria ser, de uma noção aproximativa acerca de quais distritos tinham a terra mais valorizada, quais a tinham menos valorizada, e em alguns casos o porquê da valorização ou da desvalorização.

Os valores apresentados na \autoref{tab:imoveis1924geral} (p. \pageref{tab:imoveis1924geral}) permitem obter um \index{valor locativo}\textit{valor locativo médio dos imóveis} em cada distrito. Chega-se a este valor dividindo-se a massa do \index{valor locativo}valor locativo dos imóveis de cada distrito, constante na segunda coluna da \autoref{tab:imoveis1924geral}, pelo número de imóveis do mesmo distrito, constante na última coluna da mesma tabela. O procedimento, rudimentar no mínimo, foi a solução encontrada para evitar as distorções verificadas quando tentada a comparação direta entre as massas de \index{valor locativo}valores locativos de cada distrito, por força das diferenças nos \index{valor locativo}valores locativos de cada imóvel e das grandes variações no número de imóveis por distrito. Os resultados deste procedimento podem ser vistos na \autoref{tab:valorlocativo1924geral} (p. \pageref{tab:valorlocativo1924geral}), e permitem chegar a outras conclusões de igual relevância para a pesquisa.

\begin{table}[!htp]
\IBGEtab{
\caption{Valor locativo médio dos imóveis nos distritos arrolados pelo município de Salvador em 1924}\label{tab:valorlocativo1924geral}}
{
\begin{tabular}{rr}
\hline
Locais	&Valor locativo médio por imóvel\\
\hline
\hline
\multicolumn{2}{c}{Urbanos}\\
\hline
Sé	&2:990\$060\\
São Pedro	&2:045\$480\\
Passo	&1:923\$270\\
Conceição da Praia	&8:688\$030\\
Pilar	&1:897\$360\\
Mares	&674\$900\\
Nazaré	&1:291\$690\\
Vitória	&1:007\$210\\
Santana	&1:118\$060\\
Penha	&666\$840\\
Santo Antônio	&475\$910\\
Brotas	&493\$540\\
Valor médio urbanos	&1:939\$360\\
\hline
\multicolumn{2}{c}{Suburbanos}\\
\hline
Itapuã	&14\$210\\
Pirajá	&974\$100\\
Paripe	&84\$320\\
Passé	&611\$380\\
Matoim	&39\$810\\
Maré	&374\$260\\
Cotegipe	&3\$080\\
Valor médio suburbanos	&300\$170\\
\hline
VALOR MÉDIO GERAL	&1:335\$450\\
\hline
\end{tabular} 
}
{\fonte{Elaboração do autor, com base em \citeonline[pp.~263-264]{bahia_annuario_1926}.}}
\end{table}

Embora o distrito da \index{distritos!urbanos!Conceição da Praia}Conceição da Praia concentrasse os imóveis ``mais caros'' entre todos os distritos (entre aspas por se tratar de valores médios, aproximativos, não de valores reais), cabe lembrar que nele se concentravam tanto o porto da cidade quanto número significante de casas de negócios e escritórios e os imóveis residenciais eram escassos, como visto na \autoref{tab:domsaldist1-1920} (p. \pageref{tab:domsaldist1-1920}), o que pode explicar a enorme discrepância entre o valor médio de seus imóveis e o de outros distritos.

Ressalvada a discrepância encontrada no distrito da \index{distritos!urbanos!Conceição da Praia}Conceição da Praia, o segundo distrito com imóveis ``mais caros'' era o da \index{distritos!urbanos!Sé}Sé, seguido por \index{distritos!urbanos!São Pedro}São Pedro, \index{distritos!urbanos!Paço}Paço, \index{distritos!urbanos!Pilar}Pilar, \index{distritos!urbanos!Nazaré}Nazaré, \index{distritos!urbanos!Santana}Santana e \index{distritos!urbanos!Vitória}Vitória. Isto reflete o fato de a burguesia e os latifundiários, assim como os profissionais liberais, encontrarem-se no período abrangido pelo censo em mudança rumo aos distritos de \index{distritos!urbanos!São Pedro}São Pedro e \index{distritos!urbanos!Vitória}Vitória, principalmente ao primeiro.

Os distritos urbanos com imóveis ``mais baratos'' entre os urbanos eram os dos \index{distritos!urbanos!Mares}Mares, da \index{distritos!urbanos!Penha}Penha, de Brotas e de \index{distritos!urbanos!Santo Antônio}Santo Antônio. As razões para valores tão baixos serão mais bem explicadas no \autoref{cap:3} (p. \pageref{cap:3}) a partir da análise dos \index{valor locativo}valores locativos de imóveis no distrito de Brotas.

Os distritos suburbanos com imóveis ``mais caros'', superando inclusive os \index{valor locativo}valores locativos médios por imóvel dos distritos urbanos de Brotas e \index{valor locativo}Santo Antônio, eram os de \index{distritos!rurais!Pirajá}Pirajá e \index{distritos!rurais!Passé}Passé, seguidos pelo de \index{distritos!rurais!Maré}Maré, este inaugurando a faixa dos distritos suburbanos com imóveis ``mais baratos'' que os dos distritos urbanos. O distrito de \index{distritos!rurais!Pirajá}Pirajá teve notável crescimento demográfico, como visto anteriormente, e a presença de fábricas entre seus imóveis certamente explica valores tão elevados, de outro modo inexplicáveis.

Os \index{valor locativo}valores locativos médios dos imóveis nos distritos de \index{distritos!rurais!Paripe}Paripe, \index{distritos!rurais!Matoim}Matoim, \index{distritos!rurais!Itapuã}Itapuã e \index{distritos!rurais!Cotegipe}Cotegipe eram quase irrisórios, fato facilmente explicável pela sua distância relativamente ao centro de Salvador e pelo fato de serem, realmente, áreas predominantemente rurais, sem qualquer interesse por parte dos agentes de produção do espaço urbano com maior atuação em Salvador no período estudado.

Há ainda outro indicador a observar, que pode reforçar as conclusões a que se chegou por meio do método anterior. Na medida em que os imóveis \textit{térreos} costumam ter menor valor agregado que os \textit{sobrados}\footnote{Somente a construção em dois andares já seria indício suficiente do maior valor agregado, mas o \textit{intérieur} dos sobrados, sua ornamentação, a riqueza das fachadas e estilo são suficientes para diferenciá-los da casa popular, térrea, muito mais simples e rústica \cite[pp.~87-111]{ott_formaet1_1955}.}, a \textit{proporção de cada um no total de imóveis do distrito} pode servir como outro indicador do valor da terra, ou do volume de investimento em cada distrito. Tendo como base os dados apresentados na \autoref{tab:imoveis1924geral} (p. \pageref{tab:imoveis1924geral}), é possível chegar a conclusões adicionais.

Surge uma possível explicação para a discrepância do \index{valor locativo}valor locativo médio dos imóveis do distrito da \index{distritos!urbanos!Conceição da Praia}Conceição da Praia frente aos demais distritos: ele tem a maior concentração de sobrados (77,07\%) frente aos imóveis térreos (16,31\%). Seguem-se à \index{distritos!urbanos!Conceição da Praia}Conceição da Praia no quesito de concentração de sobrados, em ordem decrescente, os distritos da \index{distritos!urbanos!Sé}Sé (60\%), \index{distritos!urbanos!Paço}Paço (41,59\%), \index{distritos!urbanos!São Pedro}São Pedro (36,34\%), \index{distritos!urbanos!Pilar}Pilar (24,09\%), \index{distritos!urbanos!Nazaré}Nazaré (16,32\%) e \index{distritos!urbanos!Vitória}Vitória (12,02\%). As menores concentrações de sobrados encontravam-se nos distritos de \index{distritos!urbanos!Santana}Santana (9,68\%), \index{distritos!urbanos!Penha}Penha (4,83\%), \index{distritos!urbanos!Santo Antônio}Santo Antônio (4,16\%), \index{distritos!urbanos!Mares}Mares (3,63\%) e Brotas (0,48\%), este último quase rivalizando com os distritos suburbanos, que não tinham sobrado algum.

Por meio destes dois métodos um tanto oblíquos, desenha-se assim um cenário no qual \index{distritos!urbanos!Conceição da Praia}Conceição da Praia, \index{distritos!urbanos!Sé}Sé, \index{distritos!urbanos!São Pedro}São Pedro e \index{distritos!urbanos!Paço}Paço surgem como os distritos urbanos onde há maior valorização da terra, e os distritos urbanos dos \index{distritos!urbanos!Mares}Mares, da \index{distritos!urbanos!Penha}Penha, de Brotas e do \index{distritos!urbanos!Santo Antônio}Santo Antônio aparecem como aqueles onde a terra era menos valorizada.

Em segundo lugar, é preciso encontrar a taxa de ocupação residencial.

Salvador, com 72,21 prédios por $km^{2}$ em 1920, era uma das capitais com maior \textit{densidade de prédios por metros quadrado}, ultrapassada apenas por Fortaleza (318,09 prédios/$km^{2}$), Recife (231,25 prédios/$km^{2}$), Niterói (162,09 prédios/$km^{2}$) e São Paulo (82,16 prédios/$km^{2}$) \cite[p.~XV]{brasil_censo46_1920}. O recenseamento de 1920, entretanto, não informa se esta densidade foi calculada tendo em vista a área total de cada município, ou apenas sua área urbana. Seria preciso detalhar a questão, e felizmente o próprio recenseamento de 1920, assim como outras fontes, permitem analisar o espaço urbano de Salvador com precisão um pouco maior.

\begin{table}[!htp]
\IBGEtab{
\caption{Estatística predial e domiciliária de Salvador, 1872-1920}\label{tab:estpredomsal1872-1920}}
{
\begin{tabular}{ccc}
\hline 
\multirow{2}{*}{Categoria} & \multicolumn{2}{c|}{Censos} \\ 
\cline{2-3} 
 & 1872 & 1920 \\ 
\hline 
População & 129.109 & 283.422 \\ 
\hline 
Prédios & 18.460 & 39.717 \\ 
\hline 
Domicílios & 24.894 & 40.615 \\ 
\hline 
Densidade predial & 6,99 & 7,14 \\ 
\hline 
Densidade domiciliária & 5,19 & 6,98 \\ 
\hline 
\end{tabular} 
}
{\fonte{Elaboração do autor, com dados de \citeonline[p.~XVI]{brasil_censo421_1920}.}}
\end{table}

A \autoref{tab:domsaldist1-1920} (p. \pageref{tab:domsaldist1-1920}) e a \autoref{tab:domsaldist2-1920} (p. \pageref{tab:domsaldist2-1920}), ambas do recenseamento de 1920, detalham os dados para Salvador acerca dos imóveis existentes e seus usos, explicitando inclusive o número de domicílios ocupados e vazios.

\begin{landscape}
\begin{table}[!htp]
\IBGEtab{
\caption{Estatística predial e domiciliar soteropolitana, por distrito, em 1920 (parte 1)}\label{tab:domsaldist1-1920}}
{
\begin{tiny}
\begin{tabular}{rrrrrrrrrrrrrr}
\hline
\multirow{3}{*}{Distrito} & \multicolumn{13}{c}{Domicílios}\\
\cline{2-14}
 & \multicolumn{2}{c|}{Particulares / residências} & \multicolumn{11}{c}{Coletivos} \\
\cline{2-14}
 & Ocupados & Desocup. & Asilos & Cadeias & Escolas & Fábricas& Fazendas e outros & Hospitais & Hotéis & Pensões ou & Quartéis & Diversos & TOTAL \\
 & & & & & & ou oficinas & estabelecimentos & & & casas de & & & \\
 & & & & & & & agrícolas & & & cômodos & & & \\
\hline
\multicolumn{14}{c}{Urbanos} \\
\hline
Brotas	&3.427	&140	&1	&0	&0	&0	&0	&1	&0	&3	&2	&1	&8\\
Conceição	&428	&10	&0	&0	&0	&0	&0	&0	&0	&7	&1	&0	&8\\
Mares	&1.786	&53	&1	&1	&0	&0	&0	&0	&0	&0	&0	&0	&2\\
Nazaré	&1.587	&70	&1	&0	&4	&0	&0	&2	&0	&0	&1	&0	&8\\
Paço	&990	&7	&1	&0	&0	&0	&0	&0	&0	&1	&1	&0	&3\\
Penha	&2.798	&142	&1	&0	&1	&0	&0	&2	&0	&1	&1	&0	&6\\
Pilar	&1.336	&16	&0	&0	&1	&0	&0	&0	&0	&18	&3	&0	&22\\
Santana	&1.998	&65	&1	&0	&3	&0	&0	&0	&0	&16	&2	&0	&22\\
Santo Antônio	&8.721	&251	&2	&0	&3	&0	&0	&1	&0	&2	&5	&0	&13\\
São Pedro	&2.206	&94	&2	&0	&8	&0	&0	&1	&2	&23	&3	&0	&39\\
Sé	&1.914	&15	&1	&0	&1	&0	&0	&0	&2	&78	&3	&0	&85\\
Vitória	&5.964	&148	&1	&1	&3	&0	&2	&2	&0	&18	&5	&0	&32\\
\hline
\multicolumn{14}{c}{Rurais} \\
\hline
Cotegipe	&691	&1	&0	&0	&0	&0	&0	&0	&0	&2	&0	&0	&2\\
Itapuã	&520	&3	&0	&0	&0	&0	&4	&0	&0	&0	&1	&0	&5\\
Maré	&413	&23	&0	&0	&0	&0	&0	&0	&0	&0	&1	&0	&1\\
Matoim	&571	&6	&0	&0	&0	&0	&0	&0	&0	&0	&1	&0	&1\\
Paripe	&609	&36	&0	&0	&0	&0	&1	&0	&0	&0	&0	&0	&1\\
Passé	&1.497	&49	&0	&0	&0	&0	&0	&0	&0	&11	&0	&0	&11\\
Pirajá	&2.883	&91	&0	&0	&0	&0	&0	&0	&0	&4	&3	&0	&7\\
\hline
TOTAL	&40.339	&1.220	&12	&2	&24	&0	&7	&9	&4	&184	&33	&1	&276\\
\hline
\end{tabular} 
\end{tiny}
}
{\fonte{Elaboração do autor, com dados de \citeonline[p.~108-109]{brasil_censo46_1920}.}}
\end{table}
\end{landscape}
\begin{landscape}
\begin{table}[!htp]
\centering
\IBGEtab{
\caption{Estatística predial e domiciliar soteropolitana, por distrito, 1920 (parte 2)}\label{tab:domsaldist2-1920}}
{
\begin{tiny}
\begin{tabular}{rrrrrrrrrrrrrrr}
\hline
\multirow{3}{*}{Distrito} & \multicolumn{12}{c}{Outras aplicações} & \multirow{3}{*}{Pop.} & \multirow{3}{*}{Dens.} \\
\cline{2-13}
 & \multirow{2}{*}{Depósitos} & \multirow{2}{1cm}{Escolas} & \multirow{2}{*}{Escritórios} & \multirow{2}{*}{Estações} & \multirow{2}{*}{Fábricas} & \multirow{2}{*}{Casas} & \multicolumn{3}{c}{Repartições administrativas} & \multirow{2}{*}{Templos} & \multirow{2}{*}{Diversas} & \multirow{2}{*}{TOTAL} & & \\
\cline{8-10} & & & & & ou oficinas & de negócio & Federais & Estaduais & Municipais & & & & & \\
\hline
\multicolumn{15}{c}{Urbanos} \\
\hline
Brotas	&3	&1	&0	&0	&1	&58	&1	&0	&0	&2	&0	&66	&23.121	&6,73 \\
Conceição	&53	&2	&230	&0	&30	&336	&4	&1	&0	&1	&18	&675	&4.589	&10,53 \\
Mares	&5	&2	&1	&0	&18	&76	&0	&1	&0	&2	&0	&105	&14.272	&7,98 \\
Nazaré	&2	&5	&2	&0	&4	&123	&0	&0	&0	&1	&0	&137	&13.438	&8,43 \\
Paço	&5	&1	&0	&0	&9	&98	&0	&0	&0	&2	&0	&115	&7.074	&7,12 \\
Penha	&5	&5	&0	&0	&4	&102	&0	&0	&0	&4	&1	&121	&19.751	&7,04 \\
Pilar	&73	&2	&45	&0	&45	&121	&0	&1	&1	&1	&3	&292	&10.108	&7,44 \\
Santana	&3	&3	&0	&1	&1	&161	&0	&3	&1	&7	&3	&183	&15.739	&7,79 \\
Santo Antônio	&5	&4	&0	&0	&7	&169	&3	&0	&0	&8	&3	&199	&56.842	&6,51 \\
São Pedro	&11	&7	&9	&0	&7	&240	&4	&3	&1	&2	&7	&291	&18.666	&8,31 \\
Sé	&4	&11	&15	&1	&11	&275	&3	&4	&4	&9	&13	&350	&15.408	&7,71 \\
Vitória	&6	&4	&0	&1	&2	&133	&1	&1	&0	&9	&7	&164	&42.540	&7,09 \\
\hline
\multicolumn{15}{c}{Rurais} \\
\hline
Cotegipe	&0	&1	&0	&1	&1	&2	&0	&0	&0	&0	&0	&5	&4.263	&6,15 \\
Itapuã	&1	&2	&0	&0	&1	&11	&0	&0	&1	&1	&0	&17	&3.457	&6,58 \\
Maré	&9	&0	&0	&0	&0	&14	&1	&0	&0	&3	&0	&27	&2.729	&6,59 \\
Matoim	&0	&1	&0	&0	&0	&4	&0	&0	&0	&1	&0	&6	&3.186	&5,57 \\
Paripe	&0	&3	&0	&2	&5	&10	&0	&0	&0	&1	&0	&21	&4.135	&6,78 \\
Passé	&0	&0	&0	&0	&1	&9	&0	&0	&0	&1	&0	&11	&8.029	&5,32 \\
Pirajá	&2	&6	&2	&0	&12	&41	&2	&1	&1	&6	&3	&76	&16.075	&5,56 \\
\hline
TOTAL	&187	&60	&304	&6	&159	&1.983	&19	&15	&9	&61	&58	&2.861	&283.422	&6,98 \\
\hline
\end{tabular} 
\end{tiny}
}
{\fonte{Elaboração do autor, com dados de \citeonline[p.~108-109]{brasil_censo46_1920}.}}
\end{table}
\end{landscape}	

\subsection{Intervenções no espaço urbano e seus agentes}\label{subsec:1.4.3}

Os dados populacionais e imobiliários acima não se bastam por si; são apenas expressões de mudanças operadas em Salvador pelos \textit{agentes de produção do espaço urbano}, já conceituados na \autoref{subsec:sociogeogrurb} (p. \pageref{subsec:sociogeogrurb}). Descendo do plano conceitual ao historicamente concreto, trata-se, no caso soteropolitano, de encontrar os \textit{agentes de produção do espaço urbano} na Salvador republicana em meio às intervenções no espaço urbano ocorridas entre 1889 e 1930. Um recorte: não se trata, evidentemente, de \textit{todas} as intervenções, pois a construção de uma casa, por si só, já caracteriza uma intervenção sobre o espaço; trata-se das intervenções de grande monta, envolvendo centenas ou milhares de imóveis, ou a implementação de grandes infraestruturas urbanas (ruas, esgoto etc.).

Tendo em mente este recorte, e ao contrário do senso comum que apregoa uma suposta ``falta de planejamento'' em Salvador, pode-se dizer que o espaço urbano de Salvador foi objeto de uma quantidade de intervenções, algumas somente projetadas, outras executadas parcial ou totalmente, mas todas razoavelmente planejadas. Além do \textit{plano de Manoel Rodrigues Teixeira} (1786) \cite[p.~318]{ruy_politica_1949} e das \textit{reformas do Conde dos Arcos} (séc. XIX) \cite{sampaio_formas_1999,VASCONCELOS2002}, que excedem o escopo desta pesquisa, as primeiras décadas do século XX foram prolíficas neste tipo de intervenções.

A primeira a merecer comentário, pela sua extensão, é o \textit{Plano de saneamento de Theodoro Sampaio} (1904). Como decorrência da encampação pela Intendência Municipal dos serviços, sempre precários, da Companhia do Queimado, foi aberta em 8 de novembro de 1904 concorrência para um plano de melhoramentos no setor, vencida pelo engenheiro baiano Theodoro Sampaio, único a apresentar proposta \cite[150]{gordilhobarbosa_eau_2004}. O trabalho foi financiado por um empréstimo feito pelo município de Salvador junto ao \textit{Banque de l'Union Parisienne} no valor de 25 milhões de francos \cite[p.~150]{gordilhobarbosa_eau_2004}, e marcou o início de um processo de intervenção planificada e efetiva do poder público sobre o setor de saneamento de Salvador \cite[p.~150]{gordilhobarbosa_eau_2004}. Os trabalhos foram parcialmente inaugurados em 1907, e abrangiam reformas e prolongamentos da rede de distribuição, assim como os primeiros 27km de redes de esgoto da cidade \cite[p.~151]{gordilhobarbosa_eau_2004}. O trabalho foi interrompido, segundo o próprio Theodoro Sampaio, porque o intendente municipal da época (Antônio Victor de Araújo Falcão) interrompeu o pagamento e o fornecimento de materiais, paralisando os trabalhos previstos, e passou a contratar terceiros para a canalização provisória dos esgotos da cidade \cite[p.~152]{gordilhobarbosa_eau_2004}.

Em seguida ao plano de Theodoro Sampaio, foi apresentado o \textit{Plano de melhoramentos de Alencar Lima} (1910). Quando apresentou seu ``Plano geral de melhoramentos em parte da cidade do Salvador'' à Intendência Municipal de Salvador, em 1910, o engenheiro Jerônimo Teixeira de Alencar Lima já era um dos sócios da Estrada de Ferro Central da Bahia, junto com Austricliano Honório de Carvalho \cite{souza_trabalholivre_2011}, e comprara o Lloyd Brasileiro em 1903, para depois vendê-lo ao Governo da Bahia e arrendá-lo novamente \cite[p.~220]{CUNHA2011}; não era, por assim dizer, um neófito no ramo da prestação de serviços públicos, tampouco desconhecido no cenário político baiano. Em resumo apertado, o projeto de Alencar Lima consistia em:

\begin{itemize}
\item Abertura da avenida Sete de Setembro;
\item Alargamento de ruas, em especial da avenida Carlos Gomes;
\item Construção de novas casas seguindo os critérios de higiene, arte e incombustibilidade;
\item Ajardinamento de praças e arborização de ruas;
\item Recomendação ao uso preferencial do concreto armado nas novas construções.
\end{itemize}

O projeto visava os distritos da \index{distritos!urbanos!Vitória}Vitória e de \index{distritos!urbanos!São Pedro}São Pedro, escolhidos explicitamente em virtude de serem zonas ``capazes de compensar-lhes o capital empregado'' \cite[p.~95]{CUNHA2011} -- e como visto na \autoref{subsubsec:polfundvalter} (p. \pageref{subsubsec:polfundvalter}), isto fazia todo o sentido. Este projeto interessa principalmente por antecipar, em linhas muito gerais, os polêmicos \index{melhoramentos urbanos!de Salvador (1912-1916)}\textit{``melhoramentos'' feitos no Centro de Salvador durante o primeiro período de \index{Primeira República brasileira (1889-1930)!política baiana!J. J. Seabra}J. J. Seabra à frente do Governo da Bahia (1912-1916)}. Tais melhoramentos, entretanto, não podem ser compreendidos de forma isolada; sucedem-se imediatamente às vultosas \index{melhoramentos urbanos!no porto de Salvador (1906-1921)}\textit{obras de melhoramento do porto de Salvador}, e por isto -- e outras razões, a serem vistas adiante -- precisam ser vistos em conjunto. É aqui onde se poderá perceber mais explicitamente o rebatimento, em Salvador, dos muitos conflitos sociais vistos nas seções anteriores com mais detalhes.

Ao contrário do que se costuma afirmar, não foi tanto a experiência de \index{Primeira República brasileira (1889-1930)!política baiana!J. J. Seabra}J. J. Seabra à frente do Ministério da Justiça, diretamente envolvido em certos aspectos da reforma urbana carioca e da infraestrutura portuária nacional, o fator determinante para sua postura mudancista relativamente a Salvador, como que incorporando em sua estadia à frente da pasta uma ``ideologia da reforma urbana radical'' \cite{sampaio_formas_1999}, e como se as ideologias, somente, movessem os sujeitos históricos, não interesses muito mais concretos; muito mais importante como força motriz do ``urbanismo demolidor'' do seu mandato governativo foi sua gestão no \index{governo federal!Ministério da Viação e Obras Públicas}\textit{Ministério da Viação e Obras Públicas} (1910-1912), onde, dando seguimento a uma política estabelecida por \index{Primeira República brasileira (1889-1930)!política baiana!Miguel Calmon}Miguel Calmon quando à frente da mesma pasta (1906-1909), estabeleceu frutuosos contatos com capitalistas franceses, nomeadamente com representantes da \textit{Caisse Commerciale et Industriale de Paris} e do \textit{Credit Mobilier Français}, subsidiária a primeira deste último, duas instituições financeiras já enfronhadas na economia baiana por meio da casa \textit{Wildberger \& Cia.}. 

A \textit{Caisse Commerciale} já havia constituído, em diálogo com \index{Primeira República brasileira (1889-1930)!política baiana!Miguel Calmon}Miguel Calmon, a \textit{Societé de Construction du Port de Bahia} para dar início às obras de ampliação e renovação tecnológica do porto de Salvador, que antes de sua chegada se arrastavam desde 1891 sem qualquer desenvolvimento \cite[p.~176-180]{rosado_porto_2016}\footnote{Diga-se de passagem que antes disto houve outra tentativa de modernização do porto de Salvador entre 1871 e 1879, de que foram concessionários primeiro os irmãos Francisco Ignácio Ferreira e Manuel Jesuíno Ferreira, com base num projeto de natureza semelhante elaborado por seu pai e indeferido em 1854; em seguida, a \textit{Bahia Docks Company Limited}, formada com um capital de 900 mil libras esterlinas e cuja diretoria era capitaneada por ninguém menos que o Visconde de Mauá; apesar de o engenheiro Charles Neat ter sido contratado para estudar os planos da obra, e de o mesmo engenheiro ter fácil acesso à Corte graças a outros projetos semelhantes (portos do Rio de Janeiro, Natal, João Pessoa e Recife), a obra mais uma vez ficou encalhada, a \textit{Bahia Docks} foi dissolvida em 1879 e a concessão foi declarada caduca em 1887 \cite[p.~175-176]{rosado_porto_2016}. A ampliação e atualização tecnológica do porto de Salvador eram reivindicação constante da Associação Comercial da Bahia \cite{CUNHA2011,joaci_porto_2016,rosado_porto_2016}}. Tratava-se de projeto realmente ambicioso: segundo o \textit{Plano da Viação Geral do Estado da Bahia} então acertado entre o ministro e os capitalistas franceses, Salvador seria o centro de um verdadeiro \textit{hub} de transportes; o porto soteropolitano seria substancialmente ampliado e modernizado, depois interligado à estação ferroviária da Calçada, ponto final de uma ampla malha férrea:

\begin{itemize}
\item Ligação da Estrada de Ferro S. Francisco, de Senhor do Bonfim, à Estrada Central da Bahia, em Iaçu (Sítio Novo), servindo a Campo Formoso, Jacobina, Morro do Chapéu, Mundo Novo, Orobó e Itaberaba. Tais ligações seriam feitas por ramais ou diretamente, a critério do resultado de estudos a serem realizados e, no caso de Campo Formoso e Morro do Chapéu, apenas se o governo estadual assim o aprovasse.
\item Ramal da Estrada de Ferro Central da Bahia, de Bandeira de Mello (Itaetê) até Brotas, por Andaraí e Lençóis.
\item Prolongamento da Estrada Central da Bahia, de Machado Portella (município de Marcionílio Souza), por Ituaçu, Bom Jesus dos Meiras (Brumado), Caetité, Monte Alto e Carinhanha, com um ramal por Condeúba até Montes Claros (MG), ponto final da Central do Brasil.
\item Ligação do ponto terminal da Central da Bahia, no Norte de Minas, à Estrada de Ferro Bahia e Minas, em Teófilo Ottoni (MG).
\item Ramal final da linha do Timbó, servindo a Itapicuru e Cipó \cite[pp.~215-217]{joaci_porto_2016}.
\end{itemize}

O plano, para conseguir tais objetivos, disciplinava a rede ferroviária no Estado prevendo o arrendamento de ferrovias federais, ampliação de linhas, construção de ramais, encampação de estradas estaduais pelo governo federal e a transferência destas ao arrendatário francês, resultando, caso concretizado o plano, num salto de 1.500 para 3 mil quilômetros de trilhos interligando várias regiões produtoras de matérias-primas ao porto de Salvador \cite[pp.~217]{joaci_porto_2016}. Iniciativa de grandíssimo porte, revolucionadora da infraestrutura de transportes na Bahia e igualmente da de comunicações, dado que a rede telegráfica acompanhava a malha ferroviária. Substituído \index{Primeira República brasileira (1889-1930)!política baiana!Miguel Calmon}Miguel Calmon no Ministério de Viação e Obras Públicas por Francisco Sá -- um mineiro --  durante o governo interino de Nilo Peçanha, a publicação dos decretos autorizativos da concessão das obras do porto e das ferrovias foi sendo retardada, efetivando-se um mês antes da posse de \index{Primeira República brasileira (1889-1930)!política baiana!J. J. Seabra}J. J. Seabra na pasta. Seabra, como visto, deu continuidade ao relacionamento, reformulando e ampliando o contrato inicial.

Ocorre que, no contexto da política dos Estados, para que um plano tão ousado desse certo era necessário manter-se alinhado ao governo federal. Seabra, entretanto, desentendido com o governo federal deste 1913, descarrilhado dos trilhos governistas, renitente nas apostas políticas equivocadas, viu fechadas as portas dos cofres federais garantidores dos empréstimos necessários à execução do Plano da Viação Geral do Estado da Bahia e lançou-se à banca internacional atrás de recursos, como tábua de salvação. Na acerba disputa política com seus opositores, \index{Primeira República brasileira (1889-1930)!política baiana!J. J. Seabra}J. J. Seabra e seus correligionários viram-se envolvidos numa longa série de denúncias de corrupção, que ao final mostraram-se verdadeiras: enquanto houve empréstimos externos, os negócios públicos corriam sem qualquer controle, a malversação dos recursos públicos tornou-se regra entre 1912 e 1916, o município de Salvador foi à falência\dots e para remediar a situação, novos empréstimos externos foram feitos por \index{Primeira República brasileira (1889-1930)!política baiana!J. J. Seabra}J. J. Seabra e seus correligionários, resultando em dois \textit{funding loans} junto à banca britânica \cite[pp.~217-221]{joaci_porto_2016}.

Em meio a este turbilhão, as obras do porto de Salvador deveriam ter sido iniciadas em 1906. O projeto -- com quatro capítulos e trinta e nove páginas --- detalhava profundas mudanças tecnológicas, que iam desde o aterro até a instalação de guindastes, vias férreas, diques para reparação de navios, quebra-mares (no plural mesmo), diversos armazéns especializados etc. \cite[p.~186]{rosado_porto_2016}. Ocorre que, contratada a obra e constituída a \textit{Companhia Cessionária das Docas do Porto da Bahia} pela \textit{Caisse Commerciale} como concessionária, o início dos serviços foi sendo protelado, o escopo da obra foi sendo encolhido, até que a Companhia Cessionária finalmente entregou a primeira parte da obra, o Cais da Alfândega, em 1911; apesar da cerimônia de inauguração das primeiras obras do cais em 13 de maio de 1913, ainda havia muito a construir, em especial o ``embelezamento'' de parte do Comércio. Mais prorrogações, novos decretos, novas modificações do escopo da obra seguiram-se, em 1920 --- treze anos depois do início das obras! --- a concessão passou por intensa revisão que resultou num termo de retificação \dots e em 1929 as obras prosseguiam, sem perspectiva de acabar, embora já em 1922 estivessem concluídas a linha férrea provisória do porto até a estação da Calçada, o edifício dos Correios, o Mercado Modelo, o alargamento da rua do Arsenal, o edifício da Capitania do Porto e os oito primeiros armazéns \cite[p.~186-196]{rosado_porto_2016}. 

No plano estritamente urbanístico, as obras do porto, mesmo atribuladas, haviam modificado significativamente a Cidade Baixa. A área entre o edifício da Alfândega e o sétimo armazém fora vendida pela Companhia das Docas à \textit{Companhia Imobiliária da Bahia}, que a loteou como ``Bairro das Nações''; a venda foi contestada pela Inspetoria dos Portos por exceder o estabelecido no contrato de concessão, e a ocupação da área foi muito lenta, efetivando-se apenas depois da Segunda Guerra Mundial \cite[p.~205-206]{almeida_vitrinescomercio_2014}\footnote{Este loteamento será visto mais uma vez no \autoref{cap:3}, pois parece ter-se tornado uma espécie de parâmetro por meio do qual a Diretoria de Obras de Salvador balizava a concessão de benesses a terratenentes loteadores.}. Muito antes disto, entretanto, pequenas intervenções tentavam, de um lado, facilitar a interligação entre o porto e outras áreas de Salvador (\index{infraestruturas urbanas!transporte público}bonde Corpo Santo/Itapagipe, navegação até a Barra, Itapagipe e Rio Vermelho), e de outro regularizar o arruamento da área (calçamento das ruas entre o Taboão e o Arsenal da Marinha, em 1897; irrigação das ruas entre a praça Riachuelo e a Alfândega, em 1898) \cite[p.~208-209]{almeida_vitrinescomercio_2014}. Além disto, o bairro do Comércio foi palco de grande remodelação estética seguindo os moldes ecléticos, da qual sobrevivem até hoje belíssimos exemplos, promovida pelos proprietários de imóveis desejosos de aproveitar a valorização fundiária vista nos distritos da Conceição e do Pilar em seguida à implementação de serviços públicos e à expectativa de início das obras do porto \cite[p.~251-304]{almeida_vitrinescomercio_2014}.

No que diz respeito ao famosíssimo \textit{plano de melhoramentos} de Seabra, tão comentado e esquadrinhado pela pesquisa acadêmica nas últimas décadas que torna desnecessária qualquer explicação ou narrativa mais detalhada, vê-se mais um aspecto a contradizer um suposto pendor seu por uma ``ideologia da reforma urbana radical'': em entrevista ao carioca \textbf{Jornal Imprensa} republicada pelo \textbf{Diário de Notícias} em 23 de março de 1912 as reformas urbanas sequer são mencionadas em meio a um vasto programa de governo então desenhado, e a mensagem inaugural de \index{Primeira República brasileira (1889-1930)!política baiana!J. J. Seabra}J. J. Seabra à frente do governo apresentada à Assembleia Legislativa em abril de 1912 fala em tudo, menos em reformas urbanas \cite[pp.~92-94]{CUNHA2011}. O mais provável, ao invés de uma ``ideologia da reforma urbana radical'' a mover \index{Primeira República brasileira (1889-1930)!política baiana!J. J. Seabra}J. J. Seabra rumo ao reordenamento urbano de Salvador, tenha sido a confluência de alguns fatores.

Destaca-se em primeiro lugar a influência do \textit{Crédit Mobilier Français} e da \textit{Caisse Commerciale et Industrielle de Paris}, que praticamente forçou \index{Primeira República brasileira (1889-1930)!política baiana!Miguel Calmon}Miguel Calmon, \index{Primeira República brasileira (1889-1930)!política baiana!J. J. Seabra}J. J. Seabra e o intendente Júlio Brandão a aceitar empréstimos vultosos para aplicação em ``trabalhos públicos''. Um destes empréstimos, contraído pelo intendente soteropolitano Júlio Brandão em 1913, foi uma das razões da falência de Salvador em 1922 \cite[pp.~119-122,~291]{CUNHA2011}. 

A disputa pela hegemonia na construção e gestão de infraestruturas urbanas (energia e transportes públicos) entre \textit{Percival Farquhar}, representante da \textit{Light} na Bahia, empresa controladora da \textit{Veículos Econômicos}, da \textit{Carris Elétricos} e da \textit{Compagnie d'Éclairage}, e a família Guinle, dona do grupo \textit{Guinle \& Cia}, empresa controladora da \textit{Companhia Linha Circular de Carris}, da \textit{Companhia de Transportes Urbanos} e da \textit{Trilhos Centrais}, reproduziu também em Salvador disputa verificada noutras cidades brasileiras. A produção e gestão oligopólicas da energia elétrica e dos transportes públicos em Salvador por estes dois grupos manteve estreita relação com a polarização política na Bahia entre 1911 e 1912, e antes disso é muito provável que tenha sido uma de suas causas.

É neste contexto que se pode entender a aliança estreita criada entre o grupo \textit{Guinle \& Cia} e J. J. Seabra, em especial por meio de Júlio Viveiros Brandão, gerente e acionista minoritário dos irmãos Guinle na Companhia Trilhos Centrais. A família Guinle, além de controlar a produção e a gestão da energia elétrica e dos transportes públicos em Salvador por meio da Guinle \& Cia, mantinha um irmão, Guilherme, como acionista da Linha Circular, e outro, Eduardo, como acionista da \textit{Companhia de Melhoramentos da Bahia}, empreiteira cuja atuação se verá adiante. Além disto, seis entre nove acionistas da Linha Circular sem vínculos familiares com os Guinle eram também sócios desta última \cite[pp.~122]{CUNHA2011}, todos eles então perfilando-se politicamente junto a Seabra.

Outro fato importante a destacar é a visita a Salvador, em agosto de 1911, do \textit{barão Amédée Reille-Soult-Dalmatie}, representante do \textit{Crédit Foncier du Brésil}, que viera fundar filial desta instituição financeira no mesmo prédio da \textit{Casa Wildberger \& Cia.} para ``desenvolver o crédito hipotecário e comercial, fazendo empréstimos sob garantia de imóveis urbanos, terrenos etc.`` e tinha assumidos interesses na ``construção de prédios e equipamentos urbanos'' \cite[pp.~115-116]{CUNHA2011}. Chegava a Salvador, portanto, um grupo capitalista interessado em atuar na produção de espaço urbanizado e também em ter imóveis como garantias para empréstimos, algo de certo modo inédito em Salvador, ao menos nesta escala.

Destaca-se também a retomada do plano Alencar Lima, ligeiramente modificado\footnote{A distribuição de obras entre município e Estado mudou: seria o primeiro responsável pelas obras entre a praça Castro Alves e a praça do Palácio, enquanto o último geriria as obras entre a ladeira de São Bento e o Rio Vermelho. A avenida que depois viria a ser chamada de Sete de Setembro teve seu traçado ampliado para ultrapassar o Campo Grande e chegar até a Barra; daí em diante seria aberta outra avenida, hoje conhecida como a Oceânica, que ligasse a Barra ao Rio Vermelho. Por último, a reconstrução dos prédios bombardeados em janeiro de 1912 foi incorporada ao plano \cite[p.~101]{CUNHA2011}.}, para satisfazer a opinião pública e a imprensa, desejosas ambas --- estas sim --- de alterações profundas na estrutura e estética urbanas de Salvador.

Os \index{melhoramentos urbanos!de Salvador (1912-1916)}melhoramentos da era \index{Primeira República brasileira (1889-1930)!política baiana!J. J. Seabra}J. J. Seabra tiveram, sim, um componente ideológico ``modernizante'', mas é necessário vê-lo em contexto. Já se mencionou na \autoref{subsubsec:ecobasa} (p. \pageref{subsubsec:ecobasa}) e na \autoref{subsec:cultespubsaba} (p. \pageref{subsec:cultespubsaba}) como a burguesia e os gestores na Bahia associavam ao ``atraso'' quaisquer expressões laborais, culturais, arquitetônicas, urbanísticas ou estéticas do passado colonial e imperial, e como tal aversão a tudo quanto representasse o ``atraso'' era uma das formas de distinção entre classes sociais e de afirmação ideológica destas duas classes capitalistas, na medida em que as mais vituperadas expressões do ``atraso'' ligavam-se a práticas da classe trabalhadora e da multidão escravizada que a antecedeu. Os \index{melhoramentos urbanos!de Salvador (1912-1916)}``melhoramentos'' seabristas, ao tempo em que se apresentam como ponte para o futuro e reforçam a negação do ``atraso'', servem também a interesses capitalistas muito precisamente identificados, em especial a especulação imobiliária tocada, entre outros agentes, pelo \textit{Crédit Foncier du Brésil}, pela Companhia de Melhoramentos da Bahia e pela sua subempreiteira, a Laffayette \& Comp. A ação especulativa da Companhia de Melhoramentos teve inclusive desdobramentos em Brotas, como se verá no \autoref{cap:3} (p. \pageref{cap:3}).

Deste modo, a partir das intervenções no espaço urbano é possível encontrar alguns dos \index{agentes de produção do espaço urbano}agentes de produção do espaço urbano de Salvador:

\begin{itemize}
\item \index{agentes de produção do espaço urbano!Estado}\textbf{Governo federal:} não agiu diretamente, mas, por meio do Ministério de Viação e Obras Públicas, foi responsável por viabilizar as obras do porto de Salvador, redefinidoras do espaço urbano da Cidade Baixa.
\item \index{agentes de produção do espaço urbano!Estado}\textbf{Governo estadual:} aparentemente o principal agente da produção do espaço urbano de Salvador durante a Primeira República, dado o vulto dos investimentos em energia elétrica, iluminação pública e ``melhoramentos urbanos'' em geral.
\item \index{agentes de produção do espaço urbano!Estado}\textbf{Intendência municipal:} sócia menor da produção do espaço urbano de Salvador, responsável pela gestão dos serviços públicos, viu-se transformada subitamente em entreposto de empresas nacionais e estrangeiras, em especial sendo levada a encampar os serviços da \textit{Bahia Tramway Light \& Power} derrotada na concorrência com a Guinle \& Cia para assim minimizar os prejuízos de Percival Farquhar em sua retirada estratégica da praça baiana em 1913 \cite[p.~76-89]{CUNHA2011}. 
\item \index{agentes de produção do espaço urbano!promotores imobiliários}\textbf{Empresas estrangeiras:} as empresas creditícias francesas multicitadas --- \textit{Caisse Commerciale et Industriale de Paris} e \textit{Credit Mobilier Français} --- foram dominantes no período. No campo da construção e gestão de infraestruturas urbanas, a alemã \textit{Siemens \& Halske S. A.} comprou em 1898 a \textit{Veículos Econômicos} e a \textit{Carris Eléctricos da Bahia}, empresas locais de fornecimento de energia e de transporte público, reunindo-as num só conglomerado \cite[p.~36]{CUNHA2011}. Em 1906 este conjunto, e também a belga \textit{Compagnie d'Éclairage da Bahia} e a \textit{Bahia Gaz and Electric Company}, foram compradas pelo grupo \textit{Light} para formar a criação da \textit{Bahia Tramway, Light \& Power Company}, compra esta previamente acertada entre a \textit{Siemens} e a \textit{Light} em acordos firmados no México (1903) e no Rio de Janeiro (1904) \cite[p.~34]{CUNHA2011}. A \textit{Light} saiu da praça baiana em 1913 com a encampação municipal já referida.
\item \index{agentes de produção do espaço urbano!promotores imobiliários}\textbf{Empresas nacionais:}\footnote{Trata-se de empresas com origem em outros Estados brasileiros.} Guinle \& Cia., embora atuando associadamente à \textit{General Electric Co.} em algumas frentes, tinha nos contratos de obras públicas sua principal fonte de capitalização, em especial nos setores ferroviário, portuário, eletricitário e de transportes públicos \cite[p.~43-44]{CUNHA2011}.
\item \index{agentes de produção do espaço urbano!promotores imobiliários}\textbf{Empresas locais:} o período dos ``melhoramentos urbanos'' foi oportuno também para o surgimento de \textit{construtoras}: datam de então, além da Companhia Melhoramentos propriamente dita, a \textit{Companhia Comercial Construtora}, a \textit{Companhia Empreiteira Lafayette \& Cia.}, a \textit{Laffayette \& Comp.} e a \textit{Sociedade Construtura Brasileira Alencar Lima \& Cia.} \cite{flexor_higi_2011}.
\item \index{agentes de produção do espaço urbano!proprietários fundiários}\textbf{Proprietários de terras e imóveis:} durante toda a Primeira República perceberam oportunidades no crescimento populacional, na expansão territorial e nas obras públicas prometidas ou realizadas, e investiram pesadamente na construção e reforma de imóveis para valorizá-los \cite{almeida_victoria_1997,almeida_vitrinescomercio_2014}.
\end{itemize}

A influência destes agentes, como visto, se dá em torno das áreas onde houve maior expectativa de \textit{valorização da terra por meio de obras públicas}. A esta constatação, já adiantando parte do assunto a ser exposto no \autoref{cap:3} (p. \pageref{cap:3}), deve-se somar o fato de que lá onde não existia tal expectativa, os agentes de produção do espaço urbano eram, além destes, outros, cuja atuação somente se verifica por meio da análise pormenorizada do desenvolvimento urbano de cada freguesia/distrito de Salvador.