\section{Espaço urbano soteropolitano em momento de reformas}\label{sec:1.4}

\begin{figure}[!htp]
\centering
\includegraphics[width=1\textwidth]{2-cap1/complementos/mapas/1855weyll.eps} 
\caption{``Mappa topographica da cidade de S. Salvador e seus subúrbios'', elaborada pelo engenheiro alemão Carlos Augusto Weyll e posteriormente datada como sendo de 1851. Ainda é a mais detalhada fonte cartográfica para o estudo de Salvador no período que vai até os anos 1930. \textbf{Fonte:} \citeonline{weyll_mappa_1851}.}
\end{figure}

Em seguida, esta sociedade será relacionada à produção, uso e apropriação do espaço urbano onde está situada mediante uma análise do grau de urbanização dos dez distritos urbanos \cite{NASCIMENTO2007, VASCONCELOS2002}, do regime de uso e apropriação da terra então vigentes \cite{CEDURB1978}, dos agentes públicos e privados de produção do espaço urbano e das reformas urbanas \cite{cardoso_vilas_1991, CUNHA2011}, sejam as projetadas, sejam as efetivamente executadas.

Se o fortíssimo influxo populacional e o dinamismo econômico do século XIX resultaram na percepção da necessidade de mudanças na malha urbana de Salvador e na efetiva realização de reformas, como a construção do Teatro São João e do Passeio Público (1812), o código de posturas de 1844, o calçamento do Largo do Teatro (1846), a abertura da Rua da Vala (1849), a construção do Campo Grande (1851), o início do abastecimento de água pelos chafarizes da Companhia do Queimado (1852), a instalação de uma malha viária de bondes \cite{fernandesgomes1992, fernandessampaiogomes1999,NASCIMENTO2007,sampaio_50_2005}, elas estiveram aquém do necessário.

\subsection{Distritos de Salvador}\label{subsec:1.4.1}

Salvador era subdividida, no período estudado, nos distritos urbanos de \textit{Brotas} (fundada em 1718), \textit{Conceição} (f. 1623), \textit{Mares} (f. 1870), \textit{Nazaré} (f. 1897), \textit{Paço} (f. 1718), \textit{Penha} (f. 1745), \textit{Pilar} (f. 1720), \textit{Santana} (1679), \textit{Santo Antônio }(1646), \textit{São Pedro Velho} (1679), \textit{Sé} (f. 1549) e \textit{Vitória} (f. 1561) \cite[259-307]{VASCONCELOS2002}, e nos distritos suburbanos de \textit{Cotegipe} (f. 1608), \textit{Itapuã} (f. 1608), \textit{Maré} (f. 1832), \textit{Matoim} (f. 1609, hoje integrante do município de Candeias), \textit{Paripe} (f. 1608), \textit{Passé} (f. 1608, hoje integrante do município de Candeias) e \textit{Pirajá} (f. 1608) \cite[p.~53-62]{NASCIMENTO2007}. O nome, tamanho, limites e outras informações sobre os distritos foram estabelecidas por uma Lei Municipal de 05 de agosto de 1892.

Os únicos entre os censos mais antigos que permitem destrinchar a população soteropolitana por distrito são os de 1872 e 1920. Desta forma, é possível desenhar um ``antes'' e um ``durante'' dos distritos soteropolitanos no período estudado. A conhecida inexistência de um censo de 1930 e a falta dos dados distritais no censo de 1940 não permitem avançar muito além desta data. 

São de especial interesse para esta pesquisa:

\begin{itemize}
\item O surpreendente crescimento populacional do distrito rural de Pirajá (461,47\%), maior que o crescimento de qualquer outro distrito soteropolitano, seja ele rural ou urbano. É possível, a julgar pelo fato de o EPUCS, décadas depois, haver projetado concentrar a moradia operária neste distrito \cite{PREFEITURA1978,sampaio_formas_1999}, e pela proximidade dos empreendimentos industriais da península de Itapagipe, levantar a hipótese de que este crescimento tenha sido majoritariamente causado pela \textit{moradia operária} \cite{cardoso_vilas_1991}.
\item O salto populacional dos distritos de Mares (283,45\%), Penha (242,48\%) e Santo Antônio (242,15\%), refletindo, ao que indicam os dados sobre a distribuição da moradia proletária entre os distritos da cidade \cite[p.~126]{cardoso_vilas_1991}, o crescimento populacional no entorno dos empreendimentos industriais e o surgimento da \textit{moradia operária} tanto como demanda quanto como modelo de atuação do mercado imobiliário.
\item O vertiginoso crescimento populacional da ordem de 354,28\% no distrito de Brotas, maior expansão demográfica entre todos os distritos urbanos de Salvador no período, a ser tratado com maior detalhe nos dois capítulos seguintes.
\item O acelerado crescimento populacional no distrito da Vitória (264,65\%), que reflete o desenvolvimento de áreas populares como a Fazenda Garcia (405 construções entre 1889 e 1930) e Federação (93 construções), bem como a expansão da cidade rumo às áreas nobres do Campo Grande (24 construções), Vitória (74 construções), Graça/Barra Avenida (145 construções) e Barra (225 construções) (ALMEIDA, 1997, p. 295).
\item A relativa estagnação populacional dos distritos de São Pedro (26,61\%) e Pilar (21,11\%) quando comparado seu crescimento populacional com o dos demais. 
\item A completa estagnação populacional do distrito da Sé (1,97\%), com variação de 297 pessoas nos quarenta e oito anos que separam os dois censos empregues (comparativamente, Brotas recebeu acréscimo de 18.031 pessoas no mesmo período).
\item A perda de população no distrito de Santana (-12,35\%), explicada pelo desmembramento do distrito de Nazaré.
\item A perda de população no distrito da Conceição (-16,41\%), refletindo mudanças no perfil do seu uso.
\item A concentração do crescimento populacional nos distritos urbanos industriais (Mares, Penha, Santo Antônio) ou nos distritos semi-urbanos, compostos por um setor urbano e um setor rural (Santo Antônio, Brotas) ou nos distritos rurais fronteiriços com os distritos industriais (Pirajá), reflexos simultaneamente de pequena mudança no perfil de trabalho da população soteropolitana e de esgotamento da malha urbana consolidada, já incapaz de suportar o acréscimo populacional.
\end{itemize}

A \autoref{tab:evodisal1872} e a \autoref{tab:evodisal1920}, que sintetizam os dados destes dois censos para a população dos distritos soteropolitanos, permitem comparar sua evolução populacional e observar as diferenças no desenvolvimento dos diferentes distritos. 

\begin{table}[!htp]
\IBGEtab{
\caption{População dos distritos soteropolitanos (1872)}\label{tab:evodisal1872}}
{
\begin{minipage}{18cm}
\begin{tiny}
\begin{tabular}{m{1.3cm} m{1cm} m{1cm} m{1.1cm}  m{1cm} m{0.7cm} m{0.7cm} m{1.1cm} m{1cm} m{1cm} m{1cm}}
\hline
\multirow{2}{*}{Distrito} & \multicolumn{4}{c}{Livres} & \multicolumn{4}{c}{Escravos} & \multicolumn{2}{c}{TOTAL} \\
\cline{2-11}	&Hom.	&Mul.	&TOTAL	&\% do segm.	&Hom.	&Mul. 	&TOTAL	&\% do segm.	&absol.	& \% da pop. \\
\hline
\multicolumn{11}{c}{Urbanos}\\
\hline
Brotas	&3.490	&1.006	&4.496	&3,48\%	&317	&277	&594	&0,46\%	&5.090	&3,94\%	\\
Conceição	&3.330	&1.010	&4.340	&3,36\%	&415	&735	&1.150	&0,89\%	&5.490	&4,25\%	\\
Mares	&1.828	&1.750	&3.578	&2,77\%	&84	&60	&144	&0,11\%	&3.722	&2,88\%	\\
Paço	&1.602	&1.596	&3.198	&2,48\%	&210	&228	&438	&0,34\%	&3.636	&2,82\%	\\
Penha	&2.341	&2.412	&4.753	&3,68\%	&543	&471	&1.014	&0,79\%	&5.767	&4,47\%	\\
Pilar	&3.868	&3.569	&7.437	&5,76\%	&490	&419	&909	&0,70\%	&8.346	&6,46\%	\\
Santana	&9.447	&8.047	&17.494	&13,55\%	&296	&164	&460	&0,36\%	&17.954	&13,91\%	\\
Santo Antônio	&7.257	&8.246	&15.503	&12,01\%	&515	&595	&1.110	&0,86\%	&16.613	&12,87\%	\\
São Pedro	&5.989	&6.408	&12.397	&9,60\%	&1.121	&1.225	&2.346	&1,82\%	&14.743	&11,42\%	\\
Sé	&5.874	&7.139	&13.013	&10,08\%	&1.105	&993	&2.098	&1,62\%	&15.111	&11,70\%	\\
Vitória	&5.493	&3.935	&9.428	&7,30\%	&989	&1.249	&2.238	&1,73\%	&11.666	&9,04\%	\\
TOTAL	&50.519	&45.118	&95.637	&74,07\%	&6.085	&6.416	&12.501	&9,68\%	&108.138	&83,76\%	\\
\hline
\multicolumn{11}{c}{Rurais}\\
\hline
Cotegipe	&1.052	&700	&1.752	&1,36\%	&180	&120	&300	&0,23\%	&2.052	&1,59\%	\\
Itapuã	&2.015	&2.266	&4.281	&3,32\%	&270	&384	&654	&0,51\%	&4.935	&3,82\%	\\
Maré	&496	&453	&949	&0,74\%	&95	&80	&175	&0,14\%	&1.124	&0,87\%	\\
Matoim	&837	&596	&1.433	&1,11\%	&447	&566	&1.013	&0,78\%	&2.446	&1,89\%	\\
Paripe	&1.189	&1.065	&2.254	&1,75\%	&488	&366	&854	&0,66\%	&3.108	&2,41\%	\\
Passé	&2.465	&1.334	&3.799	&2,94\%	&464	&180	&644	&0,50\%	&4.443	&3,44\%	\\
Pirajá	&1.246	&1.290	&2.536	&1,96\%	&172	&155	&327	&0,25\%	&2.863	&2,22\% \\
TOTAL	&9.300	&7.704	&17.004	&13,17\%	&2.116	&1.851	&3.967	&3,07\%	&20.971	&16,24\%	\\
\hline
\multicolumn{11}{c}{TOTAL GERAL}\\
\hline
Salvador	&59.819	&52.822	&112.641	&87,24\%	&8.201	&8.267	&16.468	&12,76\%	&129.109	&100\% \\
\hline
\end{tabular} 
\end{tiny}
\end{minipage}
}
{\fonte{Elaboração do autor, com dados de \citeonline[p.~508-510]{brasil_censo3_1876}.}}
\end{table}

\begin{table}[!htp]
\IBGEtab{
\caption{População dos distritos soteropolitanos (1920)}\label{tab:evodisal1920}}
{
\begin{tiny}
\begin{tabular}{m{1.3cm} m{1cm} m{1cm} m{1.1cm}  m{1cm} m{0.7cm} m{0.7cm} m{1.1cm} m{1cm} m{1cm} m{1cm} }
\hline
\multirow{2}{*}{Distrito} & \multicolumn{4}{c}{Brasileiros} & \multicolumn{4}{c}{Estrangeiros} & \multicolumn{2}{c}{TOTAL} \\
\cline{2-11}	&Hom.	&Mul.	&TOTAL	&\% do segm.	&Hom.	&Mul. 	&TOTAL	&\% do segm.	&absol.	& \% da pop. \\
\hline
\multicolumn{11}{c}{Urbanos}	\\
\hline
Brotas	&10.466	&12.274	&22.740	&8,02\%	&300	&81	&381	&0,13\%	&23.121	&8,16\% \\
Conceição	&2.429	&1.949	&4.378	&1,54\%	&177	&34	&211	&0,07\%	&4.589	&1,62\% \\
Mares	&6.331	&7.587	&13.918	&4,91\%	&252	&102	&354	&0,12\%	&14.272	&5,04\% \\
Nazaré	&5.189	&7.735	&12.924	&4,56\%	&391	&123	&514	&0,18\%	&13.438	&4,74\% \\
Paço	&2.948	&3.668	&6.616	&2,33\%	&392	&66	&458	&0,16\%	&7.074	&2,50\% \\
Penha	&8.413	&10.987	&19.400	&6,84\%	&276	&75	&351	&0,12\%	&19.751	&6,97\% \\
Pilar	&4.708	&5.059	&9.767	&3,45\%	&279	&62	&341	&0,12\%	&10.108	&3,57\% \\
Santana	&6.259	&9.064	&15.323	&5,41\%	&346	&70	&416	&0,15\%	&15.739	&5,55\% \\
Santo Antônio	&26.389	&29.620	&56.009	&19,76\%	&642	&191	&833	&0,29\%	&56.842	&20,06\% \\
São Pedro	&6.944	&10.727	&17.671	&6,23\%	&717	&278	&995	&0,35\%	&18.666	&6,59\% \\
Sé	&6.042	&8.175	&14.217	&5,02\%	&897	&294	&1.191	&0,42\%	&15.408	&5,44\% \\
Vitória	&18.999	&21.908	&40.907	&14,43\%	&1.135	&498	&1.633	&0,58\%	&42.540	&15,01\% \\
TOTAL	&105.117	&128.753	&233.870	&82,52\%	&5.804	&1.874	&7.678	&2,71\%	&241.548	&85,23\% \\
\hline
\multicolumn{11}{c}{Rurais}	\\
\hline
Cotegipe	&2.637	&1.616	&4.253	&1,50\%	&9	&1	&10	&0,00\%	&4.263	&1,50\% \\
Itapuã	&1.702	&1.744	&3.446	&1,22\%	&8	&3	&11	&0,00\%	&3.457	&1,22\% \\
Maré	&1.310	&1.400	&2.710	&0,96\%	&14	&5	&19	&0,01\%	&2.729	&0,96\% \\
Matoim	&1.664	&1.515	&3.179	&1,12\%	&6	&1	&7	&0,00\%	&3.186	&1,12\% \\
Paripe	&2.166	&1.965	&4.131	&1,46\%	&4	&0	&4	&0,00\%	&4.135	&1,46\% \\
Passé	&4.001	&3.981	&7.982	&2,82\%	&32	&15	&47	&0,02\%	&8.029	&2,83\% \\
Pirajá	&7.543	&8.388	&15.931	&5,62\%	&111	&33	&144	&0,05\%	&16.075	&5,67\% \\
TOTAL	&21.023	&20.609	&41.632	&14,69\%	&184	&58	&242	&0,09\%	&41.874	&14,77\% \\
\hline
\multicolumn{11}{c}{TOTAL GERAL} \\
\hline
Salvador	&126140	&149362	&275502	&97,21\%	&5988	&1932	&7920	&2,79\%	&283422	&100\% \\
\hline
\end{tabular} 
\end{tiny}
}
{ \fonte{Elaboração do autor, com dados de \citeonline[p.~32-43]{brasil_censo421_1920} .} }
\end{table}

\subsection{Regime de terras e densidade domiciliar}\label{subsec:1.4.2}

Não há política urbana sem política fundiária, e não há política fundiária sem um regime de terras correspondente. No caso soteropolitano, a ocupação da terra se dava de maneira relativamente simples, dada a abundância do espaço e sua mercantilização quase inexistente \cite[p.~25]{MOURA1990}; o complicado era a documentação, o registro, a titulação da posse ou da propriedade.

As terras de Salvador pertenciam basicamente a algumas ordens religiosas, a poucos proprietários individuais e à Prefeitura \cite{CEDURB1978}; sendo assim, era comum que o soteropolitano, mesmo quando proprietário de sua casa, fosse mero ``foreiro'', ``rendeiro'' ou ``morador'' de terras de terceiros \cite[p.~139]{BRANDAO1980}. Embora houvesse proibições legais à construção de cortiços pelo menos desde a Postura nº 39 do Código de Posturas Municipais de 1921 \cite{PREFEITURA1921}, somente em 1944, no contexto da atuação do EPUCS foi promulgada a primeira \textit{lei anti-cortiço} soteropolitana (Decreto-lei 347, de 06 out. 1944).

Salvador, com 72,21 prédios por $km^{2}$ em 1920, era uma das capitais com maior densidade de prédios por metros quadrado, ultrapassada apenas por Fortaleza (318,09 prédios/$km^{2}$), Recife (231,25 prédios/$km^{2}$), Niterói (162,09 prédios//$km^{2}$) e São Paulo (82,16 prédios//$km^{2}$) \cite[p.~XV]{brasil_censo46_1920}.

% \begin{table}[!htp]
\IBGEtab{
\caption{Estatística predial e domiciliária de Salvador, 1872-1920}\label{tab:estpredomsal1872-1920}}
{
\begin{tabular}{ccc}
\hline 
\multirow{2}{*}{Categoria} & \multicolumn{2}{c|}{Censos} \\ 
\cline{2-3} 
 & 1872 & 1920 \\ 
\hline 
População & 129.109 & 283.422 \\ 
\hline 
Prédios & 18.460 & 39.717 \\ 
\hline 
Domicílios & 24.894 & 40.615 \\ 
\hline 
Densidade predial & 6,99 & 7,14 \\ 
\hline 
Densidade domiciliária & 5,19 & 6,98 \\ 
\hline 
\end{tabular} 
}
{\fonte{Elaboração do autor, com dados de \citeonline[p.~XVI]{brasil_censo421_1920}.}}
\end{table}

No que diz respeito à densidade domiciliar, vista a questão distrito a distrito através da \autoref{tab:domsaldist1-1920} e da \autoref{tab:domsaldist2-1920}, fica evidente quanto aos distritos urbanos que:

\begin{itemize}
 \item Brotas , tinha a sexta maior taxa de domicílios desocupados (4,09\%), a quarta maior entre os distritos urbanos.
 \item Conceição 	
 \item Mares
 \item Nazaré
 \item Paço
 \item Penha , tinha a terceira maior taxa de domicílios desocupados entre todos os distritos (5,08\%), a maior entre os distritos urbanos.
 \item Pilar
 \item Santana
 \item Santo Antônio era o distrito mais populoso da cidade, seguido por Vitória e Brotas, mas a diferença de população entre Santo Antônio e Brotas era maior que a soma das populações dos distritos de Pirajá e Santana. A enorme extensão de um distrito composto por uma zona rural e uma zona urbana, entretanto, conferia-lhe baixíssima densidade predial, na verdade a menor entre todos os distritos urbanos. Tinha o maior número de domicílios particulares desocupados, mas sua proporção no conjunto de domicílios do distrito era menor que a média soteropolitana.
 \item São Pedro
 \item Sé
 \item Vitória
\end{itemize} 	

Já quanto aos distritos rurais, os resultados do Recenseamento 1920 indicam que:
 
\begin{itemize}
 \item Cotegipe
 \item Itapuã
 \item Maré , tinha a segunda maior taxa de domicílios desocupados entre todos os distritos.
 \item Matoim
 \item Paripe , tinha a maior taxa de domicílios desocupados entre todos os distritos.
 \item Passé
 \item Pirajá, em que pese ser um distrito suburbano, era mais populoso que os distritos de Santana, Sé, Mares, Nazaré, Pilar, Paço e Conceição, mas tinha a segunda menor densidade domiciliar do município.
\end{itemize}

\begin{landscape}
\begin{table}[!htp]
\IBGEtab{
\caption{Estatística predial e domiciliar soteropolitana, por distrito, em 1920 (parte 1)}\label{tab:domsaldist1-1920}}
{
\begin{tiny}
\begin{tabular}{rrrrrrrrrrrrrr}
\hline
\multirow{3}{*}{Distrito} & \multicolumn{13}{c}{Domicílios}\\
\cline{2-14}
 & \multicolumn{2}{c|}{Particulares / residências} & \multicolumn{11}{c}{Coletivos} \\
\cline{2-14}
 & Ocupados & Desocup. & Asilos & Cadeias & Escolas & Fábricas& Fazendas e outros & Hospitais & Hotéis & Pensões ou & Quartéis & Diversos & TOTAL \\
 & & & & & & ou oficinas & estabelecimentos & & & casas de & & & \\
 & & & & & & & agrícolas & & & cômodos & & & \\
\hline
\multicolumn{14}{c}{Urbanos} \\
\hline
Brotas	&3.427	&140	&1	&0	&0	&0	&0	&1	&0	&3	&2	&1	&8\\
Conceição	&428	&10	&0	&0	&0	&0	&0	&0	&0	&7	&1	&0	&8\\
Mares	&1.786	&53	&1	&1	&0	&0	&0	&0	&0	&0	&0	&0	&2\\
Nazaré	&1.587	&70	&1	&0	&4	&0	&0	&2	&0	&0	&1	&0	&8\\
Paço	&990	&7	&1	&0	&0	&0	&0	&0	&0	&1	&1	&0	&3\\
Penha	&2.798	&142	&1	&0	&1	&0	&0	&2	&0	&1	&1	&0	&6\\
Pilar	&1.336	&16	&0	&0	&1	&0	&0	&0	&0	&18	&3	&0	&22\\
Santana	&1.998	&65	&1	&0	&3	&0	&0	&0	&0	&16	&2	&0	&22\\
Santo Antônio	&8.721	&251	&2	&0	&3	&0	&0	&1	&0	&2	&5	&0	&13\\
São Pedro	&2.206	&94	&2	&0	&8	&0	&0	&1	&2	&23	&3	&0	&39\\
Sé	&1.914	&15	&1	&0	&1	&0	&0	&0	&2	&78	&3	&0	&85\\
Vitória	&5.964	&148	&1	&1	&3	&0	&2	&2	&0	&18	&5	&0	&32\\
\hline
\multicolumn{14}{c}{Rurais} \\
\hline
Cotegipe	&691	&1	&0	&0	&0	&0	&0	&0	&0	&2	&0	&0	&2\\
Itapuã	&520	&3	&0	&0	&0	&0	&4	&0	&0	&0	&1	&0	&5\\
Maré	&413	&23	&0	&0	&0	&0	&0	&0	&0	&0	&1	&0	&1\\
Matoim	&571	&6	&0	&0	&0	&0	&0	&0	&0	&0	&1	&0	&1\\
Paripe	&609	&36	&0	&0	&0	&0	&1	&0	&0	&0	&0	&0	&1\\
Passé	&1.497	&49	&0	&0	&0	&0	&0	&0	&0	&11	&0	&0	&11\\
Pirajá	&2.883	&91	&0	&0	&0	&0	&0	&0	&0	&4	&3	&0	&7\\
\hline
TOTAL	&40.339	&1.220	&12	&2	&24	&0	&7	&9	&4	&184	&33	&1	&276\\
\hline
\end{tabular} 
\end{tiny}
}
{\fonte{Elaboração do autor, com dados de \citeonline[p.~108-109]{brasil_censo46_1920}.}}
\end{table}
\end{landscape}
\begin{landscape}
\begin{table}[!htp]
\centering
\IBGEtab{
\caption{Estatística predial e domiciliar soteropolitana, por distrito, 1920 (parte 2)}\label{tab:domsaldist2-1920}}
{
\begin{tiny}
\begin{tabular}{rrrrrrrrrrrrrrr}
\hline
\multirow{3}{*}{Distrito} & \multicolumn{12}{c}{Outras aplicações} & \multirow{3}{*}{Pop.} & \multirow{3}{*}{Dens.} \\
\cline{2-13}
 & \multirow{2}{*}{Depósitos} & \multirow{2}{1cm}{Escolas} & \multirow{2}{*}{Escritórios} & \multirow{2}{*}{Estações} & \multirow{2}{*}{Fábricas} & \multirow{2}{*}{Casas} & \multicolumn{3}{c}{Repartições administrativas} & \multirow{2}{*}{Templos} & \multirow{2}{*}{Diversas} & \multirow{2}{*}{TOTAL} & & \\
\cline{8-10} & & & & & ou oficinas & de negócio & Federais & Estaduais & Municipais & & & & & \\
\hline
\multicolumn{15}{c}{Urbanos} \\
\hline
Brotas	&3	&1	&0	&0	&1	&58	&1	&0	&0	&2	&0	&66	&23.121	&6,73 \\
Conceição	&53	&2	&230	&0	&30	&336	&4	&1	&0	&1	&18	&675	&4.589	&10,53 \\
Mares	&5	&2	&1	&0	&18	&76	&0	&1	&0	&2	&0	&105	&14.272	&7,98 \\
Nazaré	&2	&5	&2	&0	&4	&123	&0	&0	&0	&1	&0	&137	&13.438	&8,43 \\
Paço	&5	&1	&0	&0	&9	&98	&0	&0	&0	&2	&0	&115	&7.074	&7,12 \\
Penha	&5	&5	&0	&0	&4	&102	&0	&0	&0	&4	&1	&121	&19.751	&7,04 \\
Pilar	&73	&2	&45	&0	&45	&121	&0	&1	&1	&1	&3	&292	&10.108	&7,44 \\
Santana	&3	&3	&0	&1	&1	&161	&0	&3	&1	&7	&3	&183	&15.739	&7,79 \\
Santo Antônio	&5	&4	&0	&0	&7	&169	&3	&0	&0	&8	&3	&199	&56.842	&6,51 \\
São Pedro	&11	&7	&9	&0	&7	&240	&4	&3	&1	&2	&7	&291	&18.666	&8,31 \\
Sé	&4	&11	&15	&1	&11	&275	&3	&4	&4	&9	&13	&350	&15.408	&7,71 \\
Vitória	&6	&4	&0	&1	&2	&133	&1	&1	&0	&9	&7	&164	&42.540	&7,09 \\
\hline
\multicolumn{15}{c}{Rurais} \\
\hline
Cotegipe	&0	&1	&0	&1	&1	&2	&0	&0	&0	&0	&0	&5	&4.263	&6,15 \\
Itapuã	&1	&2	&0	&0	&1	&11	&0	&0	&1	&1	&0	&17	&3.457	&6,58 \\
Maré	&9	&0	&0	&0	&0	&14	&1	&0	&0	&3	&0	&27	&2.729	&6,59 \\
Matoim	&0	&1	&0	&0	&0	&4	&0	&0	&0	&1	&0	&6	&3.186	&5,57 \\
Paripe	&0	&3	&0	&2	&5	&10	&0	&0	&0	&1	&0	&21	&4.135	&6,78 \\
Passé	&0	&0	&0	&0	&1	&9	&0	&0	&0	&1	&0	&11	&8.029	&5,32 \\
Pirajá	&2	&6	&2	&0	&12	&41	&2	&1	&1	&6	&3	&76	&16.075	&5,56 \\
\hline
TOTAL	&187	&60	&304	&6	&159	&1.983	&19	&15	&9	&61	&58	&2.861	&283.422	&6,98 \\
\hline
\end{tabular} 
\end{tiny}
}
{\fonte{Elaboração do autor, com dados de \citeonline[p.~108-109]{brasil_censo46_1920}.}}
\end{table}
\end{landscape}	

\subsection{Intervenções no espaço urbano e seus agentes}\label{subsec:1.4.3}

\begin{figure}[!htp]
\centering
\includegraphics[width=1\textwidth]{2-cap1/complementos/mapas/1894moralesdelosrios.eps} 
\caption{``Planta da cidade de São Salvador, capital do Estado federado da Bahia'', elaborada em 1894 por Afonso Morales de Los Rios, arquiteto-engenheiro participante, em 1893, de concurso para realização de obras de esgotamento em Salvador. Conforme observação da profª Odete Dourado, quando vista mais de perto, a planta mostra representação equivocada da localização do Largo do Teatro (atual Praça Castro Alves). \textbf{Fonte:} \citeonline{morales_planta_1894}.}
\end{figure}

\subsubsection{Plano de saneamento de Theodoro Sampaio (1904)}\label{subsubsec:1.4.3.1}

Como decorrência da encampação pela Intendência Municipal dos serviços, sempre precários, da Companhia do Queimado, foi aberta em 8 de novembro de 1904 concorrência para um plano de melhoramentos no setor, vencida pelo engenheiro baiano Theodoro Sampaio, único a apresentar proposta \cite[150]{gordilhobarbosa_eau_2004}. O trabalho foi financiado por um empréstimo feito pelo município de Salvador junto ao Banque de l'Union Parisienne no valor de 25 milhões de francos \cite[p.~150]{gordilhobarbosa_eau_2004}, e marcou o início de um processo de intervenção planificada e efetiva do poder público sobre o setor de saneamento de Salvador \cite[p.~150]{gordilhobarbosa_eau_2004} 

Os trabalhos foram parcialmente inaugurados em 1907, e abrangiam reformas e prolongamentos da rede de distribuição, assim como os primeiros 27km de redes de esgoto da cidade \cite[p.~151]{gordilhobarbosa_eau_2004}. O trabalho foi interrompido, segundo o próprio Theodoro Sampaio, porque o intendente municipal da época (Antônio Victor de Araújo Falcão) interrompeu o pagamento e o fornecimento de materiais, paralisando os trabalhos previstos, e passou a contratar terceiros para a canalização provisória dos esgotos da cidade \cite[p.~152]{gordilhobarbosa_eau_2004}.

\subsubsection{Plano de melhoramentos de Alencar Lima (1910)}\label{subsubsec:1.4.3.2}

Quando apresentou seu ``Plano geral de melhoramentos em parte da cidade do Salvador'' à Intendência Municipal de Salvador, em 1910, o engenheiro Jerônimo Teixeira de Alencar Lima já era um dos sócios da Estrada de Ferro Central da Bahia, junto com Austricliano Honório de Carvalho \cite{souza_trabalholivre_2011}, e comprara o Lloyd Brasileiro em 1903, para depois vendê-lo ao Governo da Bahia e arrendá-lo novamente \cite[p.~220]{CUNHA2011}; não era, por assim dizer, um neófito no ramo da prestação de serviços públicos, tampouco desconhecido no cenário político baiano.

Em resumo apertado, o projeto de Alencar Lima consistia em:

\begin{itemize}
\item Abertura da avenida Sete de Setembro;
\item Alargamento de ruas, em especial da avenida Carlos Gomes;
\item Construção de novas casas seguindo os critérios de higiene, arte e incombustibilidade;
\item Ajardinamento de praças e arborização de ruas;
\item Recomendação ao uso preferencial do concreto armado nas novas construções;
\end{itemize}

O projeto visava os distritos da Vitória e de São Pedro, escolhidos explicitamente em virtude de serem zonas ``capazes de compensar-lhes o capital empregado'' \cite[p.~95]{CUNHA2011}. 

\subsubsection{O porto (1906-1912)}\label{subsubsec:1.4.3.3}



\subsubsection{As reformas do Centro (1912-1916)}\label{subsubsec:1.4.3.4}

