\section{A política baiana (e soteropolitana) sob constante agitação}\label{sec:1.3}

Assim desenhada a situação econômica e social baiana, não parece ser difícil entender por que a política baiana viveu tempos tumultuados durante a Primeira República. A participação marginal da economia baiana na produção agroexportadora; a concentração da produção voltada à exportação em duas regiões do Estado e a debilidade econômica das demais regiões; a insignificância da malha rodoviária, a deterioração da rede fluvial de transportes e a insuficiência da malha ferroviária para interligar mais intensamente as diversas regiões do Estado entre si, com sua capital e principal porto e também com cidades outras no interior do Brasil que em tempos passados haviam sido praça para as mercadorias baianas; a dependência imposta aos latifundiários pela burguesia e pelos gestores atuantes no comércio; a extrema concentração de capitais e o pauperismo generalizado; tudo isto cria as condições para que seja o Estado a principal fonte de acesso a recursos para investimentos, e faz do acesso aos postos governamentais a condição para a sustentação do prestígio político de chefes políticos locais em franca decadência econômica. A simples construção de um quadro econômico, entretanto, não é suficiente para entender as forças motrizes da política baiana, nem consegue explicar, por si só, a ação política das classes sociais voltadas à superação da situação desoladora já descrita e redescrita; é preciso ver, na esfera propriamente política, como se movimentaram as classes sociais para superar este quadro, e como sua ação política, direta ou indiretamente, teria dado ainda outros elementos para aprofundar o declínio econômico a que gerações de políticos e economistas chamaram de ``enigma baiano'' \cite{aguiar_notas_1958}.

Resistente à república como via de regra o foram políticos das demais províncias ditas ``nortistas'', cedo --- cedo mesmo, em menos de uma semana --- os políticos baianos saíram da oposição ao regime à sua total integração \cite{sampaio_legislativo_1985}. Daí em diante, a política baiana durante a Primeira República, para facilitar o entendimento, costuma ser periodizada conforme os grupos a hegemonizá-la em cada momento, tendo em vista que os partidos políticos que então se formavam eram efêmeros, existiam apenas nas vésperas das eleições, e durante todo o tempo restante o fator aglutinador na política era não a organização partidiária, mas a fidelidade a chefes políticos e suas alianças ou cisões, tornando-se assim as organizações partidárias verdadeiras claques \cite[p.~18]{sampaio_partidos_1978}. Assim, há um período de \textit{consolidação da República} (1893-1896) onde os blocos de poder não estavam claramente configurados e o conflito intergrupal era a regra. Seguiu-se um período dominado, sucessivamente, pelos partidários de \textit{Luís Viana}, contestados pelos sequazes de \textit{José Gonçalves} (1896-1900); pelos partidários de \textit{Severino Vieira}, contestados pelos gonçalvistas e pelos vianistas (1900-1904); pelos partidários de \textit{José Marcelino de Sousa}, a princípio em aliança com os severinistas (1904-1907) mas pouco depois em oposição desabrida (1907-1912). A intervenção federal de janeiro de 1912, parte da política das salvações do presidente Hermes da Fonseca, marca o início da longa e turbulenta hegemonia de \textit{José Joaquim Seabra} e seus partidários (1912-1924), seguida, encerrando o período da Primeira República, pela hegemonia do grupo capitaneado por \textit{Francisco Goes Calmon} (1924-1930). 

Ressalte-se, para evitar confusões, que \textit{hegemonia} não significa, necessariamente, \textit{posse do chefe político no cargo de governador}, mas sim a capacidade de um grupo político de manter-se no poder por meio do consenso e da coerção; a constituição baiana impedia a recondução do governador ao cargo sem o intervalo mínimo de um mandato, o que impedia as reeleições e impunha alguma rotatividade no cargo máximo do Executivo baiano entre integrantes do mesmo bloco político. Seabra, por exemplo, foi governador duas vezes (1912-1916 e 1920-1924), mas continuou hegemônico na política estadual durante o mandato de Antônio Moniz Sodré de Aragão (1916-1920), seu correligionário. De igual maneira, pode-se dizer que os mecanismos de engenharia política vigentes na esfera federal, consolidados na ``política dos Estados'', foram adaptados às instituições políticas baianas \cite{sampaio_legislativo_1985}, resultando igualmente numa circularidade de favores e poderes: as eleições para o Executivo e o Legislativo estaduais eram controladas nos municípios pelos coroneis, cuja fidelidade política ao grupo hegemônico do momento assegurava a eleição dos candidatos deste grupo, fidelidade esta recompensada pelo controle sobre os cargos municipais conferido pelo grupo político hegemônico de cada momento.

Interessa a esta pesquisa, como já dito, muito menos a simples sucessão das personagens a ocupar o Palácio Rio Branco que as turbulências da política baiana da Primeira República, ou, melhor dizendo, as \textit{forças motrizes} destas turbulências. Não obstante, levando-se em conta o fato de os conflitos políticos na Bahia da Primeira República serem pouco conhecidos fora de um público especializado, não será possível fazer uma análise estritamente sociológico-categorial do processo político baiano do período, que deverá entremear-se com uma narrativa histórica simplificada dos fatos ocorridos. Por outro lado, o quadro da ``política dos Estados'' em nível federal impunha aos grupos hegemônicos na Bahia a cada momento alinhar-se aos sucessivos titulares do Executivo federal, sob pena de ostracismo político; interessa saber, portanto, além das forças motrizes das turbulências políticas baianas, quando houve alinhamento do governo baiano ao governo federal nos diferentes períodos em que se costuma categorizar a política baiana durante a Primeira República, na medida em que o alinhamento ao grupo em exercício no governo federal facilitava o acesso aos recursos federais.

\subsection{Parâmetros de análise}

Há três obras clássicas para estudar a política baiana neste período. A primeira, mais antiga e frequentemente reeditada \cite{TAVARES2008}, justamente por seu caráter didático e introdutório pouco avança além do elenco de governadores e de algumas palavras sobre a economia baiana do período, inseridos, o elenco e a breve notícia econômica, no quadro da longa duração. A segunda, de escopo amplo \cite{pang_coronelismo_1979}, peca por anacronismo: pretende explicar as disputas políticas baianas por meio de uma complexa tipologia das oligarquias e dos coroneis e pelo emprego francamente anacrônico das estruturas de \textit{clã} e de \textit{tribo}, centradas no \textit{paterfamilias}, como sustentáculo da política baiana (e brasileira por extensão, dado que a Bahia foi escolhida por ser ``caso de estudo do coronelismo''). A terceira, mais completista e detalhista \cite{sampaio_partidos_1978}, peca entretanto por circularidade e anacronismo: \textit{circularidade}, porque nesta obra a fragmentação política baiana do período teria como causa\dots a própria fragmentação econômica baiana; \textit{anacronismo}, porque projeta ao passado uma estrutura jurídico-legal ordenadora do processo eleitoral que, se vigente durante a Primeira República, talvez estancasse os conflitos interpartidários -- como se não existissem então, além de arcabouço legal próprio regulamentador das eleições, mecanismos políticos de regulação tanto das eleições, quanto da competição política, eficazes para aquilo a que seus criadores se propunham (o controle das eleições, a hegemonia política e a redução das incertezas sucessórias). Fundamentais antes como agora pela riqueza das informações recolhidas, estes três clássicos não conseguiram, entretanto, romper a superfície das disputas internas aos sucessivos blocos de poder e entender que forças levavam a tais disputas. 

Obras mais recentes alargaram o foco e mudaram o ângulo de análise; duas destacam-se do conjunto, pela amplitude de seus respectivos objetos e pela complementariedade entre eles. A primeira entre elas \cite{castellucci_maquina_2008} trata da política baiana não mais pelo ponto de vista dos latifundiários, burgueses e gestores, de suas intrigas palacianas, dos conflitos às vezes armados entre eles, mas pelo ponto de vista dos \textit{trabalhadores} e de sua difícil \textit{articulação política} durante a Primeira República; complementa muito positivamente neste aspecto a análise das condições de vida na Bahia, com especial foco nas movimentações dos trabalhadores em torno da questão do ``custo de vida'', iniciada anos atrás por outro autor \cite{santos_repovo_2001}. A segunda \cite{CUNHA2011} tenta enriquecer a análise das disputas de poder em meio às classes dominantes ultrapassando o âmbito das disputas na imprensa e nas tribunas parlamentares para ir mais a fundo, em meio aos arquivos privados, arquivos empresariais e correspondências pessoais sem entretanto rejeitar a pesquisa em arquivos públicos e na imprensa de época, e encontrar nas disputas entre empresas pela produção e gestão de infraestruturas urbanas uma das forças motrizes dos conflitos políticos da Primeira República.

As informações constantes nestas obras clássicas e recentes permitem alinhavar numa só narrativa os conflitos de classe esboçados na \autoref{sec:sobasotconv} (p. \pageref{sec:sobasotconv}). Para isto, em primeiro lugar, é preciso dialogar com a tese das ``oligarquias regionais'' \cite{pang_coronelismo_1979,sampaio_partidos_1978,TAVARES2008}, recebendo dela a radicação geográfica de famílias de destaque na política baiana\footnote{``Os clãs políticos dominantes demarcaram duas áreas de influência ao longo de limites \textbf{geoeconômicos}, e, dentro de cada zona, uma ou mais famílias surgiu como oligarquia municipal'' \cite[p.~76, \textbf{grifo nosso}]{pang_coronelismo_1979}.}; se o foco for transferido das ``oligarquias regionais'' para o funcionamento da economia baiana estruturado de modo geográfico, em especial numa hierarquização de centros urbanos regionais ainda que precária, e sendo as classes sociais radicadas nos diferentes estratos da rede urbana simultaneamente tributárias e construtoras desta estrutura hierarquizada por meio de sua atuação econômica, o arrolamento de patronímicos ilustres fará mais sentido. A análise da rede urbana baiana terá como base obras clássicas sobre o tema \cite{geiger_rede_1963,SANTOS1959}, e o cruzamento com a inserção econômica e geográfica dos coroneis será feito por meio de um dos mapas do \textbf{Atlas Histórico do Brasil} (cf. \autoref{2016-coroneisfgv}, na p. \pageref{2016-coroneisfgv}), produzido pela Fundação Getúlio Vargas (FGV) \cite{fgv_coroneis_2016} com base no anexo IV (``Cinquenta dos coroneis mais 'ativos' da Bahia, 1889-1937'') do livro de \citeonline{pang_coronelismo_1979}, incorporando inclusive alguns de seus erros mais flagrantes\footnote{Um exemplo: \textit{Abílio Rodrigues de Araújo}, chefe político de Rio Preto, no referido anexo de \citeonline[pp.~246-249]{pang_coronelismo_1979} é creditado como sendo de Feira de Santana.}. Para evitar repetições desnecessárias e a saturação visual da mancha gráfica desta dissertação, a referência a estas obras será feita apenas quando de citações diretas.

\begin{figure}[!htp]
\centering
\includegraphics[width=1\textwidth]{2-cap1/complementos/mapas/2016-coroneisfgv.eps}{\footnotesize \par \textbf{Fonte:} \citeonline{fgv_coroneis_2016}.}
\caption{Distribuição geográfica dos principais coroneis da Bahia (1889-1937).}\label{2016-coroneisfgv}
\end{figure}

Salvador e o \textit{Recôncavo} eram áreas de economia hegemonizada pelas plantagens de açúcar, com algumas culturas complementares. Salvador, centro macrocéfalo de vasta hinterlândia, obedecia, decerto, a outras dinâmicas; era o Recôncavo e sua sucrocultura decadente, entretanto, quem lhe dava régua e compasso, apesar de serem dependentes do mercado e do porto soteropolitano cidades como Cachoeira, São Félix, Santo Amaro e Nazaré. Feira de Santana, durante a Primeira República, entrava no Recôncavo de Salvador como seu limite extremo rumo ao sertão. Desta região açucareira vieram os \textit{Araújo Pinho} (de Santo Amaro), os \textit{Meireles} (de Mata de São João), os \textit{Calmon}, os \textit{Mangabeira}, os \textit{Prisco Paraíso} (de Cachoeira), os \textit{Costa Pinto} (de Santo Amaro), os \textit{Tosta} (de São Félix), os \textit{Sousa} (de Nazaré), os \textit{Bulcão} (de São Francisco do Conde), os \textit{Moniz}, os \textit{Aragão} e os \textit{Vilas Boas}, famílias de onde vieram figuras de proa da política baiana durante a Primeira República. Foram ainda os reconcavinos liberais e conservadores do Império, ao se conciliar, quem fundou o \textit{Partido Republicano da Bahia} (PRB), hegemônico na política baiana até 1912.

A \textit{zona cacaueira}, de economia voltada para a exportação de cacau, era encabeçada por Ilhéus e Itabuna, onde também se radicam Canavieiras e Itapetinga. Aí se radicaram \textit{Misael Tavares}, \textit{Pedro Levino Catalão}, \textit{Domingos Adami de Sá} e \textit{Oscar Falcão}. Nas \textit{Lavras Diamantinas}, onde a mineração fazia e desfazia fortunas, destacavam-se do Centro-Sul, tendo Lençóis como centro econômico e Hegemonizava a política lençoense o coronel \textit{Felisberto Augusto de Sá}. Eram também centros importantes: Mucugê (terra dos \textit{Rocha Medrado} e do coronel \textit{Douca}); Andaraí (terra de \textit{Aureliano Brito de Gondim}); Brotas (terra de \textit{Militâo Rodrigues Coelho}); Macaúbas (terra do monsenhor \textit{Hermelino Marques de Leão}); Morro do Chapéu (terra de \textit{Francisco Dias Coelho}); Campestre, atual Seabra (terra de \textit{Manuel Fabrício de Oliveira}). Na região pecuarista do \textit{Nordeste baiano}, vocacionada para o mercado interno,  surgiram os \textit{Dantas Bião} (Alagoinhas), os \textit{Dantas} (Jeremoabo e Itapicuru) e \textit{José Gonçalves} (Bonfim). O \textit{Centro-Sul}, onde na Primeira República a economia centrava-se na pecuária e na plantagem de café e cacau, encontram-se Jequié, Jaguaquara e Vitória da Conquista, então muito menor que a primeira. 

O extenso \textit{vale do São Francisco}, divisor do sertão baiano, tinha vários polos regionais, bocas de entrada para o sertão a partir da navegação ao longo do rio homônimo. Centros como Juazeiro, centro de armazendamento e polo comercial sanfranciscano e terra de coroneis comerciantes como \textit{José Alves Pereira}, \textit{João Evangelista Pereira e Melo} e \textit{Aprígio Duarte Filho}, mas tinha como outros centros de destaque: Casa Nova (cidade de origem dos \textit{Viana} e dos \textit{Castro}); Sento Sé (território da família homônima e também dos \textit{Sousa}); Pilão Arcado (terra dos \textit{Albuquerque}) e Remanso (terra dos \textit{Castelo Branco} e dos \textit{França Antunes}); Barra (terra dos \textit{Wanderley} e dos \textit{Mariani}). O oeste do vale, região por muito tempo conhecida como ``além São Francisco'', era região extensa e pouquíssimo povoada, e a precariedade das ligações com a capital certamente acentuava a dependência da população local frente aos coroneis. Suas principais cidades: Barreiras (terra de \textit{Antônio Balbino de Carvalho}, \textit{Abílio Wolney} e dos \textit{Rocha}); Correntina (terra de \textit{Félix Joaquim de Araújo}); Santa Maria da Vitória (terra de \textit{Clemente Araújo de Castro}); Santana dos Brejos (terra de \textit{Francisco Joaquim Flores}); Rio Preto, atual Santa Rita de Cássia (terra de \textit{Abílio Rodrigues de Araújo}). Carinhanha emergia no cenário por sua particularidade geográfica: fronteiriça entre Bahia e Minas, servia de entreposto para o comércio destes dois estados e também para o de Goiás, resultando em que a política local, hegemonizada pelos \textit{Duque} e pelos \textit{Alkmin}, vivesse pendendo ora para o lado baiano, ora para o lado mineiro.

Seria aplicável à política estadual o mesmo modelo discutido na \autoref{subsec:espadaleite} (p. \pageref{subsec:espadaleite})? Ou seja, seria a dinâmica da política baiana movida por um conflito entre classes sociais distintas e entre frações da mesma classe, confundidas pela aparência de conflitos entre ``oligarquias regionais''? Ou, ao contrário, teria vigido na política baiana a simples oposição ``litoral'' e ``interior'', ainda hoje reproduzida nos mesmos termos estabelecidos há mais de século por Euclides da Cunha n'\textbf{Os Sertões}? Por outro lado, seria a capacidade de apaziguar conflitos internos determinante para estabelecimento de boas relações com o governo federal?

\subsection{Consolidação republicana, vianismo, severinismo, marcelinismo (1889-1912)}

Os conflitos ocorridos na fase de consolidação da república na Bahia (1889-1896) não permitem discernir um grupo hegemônico.

Durante a hegemonia vianista (1896-1900), talvez o conflito principal, além do massacre a Canudos (Viana era acusado por seus adversários de ``complacência'' com a comunidade sertaneja) e das expedições policiais promovidas contra os inimigos políticos de Luiz Vianna em Ilhéus, Belmonte, Canavieiras, no Nordeste baiano e nas Lavras Diamantinas, talvez tenha sido a eleição municipal soteropolitana de 1899, em que a polícia investiu contra comerciantes, fato determinante do declínio político de Luiz Vianna, anatematizado pela burguesia comercial baiana até o fim de sua carreira política. 

Durante a hegemonia severinista (1900-1904) foi fundado o \textit{Partido Republicano da Bahia} (PRB), tentativa de aglutinar numa só organização política representantes de regiões distintas do Estado mas muito solidamente fundado em meio aos latifundiários do Recôncavo baiano e aos burgueses comerciantes, frustrada ela enfim pela política vingancista do governador contra seus antigos desafetos interioranos. Severino Vieira foi Ministro da Viação e Obras Públicas durante a presidência de Campos Sales (1898-1900), sucedido pelo paulista, aliás nascido carioca, Alfredo Eugênio de Almeida Maia; sua indicação como candidato oficial ao governo baiano servia tanto como consolidação da ``política dos Estados'' como forma de afastar Severino do governo federal, onde já se desentendia com o Ministro da Fazenda, Joaquim Murtinho -- desentendimento aliás compartilhado com a burguesia comercial baiana, insatisfeita com as políticas daquele a quem chamavam de ``financista homeopata''. Ao ascender ao governo da Bahia, entretanto, Severino viu nomeado para o Ministério da Justiça do governo Rodrigues Alves (1902-1906) seu adversário José Joaquim Seabra, nomeação interpretada como expressão de desagrado do presidente recém-eleito com o grupo hegemônico na Bahia. 

Durante a hegemonia marcelinista (1904-1912) reforçou-se a hegemonia açucareira -- o próprio José Marcelino era proprietário da Usina Conceição, em Nazaré -- e as medidas pró-agricultura do governo estadual (manutenção de Miguel Calmon na Secretaria de Agricultura; estudos de aperfeiçoamento da lavoura cacaueira por agrônomos alemães; estudos de implantação da cotonicultura no sertão baiano; investimentos no transporte fluvial etc.) eram muito bem-vistas pelos latifundiários baianos. A estruturação do \textit{Banco de Crédito da Lavoura da Bahia} pelo governador, tentativa de aumentar a disponibilidade de crédito na praça baiana, serviu também para romper a dependência dos comerciantes em que os latifundiários se viam envolvidos e fortaleceu ainda mais a hegemonia açucareira: o presidente do banco era João Ferreira de Araújo Pinho, promotor santamarense membro de antiga família açucareira; o diretor do banco, Henrique Teixeira, tinha trânsito fácil entre os comerciantes exportadores; e o secretário do banco, Viriato Ferreira Maia Bittencourt, era ligado à família Calmon, também do ramo sucrocultor. José Marcelino de Sousa, além de apaziguar os latifundiários interioranos em seu favor com medidas de estímulo à economia local, disponibilização de crédito barato para a agricultura e instalação ou renovação de infraestruturas de transporte, viu seu secretário de agricultura, o engenheiro Miguel Calmon, ser cooptado pelo novo presidente Afonso Pena (1906-1909) para o Ministério da Viação e Obras Públicas. Calmon foi sucedido neste ministério em 1909 pelo mineiro, aliás nascido cearense, Francisco Sá. José Marcelino envolveu-se também numa das mais famosas negociatas da Primeira República na Bahia: a \textit{Companhia Baiana de Navegação}, objeto de uma disputa entre comerciantes baianos e o Lloyd Brasileiro que terminou por comprá-la em 1903, foi estatizada em 1905 para ser imediatamente concedida ao especulador José Teixeira de Alencar Lima, numa negociata que marcou época \cite[p.~220]{CUNHA2011}. Por outro lado, no vale sanfranciscano a Viação do São Francisco, igualmente estatizada por José Marcelino em 1905, adquiriu novos barcos e desobstruiu o leito do rio Preto (para facilitar a passagem das embarcações), o que permitiu estender a navegação baiana no São Francisco até o porto de São Marcelo, na foz do rio Sapão, estendendo portanto até Goiás, Piauí e Maranhão a influência do comércio baiano (e da burguesia e dos gestores que nele se estribavam), que por meio do São Francisco já se estendia até Pernambuco e Minas Gerais \cite[p.~220-221]{CUNHA2011}.

Foi durante a hegemonia marcelinista que surgiu um novo tipo de político, iniciamente apelidados de ``jovens turcos'', bachareis oriundos de famílias latifundiárias. Os irmãos Miguel e Antônio Calmon, o promotor Araújo Pinho, o jornalista Ernesto Simões Filho, o advogado e jornalista Pedro Lago, o advogado João Mangabeira, o jurista e jornalista Antônio Moniz e o engenheiro Otávio Mangabeira, estes e outros tantos, ao invés de representarem uma ``mistura de classes'' \cite[p.~93]{pang_coronelismo_1979}, são elementos que, por meio da educação, da inserção profissional e em especial da atuação política, \textit{transitaram de uma classe à outra}, nomeadamente da classe latifundiária à classe dos gestores.

Foi também durante a hegemonia marcelinista que teve início a verdadeira guerra em torno da produção e distribuição de energia elétrica e do transporte público em Salvador, envolvendo pesos pesados das burguesias brasileira e internacional. De um lado colocava-se o grupo \textit{Guinle \& Cia}\footnote{A bibliografia consultada, em especial \citeonline[pp.~42-44]{CUNHA2011}, diz que o grupo Guinle tem origem numa casa carioca de importação e comercialização de tecidos, a \textit{Gaffrée \& Guinle}, aberta em 1871 por Eduardo Palassin Guinle e Cândido Gaffrée. Cedo passaram os dois a atuar também na construção de estradas de ferro no Nordeste brasileiro por meio de subempreitadas onde adquiriram experiência; já haviam coordenado a construção de 1.500km de estradas de ferro em 1888, quando obtiveram concessão, junto com Francisco de Paula Ribeiro, para construir o novo porto de Santos. Das ferrovias e portos, Gaffrée \& Guinle começou a trabalhar com energia elétrica, obtendo em 1899 no Rio de Janeiro a concessão para explorar a queda d'água do rio Paquequer, continuamente postergada pelas enormes demandas da construção do novo porto de Santos até que, em 1901, a demanda energética do próprio porto levou a empresa a conseguir autorização para implementação de uma hidrelétrica no rio Itatinga (até hoje existente e funcional, com produção de 15MW, o que a classifica, pelos atuais critérios da resolução ANEEL nº 652, como uma \textit{Pequena Central Hidrelétrica (PCH)}). Em 1903 a segunda geração dos Guinle fundou com o engenheiro estadunidense Adolf Aschoff a empresa \textit{Aschoff \& Guinle}; a morte do sócio em 1904 levou à mudança da razão social da firma para \textit{Guinle \& Cia.}. Daí em diante, em especial depois da aquisição da \textit{James Mitchel \& Cia.} sediada no Rio de Janeiro e em São Paulo, o grupo Guinle alargou enormemente os negócios da família, abarcando a representação de vários fabricantes internacionais de bens de consumo durável: \textit{General Electric Co.} (conglomerado estadunidense do setor de eletricidade e maquinário elétrico que veio a se tornar parceiro estratégico do grupo), \textit{Pelton Waterwheel Co.} (rodas d'água, turbinas), \textit{McIntosh \& Seymour Co.} (máquinas a vapor), \textit{Babcock \& Wilcox Co.} (caldeiras), \textit{The Pekan Manufacturing Co.} (eixos para carros e vagões), além de fabricantes de máquinas de escrever, fotografia, gramofones etc. Em junho de 1909, o grupo fundou a \textit{Companhia Brasileira de Energia Elétrica}, por meio da qual unificou seus negócios de geração e distribuição de energia elétrica em vários Estados.}, que adquiriu as empresas \textit{Companhia Linha Circular de Carris da Bahia} (em 1906) e \textit{Companhia Trilhos Centrais} (em 1907). Como o grupo Guinle decidira na mesma época também operar no mercado de energia elétrica para eletrificar os bondes de suas linhas, entrechocou-se com a empresa \textit{Bahia Tramway, Light \& Power}, por meio da qual o especulador estadunidense \textit{Percival Farquhar} representava o grupo \textit{Light}\footnote{Uma ``genealogia acionária'' deste grupo, feita a partir da bibliografia consultada (em especial \citeonline[pp.~34-40]{CUNHA2011}), mostra a complexidade do processo de exportação de capitais característico do imperialismo nas primeiras décadas do século XX. Em 1899 os canadenses \textit{Alexander Mackenzie}, empreiteiro ferroviário, e \textit{Frederick Stark Pearson}, empresário e engenheiro elétrico, fundaram com outros sócios a \textit{São Paulo Tramway, Light and Power}, que por quase século explorou a produção e distribuição de energia elétrica no Estado até sua compra pela Eletropaulo em 1981. Já em 1901 estes senhores compraram todas as ações da \textit{Companhia Água e Luz de São Paulo} para fundi-la à \textit{Light} paulista, unificando assim a produção e distribuição de energia elétrica com a prestação de serviço de transporte público na cidade de São Paulo. Em 1904, a partir desta experiência, Mackenzie e Pearson compraram todos os ativos da \textit{Companhia Nacional de Eletricidade}, que produzia e fornecia energia elétrica no Rio de Janeiro, e fundaram em Toronto a \textit{The Rio de Janeiro Tramway, Light and Power Co. Ltd.}, cuja sucessora, a \textit{Light S. A.}, ainda hoje -- sob controle da mineira CEMIG -- fornece energia ao Rio de Janeiro e outros trinta municípios fluminenses. Estes e outros negócios foram agrupados por Mackenzie e Pearson em 1912 numa só controladora, a \textit{Brazilian Traction, Light and Power}, também sediada em Toronto. Percival Farquhar foi parceiro e sócio de Mackenzie e Pearson em muitas destas empreitadas.}, que atuava em Salvador desde 1906 a partir da fusão de dois polos distintos: de um lado, como consequência de acordos firmados na Cidade do México (1903) e no Rio de Janeiro (1904), pela compra dos negócios sul-americanos da empresa alemã \textit{Siemens \& Halske Aktien Gesellschaft}, compradora esta última em 1898 das empresas baianas \textit{Veículos Econômicos} e \textit{Carris Eléctricos da Bahia}, o que tornou o grupo \textit{Light} uma das peças do oligopólio do transporte público soteropolitano; e de outro pela compra da empresa inglesa \textit{Bahia Gaz and Electric Company} e da empresa belga \textit{Compagnie d’Éclairage da Bahia}, detentoras (desde 1857 a primeira, desde 1901 a segunda \cite[pp.~86-87]{vieira_regula_2014}) das concessões para produção e distribuição de energia elétrica em Salvador; estes dois polos foram agrupados pelo grupo \textit{Light} em 1906 numa só empresa, a \textit{Bahia Tramway, Light \& Power} \cite[pp.~34-36]{CUNHA2011}. 

A disputa entre os grupos \textit{Guinle} e \textit{Light} agravava-se pelo fato de Guilherme Guinle residir em Salvador desde 1907 e entremear-se nos meios políticos soteropolitanos, mantendo como sócios minoritários nas empresas adquiridas membros de famílias poderosas e influentes como Góes Calmon, Egas Muniz, Castro Rabelo, Machado, Brandão e Carneiro; Guilherme fora também colega de faculdade do ministro Miguel Calmon \cite[p.~50]{CUNHA2011}. Tamanho arsenal político, ao mesmo tempo em que baldava os esforços do grupo \textit{Light} para permanecer na praça soteropolitana, ainda que o grupo americano-canadense contasse em sua folha de pagamento com ninguém menos que o próprio Ruy Barbosa, a um soldo que variava entre 47:000\$000 e 60:000\$000 ao ano \cite[pp.~56-58]{CUNHA2011}\footnote{O notável político baiano, advogado por formação e militância profissional, foi parecerista da \textit{Light} entre 1905 e 1922, emitindo pareceres em favor deste grupo em questões decisivas como a da ``concessão Reid'', que garantiu à \textit{Light} o monopólio da geração e fornecimento de energia elétrica no Rio de Janeiro; a relação com a \textit{Light} dava a Ruy Barbosa uma fonte fixa de trabalho e um meio para a prática e obtenção de favores para sua clientela política e familiar \cite[pp.~56-57]{CUNHA2011}.}. As disputas entre os grupos \textit{Light} e Guinle terminaram por ligar às disputas políticas baianas os destinos das infraestruturas elétrica e de transporte público -- duas importantes condições gerais de produção e reprodução da força de trabalho, e de também de operacionalidade das unidades de produção e do mercado (cf. \autoref{subsec:cgpcsjobe}, p. \pageref{subsec:cgpcsjobe}).

Ao final do governo de José Marcelino, a escolha do candidato oficial à sucessão governamental em 1908 mostra como a oferta de crédito fácil para os latifundiários foi um dos elementos determinantes da hegemonia marcelinista: Araújo Pinho, o indicado pelo governador, recebeu quase imediatamente apoio irrestrito de Dias Coelho (Morro do Chapéu) e de lideranças políticas de todo o vale sanfranciscano, pecuaristas e comerciantes dependentes da atuação do governo estadual para a criação, manutenção e gestão de condições gerais de produção (transporte, mercados etc.); por outro lado, em Ilhéus e Itabuna, vivendo então rápido desenvolvimento econômico por força das exportações cacaueiras, a disputa foi acirradíssima. Complementarmente, o apoio de Afonso Pena, Pinheiro Machado e Rui Barbosa à candidatura de Araújo Pinho mostra como a \textit{pax baiana} facilitara as relações entre o governo estadual e o governo federal; conta aí certamente a presença de Miguel Calmon no ministério, azeitando as relações. 

A dissolução da \textit{pax baiana} marcelinista emergiu da disputa entre Severino Vieira, eleito deputado ao deixar o governo, e José Marcelino; tal disputa, ao invés de ser creditada ao puro conflito de personalidades entre dois líderes carismáticos, deve ser vista, ao contrário, como conflito pelo acesso ao governo federal, sabidamente o meio mais fácil para acesso aos recursos necessários para o desenvolvimento das condições gerais de produção durante a Primeira República e, portanto, para a hegemonia política. Decerto o fator pessoal pesou, mas ele, por si só, explica pouco. Esfacelado o partido ainda antes da eleição governamental e fraturado o Estado quando da apuração dos resultados -- pois a campanha de Joaquim Inácio Tosta, apoiado por Severino Vieira, fora declarada vencedora em trinta municípios, contra oitenta onde a de Araújo Pinho fora declarada vencedora -- o mandato de Araújo Pinho à frente do governo da Bahia, à frente de um partido esfacelado entre severinistas e marcelinistas, foi incapaz de manter as boas relações com o governo federal; demonstra-o a incapacidade de apoiar a demanda dos cacauicultores de obter da presidência apoio à cacauicultura semelhante ao que se fizera ao café por meio do Convênio de Taubaté (1906). Por isto mesmo, por sua incapacidade de manter o mesmo ritmo de apoio à agricultura que seu antecessor e padrinho político, Araújo Pinho não conseguiu reunificar o PRB nem tampouco reconciliar as diferenças entre os latifundiários situacionistas e opositores, transformadas não raro em verdadeiras batalhas campais. Araújo Pinho fez a Companhia Baiana de Navegação retornar ao controle público e tentou privatizá-la novamente, sem sucesso; ainda em 1909 a Viação do São Francisco foi arrendada pelo governo estadual a capitalistas juazeirenses, beneficiados agora por toda a expansão havida durante a gestão de José Marcelino \cite[pp.~220-221]{CUNHA2011}.

\subsection{A turbulenta hegemonia seabrista (1912-1924)}

Em 1910 Araújo Pinho, José Marcelino e Severino Vieira, opositores ao governo federal, apoiaram ativamente a campanha civilista de Ruy Barbosa, enquanto Hermes da Fonseca foi apoiado por Luiz Vianna, José Gonçalves da Silva (então decadente) e, curiosamente, por um inimigo dos vianistas, José Joaquim Seabra. Vitorioso Hermes, tratou Seabra, mesmo frustrado em sua missão de cabo eleitoral hermista, de manter a aliança temporária com Luiz Vianna para isolar Severino Vieira, e com isto obteve em primeiro lugar o cargo de Ministro da Viação e Obras Públicas no governo federal; tal posto garantiu-lhe contatos preciosos com capitalistas internacionais e permitiu-lhe gerir demandas dos coroneis do interior, além de granjear-lhe a popularidade necessária à eleição ao governo em 1912. Tal eleição, entretanto, foi precedida por manobras políticas e conflitos que resultaram não apenas no bombardeio de Salvador em 10 de janeiro de 1912, mas igualmente na verdadeira guerra civil que durou entre 22 a 27 de janeiro e envolveu milhares de pessoas, barricadas, tiroteio entre a polícia e a turbamulta armada na capital etc. 

Teve início, com a eleição de Seabra em 1912, nova fase de alinhamento com o governo federal, com os sectários de Seabra agrupando-se no \textit{Partido Republicano Democrata} (PRD), fundado em 1910 por uma constelação de bachareis\footnote{Trata-se do primeiro partido de bases nitidamente urbanas na Bahia: entre seus fundadores de 1910 contavam-se 40 ``doutores'', 33 ``coroneis'', 3 ``cônegos'', 2 ``conselheiros'', 1 ``comendador'', 1 ``desembargador'', 1 ``capitão'', 1 ``farmacêutico'' e 2 ``não portadores de títulos'' \cite[p.~70]{sampaio_partidos_1978}. Comparativamente, o Conselho Geral do Partido Republicano da Bahia, instrumento político da hegemonia severinista e marcelinista, era composto em 1901 por 29 ``doutores'', 10 ``coroneis'', 3 ``batinas'',  4 ``nobres'', 2 ``comendadores'', 1 ``professor'' e 1 ``senador'' \cite[p.~49]{sampaio_partidos_1978}.} e que aglutinou inclusive seus antigos opositores no Senado federal. Opuseram-se a Seabra de início alguns latifundiários no Recôncavo, no Sul, no vale sanfranciscano e nas Lavras Diamantinas; Seabra optou por não intervir nas disputas locais entre os latifundiários, apoiando em seguida o vencedor (ainda que se tratasse de adversário seu) em busca da construção de bases sólidas em meio a esta classe social. Sua busca por uma governança unipartidária culminou com a promulgação da Lei Estadual 1.102, de 11 de agosto de 1915: uma reforma municipal que mudou a forma de acesso às intendências municipais (como então eram chamadas as prefeituras): antes eletivas, as 141 intendências baianas passaram a ser ocupadas por pessoas nomeadas pelo governador, no caso o próprio Seabra. Com tal mecanismo, o poder dos latifundiários encontrou-se de repente ameaçado, podados que foram de sua fonte de poder: o controle sobre as eleições municipais. Fez assim Seabra a base de poder com que construiu a eleição de seu sucessor, Antônio Ferrão Moniz de Aragão, e elegeu-se deputado federal para assim circular novamente nos meios políticos do Rio de Janeiro. 

Tratava-se de uma derrota para os latifundiários e uma vitória para os burgueses e gestores, base social da hegemonia seabrista; mas não se tratava de qualquer burguesia. Ocorre que Seabra já em 1912 se desentendera com Luís Vianna, aliado de ocasião em sua ascensão ao poder estadual, em torno do processo de reconhecimento de poderes nas eleições a deputado federal naquele ano, da subsequente escolha da liderança da bancada federal baiana (o senador Vianna apoiando Joaquim Pires Moniz de Carvalho contra a indicação de Mário Hermes da Fonseca, filho do presidente e eleito deputado federal pela Bahia com a força e o apoio de Seabra) e de um empréstimo externo ao Estado; Vianna tornou-se minoritário durante os seabrismo, angariando poucos apoios, como o de Simões Filho. Na esfera federal, afastara-se Seabra do porto seguro situacionista ao afrontar o senador gaúcho Pinheiro Machado, eminência parda do presidente Hermes da Fonseca; o enfrentamento entre Seabra e Vianna e o posterior expurgo deste último da seção baiana do \textit{Partido Republicano Conservador} (PRC) promovida por Seabra desagradara o sul-riograndense, e o apoio de Seabra desde as negociações precedentes às eleições de 1914 à candidatura presidencial de Ruy Barbosa -- um de seus mais tradicionais adversários -- selara seu destino: oposição ao governo federal.

Tudo isto afastava Seabra das possibilidades de angariar os recursos federais para as obras vultosas que prometera -- a serem vistas em maior detalhe na \autoref{subsec:1.4.3} (p. \pageref{subsec:1.4.3}) -- e aproximava-o a uma burguesia mais capitalizada, já presente na praça mercantil baiana mas desejosa de maiores possibilidades de atuação junto ao Estado: entre outros, tratava-se do grupo \textit{Guinle \& Cia.}, que pragmaticamente se aproximara de Seabra ainda quando de sua ascensão ao governo da Bahia. Os Guinle colheram os frutos de sua relação com Seabra, iniciada ainda na estadia deste último no Ministério da Viação e Obras Públicas. Ainda em junho de 1912 Eduardo Guinle fundou, com Francisco Marques de Góes Calmon e outros, a \textit{Companhia de Melhoramentos}, coincidentemente algumas semanas antes de Seabra anunciar ao público o amplo programa de reformas urbanas a que se propunha; esta empresa foi a responsável por realizar tais obras e por explorar economicamente vários imóveis circunvizinhos às areas afetadas, em negócios milionários. Não bastasse isto, os Guinle acuaram o grupo \textit{Light}, lançaram-no às cordas, resultando por fim na encampação da \textit{Light} pela intendência municipal em abril de 1913. Como dar seguimento a tão vultosos investimentos contando com capitais ainda mais vultosos que aqueles disponibilizados pelos Guinle? Por meio de \textit{empréstimos externos} vultuosíssimos acertados em especial com os bancos franceses \textit{Caisse Commerciale et Industriale de Paris} e \textit{Credit Mobilier Français}, que por intermédio de Seabra envolveram-se com as obras do porto de Salvador e desde 1911 haviam posto o mercado imobiliário e de obras públicas de Salvador em sua alça de mira (cf. \autoref{subsec:1.4.3}, p. \pageref{subsec:1.4.3}). 

A preferência dada por Seabra a investidores burgueses de fora da Bahia em negócios tão alentados decerto acunhou as relações entre ele e a burguesia comercial e financeira baiana; para piorar, pipocaram na imprensa oposicionista acusações, amiúde confirmadas, de malversação do dinheiro dos empréstimos contraídos para as reformas urbanas de Salvador; de envolvimento do intendente e protegido de Seabra, Júlio Viveiros Brandão, em todo o processo de malversação, o que resultou em 1914 na sua ruptura com os Guinle -- a quem servira antes da Intendência como gerente e acionista minoritário na Companhia Trilhos Centrais -- e com o próprio Seabra; tudo resultou em paralisia das obras de reforma urbana e também do fornecimento de luz e água, levando Salvador ao caos. Não bastasse isto, ainda havia o escândalo da encampação em 1914 do Banco de Crédito da Lavoura da Bahia pelo \textit{Banco de Crédito Hipotecário e Agrícola}, pactuada quase como uma ação entre amigos por Seabra e Eduardo Guinle, que resultou num prejuízo de 3.200:000\$000 aos cofres públicos baianos sem que houvesse qualquer punição aos envolvidos -- incluindo-se aí \textit{Emil Wildberger}, cuja casa comercial \textit{Wildberger \& Cia.}, além de estar então a caminho de se tornar a maior exportadora de cacau do mundo, representava interesses de capitalistas franceses como a \textit{Societé Génerale des Métaux de Paris}, o \textit{Crédit Lyonnais de Paris}, o \textit{Crédit Mobilier Français}, a \textit{Caisse Commerciale et Industrielle de Paris} o \textit{Crédit Foncier du Brésil}, o \textit{Nationale de Crédit de Paris} e outros. Explica-se assim a aproximação e posterior filiação, em 1913, do então presidente da Associação Comercial, Antônio Cabussú, ao PRC vianista e portanto à oposição antisseabrista.

Para piorar, o mandato de Antônio Moniz, correligionário e sucessor de Seabra no governo da Bahia, foi um verdadeiro descalabro, em parte devido à inabilidade do governador em negociar e transigir, como tanto fizera seu mentor, mas em grande parte porque as más escolhas de Seabra à frente de seu governo começavam a dar seus maus frutos. Os oposicionistas ao seabrismo viram ser paulatinamente bloqueados os canais de diálogo junto ao governo estadual abertos por Seabra, e portanto perderam o acesso às obras e cargos públicos que não apenas garantiam-lhe o prestígio, como também impactavam positivamente a economia local; por isso enfrentaram, inclusive de armas na mão, o que consideravam ``abusos'' do governador. No vale do São Francisco, primeiro exemplo dos conflitos do período, Franklin Lins de Albuquerque deu início, em janeiro de 1918, a enfrentamentos armados contra José Correia de Lacerda, Antônio Joaquim Correia e Adolfo Gomes de Queiroz; vencidos os três, enviou o governo a Força Pública\footnote{Diga-se de passagem que no Brasil da Primeira República as polícias estaduais eram tropas via de regra mal pagas e mal equipadas, postas em ação ao arbítrio dos governadores para subjugar seus adversários. No caso baiano, tratava-se de um contingente pequeno (média de 10 praças por município e 400, incluindo oficiais, estacionados em Salvador), mal-pago (o salário de uma praça, 1\$600, equivalia ao de um servente de pedreiro) e mal-equipado, e além do mais responsável pela segurança num Estado de grande tamanho vitimado pela precariedade de sua malha viária, tornando-a portanto incapaz de fornecer uma resposta tática rápida às situações a que era enviada para intervir \cite[pp.~46-47]{sampaio_legislativo_1985}.} para conter a revolta, que durou quase um ano e espalhou-se por diversas cidades no vale sanfranciscano, interrompendo a navegação fluvial e paralisando o comércio. Em Carinhanha, a indisciplina da Força Pública baiana -- era comum que os policiais compensassem os constantes atrasos em seus salários minguados com saques a lojas e fazendas dos opositores -- voltou-se contra João Duque, latifundiário com ligações também em Minas Gerais, de onde conseguia apoio por meio de incursões constantes da polícia mineira contra os policiais baianos; os opositores do seabrismo cedo transformaram as incursões policiais contra Duque numa bandeira de batalha, acusando Antônio Moniz de perseguição. Nas Lavras Diamantinas, a trégua armada de 1915 foi quebrada quando jagunços do pecuarista Militão Rodrigues Coelho, seabrista, invadiram o território de Horácio de Matos, latifundiário e comerciante de diamante e carbonato; o governador Moniz interveio em favor de Militão enviando a Força Pública, derrotada e desmoralizada pela jagunçada de Horácio de Matos em sucessivos combates. O governador Moniz viu-se forçado a ceder: ao apertar o cerco institucional contra os coroneis interioranos por meio da Lei Estadual 1.104, de 9 de maio de 1916, que obrigou os intendentes a remeter ao governador até o dia 15 de janeiro de cada ano cópia autêntica do orçamento municipal em vigor e quebrou, deste modo, o caráter absoluto da ingerência latifundiária sobre os assuntos fiscais municipais, Moniz distribuiu o quanto pôde de cargos públicos interior afora, proliferando o ingresso de representantes e protegidos políticos na burocracia estatal como juízes substitutos, coletores estaduais ou federais, diretores de agências de correios, inspetores de saúde etc., além da multiplicação das nomeações de esposas ou filhas de latifundiários locais como professoras em troca da concórdia política.

Nâo por acaso foi neste período que se deu a unificação das oposições ao seabrismo. Os latifundiários sertanejos, via de regra inseridos nos setores marginais da economia baiana voltados ao mercado interno, já se encontravam em pé de guerra contra o governo estadual. A burguesia comercial baiana, severamente impactada pela queda nas exportações de fumo e cacau durante a Primeira Guerra Mundial (1914-1918), angustiava-se com a a incapacidade do governador Moniz de influenciar o governo federal em seu favor, como no caso da apreensão pela marinha britânica de 1908 sacas de cacau e 1600 sacas de café despachadas a Copenhaga em 1915. Setores significativos dos jovens bachareis -- como Pedro Lago, os irmãos Calmon, Simões Filho e os irmãos Octavio e João Mangabeira -- bandeavam-se para a oposição, pela notória incapacidade do governador Moniz de entabular com eles qualquer diálogo conciliador. Fundamental para a articulação entre oposicionistas sertanejos e urbanos foi Manoel Alcântara de Carvalho, comerciante de diamantes, fazendeiro, homem de letras e correspondente de \textit{A Tarde} nas Lavras Diamantinas: foi ele quem pôs em contato Horácio de Carvalho com Pedro Lago e João Mangabeira, deslocados até as Lavras para acertar planos com os latifundiários antisseabristas, acertar a compra de armas e munição e tentar convencer -- com relativo sucesso -- os policiais a unir-se à revolta \cite[p.~201]{CUNHA2011}; foi também ele quem articulou a ocupação das vilas de Poções e Boa Nova (distritos de Vitória da Conquista) e das cidades de Jequié e Jaguaquara por Marcionílio de Souza, visando assim estabelecer uma cabeça-de-ponte na estação final da Estrada de Ferro de Nazaré e assim ameaçar diretamente o Recôncavo e a capital \cite[p.~202]{CUNHA2011}. Estava aberta, às vésperas da eleição para o governo baiano em 1919, a guerra pelo Estado; objetivava a oposição forçar uma intervenção federal e com isto jogar por terra as esperanças de Seabra de eleger-se mais uma vez governador.

Seabra, do Senado no Rio de Janeiro onde ocupara desde 1917 a vaga do falecido José Marcelino, via o ladeira-abaixo em que se convertera a política baiana e operava em sua própria causa. Apoiou a candidatura de Epitácio Pessoa à presidência em 1919 para garantir o apoio federal à sua reascensão ao governo baiano -- gesto imitado pelos oposicionistas quando da indicação, com o beneplácito de Ruy Barbosa, do nome de Paulo Fontes como candidato das ``classes conservadoras''. A revolta dos latifundiários do sertão alastrava-se, e seus jagunços preparavam-se para marchar rumo a Salvador; a notícia desta rebelião aterrorizou os blocos de poder noutros Estados, receosos de que o sucesso do levante sertanejo acendesse a faísca ignitora de sedições semelhantes noutros lugares e abalassem, portanto, as frágeis hegemonias; organizaram-se para pressionar o presidente Epitácio Pessoa a sair de sua postura não-intervencionista e intervir militarmente na situação baiana. Em fevereiro de 1920 Pessoa decretou estado de sítio em toda a Bahia, e tropas federais comandadas pelo general Cardoso de Aguiar concentraram-se em Bonfim, Cachoeira e Castro Alves; em seguida a negociações tensas com Horácio de Matos, Marcionílio de Souza e Anfilófio Castelo Branco, onde empregou até ameaças de uso da aviação militar contra os insurretos, o general mediou a assinatura de um documento que só não leva o nome de ``tratado de paz'' ou ``armistício'' por puro preciosismo (afinal, não se tratava de guerra entre Estados soberanos, mas entre cidadãos do mesmo país). Venceu Seabra, eleito enfim governador; venceram os latifundiários sertanejos baianos, que pela força das armas saíram da posição de solicitantes à de solicitadores e de colaboradores em pé de igualdade com o governo estadual, e de chefetes locais passaram a interlocutores diretos com o governo federal (por meio dos deputados que o ``armistício'' facultara-lhes indicar). Perdeu Seabra, forçado a dividir o poder que desejava concentrar e, mais uma vez, arrostado à oposição ao governo federal por indispor-se com Epitácio Pessoa, com cuja eleição colaborara; perderam os latifundiários sertanejos, pois galgaram melhores posições, aumentaram seus respectivos territórios, mas pouco a pouco retornaram ao relativo isolamento de seus domínios.

Um tal histórico dificilmente facilitaria o exercicio por Seabra de seu segundo mandato à frente do governo baiano. Esgarçada a política pelos conflitos imediatamente pregressos, não podia governar sem ceder muito além do que desejaria. Reflexo disto, a Lei Estadual 1.387, alargadora da autonomia municipal e afrouxadora do garrote arrochado pelas duas reformas municipais anteriores, impusera o diálogo com os intendentes, levando Seabra a realizar em 15 de março de 1921 o primeiro Congresso de Intententes Municipais da história da Bahia. Falido financeiramente o Estado, pouco podia fazer de grandes obras infraestruturais; esquivava-se Seabra de quaisquer compromissos de ordem material, e mesmo a demissão massiva de funcionários públicos e o aumento de impostos foram incapazes de restaurar o equilíbrio orçamentário, levando à contração de vultoso empréstimo junto à banca internacional -- e, dado o baixíssimo dinamismo econômico baiano, que resultava numa baixíssima capacidade de arrecadação tributária, cedo foi necessário reconhecer a total insolvência da Bahia frente a seus credores estrangeiros. 

Seabra, na tentativa de cair nas graças da burguesia comercial baiana, tentou ainda outras cartadas. Frente às constantes reclamações dos comerciantes em torno do aumento dos impostos, chamou a Associação Comercial da Bahia a debater o orçamento e a tributação estadual numa audiência pública, em especial porque os comerciantes das praças de Barreiras, Jaguaquara, Jequié, Ilhéus e Canavieiras -- precisamente, à exceção de Jaguaquara, aquelas não articuladas a Salvador por meio de ferrovias -- fugiam ao pagamentos dos tributos majorados abastecendo-se em atacadistas de outros Estados \cite[pp.~223-227]{CUNHA2011}; a solução encontrada foi um empréstimo global junto ao Banco Econômico para unificar a dívida interna -- fazendo assim de Francisco Marques de Góes Calmon o novo operador financeiro do Governo da Bahia e abrindo junto aos demais sócios do banco -- Bernardo Martins Catarino, Vital Batista Soares, Francisco Sá, Artur Cesar Rios, José Batista das Neves e outros altos representantes das finanças e do comércio baiano -- interlocutores privilegiado no tocante aos assuntos financeiros do Estado \cite[pp.~231-232]{CUNHA2011}. Respondendo a protestos de comerciantes baianos contra a insuficiência da Companhia Baiana de Navegação e contra a ``desorganização e imprestabilidade'' da Viação do São Francisco, de fato abandonadas à própria sorte durante a hegemonia seabrista, em 1921 Seabra extinguiu a primeira e mediou junto à Assembleia Geral da Bahia a construção de outra empresa, esta agora com maioria de ações pertencentes ao Governo da Bahia mas partilhando o poder acionário com renomados comerciantes baianos; quanto à Viação do São Francisco, menos rentável, Seabra arrendou-a pura e simplesmente \cite[pp.~222-223]{CUNHA2011}. Soma-se a isto o uso estratégico de concessões ferroviárias para angariar apoio político: ainda sob o governo de Antônio Moniz, entre 1918 e 1919, o comerciante de origem portuguesa Raimundo Pereira Magalhães, dono da casa Magalhães \& Cia., conseguira a concessão da Ferrovia de Santo Amaro, facilitando assim o escoamento da produção das usinas açucareiras reconcavinas agrupadas na empresa \textit{Lavoura e Indústrias Reunidas} (LIR) de sua propriedade; monopolizando assim o circuito ferroviário do Recôncavo, a LIR foi paulatinamente incorporando outras usinas menos favorecidas pela malha ferroviária e pelo monopólio da Magalhães \& Cia. sobre esta última, tudo isto fazendo de Raimundo Pereira Magalhães o mais poderoso comerciante de açúcar da Bahia e o principal interlocutor de Seabra e de Antônio Moniz frente à burguesia comercial \cite[pp.~227-228]{CUNHA2011}. De igual maneira, Seabra privatizou a ferrovia de Nazaré, concedendo-a em 1921 à empresa Bahia \& Mattazzi, da qual era sócio o comerciante Henrique Soares Bahia \cite[p.~228]{CUNHA2011}. 

A carreira política de Seabra na Primeira República seria marcada ainda por outro revés, na verdade o último: candidato à vice-presidência na chapa presidencial encabeçada por Nilo Peçanha e apoiada pelos blocos hegemônicos na Bahia, Rio Grande do Sul, Pernambuco e Rio de Janeiro, assentado internamente no apoio da burguesia comercial baiana, foi derrotado pela chapa encabeçada por Artur Bernardes e apoiada por todos os demais Estados brasileiros. As sucessivas ausências de Seabra durante a campanha aprofundaram as divisões internas no bloco seabrista, na medida em que era substituído por Frederico Costa, desafeto do grupo próximo ao ex-governador Antônio Moniz. Nem o apoio explícito do eterno desafeto Ruy Barbosa à chapa Peçanha-Seabra pacificou a oposição: os irmãos Mangabeira, Pedro Lago e Miguel Calmon apoiaram timidamente a chapa, como retribuição ao apoio dado por Seabra à nova eleição de Ruy Barbosa ao Senado federal, mas outros oposicionistas, como Aurelino Leal, Virgílio de Lemos, Aurélio Vianna, Ubaldino de Assis e Bráulio Xavier fizeram campanha aberta em favor da chapa Bernardes-Urbano. 

Eleito Bernardes, o oposicionista Miguel Calmon foi cooptado para o Ministério da Agricultura, Indústria e Comércio -- indicativo de que os dias políticos de Seabra estavam contados. Daí em diante, todos os esforços de Seabra para reconciliar-se com a burguesia comercial e com os latifundiários pareceram baldados: já quando do \textit{habeas corpus} dirigido ao Supremo Tribunal Federal onde requeria ser conduzido à vice-presidência por força da morte de Urbano Santos, o eleito, a Associação Comercial da Bahia fazia publicar n'\textbf{O Imparcial} matéria intitulada ``O \textit{habeas corpus} de Seabra, instrumento de subversão e anarquia''. Os últimos dois anos de Seabra à frente do governo estadual foram de verdadeiro pesadelo: acossado pelo governo federal e pelos oposicionistas agora unificados, tentava negociar como sempre, mas seu poder de barganha, antes enorme, agora era quase nulo. A debandada das hostes do seabrismo frutificara: integravam a oposição não apenas seus integrantes históricos ou aliados de ocasião, mas figuras de proa como Geraldo Rocha, advogado e latifundiário pecuarista, político de destaque em Barreiras tal como seus cunhados Francisco Rocha e Antônio Balbino de Carvalho; homem de negócios influente por todo o Brasil e conhecedor íntimo dos corredores do poder e das casas bancárias europeias, Geraldo Rocha amenizara por longo tempo as pressões dos credores da dívida externa baiana, e agora abandonava o barco seabrista. Revigorava a oposição antisseabrista uma nova leva de bachareis como Álvaro de Carvalho, Fernando São Paulo e Mário Leal, todos da Faculdade de Medicina; Virgílio de Lemos e Homero Pires, da Faculdade de Direito; entre outros.

A estratégia dos oposicionistas para a derrubada de Seabra parecia-se com o que já haviam tentado em 1919, agora sem o componente armado: forçariam uma intervenção federal, aos moldes do que acontecera no Rio de Janeiro pouco tempo antes, desta vez por meio de uma \textit{duplicata} do Legislativo. Seguiu-se à risca o roteiro das duplicatas: duas Juntas Apuradoras -- uma governista, outra oposicionista -- abriram seus trabalhos quase simultaneamente em meio a uma guerra de \textit{habeas corpus}, e declararam vencedores seus respectivos candidatos; instado o Supremo Tribunal Federal a intervir na questão, tendo em vista o bloqueio do acesso dos deputados oposicionistas ao prédio da Câmara dos Deputados, no Campo Grande, remeteu o STF a decisão ao Ministério da Justiça, que, controlado por Artur Bernardes, advertiu o governador Seabra a garantir acesso aos deputados sob pena de intervenção federal; a Câmara estadual governista abriu seus trabalhos no prédio da Biblioteca Pública, sob a presidência de Frederico Costa, e a Câmara estadual oposicionista instalou-se no prédio oficial do Campo Grande amparada pelo Exército.

Perdida a batalha no Legislativo; \textit{solus, totus et unus} ao fim de seu mandato, por suas próprias palavras; considerando-se ``generalíssimo abandonado pela maioria do seu estado maior e por outros generais do seu exército político''; encetou mesmo assim uma campanha eleitoral encarniçada contra seus opositores na eleição governamental de 1923/1924, ao ver a impossibilidade de qualquer transigência. Trocou correspondência com seus opositores em busca de termos viáveis de ``rendição'' política, sem sucesso: era ou derrota humilhante, ou vitória sob cabresto. Verdadeiro ``capoeirista político'' \cite[p.~133]{sampaio_partidos_1978}, saiu-se com uma jogada de mestre: indicou à sua sucessão seu antigo aluno na Faculdade de Direito do Recife, Francisco Marques de Góes Calmon, advogado comercialista, intermediador financeiro do Governo da Bahia e burocrata bancário com longa experiência, ao mesmo tempo presidente do Banco Econômico da Bahia, do Instituto dos Advogados da Bahia, irmão do ministro Miguel Calmon e sem qualquer trajetória política prévia a indicar suas afinidades. Aceitar Góes Calmon como candidato era ceder a Seabra; rejeitá-lo era afrontar o Calmon ministro. Para cúmulo das aflições dos oposicionistas, Ruy Barbosa morreu ainda durante as negociações pré-eleitorais em 1º de março de 1923, a oposição perdera não apenas um símbolo e um líder político com décadas de experiência, mas igualmente um contato seguro com o meio empresarial baiano e internacional. Enquanto a oposição, desarmada e atônita, conferenciava com aliados no Rio de Janeiro e em São Paulo, protelava decidir-se quanto à candidatura indicada e perdia tempo, brotavam de todos os cantos manifestações de apoio a Góes Calmon: a burguesia comercial, o arcebispo da Bahia e primaz do Brasil, funcionários públicos, médicos, corretores, professores, Horácio de Matos, Franklin de Albuquerque, Douca Medrado, Marcionílio de Souza, João Duque (retonado a Carinhanha), Rafael Jambeiro, Durval Fraga, a candidatura parecia verdadeira unanimidade e concórdia.

Parecia salvo o seabrismo. A eleição para o substituto de Ruy Barbosa no Senado federal mostrou-se verdadeira prévia da disputa estadual, com nova duplicata: Pedro Lago declarado vencedor pela oposição, Arlindo Leoni declarado vencedor pelos seabristas. Bernardes interveio parabenizando Lago pela vitória, e o resultado definiu-se em favor dos oposicionistas. A disputa chegou ao cúmulo do paroxismo: em outubro de 1923 Góes Calmon foi indicado como candidato tanto pelos oposicionistas quanto pelos seabristas, ainda que cada campo mantivesse a beligerância contra o outro, e obteve ainda por cima o apoio do presidente Artur Bernardes. Menos de duas semanas depois, a bomba: Seabra retirou seu apoio à candidatura calmonista, e Artur Bernardes declarou não querer interferir nas intrigas locais baianas; Arlindo Leoni foi indicado por Seabra como candidato ao governo quase de modo simbólico, às vésperas das eleições e quase sem apoio algum. Entre as eleições, em dezembro de 1923, e a proclamação do resultado, em fevereiro de 1924, as tensões aumentaram, e mais uma duplicata eleitoral resultou na decretação por Artur Bernardes de estado de sítio na Bahia por trinta dias; eleito enfim Góes Calmon, sua posse foi garantida pelas baionetas do Exército, que impediram Seabra de retirar-se rumo ao interior do Estado para não se ver forçado a empossar o novo governador. Enfim derrotado, humilhado em todas as esferas, apeado do poder político, Seabra embarcou rumo ao Rio de Janeiro no mesmo 28 de março em que empossara Góes Calmon governador, partindo de lá para um exílio autoimposto na Argentina e na França.

\subsection{A inquieta consolidação calmonista (1924-1930)}

A torrente adesista durante a campanha eleitoral calmonista parecia indicar um governo tranquilo. Errado: o bloco de poder encabeçado por Calmon era frágil, composto pelos retalhos dos grupos políticos unificados apenas por força da oposição a Seabra. Ex-seabristas, herdeiros do severinismo e epígonos do marcelinismo, anulado o inimigo comum, tornaram a digladiar-se, agora agrupados nos blocos \textit{calmonista} e \textit{mangabeirista}. 

Ainda esfacelada a política baiana, desinteressados os irmãos Calmon pelas intrigas partidárias diante das graves e urgentes questões administrativas que premiam o Estado, foi a intervenção pessoal de Artur Bernardes quem estabilizou a princípio a situação durante as eleições federais de fevereiro de 1924, legitimando inclusive a posição privilegiada à frente do Senado baiano ocupada por Frederico Costa, chefe agora dos ex-seabristas à falta de outra liderança que não o ex-governador Antônio Moniz caído em desgraça. 

No interior não foram poucos os latifundiários que, como Franklin de Albuquerque, Francisco Rocha, Francisco Leobas de França Antunes, Antônio Balbino, Abílio Wolney, Horácio de Matos e outros tantos, cedo puseram-se em oposição ao governador, ameaçando nova guerra nos sertões -- que teria inclusive acendido esperanças nos revoltosos sul-riograndenses de 1924 não fosse, depois da trágica \textit{batalha de Lençóis} em que as tropas estaduais foram dizimadas pela jagunçada de Horácio de Matos, a assinatura do \textit{Acordo de Mucugê}, novo ``armistício'' em que o ocupante do governo estadual se viu, como seus antecessores, premido pela força das armas a reconhecer a autoridade local dos latifundiários.

A permanente inquietação, entretanto, foi aplacada por dois fatos políticos externos às intrigas políticas baianas. 

Inovando no processo de indicação das candidaturas oficiais, Artur Bernardes orientou os governadores estaduais a procederem a convenções eleitorais consultivas onde as candidaturas, submetidas a debate e escrutínio aberto, fossem construídas num processo ``democrático'' -- sob estado de sítio semipermanente, mas ``democrático''. Tal orientação foi executada à risca por Góes Calmon, que em agosto de 1924 aproveitou a oportunidade em que se reuniam em Salvador representantes políticos de todos os 141 municípios baianos para dialogar e negociar com eles as condições de recuperação de seu proprio prestígio político. 

Quando a nomeação de Otávio Mangabeira ao Ministério das Relações Exteriores pelo recém-eleito Washington Luís (1926-1930) parecia dar novo alento à oposição mangabeirista, entrou sertão baiano adentro a \textit{coluna Miguel Costa-Prestes}, que desde 1925 varava os sertões brasileiros. Tal como em outros Estados, o governo federal subvencionou até fevereiro de 1927 a formação de ``batalhões patrióticos'' para combater os revoltosos; na Bahia, tais batalhões foram formados por jagunços a mando de latifundiários como Franklin de Albuquerque, Castelo Branco, Douca Medrado e Horácio de Matos. 

Com o ``inimigo externo'' varando os sertões, acalmaram-se todas as refregas políticas baianas; retornada a normalidade, a acalmia foi aproveitada por Góes Calmon para fundar um novo \textit{Partido Republicano da Bahia} (PRB), cujo funcionamento beneficiou-se da experiência das convenções bernardistas para estruturar um partido de novo tipo: a \textit{Convenção das Municipalidades} ocorrida entre 14 a 16 de janeiro de 1927 marcou a segunda congregação de representantes dos 151 municípios baianos para resolver de forma negociada suas dissensões e também a passagem de poder de velhos latifundiários para seus sucessores políticos, via de regra jovens bachareis súbito tornados representantes dos interesses políticos de seus mentores latifundiários. Deocleciano Teixeira, de Caetité, por exemplo, fez representar seu território por seus dois filhos, \textit{Anísio} e \textit{Mário Spínola Teixeira}; os Rocha de Barreiras, talvez por acordo com Horácio de Matos, mandataram o deputado federal \textit{Francisco Rocha} para representar Barreiras, Brotas de Macaúbas e Lençóis; \textit{Cícero Dantas Martins} e \textit{João da Costa Pinto Dantas Júnior}, filhos do senador João da Costa Pinto Dantas e descendentes do barão de Jeremoabo, representaram o bastião da família; outros tantos seguiram o mesmo protocolo. Os mais destacados latifundiários -- Antônio Honorato de Castro, Franklin Leobas de França Antunes, Franklin Lins de Albuquerque, Horácio de Matos, João Correia Duque, Marcionílio de Souza -- tinham asssento no Conselho Geral do PRB. Tensões políticas remanescentes quanto à distribuição de candidaturas para as eleições para o Congresso federal de 1927 foram ajustadas por meio da arbitragem do presidente Washington Luís e do Ministro da Justiça, Augusto Viana do Castelo.

Apaziguadas as rusgas, Góes Calmon pôde enfim voltar todas as suas atenções para o saneamento dos problemas administrativos legados por Seabra. Tentou instituir o \textit{imposto territorial} em substituição gradual e progressiva ao imposto de exportação, atendendo a pedidos da burguesia comercial e dos latifundiários cacauicultores. Ampliou a malha rodoviária. Além de quase triplicar a dotação orçamentária estadual destinada à educação, promoveu uma completa reforma no setor por meio dos 268 artigos da Lei Estadual 1.846, que tornou obrigatória a frequência escolar gratuita de crianças de 7 a 12 anos; obrigou municípios a destinar 6\% de sua renda à educação primária, podendo para isto instituir impostos especiais; e impôs aos estabelecimentos industriais a manutenção às suas custas de escolas para a alfabetização de operários e de seus filhos. Regulamentou a carreira funcional dos policiais, tentando eliminar da corporação as influências partidárias, e estabeleceu a obrigatoriedade do título de bacharel em ciências jurídicas e sociais para o exercício do cargo de delegado. Multiplicou as nomeações à burocracia estatal, aumentando enormemente o quadro do funcionalismo público -- que, como visto anteriormente, havia sido espremido ao máximo pela tentativa de saneamento fiscal de Seabra em seu segundo mandato -- e garantido, com as nomeações, a fidelidade política dos latifundiários interioranos; apesar disto, restringiu o acesso a cargos como juiz, delegado e coletor de impostos por meio da imposição do bacharelado como critério de acesso. Góes Calmon dava claros sinais de que pretendia ampliar as infraestruturas econômicas mais elementares (malha viária, por exemplo) e ampliar a participação dos bachareis na burocracia estatal, a pouco e pouco transformada de moeda política em carreira regular.

Todas estas medidas foram recebidas com hostilidade. A gritaria generalizada de todos os demais setores latifundiários impediu a consolidação do imposto territorial, levando à sua extinção em 1929\footnote{É de se observar que em outros Estados este imposto constituía a principal fonte de arrecadação de rendas tributárias \cite[p.~160]{sampaio_partidos_1978}.}; em Jequié, por exemplo, o latifundiário João Borges recusou-se a pagar o imposto e insuflou a população local a protestar contra o governo, e mesmo sua remoção do cargo de intendente e posterior expulsão do PRB não bastou para que parasse a agitação. A penúria dos orçamentos municipais impedia a aplicação dos 6\% de sua renda à educação, e o pauperismo generalizado inviabilizava a criação de qualquer tributo adicional -- afinal, como cobrar de uma população que não tinha sequer o que comer? A imposição do bacharelado como critério para nomeação às delegacias de polícia não impediu os ``acertos'' políticos prévios às indicações.

Não obstante, a hostilidade não resultou, como antes, numa revolta generalizada dos latifundiários. Reinava nas relações entre o governo estadual e os latifundiários uma paz poucas vezes vista. Paz tensa, como o indica a tentativa de implementação do imposto territorial; mesmo assim, paz, pois em tempos pregressos a situação teria sido resolvida a bala. A bibliografia consultada limitou-se a registrar o fato sem deter-se em qualquer análise pormenorizada\footnote{\citeonline{TAVARES2008}, dentro dos limites didáticos que caracterizam sua obra mais conhecida, registrou o fato sem maiores explicações. \citeonline[pp.~160-165]{sampaio_partidos_1978} atribuiu esta acalmia ao ``surgimento de uma mentalidade empresarial no Estado'' pela ação de Góes Calmon e ao surgimento de uma nova técnica do clientelismo, embasada não mais na ação direta dos latifundiários, mas ao uso de bacharéis como intermediários dos latifundiários junto ao governo, que funcionavam como verdadeiros advogados, angariando prestígio à medida em que a representação de suas ``causas'' junto ao governo obtinham sucesso. \citeonline{pang_coronelismo_1979} sequer dá atenção ao assunto, observando aqui e ali como advogados e engenheiros, além de tornarem-se políticos profissionais de pleno direito, agiam como ``conselheiros políticos de coroneis''; a \textit{pax calmoniana} é creditada pelo autor à emergência de uma ``oligarquia colegiada'', sem qualquer explicação sobre o \textit{modus operandi} desta colegialidade ou sobre as razões de suas tensões intestinas que não os sempiternos e desequilibrados conflitos interoligárquicos.}. Sem embargo, na tentativa de formular hipóteses explicativas, não se deve negligenciar que, nos últimos anos da República Velha, o número de bachareis politicamente ativos multiplicara-se, como o testemunham a emergência dos ``jovens turcos'' na década de 1910 e o verdadeiro ``rito de passagem'' representado na convenção municipal de 1925. Tome-se como amostra desta proliferação os bachareis em Direito\footnote{Uma análise completa exigiria a aplicação do mesmo método empregue adiante também com egressos da Faculdade de Medicina da Bahia (1808), da Escola Agrícola da Bahia (1877) e da Escola Politécnica da Bahia (1897), mas fugiria ao tema da pesquisa que aqui se expõe.}. Não bastassem os bachareis egressos das faculdades de Direito de Recife e São Paulo, tradicionais escolas de políticos e burocratas desde os últimos anos coloniais plenamente ativas durante a Primeira República, foi inaugurada em 1891 a \textit{Faculdade Livre de Direito da Bahia}, terceira de seu gênero no país e, estando em Salvador, muito mais acessível aos estudantes oriundos das decadentes famílias latifundiárias baianas que as duas primeiras. Formando a cada ano entre sete a sessenta e seis novos bachareis em ``ciências jurídicas e sociais'', totalizando 1.013 novos bachareis formados entre 1891 e 1930 (média de 25 por ano) \cite[p.~25-175]{modesto_bachareis_1996}\footnote{Apenas para comparação, a Escola Politécnica da Bahia, hoje integrante da Universidade Federal da Bahia, formou entre 1901 (ano de sua fundação) e 1930 um total de 114 engenheiros geógrafos, 380 engenheiros civis e 9 bachareis em ciências matemáticas e físicas, totalizando 503 diplomados no período; as turmas formaram a cada ano entre 8 (1908) a 47 (1918) diplomados, numa média aproximada de 17 por ano \cite[pp.~53-68]{costa_politecnica_2005}. Paralelamente, a Faculdade de Medicina da Bahia diplomara até 1889 um total de 1.934 bacharéis, e entre 1890 e 1930 diplomou ainda outros 2.053 profissionais em turmas que variavam de 4 (1901) a 115 (1919) formandos (média aproximada de 50 por ano) \cite{fameb_formados_2008}. Como o lugar ocupado por médicos e engenheiros na política, na estrutura burocrática estatal e nas empresas capitalistas era, no período, muito distinto daquele ocupado pelos bacharéis em Direito, qualquer comparação exigiria mediações incompatíveis com o objeto e a finalidade deste estudo. Mais adiante, no lugar adequado (\autoref{subsec:gestprodespbrotas}, p. \pageref{subsec:gestprodespbrotas}), o papel dos engenheiros na burocracia estatal e no mercado da construção civil soteropolitanos será descrito, analisado e criticado.} a espalhar-se Bahia adentro nas mais diversas funções, desde a advocacia, a magistratura e outras funções burocráticas até mesmo ao governo da Bahia (Antônio Moniz Sodré de Aragão, formado na turma de 1903, e Vital Soares, formado na turma de 1897)\footnote{Segundo a mesma fonte empregue no corpo do texto, destacaram-se também na política os seguintes bachareis formados pela Faculdade Livre de Direito da Bahia neste período: Aurelino Leal (1894); Pedro Veloso Gordilho (1895); José Álvaro Cova (1896); João Mangabeira (1897); João Vicente Bulcão Vianna (1900); Ernesto Simões Filho, Pedro de Azevedo Gordilho (1907); Demétrio Tourinho (1909); Isaías Alves, Wanderley Pinho (1910); Nelson Sampaio da Costa Dória (avô do atual prefeito de São Paulo, João Dória) (1913); Guilherme Marback (1919); Clemente Mariani (1920); Junqueira Ayres (1922); Nestor Duarte Guimarães (1924); Aliomar Baleeiro, Hermes Lima (1925); Lafayette Pondé, Luiz Viana Filho, Osvaldo Veloso Gordilho (1929). Esta faculdade é hoje a Faculdade de Direito da UFBA.}. Nâo é de espantar, portanto, sua presença cada vez mais acentuada na política baiana, marcando a passagem a uma época de predomínio dos ``doutores'' e dos burocratas. Isto e o domínio das técnicas jurídicas, administrativas e retóricas adquirido nos bancos dos cursos de ``ciências jurídicas e sociais'', tarimba antes adquirida apenas mediante longa e penosa vivência das rusgas eleitorais e da prática administrativa já dentro da máquina do Estado, levam a crer que estes jovens ``doutores'' -- projetos de burocratas, aprendizes de gestores -- chegavam à esfera política em condições muito mais vantajosas que latifundiários ou burgueses de mesma idade desprovidos de educação universitária, suprimindo-os na competição política graças à maior facilidade com que traziam ``resultados'' para seus ``clientes'' políticos. Os ``doutores'' não eram infensos ao uso das armas, como bem o mostra sua participação no levantes sertanejos de 1919, mas, posto tudo na ponta do lápis, a alternativa das duplicatas eleitorais, além de menos custosa, desenvolvia-se num campo de seu completo domínio -- e não naquele da mobilização de hostes armadas característicos dos latifundiários sertanejos. 

Trabalha-se aqui, portanto, com a hipótese de que o custo econômico da luta armada, a proliferação dos ``doutores'' e sua \textit{expertise} nos meandros das leis e da administração pública condicionaram, no conjunto, a passagem da \textit{política da bala} para a \textit{política da pena}, para um modelo dialogado e concertado de solução de divergências, enfim, para uma política já acomodada aos moldes de um republicanismo cada vez mais burocratizado (em comparação com o verificado nos anos anteriores). Isto é de suma importância para que se compreenda adequadamente em seu contexto a atuação dos engenheiros da Diretoria de Obras em Salvador e também em Brotas, a ser vista no \autoref{cap:3} (p. \pageref{cap:3}).

Calmon indicou para substituí-lo \textit{Vital Henrique Batista Soares}, eleito em 1928 com as formalidades do novo estilo consensual, convencional e congressual da política baiana. Seguiu Vital Soares todas as linhas políticas de seu antecessor -- incluindo sua política de desmembramento de municípios cujos latifundiários hegemônicos o antagonizassem ou a substituição dos juízes, delegados e coletores indicados pelos eventuais oposicionistas. A mesma máquina partidária do PRB garantidora da tranquilidade sucessória servia para apaziguar os ânimos de opositores. Tamanha calmaria interna e o sólido apoio do presidente Washington Luiz levaram Vital Soares a arriscar-se na candidatura à vice-presidência em meados de 1929 -- e o resto, com a vitória da chapa Júlio Prestes-Vital Soares contra a Aliança Liberal, é História.

\subsection{Tendências a destacar no período}

Viu-se, portanto, que na poĺítica baiana da Primeira República desenharam-se algumas tendências.

A primeira delas é a persistência do poder local dos latifundiários, que o defendiam inclusive de armas na mão (fossem suas as mãos, ou as de seus jagunços). A contradição entre latifundiários exportadores e latifundiários voltados para o mercado interno, se não teve na Bahia o caráter explosivo visto na escala nacional, manifestou-se com toda a força quando da tentativa de implementação do imposto territorial, mostrando estar latente nas relações internas à classe latifundiária.

O mecanismo simultaneamente oligopsônico e creditício de submissão dos latifundiários à burguesia comercial verificou-se também no caso baiano; verificou-se também na Bahia o caráter politicamente ativo da burguesia comercial exportadora, embora a bibliografia consultada não tenha sido nem clara nem unânime quanto ao papel político da burguesia comercial importadora no período.

A bibliografia consultada pôs num só quadro explicativo a ação política dos gestores e da pequena burguesia, mas foi possível verificar como os estratos superiores destas duas classes sociais viveram um período de ascensão política que teve sua maior expressão na conferência municipalista de 1925. A classe trabalhadora, grande ausente de toda a política eleitoral brasileira durante a Primeira República, na Bahia interveio nas disputas políticas apenas mediante as grandes greves do ciclo 1919-1920 e de protestos contra a ``carestia de vida''.