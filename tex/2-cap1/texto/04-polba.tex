\section{A política baiana (e soteropolitana) sob constante agitação}\label{sec:1.3}

Assim desenhada a situação econômica e social baiana, não parece ser difícil entender por que a política baiana viveu tempos tumultuados durante a Primeira República. A participação marginal da economia baiana na produção agroexportadora; a concentração da produção voltada à exportação em duas regiões do Estado e a debilidade econômica das demais regiões; a insignificância da malha rodoviária, a deterioração da rede fluvial de transportes e a insuficiência da malha ferroviária para interligar mais intensamente as diversas regiões do Estado com sua capital e principal porto; a dependência imposta pela burguesia e pelos gestores ligados ao comércio aos latifundiários; a extrema concentração de capitais e o pauperismo generalizado; tudo isto cria as condições para que seja o Estado a principal fonte de acesso a recursos para investimentos, e faz do acesso aos postos governamentais a condição para a sustentação do prestígio político de chefes políticos locais em franca decadência econômica. A simples construção de um quadro econômico, entretanto, não é suficiente para entender as forças motrizes da política baiana, nem consegue explicar, por si só, a ação política das classes sociais voltadas à superação da situação desoladora já descrita e redescrita; é preciso ver, na esfera propriamente política, como se movimentaram as classes sociais para superar este quadro, e como sua ação política, direta ou indiretamente, teria dado ainda outros elementos para aprofundar o declínio econômico a que gerações de políticos e economistas chamaram de ``enigma baiano'' \cite{aguiar_notas_1958}.

Resistente à república como via de regra o foram as demais províncias ditas ``nortistas'', cedo, cedo mesmo, em menos de uma semana a política baiana encontrava-se plenamente integrada ás instituições republicanas \cite{sampaio_legislativo_1985}. Daí em diante, a política baiana durante a Primeira República, para facilitar o entendimento, costuma ser periodizada conforme os grupos a hegemonizá-la em cada momento, tendo em vista que os partidos políticos que então se formavam eram efêmeros, existiam apenas nas vésperas das eleições, e durante todo o tempo restante o fator aglutinador na política era não a organização partidiária, mas a fidelidade a chefes políticos e suas alianças ou cisões, tornando-se assim as organizações partidárias verdadeiras claques \cite[p.~18]{sampaio_partidos_1978}. Assim, há um período de \textit{consolidação da República} (1893-1896) onde os blocos de poder não estavam claramente configurados e o conflito intergrupal era a regra. Seguiu-se um período dominado, sucessivamente, pelos partidários de \textit{Luís Viana}, contestados pelos sequazes de \textit{José Gonçalves} (1896-1900); pelos partidários de \textit{Severino Vieira}, contestados pelos gonçalvistas e pelos vianistas (1900-1904); pelos partidários de \textit{José Marcelino de Sousa}, a princípio em aliança com os severinistas (1904-1907) mas pouco depois em oposição desabrida (1907-1912). A intervenção federal de janeiro de 1912, parte da política das salvações do presidente Hermes da Fonseca, marca o início da longa e turbulenta hegemonia de \textit{José Joaquim Seabra} e seus partidários (1912-1924), seguida, encerrando o período da Primeira República, pela hegemonia do grupo capitaneado por \textit{Francisco Goes Calmon} (1924-1930). 

Ressalte-se, para evitar confusões, que \textit{hegemonia} não significa, necessariamente, \textit{posse do chefe político no cargo de governador}, mas sim a capacidade de um grupo político de manter-se no poder por meio do consenso e da coerção; a constituição baiana impedia a recondução do governador ao cargo sem o intervalo mínimo de um mandato, o que impedia as reeleições e impunha alguma rotatividade no cargo máximo do Executivo baiano entre integrantes do mesmo bloco político. Seabra, por exemplo, foi governador duas vezes (1912-1916 e 1920-1924), mas continuou hegemônico na política estadual durante o mandato de Antônio Moniz Sodré de Aragão (1916-1920), seu correligionário. De igual maneira, pode-se dizer que os mecanismos de engenharia política vigentes na esfera federal, consolidados na ``política dos Estados'', foram adaptados às instituições políticas baianas \cite{sampaio_legislativo_1985}, resultando igualmente numa circularidade de favores e poderes: as eleições para o Executivo e o Legislativo estaduais eram controladas nos municípios pelos coroneis, cuja fidelidade política ao grupo hegemônico do momento assegurava a eleição dos candidatos deste grupo, fidelidade esta recompensada pelo controle sobre os cargos municipais conferido pelo grupo político hegemônico de cada momento.

Interessa a esta pesquisa, como já dito, muito menos a simples sucessão das personagens a ocupar o Palácio Rio Branco que as turbulências da política baiana da Primeira República, ou, melhor dizendo, as \textit{forças motrizes} destas turbulências. Não obstante, levando-se em conta o fato de os conflitos políticos na Bahia da Primeira República serem pouco conhecidos fora de um público especializado, não será possível fazer uma análise estritamente sociológico-categorial do processo político baiano do período, que deverá entremear-se com uma narrativa histórica simplificada dos fatos ocorridos. Por outro lado, o quadro da ``política dos Estados'' em nível federal impunha aos grupos hegemônicos na Bahia a cada momento alinhar-se aos sucessivos titulares do Executivo federal, sob pena de ostracismo político; interessa saber, portanto, além das forças motrizes das turbulências políticas baianas, quando houve alinhamento do governo baiano ao governo federal nos diferentes períodos em que se costuma categorizar a política baiana durante a Primeira República.

\subsection{Parâmetros de análise}

Há três obras clássicas para estudar a política baiana neste período. A primeira, mais antiga e frequentemente reeditada \cite{TAVARES2008}, justamente por seu caráter didático e introdutório pouco avança além do elenco de governadores e de algumas palavras sobre a economia baiana do período, inseridos, o elenco e a breve notícia econômica, no quadro da longa duração. A segunda, de escopo amplo \cite{pang_coronelismo_1979}, peca por anacronismo: pretende explicar as disputas políticas baianas por meio de uma complexa tipologia das oligarquias e dos coroneis e pelo emprego francamente anacrônico das estruturas de \textit{clã} e de \textit{tribo}, centradas no \textit{paterfamilias}, como sustentáculo da política baiana (e brasileira por extensão, dado que a Bahia foi escolhida por ser ``caso de estudo do coronelismo''). A terceira, mais completista e detalhista \cite{sampaio_partidos_1978}, peca entretanto por circularidade e anacronismo: \textit{circularidade}, porque nesta obra a fragmentação política baiana do período teria como causa a fragmentação econômica baiana; \textit{anacronismo}, porque projeta ao passado uma estrutura jurídico-legal ordenadora do processo eleitoral que, se vigente durante a Primeira República, talvez estancasse os conflitos interpartidários -- como se não existissem então, além de arcabouço legal próprio regulamentador das eleições, mecanismos políticos de regulação tanto das eleições, quanto da competição política, eficazes para aquilo a que seus criadores se propunham (o controle das eleições, a hegemonia política e a redução das incertezas sucessórias). Fundamentais antes como agora pela riqueza das informações recolhidas, estes três clássicos não conseguiram, entretanto, romper a superfície das disputas internas aos sucessivos blocos de poder e entender que forças levavam a tais disputas. 

Obras mais recentes alargaram o foco e mudaram o ângulo de análise; duas destacam-se do conjunto, pela amplitude de seus respectivos objetos e pela complementariedade entre eles. A primeira entre elas \cite{castellucci_maquina_2008} trata da política baiana não mais pelo ponto de vista dos latifundiários, burgueses e gestores, de suas intrigas palacianas, dos conflitos às vezes armados entre eles, mas pelo ponto de vista dos \textit{trabalhadores} e de sua difícil \textit{articulação política} durante a Primeira República. A segunda \cite{CUNHA2011} tenta enriquecer a análise das disputas de poder em meio às classes dominantes ultrapassando o âmbito das disputas na imprensa e nas tribunas parlamentares para ir mais a fundo, em meio aos arquivos privados, arquivos empresariais e correspondências pessoais sem entretanto rejeitar a pesquisa em arquivos públicos e na imprensa de época, e encontrar nas disputas entre empresas pela produção e gestão de infraestruturas urbanas uma das forças motrizes dos conflitos políticos da Primeira República.

As informações constantes nestas obras clássicas e recentes permitem alinhavar numa só narrativa os conflitos de classe esboçados na \autoref{sec:sobasotconv} (p. \pageref{sec:sobasotconv}). Para isto, em primeiro lugar, é preciso dialogar com a tese das ``oligarquias regionais'' \cite{pang_coronelismo_1979,sampaio_partidos_1978,TAVARES2008}, recebendo dela a radicação geográfica de famílias de destaque na política baiana\footnote{``Os clãs políticos dominantes demarcaram duas áreas de influência ao longo de limites \textbf{geoeconômicos}, e, dentro de cada zona, uma ou mais famílias surgiu como oligarquia municipal'' \cite[p.~76, \textbf{grifo nosso}]{pang_coronelismo_1979}.}; se o foco for transferido das ``oligarquias regionais'' para o funcionamento da economia baiana estruturado de modo geográfico, em especial numa hierarquização de centros urbanos regionais ainda que precária, e sendo as classes sociais radicadas nos diferentes estratos da rede urbana simultaneamente tributárias e construtoras desta estrutura hierarquizada por meio de sua atuação econômica, o arrolamento de patronímicos ilustres fará mais sentido. A análise da rede urbana baiana terá como base obras clássicas sobre o tema \cite{geiger_rede_1963,SANTOS1959}, e o cruzamento com a inserção econômica e geográfica dos coroneis será feito por meio de um dos mapas do \textbf{Atlas Histórico do Brasil} (cf. \autoref{2016-coroneisfgv}, na p. \pageref{2016-coroneisfgv}), produzido pela Fundação Getúlio Vargas (FGV) \cite{fgv_coroneis_2016} com base no anexo IV (``Cinquenta dos coroneis mais 'ativos' da Bahia, 1889-1937'') do livro de \citeonline{pang_coronelismo_1979}, incorporando inclusive alguns de seus erros mais flagrantes\footnote{Um exemplo: \textit{Abílio Rodrigues de Araújo}, chefe político de Rio Preto, no referido anexo de \citeonline[pp.~246-249]{pang_coronelismo_1979} é creditado como sendo de Feira de Santana.}. Para evitar repetições desnecessárias e a saturação visual da mancha gráfica desta dissertação, a referência a estas obras será feita apenas quando de citações diretas.

\begin{figure}[!htp]
\centering
\includegraphics[width=1\textwidth]{2-cap1/complementos/mapas/2016-coroneisfgv.eps}{\footnotesize \par \textbf{Fonte:} \citeonline{fgv_coroneis_2016}.}
\caption{Distribuição geográfica dos principais coroneis da Bahia (1889-1937).}\label{2016-coroneisfgv}
\end{figure}

Salvador e o \textit{Recôncavo} eram áreas de economia hegemonizada pelas plantagens de açúcar, com algumas culturas complementares. Salvador, centro macrocéfalo de vasta hinterlândia, obedecia, decerto, a outras dinâmicas; era o Recôncavo e sua sucrocultura decadente, entretanto, quem lhe dava régua e compasso, apesar de serem dependentes do mercado e do porto soteropolitano cidades como Cachoeira, São Félix, Santo Amaro e Nazaré. Feira de Santana, durante a Primeira República, entrava no Recôncavo de Salvador como seu limite extremo rumo ao sertão. Desta região açucareira vieram os Araújo Pinho (de Santo Amaro), os Meireles (de Mata de São João), os Calmon, os Mangabeira, os Prisco Paraíso (de Cachoeira), os Costa Pinto (de Santo Amaro), os Tosta (de São Félix), os Sousa (de Nazaré), os Bulcão (de São Francisco do Conde), os Moniz, os Aragão e os Vilas Boas, famílias de onde vieram figuras de proa da política baiana durante a Primeira República. Foram ainda os reconcavinos liberais e conservadores do Império, ao se conciliar, quem fundou o Partido Republicano da Bahia (PRB), hegemônico na política baiana até 1912.

A \textit{zona cacaueira}, de economia voltada para a exportação de cacau, era encabeçada por Ilhéus e Itabuna, onde também se radicam Canavieiras e Itapetinga. Aí se radicaram Misael Tavares, Pedro Levino Catalão, Domingos Adami de Sá e Oscar Falcão. Nas \textit{Lavras Diamantinas}, onde a mineração fazia e desfazia fortunas, destacavam-se do Centro-Sul, tendo Lençóis como centro econômico e Hegemonizava a política lençoense o coronel Felisberto Augusto de Sá. Eram também centros importantes: Mucugê (terra dos Rocha Medrado e do coronel Douca); Andaraí (terra de Aureliano Brito de Gondim); Brotas (terra de Militâo Rodrigues Coelho); Macaúbas (terra do monsenhor Hermelino Marques de Leão); Morro do Chapéu (terra de Francisco Dias Coelho); Campestre, atual Seabra (terra de Manuel Fabrício de Oliveira). Na região pecuarista do \textit{Nordeste baiano}, vocacionada para o mercado interno,  surgiram os Dantas Bião (Alagoinhas), os Dantas (Jeremoabo e Itapicuru) e José Gonçalves (Bonfim). O \textit{Centro-Sul}, onde na Primeira República a economia centrava-se na pecuária e na plantagem de café e cacau, encontram-se Jequié, Jaguaquara e Vitória da Conquista, então muito menor que a primeira. 

O extenso \textit{vale do São Francisco}, divisor do sertão baiano, tinha vários polos regionais, bocas de entrada para o sertão a partir da navegação ao longo do rio homônimo. Centros como Juazeiro, centro de armazendamento e polo comercial sanfranciscano e terra de coroneis comerciantes como José Alves Pereira, João Evangelista Pereira e Melo, Aprígio Duarte Filho  mas tinha como outros centros de destaque: Casa Nova (cidade de origem dos Viana e dos Castro); Sento Sé (território da família homônima e também dos Sousa); Pilão Arcado (terra dos Albuquerque) e Remanso (terra dos Castelo Branco e dos França Antunes); Barra (terra dos Wanderley e dos Mariani). O oeste do vale, região por muito tempo conhecida como ``além São Francisco'', era região extensa e pouquíssimo povoada, e a precariedade das ligações com a capital certamente acentuava a dependência da população local frente aos coroneis. Suas principais cidades: Barreiras (terra de Antônio Balbino de Carvalho, Abílio Wolney e dos Rocha); Correntina (terra de Félix Joaquim de Araújo); Santa Maria da Vitória (terra de Clemente Araújo de Castro); Santana dos Brejos (terra de Francisco Joaquim Flores); Rio Preto, atual Santa Rita de Cássia (terra de Abílio Rodrigues de Araújo). Carinhanha emergia no cenário por sua particularidade geográfica: fronteiriça entre Bahia e Minas, servia de entreposto para o comércio destes dois estados e também para o de Goiás, resultando em que a política local, hegemonizada pelos Duque e pelos Alkmin, vivesse pendendo ora para o lado baiano, ora para o lado mineiro.

Seria aplicável à política estadual o mesmo modelo discutido na \autoref{subsec:espadaleite} (p. \pageref{subsec:espadaleite})? Ou seja, seria a dinâmica da política baiana movida por um conflito entre classes sociais distintas, confundidas pela aparência de conflitos entre ``oligarquias regionais''? Ou, ao contrário, teria vigido na política baiana a simples oposição ``litoral'' e ``interior'', ainda hoje reproduzida nos mesmos termos estabelecidos há mais de século por Euclides da Cunha n'\textbf{Os Sertões}? Por outro lado, seria a capacidade de apaziguar conflitos internos determinante para estabelecimento de boas relações com o governo federal?

\subsection{Consolidação republicana, vianismo, severinismo, marcelinismo (1889-1912)}

Os conflitos ocorridos na fase de consolidação da república na Bahia (1889-1896) não permitem discernir um grupo hegemônico.

Durante a hegemonia vianista (1896-1900), talvez o conflito principal, além do massacre a Canudos (Viana era acusado por seus adversários de ``complacência'' com a comunidade sertaneja) e das expedições policiais promovidas contra os inimigos políticos de Luiz Vianna em Ilhéus, Belmonte, Canavieiras, no Nordeste baiano e nas Lavras Diamantinas, tenha sido a eleição municipal soteropolitana de 1899, em que a polícia investiu contra comerciantes, fato determinante do declínio político de Luiz Vianna, anatematizado pela burguesia comercial baiana até o fim de sua carreira política. 

Durante a hegemonia severinista (1900-1904) foi fundado o Partido Republicano da Bahia (PRB), tentativa de aglutinar numa só organização política representantes de regiões distintas do Estado mas muito solidamente fundado em meio aos latifundiários do Recôncavo baiano e aos burgueses comerciantes, frustrada ela enfim pela política vingancista do governador contra seus antigos desafetos interioranos. Severino Vieira foi Ministro da Viação e Obras Públicas durante a presidência de Campos Sales (1898-1900), sucedido pelo paulista, aliás nascido carioca, Alfredo Eugênio de Almeida Maia; sua escolha como candidato oficial ao governo baiano servia tanto como consolidação da ``política dos Estados'' salista como forma de afastar Severino do governo federal, onde já se desentendia com o Ministro da Fazenda, Joaquim Murtinho -- desentendimento aliás compartilhado com a burguesia comercial baiana, insatisfeita com as políticas daquele a quem chamavam de ``financista homeopata''. Ao ascender ao governo da Bahia, entretanto, Severino viu nomeado para o Ministério da Justiça do governo Rodrigues Alves seu adversário José Joaquim Seabra, numa expressão de desagrado do presidente pelo grupo hegemônico na Bahia. 

Durante a hegemonia marcelinista (1904-1912) reforçou-se a hegemonia açucareira -- o próprio José Marcelino era proprietário da Usina Conceição, em Nazaré -- e as medidas pró-agricultura do governo estadual (manutenção de Miguel Calmon na Secretaria de Agricultura; estudos de aperfeiçoamento da lavoura cacaueira por agrônomos alemães; estudos de implantação da cotonicultura no sertão baiano; investimentos no transporte fluvial etc.) eram muito bem-vistas pelos latifundiários baianos. A estruturação do Banco de Crédito da Lavoura da Bahia, ademais, servia para romper a dependência dos comerciantes em que os latifundiários se viam envolvidos, e fortaleceu ainda mais a hegemonia açucareira: o presidente era João Ferreira de Araújo Pinho, promotor santamarense membro de antiga família açucareira; o diretor Henrique Teixeira tinha trânsito fácil entre os comerciantes exportadores; e o secretário Viriato Ferreira Maia Bittencourt era ligado à família Calmon, também do ramo. José Marcelino de Sousa, além de apaziguar os latifundiários interioranos em seu favor com medidas de estímulo à economia local, disponibilização de crédito barato para a agricultura e instalação ou renovação de infraestruturas de transporte, viu seu secretário de agricultura, Miguel Calmon, ser cooptado pelo novo presidente Afonso Pena para o Ministro da Viação e Obras Públicas durante a presidência de Campos Sales (1906-1909). Calmon foi sucedido neste ministério pelo mineiro, aliás nascido cearense, Francisco Sá.

Foi durante a hegemonia marcelinista que surgiu um novo tipo de político, iniciamente apelidados de ``jovens turcos'', bachareis oriundos de famílias latifundiárias. Os irmãos Miguel e Antônio Calmon, o promotor Araújo Pinho, o jornalista Ernesto Simões Filho, o advogado e jornalista Pedro Lago, o advogado João Mangabeira, o jurista e jornalista Antônio Moniz e o engenheiro Otávio Mangabeira, ao invés de representarem uma ``mistura de classes'' \cite[p.~93]{pang_coronelismo_1979}, são elementos que, por meio da educação, da inserção profissional e em especial pela atuação política, \textit{transitaram de uma classe à outra}, nomeadamente da classe latifundiária à classe dos gestores.

A escolha de Araújo Pinho como candidato oficial à sucessão governamental em 1908 mostra como a oferta de crédito fácil para os latifundiários foi um dos elementos determinantes da hegemonia marcelinista: recebeu quase imediatamente apoio irrestrito de Dias Coelho (Morro do Chapéu) e de lideranças políticas de todo o vale sanfranciscano, pecuaristas e comerciantes dependentes da atuação do governo estadual para a criação, manutenção e gestão de condições gerais de produção (transporte, mercados etc.); por outro lado, em Ilhéus e Itabuna, vivendo então rápido desenvolvimento econômico por força das exportações cacaueiras, a disputa foi acirradíssima. Complementarmente, o apoio de Afonso Pena, Pinheiro Machado e Rui Barbosa à candidatura de Araújo Pinho mostra como a \textit{pax baiana} facilitara as relações entre o governo estadual e o governo federal; conta aí certamente a presença de Miguel Calmon no ministério, azeitando as relações.

A dissolução da \textit{pax baiana} marcelinista emergiu da disputa entre Severino Vieira, eleito deputado ao deixar o governo, e José Marcelino; tal disputa, ao invés de ser creditada ao puro conflito de personalidades entre dois líderes carismáticos, deve ser vista, ao contrário, como conflito pelo acesso ao governo federal, sabidamente o meio mais fácil para acesso aos recursos necessários para o desenvolvimento das condições gerais de produção durante a Primeira República e, portanto, para a criação das condições para a hegemonia política. Decerto o fator pessoal pesou, mas ele, por si só, explica pouco. Esfacelado o partido ainda antes da eleição governamental e fraturado o Estado quando da apuração dos resultados -- pois a campanha de Joaquim Inácio Tosta, apoiado por Severino Vieira, fora declarada vencedora em trinta municípios, contra oitenta onde Araújo Pinho fora declarada vencedora -- o mandato de Araújo Pinho à frente do governo da Bahia, à frente de um partido esfacelado entre severinistas e marcelinistas, foi incapaz de manter as boas relações com o governo federal; demonstra-o a incapacidade de apoiar a demanda dos cacauicultores de obter apoio à cacauicultura semelhante ao que se fizera ao café por meio do Convênio de Taubaté (1906). Por isto mesmo, por sua incapacidade de manter o mesmo ritmo de apoio à agricultura que seu antecessor e padrinho político, Araújo Pinho não conseguiu reunificar o PRB nem tampouco reconciliar as diferenças transformadas em verdadeiras batalhas campais.

\subsection{A turbulenta hegemonia seabrista (1912-1924)}

Em 1910 Araújo Pinho, José Marcelino e Severino Vieira apoiaram ativamente a campanha civilista de Rui Barbosa, enquanto Hermes da Fonseca foi apoiado por Luiz Vianna, José Gonçalves da Silva (então decadente) e, curiosamente, por um inimigo dos vianistas, José Joaquim Seabra. Vitorioso Hermes, tratou Seabra, mesmo frustrado em sua missão de cabo eleitoral hermista, de manter a aliança temporária com Luiz Vianna para isolar Severino Vieira, e com isto obteve em primeiro lugar o cargo de Ministro da Viação e Obras Públicas no governo federal; tal posto garantiu-lhe contatos preciosos com capitalistas internacionais e permitiu-lhe gerir demandas dos coroneis do interior, além de granjear-lhe a popularidade necessária à eleição ao governo em 1912. Tal eleição, entretanto, foi precedida por manobras políticas e conflitos que resultaram não apenas no bombardeio de Salvador em 10 de janeiro de 1912, mas igualmente na verdadeira guerra civil que durou entre 22 a 27 de janeiro e envolveu milhares de pessoas, barricadas, tiroteio entre a polícia e a turbamulta armada na capital etc. 

Teve início, com a eleição de Seabra em 1912, nova fase de alinhamento com o governo federal, com os sectários de Seabra agrupando-se no Partido Republicano Democrata (PRD), fundado em 1910 por uma constelação de bachareis\footnote{Trata-se do primeiro partido de bases nitidamente urbanas na Bahia: entre seus fundadores de 1910 contavam-se 40 ``doutores'', 33 ``coroneis'', 3 ``cônegos'', 2 ``conselheiros'', 1 ``comendador'', 1 ``desembargador'', 1 ``capitão'', 1 ``farmacêutico'' e 2 ``não portadores de títulos'' \cite[p.~70]{sampaio_partidos_1978}. Comparativamente, o Conselho Geral do Partido Republicano da Bahia, instrumento político da hegemonia severinista e marcelinista, era composto em 1901 por 29 ``doutores'', 10 ``coroneis'', 3 ``batinas'',  4 ``nobres'', 2 ``comendadores'', 1 ``professor'' e 1 ``senador'' \cite[p.~49]{sampaio_partidos_1978}.} e que aglutinou inclusive seus antigos opositores no Senado federal. Opuseram-se a Seabra de início alguns latifundiários no Recôncavo, no Sul, no vale sanfranciscano e nas Lavras Diamantinas; Seabra optou por não intervir nas disputas locais entre os latifundiários, apoiando em seguida o vencedor (ainda que se tratasse de adversário seu) em busca da construção de bases sólidas em meio a esta classe social. Sua busca por uma governança unipartidária culminou com a promulgação da Lei Estadual 1.102, de 11 de agosto de 1915: uma reforma municipal que mudou a forma de acesso às intendências municipais (como então eram chamadas as prefeituras): antes eletivas, as 141 intendências baianas passaram a ser ocupadas por pessoas nomeadas pelo governador, no caso o próprio Seabra. Com tal mecanismo, o poder dos latifundiários encontrou-se de repente ameaçado, podados que foram de sua fonte de poder: o controle sobre as eleições municipais. Fez assim Seabra a base de poder com que construiu a eleição de seu sucessor, Antônio Ferrão Moniz de Aragão, e elegeu-se deputado federal para assim circular novamente nos meios políticos do Rio de Janeiro. Tratava-se de uma derrota para os latifundiários e uma vitória para os burgueses e gestores, base social da hegemonia seabrista.

Ocorre que Seabra, já em 1912, se desentendera com Luís Vianna, aliado de ocasião em sua ascensão ao poder estadual, em torno do processo de reconhecimento de poderes nas eleições a deputado federal naquele ano, da subsequente escolha da liderança da bancada federal baiana (o senador Vianna apoiando Joaquim Pires Moniz de Carvalho contra a indicação de Mário Hermes da Fonseca, filho do presidente e eleito deputado federal pela Bahia com a força e o apoio de Seabra), e de um empréstimo externo ao Estado; Vianna tornou-se minoritário durante os seabrismo, angariando poucos apoios, como o de Simões Filho. Na esfera federal, afastara-se Seabra do porto seguro situacionista ao afrontar o senador gaúcho Pinheiro Machado, eminência parda do presidente Hermes da Fonseca; o enfrentamento entre Seabra e Vianna e o posterior expurgo deste último da seção baiana do Partido Republicano Conservador (PRC) promovida por Seabra desagradara o sul-riograndense, e o apoio de Seabra à candidatura presidencial de Ruy Barbosa desde as negociações precedentes às eleições de 1914 selara seu destino: oposição ao governo federal.

Para piorar, o mandato de Antônio Moniz, sucessor de Seabra no governo da Bahia, foi um verdadeiro descalabro, em parte devido à inabilidade do governador em negociar e transigir, como tanto fizera seu mentor. Os oposicionistas ao seabrismo viram ser paulatinamente bloqueados os canais de diálogo junto ao governo estadual abertos por Seabra, e portanto perderam o acesso às obras e cargos públicos que não apenas garantiam-lhe o prestígio, como também impactavam positivamente a economia local; por isso enfrentaram, inclusive de armas na mão, o que consideravam ``abusos'' do governador. No vale do São Francisco, primeiro exemplo dos conflitos do período, Franklin Lins de Albuquerque deu início, em janeiro de 1918, a enfrentamentos armados contra José Correia de Lacerda, Antônio Joaquim Correia e Adolfo Gomes de Queiroz; vencidos os três, enviou o governo a Força Pública\footnote{Diga-se de passagem que no Brasil da Primeira República as polícias estaduais eram tropas via de regra mal pagas e mal equipadas, postas em ação ao arbítrio dos governadores para subjugar seus adversários. No caso baiano, tratava-se de um contingente pequeno (média de 10 praças por município e 400, incluindo oficiais, estacionados em Salvador), mal-pago (o salário de uma praça, 1\$600, equivalia ao de um servente de pedreiro) e mal-equipado, e além do mais responsável pela segurança num Estado de grande tamanho vitimado pela precariedade de sua malha viária, tornando-a portanto incapaz de fornecer uma resposta tática rápida às situações a que era enviada para intervir \cite[pp.~46-47]{sampaio_legislativo_1985}.} para conter a revolta, que durou quase um ano e espalhou-se por diversas cidades no vale sanfranciscano, interrompendo a navegação fluvial e paralisando o comércio. Em Carinhanha, a indisciplina da Força Pública baiana -- era comum que os policiais compensassem os constantes atrasos em seus salários minguados com saques a lojas e fazendas dos opositores -- voltou-se contra João Duque, latifundiário com ligações também em Minas Gerais, de onde conseguia apoio por meio de incursões constantes da polícia mineira contra os policiais baianos; os opositores do seabrismo cedo transformaram as incursões policiais contra Duque numa bandeira de batalha, acusando Antônio Moniz de perseguição. Nas Lavras Diamantinas, a trégua armada de 1915 foi quebrada quando jagunços do pecuarista Militão Rodrigues Coelho, seabrista, invadiram o território de Horácio de Matos, latifundiário e comerciante de diamante e carbonato; o governador Moniz interveio em favor de Militão enviando a Força Pública, derrotada e desmoralizada pela jagunçada de Horácio de Matos em sucessivos combates. O governador Moniz viu-se forçado a ceder: ao apertar o cerco institucional contra os coroneis interioranos por meio da Lei Estadual 1.104, de 9 de maio de 1916, que obrigou os intendentes a remeter ao governador até o dia 15 de janeiro de cada ano cópia autêntica do orçamento municipal em vigor e quebrou, deste modo, o caráter absoluto da ingerência latifundiária sobre os assuntos fiscais municipais, Moniz distribuiu o quanto pôde de cargos públicos interior afora, proliferando o ingresso de representantes e protegidos políticos na burocracia estatal como juízes substitutos, coletores estaduais ou federais, diretores de agências de correios, inspetores de saúde etc., além da multiplicação das professoras, esposas ou filhas de latifundiários locais assim nomeadas em troca da concórdia política.

Nâo por acaso foi neste período que se deu a unificação das oposições ao seabrismo. Os latifundiários sertanejos, via de regra inseridos nos setores marginais da economia baiana voltados ao mercado interno, já se encontravam em pé de guerra contra o governo estadual. A burguesia comercial baiana, severamente impactada pela queda nas exportações de fumo e cacau durante a Primeira Guerra Mundial (1914-1918), angustiava-se com a a incapacidade do governador Moniz de influenciar o governo federal em seu favor, como no caso da apreensão pela marinha britânica de 1908 sacas de cacau e 1600 sacas de café despachadas a Copenhaga em 1915. Setores significativos dos jovens bachareis -- como Pedro Lago, os irmãos Calmon, Simões Filho e os irmãos Octavio e João Mangabeira -- bandeavam-se para a oposição, pela notória incapacidade do governador Moniz de entabular com eles qualquer diálogo conciliador. Fundamental para a articulação entre oposicionistas sertanejos e urbanos foi Manoel Alcântara de Carvalho, comerciante de diamantes, fazendeiro, homem de letras e correspondente de \textit{A Tarde} nas Lavras Diamantinas: foi ele quem pôs em contato Horácio de Carvalho com Pedro Lago e João Mangabeira, deslocados até as Lavras Diamantinas para acertar planos com os latifundiários antisseabristas, acertar a compra de armas e munição e tentar convencer -- com relativo sucesso -- os policiais a unir-se à revolta \cite[p.~201]{CUNHA2011}; foi também ele quem articulou a ocupação das vilas de Poções e Boa Nova (distritos de Vitória da Conquista) e das cidades de Jequié e Jaguaquara por Marcionílio de Souza, visando assim estabelecer uma cabeça-de-ponte na estação final da Estrada de Ferro de Nazaré e assim ameaçar diretamente o Recôncavo e a capital \cite[p.~202]{CUNHA2011}. Estava aberta, às vésperas da eleição para o governo baiano em 1919, a guerra pelo Estado; objetivava a oposição forçar uma intervenção federal e com isto jogar por terra as esperanças de Seabra de eleger-se mais uma vez governador.

Seabra, do Senado no Rio de Janeiro onde ocupara desde 1917 a vaga do falecido José Marcelino, via o ladeira-abaixo em que se convertera a política baiana e operava em sua própria causa. Apoiou a candidatura de Epitácio Pessoa à presidência em 1919 para garantir o apoio federal à sua reascensão ao governo baiano -- gesto imitado pelos oposicionistas quando da indicação, com o beneplácito de Ruy Barbosa, do nome de Paulo Fontes como candidato das ``classes conservadoras''. A revolta dos latifundiários do sertão alastrava-se, e seus jagunços preparavam-se para marchar rumo a Salvador; a notícia desta rebelião aterrorizou os blocos de poder noutros Estados, receosos de que o sucesso do levante sertanejo acendesse a faísca ignitora de sedições semelhantes noutros lugares e abalassem, portanto, as frágeis hegemonias; organizaram-se para pressionar o presidente Epitácio Pessoa a sair de sua postura não-intervencionista e intervir militarmente na situação baiana. Em fevereiro de 1920 Pessoa decretou estado de sítio em toda a Bahia, e tropas federais comandadas pelo general Cardoso de Aguiar concentraram-se em Bonfim, Cachoeira e Castro Alves; em seguida a negociações tensas com Horácio de Matos, Marcionílio de Souza e Anfilófio Castelo Branco, onde empregou até ameaças de uso da aviação militar contra os insurretos, o general mediou a assinatura de um documento que só não leva o nome de ``tratado de paz'' ou ``armistício'' por puro preciosismo (afinal, não se tratava de guerra entre Estados soberanos, mas entre cidadãos do mesmo país). Venceu Seabra, eleito enfim governador; venceram os latifundiários sertanejos baianos, que pela força das armas saíram da posição de solicitantes à de solicitadores e de colaboradores em pé de igualdade com o governo estadual, e de chefetes locais passaram a interlocutores diretos com o governo federal (por meio dos deputados que o ``armistício'' facultara-lhes indicar). Perdeu Seabra, forçado a dividir o poder que desejava concentrar e, mais uma vez, arrostado à oposição ao governo federal por indispor-se com Epitácio Pessoa, com cuja eleição colaborara; perderam os latifundiários sertanejos, pois galgaram melhores posições, aumentaram seus respectivos territórios, mas pouco a pouco retornaram ao relativo isolamento de seus domínios.




A carreira política de Seabra na Primeira República seria marcada ainda por outro revés: candidato à vice-presidência na chapa presidencial encabeçada por Nilo Peçanha e apoiada pelos blocos hegemônicos na Bahia, Rio Grande do Sul, Pernambuco e Rio de Janeiro, foi derrotado pela chapa encabeçada por Artur Bernardes e apoiada por todos os demais Estados brasileiros. Seabra apoiou a revolução de 1930 na esperança de sair do ostracismo em que se encontrava depois de sua derrocada em 1924 e foi um dos dois parlamentares a assinar as duas primeiras constituições brasileiras (1891 e 1934), mas já então encontrava-se na oposição ao regime varguista.

\subsection{A consolidação calmonista (1924-1930)}