\section{A inserção brasileira no contexto internacional}\label{sec:insbrascontint}

É tempo, agora, de entender a inserção brasileira num contexto internacional de imperialismo, guerras, trustes e carteis. Nesta escala, já é possível analisar mais cerradamente a formação social e analisar, ainda que superficialmente, sua estrutura de classes, para, posteriormente, verificar a inserção da sociedade soteropolitana neste quadro.

\subsection{Da República da Espada (1889-1894) à República do Café-com-Leite (1894-1930)}\label{subsec:espadaleite}

A proclamação da república no Brasil (1889) resulta não apenas das questões \textit{religiosa}\footnote{Costuma-se dar este nome a uma série de conflitos ocorridos entre 1873 e 1876 entre o clero e a maçonaria, de um lado, e entre o clero e a instituição regalista do \textit{padroado}, de outro; ambos podem ser enquadrados na \textit{reação ultramontana católica} iniciada no papado de Gregório XVI (1831-1846) e continuada no papado de Pio IX (1846-1878), especialmente por meio da encílica \textit{Quanta Cura} e seu infame anexo \textit{Sílabo dos Erros} (1864) e das posturas mais duras do Concílio Vaticano I (1869-1870), como resposta às revoluções liberais e ao secularismo. O conflito do clero com a maçonaria já se antecipava enquanto ordem papal em \textit{Quanta Cura} e no \textit{Sílabo dos Erros}, ambos contrários à liberdade de consciência e ao primado da razão; restou que Vital de Oliveira, bispo de Olinda, e Macedo Costa, bispo do Pará, acendessem o pavio aplicando tais doutrinas a seu pastorado, proibindo maçons em irmandades católicas, punindo padres maçons e engajando-se em polêmica impressa contra a maçonaria. O caso chegou até à Coroa, pois o regalismo instituído pelo padroado facultava ao imperador brasileiro interferir em assuntos clericais -- na prática, a igreja era quase totalmente submissa à Coroa, fato condenado tanto em \textit{Quanta Cura} quanto no \textit{Sílabo dos Erros}. Com a subsequente prisão dos bispos por desobedecerem à ordem imperial de suspender as sanções religiosas que haviam imposto aos maçons, a questão tomou vulto, transformou-se em transtorno diplomático com o Vaticano, resolvido com a absolvição imperial dos bispos em 1876, passando assim a Pedro II a imagem de ``submisso ao Papa'' tão fortemente aproveitada pela campanha republicana então nascente.}, \textit{militar}\footnote{Entre 1884 e 1887, uma série de incidentes envolvendo o tenente-coronel Antonio de Sena Madureira e o coronel Ernesto Augusto da Cunha Matos em questões que iam desde protestos quanto à contribuição obrigatória para o montepio militar ou o afastamento de oficiais acusados de corrupção geraram intensa polêmica impressa, resultando na proibição, por parte do Ministério da Guerra, de qualquer manifestação de militares através da imprensa. A mordaça gerou insatisfação na caserna, especialmente na Escola Militar da Praia Vermelha, onde já floresciam a filosofia positivista e o republicanismo.} ou mesmo da questão \textit{sucessória}\footnote{Uma vez que Pedro II teve apenas filhas como herdeiras e a constituição brasileira de 1824 instituíra a sucessão semi-agnática, que não exclui herdeiras do processo sucessório, Isabel era a herdeira do trono brasileiro; por ser casada com Luís Filipe Maria Fernando Gastão, conde d'Eu, tido como largamente impopular em razão de sua nacionalidade francesa, sua futura ascensão ao trono criou entre as classes populares, a classe média, os militares e outros a má expectativa de serem governados por um estrangeiro.}; por importantes que sejam estas questões como expressão das contradições e conflitos sociais do último período do Império, foi fundamentalmente da crise aberta pela \index{abolição da escravidão}\textit{abolição da escravidão}, como corolário da degenerescência do regime escravista, que resultaram os problemas sociais e políticos que levaram à derrocada do Império. 

Foi, na verdade, nos anos 1860 que começaram a se acumular fatores contrários à sustentação do regime escravista: a crise econômica dos anos 1860, causada pelo \textit{declínio nos preços do café} (principal pauta de exportação brasileira na época); a \textit{crise financeira de 1864}; a \textit{vitória dos Estados antiescravistas na Guerra de Secessão estadunidense}, com o consequente debilitamento dos Estados escravocratas (Brasil e Cuba) perante a opinião pública internacional; a \textit{Guerra do Paraguai}, onde massas de recém-libertos incorporadas à tropa foram tomadas pelas ideias de liberdade e insuflaram-nas entre a oficialidade; o \textit{declínio da população escravizada} e as \textit{migrações internas de escravos}, especialmente do Norte-Nordeste, para as regiões cafeeiras; tudo isto, enfim, resultou não apenas numa cúpula ministerial favorável à abolição, mas também ao florescimento de uma opinião pública também abolicionista, e ao surgimento das primeiras associações dedicadas à propaganda anti-escravista e à coleta de donativos para compra de alforrias \cite[p.~141-143]{gorender_escrareab_1990}. É o momento em que a rebelião negra contra a escravidão, afogada pela maré montante da repressão no início do período, assume ao seu final novas formas e se intensifica; é de igual modo momento do dealbar, na cena política e social, de uma ``classe média'' urbana -- as aspas serão explicadas na \autoref{subsubsec:clamed} (p. \pageref{subsubsec:clamed}) -- patrocinadora de um \textit{movimento abolicionista} radicalizado, promotor não só da cotização para alforrias, mas igualmente de fugas individuais e coletivas de escravos \cite[p.~267-336]{saes_estadoburgues_1985}.

A \textit{abolição da escravidão} (1888) e a \textit{proclamação da república} (1889) fazem parte de um só processo de conflitos sociais no Império, em que a fração então hegemônica dos latifundiários enfrentou não apenas os escravos rebeldes, mas de igual modo uma ``classe média'' urbana estreante no cenário político e frações dos latifundiários com pouco acesso às esferas de poder político e de controle de recursos; entender os dois processos separadamente implica numa separação injustificada entre entre uma esfera econômica e uma esfera política que só se podem compreender juntas. E muitas das contradições e conflitos sociais da Primeira República foram ensaiados já aqui, nos últimos anos do Império.

O período imediatamente posterior à proclamação da república no Brasil, conhecido como \textit{república da espada}, foi na verdade a primeira ditadura militar republicana, convulsionada por agitações políticas de todos os tipos. Em disputa, não somente projetos políticos, mas o poder, e, em última instância, mesmo o regime.

A crônica da época diz que o golpe militar responsável pela proclamação da república foi articulado por um grupo de jovens oficiais sem muita inserção entre a base da tropa e sem maior articulação com o oficialato superior, convocado à última hora para a ação \cite[p.~16]{cardoso_govmil_1977}; diz-se inclusive, como Aristides Lobo, que o povo assistiu ``bestializado'' a tudo aquilo, e que para o grosso da população brasileira, eminentemente rural e alijada dos processos políticos, ``a mudança do regime político a afetara tanto quanto a morte de um gato na China'' \cite[p.~43]{basbaum_histsinc_1967}. Frágil como fosse, a proclamação da república abriu um período de rearranjo das bases e das forças sociais hegemônicas do país. 

Os governos de \textit{Deodoro da Fonseca} e \textit{Floriano Peixoto} foram verdadeiras ditaduras militares, conhecidos para a História como \textit{República da Espada}. Deodoro governou pouco; Floriano pensava em construir um governo estável, acima das disputas locais, estaduais e regionais, cooptando quadros nas escolas civis e militares. Teria tudo para ser ferrenho adversário dos latifundiários, mas rapidamente surgiu uma aliança entre Floriano e os cafeicultores organizados no Partido Republicano Progressista (PRP), pois ambas as partes percebiam os riscos que corria a jovem república e viam-se reciprocamente como garantidores do novo regime: os latifundiários viam em Floriano a única possibilidade de garantir a sobrevivência do regime contra as forças centrífugas já então em pleno curso\footnote{Algumas destas forças centrífugas serão vistas adiante, na \autoref{subsec:clapolprire} (p. \pageref{subsec:clapolprire}); outras, como os \textit{monarquistas}, eram muito mais um espantalho a agitar nas peças de propaganda que uma ameaça real, tanto assim que deixaram de ser uma força política relevante já na década de 1910 \cite{CARONE1970inst,janotti_subversivos_1986}.}, e Floriano via nos cafeicultores paulistas uma base de sustentação sobre a qual estruturar o projeto de um Estado forte. Não que se gostassem: \textit{aturavam-se} apenas, cada parte querendo avançar seus projetos políticos às custas da outra, como o episódio da passagem do poder de Floriano a \textit{Prudente de Morais} bem o exemplifica\footnote{``No dia 15 de novembro Prudente, trajado de acordo com o protocolo, aguardou, no hotel, que o viessem buscar. Só apareceu André Cavalcanti, convidado para Chefe de Polícia do seu Governo. Esperaram. Quando se convenceu de que não vinha ninguém, pediu ao amigo que se desse ao incômodo de ir ao Largo do Machado buscar condução. Veio o fiacre que ele conseguiu, um calhambeque em péssimo estado, o cocheiro mal enjambrado, e duas pilecas maltratadas. Foi nesse veículo sem pompa que o novo Presidente se transportou para o velho Palácio do Conde dos Arcos, onde prestou o compromisso legal. Acabada a cerimônia, o representante da Inglaterra, sabendo que o Presidente da República estava sem condução, ofereceu a sua esplêndida carruagem. Nela, Prudente se transportou para a sede do Governo. [\dots] O Palácio havia sido abanbdonado e entregue à discrição do público. O Itamarati recebeu o Presidente de portas abertas e salões vazios. Não apresentava o aspecto de uma casa de governo. Nâo havia uma mesa de trabalho, uma estante de livros, a menor demonstração de vida burocrática. [\dots] Seguido de poucas pessoas, esguio e solene, \textit{um doloroso sorriso} em meio àquele cenário, Prudente atravessa as dependências descuidadas. Na grande sala dos fundosm dando para o parque, jazia sobre o assoalho de custoso mosaico de madeira, um caixão aberto, contendo jornais, papeis rasgados, garrafas vazias de cerveja e a palha que as envolvera. Os estofos de alguns móveis foram rasgados a pontaços de baionetas. Era inacreditável! Só então apareceu alguém do mundo oficial: era o Sr. Cassiano do Nascimento, ministro de quase todas as pastas do Governo Floriano. Fez um pequeno discurso dizendo que, em nome do Vice-Presidente fazia a transmissão do Governo. Depois disso, em consequência de uma pergunta de Prudente, o grupo dirigido por aquele ministro se encaminhou para ume pequena sala à esquerda que servia para o despacho presidencial. Então o Sr. Cassiano despediu-se e se retirou. Prudente instalava-se na Presidência da República do modo mais informal'' \cite[pp.~239-241]{silva_republica_1972}.}.

Em defesa da república, surgiram durante o governo Floriano Peixoto os ``jacobinos'' de 1893-1897, agrupados em torno de jornais como \textit{O Jacobino} e \textit{O Nacional}: gente como Júlio de Castilhos, Francisco Glicério, Deocleciano Martyr, Aníbal Mascarenhas e outros. Agitadores políticos profissionais, autoritários, anticlericais, defensores de medidas nacionalistas (tarifas de proteção à indústria e nacionalização do solo) e protetivas dos trabalhadores (como a jornada de oito horas e a regulamentação dos alugueis para operários), americanófilos e antilusitanos, atuavam ameaçando de morte os inimigos, intimidando-os com a publicação de seus nomes na sua imprensa (de longe a mais radical do período), provocando confrontos de rua, agitando o povo para depredações, insuflando ataques a portugueses (que tratavam, sem mais, como monarquistas) etc. \cite{queiroz_radicais_1986}\dots Tinham como base social principal o pequeno funcionalismo público das cidades e os militares de baixa e média patente. Com a perseguição aos suspeitos de envolvimento no atentado contra o presidente Prudente de Morais (5 nov. 1897), o movimento perdeu força e terminou dissolvendo-se em poucos anos.

Prudente de Morais deu início à fase civil da primeira república brasileira e terminou seu mandato aclamado, popularíssimo. Mas foi seu sucessor, \textit{Campos Salles}, quem estabeleceu o mecanismo de ajuste político por que ficaria conhecido todo o período restante: era a \textit{política dos governadores}. 



DESENVOVER, EM SÍNTESE APERTADA A POLÍTICA DOS GOVERNADORES - funcionamento, altos e baixos 


DESENVOLVER EM SÍNTESE APERTADA, USANDO A Periodização de Edgar Carone: apogeu (Prudente de Moraes, Campos Sales, Rodrigues Alves, Afonso Pena), abalos (Hermes da Fonseca, Wenceslau Braz), contestações (Epitácio Pessoa, Artur Bernardes, Washington Luiz) \cite{carone_evolucao_1977}



Aquilo da política dos governadores que interessa a esta pesquisa, mais que a longa narrativa de seu apogeu e declínio, é seu móbil: os \textit{conflitos entre diferentes frações regionais da classe dos latifundiários} e os \textit{conflitos entre latifundiários exportadores e latifundiários voltados para o mercado interno}. As diatribes, catilinárias e polêmicas entre deputados, senadores e governadores nas sessões parlamentares e na imprensa, os conflitos armados entre coroneis, a disputa em torno das tarifas alfandegárias, tudo isto costuma ser debitado na conta de um certo sistema \textit{regionalista} de poder em que São Paulo e Minas Gerais exerceriam preponderância em nível nacional

DESENVOLVER O ARGUMENTO DE PERISSINOTO, NELSON WERNECK SODRÉ E BORIS FAUSTO; CONFLITOS REGIONAIS COMO CONFLITO ENTRE DIFERENTES FRAÇÕES REGIONAIS DOS LATIFUNDIÁRIOS, DIVIDIDOS ENTRE EXPORTADORES E PRODUTORES PARA O MERCADO INTERNO

\subsection{O Brasil, a banca internacional, o imperialismo}\label{subsec:brasimper}

A dinâmica mundial do capitalismo, ao que se viu na literatura consultada, não parece ter desempenhado papel decisivo na história brasileira antes e durante as fases iniciais da Primeira República porque a integração da economia brasileira na divisão internacional do trabalho era muito parcial e fragmentária: o setor de mercado externo compunha-se de uma série de manchas no mapa do país, articuladas com o exterior por meio de uma economia urbana incipiente e centrada em cidades portuárias precariamente interligadas; entre as manchas de produção primária para exportação e as cidades que a polarizavam havia um mundo semifechado, quase autárcico, de fazendas, estâncias, pequenas propriedades, vendas, mascates, tropeiros e pequenas cidades dificilmente afetadas pelo ``exterior'' \cite[p.~35o]{singer_braecomu_1977}.

A abolição da escravatura e a proclamação da República criaram o quadro institucional adequado para a crescente integração do Brasil na economia capitalista mundial e colocaram o Brasil em posição de maior destaque na divisão internacional do trabalho, com um crescimento de 31,6\% nas exportações brasileiras entre 1880 e 1900 e de 63,7\% na primeira década do século XX \cite[p.~352]{singer_braecomu_1977}. Esta maior inserção, entretanto, se deu ainda no papel de \textit{fornecedor de matérias-primas e de produtos agrícolas}, especialmente café (o principal produto da pauta de exportação brasileira), açúcar, algodão, borracha e derivados do couro. 

\begin{table}[!htp]
\IBGEtab{
\caption{Brasil, principais produtos de exportação, 1889-1929 (em \%)}\label{tab:exportabrasil}}
{
\begin{minipage}{21cm}
\begin{tabular}{cccccccccc}
\hline
Períodos & Café & Açúcar & Cacau & Mate & Fumo & Algodão & Borracha & Couros/Peles & Outros \\
\hline\hline
1889-1897 & 67,8 & 6,5 & 1,1 & 1,2 & 1,7 & 2,9 & 11,8 & 2,4 & 4,8 \\
1898-1910 & 52,7 & 1,9 & 2,7 & 2,7 & 2,8 & 2,1 & 25,7 & 4,2 & 5,2 \\
1911-1913 & 61,7 & 0,3 & 2,3 & 3,1 & 1,9 & 2,1 & 20,0 & 4,2 & 4,4 \\
1914-1918 & 47,4 & 3,9 & 4,2 & 3,4 & 2,8 & 1,4 & 12,0 & 7,5 & 17,4 \\
1919-1923 & 58,8 & 4,7 & 3,3 & 2,4 & 2,6 & 3,4 & 3,0 & 5,3 & 16,5 \\
1924-1929 & 72,5 & 0,4 & 3,3 & 2,9 & 2,0 & 1,9 & 2,8 & 4,5 & 9,7 \\
\hline
\end{tabular} 
\end{minipage}
}
{ \fonte{Elaboração do autor, com dados de \citeonline[p.~63]{suzigan_polgov_2001}.} }
\end{table}

\begin{table}[!htp]
\centering
\IBGEtab{
\caption{Principais parceiros do Brasil no comércio internacional 1853-1928}\label{tab:comerbras}}
{\begin{tabular}{ccccccccc}
\hline
\multicolumn{9}{c}{Participação em \% no comércio exterior do Brasil} \\
\hline & \multicolumn{2}{c}{Grã-Bretanha} & \multicolumn{2}{c}{Alemanha} & \multicolumn{2}{c}{Estados Unidos} & \multicolumn{2}{c}{França} \\
\cline{2-9} Datas & Exp. & Imp. & Exp. & Imp. & Exp. & Imp. & Exp. & Imp. \\
\hline\hline
1853/4 a 1857/8 & 32,9 & 54,8 & 6,0 & 5,9 & 28,1 & 7,0 & 7,8 & 12,7 \\
1870/1 a 1872/4 & 39,4 & 53,4 & 5,9 & 6,5 & 28,8 & 5,4 & 7,5 & 12,2 \\
1902 a 1904 & 18,0 & 28,1 & 15,0 & 12,2 & 43,0 & 11,5 & 7,8 & 8,8 \\
1908 a 1912 & 17,0 & 27,5 & 14,3 & 16,2 & 38,2 & 13,5 & 8,6 & 9,4 \\
1920 & 8,2 & 21,4 & 5,8 & 4,6 & 42,0 & 40,6 & 12,0 & 5,4 \\
1928 & 3,4 & 21,0 & 11,0 & 12,3 & 44,6 & 26,2 & 9,0 & 6,2 \\
\hline
\end{tabular} }
{ \fonte{Artigo ``O Brasil no contexto do capitalismo internacional 1889-1930'', de Paul \citeonline[p.~369]{singer_braecomu_1977}} }
\end{table}

Para piorar, a fase monopolista do capitalismo, já detalhada na \autoref{sec:impercol}, implicava em enorme integração vertical e horizontal dos conglomerados empresariais, e igualmente de preferência pelos produtos destes conglomerados nas ``zonas de influência'' de seus países de origem; por isto, à exceção do café, para cuja produção o Brasil tinha características ecológicas excelentes, todos os demais produtos da pauta de exportação encontravam ou a concorrência de similares produzidos nas áreas de atuação dos conglomerados imperialistas (p. ex., açúcar de beterraba), ou o obstáculo de taxas aduaneiras protecionistas.

Percival Farquhar, conhecidíssimo capitalista estadunidense atuante no Brasil durante a República Velha e depois \cite[pp~28, 35-41, 53-54, 60-61, 76, 83-84, 121, 150-152]{CUNHA2011}, praticamente delineou, num artigo publicado em meio à guerra, o programa da ação dos investidores estrangeiros no país:

\begin{citacao}
Os notáveis investimentos na América do Sul serão, naturalmente, em estradas de ferro; serviços públicos urbanos; desenvolvimento da energia hidrelétrica; propriedades cujos produtos sejam consumidos nos Estados Unidos; títulos da dívida do governo federal, dos governos estaduais e dos municípios. \cite[p.~398]{farquhar_invest_1916}
\end{citacao}

Com algumas variações, este foi, na verdade o programa de atuação de todos os investidores estrangeiros no Brasil durante a Primeira República -- e o próprio Farquhar, por agir nas bancas de Londres, Paris e Bruxelas em busca de capital para seus empreendimentos, pode ser tomado como personagem-síntese desta atuação.

No que diz respeito aos \textit{empréstimos tomados pelo Brasil junto à banca internacional}, se apenas dois dos dos 17 empréstimos tomados durante o Império se destinaram a investimentos em infraestrutura (estradas), após 1890 passam a se destinar majoritariamente a obras públicas (construção de portos e ferrovias) ou à sustentação das cotações externas de café. As dificuldades na quitação destas dívidas, comuns no Império, persistiram nas primeiras décadas da República: os dois \textit{funding loans} (1898 e 1914) foram pactuados pelo governo federal com a banca internacional mediante a cessão a exigências draconianas por esta última \cite[p.~365]{singer_braecomu_1977}.

Os \textit{britânicos} foram os principais beneficiados pela abertura dos portos brasileiros em 1808, e desde então dominaram o comércio externo e a banca brasileira.

DSENVOLVER USANDO A TABELA DE RIPPY

A contração gradual começou por volta de 1928-1929 com a venda de diversos serviços públicos para corporações controladas por capitalistas estadunidenses, iniciando uma tendência sem retorno \cite{rippy_britlat_1954}.

No que diz respeito ao capital de origem \textit{francesa}, se os empresários da França se faziam presentes no Brasil desde há muito, foi só durante a Primeira República brasileira que os investimentos franceses floresceram, como parte de uma tendência geral para investimento francês na América Latina (concentrado no Brasil, Argentina e México). Entre 1902 e 1914, os investimentos franceses na América Latina duplicaram, e os investimentos no Brasil passaram de 20\% a 42\% do total; além disso, em 1913 os investimentos diretos (ou seja, em empresas) ultrapassaram os empréstimos públicos, que até 1902 sempre haviam sido muito maiores em comparação \cite[p~83-84]{mauro_empfran_1999}. Entre 1904 e 1913 o Brasil foi o maior cliente da banca francesa na região, e o segundo em escala mundial \cite{rippy_french_1949}. Houve três destinos principais para os investimentos franceses: \textit{(a)} os \textit{portos}, em especial os de Recife, Porto Alegre e Rio de Janeiro; \textit{(b)} as \textit{ferrovias}, com especial destaque para a criação de seis companhias francesas específicas no setor e da  (1910), com capital de 3 milhões de francos, para a construção de ferrovias na Bahia; \textit{(c)} os \textit{bancos} \cite[p~84]{mauro_empfran_1999}, que merecem destaque. O \textit{Banque Française au Brésil}, fundado em 1872 com capital de 10 milhões de francos, tornou-se lucrativo a partir de 1880, e mais ainda depois de 1900; o sistema financeiro francês, entretanto, ainda era insuficiente, o que levou à criação em 1909 do \textit{Banque Française et Italienne por l'Amérique du Sud} (conhecido posteriormente como \textit{Banco Sudameris}) \cite[p~84]{mauro_empfran_1999}. Entre um e outro, foram criados também o \textit{Banque Nationale du Brésil} (1893) e o \textit{Crédit Foncier du Brésil et de l'Amérique du Sud} (1907), este último tendo especial relevo nas muitas reformas urbanas realizadas no Brasil da Primeira República. Os investimentos franceses no período gozaram de alta rentabilidade (taxas anuais de 5\%, com retorno rápido e prazos de amortização superiores a 35 anos), mas após a Primeira Guerra Mundial o franco se desvalorizou frente à libra esterlina, levando a uma conflituosa redução da dívida que só foi resolvida quando a Corte Internacional de Haia obrigou os devedores brasileiros a indexar a dívida segundo o franco-ouro \cite[p.~87]{mauro_empfran_1999}. Mesmo assim, em 1922 já havia 4 bilhões de francos investidos no Brasil\footnote{Distribuídos da seguinte maneira: 2,5 bi para empréstimos públicos, 1,25 bilhão para ferrovias, 170 milhões para bancos e 138 milhões para a indústria \cite[p~84]{mauro_empfran_1999}.}. Até 1930, Pierre Louis Marcel Bouilloux-Lafont era o mais importante capitalista francês a se relacionar com investidores e com o Estado brasileiro. Dono da \textit{Caisse Commerciale et Industriale} (fund. 1907), banco especializado em empréstimos estrangeiros, veio ao Brasil em 1909 para assumir a construção do porto de Salvador, e em 1911 conseguiu decreto autorizador do funcionamento da sua \textit{Societé Franco-Sud-américaine de Travaux Publics} no ramo da construção de estradas de ferro no Brasil; os 326 milhões de francos do grupo Boilloux-Lafont investidos no Brasil em 1914 representavam 10\% do total do investimento francês no país \cite{somogyi_lafont_1990}. É de se notar também a atuação na Bahia do barão Amédée Reille-Soult-Dalmatie \cite[pp.~116-120]{CUNHA2011}, político e banqueiro francês.

No que diz respeito ao \textit{capital de origem germânica}, as estatísticas divergem em alguns aspectos, mas é certo que a migração alemã para a América Latina não tomou vulto antes da década de 1850 \cite[p.~65]{rippy_german_1948} e só se tornou realmente significativa entre os anos 1880 e 1910 (cf. \autoref{tab:imigra}). Os primeiros migrantes foram responsáveis pelos primeiros empreendimentos comerciais de países latino-americanos com países de língua alemã (antes mesmo da unificação) e por atrair investimentos germânicos em terras, pecuária, imóveis, suprimentos agrícolas, cervejarias, hoteis e estabelecimentos mercantis. Havia na América Latina inteira em 1918 pelo menos 1.019 empresas com capital alemão, mobilizando US\$ 677 milhões \cite[p.~64-65]{rippy_german_1948}\footnote{As cifras estão em dólares, unusualmente para as estimativas do período, porque retiradas de \textit{blacklists} estadunidenses de empresas com participação alemã, usadas pelo governo dos EUA durante a Primeira Guerra Mundial como instrumento político para convencer governos latino-americanos a expulsar de seus respectivos países, ou ao menos de romper negócios e contratos pré-existentes.}. Ainda antes da Primeira Guerra Mundial, havia um pequeno circuito bancário alemão -- pequeno quando comparado com os circuitos britânico e francês -- constituído na América Latina: \textit{Banco Aleman Transatlantico}, \textit{Banco Germanico de la América del Sud}, \textit{Brasilianische Bank für Deutschland}, \textit{Banco de Chile y Alemania}, \textit{Banco Antioqueña}, \textit{Banco Mexicano de Comercio e Industria} e \textit{ZentralAmerika Bank}. Havia também outra especialidade alemã na América Latina, as \textit{empresas de navegação}: \textit{Hamburg-American}, \textit{Hamburg-South American}, \textit{North German Lloyd}, \textit{Hansa}, \textit{Kosmos}, \textit{Roland}, \textit{Atlas} e \textit{Kirsten Line} \cite{rippy_german_1948}. Não obstante sua presença marcante no campo da eletricidade e da química, reais especialidades da indústria alemã no período, em outros aspectos o capital alemão na América Latina era insignificante frente aos capitais britânico e francês. Antes da Primeira Guerra Mundial o capital alemão tinha pouca participação no setor de serviços públicos, somente duas petrolíferas, menos de doze mineradoras, quatro companhias de nitratos e três ferrovias (no valor total de US\$ 25 milhões), e participação tímida na indústia de processamento de carne, na navegação fluvial e na telefonia \cite{rippy_german_1948}. No Brasil, entretanto, sua presença foi marcante no setor agrícola.



\subsubsection{Quadro geral}\label{subsubsec:quager}

Embora exista, atualmente, tendência a considerar a República Velha como um período de enorme prosperidade econômica \cite{caldeira_riqueza_2017}, é preciso cautela\footnote{Esta tendência expressa as posições de certa historiografia que vê nos empresários o único motor da atividade econômica, desconsiderando a atuação dos trabalhadores e o papel central desempenhado pelo conflito entre classes sociais no desenvolvimento econômico. \citeauthoronline{caldeira_riqueza_2017}, principal representante desta tendência, força tanto a intepretação da Primeira República como um período de enorme prosperidade econômica que chega ao absurdo de considerar Jeca Tatu -- sim, o próprio, a cria de Monteiro Lobato -- como\dots um \textit{empreendedor rural} \cite[pp.~517-518]{caldeira_riqueza_2017}!}. Se os indicadores microeconômicos e as tendências macroeconômicas parecem promissores 

O Brasil já estava em recessão em 1928, em seguida a uma queda internacional de preços agrícolas que se prolongava desde 1925. É neste contexto que as politicas de compra de excedentes inauguradas com o Convênio de Taubaté mostraram seu lado negativo: o preço mundial do café caiu de cerca de 20 centavos por libra em 1924-1926 para 15 em 1927. O aumento dos estoques brasileiros de café (que mais do que duplicaram em 1927 e absorveram quase um terço da produção) levou a um ligeiro aumento nos preços em 1928-1929, para 18 centavos por libra. Mas quando, devido à falta de recursos, o governo parou de aumentar os estoques, os preços caíram, atingindo 10 centavos no final de 1929, depois 6 em 1931, apesar do aumento maciço da estocagem governamental, que dobrou em 1930 graças a um empréstimo US\$ 100 milhões contraído com sucesso pelo Estado de São Paulo em Londres \cite[p.~25]{hautcoeur_1929_2009}\footnote{Este processo precisa ser visto em seu devido contexto. Em 1929 ``a recessão, que já havia começado em certos países (em particular na Alemanha, no Brasil e no Canadá) não inquietava; muitos políticos e economistas criam que uma nova era de crescimento permanente havia começado, na qual as crises sérias estariam excluídas'' \cite[p.~4]{hautcoeur_1929_2009}. Este seria um fator essencial a se ter em conta, pois ``numerosas análises limitam-se à crise americana. De fato, os Estados Unidos estão no coração da crise, de gravidade excepcional. As políticas conduzidas por Roosevelt exerceram influência mundial. Mas a crise se origina nos Estados Unidos, ou acreditamos nisto porque os Estados Unidos são o símbolo da prosperidade aparentemente inalterável da década de 1920 (sua taxa de crescimento econômico alcançou quase 5\% ao ano neste período)? Na verdade, vários países estavam em recessão antes dos Estados Unidos: é o caso, já em 1928, da Alemanha, da Polônia, e também da Argentina, do Canadá, da Austrália e do Brasil, o que torna indispensável considerar a dimensão internacional do fenômeno'' \cite[p.~7]{hautcoeur_1929_2009}. Esta recessão teria sido causada por uma \textit{crise global na agricultura e na produção de produtos primários} (que representavam à época 60\% do comércio mundial e mais de um terço da força de trabalho na maioria dos países), com forte queda nos preços destes produtos; a recessão seria o produto de uma \textit{superprodução} induzida por políticas governamentais (fixação de preços e compra de excedentes; proteção aduaneira; subsídios diretos ou indenizações de guerra, através das quais muitas fazendas europeias foram reconstruídas) e do mercado financeiro (que financiou a compra de terras a preços altos no imediato pós-guerra e também os investimentos necessários em maquinário e infraestrutura necessários para a exploração agrícola, amarrando seu destino, portanto, àquele do setor em que investira) \cite[pp.~22-25]{hautcoeur_1929_2009}. }. A Bolsa do Café de Santos, um belíssimo prédio onde eram feitos os pregões, registrava a tragédia em cifras: se em agosto de 1929, dois meses antes da implosão da bolsa nova-iorquina, a saca do café estava cotada no mercado internacional em 200 mil-réis, em janeiro de 1930 desabara para 21 mil-réis. A praça de Santos, o maior centro brasileiro de atividades comerciais ficou virtualmente em moratória. Sem preços, o Brasil, que possuía 60\% do mercado internacional do café, não podia exportar o produto, e acumulava grandes estoques nos diversos armazéns gerais da cidade -- o que comprometeu os preços.

\subsection{Classes sociais e política na Primeira República}\label{subsec:clapolprire}

Seria muito simples pegar a estrutura econômica brasileira e fazer derivar uma categoria profissional de cada um dos lugares na estrutura produtiva e, posteriormente, agrupá-los em classes mediante o critério da posse, propriedade ou controle dos meios de produção. 

Mas é isto suficiente?

Há vasta literatura recomendando prudência nesta caracterização \cite{aguiar_hierarquias_1974, ossowski_classes_1964, schumpeter_imperialismo_1961, velho_classes_1977}. Não há uma só entre as referências metodológicas consultadas que recomende tal simplismo empirista. É preciso, sim, entender como as classes se relacionam com seu lugar na produção econômica; entretanto, sem a força viva dos embates cotidianos e das pequenas coisas extra-laborais que fazem qualquer classe social fazer-se enquanto classe ao confrontar-se com outras (costumes, formas de sociabilidade, lazeres, produção cultural etc.), a análise terminaria manca \cite{aguiar_classe_2009, BERNARDO1991, bernardo_fascismo_2015}.

Há ainda outra questão. É comum falar-se em ``elites'' para denominar qualquer estrato social dominante em determinado tempo e lugar. Tal classificação não será empregue nesta pesquisa, primeiro por ser vaga e imprecisa; segundo, por não respeitar a longa -- e controversa -- produção sociológica acerca do assunto \cite{bottomore_elites_1965,michels_partidos_1982,mosca_elementi_1923,pareto_mind_1935,
schumpeter_capitalismo_1961} terceiro, por só fazer sentido quando inserida num contexto teórico onde as classes sociais fornecem o \textit{substrato básico} e as elites, um \textit{modo de compreensão da mobilidade social entre diferentes classes}. Uma longa citação ajudará a situar o problema:

\begin{citacao}
A referência a uma classe social só adquire sentido através da referência a uma classe oposta. A dialéctica da exploração e da opressão liga intimamente as características e a estrutura interna das várias classes, e sob este ponto de vista a luta entre as classes consiste na transformação conjunta e contraditória de todas elas. Mas não se passa o mesmo com a noção de elite, que pode ser definida de maneira independente, enquanto estrato privilegiado. A estrutura interna de uma elite nem se relaciona com a das massas, pois os teóricos das elites definem a massa precisamente pela sua incapacidade de organização própria, nem está em relação necessária com a estrutura interna de qualquer outra elite, porque a elite governa sozinha e se aparece uma nova é para liquidá-la e substituí-la. [\dots] a teoria das elites é incapaz de explicar, ou sequer conceber, esta transformação dos membros de uma elite em membros de uma classe. Os autores que pretendem que o fenómeno da mobilidade social invalida, ou pelo menos compromete, a teoria das classes e justifica a aplicação de uma perspectiva de elites estão a confundir classe com casta. É precisamente a mobilidade social que permite inserir o fenómeno das elites no quadro geral das classes, pois a formação de uma elite no interior de uma classe inferior prepara a projecção desta elite para a classe superior, alimentada periodicamente por estas novas elites [\dots] As elites só têm sentido porque são elites de uma classe ou elites de uma classe transformando-se em componentes de outra classe. O conceito de elite padece, portanto, de uma assimetria, porque as elites capitalistas continuam a ser capitalistas, enquanto as elites proletárias abandonam a sua classe de origem. [\dots] a questão decisiva é que não ocorre nenhuma conversão de uma elite numa classe. Ou as elites se formam no interior de uma dada classe exploradora ou os membros da elite da classe explorada se convertem em membros de uma classe exploradora. \cite[p.~387-388]{bernardo_fascismo_2015}.
\end{citacao}

São igualmente inaplicáveis, para os fins desta pesquisa, os conceitos clássicos de \textit{oligarquia} e \textit{aristocracia}: o primeiro, porque diz respeito a uma \textit{forma de governo} em que o exercício do poder político está restrito a um pequeno número de pessoas, não a uma \textit{classe social}, ou seja, a um conjunto de indivíduos considerados em sua posição num modo de produção ou numa formação social específicos e correlacionados a outros conjuntos de indivíduos situados noutras posições, antagônicas à sua ou não, do mesmo modo de produção ou formação social\footnote{Esta distinção foi ressaltada especialmente porque \citeonline[pp.37-45]{pang_coronelismo_1979} construiu, com base numa metodologia emprestada a Max Weber, toda uma tipologia do coronelismo em torno deste conceito, que tem influenciado estudos sobre a Primeira República desde sua primeira publicação em 1978; a metodologia empregue na pesquisa aqui exposta conflita com tal concepção do fenômeno coronelista, por razões a serem elucidadas adiante.}; o segundo, porque com a proclamação da república foram extintos os títulos aristocráticos e nobiliárquicos, tornados eletivos todos os cargos políticos pela Constituição de 1891. 

\subsubsection{Os latifundiários}\label{subsubsec:clagraris}

Dado o fato de a economia brasileira manter sua característica agroexportadora herdada do Império, são os \textit{latifundiários}, sem sombra de dúvida, uma das classes participantes do bloco político hegemônico durante a República Velha \cite{gorender_burguesia_1990,oliveira_emopro_1977,CARONE1970inst}. Há um debate em aberto acerca da natureza sociológica desta classe, em especial no período de transição do Império à República, girando em torno de ter-se ou não transformado numa \textit{burguesia agrária} por força da mudança de padrão da exploração do trabalho (da escravidão ao assalariamento) \cite{gorender_burguesia_1990,oliveira_emopro_1977}; para evitar as polêmicas, serão aqui chamados apenas de \textit{latifundiários} pelo fato de a raiz de seu poder político encontrar-se na exploração da produção agrícola em regime de \textit{plantagem}\footnote{Foi adotada aqui a denominação proposta por Jacob Gorender para o que tradicionalmente se chama \textit{plantation}. Eis a explicação, pelo próprio: ``As grandes explorações agrícolas com trabalho escravo, surgidas no continente americano à época do mercantilismo, têm sido designadas, na literatura de língua portuguesa, pelo nome de \textit{plantation}, vocábulo emprestado ao inglês e sempre impresso em itálico Mas os ingleses [\dots] tomaram o termo emprestado ao francês. [\dots] O esdrúxulo consiste em qu escritores de língua portuguesa precisem desse vocábulo estrangeiro a fim de indicar uma forma de organização econômica que Portugal teve muito antes da França e da Inglaterra (nas ilhas atlânticas) e que, no Brasil, apresentou-se sob um modelo clássico e de duração mais prolongada do que em outras regiões. Em lugar de \textit{plantation}, alguns autores empregam `plantação' ou `grande lavoura'. Ambas essas expressões linguísticas sofrem da desvantagem de carência de univocidade, prestando-se a confusões. Proponho substituir \textit{plantation}, em vernáculo, por plantagem. Não se trata aí de invenção léxica, porquanto plantagem está há muito dicionarizada. Mas, sendo vocábulo em desuso na linguagem comum e de todo ausente na literatura historiográfica e econômica, terá significação unívoca, além de dispensar o grifo e a pronúncia à inglesa \cite[pp.~119-120]{gorender_escracolo_2010}''.}. Esta classe social se apropriava do grosso do excedente econômico produzido pela agricultura de exportação sobre a qual se estruturou a economia brasileira do perído (café, açúcar, borracha, cacau, charque etc.); por dominar a economia, reunia forças para dominar também a política e a sociedade.

As esquinas da História, entretanto, são numerosas; seria erro crasso agrupar uma classe com tanta diversificação interna sem quaisquer considerações ou ressalvas. Tal a diversidade interna era proporcional à \textit{extensão do país}, à \textit{variedade da produção agropecuária} nesta mesma extensão, à \textit{preponderância de tal ou qual produto agrícola sobre a economia regional em momentos distintos do período estudado} e à \textit{destinação dos produtos finais} (mercado interno ou externo), cada qual influindo sobre a ascensão ou declínio de cada setor desta classe. Alguns exemplos indicarão as dificuldades e as possibilidades desta classificação.

No Sul do país, em especial no Rio Grande do Sul, a pecuária é a chave para a compreensão da economia; seria simples admitir os pecuaristas como classe dominante \textit{tout court}, e eliminar, assim, as diferenças entre os pecuaristas (e comerciantes associados) das regiões \textit{litorânea} (Pelotas, Rio Grande e Bagé\footnote{Embora Bagé não seja litorânea, estava incorporada na mesma dinâmica econômica das duas primeiras.}) \textit{serrana} (Vacaria, Cruz Alta), mas as encruzilhadas da História são mais complexas. Os litorâneos estiveram no poder nas últimas décadas da monarquia, consolidados pela possibilidade de os migrantes alemães naturalizados brasileiros terem garantido seu direito ao voto pela lei Saraiva, de 1881, e, tendo entre os liberais chefiados por \textit{Gaspar da Silveira Martins} (1835-1901) um instrumento político para influenciar o poder provicial e a Corte, favoreceram projetos de seu interesse, como a construção de linhas férreas, tarifas especiais para a exportação de charque e concessão de crédito para as charqueadas e estâncias; os serranos, emergentes nas últimas décadas do império, condenados à oposição pela falta de acesso ao poder provincial ou à Corte, foram -- junto com imigrantes italianos, comerciantes de Porto Alegre e militares -- a base do Partido Republicano Riograndense (PRR) de onde despontaram \textit{Júlio de Castilhos}, \textit{Borges de Medeiros} e \textit{Pinheiro Machado}. Proclamada a república, os liberais foram apeados do poder e exilatos, sendo substituídos pelos quadros dirigentes do PRR, o que deu aos serranos acesso ao poder na província e na Capital, neste último caso graças à confiança que era depositada por Floriano Peixoto sobre o PRR; os liberais, ao retornarem do exílio, formaram o Partido Federalista do Rio Grande do Sul (PF) junto com suas bases litorâneas tradicionais, com pecuaristas da Campanha gaúcha e com republicanos dissidentes. Os primeiros anos da política gaúcha sob a república foram marcados pela tensão entre estes grupos de latifundiários, maldisfarçada sob a tensão personalista entre \textit{maragatos} e \textit{ximangos} ou \textit{pica-paus}, materializada em golpes (como o que derrubou Júlio de Castilhos em 1891 e o que o reconduziu ao poder em 1892) e revoltas (como a revolta federalista de 1893/1895). Toda a política riograndense na Primeira República foi marcada pela tensão entre grupos distintos de latifundiários.

No Sudeste, embora se possa falar de maior dinamismo, maior capital excedente e maior disponibilidade dos grandes cafeicultores para novos investimentos em tecnologia ou em outros ramos econômicos (como o comércio ou a indústria) \cite[p.~153-154]{CARONE1970inst}, não se pode esquecer que nem todos os cafeicultores seguiram este padrão, restrito a uma pequena fração da classe \cite[p.~32-38]{gorender_burguesia_1990}, e que a vasta maioria dos cafeicultores, por depender do mercado externo e suas flutuações, vivia endividada e sobre suas terras sobrepunham-se seguidas hipotecas \cite[p.~154]{CARONE1970inst}. Há inclusive quem divida a classe dos latifundiários na região cafeeira em duas: a dos \textit{coroneis}, resultante de um empobrecimento contínuo e de longa data dos velhos fazendeiros do império, ``empobrecida pelas hipotecas, pelos algos juros, pela subdivisão ao infinito das terras herdadas de seus antepassados aristocratas rurais'', mas fundamental na república porque ``é ele, esse homenzinho completamente ignorado nas cidades e tão grande na sua aldeia, que faz os deputados, os senadores e os presidentes da República'' \cite[pp.~146-148]{basbaum_histsinc_1967} ; e a dos \textit{fazendeiros}, ``nova classe de proprietários mais prósperos, mais ricos e mais poderosos que o simples coronel [\dots], com mais crédito bancário, mais financiamento e mais dinheiro'', e que ``constitui a cabeça política do país, o estado-maior das `classes conservadoras' da nação'' \cite[p.~149]{basbaum_histsinc_1967}.

Especialmente no Nordeste açucareiro, os impactos da abolição da escravidão foram drásticos: sendo os escravos bens de raiz que custavam, em conjunto, tanto ou mais que a própria terra que lavravam, sua libertação descapitalizou os donos dos velhos banguês, situação agravada pelo baixo preço internacional do açúcar, impeditivo da rápida recomposição do capital. Estas perdas, e também a baixa disponibilidade de capital excedente, levam os sucrocultores nordestinos a serem menos dispostos a arriscar em inovações tecnológicas -- exceto no caso pernambucano, onde a transição dos banguês para as usinas garantiu sobrevida econômica e política às frações inovadoras desta classe -- ou em mudanças de ramo de investimento \cite[p.~153]{CARONE1970inst}. 

XXXXXXXXX

A historiografia tradicional costuma tratar apenas destes três blocos regionais ao tratar da política dos governadores e sua imbricação com os latifundiários. Fica sem explicação por que se dá tão pouca atenção aos estados do \textit{Norte} e do \textit{Centro-Oeste}, pois governar estes estados, ainda quando orçamentariamente deficitários, geograficamente isolados e de escassíssima população neste período, significava governar territórios gigantescos, maiores mesmo que vários países industrializados.

No \textit{Centro-Oeste}, vivia-se um completo isolamento do país. O Mato Grosso tinha população de oitenta mil habitantes em 1889, e era tão segregado que eram necessários trinta dias de viagem e a passagem por três países estrangeiros para alcançar sua capital, Campo Grande, por meio do rio Paraguai; tal situação só mudou em 1914 com a construção da Ferrovia Noroeste do Brasil \cite{almeida_matogrosso_2011}. Situação semelhante se dava em Goiás: a produção da pecuária e derivados era escoada do estado por meio de tropas de burros até o rio Araguari, de onde seguia por barco até seus destinos finais.  

A política do Mato Grosso, embora dominada pelos pecuaristas desde há muito, era atravessada no século XIX por uma contradição: o extremo sul da antiga província imperial era uma típica frente de expansão, recebendo migrantes do Paraguai e de outras províncias brasileiras (Minas Gerais, Goiás, São Paulo, Paraná, Rio Grande do Sul) dedicados a fazer fortuna com a pecuária e a cultura ervateira que então se desenvolviam sobre terras devolutas; eram os latifundiários do norte da província, entretanto, quem detinha o acesso aos recursos públicos provinciais e quem transitava na Corte. Embora os Murtinho viessem agindo quase sem oposição desde o império e se configurassem como virtuais donos do Estado, pesquisas recentes \cite{queiroz_murtinho_2010} indicam que eles eram sócios menores, quase que uns gerentes de alto escalão, de financistas cariocas como Francisco de Paula Mayrink, especialmente por meio do \textit{Banco Rio e Mato Grosso} e sua subsidiária mais famosa, a \textit{Companhia Mate Laranjeira}, que desde sua fundação em 1891 exerceu autêntico monopólio sobre a produção ervateira do sul do Mato Grosso focada no mercado argentino, criou sua própria infraestrutura de extração e comercialização do mate, comandava milhares de trabalhadores e dominou o setor até os anos 1940. Com a proclamação da república e a nomeação de Antônio Maria Coelho por Deodoro da Fonseca para a presidência do Estado, o novo governador arregimentou os militares atuantes na antiga província e os comerciantes de Corumbá para fundar o Partido Nacional Republicano (PNR), que estruturou sua base política principalmente entre os antigos conservadores; já os Murtinho, subitamente alijados do poder com os Ponce, seus antigos aliados, lançaram manifesto e fundaram o Partido Republicano arregimentando os antigos liberais. A crise resultou no movimento separatista fundador do Estado Livre do Mato Grosso, movimento este destruído pelas armas por meio de um cerco de Campo Grande sendo conduzido pelo próprio Generoso Ponce; por fim Antonio Maria Coelho foi destituído e Generoso Ponce venceu eleições tipicamente fraudadas, voltando ao poder. Sua sucessão, quando já sócio do Banco Rio e Mato Grosso com os Murtinho, gerou mais disputas, desta vez com seus antigos aliados, que saíram vencedores \cite{almeida_matogrosso_2011}. 

No que diz respeito à região \textit{Norte} do país, não se pode esquecer que a proclamação da república se deu em meio ao primeiro \textit{ciclo da borracha}, que tornara a Amazônia uma das regiões mais prósperas do país. 

José Leopoldo de Bulhões Jardim (1856-1928), fundador do clã dos Bulhões, Somente em 1909, com a fundação do Partido Democrata de Goiás 





Não obstantes tantas contradições internas, a presença dos latifundiários em todos os estratos governamentais brasileiros significava a garantia de políticas estatais de apoio financeiro à agricultura; o poder político é absolutamente dominado por esta classe durante toda a República Velha. No plano federal, todos os presidentes civis foram fazendeiros ou latifundiários; no plano dos Estados, a regra se repetiu sem variações \cite[p.~155]{CARONE1970inst}.

\subsubsection{A burguesia}\label{subsubsec:claburg}

Burguesia comercial DESENVOLVER, MOSTRANDO COMO OS COMERCIANTES SUBMETIAM OS LATIFUNDIÁRIOS POR MEIO DE CRÉDITO, DÍVIDAS E OLIGOPSÔNIO

Burguesia industrial DESENVOLVER, MOSTRANDO O CARÁTER SUBSIDIÁRIO FRENTE À PLANTAGEM

Os bons preços do café e a proibição de novas plantações, implementada em 1902, leva a camada mais dinâmica dos fazendeiros, como Rodrigues Alves, a aplicar capital próprio, retirado de suas lavouras, em comércio, bancos, indústrias e energia elétrica; o fenômeno se repetiu onde quer que boas safras ou inovações tecnológicas permitissem a formação de capital excedente, apto a ser reinvestido \cite[p.~147]{CARONE1970inst}

\subsubsection{O enigma da ``classe média'' urbana e a formação da classe dos gestores no Brasil}\label{subsubsec:clamed}

Sociólogos, economistas e historiadores criticam o uso do termo ``classe média'' por ser vago, incerto e não ter ``base conceitual de origem controlada'' \cite[p.~19]{POCHMANN2014};  mesmo as ilustrações históricas do papel desta classe seriam ``insatisfatórias'' \cite[p.~9]{pinheiro_clamed_1977}. Há, inclusive, quem prefira, prudentemente, passar reta e silenciosamente por qualquer esforço conceitual, confiando no puro empirismo para analisar a ação política desta ``classe''. O máximo que se tentou fazer quanto a esta ``classe'' no período estudado foi subdividi-la em dois grandes grupos: a classe média \textit{antiga}, composta pelos pequenos produtores e pequenos comerciantes, e a classe média \textit{nova}, advinda com a industrialização e a correspondente intensificação da divisão do trabalho \cite[p.~11]{pinheiro_clamed_1977}. Ou, ainda, dividi-la numa camada \textit{alta}, oriunda de setores da classe latifundiária por meio do bacharelismo; numa camada \textit{média}, composta por imigrantes, segmentos das classes decadentes, profissionais liberais, exército etc.; e numa camada \textit{baixa}, composta por artesãos e funcionários públicos \cite[p. ~175-176]{CARONE1970inst}. Foi tentado, também, diferenciar esta ``classe média'' segundo as características de sua inserção na estratificação social segundo o desenvolvimento histórico em cada região do país: se no Sul esta classe é formada por ``pequenos fazendeiros que abandonavam o campo, assim como colonos e seus descendentes que pretendiam subir na escala social'', no Norte ``as grandes famílias proprietárias decadentes forneciam contingentes de funcionários públicos, grupos profissionais, empregados de indústria e comércio, proprietários de pequenos negócios''  \cite[p.~16]{pinheiro_clamed_1977}.

A profusão de estratificações da ``classe média'', tal como as reiteradas confissões sobre as dificuldades de distingui-la de outras classes, testemunham o caráter \textit{problemático} da categorização homogênea dos tantos elementos heteróclitos que a compõem. A ``classe média'' é tratada como classe distinta das demais por \textit{hábito}, mais que por \textit{construção conceitual precisa}. Como quer que seja categorizada (e por maiores os problemas encontrados na sua conceituação), há vasta produção historiográfica sobre a atuação desta ``classe'' durante a República Velha, indicando não apenas a atuação de grupos sociais específicos e setores de classe bem delimitados. É destrinchando seus componentes que será possível afirmar não a pertinência conceitual e histórica de uma ``classe média'', mas a categorização mais precisa de seus compontentes nas classes sociais que realmente integram -- e a República Velha reuniu as condições ideais para o florescimento destes estamentos, grupos sociais e setores de classe. 

Uma primeira condição é a proliferação das \textit{faculdades}, os criadouros de \textit{bacharéis}, futuros \textit{burocratas}. Em 1916, por exemplo, já havia 16 faculdades de Direito, que formavam cerca de 408 bacharéis por ano; não se pode esquecer que, na falta de cursos formais de Administração, Sociologia ou Economia, eram os bacharéis em Direito quem cumpria com as atribuições destes profissionais em diversos postos (burocratas estatais de carreira, administradores empresariais, agentes bancários etc.), e muitos dos cursos jurídicos então existentes nomeavam-se de ``Ciências Jurídicas e Sociais''. Em 1920 foi criada a Universidade do Rio de Janeiro, atual universidade federal, primeira do país; em 1930, havia 350 estabelecimentos de ensino secundário e 200 de ensino superior \cite[p.~17]{pinheiro_clamed_1977}. As faculdades encontraram espaço para sua proliferação principalmente por causa do desenvolvimento das \textit{grandes cidades}, que não apenas reuniam num só espaço as repartições públicas e os cursos superiores como também eram, desde os tempos coloniais, o lugar por excelência de exercício das \textit{profissões artesanais} \cite{REIS2012}, e num contexto pós-abolicionista agregavam também os negros recém-libertos que a elas acorriam em massa para fugir -- temporária ou definitivamente -- da degradação do trabalho comum nos campos em sequência à conquista de sua liberdade\cite{AZEVEDO2004, bacelar_negrosalvador_1994}. No que diz respeito à sua atuação política, o aparente antagonismo entre os latifundiários e estes burocratas, primeiro e mais visível estrato da ``classe média'', era superficial, não correspondia a um antagonismo econômico: esta ``classe'' era economicamente dependente dos latifundiários, na medida em que durante toda a Primeira República brasileira desenvolveu-se o chamado ``estado cartorial'', uma política de angariamento de apoio político em troca de cargos na máquina pública \cite[p.~20]{pinheiro_clamed_1977}. O fato de o regime instaurado pelo golpe de 1930 haver operado uma ampla substituição no quadro burocrático preexistente e, pouco a pouco, haver renovado e ampliado a estrutura do Estado para compatibilizá-la com o programa de centralização política, tributária e administrativa exigido para uma mais completa coordenação do processo de industrialização \cite{araujo_dasp_2017} indica tanto a necessidade de burocratas estatais para o funcionamento de um Estado centralizado quanto, por via transversal, a existência e a necessidade deste funcionalismo público nos quase quarenta anos de vigência da Primeira República para o pleno funcionamento do sistema.

Outra condição é a proliferação do \textit{comércio} e do \textit{setor terciário} nestas grandes cidades. O Rio de Janeiro, por ser o maior entreposto comercial do país (com o consequente surgimento de postos de trabalho nos escritórios comerciais) e por ser a capital federal (com o consequente agrupamento espacial da burocracia correspondente a esta esfera de governo), era o lugar por excelência das ``classes médias'' \cite[p.~119]{pinheiro_clamed_1977}, mas a emergência tanto de uma burguesia comercial quanto de uma burocracia privada atrelada às grandes casas comerciais (guarda-livros, contadores, escriturários etc.) é fenômeno verificável em todas as capitais estaduais ou em cidades com economia comercial destacada. 

Os \textit{militares}, conquanto costumem ser inseridos na ``classe média'', guardam mais a característica de um \textit{estamento}, ao menos em seus estratos superiores, que de uma \textit{classe} relativamente coesa e unificada. Compunham uma ``instituição total'', ou seja, um tipo de instituição social que envolvia ``todos os aspectos da vida de seus membros'', chegando ao ponto de requerer de seus membros ``uma radical transformação de personalidade'' e a desenvolver uma ``identidade mais marcada'' -- facilmente verificável na distinção entre ``militares'' e ``paisanos'', comum à época \cite[p.~181]{carvalho_militares_1977}. No caso brasileiro, as formas de recrutamento existentes antes e durante a Primeira República demonstram que esta ``instituição total'' reproduzia em seu interior as distinções de classe existentes na sociedade brasileira: durante o Império as patentes superiores eram preenchidas por filhos da nobreza, em especial dos militares tornados nobres, enquanto as praças eram recrutadas à força em meio aos estratos inferiores da sociedade \cite[pp.~186-192]{carvalho_militares_1977}. Os republicanos tentaram modificar este sistema depois de 1891, sem sucesso: durante a Primeira República a maioria do oficialato continuava sendo recrutada em meio aos filhos de comerciantes, servidores públicos, militares e profissionais liberais \cite[p.~188]{carvalho_militares_1977}, enquanto as praças eram recrutadas entre os nordestinos afugentados pelas secas, entre os desocupados das grandes cidades que procuravam o serviço militar como emprego, entre os criminosos mandados pela polícia e entre os inaptos para o trabalho \cite[p.~190]{carvalho_militares_1977}. 

Um aspecto importante, talvez decisivo para melhor localizar o lugar dos militares na estrutura produtiva brasileira -- além, evidentemente, de seu papel enquanto força repressiva permantente -- está no \textit{processo formativo do oficialato}, em especial no âmbito da Escola Militar da Praia Vermelha (1858-1904): em especial durante a ``invasão positivista'' iniciada em 1872 pelo ingresso de Benjamin Constant em seu corpo docente, esta escola foi, no testemunho de seus ex-alunos, menos uma escola de oficiais militares e mais uma escola de ``burocratas, literatos, publicistas e filósofos, engenheiros e arquitetos notáveis, políticos sôfregos e espertíssimos, eruditos professores de matemáticas, ciências físicas e naturais'', chegando um historiador e sociólogo a dizer que ``o que na verdade produzia a Escola eram bachareis fardados, a competir com os bachareis sem farda das escolas de Direito e Medicina'' \cite[p.~196]{carvalho_militares_1977}. Outro ex-aluno afirmou categoricamente que ``o ambiente quase nada tinha de militar'' \cite[p.~196]{carvalho_militares_1977}. Num tão curioso ambiente, por vezes antimilitarista em meio à formação de militares, imbuído da ideologia do ``soldado-cidadão'' tão cara aos positivistas e aos militares  não é de se espantar, portanto, encontrarem-se embrenhados por todo o funcionalismo público brasileiro tenentes, capitães, majores e coroneis; mesmo no âmbito mais estritamente militar e livre das influências positivistas visto na Escola Militar do Realengo (1913-1944) um ex-aluno ilustre, Juarez Távora, dizia que a cadeira de direito público permitia-lhes ``ombrear com o bacharelismo de nosos políticos profissionais'' \cite[p.~211]{carvalho_militares_1977}.

Nos três casos nota-se a estruturação, a partir de diversas fontes, de uma classe \textit{burocrática}, ou seja, de uma classe social cujo lugar na produção econômica não está nem na propriedade dos meios de produção, nem na simples venda de sua força de trabalho para fins produtivos, pois ela era empregue em atividades de \textit{supervisão} e \textit{gerenciamento}. Ainda que as condições gerais de produção e da reprodução da força de trabalho estivessem ainda em suas fases mais elementares, ainda que não estivessem plenamente desenvolvidas na fragmentária estrutura econômica brasileira (cf. \autoref{subsec:brasimper}, p. \pageref{subsec:brasimper}), nem por isto se pode dizer que não existissem durante a Primeira República. No campo educacional, por exemplo, a tenacidade da propaganda ``pedagogista'' nas muitas campanhas nacionalistas das três primeiras décadas republicanas e as reformas educacionais escolanovistas da década de 1920 \cite{nagle_educacao_1977}, somadas aos esforços educacionais vistos entre os próprios trabalhadores com a fundação de escolas pelos sindicatos e por grupos de militantes anarquistas, socialistas e católicos \cite{andradeneto_educana_2014,ghiraldelli_educmovop_1987}, se, em seu conjunto, não conseguiram reverter o grave quadro de analfabetismo que assolava o Brasil na Primeira República, serviram como campo de atuação para toda uma burocracia ligada ao Ministério da Educação e Saúde Pública e para as secretarias estaduais e municipais ligadas à área -- campo este, por definição, controlado pelos burocratas estatais. Por outro lado, pode-se facilmente verificar a existência na Primeira República de diversas das condições gerais de produção (cf. a \autoref{subsec:cgpcsjobe} (p. \pageref{subsec:cgpcsjobe}) para retomar o debate conceitual):

\begin{itemize}
\item As \textit{condições gerais de realização social da exploração}, em especial nas grandes cidades ligadas à economia agroexportadora, eram instituídas tanto pela polícia quanto pelas reformas urbanas que resultaram na separação definitiva entre os bairros concentradores dos postos de trabalho e os bairros de residência proletária, controladas e geridas tais reformas, quando não também concebidas, por engenheiros amiúde lotados nos órgãos municipais e estaduais responsáveis pela gestão de obras públicas e engenharia sanitária; 
\item As \textit{condições gerais de operatividade do processo de trabalho} eram instituídas tanto pelos burgueses quanto por um incipiente corpo de administradores intermédios (guarda-livros, contadores, escriturários, contínuos, praticantes, autografistas, tesoureiros, fiéis, almoxarifes etc.);
\item As \textit{condições gerais de operacionalidade das unidades de produção} eram instituídas tanto por pequenos empreiteiros quanto por grandes obras públicas, também estas submetidas ao controle e gestão de engenheiros ligados aos órgãos municipais e estaduais responsáveis pela gestão de obras públicas e engenharia sanitária; 
\item As \textit{condições gerais de operatividade do mercado} eram instituídas tanto por administradores intermédios públicos e privados (guarda-livros, contadores, escriturários, contínuos, praticantes, autografistas, tesoureiros, fiéis, almoxarifes etc.) quanto, mais uma vez, por engenheiros ligados aos órgãos municipais e estaduais responsáveis pela gestão de obras públicas e engenharia sanitária; 
\item As \textit{condições gerais de realização social do mercado}, instituídas na época de forma ainda difusa e pouquíssimo especializada, via nas propagandas inseridas em jornais e revistas os primeiros passos para a constituição de uma indústria da comunicação e propaganda.
\end{itemize}

Como se vê, na Primeira República desenvolviam-se muitas das atividades de integração econômica e de relacionamento entre empresas onde se radicavam aqueles a quem, nesta pesquisa, se chama de \textit{gestores}\footnote{Um debate mais esmiuçado sobre os gestores na Primeira República poderá ser encontrado no \autoref{ap:1} (p. \pageref{ap:1}).}. A esta classe gestorial historiadores e sociólogos costumam adicionar a \textit{pequena burguesia} clássica, ou seja, profissionais liberais (médicos, advogados etc.) e artesãos independentes, todos proprietários de meios de produção, mas trabalhando com quadro reduzidíssimo de funcionários, ou mesmo sem nenhum -- diferenciando-se assim da burguesia, capaz esta de mobilizar maior volume de capital e de comprar a força de trabalho de pessoas cujo trabalho comanda diretamente ou por intermédio de uma camada de burocratas.

Vem desta mistura entre gestores e pequena burguesia as confusões na historiografia acerca do papel político da chamada ``classe média''. Para a historiografia que a toma como categoria social válida, a atuação política desta ``classe'' durante a Primeira República foi oscilante, a demonstrar sua própria (in)definição enquanto classe. Um exemplo: na década de 1890 o florianismo e sua vertente radical, o jacobinismo, floresceram entre a ``classe média'' das grandes cidades brasileiras \cite{queiroz_radicais_1986}, mas já em 1910 esta mesma ``classe média'' apoiou decididamente a campanha civilista de Rui Barbosa. Não era de se esperar outra coisa: a atuação dos militares, da pequena burguesia e dos burocratas públicos e privados diferenciava-se pela sua situação na estrutura social brasileira e também pelos diferentes interesses daí advindos. 

Em suma: crescente em termos demográficos, por força da crescente complexificação da divisão social do trabalho no país, a ``classe média urbana'' aparentemente não teve durante a República Velha um desempenho político que visasse o aumento de seu poder no sistema político vigente, nem tampouco pautou questões voltadas à transformação radical do regime vigente \cite[p.~36]{pinheiro_clamed_1977}; como visto, entretanto, esta ``classe'' não é propriamente uma classe social, mas sim o agrupamento informe, literalmente \textit{ad hoc}, de grupos sociais, classes sociais e estamentos distintos, ao que tudo indica à falta de um marco teórico apto a identificar mais precisamente seus caracteres distintivos. Só é possível entender adequadamente a tal ``classe média'' e sua ação política, portanto, quando a destrinchamos de acordo com as classes, setores de classe e estamentos que se costuma agrupar sob seu nome, e quando se analisa cada um destes em separado.

\subsubsection{Os trabalhadores: a complexa formação de uma classe}\label{subsubsec:clatrab}

Os trabalhadores são uma das classes globais do regime capitalista; conquanto esta afirmação tenha validade num plano lógico, teórico, num plano histórico, prático, sua formação assenta-se nos processos históricos de cada tempo e lugar. O que os põe juntos enquanto classe, num primeiro momento, é sua posição no processo de trabalho global, em oposição aos burgueses e aos gestores; quaisquer outras ligações entre estes elementos da classe trabalhadora global dependem de sua ação nos campos político e cultural. É esta ação, assim como os processos históricos de sua formação enquanto classe, que precisam ser compreendidos em cada caso \cite{aguiar_classe_2009}.

No caso brasileiro, para que se formasse uma classe trabalhadora houve uma confluência de dois fatores \cite[p.~27]{chalhoub_botequim_1986} o longo processo de \textit{luta contra a escravidão} que desembocou na sua abolição, no qual negros escravizados criaram formas próprias de luta, negociação e resistência \cite{AZEVEDO2004, bethell_trafico_2002, chalhoub_liberdade_1990, conrad_ultimosanos_1978, farias_cidadesnegras_2006, fraga_encruzilhadas_2014, luna_lutaescravidao_1976, reis_elitemovsoc_1976, REISSILVA1989, REIS2004males, reis_familiareal_2008, schwartz_1814_1996, silva2007caminhos}, carreadas por eles ao mercado de trabalho livre assim como livre as práticas, técnicas e saberes trazidos da África ou adquiridos durante o cativeiro \cite{fraga_encruzilhadas_2014, souza_trabalholivre_2011, REIS2012}; e a chegada de uma massa de migrantes (italianos, espanhóis, portugueses, japoneses, alemães, poloneses, austríacos, lituanos, iugoslavos, húngaros, tchecos, romenos, russos etc.) para as cidades e campos, especialmente do Sul e Sudeste, para servir como trabalhadores de baixa ou média qualificação, muitos dos quais -- não todos, contudo -- trazendo de seus países de origem ideologias e tradições próprias de organização, como o anarquismo e o socialismo \cite{petrone_imigra_1977}.

Não obstante ser possível entender que entre migrantes recém-chegados e negros recém-libertos do cativeiro houvesse sérios estranhamentos (especialmente por causa do racismo anti-negro) \cite[pp.~35-76]{chalhoub_botequim_1986}; que correntes intelectuais como o socialismo e o anarquismo em cidades de menor porte permanecessem restritas a pequenos círculos intelectuais \cite{duarte_rebelde_1991}; que tais correntes tivessem problemas em adaptar-se a práticas e costumes locais, especialmente aos de origem africana \cite{goes_formacao_1988}; não obstante tudo isso, é certo que desde os primeiros anos da República, quando os migrantes ultrapassavam o racismo anti-negro em prol de questões comuns, estes dois setores envolveram-se em lutas conjuntas, e que os trabalhadores interessados na chamada ``questão social'' -- ou seja, na superação de sua condição de classe explorada e oprimida -- discutiam-na abertamente com seus companheiros de labor \cite[p.~73-85]{gomes_velhos_1988}; que formaram um potente movimento operário, simultaneamente reivindicativo e revolucionário \cite{samis_anabras_2004}, capaz de organizar as forças do trabalho nos planos político e cultural \cite{farinha_federa_2002,hardman_patripatr_2002}, de paralisar todo o trabalho de uma cidade por meio de greves gerais \cite{castellucci_salvador_2001,magnani_anarsp_1982} e inclusive de promover atos insurrecionais \cite{dulles_anacombras_1977,koval_prolbras_1982}.  

No que diz respeito à \textit{composição técnica} desta classe, as fontes censitárias de 1872 e 1920 refletem a divisão social do trabalho existente no país ao classificar como ``industriais'' profissões tão díspares quanto as ``artes e ofícios'' (marceneiros, ferreiros, mecânicos etc.), os trabalhadores artesanais e as indústrias caseiras \cite[p.~141]{pinheiro_prolind_1977}. Os trabalhadores ditos ``qualificados'', os da construção civil e os dos transportes (terrestres e marítimos) conseguiam razoável grau de organização, mas os trabalhadores fabris de mais baixa especialização técnica eram, em sua maioria, mulheres e crianças, considerados por seus contemporâneos como mais difíceis de organizar \cite[p.~152]{pinheiro_prolind_1977}. É possível dizer que, dada a pequena relevância da produção fabril na vasta maioria do território brasileiro, estes trabalhadores artesanais constituíam a maioria da classe trabalhadora no período.

É de se indagar, no caso brasileiro, se chegou a se formar nestes movimentos a \textit{aristocracia operária} vituperada num só coro por anarquistas, socialistas e comunistas \cite{bakunin_contramarx_2015,engels_1892pref_1990,lenin_imperialismo_1987}. Os estudos realizados até o momento indicam a formação de uma \textit{camada superior} entre os trabalhadores urbanos, em geral formada por aqueles ligados às profissões artesanais (sapateiros, alfaiates, vidreiros, estucadores, marmoristas, calceteiros etc.) ou ligadas de algum modo à cultura (gráficos, professores etc.); e entre eles, formou-se como camada ainda mais coesa o grupo daqueles que, por saber ler e escrever -- não se pode esquecer que no Brasil da época a taxa de analfabetismo variou entre 83\% (1890) a 65\% (1920) -- capitanearam as incontáveis iniciativas culturais operárias do período (escolas, grupos de teatro, círculos literários etc.) \cite{gomes_velhos_1988,goes_formacao_1988,hardman_patripatr_2002,pinheiro_prolind_1977}. Há, inclusive, quem classifique esta camada superior da classe trabalhadora já como ``classe média'' -- problema a ser discutido na \autoref{subsubsec:clamed}.

É importante observar, por outro lado, que esta camada estava apta apenas a exercer hegemonia \textit{cultural} sobre a classe, que não coincidia com a hegemonia \textit{política}. Os sindicatos do período, instrumento político por excelência dos trabalhadores num momento em que a proibição do voto aos analfabetos impedia-os de participar da política eleitoral mesmo no papel passivo de eleitores, eram organizados por \textit{ofícios} (ou seja, para cada profissão um sindicato), e não por \textit{ramo industrial} (ou seja, para cada cadeia produtiva um sindicato); isto garantia que mesmo os trabalhadores menos privilegiados podiam liderar suas categorias, e assim participar da ação política em pé de igualdade com as categorias profissionais mais elitizadas. As reivindicações trabalhistas eram tratadas no período pelos empresários com supremo desdém, quando não com violência; isto, e a criminalização das greves no Código Penal de 1890 (arts. 204 a 206), gerou a reação de ações igualmente violentas por parte dos trabalhadores, transformando cada greve numa potencial insurreição. Adicionalmente, embora a ação sindical existisse no Brasil desde a alvorada da república (ou mesmo antes dela \cite[p.~69-77]{koval_prolbras_1982}), o reconhecimento dos sindicatos como interlocutores pelos empresários via de regra era nulo, e os acordos ao final de cada greve eram feitos diretamente entre os patrões e os trabalhadores \cite{dulles_anacombras_1977,koval_prolbras_1982}. Soma-se a isso o fato de os poucos partidos denominados ``operários'' ou ``socialistas'' no período, além de absolutamente inexpressivos em termos eleitorais, serem em geral dominados pelas chamadas ``classes médias'' \cite[p.~150]{pinheiro_prolind_1977}, gerando um estranhamento impeditivo de sua transformação em reais instrumentos políticos pelos trabalhadores. Para piorar, fora dos períodos de greve os sindicatos não conseguiam a mesma audiência dos períodos paredistas \cite[p.~152]{pinheiro_prolind_1977}. Sendo assim, esta camada superior, por privilegiada que fosse no seio da própria classe, não dispunha das condições para o mesmo tipo de ``aburguesamento'' verificado nas aristocracias operárias europeias. Esta aristocracia, conhecida no Brasil pelo nome de ``pelego'', só veio a ser formada como efeito da reestruturação corporativista do Estado brasileiro em 1937, quando os sindicatos foram transformados em órgãos estatais.

Cabe, a esta altura, a pergunta: e quanto a quem não se adequou à ordem social cuja descrição sumária se pretendeu a partir da apresentação das classes acima e de suas relações recíprocas? 

Especialmente no contexto da transição de uma economia baseada na exploração do trabalho escravo para outra assentada na exploração do trabalho assalariado, esta pergunta tem a ver com o \textit{controle social} imposto àqueles que haviam recém conquistado sua liberdade, ou seja, com a imposição aos recém-libertos de novas formas de sujeição por meios diversos: a frustração de qualquer expectativa sua de liberdade substantiva, de trabalho autônomo, de propriedade de meios de produção (especialmente, numa economia agrária, a terra); a humilhação quase ritualística, a inculcação mental de uma suposta inferioridade biológica, a exigência de comportamento ``obediente'' e submisso no trabalho e fora dele; a obstaculização à qualificação da própria força de trabalho por meio da educação e a facilitação à ocupação de postos de trabalho aviltados ou demandantes de pouca qualificação profissional (como, por exemplo, estivador ou ``caixeiro''); entre outras \cite{bacelar_negrosalvador_1994, chalhoub_botequim_1986}.

Se é certa a inserção dos recém-libertos nas novas formas de exploração principalmente por meio de seu assalariamento nas cidades e de seu assentamento em pequenas propriedades rurais como parceiros, meeiros, agregados etc., não menos certa é a luta dos recém-libertos pela sua dignidade de cidadãos da república. Testemunho disto é a proliferação, de Porto Alegre a Recife, da \textit{imprensa negra}, vinculadas a associações, clubes e grêmios como a \textit{Associação Protetora dos Brasileiros Pretos}, o \textit{Centro Cultural Henrique Dias}, \textit{C. G. Campos Elíseos}, \textit{Grêmio Bandeirantes}, \textit{Grêmio Dramático Recreativo e Literário ``Elite da Liberdade''}, \textit{Smart}, \textit{Sociedade Propugnadora 13 de Maio}, \textit{Treze de Maio}, entre outros; jornais como \textbf{O Treze de Maio} (1888), \textbf{A Pátria} (1889), \textbf{O Exemplo} (1892), \textbf{A Redenção} (1899), \textbf{O Baluarte} (1903), \textbf{O Propugnador} (1907), \textbf{O Combate} (1912), \textbf{O Patrocínio} (1913) e outros, ainda que efêmeros, dada a falta de patrocínio adequado ou, em alguns casos, por força de sua natureza propositalmente temporária, serviam não apenas de instrumentos de educação e formação por meio do debate público de questões relativas à inserção dos recém-libertos na sociedade, como também estabeleciam laços comunitários entre os negros, criavam expectativas comportamentais adequadas em meio a esta comunidade da ``classe de cor'' que a diferenciassem do ``preto comum'', traduziam artigos da imprensa negra de outros países\footnote{Ressaltam do contexto artigos do líder negro estadunidense Marcus Garvey publicados n'\textbf{O Clarim d'Alvorada}. José Correia Leite, integrante da redação de \textbf{O Clarim d'Alvorada}, menciona como havia ``um reduzido grupo de garveyristas'' entre seus colegas, como Alcino dos Santos e João Sótero da Silva, baianos tornados representantes do jornal em Salvador \cite[p.~40]{gomes_negrosepolitica_2005}. Por isto mesmo \textbf{O Clarim d'Alvorada} foi acusado na própria imprensa negra de ``mimetismo'', ``importação de outras realidades'', de ``fazer um movimento que era importado'' e de proporem um ``modelo racista'' para o Brasil \cite[p.~42]{gomes_negrosepolitica_2005}} e, em alguns casos, estabeleciam contatos internacionais\footnote{Além das relações com Marcus Garvey, ao que tudo indica restritas à tradução de textos, houve muita permuta e intercâmbio de \textbf{O Clarim d'Alvorada} com o jornalista negro estadunidense Robert Abbot, do \textbf{Chicago Defender}, entre 1923 e 1926; o Centro Cívico Palmares teve em sua diretoria um negro inglês conhecido como Mr. Gids, gerente da grande papelaria paulista Casa Vanote; e a biblioteca do Clube Negro de Cultura Social esteve a cargo de um negro de Trinidade e Tobago mencionado como John \cite[pp.~43-44]{gomes_negrosepolitica_2005}} \cite[pp.~27-44]{gomes_negrosepolitica_2005}.

Aos que, submissa ou matreiramente, inseriam-se num tal quadro de controle social, abriam-se as possibilidades -- exíguas, mas possíveis -- de mobilidade social ascendente. Aos ``vagabundos'', aos ``vadios'', aos ``gatunos'', aos ``desempregados crônicos'' (``doença'' grave!), aos ``capoeiras'', aos ``malandros'', aos ``libertinos'', aos ``capangas'', aos ``desordeiros'', às ``mulheres da gandaia'', aos ``jagunços'', aos ``viciados'', aos ``insubmissos'', aos ``incorrigíveis'', aos que diante do inferno cotidiano do pauperismo restava o recurso eventual ou costumeiro a expedientes postos à margem da ordem social  -- a estes, os rigores da lei e a pecha de pertencerem às ``classes perigosas'' \cite{chalhoub_botequim_1986, guimaraes_classper_1981}. Conquanto não constituam uma classe no sentido rigoroso do termo, mas sim um \textit{epíteto dado pelas classes dominantes àqueles segmentos da classe trabalhadora que escapam ao seu controle}, foi termo corrente nos debates parlamentares imediatamente precedentes à abolição da escravatura e persistiram na imprensa da Primeira República, pelo que vale o registro. 

Este movimento, entretanto, circunscreveu-se aos trabalhadores \textit{urbanos}; os \textit{trabalhadores rurais}, em suas diversas formas históricas (parceiros, meeiros, moradores, arrendatários, safreiros, foreiros, boias-frias, agregados, colonos etc.)\footnote{\citeonline{basbaum_histsinc_1967} faz interessante classificação do que chamou de ``subclasses rurais'' como os \textit{pequenos proprietários}, os \textit{arrendatários}, os \textit{colonos}, os \textit{trabalhadores agrícolas propriamente ditos} e os \textit{camaradas}; dada a vivência do autor entre estes trabalhadores enquanto dirigente do PCB durante a Primeira República e o período varguista, é certo que esta categorização vem de observação empírica e seria muito interessante de desenvolver, mas um trabalho assim não está no escopo desta pesquisa.} pouco se integraram às lutas dos trabalhadores urbanos no período estudado. Embora movimentos como os de \textit{Canudos}, do \textit{Contestado} e a \textit{Revolta do Capim} sejam de extrema relevância em seus respectivos contextos \cite{mottazarth_rescamp1_2008}, não foi possível encontrar nas obras historiográficas consultadas acerca destas lutas maiores indícios de sua integração com os processos de luta dos trabalhadores urbanos -- o que não impossibilita de forma alguma sua ocorrência.

\subsection{As cidades brasileiras: reformas urbanas e regime de terras em tempo de monopólios}\label{subsec:cidbraref}

Tendo chegado à escala nacional, já é possível falar, malgrado as inevitáveis particularidades, de um \textit{contexto urbano} um pouco mais homogêneo.

Censitariamente, e mesmo com os cuidados a serem tomados no uso dos dados censitários anteriores a 1940\footnote{\citeonline[p.~24]{santos_urbanizacao_2005} observa que ``somente após 1940 as contagens separavam a população urbana (cidades e vilas) da população rural do mesmo município''.}, a evolução da urbanização brasileira, conquanto ``pequena e frágil'' \cite[p.~303]{suzigan_polgov_2001} e longe de alcançar os patamares do período iniciado na década de 1940, começava a se destacar (cf. \autoref{tab:popurbra}).

\begin{table}[!htp]
\IBGEtab{
\caption{Grau de urbanização do Brasil (1872-1920)}\label{tab:popurbra}
}{
\begin{minipage}{18cm}
\begin{tabular}{|m{1cm}|m{1.8cm}|m{0.4cm} m{1.5cm} m{0.4cm} m{1.5cm} m{0.4cm} m{1.5cm} m{1cm} m{1cm} m{1cm}|}
\hline 
\multirow{2}{*}{Censo} & \multirow{2}{*}{Pop. total} & \multicolumn{2}{c}{50 mil ou +} & \multicolumn{2}{c}{100 mil ou +} & \multicolumn{2}{c}{500 mil ou +} & \multicolumn{3}{c|}{Pop. urbana (\%)} \\ 
\cline{3-11} & & nº & pop. & nº & pop. & nº & pop. & 50 mil ou + & 100 mil ou + & 500 mil ou + \\ 
\hline
1872 & 9.930.478 & 4 & 582.749 & 3 & 520.752 & -- & -- & 5,9 & 5,6 & -- \\ 
1890 & 14.333.915 & 6 & 976.038 & 3 & 808.619 & -- & -- & 6,8 & 5,6 & -- \\ 
1900 & 17.438.434 & 8 & 1.644.149 & 4 & 1.370.182 & -- & -- & 9,4 & 7,9 & -- \\ 
1920 & 30.635.605 & 15 & 3.287.448 & 6 & 2.674.836 & 1 & 1.157.873 & 10,7 & 8,7 & 3,8 \\ 
\hline 
\end{tabular} 
\end{minipage}
}
{\fonte{\citeonline{cardoso_govmil_1977}}.}
\end{table}


Durante a República Velha, as cidades se desenvolveram dentro da dinâmica do sistema agrário-exportador; a urbanização se deu ``à sombra do fortalecimento da economia agrário-exportadora, que a longo prazo conformará o Estado à sua própria imagem, portanto, à própria burocracia''  \cite[p.~22-23]{pinheiro_clamed_1977}.

Veja-se este desenvolvimento à luz de alguns exemplos. No Rio de Janeiro, \cite{singer_evourb_1968}. Em São Paulo, \cite{singer_evourb_1968}. Em Blumenau, \cite{singer_evourb_1968}. Em Porto Alegre, \cite{singer_evourb_1968}. Em Belo Horizonte, \cite{singer_evourb_1968}. Em Recife, \cite{singer_evourb_1968}. 

O \textit{higienismo}, como alhures, era a ideologia animadora dos debates em torno da situação das cidades; medicina e engenharia sobrepunham-se na definição do que era mais salubre para as cidades, propondo invariavelmente ambientes capazes de deixar sair os ``maus odores''; e as estratégias dos sanitaristas pós-pasteurianos, adeptos da teoria microbiana do contágio, não diferiam daquelas adotadas por seus antecessores pré-pasteurianos e pró-miasma: vigia a mesma associação entre pobreza e imoralidade, as mesmas soluções de combate às habitações ditas ``insalubres'' sem oferta de alternativas acessíveis a curto prazo \dots \cite{CAPONI2002} Engenheiros sanitaristas como Saturnino de Brito, Lourenço Baeta Neves, Miguel Presgrave, Teodoro Sampaio, Bernardino de Queiroga, Victor da Silva Freire, Manoel Pereira Reis, Américo Rangel e José Pereira Rebouças desenvolveram planos urbanos (alguns chamados de ``planos de melhoramentos'' \cite{leme_urbasp_1991}) e coordenaram a execução de obras de saneamento; se não se diziam ``modernos'', suas concepções eram profundamente modernas \cite{andrade_saturnino_1991}. Entre os vários ``planos de melhoramentos'' da época, o mais conhecido é o realizado, em 1927, por convite do prefeito do Rio de Janeiro, Antonio Prado Junior, ao urbanista francês Alfred Agache: resultou daí um plano de extensão, embelezamento e remodelação para o Rio de Janeiro, apresentado em 1930 \cite{pinheiro_capiconsul_2009}. 

Em termos estéticos, o \textit{ecletismo} encontrou no Brasil terreno fértil:

\begin{citacao}
A ideologia progressista da República encontrará no Ecletismo arquitetônico a linguagem que permite equiparar imagens de realidades distintas, derivaas no plano teórico, das mesmas idealizações de modernidade. O Estado republicano conduzirá a organização e o controle da produção do espaço edificado, realizando o projeto estético há muito acalentado para as cidades, cujos desdobramentos serão peculiares em cada lugar [\dots].

De fato, a partir de meados do século XIX, as várias tendências do chamado período eclético -- do Historicismo tipológico aos pastiches compositivos --, quase que simultaneamente, são assimiladas pelo repertório construtivo das diversas cidades brasileiras que, em escalas distintas, procuram se aproximar do ideal de civilização, modernidade e progresso. \cite[p.~259-260]{almeida_vitrinescomercio_2014}
\end{citacao}

Assim, outra vez a estética é expressão de conflitos sociais. Vimos na \autoref{subsec:armor} como, na América Latina, a transposição de elementos arquitetônicos europeus, eles próprios expressivos de conflitos sociais, tendia a resultar em efeitos diversos daqueles obtidos em seus ambientes originais, e que estes efeitos resultantes das transposições estéticas é que deveriam ser estudados, sempre em relação a seus conflitos locais. Vemos aqui, agora, estes efeitos em funcionamento. Enquanto ideologia, o republicanismo, no Brasil como em outros países da América Latina, representou a tentativa de aproximação de ideais de ``civilização, modernidade e progresso'', em tudo opostos aos valores e práticas do período colonial (e, no caso brasileiro, do período imperial); o ecletismo foi, assim, a forma de botar abaixo as reminiscências do passado e fazer das edificações -- públicas especialmente, mas também as privadas quando as condições o permitiam -- instrumento de educação dos cidadãos a partir de uma estética monumental e cívica.