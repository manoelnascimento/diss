\section{Conclusão}\label{sec:1.5}

Ao final do período 1889-1930, pelo que se apresentou no cenário, foi possível verificar a consolidação de uma dupla hegemonia: da sociedade por ações, da corporação impessoal, da grande empresa enquanto forma preferencial do empreendimento capitalista e ``órgão supremo do imperialismo'' \cite{PEDROSA1966a}, e dos EUA enquanto principal ator geopolítico. Pode-se dizer que as tensões do período se avolumaram até a crise de 1929, que é, no plano internacional, o marco temporal extremo desta pesquisa. Suas consequências -- o \textit{New Deal} rooseveltiano e outros planos de recuperação econômica, ou o surgimento do \textit{fascismo} em diversos países (e não só na \index{Itália}Itália ou na \index{Alemanha}Alemanha) -- extrapolam os limites temporais a que esta investigação se circunscreve. Resta, portanto, deixar a crise em suspenso e retomar a narrativa, desta vez analisando os conflitos sociais em escala nacional.