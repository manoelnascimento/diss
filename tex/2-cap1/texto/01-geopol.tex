\section{Situação geopolítica internacional}\label{sec:1.1}

Costuma-se estabelecer como marco histórico da passagem do século XIX para o XX o novo \index{imperialismo}imperialismo colonialista pactuado na \index{imperialismo!Conferência de Berlim (1884-1885)}Conferência de Berlim (1884-1885), mas tal marco coloca em segundo plano questões importantes mais remotas, definidoras de tendências desta época. O período escolhido para esta pesquisa está inserido numa longa e turbulenta linha de acontecimentos inserida num ciclo que vai de 1870 a 1929 \cite{bernardo_fascismo_2003,bukharin_imperialismo_1986,
hobsbawm_empire_1989,hobsbawm_extremes_1995,hobson_imperialism_1902,
hobson_capitmoderno_1983,KROPOTKIN1901,lenin_imperialismo_1987,luxemburg_acumula_1985,
morris_magnatas_2010}, iniciado com a Guerra Franco-Prussiana (19 de julho de 1870 - 10 de maio de 1871) e encerrado com a crise de 1929. Para entender a dinâmica deste ciclo, é preciso compreender seus fatores principais e extrair deles os elementos que interessam ao recorte temporal e espacial da presente pesquisa. 

\subsection{Precedentes: as unificações políticas e a crise econômica global de 1873-1896}

Dois fatores preponderam como antecedentes históricos relevantes, em escala global, do período escolhido para o estudo: as \index{unificação ou reunificação política}\textit{unificações e reunificações políticas} de países europeus anteriormente fragmentados, concluídas nos anos 1880; e a \index{crises econômicas!crise de 1873-1896}\textit{crise econômica de 1873-1896}.

\subsubsection{Unificações e reunificações políticas (1815-1871)}

O \index{Itália!risorgimento@\textsl{risorgimento}}\textit{risorgimento} da \index{Itália}Itália foi um longo processo, iniciado com a confirmação da partição da \index{Itália}Itália pelo \textit{Congresso de Viena} (1815) e concluído com a tomada de Roma (1871). Levantes e insurreições contra o domínio austríaco sobre a Itália já aconteciam desde 1820 sob a influência dos \index{Itália!risorgimento@\textsl{risorgimento}!carbonários}\textit{carbonários}, mas foram a \index{Itália!risorgimento@\textsl{risorgimento}!Insurreição de 1830}\textit{Insurreição de 1830} e a \index{Itália!risorgimento@\textsl{risorgimento}!Primeira Guerra Italiana de Independência}\textit{Primeira Guerra Italiana de Independência} (1848-1849), esta última com presença marcante do movimento \index{Itália!risorgimento@\textsl{risorgimento}!\textit{Giovane Italia}}\textit{Giovane Italia}, os primeiros movimentos políticos a pautar seriamente a questão. A \index{Itália!risorgimento@\textsl{risorgimento}!Segunda Guerra Italiana de Independência}\textit{Segunda Guerra Italiana de Independência} (1859), iniciada pelo Reino da Sardenha governado pelo rei \index{Itália!risorgimento@\textsl{risorgimento}!Vítor Emanuel II}\textit{Vítor Emanuel II} e pelo primeiro-ministro \index{Itália!risorgimento@\textsl{risorgimento}!Camillo Benso, Conde de Cavour}\textit{Camillo Benso, Conde de Cavour}, trouxe os primeiros resultados expressivos para a unificação italiana com a anexação do Reino das Duas Sicílias ao Reino da Sardenha. No ano seguinte, o Reino da Sicília anexou, com o beneplácito da \index{França}França e da \index{Grã-Bretanha}Grã-Bretanha, o Ducado de Parma, o Ducado de Modena, o Grão-Ducado da Toscana e os Estados Papais; em troca, a França anexou a Saboia e Nice. Já em 17 de março de 1861 o parlamento sardo proclamou a mudança do nome do Reino da Sardenha para Reino da Itália, e proclamou \index{Itália!risorgimento@\textsl{risorgimento}!Vítor Emanuel II}Vítor Emanuel como \textit{Rei da Itália}. Restavam, para completar o quadro, o Reino Lombardo-Vêneto e o território restante dos antigos Estados Papais; o primeiro foi anexado em 1866 depois da \index{Itália!risorgimento@\textsl{risorgimento}!Terceira Guerra Italiana de Independência}\textit{Terceira Guerra Italiana de Independência} (1866), e o segundo depois da \index{Itália!risorgimento@\textsl{risorgimento}!captura de Roma}\textit{captura de Roma} (1870) \cite{hobsbawm_capital_1977}.

A \index{EUA!Guerra de Secessão}\textit{Guerra de Secessão} nos \index{EUA}EUA (12 abr. 1861 - 22 jun. 1865), ao mesmo tempo em que reunificou os \index{EUA}EUA, colocou-os sob a hegemonia dos estados industriais do \index{EUA!Norte}Norte, extinguiu a escravidão nos estados do \index{EUA!Sul}Sul e facilitou enormemente a conflagração de guerras de conquista contra os povos indígenas ao \index{EUA!Oeste}Oeste. FALAR DOS CONFLITOS RACIAIS E POLÍTICOS INTERNOS, E DA RECONSTRUÇÃO, E DOS ROBBER BARONS.

A \index{Alemanha!unificação da Alemanha}\textit{unificação da Alemanha} em 18 de janeiro de 1871 foi o ápice de um processo iniciado em 1862. A vitória da \index{Alemanha!Confederação da Alemanha do Norte}Confederação da Alemanha do Norte sobre a \index{França}França de \index{Napoleão III}Napoleão III na \index{Guerra Franco-Prussiana}Guerra Franco-Prussiana, resultante, entre outras coisas, do enorme avanço técnico e industrial experimentado durante a segunda metade do século XIX nos pequenos reinos (em especial a \index{Alemanha!Prússia}Prússia) que viriam a formar a \index{Alemanha}Alemanha unificada.  A tomada da \index{Alsácia-Lorena}Alsácia-Lorena aos franceses, outro resultado da guerra, garantiu à \index{Alemanha}Alemanha a satisfação de velhas aspirações nacionalistas, uma vantagem estratégica (afastar do \index{Alemanha!Reno}Reno a fronteira com a \index{França}França e fixá-la na cordilheira dos \index{França!Vosges}Vosges, obstáculo natural militarmente mais eficaz \cite{eckhardt_alsace_1918}) e uma vantagem econômica (\index{matérias-primas!carvão}carvão e \index{matérias-primas!ferro}ferro, fundamentais para uma \index{indústria}indústria ainda baseada no \index{energia!vapor}vapor e, posteriormente, na \index{energia!energia termelétrica}energia termelétrica \cite{brooks_alsace_1917}), mas assegurou permanente animosidade entre os dois países. O \index{Áustria!Império Austro-Húngaro}Império Austro-Húngaro, que disputara ferozmente com a \index{Alemanha!Prússia}Prússia a hegemonia sobre a \index{Alemanha!Confederação Alemã|seealso{Confederação da Alemanha do Norte}}Confederação Alemã até então, ficou para trás, malgrado seus enormes avanços tecnológicos \cite{schulze_engin_1996}, vivendo em suas últimas décadas (1900-1918) notável degeneração institucional. 

Estes processos de unificação (\index{Alemanha}Alemanha e \index{Itália}Itália) e reunificação (\index{EUA}EUA) criaram as condições para que a hegemonia política global da \index{Grã-Bretanha}Grã-Bretanha no século XIX fosse desafiada. Pode-se dizer que parte de seu sucesso econômico e político nos três primeiros quartos do século se deveu à fragmentação política de outros países que poderiam ser seus sérios concorrentes nos campos geopolítico e econômico; para azar dos britânicos, os processos de unificação e reunificação foram animados por elementos ideológicos favorecedores do expansionismo, do anexionismo e da disputa geopolítica internacional. Na medida em que \index{Itália}Itália e \index{Alemanha}Alemanha foram agrupadas em torno de fronteiras engenhosamente justificadas por meio da \textit{invenção de tradições} \cite{hobsbawm_prodtrad_2012} e de um \index{nacionalismo}nacionalismo reforçado por \index{nacionalismo!nacionalismo linguístico}argumentos linguísticos e \index{nacionalismo!racismo}``raciais'' \cite{hobsbawm_transfnac_2011}, criam para si não apenas os instrumentos necessários a uma coesão social de tipo novo, cujos efeitos seriam percebidos na \index{imperialismo!Conferência de Berlim (1884-1885)}Conferência de Berlim, mas igualmente a justificativa para uma ``vontade de poder'' de caráter expansionista \cite{rocker_nacult_1954} cujos efeitos mais nefastos só seriam percebidos quase meio século depois. A arquitetura pública e os monumentos cívicos tiveram papel fundamental nesta invenção de tradições, na medida em que a ornamentação dos prédios, o simbolismo dos monumentos, a escolha dos motivos, tudo, enfim, remetia às tradições que se pretendia inventar ou reinventar \cite{hobsbawm_prodtrad_2012}.

Nos \index{EUA}EUA, as mesmas teorias \index{nacionalismo!nacionalismo linguístico}linguísticas e \index{nacionalismo!racismo}``raciais'' se somaram, para os mesmos efeitos, à ideologização das \textit{guerras de fronteira com os índios} como um dos fundamentos da democracia \cite{turner_frontier_1920} e à vaga noção de um \index{nacionalismo!``destino manifesto'' (teoria)}``destino manifesto'', de caráter dito ``civilizatório'', dos estadunidenses de ascendência anglo-saxã \cite{brown_mandest__1980,horsman_mandest_1981}, envolvendo com este véu ideológico a \index{imperialismo!Guerra Hispano-Americana}\textit{Guerra Hispano-Americana} (1898), as \index{imperialismo!Guerras das Bananas}\textit{Guerras das Bananas} (1898-1934) e a consequente conquista de Cuba, das Filipinas, de Porto Rico e da ilha de Guam.

\subsubsection{Crise de 1873-1896}

Outra fator de extrema relevância: a \textit{\index{crises econômicas!crise de 1873-1896}crise de 1873-1896}, duas décadas de estagnação financeira e comercial menos conhecidas que a \index{crises econômicas!crise de 1929}crise de 1929, mas igualmente importantes na história econômica global \cite{Fels1949,Fels1951,hobsbawm_empire_1989,Musson1959,Rezneck1950,Sprague1910,Persons1920}. 

As indenizações de guerra impostas à \index{França}França pela \index{Alemanha}Alemanha como resultado da \index{Guerra Franco-Prussiana}Guerra Franco-Prussiana baquearam severamente a \index{especulação imobiliária}especulação imobiliária em Paris (via \index{Georges-Eugène Haussmann}\index{Haussmann|seealso{Georges-Eugène Haussmann}}Haussmann), enquanto na \index{Alemanha}Alemanha os bancos e bolsas de valores, por sua vez, lançaram-se à especulação desenfreada graças ao dinheiro destas indenizações; é o tempo a que os alemães chamam \textit{\index{Alemanha!Gründerzeit@\textsl{Gründerzeit}}Gründerzeit}, ou seja, o ``tempo dos empresários'', quando mesmo as mais descaradamente fraudulentas iniciativas e os mais absurdos negócios encontravam investidores com enorme facilidade \cite[p.~61]{hobsbawm_capital_1977}. Na \index{Áustria}Áustria, grande parceira econômica da \index{Alemanha}Alemanha, o influxo de dinheiro criou condições para a reconstrução de \index{Áustria!Viena}Viena, e para a \index{especulação imobiliária}especulação imobiliária sem precedentes que levou ao primeiro \index{crises econômicas!crack@\textit{crack}}\textit{crack} da série, em 8 de maio de 1873, rapidamente espraiado pelas bolsas de \index{Alemanha!Berlim}Berlim e \index{França!Paris}Paris.

Nos \index{EUA}EUA, a bolha criada pela enorme \index{transportes!expansão ferroviária|seealso{ferrovias}}expansão ferroviária estourou: um dos grandes financiadores do Norte na \index{EUA!Guerra de Secessão}Guerra de Secessão, o banco \index{carteis e trustes!EUA!Jay Cooke \& Co.}Jay Cooke \& Co., abriu falência em 18 de setembro de 1873, como consequência de sua dificuldade em negociar ações da \index{transportes!ferrovias!Northern Pacific Railroad}Northern Pacific Railroad após mais uma derrota do exército estadunidense para os \index{EUA!Oeste!\textit{hunkpapa}}\textit{hunkpapa} liderados por \index{EUA!Oeste!Touro Sentado}Touro Sentado \cite[p.~241-242]{utley_frontier_1973}; a falência desencadeou uma crise na \index{bolsas de valores!Bolsa de Nova Iorque}Bolsa de Nova Iorque, fechada por dez dias para conter as quebras. Várias companhias ferroviárias e bancos americanos faliram. 

Como a especulação em torno das ferrovias estadunidenses envolvia a \index{indústria!siderurgia}siderurgia e diversos bancos alemães, britânicos e franceses, já combalidos pela \index{crises econômicas!crise austríaca}crise austríaca, a crise voltou a se espalhar pelo globo, e retornou com força total noutros dois \index{crises econômicas!cracks em 1882 e 1884}\textit{cracks} em 1882 e 1884.

Para piorar a situação, embora o comércio e as finanças vivessem um período de franca \index{crises econômicas!depressão}depressão, as inovações técnicas na \index{indústria!produção industrial}produção industrial pautaram um ritmo crescente na \index{produtividade}produtividade \cite{hobsbawm_empire_1989}, seja nos setores considerados \textit{\index{condições gerais da produção}condições gerais da produção} \cite[p.~155-162]{BERNARDO1991}, seja em setores de menor impacto sobre os processos produtivos. Um exemplo: a produção de \index{matérias-primas!ferro}ferro nos cinco maiores países produtores passou de 11 milhões de toneladas para 23 milhões entre 1870 e 1890, enquanto a produção global de \index{matérias-primas!aço}aço no mesmo período saltou de meio milhão de toneladas para 11 milhões \cite[p.~35]{hobsbawm_empire_1989}.

Uma das consequências da crise destes anos foi a \index{onda migratória}\textit{onda migratória da} \index{Europa}\textit{Europa para a }\index{América}\textit{América}, sensivelmente aumentada na década de 1880 (cf. \autoref{tab:emigra} e \autoref{tab:imigra}).

\begin{table}[!htp]
\centering
\IBGEtab{
\caption{Emigração europeia 1846-1930, por país de origem (em milhões de pessoas)}\label{tab:emigra}}
{\begin{tabular}{cccccc}
\toprule
Ano & Total & Reino Unido & Espanha e Portugal & Alemanha e Áustria & Outros \\
\midrule \midrule
1846-1850 & 0,5 & 0,2 & -- & 0,2 & 0,1 \\
1851-1860 & 2,2 & 1,3 & 0,85 & 0,65 & 0,2 \\
1861-1870 & 2,6 & 1,6 & 0,1 & 0,7 & 0,2 \\
1871-1880 & 3,1 & 1,85 & 0,15 & 0,75 & 0,35 \\
1881-1890 & 7,0 & 3,25 & 0,75 & 1,8 & 1,2 \\
1891-1900 & 6,2 & 2,15 & 1,0 & 1,25 & 1,8 \\
1901-1910 & 11,3 & 3,15 & 1,4 & 2,6 & 4,15 \\
1911-1920 & 7,6 & 2,6 & 1,7 & 0,9 & 2,4 \\
1921-1930 & 6,6 & 2,15 & 1,6 & 1,1 & 1,75 \\
\bottomrule
\end{tabular} }
{ \fonte{Elaboração do autor, com dados recolhidos n'\textbf{A economia política do imperialismo} de Michael \citeonline[p.~127]{brown_imper_1978}. } }
\end{table}


\begin{table}[!htp]
\centering
\IBGEtab{
\caption{Imigração em países de colonização europeia 1846-1930, por país de destino (em milhões de pessoas)}\label{tab:imigra}}
{\begin{tabular}{|ccccccc|}
\hline
Ano & Total & EUA & Canadá & Argentina e Brasil & Austrália e Nova Zelândia & Outros \\
\hline
1846-1850 & 1,6 & 1,25 & 0,25 & -- & -- & 0,1 \\
1851-1860 & 3,4 & 2,6 & 0,3 & 0,05 & 0,05 & 0,4 \\
1861-1870 & 3,4 & 2,3 & 0,3 & 0,2 & 0,2 & 0,4 \\
1871-1880 & 4,0 & 2,8 & 0,2 & 0,5 & 0,2 & 0,3 \\
1881-1890 & 7,5 & 5,2 & 0,4 & 1,4 & 0,3 & 0,2 \\
1891-1900 & 6,4 & 3,7 & 0,2 & 1,8 & 0,45 & 0,25 \\
1901-1910 & 14,9 & 8,8 & 1,1 & 2,45 & 1,6 & 0,95 \\
1911-1920 & 11,1 & 5,7 & 1,1 & 2,0 & 1,0 & 1,3 \\
1921-1930 & 8,7 & 4,0 & 1,0 & 2,15 & 0,7 & 0,85 \\
\hline
\end{tabular} }
{ \fonte{Elaboração do autor, com dados de \citeonline[p.~127]{brown_imper_1978}.} }
\end{table}

Outra das consequências da crise foi a confirmação do \index{declínio da hegemonia britânica!Grã-Bretanha}declínio da hegemonia britânica sobre a economia global; embora o papel dos capitais britânicos na economia global ainda fosse inquestionável \cite{goetzmann_british_2006,rippy_britlat_1954,stone_british_1977}, sua produção industrial, embora crescesse, fazia-o em taxas declinantes, e o desenvolvimento de suas indústrias nos ramos de ponta como a química e a engenharia elétrica estava claramente a reboque do que se fazia na \index{Alemanha}Alemanha e nos \index{EUA}EUA \cite[p.~207]{Musson1959}. 

Uma última consequência da crise foi a criação dos \index{carteis|seealso{carteis e trustes}}\textit{carteis} na \index{Alemanha}Alemanha, dos \index{trusts|seealso{carteis e trustes}}\textit{trusts} nos \index{EUA}EUA e de outras formas de organização monopolista de empresas, apontadas por \citeonline[p.~125-200]{hobson_capitmoderno_1983}, \citeonline[p.~16-29]{lenin_imperialismo_1987}, e \citeonline[p.~47-54]{bukharin_imperialismo_1986} como características de uma nova fase da economia mundial, precursoras daquilo que viriam a ser as grandes corporações e empresas do primeiro \index{pós-guerra}pós-guerra e também dos grandes conglomerados transnacionais do segundo \index{pós-guerra}pós-guerra.

\subsection{Imperialismo, colonialismo, carteis, trustes e a Primeira Guerra Mundial (1881-1914)}

Somente depois de entender estas preliminares é possível entender o imperialismo de um ponto de vista \textit{histórico}. Em que pesem as considerações de um \index{Schumpeter}Schumpeter sobre incompatibilidades entre capitalismo e imperialismo, lançando o último na conta dos restos do absolutismo monárquico que persistiam no período \cite{schumpeter_imperialismo_1961}, vista a questão pelo ponto de vista histórico, o imperialismo foi, no campo dos Estados-nação, uma das saídas encontradas para a crise dos anos 1870-1880, assim como os carteis e os trustes foram saídas para a mesma crise encontradas no campo das empresas; são dois lados da mesma moeda, funcionaram juntos e um não pode ser compreendido sem remissão ao outro.

Somente agora a \index{imperialismo!Conferência de Berlim (1884-1885)}Conferência de Berlim passa a fazer sentido como marco histórico de um período. Se anteriormente à crise dos anos 1870-1880 a relação dos países europeus com as sociedades africanas oscilou entre o comércio e a guerra de conquista \cite{ogot_hisaf5_2010,AJAYI2010}, neste período passam a ser de dominação territorial e anexação. É certo que fatores endógenos aos próprios Estados africanos, como as diversas lutas interestatais e intraestatais, e a superioridade europeia em diversos aspectos conjunturais (conhecimento geográfico do continente, saber médico contra doenças endêmicas e capacidade de sustentar guerras prolongadas), facilitaram enormemente a conquista \cite{uzoigwe_partilha_2010}, mas ela foi seguida por uma ferrenha e encarniçada resistência onde quer que as administrações coloniais hajam sido instaladas \cite[p.~51-318]{boahen_hisaf7_2010}. Com particularidades próprias e algumas diferenças marcantes, algo parecido se pode dizer do avanço da dominação europeia sobre a Ásia \cite{panikkar_domasia_1977}. 

Diante da necessidade premente de conquistar mais matérias-primas e mercados para seus produtos industriais num contexto de severa depressão comercial, o avanço sobre a África e a Ásia, já anteriormente pontilhada por colônias de todo tipo, pareceu uma decorrência ``natural'' da disputa por novas colônias; o princípio da ``ocupação efetiva'' e a divisão da África em 50 colônias coroou o processo. Não por acaso, uma economista do porte de Rosa Luxemburg teorizou sobre a necessidade constante do capitalismo de recorrer a sociedades não-capitalistas para ``fechar as contas'' de sua reprodução ampliada; a \index{imperialismo!partilha da África}partilha da África, conquanto representasse uma mudança de perfil, inseria-se numa longa história de saques, pilhagens, ataque à chamada ``economia natural'' (ou seja, o feudalismo, a economia camponesa, a economia doméstica etc.), de que a colonização e a ocupação territorial direta seriam apenas o corolário \cite{luxemburg_acumula_1985}.

A longa história da formação das colônias e da resistência anticolonial é certamente apaixonante, mas, dado o fato de o Brasil ser país politicamente independente desde 1822 e de este novo modelo de colonialismo pouco ter afetado a América do Sul, apenas dois traços do imperialismo interessam à presente pesquisa.

Os \index{imperialismo!monopólios}\textit{monopólios} são o primeiro. No nascedouro do moderno sistema de crédito e da sociedades por ações, Karl Marx vira como simples tendência: \textit{(a)} que as sociedades por ações expandissem imensamente a escala de produção das empresas a patamares impossíveis de serem atingidos por capitais isolados, levando assim à constituição de sociedades por ações em ramos onde antes predominavam companhias governamentais (p. ex., companhias majestáticas como a Companhia de Moçambique, a Companhia do Niassa e as Companhias das Índias Orientais criadas pela França, pela Inglaterra e pelos Países Baixos); \textit{(b)} que, portanto, o capital assumiria a forma de \textit{capital social} em oposição ao \textit{capital privado}, e as empresas passariam a ser sociais em contraste com as empresas privadas, promovendo algo como ``a abolição do capital como propriedade privada dentro dos limites do próprio modo capitalista de produção''; \textit{(c)} que o capitalista realmente ativo seria transformado em mero dirigente, administrador do capital alheio, e os proprietários de capital passariam a ser puros proprietários, ``simples capitalistas financeiros''; \textit{(d)} tanto as empresas capitalistas quanto as cooperativas industriais de trabalhadores eram beneficiadas pelo moderno sistema de crédito, e deveriam ser, para Marx, consideradas ``formas de transição entre o modo capitalista de produção e o modo associado'' -- ou seja, o socialismo -- com a diferença de que, num caso, a contradição entre a propriedade privada dos meios de produção, típica do capitalismo livre-concorrencial, e a propriedade social dos meios de produção, típica do modo de produção então esboçado, era superada negativamente por meio das sociedades por ações, e positivamente no caso das cooperativas industriais de trabalhadores \cite[p.~581-588]{MARX2008a}. Em adendo a este mesmo capítulo, Friedrich Engels ligou esta teoria de seu amigo aos \index{carteis|seealso{carteis e trustes}}\textit{carteis} internacionais e à concentração de toda a produção de determinado ramo industrial numa só sociedade por ações com direção única \cite[p.~584]{MARX2008a} -- o que, para ser um \index{trustes|seealso{carteis e trustes}}\textit{trust}, só precisa do nome. Se a previsão teórica de Marx e Engels a respeito da sociedade por ações como elemento de transição para um novo regime, vista com os olhos de hoje, se mostrou falha, acertou, não obstante, no que diz respeito à separação entre administradores diretos e acionistas. 

É esta teoria de Marx e Engels sobre as sociedades por ações, os \textit{trusts} e os cartéis que levou Lenin, leitor fiel de ambos, a especular sobre o caráter terminal para o capitalismo decorrente do próprio imperialismo -- embora, em termos políticos, sua teoria divergisse de seus mestres quanto ao lugar onde se dariam as primeiras fissuras \cite{lenin_imperialismo_1987}.

Fosse como fosse, os \index{trustes|seealso{carteis e trustes}}\textit{trustes} e \index{carteis|seealso{carteis e trustes}}\textit{carteis} eram uma realidade incontornável. Nos \index{EUA}EUA formaram-se trustes como a \index{carteis e trustes!EUA!Standard Oil Co.}\textit{Standard Oil Co.} de \index{John D. Rockefeller}John D. Rockefeller, a \textit{\index{carteis e trustes!EUA!American Tobacco Co.}American Tobacco Co.} de \index{J. B. Duke}J. B. Duke, a \textit{\index{carteis e trustes!EUA!US Steel}US Steel} de \index{John Pierpont Morgan}John Pierpont Morgan. Na \index{Alemanha}Alemanha desenvolveram-se grandes empresas como a \index{carteis e trustes!Alemanha!RWE}\textit{RWE} de Hugo Stinnes; a \index{carteis e trustes!Alemanha!BASF}\textit{BASF} de Friedrich Engelhorn; a \index{carteis e trustes!Alemanha!AEG}\textit{AEG} de Emil Rathenau; a \index{carteis e trustes!Alemanha!Siemens \&Halske}\textit{Siemens \& Halske} e a \index{carteis e trustes!Alemanha!Siemens-Schuckert}\textit{Siemens-Schuckert} fundadas por Werner von Siemens; a \index{carteis e trustes!Alemanha!Friedrich Krupp AG}\textit{Friedrich Krupp AG} da quadricentenária família Krupp; a \index{carteis e trustes!Alemanha!Borsig-Werke}\textit{Borsig-Werke} fundada por August Borsig...  Todas surgiram em ambientes protegidos da influência de indústrias de outros países -- especialmente as indústrias inglesas -- por meio de tarifas protecionistas, onde os capitalistas com maior capacidade de mobilização política e maiores reservas de capital tomaram progressivamente o espaço de capitalistas menores, incorporando suas empresas ou simplesmente levando-as à falência.  \cite{bukharin_imperialismo_1986,huberman_historia_1986}.  Na medida em que um \textit{trust} ou um \textit{cartel} regula a oferta para estabelecer a procura econômica, precisa ou paralisar parte de sua infraestrutura produtiva, deixando-a ociosa, ou procurar outros mercados -- e a política externa colonialista ofereceu exatamente as condições necessárias para a busca de novos mercados \cite{lenin_imperialismo_1987}. 

A \index{imperialismo!exportação de capitais}\textit{exportação de capitais} é o segundo traço do imperialismo relevante para a presente pesquisa. Decorre do alto nível de concentração de capitais nos bancos, resultante dos lucros dos \textit{trustes} e \textit{carteis}; este capital concentrado foi empregue no financiamento ao desenvolvimento de infraestruturas básicas nas novas colônias ou em países ditos ``atrasados'', gerando, assim, retorno do capital emprestado, uma vez que as ferramentas, máquinas etc. necessários à construção destas infraestruturas eram comprados das mãos dos monopolistas \cite{huberman_historia_1986,lenin_imperialismo_1987,luxemburg_acumula_1985}. 

A \index{Primeira Guerra Mundial}Primeira Guerra Mundial surge neste cenário onde a aparente harmonia das fusões e incorporações empresariais ocultou uma violentíssima concorrência entre empresas, assim como o sistema internacional bipartite da Tríplice Aliança e da Tríplice Entente estabeleceu um frágil e tenso equilíbrio nas disputas entre Estados-nação. Em ambos os casos, o expansionismo imperialista -- pela via do colonialismo e da exportação de capitais -- envolvia seus partícipes em sucessivas crises, das quais o \index{Primeira Guerra Mundial!assassinato do arquiduque Francisco Ferdinando}assassinato do arquiduque Francisco Ferdinando foi apenas a última\footnote{\citeonline{howard_guerra_2013} e \citeonline{shirer_queda1_1969} citam várias: a \index{Primeira Guerra Mundial!crise de Tânger}\textit{crise de Tânger} (1905-1906), quando \index{França}França e \index{Grã-Bretanha}Grã-Bretanha enfrentaram a \index{Alemanha}Alemanha no campo diplomático em torno da influência sobre o Marrocos; a \index{Primeira Guerra Mundial!crise bósnia}\textit{crise bósnia} (1908-1909), quando o \index{Áustria!Império Austro-Húngaro}Império Austro-Húngaro anexou a Bósnia e semeou animosidade na região inteira; a \index{Primeira Guerra Mundial!crise de Agadir}\textit{crise de Agadir} (1911), quando \index{França}França e \index{Alemanha}Alemanha quase se enfrentam militarmente, mais uma vez disputando os rumos e a influência política sobre o Marrocos; a \index{Primeira Guerra Mundial!Guerra Ítalo-Turca}\textit{Guerra Ítalo-Turca} (1911-1912), quando a derrota do \index{Império Otomano}Império Otomano para a \index{Itália}Itália agitou os nacionalistas dos países da Liga Balcânica (Sérvia, Grécia, Montenegro e Bulgária) e, como consequência, serviu de estopim para a \index{Primeira Guerra Mundial!Primeira Guerra dos Bálcãs}\textit{primeira} (1912-1913) e a \index{Primeira Guerra Mundial!Segunda Guerra dos Bálcãs}\textit{segunda Guerra dos Bálcãs} (1913).}. Para os fins desta dissertação, a Primeira Guerra Mundial só interessa por dois motivos: ela é o marco histórico da passagem da hegemonia política e econômica global para os EUA, e além disso oportunizou aos capitalistas brasileiros, agrários ou industriais, certas tendências que serão vistas adiante, em momento oportuno.

\subsection{Revoluções, os anos 1920 e a crise de 1929}

É muito natural, depois de destruídas as infraestruturas econômicas e massacrada a população por uma guerra cruenta como foi a de 1914-1918, que o esforço de reconstrução criasse certa euforia pelo novo e certa alegria pela nova vida. Para outros países, entretanto, as consequências da Primeira Guerra Mundial foram mais graves, e são de suma importância para entender algumas questões candentes do período. O fundamental foi resumido em 1919 por John Maynard Keynes:

\begin{citacao}
O Tratado de Paz [\index{Primeira Guerra Mundial!Tratado de Versalhes}\textit{Tratado de Versalhes}] não contém qualquer disposição orientada para a reabilitação econômica da Europa - nada que transforme as Potências Centrais derrotadas em bons vizinhos, nada que permita dar estabilidade aos novos Estados europeus, nada para salvar a Rússia; não promove de nenhuma forma um pacto de solidariedade econômica entre os próprios aliados. Em Paris nada se fez para restaurar as finanças desordenadas da França e da Itália, ou para ajustar os sistemas do Velho e do Novo Mundo.

O Conselho dos Quatro não se preocupou com esses temas, mas sim com outros - Clemenceau queria esmagar a economia do inimigo, Lloyd George conseguir um acordo para levar consigo a Londres, e exibi-lo durante uma semana, Wilson nada fazer que não fosse justo e correto. É um fato extraordinário, mas o problema econômico fundamental de uma Europa esfomeada que se desintegrava diante dos seus olhos era a única questão para a qual foi impossível provocar o interesse dos Quatro \cite[p.~157]{keynes_paz_2002}
\end{citacao}

A impressão de Keynes pode parecer exagerada quando contraposta às realizações da \textit{Liga das Nações}. Entretanto, enquanto funcionou, a Liga teve atuação medíocre. Se as colônias dos países derrotados ficaram sob sua responsabilidade, os mandatos conferidos às potências vencedoras para administrá-las servia como carta branca para uma colonização de fato. A mediação da Liga em questões de delimitação de fronteiras, quando não acirrou tensões pré-existentes\footnote{São elas: Albânia, 1912-1923; Alta Silésia, 1921-1923; Klaipéda (na atual Lituânia), 1923.}, foi porque tratou de disputas territoriais sem maior impacto nas relações entre as potências hegemônicas do período\footnote{É o caso da questão das ilhas Alanda (na atual Finlândia), 1921.} ou prestou-se ao jogo geopolítico destas potências\footnote{A questão de Mossul (no atual Iraque), entre 1920 e 1926, é exemplar.}. As duas principais medidas para resolver a questão das indenizações de guerra da Alemanha -- os planos Dawes (1925)\footnote{Uma cadeia de eventos iniciada com os empréstimos contraídos pelo \textit{kaiser} Guilherme II para custear o esforço de guerra e pela desvalorização do \textit{marco alemão} por força das indenizações impostas pelo Tratado de Versalhes resultou na moratória alemã de 1923, quando a hiperinflação tornou impossível pagar as indenizações de guerra. França e Bélgica retaliaram ocupando o vale do Ruhr, tradicional área carbonífera, siderúrgica e metalúrgica da Alemanha, para obter diretamente as indenizações sob a forma de mercadorias. Enquanto a população local protestava com greves e passeatas, resultando em 130 civis mortos pelas tropas de ocupação, o governo alemão emitiu moeda desenfreadamente, acelerando o processo hiperinflacionário já existente, desvalorizando ainda mais o marco. Ao final de 1923, um dólar estadunidense podia ser trocado por 4.210.500.000.000,00 marcos. Uma comissão chefiada pelo banqueiro estadunidense Charles Gates Dowes propôs em agosto de 1924 às potências aliadas, para resolver a situação ou amenizá-la, as seguintes medidas: \textit{(a)} evacuação das tropas aliadas do vale do Ruhr; \textit{(b)} o pagamento das reparações de guerra seria reiniciado no valor de um bilhão de marcos no primeiro ano do plano, aumentando progressivamente até o patamar de dois milhões e meio de marcos no decurso de cinco anos; \textit{(c)} o \textit{Reichsbank} seria reorganizado sob supervisão das potências aliadas; \textit{(d)} as indenizações poderiam ser pagas com recursos vindos de tarifas aduaneiras e tributos de circulação de mercadorias ou sobre produtos específicos; \textit{(e)} a Alemanha tomaria emprestados 800 milhões de marcos dos EUA. O programa trouxe benefícios de curto prazo à combalida economia alemã, mas pode ser entendido como uma das causas do espraiamento global da crise econômica de 1929 se se levar em conta que as indenizações de guerra, alimentadoras das economias das potências vencedoras da Primeira Guerra Mundial, eram pagas principalmente com dinheiro emprestado pelos EUA \cite[p.~85]{carr_relations_1937}.} e Young (1929)\footnote{Tentativa de superar problemas criados pelo plano Dowes, frustrada pela crise de 1929.} -- foram feitas à sua revelia. Os \textit{Tratados de Locarno}, negociados entre 5 e 16 de outubro de 1925 entre representantes dos governos alemão, francês, belga, britânico e italiano para encerrar definitivamente as disputas territoriais do pós-guerra na Europa, foram igualmente feitos à sua revelia, ainda que tenham resultado na incorporação da Alemanha à Liga. A Liga fracassou principalmente em sua missão central: o desarmamento da Europa. Contudo, sua atuação entre 1924 e 1930 é tida como o seu apogeu  -- e não poderia ser diferente, dado o desprezo a que foi relegada a partir da segunda metade da década de 1930 \cite{carr_relations_1937,carr_crisis_1981}.

O relativo fracasso de uma organização internacional dos capitalistas foi contemporâneo, curiosamente, do fracasso retumbante de uma revolução de trabalhadores em escala europeia, cujos sintomas se multiplicavam. Para começar, há severos indícios de que o alardeado patriotismo militarista alemão verificado imediatamente antes da votação dos créditos de guerra no \textit{Reichstag} esteve restrito à intelectualidade alemã e aos militantes socialistas já integrados no \textit{status quo} parlamentar e sindical, e não aos trabalhadores \cite{broue_german_2005,watson_german_2011}. Durante a guerra, os soldados de ambos os lados do conflito desenvolveram o estranho sistema de convivência entre tropas confrontantes conhecido como ``viva e deixe viver'' (\textit{live and let live}) \cite{ashworth_live_1980}\footnote{Já em 1914 ocorreram na terra de ninguém do front ocidental confraternizações entre tropas adversárias, especialmente no Natal, e durante toda a guerra as tropas evitaram o quanto puderam o confronto direto em dadas situações, num sistema de convivência conhecido na época como ``viva e deixe viver'' (\textit{live and let live}), na verdade uma ``passividade armada'' muito bem calculada entre pequenas unidades confrontantes de exércitos inimigos \cite{ashworth_live_1980}. Robert Axelrod, lendo tais práticas pela ótica da teoria dos jogos, resumiu-as desta forma: ``Durante períodos de contenção mútua de agressões, soldados inimigos esforçavam-se para mostrar uns aos outros que podiam retaliar se necessário. Por exemplo, franco-atiradores alemães mostravam sua habilidade ao mirar em pontos de muros de chalés e atirar até abrirem um buraco [\dots]. Da mesma forma, a artilharia às vezes demonstrava que com alguns poucos tiros acuradamente mirados poderia causar mais dano se assim o quisesse. [\dots] Estas demonstrações de capacidades retaliatórias [\dots] ajudavam a policiar o sistema [‘viva e deixe viver’] ao demonstrar que a contenção não se devia a fraqueza, e que a desistência [de agir dentro do sistema ‘viva e deixe viver’] pelo outro lado resultaria em sua própria derrota'' \cite[p.~79-80]{axelrod_cooperation_2006}.}; a completa desagregação do exército russo na virada de 1916 para 1917, quando centenas, milhares de soldados, majoritariamente camponeses, desertavam para voltar às suas terras \cite{trotsky_revrus01_1977}. Diante do recrudescimento dos combates, da mortandade nas trincheiras e do verdadeiro empate em que os \textit{fronts} se encontravam, paralisados que estavam nas mesmas linhas há meses com pouquíssimo avanço para um lado ou para o outro, os soldados franceses iniciaram uma onda de motins em 1916 \cite{masson_franceses_2008} que repercutiu muito fortemente nos escalões superiores do exército francês -- pautado, assim, a encontrar novas táticas e tecnologias para vencer a guerra contra os alemães e a guerra contra sua própria tropa insubmissa. 

As tropas de linha da Primeira Guerra Mundial eram em sua maioria formadas por trabalhadores e camponeses, e os horrores da guerra de trincheiras correram de boca a ouvido entre eles e seus parentes na retaguarda, deixando trabalhadores civis perplexos com as notícias lidas nas entrelinhas do que era permitido pela censura oficial. Graças a isto, as rebeliões no \textit{front} relacionaram-se comutativamente com uma intensa onda de greves movidas quase simultaneamente por camponeses e trabalhadores urbanos de quase todos os países beligerantes, sem qualquer limitação de fronteiras e com demonstrações de solidariedade internacional de classe contra a guerra. De 1916 a 1917 as greves na França aumentaram em 600\% e a quantidade de trabalhadores envolvidos nas paralisações chegou a quase trezentos mil; em 1916, o número de dias de trabalho perdidos por greve na Alemanha aumentou 500\% em relação a 1915, e em 1917 aumentou 700\% em relação a 1916; na Grã-Bretanha as importantes greves de 1916 e 1917 marcaram o início do movimento dos \textit{shop stewards} (``delegados de fábrica'') e em 1918 eclodiram motins entre tropas britânicas estacionadas na França. A lista de movimentos contra a guerra a partir de 1916 é tão espantosa quanto sua simultaneidade e sua radical internacionalização, e a passagem do caráter puramente antibelicista dos primórdios a um caráter classista, internacionalista e anticapitalista, se pode ser explicado pela incansável propaganda de socialistas e anarquistas acerca de um outro mundo além da exploração e da opressão a que os trabalhadores se viam cotidianamente sujeitos sob o capitalismo, pode também ser explicado pela percepção eminentemente prática dos combatentes de que estavam lutando uma guerra que não era sua, e da qual nada aproveitariam \cite[p.~232-251]{bernardo_fascismo_2015}.

Desta forma, se na década de 1920 ocorreram revoluções aparentemente tributárias da Revolução Russa, trata-se na verdade das últimas manifestações de um ciclo internacional de revoluções, levantes, insurreições, motins, greves etc. dos quais a Revolução Russa é apenas o momento mais drástico \cite[p.~616]{bernardo_fascismo_2015}, e também o primeiro momento em que esta difusa revolução internacional dos trabalhadores assumiu paulatinamente, em especial após a assinatura do tratado de Brest-Litovsk em 1918, feições nacionalistas que lhe eram originalmente estranhas \cite[p.~618-620]{bernardo_fascismo_2015}. A atuação da \textit{Internacional Comunista} (1919-1943) nos anos 1920 e 1930, conquanto inserida na vaga revolucionária que ainda sacudiu o mundo até 1923\footnote{São tantos os levantes, insurreições e revoluções deste período que é possível agrupá-los segundo as tendências ideológicas e objetivos de cada um. No \textit{primeiro grupo}, há as revoluções com participação ativa da Internacional Comunista: Revolução Húngara (1918-1920), Revolução Alemã (1918-1923), República Soviética Bávara (1919), \textit{bienio rosso} italiano (1919-1920), o levante de Albona (1921), o levante búlgaro de setembro de 1923 e a primeira fase da Guerra Civil Chinesa (1927-1937). No \textit{segundo grupo}, há os movimentos de esquerda, mas não necessariamente vinculados à Internacional Comunista (e às vezes mesmo contrários a ela): a revolta de Kronstadt (1921) e a Revolução Ucraniana (1918-1922). No \textit{terceiro grupo}, há revoluções nacionalistas como a mexicana (1910-1920), a grega (1919-1922), a irlandesa (1919-1921), a a maltesa (1919) e a egípcia (1919).}, já representava a tensa submissão do movimento revolucionário internacional à geopolítica da União Soviética. Não por acaso, já em 1922 o tratado de Rapallo, firmado entre a Alemanha (sob o regime democrático weimariano) e a URSS, representou tanto o primeiro passo para consolidar a tática do ``socialismo num só país'' como também o primeiro reconhecimento formal da URSS por qualquer das grandes potências internacionais. A URSS seria reconhecida, a seguir, pela Grã-Bretanha, Itália, França, Japão e pela maioria dos países europeus, demonstrando que a aceitação das regras das relações internacionais pela sua classe dirigente era apenas uma questão de tempo \cite[p.~72-77]{carr_relations_1937}.

Diante deste quadro de crise nas relações internacionais e de acirrados conflitos de classe, como explicar, então, a euforia que levou os franceses a chamar a década de 1920 de \textit{les annes foulles} (``os anos loucos''), os estadunidenses, de \textit{roaring twenties} (``os ribombantes anos vinte''), e mesmo os depauperados alemães a chamar esta década de \textit{Goldene Zwanziger} (``os dourados anos vinte'')? Como se incluiriam neste entusiasmo obras artísticas tão angustiantemente secas e sombrias como as de Otto Dix, Käthe Kollwitz, Ernst Barlach e Erich Maria Remarque, ou inquietantes como as de Raoul Hausmann, Christian Schad, Karl Hubbuch, Max Beckmann e George Grosz? Uma obra-prima como \textbf{Berlin Alexanderplatz}, de Alfred Döblin, misto de literatura e observação participante entre o \textit{lumpenproletariat} do leste berlinense, só pode ser otimista por antífrase \cite{doblin_alexanderplatz_2009}, assim como \textbf{Um homem sem qualidades}, de Robert Musil \cite{musil_quali_1989}.

Explica-se. A crise de 1929 é ainda tema de debate quente na historiografia econômica e na economia. Pode-se dizer, em linhas gerais e sem entrar no debate das correntes explicativas, que suas causas são: \textit{(a)} os desequilíbrios orçamentários no pós-guerra, alguns dos quais já vistos anteriormente nesta subseção; \textit{(b)} uma crise agrícola mundial, agravada pela urbanização acelerada; \textit{(c)} a superprodução de bens de consumo durável, resultante do desenvolvimento da produção em massa (e da maior sujeição dos trabalhadores à disciplina fabril por meio de técnicas tayloristas de gestão); \textit{(d)} o baixo consumo dos bens assim produzidos, reduzindo seus preços; \textit{(e)} bolhas especulativas formadas em torno da aviação, da radiodifusão e da indústria automobilística. Todas estas tendências se desenvolveram na segunda metade da década de 1920, e, combinadas, resultaram na quase paralisia da produção mundial \cite{gazier_1929_2009,hautcoeur_1929_2009}.

Por que é que a crise de 1929 se internacionalizou tão rápida e avassaladoramente? A Alemanha dependia dos empréstimos da banca norte-americana para pagar as reparações de guerra aos franceses e britânicos; estes últimos precisavam dessas reparações para pagar os empréstimos, contraídos durante a guerra, à banca norte-americana; esta última precisava desses pagamentos para proceder a empréstimos aos alemães. A crise num dos vértices deste triângulo, os Estados Unidos, comprometeu toda esta circulação financeira internacional. A aparente abundância em algumas regiões do globo impediu a vasta maioria daqueles que dela se beneficiaram de perceber os sinais da crise.

\subsection{Urbanização: dialética do caos e da ordem em tempos turbulentos}

É comum que a literatura sobre o urbanismo lance sobre a \textit{industrialização}, enquanto síntese de um fenômeno mais complexo, a responsabilidade sobre o crescimento desordenado das cidades no século XIX, problema para cuja solução o urbanismo teria sido criado enquanto disciplina separada do saber. 


