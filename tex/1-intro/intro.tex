% ----------------------------------------------------------
% Introdução (exemplo de capítulo sem numeração, mas presente no Sumário)
% ----------------------------------------------------------
\chapter[Introdução]{Introdução}\label{ch:intro}

A pesquisa apresentada nesta dissertação pretende compreender como os conflitos sociais interferiram na produção, apropriação e uso do espaço urbano no distrito soteropolitano de Brotas na Primeira República. Para isto, que é seu objetivo geral, passou pelas seguintes etapas:

\begin{enumerate}
\item Caracterizar em linhas gerais a sociedade e o território soteropolitanos na Primeira República, identificando os impactos das reformas de J. J. Seabra sobre este território.
\item Identificar e caracterizar as principais medidas adotadas durante as reformas de J. J. Seabra e os agentes de produção do espaço urbano neste período.
\item Recompor, em linhas gerais, a história territorial do distrito de Brotas, e identificar nela conflitos sociais recorrentes, assim como a produção, apropriação e uso de seu espaço.
\item Identificar, na literatura especializada sobre o governo J. J. Seabra e a reforma urbana que promoveu, o lugar dedicado ao distrito estudado no processo de melhorias urbanas.
\item Identificar os agentes de produção do espaço urbano soteropolitano quando das reformas de Seabra, e rastrear sua atuação no distrito selecionado.
\item Identificar em relatórios do Governo da Bahia e da Intendência Municipal de Salvador o tipo de investimento em infraestruturas urbanas (arruamento, iluminação, transporte, saneamento, limpeza pública) feitas no distrito escolhido durante o período estudado.
\item Vasculhar na imprensa de época o tipo de notícia publicada sobre Brotas, para resgatar a imagem que então se produzia do distrito.
\item Perscrutar todos os pedidos de licença para construção e reforma de imóveis feitos no distrito durante o período estudado, para compreender os padrões constutivos impostos ao distrito no período estudado.
\item Analisar amostragem decenal dos livros de décimas urbanas para identificar os tipos e formas de acesso à terra no distrito durante o período estudado.
\item Identificar, a partir das informações encontradas, os conflitos sociais existentes na Salvador da Primeira República, e como afetaram a produção, apropriação e uso do território urbano no distrito escolhido.
\end{enumerate}

A forma de exposição da pesquisa nesta dissertação será detalhada adiante, e lá se poderá perceber de que modo estes pontos, quando cerzidos, são os elementos de uma tessitura sócio-espacial complexa.

\section{Marco teórico}\label{sec:marcteor}

A teoria fundamentadora da pesquisa proposta neste projeto parte da constatação de duas questões no campo do planejamento e das políticas urbanas: uma de ordem \textit{epistemológica} e outra de ordem \textit{prática}.

Antes, todavia, de entrar no marco teórico propriamente dito, para ajudar a compreender a \textit{episteme}, o método e alguns dos resultados a que se chegou nesta pesquisa é impossível deixar de dizer algo, escrito na despretensão do teclar, sobre a História e os fazeres dos historiadores.

A História é o processo sempre aberto da concretização de um entre muitos mundos possíveis por parte de sujeitos em constante, contraditória e não raro conflituosa inter-relação. Nada no fazer histórico está determinado de antemão; tudo quanto hoje se nos apresenta como fato histórico foi, em algum momento, mera possibilidade, pura virtualidade, cuja passagem do plano hipotético à concretude fatual foi mediada pelas escolhas e entrechoques – alguns momentâneos, outros de mais longa duração; alguns intencionais, outros acidentais – próprios das relações que os sujeitos históricos desenvolvem entre si ao viver e, portanto, ao produzir fatos históricos.

Ocorre que estas escolhas e entrechoques não se dão no vazio; afinal, a experiência nos mostra a todo instante que sujeitos históricos não agem sobre \textit{tabula rasa}. A \textit{prática} deste sujeitos, necessária à sua própria existência enquanto sujeitos, se dá e interfere num mundo simultaneamente material e simbólico que lhes foi legado pelas práticas de outros sujeitos, sejam eles seus antecessores ou contemporâneos; o fazer histórico é, portanto, não somente desprovido de qualquer teleologia, como também é um processo ininterrupto, em constante desenvolvimento.

O estudo dos fatos históricos, por estas razões, não é o simples \textit{recontar de fatos passados}, como, entre outros tantos a desenvolver esta perspectiva, quiseram uma vez os gregos. Em seu panteão, uma das habitantes do monte Parnaso e quiçá a mais célebre das filhas de Zeus e Mnemósine (a memória personificada) é Clio, a musa da História. Seu nome deriva da raiz grega \textgreek{κλέω} que significa ``recontar'', ``tornar famoso'', ``celebrar''. É certamente este o modo mais tradicional de se conceber a História: a rememoração de fatos congelados em eras passadas, o contar novamente os acontecimentos cristalizados, a celebração de eventos passados, o listar dos feitos pretéritos das grandes personagens. É curioso, entretanto, que os gregos houvessem dado como lar à musa da História o monte em cujo pé se localiza Delfos, cidade-oráculo famosa pelos vaticínios de sua pitonisa extasiada por oleandro e fumos vulcânicos. Simbolicamente considerada a coincidência, não quereria a curiosa vizinhança reforçar o – óbvio – fato de o passado ser sobranceiro ao futuro? De os enigmas do futuro estarem sob o olhar de quem enuncia e rememora o que já se passou?

\textit{Estudar os fatos históricos}, nos caminhos a trilhar nesta dissertação, significa \textit{estudar a constituição do presente e as condições dos futuros possíveis}. Se o estudo dos fatos históricos, sob a égide de Clio, assemelha-se mais a uma \textit{cristalografia}, ao estudo de fatos cristalizados no passado, na perspectiva aqui adotada o estudo dos fatos históricos toma ares, digamos, de uma \textit{reologia}, de um estudo dos \textit{fluxos e deformações da matéria da História} – quais sejam, as \textit{práticas} dos sujeitos, os \textit{fatos} que delas resultam, os \textit{conflitos} e \textit{convergências} resultantes do entrecruzamento destas práticas, as \textit{instituições} criadas pela sua reiteração, as \textit{ideologias} que expressam o congelamento das práticas e das instituições em sistemas de significados apartados do fluxo de fatos que lhes deram origem, a \textit{cultura material} expressiva deste magma de significados...

O caminho adotado nesta dissertação lembra de certa forma outro olhar para a História, vindo de cantos que se supunha, até não muito tempo atrás, cegos para o passado. Não o segue irrestritamente, mas lembra-o, compartilha alguns de seus pressupostos. Em Gana e na Costa do Marfim, há entre os \textit{Akan} ideogramas chamados \textit{adinkra}; têm significados complexos, expressos em aforismos ou fábulas plenos de conteúdo filosófico. Um deles, \textit{sankofa}, é o desenho estilizado de um pássaro que olha para trás. Várias versões são dadas para o significado do ideograma: ``voltar e apanhar o que ficou para trás'', ``voltar às raízes'', ``sempre podemos retificar os nossos erros''... Em todas elas, nota-se, mais uma vez, a inescapável ligação entre o passado e o presente; novidade é a \textit{interferência do passado no presente}, para retificá-lo à luz do que já se passou, no mínimo, ou para aprender com os erros pretéritos e, assim, construir o futuro. \textit{Sankofa}, em tudo, dista da clássica imagem grega da História rememorativa, ou da parábola kafkiana, tão cara a Hannah Arendt, do sujeito premido entre dois adversários -- o passado e o futuro -- a empurrarem-no um contra o outro, desejoso do dia em que poderá escapar do campo de luta e tornar-se dela mero espectador travestido de juiz. \textit{Sankofa} em tudo necessita do verbo, pressupõe \textit{ação}, \textit{movimento}, não \textit{contemplação} ou \textit{estagnação}. Em \textit{sankofa} a ação e o movimento não são apenas atributos \textit{do passado rememorado}, mas \textit{de quem rememora}, e o rememorar mesmo é \textit{ação e intervenção interessada sobre o presente}.

Mas \textit{agir e intervir é papel de quem historia}? Outra digressão, a mais díspar entre todas que apresento, joga luzes sobre a questão. No universo Marvel -- sim, quadrinhos são tão carregados de significados quanto qualquer texto dito ``clássico'' -- os Vigias são uma espécie extraterrestre virtualmente imortal, antiga como o universo. Dotados de vastos poderes, podem voar, ler mentes, manipular energia, viajar no tempo, projetar campos de força e ilusões holográficas (o que lhes permite mudar de aparência ao alterar a forma como outros seres os percebem) e seus ``sentidos cósmicos avançados'' lhes permitem estar simultaneamente cientes de incontáveis eventos. Eras atrás, na tentativa de usar seus poderes para o bem, um Vigia forneceu aos habitantes do planeta Prosilicus o conhecimento da energia nuclear; interferiu assim na evolução da vida no planeta, e os prosilicanos, sem plena ciência das consequências do uso desta tecnologia, usaram-na para fins destrutivos. Ao retornarem ao planeta Prosilicus, os Vigias foram acusados pelos prosilicanos sobreviventes de serem os responsáveis pela destruição do planeta ao lhes fornecerem tecnologias para as quais não estavam prontos. Desde então, os Vigias juraram jamais interferir com outras civilizações, limitando-se a observar e registrar sua evolução. Uatu, entretanto, é um estranho Vigia. Filho do Vigia que entregou a tecnologia nuclear aos Prosilicanos, quando responsabilizado por observar o desenvolvimento da vida em nosso sistema solar apaixonou-se pelo potencial do \textit{homo sapiens} e violou o juramento de não-intervenção quase quatrocentas vezes, sempre em benefício da preservação da espécie humana ou da vida terrestre. Há, no universo Marvel, quem diga ser a Terra, entre milhões de outros sob sua responsabilidade, o planeta onde Uatu encontrou formas de vida capazes de lidar com estas ``intervenções'', daí derivando sua afeição. Não deve fugir à atenção de quem lê estas linhas que Stan Lee, principal roteirista da Marvel Comics nos anos 1960, criou os fundamentos de seu universo fantástico no auge da Guerra Fria e do terror nuclear, e Uatu é tributário de seu tempo: de \textit{deus ex machina} em enredos de ficção científica, ele e sua espécie logo foram transformados como que num alerta acerca dos abusos da tecnologia atômica. Personagem e criador enredam-se na mesma teia: intervém no presente, direta ou alegoricamente, porque constróem-no, integram-no; ao mesmo tempo, por dele participarem, são igualmente construídos pelo tempo de que são parte. 

Este é, portanto, o pano de fundo sobre o qual se projeta a pesquisa exposta a seguir. É partindo desta tomada de posição teórica que se poderá compreender a escolha do marco teórico desenvolvido a seguir, a partir da questão epistemológica e da questão prática já apontadas.

\subsection{Questão epistemológica}
\label{subsec:questepist}

Do ponto de vista \textit{epistemológico}, ao longo da formação do urbanismo e do planejamento urbano enquanto disciplinas técnicas especializadas com pretensão à cientificidade, consolidou-se uma tendência à explicação dos processos de formação dos territórios urbanos baseada em critérios puramente estatísticos, casuísticos ou formais \cite{benevolo_historia_1983, mumford_cidade_1998, hall_cidades_2007}. Esta tendência condicionou a própria construção das técnicas do planejamento urbano. Num breve estudo sobre a história da gestão urbana contemporânea, Luiz de Pinedo Quinto Jr. observou que 

\begin{citacao}
(…) A cultura urbanística [formada pela experiência do urbanismo alemão do século XIX, baseado na possibilidade de obter-se um maior grau de racionalidade do uso do solo baseado no conceito de unidade e coerência] procura impor e moldar a cidade capitalista partindo do pressuposto de que é possível controlar e diminuir os conflitos gerados pelas relações de mercado. (…) Desde o surgimento do corpo disciplinar [do urbanismo], a intervenção e a reestruturação da cidade tende a eliminar a história como instrumento de análise, pois esta coloca em xeque técnicas de intervenção que, ao serem aplicadas, não surtem o efeito desejado, visto que o problema urbano específico não possui necessariamente a mesma causa que levou ao surgimento da técnica de intervenção. (...) Dentro da ideologia dominante no corpo disciplinar, só passaram a ter legitimidade e valor científico os instrumentos e técnicas que reforçassem o caráter operativo da disciplina. \cite{QUINTOJR1990}
\end{citacao}

Tal tendência é válida por tratar de aspectos evidentes e incontornáveis do fenômeno urbano, mas traz consigo um problema: o encadeamento sucessivo de formas espaciais sem qualquer referência aos processos históricos que as produziram, a análise da evolução demográfica descolada de uma interpretação compreensiva dos dados etc., arrisca tomar o fim pelo meio, o resultado pelo processo, o produto pelo produtor, e assim reduzir a força operacional dos instrumentos empregues.

Além disto, mesmo quando os processos de formação dos instrumentos tradicionais de ordenamento urbano como planos diretores, códigos de obras, ordenamentos de uso do solo etc. exigem a compreensão de processos históricos de formação da cidade, pois do contrário careceriam dos elementos empíricos garantidores de sua eficácia, sua interpretação destes processos históricos é pautada por uma concepção da história onde os diferentes agentes responsáveis pela produção do espaço urbano, tal como seu fazer e suas práticas, são abstraídos ou ocultados. 

O instrumental técnico do urbanismo e do planejamento urbano concebe este fazer e estas práticas como \textit{irregularidades} \cite[p.~181-210]{ROLNIK2007}, que levam – seguindo a terminologia técnica do urbanismo atualmente em vigor – à formação de \textit{aglomerados subnormais}. A consequência mais comum, ao longo do tempo, é a estigmatização dos territórios urbanos produzidos fora dos padrões urbanísticos – estigmatização com raízes mais profundas, que a terminologia técnica oculta.

Mas a técnica do urbanismo e do planejamento urbano, como quaisquer outras, não são apenas o conjunto de instrumentos, saberes e práticas necessárias para produzir determinado resultado ou produto (material, simbólico ou afetivo); exatamente por isto, são ``expressão material de dadas relações sociais'' \cite[p.~266]{BERNARDO1977c}, são ``realização material de dadas relações sociais e, simultaneamente, a condição para a sua reprodução'' \cite[p.~285]{BERNARDO1977c}. Ou, de modo mais extenso:

\begin{citacao}
Cada modo de produção produz uma tecnologia específica, expressão e realização das suas contradições próprias. É certo que elementos de uma tecnologia, tanto tipos particulares de organização como utensílios e máquinas, podem vir a ser isolados do contexto geral em que surgiram e a que haviam pertencido e passarem a integrar outras tecnologias, de que se tornam então elementos componentes. Porém, em primeiro lugar, isso acontece exclusivamente com técnicas particulares, e nunca com um sistema tecnológico globalmente considerado. (…) Em segundo lugar, nem todas as técnicas são suscetíveis de tal processo de desestruturação e reestruturação, e a análise histórica mostra que isso tem até ocorrido com um número reduzido de técnicas particulares. Em terceiro lugar, cada técnica não é uma forma estagnada e definitivamente fixada, mas caracteriza-se precisamente pela evolução e pelas mudanças que sofre, no interior das transformações globais do sistema tecnológico em que se integra. Isolada do sistema, converte-se num fóssil. E, integrada em outro sistema, passa a desenvolver-se de outro modo, para em breve se tornar uma técnica diferente. Uma técnica, como qualquer outro elemento social, é definível apenas pelo sistema – um ou outro – em que ocupa um lugar. \cite[p.~312]{BERNARDO1991}
\end{citacao}

Não é diferente com o urbanismo e com o planejamento urbano. Na atualidade, os dois apresentam-se enquanto um conjunto de técnicas de regulação e ordenação do desenvolvimento das cidades, enquanto conjunto de técnicas de ``arbitramento'' de diferentes interesses de uso de tal ou qual parcela do espaço urbano \cite{bernardi_organizacao_2007, duarte_planejamento_2007}; a análise histórica de sua aplicação desvela, entretanto, uma longa sequência de fracassos \cite{hall_cidades_2007, SANTOS1982} que não pode ser creditada apenas às esperadas divergências entre o planejado e o executado, ainda que corrigidas por revisões constantes. Esta concepção do planejamento urbano oculta o fato de que, no quadro tecnológico do capitalismo, suas técnicas têm funcionado como um conjunto de ``mecanismos de dispersão das contradições emergentes das relações sociais de produção capitalista accionados no domínio fundiário urbano e habitacional'' \cite[p~76]{SANTOS1982} e em outros domínios das políticas urbanas. E, ao contrário do pretendido, é exatamente a crescente intervenção do Estado, enquanto instituição planejadora e executora de políticas de planejamento urbano, quem cria, através das políticas de desenvolvimento urbano que implementa, as condições para ``novas polarizações sociais (…) e novas formas de politização dos conflitos e de resistência das classes populares, enquanto classes urbanas'' (idem, p. 76).

Mas é possível falar de ``planejamento urbano'' ou de ``urbanismo'' no período escolhido? Não seria anacrônica a crítica?

Sabe-se que na Primeira República os chamados ``melhoramentos urbanos'' estavam inseridas no campo da \textit{engenharia sanitária} e do \textit{higienismo}. No campo do serviço público, desde 1891 Salvador contava com uma Diretoria de Obras Públicas responsável por analisar os pedidos de construção e reforma da cidade, e desde 1907 com um Serviço Sanitário Municipal; no campo do ensino, a Escola Politécnica fora fundada em 1896 com um curso em que Arquitetura e Engenharia Sanitária eram ministradas em simultâneo \cite{fernandessampaiogomes1999}. 

Isto posto, não se pode esquecer que, apesar da pioneira proposta de Theodoro Sampaio de um vasto plano de obras infraestruturais (1905), do \textit{Plano geral de melhoramentos} de Alencar Lima (1910) e de um plano de saneamento elaborado por Saturnino de Brito (1926), o que houve de efetiva transformação do espaço urbano de Salvador se deu no contexto da \textit{reforma do porto de Salvador} (1906-1921) e da \textit{reestruturação do centro da cidade} (1912-1916), obras decorrentes de uma preocupação estético-sanitária que, não obstante incorporar o fato de o funcionamento da cidade ter se tornado tributário de sistemas técnicos de transporte, distribuição de água, esgotamento, telefone, energia elétrica etc., não incorporara, como em fase posterior do urbanismo soteropolitano, a pretensão de pensar ou intervir de forma global na cidade \cite{fernandessampaiogomes1999}. Daí que o trabalho do EPUCS seja canonicamente considerado como o primeiro grande momento do urbanismo e do planejamento urbano de Salvador.

Do ponto de vista estritamente técnico, é evidentemente anacrônico falar de um ``planejamento urbano'', e embora um relatório da Intendência de Salvador datado de 1930 já falasse em ``urbanismo'', se verá ao longo desta dissertação que seu significado de época era diferente do atual. Por outro lado, vistas as coisas pela perspectiva dos trabalhadores; pelo ponto de vista daqueles removidos à força dos cortiços, mocambos e habitações coletivas comuns na Salvador da Primeira República \cite{cardoso1990proleta}; pelo lado daqueles cujas ocupações tradicionais foram alvo de uma luta sem quartel pelos sanitariastas \cite{barbosa2009}; vistas as coisas por esta perspectiva, pouco importava se havia ou não uma perspectiva global de intervenção urbana. Importava que, global ou não, havia uma intervenção planejada contra seus modos de fazer e de viver, e que era preciso contorná-la, adaptando-se (a seu modo, claro) ou resistindo. A mesma perspectiva indica que, global ou não, este é o traço que une o ``higienismo'', o ``sanitarismo'', os ``melhoramentos urbanos'', o ``urbanismo'' e o ``planejamento urbano'' num só \textit{continuum}; sob tal ponto de vista, adotado nesta pesquisa, \textit{tanto faz usar qualquer dos termos}. 

\subsection{Questão prática}
\label{subsec:questprat}

O conflito entre grupos sociais com interesses distintos na apropriação e uso do espaço, como visto, é o ``convidado de pedra'' no campo epistemológico do urbanismo e do planejamento urbano. E isto leva às questões de ordem prática que influenciam este projeto de pesquisa.

Na experiência profissional do autor desta dissertação enquanto assessor jurídico de grupos, comunidades e movimentos populares em luta por moradia digna ou ameaçados por remoções forçadas em Salvador e Região Metropolitana, realizada a partir da atuação do \textit{Centro de Estudos e Ação Social} (CEAS), é comum encontrar ``no outro lado da mesa de negociação'' entre Estado e movimentos populares arquitetos, urbanistas, engenheiros e geógrafos pouco dispostos a ceder um palmo sequer às reivindicações daqueles sobre cujas casas passaram seus lápis e réguas.

Igualmente, esta experiência profissional permitiu contato com fazeres e saberes de produção do espaço urbano muito distintos daqueles estudados nas universidades e praticados no âmbito dos órgãos de governo. Padrões no apropriação e uso do solo, formas de ocupação territorial, o enfrentamento às formas hegemônicas de produção territorial, tudo isso exige ``no mínimo uma certa vivência prévia da cidade, um relativo conhecimento do seu espaço, assim como a existência de uma rede de relações sociais informais'' \cite[p.~40]{MATTEDI1981}; a julgar por estudos já realizados sobre estas formas contra-hegemônicas de produção territorial entre os anos 1940-1980, há quem diga que elas têm raízes profundas em períodos anteriores, quando a ocupação da terra se dava de maneira relativamente simples, dada a abundância do espaço e sua mercantilização quase inexistente \cite[p.~25]{MOURA1990}; era possível, então, encontrar muitos soteropolitanos que, mesmo quando proprietários de sua casa, não passavam de meros ``foreiros'', ``rendeiros'' ou ``moradores'' de terras de terceiros \cite[p.~139]{BRANDAO1980}.

A comprovação empírica destas observações levou o autor desta dissertação a mitigar a vertente formalista do urbanismo – sem abandoná-la de todo, afinal, estamos num programa de pós-graduação em Arquitetura e Urbanismo – e a compreender a urbanização e a formação/consolidação de territórios urbanos como um processo de relação histórica entre sociedade e espaço \cite{CASTELLS2000, SANTOS2008} que, apesar de encerrar de modo evidente e incontornável a análise da evolução das formas, da demografia e dos tipos ideais urbanos, exatamente pelo caráter histórico da relação compreende a produção, apropriação e uso do espaço urbano como um \textit{constante fazer e refazer}, como \textit{sucessão de conjunturas espaciais}, produzidos por agentes sociais vivos e não raro conflitantes.

Eis em resumo, portanto, a perspectiva que preside a presente dissertação:

\begin{citacao}
Neste sentido, é possível então afirmar que as questões e os conflitos de interesses surgem das relações sociais e se territorializam, ou seja, materializam-se em disputas entre esses grupos e classes sociais para organizar o território da maneira mais adequada aos objetivos de cada um, ou seja, do modo mais adequado aos seus interesses. Essas disputas no interior da sociedade criam tensões e formas de organização do espaço que definem um campo importante [\dots]. \cite[p.~41]{CASTRO2005}.
\end{citacao}

Aplicada tal perspectiva a história da cidade e do planejamento urbano, alarga-se um debate travado por um historiador italiano, que ridicularizou a ``artificiosa concentração da historicidade intrínseca da cidade no núcleo antigo'' \cite[p.~74]{argan_histcid_1992}; sua polêmica, entretanto, foi contra a oposição ``absurda'' entre uma ``cidade antiga'' e uma ``cidade moderna'' conceitualmente apartadas, pois 

\begin{citacao}
\dots se se quer conservar a cidade como instituição, não se pode admitir que ela conste de uma parte histórica com um valor qualitativo e de uma parte não-histórica, com caráter puramente quantitativo. Fique bem claro que o que tem e deve ter não apenas organização, mas substância histórica é a cidade em seu conjunto, antiga e moderna. Pôr em discussão sua historicidade global equivale a pôr em discussão o valor ou a legitimidade histórica da sociedade contemporânea, o que talvez alguns queiram, mas que o historiador não pode aceitar \cite[p.~79]{argan_histcid_1992}.
\end{citacao}

O desenvolvimento da pesquisa exposta nesta dissertação evidenciou outro lado da questão. Embora os efeitos desta distinção ``absurda'' aparentemente se produzam no presente e a ele se circunscrevam, eles reverberam sobre o passado pela influência exercida sobre as agendas de pesquisa historiográfica, resultando, no caso que nos interessa mais de perto, em inflação de produtos acadêmicos acerca do \textit{centro histórico} de Salvador e num silêncio ensurdecedor quanto às zonas \textit{suburbana} e \textit{rural} da cidade. 

Como as ondas a irradiar de uma pedra atirada à água, os reflexos deste erro se multiplicam: se a produção historiográfica sobre o território soteropolitano se concentra no estudo da área mais densamente povoada, em especial nas freguesias propriamente urbanas e nas semi-rurais que as rodeavam (Santo Antônio, Vitória, Brotas), resulta disso que a historiografia dos bairros populares de Salvador surgidos da década de 1940 em diante no território destes distritos suburbanos e rurais toma como marco zero a primeira grande ``invasão'' coletiva de terras, que se disputa ter sido a da Vila Ruy Barbosa, a de Gingibirra ou a do Corta-Braço; toda a produção, apropriação e uso anterior destes territórios, inclusive por sujeitos históricos muito semelhantes aos ``invasores'' em seu respectivo contexto, resta condenada ao oblívio. É como se, negados os seus direitos mais básicos por décadas, os sujeitos históricos construtores destes territórios populares -- sejam os do século XX, sejam seus antepassados --  tivessem negado também o seu direito à representação histórica, ao status de agentes construtores da História tão legítimos quanto qualquer outro.

\section{Referencial teórico}
\label{sec:refeteor}

Dado o marco teórico apresentado, é preciso especificar os referenciais teóricos empregues na produção desta pesquisa, que se dividem em três campos: a \textit{sociologia e geografia urbanas}, a \textit{obra do historiador e economista português João Bernardo Maia Viegas Soares} e, por último, a \textit{historiografia sobre a Primeira República}.

\subsection{Primeiro campo teórico: sociologia e geografia urbanas}
\label{subsec:sociogeogrurb}

O primeiro campo teórico é composto pela \textit{sociologia urbana} e pela \textit{geografia urbana}, mais especialmente pelas obras de Élisée Reclus, Piotr Kropotkin, Roberto Lobato Corrêa, Milton Santos, Manuel Castells e Jean Lojkine.

Deste campo teórico, além da inspiração geral para este projeto de pesquisa, foram extraídos alguns dos conceitos essenciais para a construção de um modelo analítico apto a sistematizar de modo coerente os conflitos sociais no espaço.

\subsubsection{Élisée Reclus e Piotr Kropotkin: a evolução do território urbano e a análise histórica dos conflitos na produção do território}

Embora seja possível traçar seus antecessores em diversos momentos da história, o movimento anarquista, tal como o compreendemos hoje, consolidou-se como expressão de um projeto político apenas nas quatro últimas décadas do século XIX, e a força, presença e relevância deste projeto político inicial influenciou incontáveis organizações e iniciativas políticas até a década de 1930, quando eventos que culminaram na chegada ao poder político entre 1917 e 1939 de organizações políticas historicamente inimigas do anarquismo -- que vão desde comunistas até fascistas, passando por organizações influenciadas pelo liberalismo, pelo conservadorismo, pelo nacionalismo e pelo fundamentalismo religioso -- resultaram numa onda de perseguição política aos anarquistas cujo saldo foi o desmantelamento de suas organizações e a destruição de seus arquivos pessoais, além de prisões, exílios e assassinatos. Somente na década de 1960 o anarquismo retornou com força à cena política, mesmo assim perseguido e marginalizado \cite{WOODCOCK2008}. 

No que se pode chamar de a \textit{primeira onda} do movimento anarquista, descrita acima, houve dois militantes, geógrafos de profissão, que apresentaram versões alternativas e bastante funcionais a muitas das teorias atualmente empregues na compreensão do fenômeno citadino e no planejamento urbano\footnote{Não por acaso Patrick Geddes, um dos fundadores do planejamento regional e urbano modernos, foi amigo e discípulo dos dois (\citeauthor{dunbar_elisee_1989}, \citeyear{dunbar_elisee_1989}, pp. 89-90; \citeauthor{hall_cidades_2007}, \citeyear{hall_cidades_2007}, pp. 161-170).}: \textit{Elisée Reclus }e \textit{Piotr Kropotkin}. Para ambos, ``anarquismo e geografia são uma combinação lógica'' \cite[p.~78]{dunbar_elisee_1989}, seja enquanto \textit{filosofia política}, seja enquanto \textit{fundamento epistemológico}.

Kropotkin e Reclus veem na \textit{livre associação dos indivíduos} e na \textit{solidariedade} duas forças motrizes do desenvolvimento social, econômico, político e geográfico; por isto mesmo, são críticos acerbos de tudo quanto possa obstaculizar estas duas forças: Estado, capital, exploração do homem pelo homem, colonialismo, imperialismo, tirania e autoritarismo são temas constantes de seus ataques.

Por caminhos teóricos ligeiramente diferentes, Kropotkin e Reclus chegam à conclusão de que as cidades são um lugar de encontros e um espaço fértil para atuação política, por terem sido os lugares onde surgiram os embriões da democracia moderna -- embora vejam no governo representativo e no Estado, mesmo o mais democrático, entraves à solidariedade e à livre associação dos indivíduos. Por terem sido quase vizinhos em seu exílio suíço (1877-1881), Kropotkin e Reclus influenciaram-se mutuamente, um ``polindo'' o entendimento do outro através do diálogo: Reclus dando foco mais social e ecológico à geografia física de Kropotkin, e este último conferindo à geografia social e ecológica de Reclus caráter mais comunal e mais atento no fenômeno urbano \cite[p.~209-210]{WARD2010}.

\textit{Élisée Réclus} é um dos menos conhecidos entre os fundadores da geografia. Tido como um dos principais geógrafos do seu tempo, chegou inclusive, graças a sua reputação, a contribuir para a solução de problemas diplomáticos internacionais\footnote{``Pesquisando no acervo do Conselho Municipal da cidade de Genebra, encontramos um documento contendo afirmações que chamaram logo a nossa atenção. O fundo cartográfico Reclus-Perron conservado em Genebra possui particularmente um mapa manuscrito que o explorador Henri Coudreau (1859-1899) fez para Elisée Reclus (1830-1905). Ele foi decisivo para a arbitragem do Conselho federal suíço que julgou, em primeiro de dezembro de 1900, o contestado franco-brasileiro sobre as fronteiras entre o Brasil e a Guiana, anexando ao Brasil um território de 260.000 quilômetros quadrados.'' \cite[p.~2]{FERRETTI2013}}. Sua militância política e o forte acento social e histórico de sua geografia, entretanto, levaram os outros fundadores da geografia moderna -- e seus discípulos -- a dizer, textualmente, que sua obra ``contém interessantes pontos de vista geográficos, mas que é sobretudo história e sociologia''.

A metodologia de Élisée Reclus é, num primeiro contato, estranha:

\begin{citacao}
No estudo das características diversas do planeta, em suas reações mútuas de justaposição e influência, nas mudanças provocadas pela série das eras, o elemento de comparação que sempre teremos diante dos olhos será a sociedade humana. A história da Terra e aquela da humanidade em suas ações e reações continuadas, desde as origens conhecidas até os tempos que se preparam, serão o objeto de nosso estudo. Para resumir nosso pensamento, buscaremos seguir a evolução da humanidade em relação às formas terrestres e a evolução das formas terrestres em relação à humanidade. \cite[pp.~78-79]{reclus_renovacao_2010}
\end{citacao}

Haveria em Reclus mais que relações, mas praticamente uma \textit{unidade entre sociedade e natureza}, quase rompendo a distinção \textit{physis/nomos }herdada há milênios da ontologia aristotélica? A dificuldade da empreitada ontológica era reconhecida pelo próprio Reclus:

\begin{citacao}
É verdade, eu sabia de antemão que nenhuma pesquisa far-me-ia descobrir essa lei de um progresso humano cuja miragem sedutora agita-se incessantemente em nosso horizonte, e que foge de nós e dissipa-se para recriar-se uma vez mais [{\dots}] Não, mas podemos ao menos, nessa avenida dos séculos que os achados dos arqueólogos prolongam constantemente naquilo que foi a noite do passado, reconhecer a íntima ligação que une a sucessão dos fatos humanos à ação das forças telúricas: é-nos permitido perseguir no tempo cada período da vida dos povos correspondendo à mudança dos meios, observar a ação combinada da Natureza e do próprio Homem, reagindo sobre a Terra que o formou. \cite[pp. 45-46]{reclus_renovacao_2010}
\end{citacao}

Mais modestamente, o que Reclus propôs foi uma análise simultaneamente sincrônica e diacrônica do espaço: ``Ao meio-espaço, caracterizado por milhares de fenômenos exteriores, deve-se ajuntar o meio-tempo, com suas transformações incessantes, suas repercussões sem fim'' \cite[p.~110]{RECLUS1905a}. Para Reclus, ao contrário de muitos deterministas seus contemporâneos, não havia qualquer primazia do meio natural sobre o homem:

\begin{citacao}
...todas estas forças [\textit{naturais}] variam de lugar em lugar e de época em época; é portanto em vão que os geógrafos têm ensaiado classificar numa ordem definitiva, a série de elementos do meio que influem sobre o desenvolvimento de um povo; os fenômenos múltiplos, entrecruzados da vida não se permitem enumerar numa ordem metódica.

[{\dots}] A história da humanidade, em seu conjunto e suas partes, não se pode assim explicar pela soma dos meios com ``juros compostos'' pela sucessão dos séculos; mas para entender a evolução que se realiza, também temos de apreciar em que medida os próprios meios evoluem pelo fato da transformação geral [\textit{da sociedade}] e modificam sua ação em consequência.  

[{\dots}] Há também os aspectos da natureza que, sem haver mudado em nada, não deixam de exercer uma ação completamente diferente por efeito da história geral [\textit{da sociedade}] que modifica o valor relativo de todas as coisas. \cite[pp.~112-114]{RECLUS1905a}
\end{citacao}

\textit{Meio-tempo,} \textit{meio-espaço }e \textit{ação combinada e recíproca da Natureza e do Homem} -- desfazendo, assim, qualquer fetiche naturalista ou de certo ecologismo vulgar hoje em moda -- são conceitos-chave da \textit{geografia social}, cujas três ``leis'' Reclus sintetizou:

\begin{citacao}
A primeira categoria de eventos que o historiador constata mostra-nos como, por efeito de um desenvolvimento desigual entre os indivíduos e sociedades, todas as coletividades humanas, à exceção das tribos permanecidas no naturismo primitivo, se desdobram, por assim dizer, em classes ou castas, não somente diferentes, mas opostas em interesses e tendências, mesmo francamente inimigas em todos os períodos de crise. [{\dots}]

O segundo fato coletivo, consequência necessária do desdobramento dos corpos sociais, é que o equilíbrio rompido de indivíduo a indivíduo, de classe a classe, se equilibra constantemente em torno de seu eixo de repouso: a violação da justiça clama sempre por vingança. Daí as incessantes oscilações. [{\dots}]

Um terceiro grupo de fatos, ligando-se ao estudo do homem em todas as épocas e todos os países, atesta-nos que nenhuma evolução na existência dos povos pôde ser criada senão pelo esforço individual. É na pessoa humana, elemento primário da sociedade, que se deve buscar o choque impulsivo do meio, destinado a se traduzir em ações voluntárias para disseminar as ideias e participar das obras que edificarão o comportamento das nações. O equilíbrio das sociedades só é instável pela perturbação imposta aos indivíduos em sua franca expansão. A sociedade livre estabelece-se pela liberdade fornecida em seu completo desenvolvimento a cada pessoa humana, primeira célula fundamental, que, em seguida, agrega-se e associa-se como lhe apraz às outras células da mutável humanidade. [{\dots}]

A ``luta de classes'', a busca do equilíbrio e a decisão soberana do indivíduo, tais são as três ordens de fatos que nos revela o estudo da \textit{geografia social }e que, no caos das coisas, mostram-se assaz constantes para que se possa dar-lhes o nome de ``leis''. (\citeauthor{RECLUS1905a}, \citeyear{RECLUS1905a}, pp. II-IV; \citeyear{reclus_renovacao_2010}, pp. 47-50)
\end{citacao}

Muito à frente de seu tempo, e graças à influência de Piotr Kropotkin, Élisée Reclus foi um dos primeiros geógrafos a dedicar atenção especial às cidades -- e entre os pioneiros da disciplina, foi o que mais analisou a questão.

Entre as obras de Élisée Reclus, a \textbf{Nouvelle Géographie Universelle }(19 volumes) fez muito sucesso na época de sua publicação (1876-1894). É uma descrição extremamente minuciosa e rigorosa do ``estado do mundo'' na época, riquíssima em dados demográficos, corográficos, econômicos, topográficos, cartográficos, geológicos etc. Nesta obra, Reclus divide internamente os países em \textit{zonas de influência }(com bastante riqueza e detalhe), \textit{zonas de vegetação }(com menor impacto na obra), a \textit{pertença étnica }de sua população (onde isto se aplicava) e... a ``rede urbana'', entre muitas aspas. Nesta obra Reclus não conceituou a rede, não estruturou abstratamente a hierarquia  entre cidades; apenas identificou empiricamente fatos concretos que se repetiam e os descreveu o mais detalhadamente possível, sem extrair deles qualquer teoria geral.

Já em \textbf{L'homme et la terre} (1905-1908, 6 vols.) as intuições reclusianas sobre o espaço urbano chegam ao nível de uma primeira teoria sobre o assunto, especialmente no capítulo II do livro IV, intitulado ``\textit{Répartition des hommes}'' (Repartição dos homens). Este pequeno ensaio -- 41 páginas no original francês \cite[pp.~335-376]{RECLUS1905e}, com tradução recente para o português \cite{reclus_renovacao_2010} -- é repleto de ideias sobre questões como causas do êxodo rural na Europa; vantagens da convivência urbana; fatores de atração locacional na fundação de cidades; relação entre agricultura, clima e relevo na formação de cidades; método para estudo da ``personalidade'' e do ``caráter'' de uma cidade tal como se apresenta através da acumulação de obras, construções, aglomerações, bairros etc.; morfologia urbana; especulação imobiliária; usos militares da malha urbana; indústria e poluição; estética e monumentos urbanos; saneamento urbano; relação entre meios de transporte e suburbanização; cidades-jardim (Port Sunlight, Bourneville, Letchworth); superpopulação urbana, concentração urbana e sua perspectiva futura.

Lançada ao esquecimento por décadas, a obra de Élisée Reclus foi resgatada entre os anos 1960 e 1970 por Yves Lacoste e David Harvey. Atualmente, Federico Ferretti e Phillipe Pelletier esforçam-se mais detidamente para fazer pontes entre a obra de Reclus e temas como colonialismo e estudos pós-coloniais, educação popular e geopolítica. No Brasil, país objeto de monografia específica de Reclus \cite{RECLUS1900}, até onde foi possível pesquisar nenhuma obra de Reclus havia sido publicada desde o início do século XX até 1985, quando sai publicada na coleção ``Grandes Cientistas Sociais'', da editora Ática, uma coletânea de trechos da obra reclusiana. Graças ao esforço pioneiro de Plínio Augusto Coelho, que traduz e publica escritos anarquistas desde 1989 através da editora Imaginário, capítulos da obra \textbf{L'homme et la terre }estão sendo traduzidos para o português e publicados na forma pequenos livros. E o interesse por Reclus já chegou à academia brasileira: entre 6 e 10 de dezembro de 2011 realizou- e, na USP, o colóquio internacional ``Élisée Reclus e a geografia do novo mundo'', cujas memórias estão disponíveis na internet (\url{https://reclusmundusnovus.wordpress.com/memorias/}). Tudo isto é plenamente justificado:

\begin{citacao}
Apesar da utilização de metáforas biológicas, o texto de Reclus interessa pela visão dos processos, pela explicação teórica dada para a fundação das cidades, entre outras, como, por exemplo, a indicação de especuladores como agentes do desenvolvimento urbano. \cite[p.~65]{vasconcelos_dois_2012}
\end{citacao}

\textit{Piotr Kropotkin} é outro dos fundadores desconhecidos da geografia. Príncipe da dinastia ruríquida, senhora dos territórios da Rússia e Ucrânia entre 862 e 1610, \textit{Piotr Kropotkin }(1843-1921) recusou o título aos 12 anos para tornar-se uma das figuras centrais do movimento anarquista no século XIX, e repreendia duramente os amigos que ainda o tratavam como se fosse nobre \cite[p.~13]{baldwin_story_1970}. Quando jovem, na Rússia, esteve envolvido com atividades administrativas, militares e de corte; entre 1866 e 1872 dedicou-se a expedições científicas à Sibéria, tendo em seguida, numa viagem à  Suíça, tido contato com o movimento operário, filiando-se à seção genebrina da Associação Internacional dos Trabalhadores (AIT). Por influência de militantes da Federação do Jura suíço, ingressou no movimento anarquista, que não abandonou até sua morte. Voltou à Rússia ainda em 1872, e já em 1874 estava preso na fortaleza de Pedro e Paulo por sua atividade política com o \textit{Círculo Tchaikovsky}. Tendo escapado das prisões russas em 1876, passou a viver no exílio, entre a França, Suíça e Inglaterra, sempre ativo no movimento anarquista então em pleno vigor. Voltou à Rússia apenas em 1917 para colaborar com o processo revolucionário, mas a escalada dos bolcheviques ao poder através de um golpe de Estado fê-lo crítico acerbo desta forma de condução de um processo revolucionário -- em conformidade com sua longa crítica a qualquer governo, mesmo revolucionário. Morto em 8 de fevereiro de 1921, seu funeral, em 13 de fevereiro, foi o último ato público de anarquistas durante a Revolução Russa; a \textit{Cheka}, polícia política bolchevique, se encarregaria de aniquilá-los a partir de então.

Em termos atuais, pode-se dizer que Kropotkin viu na história das cidades europeias medievais o \textit{espaço urbano como produto da associação de indivíduos em busca da libertação do jugo feudal e da dominação eclesial: } ``O principal objetivo da cidade medieval era o de garantir a liberdade, a autoadministração e a paz, e sua principal base, o trabalho'' \cite[p.~142]{KROPOTKIN2009}. Não apenas o espaço físico e o desenho urbano\footnote{``Geralmente a cidade era dividida em quatro partes, ou em cinco a sete setores que se irradiavam de um centro, e cada um deles correspondia  mais ou menos a um certo comércio ou ofício que nele prevalecia, mas continha habitantes de diferentes posições sociaise ocupações -- nobres, comerciantes, artesãos ou mesmo semisservos. Cada setor ou parte constituía um aglomerado bem independente. [{\dots}] Portanto, a cidade medieval é uma dupla federação: de todos os domicílios unidos em pequenas associações territoriais -- a rua, a paróquia, o setor -- e de indivíduos ligados por juramento em corporações de ofício. A primeira foi resultante da origem na comunidade aldeã e a segunda, uma ramificação subsequente gerada por novas condições'' \cite[p.~142]{KROPOTKIN2009}.}, como as instituições sociais criadas nestas cidades foram para ele resultados desta luta. Ao contrário do que se possa imaginar à primeira leitura, o retorno de Kropotkin às cidades medievais não era uma \textit{utopia regressiva, }uma idealização do passado proposta como horizonte político, mas sim uma \textit{crítica historicista às utopias socialistas do século XIX}: ``não só as aspirações de nossos radicais modernos já eram realidade na Idade Média, assim como muito do que se chama hoje de utopia era comum naquela época'' (\citeauthor{KROPOTKIN2009}, \citeyear{KROPOTKIN2009}, p. 157; \citeauthor{HORNER1989}, \citeyear{HORNER1989}, p. 142).

A tese geográfico-política kropotkiniana, entretanto, não se encerra aí. As cidades europeias teriam sido palco desde o século XIV de uma luta encarniçada do ``povo'' contra os burgueses, senhores feudais, aristocratas e reis absolutistas pela defesa de suas liberdades e pelo uso comum da terra. Toda a história das revoluções europeias, para Kropotkin, explica-se por esta chave. A análise da  \textit{Revolução Francesa }feita por Kropotkin é exemplar neste sentido. Para Kropotkin, a Revolução Francesa foi impulsionada não apenas pelos panfletos iluministas, mas pela decidida \textit{ação popular }de libertação de obrigações feudais e de retomada de terras das mãos de senhores laicos e religiosos. Desabrochavam no seio das massas ideias ``a respeito da descentralização política, do papel preponderante que o povo queria dar às suas municipalidades, às suas seções nas grandes cidades, e às assembleias de aldeia'' \cite[p.~23]{KROPOTKIN1955}. Ou ainda: ``a revolta dos camponeses para a abolição dos direitos feudais e a reconquista das terras comunais tiradas às comunas aldeãs desde o século XVII pelos senhores laicos e eclesiásticos -- \textit{eis a própria essência, a base da grande Revolução}'' (\Idem[p.~114]{KROPOTKIN1955}). 

A luta de classes eclodiu também nas cidades. Na França do século XVIII, a autoridade real demorara duzentos anos para construir uma estrutura institucional capaz de submeter a seu jugo as cidades anteriormente livres; tais instituições -- conselhos municipais vitalícios, direitos feudais, síndicos, almotacéis, direitos eclesiais de intervenção nas instituições municipais, isenções tributárias a membros da Igreja Católica e aristocratas etc. -- encontravam-se em franca decrepitude às vésperas da Revolução Francesa (\Idem[pp.~118-120]{KROPOTKIN1955}). Assim que a notícia da queda da Bastilha circulou pela província, o povo sublevou-se, apoderando-se dos Paços dos Conselhos e elegendo  ``uma nova municipalidade''; esta foi a base da \textit{revolução comunalista }posteriormente sancionada pela Assembleia Constituinte em 1789 e 1790 (\Idem[pp.~121]{KROPOTKIN1955}). O ``povo'', territorializado a partir dos distritos (\textit{arrondissements}) de Paris e de outras cidades grandes, ``fez a revolução nas localidades, estabelece revolucionariamente uma nova administração municipal, distingue entre os impostos que aceita e os que recusa pagar, e dita o modo de repartição igualitária daqueles que pagaria ao Estado ou à Comuna'' (\Idem[p.~130]{KROPOTKIN1955}).

Como se vê, o pensamento político de Kropotkin não lida com conceitos e categorias abstratas ou idealizadas; enraíza-os num meio geográfico, \textit{territorializa-os}. A relação homem-meio, sociedade-natureza, é vista por Kropotkin como um todo unitário, pleno de relações biunívocas e complexas. O espaço, em Kropotkin, é produto também do desenvolvimento histórico, e a História se desenvolve no espaço. O espaço urbano do passado e do porvir -- como qualquer outro espaço -- é, também, fruto de lutas sociais, em especial quando as cidades são o palco principal das lutas pela liberdade. Por isso, a questão urbana, em Kropotkin, pode ser vista como \textit{o conjunto dos fatores que obstaculizam o pleno desenvolvimento dos indivíduos e sua livre organização nas cidades}, fatores estes que variam em cada momento histórico. Esta é a primeira das noções encontradas na obra de Piotr Kropotkin a permear a pesquisa cujos resultados serão apresentados nesta dissertação.

O conhecimento destes fatores, todavia, só pode se dar através da pesquisa histórica da produção e do uso do espaço de cada cidade e das lutas em torno desta produção e deste uso, na tentativa de produzir sínteses orientadoras da ação política. Num artigo escrito entre 1880 e 1882, Kropotkin explicitou o método que resultou na obra \textbf{A grande revolução}, sua grande análise da Revolução Francesa: 

\begin{citacao}
Quanto às insurreições, que precederam a revolução e sucederam-se durante o primeiro ano, o pouco que posso dizer disso, neste espaço restrito, é o resultado de um trabalho de conjunto, que realizei em 1877 e 1878, no Museu Britânico e na Biblioteca Nacional [\textit{da França}], trabalho que ainda não terminei, e no qual me propunha expor as origens da revolução e de outros movimentos na Europa. Aqueles que quiserem lançar-se neste estudo deverão consultar (além das obras conhecidas [{\dots}]) as memórias e as histórias locais [{\dots}]. Entretanto, não devem contar com o fato de poder reconstituir, só com estes documentos, uma história completa das insurreições, que precederam a revolução. Para fazê-lo, só há um meio: dirigir-se aos arquivos, onde, apsar da destruição dos documentos feudais, ordenada pela Convenção, acabar-se-á, com certeza, por encontrar fatos muito importantes (\citeauthor{KROPOTKIN2005f}, \citeyear{KROPOTKIN2005f},  nota 27).
\end{citacao}

Deste modo, ao invés de contar apenas com os debates abstratos sobre a produção do espaço urbano fornecidos pela literatura especializada e realizar um debate que, conquanto importante, distancia-se da realidade estudada, seguiu-se a orientação de incluir a pesquisa arquivística como método para a investigação da produção, apropriação e uso do espaço no concreto.

\subsubsection{Milton Santos e Roberto Lobato Corrêa: a produção do espaço urbano}\label{subsubsec:milsanroblobcor}

De \textit{Roberto Lobato Corrêa} foram aproveitados os conceitos de \textit{agentes de produção do espaço urbano} e de \textit{processo de reorganização espacial} (CORRÊA, \citeyear{CORREA1985espa}, p. 7; \citeyear{CORREA1997}, p. 122). Estes agentes compõem o ``conjunto de forças sociais que atuam ao longo do tempo e permitem localizações, relocalizações e permanência das atividades e população sobre o espaço urbano'' \cite[p.~122]{CORREA1997}, e constroem um

\begin{citacao}
processo de reorganização espacial que se faz via incorporação de novas áreas ao espaço urbano, densificação do uso do solo, deterioração de certas áreas, renovação urbana, relocação diferenciada de infraestrutura e mudança, coercitiva ou não, do conteúdo social e econômico de determinadas áreas da cidade \cite[p.~7]{CORREA1985espa}.
\end{citacao}

De \textit{Milton Santos} foi aproveitado o conceito de \textit{espaço como acumulação desigual de tempos}, onde se dá a \textit{superposição de traços de sistemas diferentes} \cite[p.~256-257]{SANTOS2008}:

\begin{citacao}
Desde que instalados sobre um pedaço de espaço, as variáveis (de tipos diferentes, de idades diferentes) formam um precipitado, um fato novo, dotado de capacidade de criar ou estabelecer novas relações: uma nova qualidade. Estas combinações diferentes condicionam, até certo ponto, a entrada de novas variáveis. As localizações são historicamente determinadas pelas combinações de variáveis novas e antigas. (…) No entanto, pelo fato de que a ação de um sistema histórico anterior deixa resíduos, há uma superposição de traços de sistemas diferentes, exceto no caso de espaços virgens, tocados pela primeira vez por um impacto modernizador cuja origem se encontra em forças externas. (…) O lugar é, pois, o resultado de ações multilaterais que se realizam em tempos desiguais sobre cada um e em todos os pontos da superfície terrestre. Daí porque o fundamento de uma teoria que deseje explicar as localizações específicas deve levar em conta as ações do presente e do passado, locais e extralocais. \cite[p.~256-258]{SANTOS2008}
\end{citacao}

Os \textit{traços} conceituados por Milton Santos assemelham-se, em escala mais ampla, aos \textit{rastros} conceituados por Walter Benjamin em relação a escalas bem mais reduzidas, quase infinitesimais relativamente aos traços:

\begin{citacao}
O \textit{intérieur} não apenas é o universo, mas também o invólucro do homem privado. Habitar significa deixar rastros. No intérieur esses rastros são acentuados. Inventam-se as colchas e protetores, caixas e estojos em profusão, nos quais se imprimem os rastros dos objetos mais cotidianos. Também os rastros do morador ficam impressos no \textit{intérieur}. Surge a história de detetive que investiga esses rastros. \cite[p.~46]{benjamin_passagens_2006}
\end{citacao}

Sabendo desde já da dificuldade de encontrar fontes primárias que tratem diretamente dos conflitos sociais sobre o distrito escolhido, e mesmo de encontrar literatura monográfica a esquadrinhá-los, a pesquisa concebida neste projeto tentará identificar os traços e os rastros dos conflitos sociais pretéritos inscritos na produção do espaço de Salvador a partir de fontes variadas, para então tentar recompor um quadro geral de interpretação destes conflitos. Tais traços e rastros podem ser materiais (como construções, ruínas, superposições de usos e apropriações do espaço etc.) ou imateriais (tradições, hábitos, modos de vida etc.). Podem remeter à presença continuada de determinados grupos sociais em determinados espaços ao longo do tempo, ou às consequências de sua saída (pacífica ou violenta). Podem expressar elementos de permanência, descontinuidade ou retomada de determinadas práticas sociais em determinado espaço. Podem comunicar aos que constroem o território no presente a memória dos conflitos passados. Podem condicionar, através de vantagens ou desvantagens locacionais, o uso de determinado espaço. 

\subsubsection{Manuel Castells e Jean Lojkine: condições gerais de produção e espaço urbano}\label{subsubsec:mancastjeanlojk}

O conceito de \textit{condições gerais de produção} nasce de uma intuição de Marx, que não chegou a ser transformada em conceito nem a ser debatida por ele com precisão. Sequer seus principais comentadores têm dedicado muitas páginas ao assunto (cf. \citeonline{BIDET2007}, \citeonline{BERLIM1978}, \citeonline{CLEAVER2000}, \citeonline{MESZAROS2011}, \citeonline{ROSDOLSKY2001}). Não obstante, os elementos deste conceito, nomeado por Marx de diversas maneiras a depender do contexto em que emergem (\textit{condições coletivas de produção}, \textit{condição geral da atividade produtiva} etc.), estão espalhados em sua obra econômica e discutidos, mesmo de forma dispersa, com certo grau de profundidade. Para o momento, pode-se dizer, em linhas gerais, que para Marx o conceito de condições gerais de produção abarca os seguintes elementos:

\begin{enumerate}
\item Os \textit{meios de transporte}, como estradas, ferrovias e navios, capazes de levar as mercadorias de um lugar a outro para efetivação de sua troca \cite[p.~522]{MARX2013}.
\item Os \textit{meios de comunicação}, como o telégrafo, que permitem superar distâncias sem os custos próprios da comunicação realizada por meio dos transportes \cite[p.~457-458]{MARX2013}.
\item Tudo o que \textit{torne possível a circulação}, ou que \textit{facilite a circulação} \cite[p.~438]{MARX2011}.
\item Todos os \textit{trabalhos de utilidade geral}, como as irrigações, ou seja, as obras e equipamentos que beneficiem não apenas um capitalista, mas grande número deles ou todos, indistintamente \cite[p.~438]{MARX2011}.
\item Edifícios, depósitos, recipientes, aparelhos, instrumentos e outros \textit{meios de produção consumidos em comum por muitos indivíduos, simultânea ou alternadamente} \cite[p.~399-400]{MARX2013}.
\item Todos os ganhos de produtividade resultantes do \textit{desenvolvimento da produtividade do trabalho na produção de matérias-primas e meios de produção}.
\item Tudo aquilo que \textit{aumente a força produtiva} \cite[p.~114]{MARX2008}.
\item Tudo aquilo que garanta a \textit{segurança da troca} \cite[p.~432-433]{MARX2011}.
\end{enumerate}

A ligação entre as condições gerais de produção e o meio urbano é muito evidente, especialmente se levarmos em conta a malha viária urbana e os meios de transporte urbanos; os ganhos de escala decorrentes da concentração espacial de meios de produção, meios de circulação e força de trabalho; os benefícios gerados pelos trabalhos de utilidade geral sobrepostos no mesmo espaço (rede elétrica, abastecimento de água e esgoto etc.). Entretanto, foi preciso aguardar quase cem anos para que o desenvolvimento conceitual das condições gerais de produção fosse retomado e sua ligação com o meio urbano fosse explicitada; foi no seio da \textit{escola francesa de sociologia urbana} que as condições gerais de produção foram recolocadas num lugar teórico de relevo, mais especificamente nas obras de \textit{Manuel Castells} (em sua primeira fase) e \textit{Jean Lojkine}. O primeiro resgata e desenvolve o que em Marx eram apenas intuições, e o segundo desenvolve mais precisamente o lugar das condições gerais de produção na produção do espaço urbano.

Tendo afirmado ao longo de \textbf{A questão urbana} que o urbano é a \textit{reprodução coletiva da força de trabalho} e que a cidade é a \textit{unidade deste processo de reprodução} \cite[p.~550]{CASTELLS2000}, \textit{Manuel Castells} precisou retificar alguns equívocos num posfácio, e estabelecer, a partir de um diálogo crítico com o próprio livro de sua autoria, elementos contidos em suas pesquisas posteriores. 

Em primeiro lugar, Castells reconheceu a diversidade de práticas e funções que se dão numa cidade; por outro lado, ligou todas estas funções e práticas à forma histórica específica de uma sociedade dada, em relação com o capital, a produção, a distribuição, a política, a ideologia, ao consumo, à acumulação do capital e às relações políticas entre classes \cite[p.~550-551]{CASTELLS2000}. 

Em segundo lugar, Castells estabeleceu o espaço e o tempo como elementos de uma conjuntura unidos por uma relação biunívoca, e mostrou como a organização do espaço em unidades específicas e articuladas segundo os arranjos e os ritmos dos meios de produção parece ser própria das \textit{regiões}, não das \textit{cidades} \cite[p.~554-555]{CASTELLS2000}.

Em terceiro lugar, Castells demonstrou como a aglomeração, a unidade urbana, são definidas não nos termos próprios das regiões, mas nos termos do processo de \textit{reprodução da força de trabalho}, articulado com a reprodução das relações sociais e com a dialética da luta de classes \cite[p.~556-557]{CASTELLS2000}. Dentro da reprodução da força de trabalho, distinguiu dois grandes processos: o \textit{consumo coletivo} e o \textit{consumo individual}, para indagar qual deles estrutura o espaço \cite[p.~557]{CASTELLS2000}. 

Na tentativa de responder à indagação, Castells principiou pela divisão, encontrada em Marx, do consumo em \textit{reprodução ampliada dos meios de produção} mediante o \textit{consumo produtivo}; em \textit{reprodução ampliada da força de trabalho} por meio do \textit{consumo individual}; e num consumo dos indivíduos que ultrapassa o nível da reprodução simples e ampliada segundo  necessidades historicamente definidas (representando, aqui, o \textit{consumo de bens de luxo} \cite[p.~568]{CASTELLS2000}. Em seguida, afirmou que o consumo é determinado pelas regras gerais do modo de produção, mas esta determinação se produz em diferentes níveis e com efeitos específicos ao se levar em cotnta a diversdade de significações sociais do consumo: \textit{apropriação do produto} pelas classes sociais, \textit{reprodução da força de trabalho} relativamente ao processo de produção, e \textit{reprodução das relações sociais} no que concerne ao modo de produção no seu conjunto \cite[p.~569]{CASTELLS2000}. 

Castells, ao reconhecer na aglomeração urbana apenas um espaço de reprodução da força de trabalho e deixar à escala regional as preocupações quanto à inserção no processo produtivo, mostrava radicar sua preocupação com o \textit{espaço intra-urbano}. Usou o conceito de \textit{consumo coletivo} para analisar o espaço intra-urbano, já sendo possível localizar o nele \textit{planejamento urbano} como resultado da necessidade de regular o espaço urbano em resposta às revoltas urbanas; consequentemente, o planejamento é forma de intervenção do Estado sobre aspectos muito minuciosos da vida urbana, resultando numa politização de diversos aspectos da vida antes não englobados pelas contradições sociais. A intervenção do Estado se dá através de investimentos nos equipamentos de consumo coletivo que, ao não serem feitos no sentido de obtenção de lucro, representam uma desvalorização de parte do capital social e, consequentemente, aumentam a taxa de lucro do setor privado. Nota-se a preocupação de Castells com os conflitos sociais e com certas formas de participação política tendentes a aprofundar a integração das classes subalternas.

Em obra posterior, dedicada ao estudo do movimento citadino madrilenho, Castells encontrou o substrato empírico para este modelo teórico, reforçando-o e descrevendo-o em termos mais simples \cite[p.~20-25]{CASTELLS1980}. Em \textbf{The city and the grassroots}, entretanto, este modelo foi abandonado em favor de um modelo pluralista e intercultural de intepretação da realidade\footnote{``O que nossa perspectiva afirma, e o que a pesquisa empírica mostra, é que a tecnologia \textit{per se}, ou mesmo a estrutura da própria economia, \textit{não} são a força motriz  por trás do processo de urbanização. Fatores econômicos e o progresso tecnológico desempenham, de fato, papel importante no estabelecimento da forma e sentido do espaço. Mas tal papel é determinado, tanto quanto a própria economia e a tecnologia, pelo processo social mediante o qual a humanidade se apropria do espaço e do tempo e constroi uma organização social, implacavelmente desafiado pela produção de novos valores e pela emergência de novos interesses sociais'' \cite[p.~291]{CASTELLS1983}} \cite{CASTELLS1983}.

Já \textit{Jean Lojkine} construiu a síntese teórica expressa em \textbf{O Estado capitalista e a questão urbana} a partir de uma crítica dura tanto ao funcionalismo sociológico quanto ao marxismo estruturalista a que Manuel Castells, mesmo a contragosto, era associado \cite[p.~79-105]{LOJKINE1997}. Sendo Lojkine também marxista, embora com maior influência de Gramsci que de Althusser (o que, na esquerda francesa dos anos 1970, significava a pertença a campos diferentes de atuação)\footnote{``Na primeira metade da década de 1970 a ala renovadora, moderada e conciliatória do Partido Comunista Francês ressuscitou os escritos de Gramsci, considerando-o um precursor de Togliatti e do eurocomunismo. Na outra extremidade deste Partido Louis Althusser [\dots] dirigia a artilharia da Escola Normal Superior contra o humanismo atribuído a Gramsci e contra a teoria da praxis, o que na realidade significava uma reacção do comunismo granítico contra um comunismo disposto a adoptar a democracia parlamentar. A polémica era esta e os seus dois termos pareciam ser os únicos existentes'' \cite{MANOLOBERNARDO2012}.}, a consequência prática desta crítica foi a necessidade retornar a Marx não apenas para construir uma teoria do Estado alternativa ao funcionalismo sociológico e ao marxismo estruturalista \cite[p.~106-141]{LOJKINE1997}, mas de fazê-lo igualmente para situar-se no debate sobre a questão urbana.

É então que o conceito de \textit{condições gerais de produção} passa a ter operatividade. Lojkine resgatou-o para ligar o \textit{processo imediato de produção} e a \textit{unidade de produção}, de um lado, e o \textit{processo global de produção e circulação do capital}, de outro. Lojkine observou, todavia, que

\begin{citacao}
\dots essa limitação do alcance do conceito [\textit{aos meios de comunicação e de transporte}] parece-nos hoje discutível por causa do aparecimento de fatores também importantes que são outras tantas \textit{condições necessárias} à reprodução global das formações capitalistas desenvolvidas. Trata-se, de um lado dos \textit{meios de consumo coletivos} que se vêm juntar aos \textit{meios de circulação material}; de outro, da \textit{concentração espacial} dos meios de produção e reprodução das formações sociais capitalistas \cite[p.~145]{LOJKINE1997}
\end{citacao}

Lojkine observou, então, que o que caracteriza a cidade capitalista, além da simples aglomeração de meios de produção e troca, é, de um lado, a crescente concentração dos meios de consumo coletivo, e de outro o modo de aglomeração específica do conjunto dos meios de reprodução do capital e da força do trabalho, que se tornam sempre condições sempre mais determinantes do desenvolvimento econômico \cite[p.~146-147]{LOJKINE1997}

O espaço intra-urbano e o espaço regional são integrados por Jean Lojkine numa só teoria, graças à sua concepção renovada de condições gerais de produção que permite incluir num só quadro teórico o consumo coletivo, a circulação material e a concentração espacial dos meios de produção e reprodução das formações sociais capitalistas. O planejamento urbano, aqui, é uma intervenção contraditória do Estado sobre a socialização das forças produtivas. Mais adiante em sua obra, Lojkine trata as lutas urbanas já quase sem se referir às condições gerais de produção, que parecem ter-lhe servido apenas como instrumento polêmico contra o funcionalismo e o estruturalismo; os movimentos políticos que analisou em \textbf{O Estado capitalista e a questão urbana} existiam fundamentalmente como reação às consequências do capital monopolista sobre a questão urbana.

\subsection{Segundo campo teórico: condições gerais de produção e conflitos sociais na obra de João Bernardo}
\label{subsec:cgpcsjobe}

Dada a centralidade dos conceitos de \textit{conflitos sociais} e de \textit{condições gerais de produção} neste projeto de pesquisa, faz-se necessário conceituá-los em sua versão mais completa e elaborada, constante na obra do historiador e economista português \textit{João Bernardo Maia Viegas Soares}. 

Expulso de todas as universidades portuguesas em 1965 por envolver-se numa discussão com o reitor da Universidade de Lisboa e ser acusado de agressão, João Bernardo foi preso três vezes entre 1965 e 1966, entrou na militância anti-salazarista clandestina em 1967 e no final de junho de 1968 exilou-se em Paris, onde viveu até pouco depois da Revolução dos Cravos (1974). Em todo este período, João Bernardo militou em organizações clandestinas e seguiu com as pesquisas críticas em torno do marxismo que iniciara ainda antes de sua expulsão das universidades, o que o levou a uma ruptura com o marxismo ortodoxo e a uma aproximação do \textit{comunismo de conselhos} e de autores como Anton Pannekoek, Karl Korsch e Herman Gorter. Com antigos companheiros de organização, João Bernardo fundou o jornal \textit{Combate}, publicado de 1974 até 1978, de tendência libertária e que esteve muito ligado às ocupações de empresas e às comissões de trabalhadores. Com o fracasso da experiência política radical do conselhismo na revolução portuguesa (1974–1978) e depois de vários anos de estudos em Portugal, em outros países europeus e nos Estados Unidos, em 1984 João Bernardo decidiu-se a vir para o Brasil, estimulado pelo professor Maurício Tragtenberg. Ministrou cursos como professor convidado em várias universidades públicas brasileiras até 2009 e deu cursos livres em sindicatos, especialmente na CUT até 1999 \cite{BERNARDO2014} 

Em \textbf{Marx crítico de Marx} (\citeyear{BERNARDO1977a}, \citeyear{BERNARDO1977b}, \citeyear{BERNARDO1977c}) e na \textbf{Economia dos conflitos sociais} (\citeyear{BERNARDO1991}), João Bernardo apresenta um quadro teórico que tem a vantagem de centrar-se mais nos \textit{conflitos concretos entre classes} que na \textit{análise abstrata das relações entre capital e Estado}. Adicionalmente, o modelo teórico de João Bernardo, conquanto retenha o núcleo central da teoria marxista -- a \textit{teoria da exploração econômica} --, promove uma crítica global ao próprio marxismo, apontando a cada passo seus pontos cegos, becos sem saída e contradições.

\subsubsection{Desequilíbrio, conflitos sociais, modelo heurístico aberto, crises}\label{subsubsec:desecsmodheuracr}

Na \textbf{Economia dos conflitos sociais} João Bernardo explicita o caráter desequilibrado da economia:

\begin{citacao}
A luta de classes é o resultado inelutável, permanente, do fato de a força de trabalho ser capaz de despender tempo de trabalho, sem que seja, porém, possível vinculá-la a um \textit{quantum} predeterminado. Por isso os resultados do processo de exploração são irregulares, em grande parte imprevisíveis, fluidos. Desta contradição fulcral resulta que o modelo da mais-valia é um modelo aberto e, como todos os mecanismos econômicos da sociedade contemporânea são, ou formas de mais-valia, ou seus aspectos subsidiários, conclui-se que uma teoria crítica da economia capitalista só pode basear-se num modelo aberto, estruturalmente desequilibrado \cite[p.~62]{BERNARDO1991}. 
\end{citacao}

O desequilíbrio permanente no plano da produção é fonte de \textbf{conflitos sociais}, que vêm a ser ``uma categoria genérica, que engloba todas as formas de manifestação social das contradições'' \cite[p.~10]{BERNARDO1997}. Os conflitos podem ou não transformar-se em \textbf{lutas}, que são ``apenas uma das categorias dos conflitos, constituindo movimentos colectivos, capazes de empregar eventualmente a violência e dotados de um programa de reivindicações sistemático'' \cite[p.~10]{BERNARDO1997}. 

Na \textbf{Dialéctica da prática e da ideologia}, João Bernardo refina a conceituação dos conflitos sociais, definindo-os como sendo o processo de seleção, entre as muitas virtualidades produzidas pelas relações práticas entre as classes sociais e pela institucionalização destas relações, daquela ou daquelas que deixarão de ser uma simples possibilidade contida no desenvolvimento das relações sociais e se transformarão em relações reais, práticas, concretas. Como este processo se dá mediante uma série de choques simultâneos entre classes sociais, o critério de seleção é a adequação, em cada momento, entre a passagem destas virtualidades à pratica e as necessidades de uma das classes em conflito \cite[p.~31-32]{BERNARDO1991a}. E são as crises (sociais, econômicas, políticas) o campo ideal para estudar tendências de curto e de longo prazo no desenvolvimento dos conflitos sociais:

\begin{citacao}
Todas as crises articulam tendências a longo prazo e contradições actuantes no curto prazo. Estas, por si só, nunca justificam o carácter catastrófico que distingue as crises; e se nos restringirmos aos processos de longa duração é impossível explicar a razão porque em dado momento, e não noutro qualquer, se precipitam e amplificam as consequências negativas. À medida que se se agravam as contradições a longo prazo é-lhes mais difícil absorver e superar os efeitos das contradições a curto prazo até que, quando o não conseguem, precipita-se a crise. Os processos de longa duração ficam então postos a nu, graças aos de curta duração. E estas, que não podiam antes desenvolver-se completamente, têm a oportunidade de se revelar na plenitude dos seus efeitos. As crises são o campo empírico ideal para estudar uns e outros. \cite[p.~128]{BERNARDO1997}.
\end{citacao}

\subsubsection{Integração econômica, condições gerais de produção (CGP), poder político e classes sociais}\label{intecocgpppcs}

Enquanto a escola francesa de sociologia trouxe as condições gerais de produção para o centro do debate sobre as cidades por meio de extensões e atualizações conceituais que levavam sua teoria de origem quase ao limite, na obra de João Bernardo esta categoria, também central para a pesquisa apresentada nesta dissertação, ganha tamanha centralidade que torna-se necessário destrinchá-la.

João Bernardo fez uma crítica profunda da contradição entre o \textit{modelo de produção da mais-valia} e o \textit{modelo de distribuição da mais-valia} na teoria marxista: para o autor, Marx teria elaborado n'\textbf{O Capital} um modelo explicativo baseado no funcionamento de \textit{uma só empresa abstrata isolada das demais empresas abstratas}, e não de \textit{várias empresas desiguais funcionando em conjunto}. O resultado, em termos teóricos, é que a relação entre proletariado e capitalistas aparece distorcida: 

\begin{citacao}
\dots enquanto sob o ponto de vista da produção esta [\textit{a relação entre o proletariado e os capitalistas}] aparecia como resultado de uma relação de classes globalizadas, sob o ponto de vista da sua distribuição a produção de mais-valia passa a apresentar-se como resultante da relação entre um grupo particular de operáriose um capitalista particular. [\dots] Parte-se do princípio, e toda a parte da obra em que o objeto da análise é a produção da mais-valia, em geral no livro primeiro, que o sistema capitalista vigora em absoluto (portanto, que não há relaçoes com outros regimes de produção), que existe uma única nação (portanto, que não há comércio externo) e, além disso, que se trata de uma única empresa \cite[p.~10-11]{BERNARDO1977b}.
\end{citacao}

Entre outros problemas desta opção teórica dissecados pelo autor \cite[p.~7-21]{BERNARDO1977b}, interessa a uma teoria do planejamento urbano a \textit{relação entre empresas}, e portanto a \textit{integração econômica}, que é também central no pensamento de João Bernardo:

\begin{citacao}
O que está em causa é a ausência de um verdadeiro modelo das relações inter-capitalistas. Na perspectiva macro-econômica da produçao da mais-valia em função dessa produção, ou da circulação da mais-valia, Marx ou reduz a totalidade económica a uma só empresa ou a considera composta de empresas absolutamente idênticas e indiferenciadas, de forma que tanto num caso como noutro a relacionação entre as empresas não é pensada enquanto problema. Em toda a obra de Marx essa relacionação não é rigorosamente definida e para exprimi-la faz-se apelo a categorias empíricas e convencionais que permitem, ao nível da forma de exposição, apresentar como resolvido um problema que nem sequer realmente é posto \cite[p.~21]{BERNARDO1977b}.
\end{citacao}

O autor pretendeu preencher esta lacuna ao apresentar as \textit{condições gerais de produção (CGP)} como \textit{campo fundamental da inter-relação capitalista} \cite[p.~110-115]{BERNARDO1977b}. Em sua obra \textbf{Economia dos conflitos sociais} \cite{BERNARDO1991}, apresentou um modelo mais bem-acabado das relações entre capitalistas a partir das CGP:

\begin{citacao}
No modelo que proponho [\dots] a integração econômica pressupõe a diferenciação recíproca dos processos produtivos. A hierarquização é a forma como esta integração se realiza. O lugar dominante cabe aos processos que surtem o maior número de efeitos tecnológicos em cadeia e o leque mais vasto desses efeitos, porque o seu \textit{output} serve de \textit{input} ao maior número de outros processos. O aumento da produtividade num dos processos produtivos dominantes constitui, portanto, uma condição necessária para que tal aumento ocorra num número muito elevado dos restantes, pelo que são eles as condições fundamentais para a integração econômica global. [\dots] A estes processos fundamentais, necessários à integração das unidades econômicas no nível da própria atividade produtora, chamo Condições Gerais de Produção (CGP) \cite[p.~157-158]{BERNARDO1991}.
\end{citacao}

Aqui percebe-se tanto a presença dos elementos indicados por Marx quanto alguns dos processos que Marx classificara como economias no emprego do capital constante\footnote{Ao discutir uma descrição dos elementos da economia no emprego do capital constante feita por Marx, João Bernardo exclama: ``Pois não é esta a base imediata do modelo de distribuição de mais-valia que tenho vindo a enunciar? Entre as 'esferas que fornecem ao capital os seus meios de produção' não estão as condições gerais de produção, que dominam a extracção, as fontes de energia e a sua repartição, bem como bom número de matérias-primas?'' \cite[p.~114]{BERNARDO1977b}}. As condições gerais de produção, concebidas deste modo, são distintas das \textit{unidades de produção particularizadas}:

\begin{citacao}
Àquelas unidades que não desempenham qualquer função de CGP, denomino Unidades de Produção Particularizadas (UPP). Considero-as particularizadas porque, servindo o seu \textit{output} de \textit{input} a um número reduzido de outros processos, não desempenham funções básicas nem centrais na propagação de aumentos de produtividade. Enquanto as CGP iniciam a generalidade das remodelações tecnológicas e dão aos seus efeitos o âmbito mais vasto possível, cada UPP limita-se a veicular tais efeitos ao longo da linha de produção em que diretamente se insere, e dessa apenas \cite[p.~158]{BERNARDO1991} 
\end{citacao}

João Bernardo apresentou uma descrição abrangente e minuciosa das condições gerais de produção, na tentativa de demonstrar seu papel enquanto elementos integradores da produção capitalista:

\begin{enumerate}
\item \textit{Condições gerais da produção e da reprodução da força de trabalho}, compostas pelas ``creches e os estabelecimentos de ensino destinados à formação das novas gerações de trabalhadores, bem como as condições várias de existência das famílias de trabalhadores'', pelas ``infra-estruturas sanitárias e os hospitais'' e pelo ``meio social em geral e, nomeadamente, o quadro urbano'', ressaltando que ``aqui se insere o urbanismo, em sentido muito lato''  \cite[p.~159]{BERNARDO1991}.
\item \textit{Condições gerais da realização social da exploração}, ou seja, ``as condições para que o processo de trabalho ocorra enquanto processo de produção de mais-valia'', isto é, ``para que os trabalhadores sejam despossuídos da possibilidade de reproduzir e formar independentemente a força de trabalho e sejam despossuídos do produto criado'', sendo, portanto, afastados também da organização do processo de trabalho. Trata-se das instituições repressivas e do urbanismo \cite[p.~159]{BERNARDO1991}.
\item \textit{Condições gerais da operatividade do processo de trabalho}. São as ``condições para que o processo de trabalho, definido como processo de exploração, possa ocorrer materialmente'', vez que a exploração econômica dos trabalhadores sob o capitalismo ``requer meios tecnológicos que, ao mesmo tempo que realizam o afastamento dos trabalhadores relativamente à administração da produção, põem à disposição dos capitalistas as formas de efetivarem essa administração''. Trata-se dos ``centros de investigação e de pesquisa, tanto teórica como aplicada, mediante os quais os capitalistas realizam e reproduzem o seu controle sobre a tecnologia empregada, dela excluindo os trabalhadores'' e as ``várias formas de captação, veiculação e armazenamento de informações, que conferem aos capitalistas o controle dos mecanismos de decisão e lhes permitem impor à força de trabalho os limites estritos em que pode expressar opiniões ou tomar decisões relativamente aos processos de fabricação'' \cite[p.~160]{BERNARDO1991}.
\item \textit{Condições gerais da operacionalidade das unidades de produção}. Trata-se das infraestruturas, ``nomeadamente as redes de produção e distribuição de energia; as redes de comunicação e transporte; os sistemas de canalização para fornecimento de água e para escoamento de detritos e, em geral, da coleta de lixo; a criação, ou preparação, ou acondicionamento dos espaços ou suportes físicos, ou do ambiente, onde se instalam processos de produção'' \cite[p.~160-161]{BERNARDO1991}.
\item \textit{Condições gerais da operatividade do mercado}. Trata-se dos ``sistemas de veiculação, cruzamento e comparação de informações que permitem o estabelecimento de relações entre produtores e consumidores'', das ``redes de transporte'' e das instalações de armazenamento de produtos cujo consumo não seja imediato, ``desde que, como freqüentemente sucede, sejam comuns ao \textit{output }de várias linhas de produção''  \cite[p.~161]{BERNARDO1991}.
\item \textit{Condições gerais da realização social do mercado}. Trata-se fundamentalmente de estimular o ``consumo de determinados bens específicos produzidos por algumas empresas'' e de condicionar ``um certo estilo de vida, a aquisição de um certo leque de bens ou até o consumo em geral''; são incluídas aqui pelo autor tanto a publicidade quanto certos aspectos da educação \cite[p.~161]{BERNARDO1991}.
\end{enumerate}

A estes conflitos sociais corresponde, neste modelo, uma superestrutura política diferenciada:

\begin{citacao}
O nível do político é o Estado, entendido como aparelho de poder das classes dominantes. Sob o ponto de vista dos trabalhadores, esse aparelho inclui as empresas. No interior de cada empresa, os capitalistas são legisladores, superintendem as decisões tomadas, são juízes das infrações cometidas, em suma, constituem um quarto poder, inteiramente concentrado e absoluto, que os teóricos dos três poderes clássicos no sistema constitucional têm sistematicamente esquecido, ou talvez preferido omitir. E, o entanto, a lucidez de Adam Smith permitira-lhe já colocar ao lado do poder  político, tanto civil quanto militar, o poder de comandar e usar o trabalho alheio. [\dots] A este aparelho, tão lato quanto o são as classes dominantes, chamo \textit{Estado Amplo}. O Estado A é constituído pelos mecanismos da produção da mais-valia, ou seja, por aqueles processos que asseguram aos capitalistas a reprodução da exploração [\dots]. 

Apenas sob o estrito ponto de vista das relações entre capitalistas, o Estado pôde se reduzir ao sistema de poderes classicamente definido, a que chamo aqui de \textit{Estado Restrito}. Os parâmetros da organização do Estado R definem-se pelos casos-limite da acumulação de capital sob forma absolutamente centralizada, e temos então a ditadura interna aos capitalistas, ou sob forma dispersa, isto é, quando existe uma pluralidade de pólos de acumulação, e temos então a democracia interna aos capitalistas. A organização do Estado R depende, em suma, do processo de constituição das classes capitalistas.

O Estado globalmente considerado, a integralidade da superestrutura política, resulta da articulação entre o Estado A e o Estado R \cite[p.~162-163]{BERNARDO1991}.
\end{citacao}

No modelo teórico de João Bernardo as \textit{classes sociais capitalistas} são igualmente radicadas no processo de integração econômica:

\begin{citacao}
O sistema de integração hierarquizada dos processos produtivos, com a superestrutura política que lhe corresponde, pressupõe que no interior do grupo social dos capitalistas se distingam a particularização e a integração. De cada um destes aspectos fundamentais decorre uma classe capitalista: a classe burguesa e a classe dos gestores. Defino a \textbf{burguesia} em função do funcionamento de cada unidade econômica enquanto unidade particularizada. Defino os \textbf{gestores} em função do funcionamento das unidades econômicas enquanto unidades em relação com o processo global. Ambas são classes capitalistas porque se apropriam da mais-valia e controlam e organizam os processos de trabalho. Encontram-se, assim, do mesmo lado na exploração, em comum antagonismo com a classe dos trabalhadores \cite[p.~202]{BERNARDO1991} (\textbf{grifo nosso}). 
\end{citacao}

Estas duas classes capitalistas diferenciam-se por critérios bastante objetivos:

\begin{enumerate}
\item Quanto às \textit{funções desempenhadas no processo produtivo}. Burgueses e gestores podem compartilhar espaços nas unidades particularizadas de produção e na produção das condições gerais de produção, assim como no Estado Amplo e no Estado Restrito, mas é sua função nestes lugares que as diferencia enquanto classe: os burgueses organizam \textit{processos econômicos particularizados} e \textit{fazem-no reproduzindo esta particularização}, e por sua vez os gestores organizam \textit{processos decorrentes do funcionamento econômico global }e da \textit{relação de cada unidade econômica com tal funcionamento} \cite[p.~203-204]{BERNARDO1991}. 
\item Quanto às \textit{superestruturas jurídicas e ideológicas}. Os burgueses se apropriam do capital através da \textit{propriedade privada dos meios de produção}, enquanto os gestores se apropriam do capital através de sua \textit{relação com a integração econômica}; estes últimos, embora possam receber salários, têm sua remuneração complementada através de \textit{suplementos}, \textit{seguros e pensões} e \textit{regalias em gêneros}, e nos lugares onde a burguesia mantém ainda ativa presença empresarial a remuneração complementar assume também a forma de \textit{ações da empresa}, \textit{empréstimos concedidos pela empresa} a juros baixíssimos, \textit{prêmios} em caso de demissão etc. A estas superestruturas \textit{jurídicas} correspondem também concepções \textit{ideológicas}, ou seja, diferentes \textit{projetos de organização da totalidade social}. Os burgueses, seguindo a atomística de sua posição na produção, concebem o funcionamento da sociedade em termos de \textit{livre-mercado} e pugnam pela sua expansão; esta transferência para o mundo das ideias da forma jurídica de sua apropriação do capital, entretanto, não corresponde a qualquer mecanismo de funcionamento da economia. Os gestores, por sua vez, concebem a sociedade em termos de \textit{planificação}, entendendo-a enquanto fenômeno inovador, inaugurado no momento em que alcançaram a hegemonia social, econômica e política, e apto a suplantar as formas tradicionais de concorrência e o mercado \cite[p.~204-208]{BERNARDO1991}. 
\item Quanto às \textit{diferentes origens históricas}. O capitalismo surge do desenvolvimento e desintegração do \textit{regime senhorial} \cite{BERNARDO1995, BERNARDO1997, BERNARDO2002}, e as classes sociais que o compõem encontram sua origem histórica no funcionamento da economia deste regime.  Enquanto a burguesia surge do chamado \textit{putting-out system}\footnote{Forma de organização da produção surgida nas fases iniciais do capitalismo, caracterizada por uma relação comercial entre um \textit{mercador-coordenador} e \textit{produtores sub-contratados}: enquanto o mercador-coordenador compra matéria-prima, os sub-contratados trabalham-na para produzir bens manufaturados, que vendem ao mercador-coordenador. Todo o trabalho de manufatura era feito no próprio domicílio do produtor sub-contratado; a ligação entre as etapas de produção era coordenada pelo mercador-comprador, que se encarregava (pessoalmente ou através da contratação de pessoal próprio) do transporte os bens em produção de casa a casa, até que toda a cadeia produtiva necessária para a transformação da matéria-prima em produto final estivesse concluída \cite[p.~215-216]{WILLIAMSON1985}.} e da fragmentação própria deste sistema pré-manufatureiro de trabalho doméstico terceirizado, os gestores formaram-se enquanto classe a partir de instituições onde os poderes se concentravam, como a \textit{burocracia de corte}, a \textit{burocracia dos grandes soberanos e príncipes} e a \textit{burocracia das cidades}, devendo esta última, segundo João Bernardo, ser considerada como uma \textit{senhoria coletiva} frente ao campesinato; estas burocracias criaram as condições gerais que permitiram ao \textit{putting-out system} e a outras formas embrionariamente empresariais\footnote{A desagregação do \textit{comunitarismo rural} nos séculos XIV e XV, em seguida às sucessivas derrotas da plebe rural nas lutas sociais que acompanharam as grandes heresias medievais e os primeiros anos da Reforma; a ascensão e o enriquecimento de \textit{camponeses abastados}; a crise econômica que levara a \textit{classe senhorial} a vender partes consideráveis de seu patrimônio; a acumulação de fortuna fundiária nas mãos dos camponeses ricos; a proliferação de \textit{jornaleiros}, ou seja, de trabalhadores rurais sem-terra a vagar pelos campos em busca de trabalho a cada safra ou entre-safra; o interesse de negociantes-empresários das cidades em aproveitar a mão-de-obra artesanal existente nas áreas rurais e implementar nestas áreas, fora do controle das corporações de ofício, pequenas manufaturas têxteis; tudo isto, para João Bernardo, cria as condições para uma \textit{economia não-senhorial} no final da Idade Média \cite[p.~579-623]{BERNARDO2002}, cujo desenvolvimento veio a resultar no regime capitalista do início da Idade Moderna.} converter-se em empresas capitalistas propriamente ditas \cite[p.~208]{BERNARDO1991}. 
\item Quanto aos \textit{diferentes desenvolvimentos históricos}. Embora compartilhem origens históricas muito próximas, embora distintas, burgueses e gestores desenvolveram-se enquanto classes sociais mediante processos históricos distintos. Nas fases iniciais do capitalismo, a classe dos gestores encontrava-se \textit{fragmentada em vários campos} e, no interior de cada um, em \textit{instituições e unidades econômicas distintas}, sem que seus integrantes relacionassem-se reciprocamente. Sendo a \textit{mais-valia relativa} -- ou seja, o \textit{aumento constante da produtividade} -- o motor do crescimento do capitalismo, ela exige o \textit{aumento da concentração da força de trabalho e da composição técnica do capital}; isto exige \textit{investimentos} cada vez mais altos, na medida em que a quantidade de capital necessária para assegurar a reprodução ampliada é elevada pelas pressões sobre a taxa de lucro. As \textit{crises econômicas}, ao desvalorizar o capital, fazem com que estes investimentos possam ser reduzidos nos períodos de recuperação próprios a cada ciclo econômico. Rapidamente, com a evolução das crises e com as necessidades de novos investimentos, foram atingidos níveis de concentração que ultrapassaram as capacidades de qualquer capital individual ou familiar, e em poucas décadas mesmo a capacidade de investimento derivada da associação alguns poucos burgueses (via sociedades limitadas) também foi ultrapassada. Os incrementos na produtividade só puderam continuar, então, na medida em que se tornou possível \textit{mobilizar a generalidade indiscriminada dos capitais} por meio de \textit{sistemas financeiros} (conceito que, para o autor, engloba tanto as operações de crédito quanto as sociedades por ações). As \textit{barreiras institucionais} entre os pequenos investidores particulares e a aplicação efetiva dos capitais investidos, na forma de diretorias de empresa, burocracias bancárias e securitárias e outras, multiplicaram-se e complexificaram-se à medida em que evoluíam as formas de crédito, seguro e sociedades acionárias. E é a partir de sua posição em tais lugares que, por exemplo, direções de bancos aplicam recursos sem consultar os correntistas, seguradoras compram e vendem ações sem consultar os componentes de seus fundos de seguro, diretorias de empresas tomam decisões sem consultar a globalidade dos acionistas etc. A \textit{concentração econômica}, ao centralizar capitais anteriormente dispersos e ao instituir barreiras entre seus titulares e sua aplicação efetiva, tornou-se ao mesmo tempo sinônimo da \textit{dispersão da propriedade privada do capital} e de \textit{progressiva hegemonia daqueles que detém o controle das instituições controladoras destes capitais centralizados} -- os gestores. Na medida em que a concentração econômica facilita igualmente a integração recíproca de unidades de produção particularizadas, o poder dos gestores resulta ainda maior. Na medida em que as instituições surgidas no processo de concentração econômica compõem o Estado Amplo, é neste lugar que começa a hegemonia dos gestores; é daí que se lançam ao que possa haver restado de significativo das instituições integrantes do Estado Restrito. E todo este processo \textit{enfraquece o poder da burguesia}, que perde paulatinamente sua hegemonia à medida em que avança a concentação de capitais e a influência dos gestores sobre o Estado Restrito; sua tendência, enquanto classe, é a de transformar-se numa classe de \textit{rentistas} \cite[p.~208-216]{BERNARDO1991}.
\end{enumerate}

\subsubsection{Meio urbano, urbanismo e arquitetura num quadro de conflitos sociais}\label{subsubsec:murbarqcs}

Na complexa arquitetura conceitual de João Bernardo, pode-se verificar que o \textit{urbanismo} é uma das condições gerais de produção. Mais especificamente, é uma das \textit{condições gerais da produção e da reprodução da força de trabalho}:

\begin{citacao}
Qualquer tipo de urbanismo capitalista, pela simultânea separação social dos \textit{habitats} e integração social das vias de comunicação, ao mesmo tempo reflete e condiciona a simultânea cisão e articulação sociais que ocorrem no processo da mais-valia. Trata-se de uma condição fundamental, tanto para a produção da força de trabalho, como para as demais formas de produção da mais-valia \cite[p.~159]{BERNARDO1991}.
\end{citacao}

A segregação urbana aparece no campo das condições gerais de produção. Para João Bernardo, o meio urbano tem um papel destacado no processo de formação de novas gerações de trabalhadores ao promover a separação de uma geração jovem de trabalhadores da plurimilenária cultura rural que precedeu o capitalismo:

\begin{citacao}
A ortogonalidade das arquiteturas e da urbanização e a ocorrência simultânea de ritmos diferentes e defasados são dois aspectos de importância primordial na formação das mentalidades e das habilidades adequadas à tecnologia industrial. Basta recordar que recentemente, quando o capitalismo precisou aumentar maciçamente a oferta de mão-de-obra apta a laborar com as novas técnicas eletrônicas, não se limitou a ministrar cursos de formação nem a introduzir o computador na escola. Difundiu-o maciçamente no meio urbano, a um ponto tal que os jogos, de mecânicos que eram, passaram a ser eletrônicos e qualquer criança educada nas cidades de hoje [\textit{1991}], pelo mero fato de brincar, torna-se mais capaz de entender o manejamento de computadores do que um adulto instruído. Assim, no ócio extradoméstico e mesmo durante os próprios períodos em que transita entre a esfera da família e a das instituições formadoras especializadas, a futura força de trabalho vai paulatinamente recebendo um adestramento manual e psíquico insubstituível \cite[p.~82-83]{BERNARDO1991}. 
\end{citacao}

Por isto mesmo, não apenas o meio urbano, mas também o urbanismo e a arquitetura podem ser entendidos como uma das \textit{condições gerais da realização social da exploração}:

\begin{citacao}
Podemos a partir daqui entender a estreita conjugação entre as formas repressivas e o urbanismo. A vigilância indireta requer a configuração especial da arquitetura e mesmo toda uma paisagem urbana, tal como, já no seu tempo, a reconstrução de Paris sob a orientação de Haussmann tivera entre os objetivos principais a adoção de novas técnicas no combate às insurreições \cite[p.~160]{BERNARDO1991}.
\end{citacao}

Há outro aspecto em que o meio urbano, a arquitetura e o urbanismo podem ser entendidos como condições gerais da realização social da exploração: o \textit{combate travado por burgueses e gestores contra o inter-relacionamento social dos trabalhadores fora dos quadros capitalitas}. Para João Bernardo, sob o ponto de vista social,

\begin{citacao}
\dots a integração dos trabalhadores no capitalismo é sinônimo da fragmentação da força de trabalho. No organograma de uma empresa, cada trabalhador encontra-se inteiramente individualizado e só lhe seria consentido um relacionamento direto com a direção ou, pelo menos, apenas dentro do quadro oficialmente determinado poderiam os trabalhadores estabelecer entre si relações diretas; as relações entre os trabalhadores seriam autorizadas na medida somente em que decorressem das necessidades do processo de trabalho, ou seja, mediante a prévia relação de cada trabalhador com as respectivas chefias. Neste esquema ideal, que constitui o sonho de qualquer capitalista, a permanente interferência da direção da empresa, esforçando-se para que o relacionamento entre trabalhadores seja apenas indireto, resultado das relações diretas de cada um com a chefia, é a garantia da individualização dos trabalhadores, da sua fragmentação. Este quadro social inspira o sistema tecnológico vigente e é por ele reproduzido. O relacionamento recíproco dos trabalhadores durante o processo material de trabalho decorre da relação de cada um com a maquinaria, que é globalmente controlada, pela administração capitalista. Explicam-se assim os sistemas salariais que dividem os trabalhadores numa quantidade tão grande de subcategorias que cada uma quase tende a ser preenchida por um indivíduo apenas, de maneira a estimular a concorrência e os conflitos internos à força de trabalho. De um modo geral, o capitalismo lança mão de todas as  tradições culturais e preconceitos, desde o racismo até o bairrismo, capazes de acentuar a fragmentação da classe trabalhadora e o individualismo dos seus membros. E, como se trata de um sistema econômico totalizante, que não rege apenas a produção de bens, mas também a própria produção de força de trabalho, tendendo, portanto a desenvolver extensiva e intensivamente até abranger a globalidade da sociedade, a individualização dos trabalhadores encontra-se reproduzida na individualização dos cidadãos \cite[p.~317]{BERNARDO1991}
\end{citacao}

Do ponto de vista dos burgueses e dos gestores, o \textit{inter-relacionamento social genérico dos trabalhadores} precisa se dar sob seu controle, pois seu aprofundamento fora das práticas e instituições em que burgueses ou gestores aparecem como intermediários resulta no \textit{fortalecimento dos trabalhadores enquanto classe}. O inter-relacionamento social genérico dos trabalhadores é, portanto, um quadro social onde podem ser desenvolvidas relações sociais alheias ao controle capitalista, podendo inclusive comportar virtualidades revolucionárias:

\begin{citacao}
No interior das empresas, os grupos informais constituem um quadro deste inter-relacionamento social mais genérico e, ao mesmo tempo, dele resultam. Grupos informais e relações humanas supraprofissionais são sistemas indissociáveis. Fora dos locais de trabalho, estas relações tecem-se em torno de pontos de convergência: as tabernas, os cafés, os bares, as associações musicais, desportivas ou recreativas; até a igreja, sobretudo quando os fiéis se recrutam apenas entre a população trabalhadora, não sendo a freqüência interclassista; e os mais simples de todos, os jardins, a praça pública. Enquanto se restringem ao aspecto formal mais aparente, enquanto o convívio parece não ter outra função senão a da mera presença em conjunto, este inter-relacionamento é um fator de conformismo, pressionando os que freqüentam um mesmo pólo de concentração a obedecer a padrões de comportamento comuns. É, então, um fator de divisão entre grupos. Mas, quando os conflitos se desenvolvem, rapidamente estes aspectos são eliminados ou, pelo menos, secundarizados, servindo o inter-relacionamento social de quadro de radicalização \cite[p.~329]{BERNARDO1991}.  
\end{citacao}

As estratégias de capitalistas e gestores para controlar este inter-relacionamento são muitas:

\begin{citacao}
Por vezes procuram retirar aos trabalhadores o controle dos pólos de inter-relacionamento, criando nas empresas clubes e centros recreativos ou conquistando, com subsídio e interesseiras benesses, aqueles que tenham sido fundados autonomamente. Em outros casos, tentam desarticular verdadeiramente as redes de inter-relacionamento genérico dos trabalhadores, destruindo por completo bairros tradicionais e forçando os habitantes a dispersarem-se por áreas residenciais novas, deliberadamente planejadas e construídas sem pontos de convergência, sem jardins e praças, sem cafés nem centros esportivos. Referi, no capítulo respectivo, as funções do urbanismo enquanto CGP. Vemos agora que o cuidadoso planejamento de cidades-dormitório é hoje uma condição geral para que o processo de produção possa ocorrer no quadro da redução dos conflitos às formas individuais e passivas \cite[p.~330]{BERNARDO1991}.  
\end{citacao}

Os quadros, ritmos, espaços e formas do inter-relacionamento social constituem, desta forma, um \textit{campo da luta de classes}:

\begin{citacao}
E vemos assim que o inter-relacionamento social genérico, se é objeto da estratégia dos capitalistas, converte-se ele próprio em campo da luta de classes onde, portanto, os trabalhadores conduzem uma ação com o objetivo de preservar, ou de restaurar, sistemas de inter-relacionamento. À desarticulação dos espaços públicos pelo novo urbanismo, opõe-se uma imaginosa recriação, o desvio de certos elementos urbanos da função prevista e o seu aproveitamento enquanto pólo de relações entre os moradores \cite[p.~331]{BERNARDO1991}. 
\end{citacao}

Os receios de burgueses e gestores quanto ao inter-relacionamento social de trabalhadores é justificado. Para ilustrar sua argumentação, João Bernardo traz um exemplo simples, no qual é possível reconhecer tantas e quantas comunidades populares mundo afora:

\begin{citacao}
Um correspondente anônimo de um obscuro jornal operário deu conta da generalização e da agudização dos conflitos trabalhistas na cidade espanhola de Reinosa, onde, durante meses, a partir de finais de 1986, as massas trabalhadoras enfrentaram unânime e ativamente, com a maior coragem e engenho, os grandes capitalistas que controlam as indústrias locais e os reforços policiais diariamente intensificados.

Espantava-se esse correspondente que uma povoação ``que fazia dos bares o principal núcleo de relacionamento'' e que fora até então conhecida como ``la ciudad de los cien bares'', pudesse ter-se convertido na cidade onde todos lutavam como um só, sem precisarem aparentemente de nenhum tipo de organização nem de receberem indicações de ninguém. Não há razão para espantos, antes ao contrário. A freqüentação dos cem bares, repetida ao longo dos anos, criou entre os trabalhadores um inter-relacionamento tão estreito que permitiu, chegada a hora do confronto, que se afirmassem como um coletivo único e que a combatividade de uns tantos se repercutisse em todos \cite[p.~329-330]{BERNARDO1991}.
\end{citacao}

\subsection{Terceiro campo teórico: história da Primeira República}
\label{subsec:histprirep}

O terceiro campo teórico é a \textit{produção historiográfica sobre a Primeira República} \cite{BRUNO1967, carone_evolucao_1977, CARONE1970inst, faoro_donos_2001, fausto_hgcb1_1977, fausto_sociedade_1977, freyre_ordem_2004, janotti_subversivos_1986, leal_coronelismo_2012, LINS1988coro, PEDROSA1966a, PEDROSA1966b, pires_eleicoes_1995, saes_classemedia_1975, silva_historiaeconomica_2002}. Estas obras permitirão reconstruir, de modo razoavelmente aproximado, a institucionalização de práticas \cite{BERNARDO1991} e as significações social-históricas \cite{CASTORIADIS1982} hegemônicas presentes no período analisado. Tal reconstrução permitirá avaliar a intensidade dos conflitos sociais encontrados na literatura de caráter monográfico, ou seja, em que grau os conflitos sociais encontrados ameaçaram colocar em xeque os usos hegemônicos do território, ou em que medida os usos alternativos do território eram tolerados pelos grupos sociais hegemônicos de cada período estudado.

Dentro dele, foi dado especial relevo, por força do recorte espacial, à \textit{produção historiográfica sobre a Salvador da Primeira República} \cite{araujo_inventario_1992, castellucci_maquina_2008, CUNHA2011, sampaio_partidos_1978, sampaio_legislativo_1985, santos_associacao_1985, pang_coronelismo_1979}. O que se fez nesta pesquisa foi investigar, a partir dos conflitos sociais próprios deste período histórico, os traços e os rastros de conflitos envolvendo o uso do território, em especial os que, com variados graus de ênfase, sejam opostos aos usos hegemônicos. Trata-se da tentativa de compreender como o fazer coletivo e cotidiano daqueles indivíduos anônimos – anônimos para nós, claro, pois cada qual teve nome, vida e trajetória – foi capaz de institucionalizar práticas \cite{BERNARDO1991} e construir significações social-históricas \cite{CASTORIADIS1982} aptas a afrontar, em maior ou menor medida, o ordenamento territorial que se lhes impunha. Ou, no dizer de Walter Benjamin, trata-se de escovar a história a contrapelo para identificar a ``corveia anônima'' responsável pela produção destes bens culturais e apontar-lhes os elementos de barbárie \cite[p.~225]{BENJAMIN1987}.

Ainda neste campo, optou-se, devido à permanência de certos aspectos da vida social e à novidade de outros, bem como a tensão entre as permanências e as novidades, por integrar a este campo teórico obras sobre a \textit{transição do trabalho escravo para o trabalho livre }\cite{ANDRADE1988, AZEVEDO2004, brito2003abolicao, COSTA1991, DIAS2004, HOLTHE2003, mata2007libertos, MATTOS2008, MATTOSO1978, MATTOSO1992, MATTOSO1988, menezesfilho2007pos, MOURA1981, NASCIMENTO2007, REIS2000, REIS2004males, REISGOMES1996, REISSILVA1989, REIS2012, COSTA1989}. Estas obras permitirão reconstruir de modo aproximado – ainda que inevitavelmente prenhes do ``agora'' \cite[p.~229-230]{BENJAMIN1987} – modos de vida, relações sociais e territorializações dos grupos sociais subjugados no período proposto, aptos a reconstituir usos alternativos ou contra-hegemônicos do espaço urbano e periurbano de Salvador, e em que medida estes usos confrontaram os usos hegemônicos.

\subsection{O modelo teórico resultante}
\label{subsec:modteoresult}

Com os elementos destes três campos teóricos, a pesquisa apresentada nesta dissertação fundamenta-se num modelo teórico através do qual os conflitos entre agentes de produção do espaço urbano de Salvador na Primeira República nos processos de reorganização espacial podem ser compreendidos dentro de seu contexto de época, para avaliar seu impacto sobre as formas hegemônicas e contra-hegemônicas de produção, apropriação e uso do espaço urbano no distrito de Brotas, e tais conflitos podem ser identificados, sistematizados e, na medida do possível, cartografados.

\section{Metodologia}
\label{sec:metodo}

Esta pesquisa, sendo fundamentalmente uma pesquisa historiográfica, envolveu a \textit{pesquisa bibliográfica em fontes primárias e secundárias}, fundamental para reconstruir, de modo razoavelmente aproximado, a institucionalização de práticas \cite{BERNARDO1991} e as significações social-históricas \cite{CASTORIADIS1982} hegemônicas presentes em cada um dos períodos da história de Salvador analisados.

Entre as fontes primárias, destacam-se no \textit{Arquivo Histórico Municipal de Salvador} (BR-BAAHMS) as 24 caixas de documentos avulsos com os projetos de construções, reformas, ampliações, loteamentos etc. do distrito de Brotas no período entre 1893 -- quando tais projetos se tornaram obrigatórios para a concessão das licenças municipais -- e os anos 1950 -- data-limite, até a presente data, de recepção de material arquivístico pela instituição. A princípio, e aproveitando o fato de que a equipe deste arquivo já havia concluído a organização destas caixas por ruas -- nos demais distritos, a organização dos projetos ainda é por ano, sem identificação de ruas --, estes projetos seriam analisados por amostragem, de acordo com ruas selecionadas; entretanto, dada a centralidade destes documentos para a pesquisa, foi feita a sacrificante, porém necessária, análise de \textit{todos} eles, sem qualquer exceção. Trabalhar com a totalidade dos requerimentos de licença permitiu conhecer muito mais a respeito da produção do espaço urbano em Brotas que qualquer dedução ou análise abstrata, e ajudou a consolidar algumas noções mais bem expostas adiante, em momento adequado.

No mesmo arquivo, os \textit{Livros das Décimas Urbanas} registram, com periodicidade anual, todos os lotes existentes em cada distrito, sujeitos à cobrança deste tributo ou dele isentos por força de lei, bem como seus proprietários, eventuais inquilinos e o valor de cada imóvel. Dada sua quantidade (entre um a três volumes por ano), e levando em conta tanto o alto custo quanto a baixa fungibilidade dos bens de raiz enquanto mercadoria, assim como o prazo para o encerramento desta pesquisa, foi necessário trabalhá-los usando uma amostragem decenal, restringindo sua leitura aos anos de 1893 (1 vol.), 1900 (1 vol.), 1910 (1 vol.), 1920 (1 vol.) e 1930 (3 vols.). Não obstante o caráter amostral, a consulta à documentação serviu para identificar, ao longo das décadas, a existência da \textit{retenção especulativa} de imóveis no distrito. Uma pesquisa mais rigorosa exigiria o cruzamento dos dados destes livros com aqueles encontrados nos pedidos de licença de obra, mas o tempo não permitiu avançar neste procedimento que permitiria sanar quaisquer dúvidas a respeito do desenvolvimento urbano deste distrito.

Outras fontes importantes no Arquivo Histórico Municipal de Salvador, entre os documentos encadernados, são os \textit{livros de posturas municipais}, o \textit{Código de Posturas Municipais de 1921}, as \textit{coletâneas de legislação municipal} custodiadas na biblioteca do arquivo, bem como as versões encadernadas do \textit{Regulamento Sanitário de 1907} e do \textit{Código de Obras de 1926}. Naquilo que tem de afirmativo, esta legislação permite desenhar a cidade-ideal dos detentores do poder político do período e, pela negativa, conhecer que práticas pretendiam reprimir -- portanto, que tipo de cidade contrastava com esta cidade-ideal, não raro confrontando-a. A partir de 1924 a Inspetoria Municipal de Higiene foi fundida à Diretoria Estadual de Saúde; esta ausência afeta apenas seis entre os 41 anos estudados e não influi tanto nos resultados da pesquisa, pois as medidas higiênicas adotadas são em grande parte as mesmas em toda a legislação consultada.

Na biblioteca do Arquivo Histórico Municipal de Salvador, há a série, infelizmente incompleta, de \textit{Relatórios da Intendência à Câmara Municipal}, em que os \textit{intendentes} soteropolitanos (como eram chamados os \textit{prefeitos} durante a Primeira República) prestavam conta de seus atos ao então chamado ''Poder Deliberativo''. Nestes relatórios é possível encontrar as despesas municipais com educação, iluminação pública, transportes, pavimentação e arruamento, saneamento básico e abastecimento de água etc., que permitirão estabelecer parâmetros comparativos da prestação destes serviços nos distritos estudados.

Entre os documentos avulsos do mesmo arquivo, também foram muito esclarecedores os \textit{projetos de lei} discutidos e aprovados no período selecionado, arquivados em seis caixas de manuscritos.

Esta dissertação poderia ter sido enormemente beneficiada por pesquisas mais demoradas no \textit{Arquivo Público do Estado da Bahia} (BR-BAAPB), mas os prazos e a natureza secundária da documentação nele custodiada relativamente ao objeto da pesquisa neste momento fizeram com que poucos documentos fossem lá consultados. É de especial interesse o \textit{Livro de Registro de Terras da Freguesia de Brotas}, único remanescente do que foi com certeza uma série, dada a função registrária exercida pelas igrejas matrizes de cada freguesia durante a Colônia e o Império. Seriam de igual interesse os livros de escrituras e os inventários de alguns indivíduos, para assim rastrear mais minudentemente o aspecto fundiário do desenvolvimento urbano do distrito estudado; conquanto interessantes, estes documentos -- salvo uma ou outra exceção importante -- não foram consultados, pois, para os fins desta pesquisa, a leitura dos \textit{Livros das Décimas Urbanas} ofereceu informações suficientes.

Ainda foi possível contar com os periódicos baianos inseridos na \textit{Hemeroteca Digital Brasileira} (BR-RJBN-HDB), da Biblioteca Nacional, que fornecerão a base a partir da qual estudar as notícias na imprensa sobre Brotas. As coleções reunidas na Hemeroteca infelizmente não estão completas, mas fornecem rapidamente acesso a documentação, especialmente a jornais e publicações oficiais, que de outro modo estaria disponível apenas em suporte físico em bibliotecas e arquivos. O uso da Hemeroteca Digital Brasileira permitiu enorme diversificação das fontes consultadas no que diz respeito a impressos de época, malgrado sua incompletude.

\section{Estrutura do trabalho}\label{sec:estrutrab}

Esta dissertação está dividida nesta introdução, em três capítulos e uma conclusão. Há, também, alguns apêndices, contendo discussões que se considerou relevantes, necessárias ao desenvolvimento da dissertação mas que destoariam demais do fluxo textual e metodológico empregue como método de exposição. De igual maneira, há alguns anexos com documentos importantes tanto para o processo de pesquisa precedente à redação desta dissertação, quanto documentos importantes para a compreensão do próprio conteúdo desta dissertação.

No \textbf{\textit{primeiro capítulo}}, será esboçado um panorama da sociedade, da política e do espaço urbano de \index{Salvador}Salvador, inserido por sua vez nos contextos global e nacional de sua época. Será dada especial atenção aos aspectos demográfico, político e cultural da cidade, à sua estratificação social e à configuração de seu espaço urbano. 

Antes de chegar às contradições e conflitos sociais próprios da sociedade soteropolitana, é preciso compreender sua inserção no \textit{contexto nacional}, e para chegar até ele é preciso entender igualmente o \textit{contexto internacional}; só por aí se tornará possível compreender algumas das contradições econômicas, políticas e sociais mais importantes do período. Nesta escala, a análise terá de ser, forçosamente, econômica e geopolítica. 

Em seguida, no mesmo capítulo, será desenhado um panorama da \textit{política brasileira} e da \textit{inserção do Brasil na economia global} durante o período, pois a inserção soteropolitana no contexto é fortemente condicionada pelas turbulências da época. Será dada especial atenção à estratificação social brasileira e ao desenvolvimento das grandes cidades no país durante o período, por comporem um ambiente em que ideias e profissionais circulavam com grande facilidade e, portanto, por formarem assim um quadro ideológico e prático comum a partir das propostas de ``melhoramentos'' urbanos. 

Posteriormente, será feita uma análise da sociedade e da política baianas. O foco estará na inserção de Salvador na economia global, na sua demografia, na sua estrutura de classes e em sua estratificação social (com algumas discussões específicas sobre a intelectualidade e a classe trabalhadora), e nas interfaces entre a cultura e o uso dos espaços públicos. Será feita também uma descrição sumária das instituições políticas estaduais e municipais, em especial daquelas que interferiram diretamente na produção do espaço urbano, além de uma caracterização dos agentes da política baiana e de suas interações e conflitos. 

Por fim, encerrando este capítulo, vistos todos os fatores que, nesta pesquisa, se considera condicionantes da produção do espaço urbano no período estudado, será possível adentrar na dinâmica da produção do espaço urbano soteropolitano. Será dada especial atenção ao regime de terras vigente na cidade e às intervenções no espaço urbano, com destaque para as reformas promovidas durante os governos de José Joaquim Seabra (1912-1924). 

Tudo isto compõe o momento \textit{sincrônico} da análise de contexto que permitirá, no terceiro capítulo, analisar mais detidamente o papel dos conflitos sociais na produção, apropriação e uso do espaço urbano no distrito soteropolitano de Brotas na Primeira República.

Como os conflitos sociais, do modo como são compreendidos na presente pesquisa, são \textit{dinâmicos}, só se deixam perceber em análise \textit{diacrônica}, e a passagem do tempo pressupõe obviamente um ``antes'' e um ``depois'', em seguida a esta descrição dos condicionantes sincrônicos da produção, apropriação e uso do espaço urbano em Brotas, será preciso desdobrar \textit{diacronicamente}, no \textbf{\textit{segundo capítulo}}, o desenvolvimento deste distrito, desde sua fundação até o marco temporal inicial desta pesquisa. Trata-se da análise das conjunturas espaciais pretéritas do distrito, o terreno para a análise dos processos de produção, apropriação e uso do território do distrito no período estudado. 

Se o primeiro capítulo permite o \textit{enquadramento sincrônico} dos conflitos sociais distritais num contexto, para percebermos as influências contemporâneas externas ao território, trata-se agora de perceber as \textit{contradições internas em processo}, para que o território e sua produção, apropriação e uso resultem da confluência entre as influências externas e as contradições internas. 

Para caracterizar este segundo vetor será preciso fazer, em primeiro lugar, um breve histórico da formação e delimitação territorial da freguesia de Brotas; em seguida, as características socioeconômicas desta freguesia serão descritas em traços largos; outros usos porventura não explicitados do seu espaço serão delimitados e caracterizados; por último, se tentará capturar o retrato que os soteropolitanos de outros distritos faziam de Brotas. 

No \textbf{\textit{terceiro capítulo}}, central a esta dissertação, será retomada a análise sincrônica dos conflitos sociais, desta vez em detalhe. ESCREVER QUANDO TERMINAR O CAPÍTULO

Por fim, na conclusão, ESCREVER AO FIM DE TUDO.

Em \textbf{\textit{apêndice}}, para não prejudicar o desenvolvimento da argumentação, foram separados os seguintes tópicos, considerados de especial relevância:

\begin{enumerate}
\item a
\item b
\item c
\item d
\item e
\item f
\end{enumerate}

Foram também trazidos em anexo documentos relevantes para a compreensão dos assuntos tratados:

\begin{enumerate}
\item Ficha de registro catalográfico de pedidos de licença de construção, usadas durante a pesquisa arquivística junto ao BR BAAHMS;
\item Algumas plantas de obras que se destacaram do conjunto;
\item Alguns documentos igualmente destacados do conjunto;
\item Lista de ruas do distrito de Brotas entre 1889 e 1930, com indicativo de quantos pedidos de licença de obra há para cada uma no BR BAAHMS;
\item Lista de proprietários rurais do distrito de Brotas constante no Censo de 1920;
\item Lista de engenheiros, arquitetos e agrimensores atuantes no distrito de Brotas entre 1889 e 1930
\end{enumerate}

A anexação da ficha catalográfica justifica-se pelo fato de poder ser instrumento útil na pesquisa de pedidos de licença de construção de outros distritos, com algumas adaptações (p. ex., a equipe da Inspetoria Municipal de Higiene varia de distrito a distrito).

Espera-se, com este trabalho, ter sido possível apresentar contribuição relevante para a história territorial de Salvador e para a história do planejamento urbano na cidade.

% ----------------------------------------------------------